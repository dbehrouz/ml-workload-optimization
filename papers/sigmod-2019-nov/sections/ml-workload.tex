\section{Machine Learning Workloads} \label{sec-ml-workloads}
We define a machine learning workload as a series of exploratory data analysis steps followed by one or multiple model building steps.
Typically users analyze the data by first performing different transformations to gain insight into the data.
Then, based on these analyses, they apply the right set of transformations to the data to train one or multiple models on the transformed training data.
Therefore, machine learning workloads typically consist of both interactive (exploratory analysis) and long-running processes (hyperparameter tuning and model training).
Using the experiment database, we can speed up the execution of the machine learning workloads.
By analyzing the experiment database, we can extract the common transformations on data and materialize them.
Thus, during the interactive exploratory analysis, we first analyze the workload to look for reuse opportunities and return the materialized data when possible.
Moreover, using the experiment database, we provide the users with already trained models or promising hyperparameters which decreases the execution time of the hyperparameter search and model training.

We analyzed several scripts from the Titanic: Machine Learning from Disaster competition in Kaggle\footnote{https://www.kaggle.com/c/titanic}.
The Kaggle platform allows users to submit their solutions as scripts (called notebooks or kernels) to the Kaggle platform.
These notebooks are available publicly for other users to view and fork in their own workspace.
Figure \ref{fig-titanic-script-hierarchy} shows some of the popular notebooks and how other users wrote their notebooks based on the existing ones.
The numbers show how many times each script is forked.
The top most-voted notebooks for the Titanic competition have been forked a total of 44,434 times.
This demonstrates that many of the workloads share similar sets of data transformations and model training.
Materializing frequent transformations and models can greatly increase the efficiency and many repeated operations can be avoided.

\begin{figure}
\centering
\includegraphics[width=\columnwidth]{../images/kaggle-titanic-scripts-graph}
\caption{The fork hierarchy of some of the popular notebooks in Kaggle's Titanic competition and how many times each notebook is forked}
\label{fig-titanic-script-hierarchy}
\end{figure}

In order to apply the materialization optimization, we first define the characteristics of the machine learning workloads.
Then, we specify how to detect reuse opportunities from the experiment database.

\subsection{Machine Learning Workload Characteristics}
We assume the main units of work are dataframe like objects that contain one or many columns, where all the data items in one column are of one data type.
We divide the operations in the ML workloads into 2 categories.

\begin{table}
\centering
\begin{tabular}{ll}
\hline
	   Feature Extraction & Feature Selection\\ \hline
        feature hasher & variance threshold  \\
        one hot encoding & select k best \\
        count vectorizer& select percentile \\ 
        tfidf transformer & recursive feature elimination \\
        hashing vectorizer & select from model \\
        extract\_patch\_2d &  \\
        \hline
\end{tabular}
\caption{List of feature extraction and feature selection operations}\label{feature-engineering-operations}
\end{table}

\textbf{1. Feature engineering:}
\begin{itemize}
\item feature selection operations
\item feature extraction operations
\item user-defined feature engineering operations
\end{itemize}

Table \ref{feature-engineering-operations} shows the list common feature extraction and feature selection operations.

\textbf{2. Model building: }
\begin{itemize}
\item model training operation that applies a training algorithm to a dataset
\item common aggregation operations (such as mean, max, min, and percentile)
\item user-defined aggregation operations 
\end{itemize}
Each model building operation results in objects that can either be used in other feature engineering operations (applying PCA to data or categorizing a column based on values in percentile aggregate) or can be a complete machine learning model that can be used to make predictions on unseen data.
\todo[inline]{How can we capture these in the graph? maybe special edges that connect these nodes to a data node? }

%%% Continue from here
\subsection{Graph Representation}\label{sub-graph-construction}
To efficiently apply our optimizations, we utilize a graph data structure (called the experiment graph) to store the meta-data and logs of the machine learning workloads.
Let $\mathcal{V}=\{v_i\}, i = 1, \cdots, n$ be a collection of artifacts that exist in the experiment database.
Each artifact is either a raw dataset, a pre-processed dataset resulting from a feature engineering operation, or a model resulting from a model training operation.
Let $\mathcal{E}=\{e_i\}, i = 1, \cdots, m$ be a collection of executed operations that exist in the experiment database.
A directed edge $e$ from $v_i$ to $v_j$ in $\mathcal{G}(\mathcal{V},\mathcal{E})$ indicates that the artifact $v_j$ is (fully or partially) derived from the artifact $v_i$ by applying the operation in $e$.
Every vertex $v$ has the attribute $\langle s \rangle$ (accessed by $v.s$), which represents the storage size of artifact when materialized.
Every edge $e$ has the attributes $\langle f, t\rangle$ (accessed by $e.f$ and $e.t$), where $f$ represents the frequency of the operation (the number of times the operation has been executed) and $t$ represents the average run-time (in seconds) of the operation.
Furthermore, each vertex contains meta-data about the artifacts (such as the name and type of the columns for datasets and name, size, value of parameters and hyperparameters, and loss value of the models) and each edge contains the meta-data about the operation (such as the function name, training algorithm, hyperparameters, and in some cases even the source code of the operation).
When a new machine learning workload is executed, we extend the graph to capture the new operations and artifacts.
If an operation already exists in the graph, we update the frequency and average run-time attributes.
Otherwise, we add a new edge and vertex to the experiment graph, representing the new operation and the artifact.
Figure \ref{fig-experiment-graph} shows an example graph constructed from the code in Listing \ref{listing-experiment-graph} based on the Avito demand prediction challenge\footnote{https://www.kaggle.com/c/avito-demand-prediction/}.

\begin{lstlisting}[language=Python, caption=Example script,captionpos=b,label = {listing-experiment-graph}]
import numpy as np
import pandas as pd

from sklearn import svm
from sklearn.feature_selection import SelectKBest
from sklearn.feature_extraction.text import CountVectorizer

train = pd.read_csv('../input/train.csv') 
print train.columns # [ad_desc,ts,u_id,price,y]
vectorizer = CountVectorizer()
count_vectorized = vectorizer.fit_transform(train['ad_desc'])
selector =  SelectKBest(k=2)
top_features = selector.fit_transform(train[['ts','u_id','price']], 
				      train['y'])
X = pd.concat([count_vectorized,top_features], axis = 1)
model = svm.SVC()
model.fit(X, train['y'])
\end{lstlisting}

\begin{figure}
\centering
\documentclass{standalone}
\usepackage{tikz}
\usetikzlibrary{graphdrawing, graphs, quotes, positioning,arrows, backgrounds, math, calc, shapes, positioning}
\usegdlibrary{trees}
\begin{document}
\begin{tikzpicture}
%\draw[help lines]  (-2,0) grid (6,6);
\tikzstyle{every node}=[inner sep=0.02cm]
\node (train) [ellipse, draw] at (2,6) {$train$};
% layer 1
\node (ad) [ellipse, draw, below left = of train] {$ad\_desc$};
\node (forselection) [ellipse, draw, below = of train] {$t\_subset$};
\node (y) [ellipse, draw, below right = of train] {$y$};
% layer 2
\node (cv) [ellipse, draw, below = of ad] {$cnt\_vect$};
\node(sk) [ellipse, draw, below = of forselection] {$top\_features$};
% layer 3
%\node (merged1) [circle, draw] at (1.5, 3.5) {$v_6$};
\node (cvsk) [ellipse, draw, below = of sk] {$X$};
% layer 4
%\node(merged2) [circle, draw] at (3, 2.8) {$v_8$};
% layer 5
\node(model) [ellipse, draw, below right = of cvsk]  {$model$};

\graph [grow down,edge quotes ={inner sep=1pt}, edges ={thick},radius=.2cm, nodes={circle, draw,font =\small}]{
(train) [label=train]
-> [anchor=east, align=center,"p1"] (ad)
-> [anchor=east,align=center,"vf1"] (cv)
%-> [anchor=east,align=center,"m"](merged1) 
-> [anchor=east,align=center,"c1"](cvsk) 
%-> [anchor=south, align=center,"m"](merged2) ;
-> [anchor=east,auto=false,align=center,"f"] (model);

(train) 
-> [anchor=east,align=center, "p2"] (forselection)
-> [anchor=east,auto=false,align=center,"s1"] (sk)
-> [anchor=east,align=center,"c1"](cvsk) ;

(train) 
-> [anchor=west, align=center,"p3"]   (y)
%-> [anchor=west, align=center,"m"](merged2) 
-> [anchor=east,auto=false,align=center,"f"] (model);
};
\end{tikzpicture}
\end{document}
\caption{Experiment graph constructed from the Listing \ref{listing-experiment-graph}}
\label{fig-experiment-graph}
\end{figure}

We make two important assumptions when constructing the graph from the workloads.
First, most of the operations are applied to one or a subset of the columns of a dataset.
In this case, we augment the graph with an edge representing a projection operation that selects the required subset of the columns.
The run-time ($t$) attribute of this edge is set to 0 as this operation is never actually executed.
Second, multiple edges merging into one vertex indicates an operation that receives as input multiple artifacts and outputs one artifact.
This is a critical assumption for our materialization algorithm in Section \ref{sec-materializaiton-and-reuse}
Currently, we support three types of merge operations:
\begin{itemize}
\item \textbf{concat}: a \textit{physical} operation that concatenates the columns of the input datasets (e.g., concatenation of $v_4$ and $v_5$ in Figure \ref{fig-experiment-graph} which represents Pandas' \textit{concat} operation).
\item \textbf{join}: a \textit{physical} operation that joins multiple datasets on a given key.
\item \textbf{combine}: a \textit{logical} operation that combines the input datasets. The combine operation is used to simplify the representation of operations that require multiple artifacts as input (e.g., combining $v_6$ and $v_3$ in Figure \ref{fig-experiment-graph} which simplifies the representation of \textit{model.fit} of scikit-learn library).
\end{itemize}
Since the \textit{combine} operation is logical, the run-time attribute ($t$) of every participating edge is set to $0$.
For the \textit{join} and \textit{concat} operations, the frequency attribute of every participating edge is set to the frequency of the merge operation.
However, the run-time attribute of every participating edge is set to the average run-time of the merge operation divided by the number of the participating edges (e.g., in Figure \ref{fig-experiment-graph}, if the \textit{concat} operation between $v_4$ and $v_5$ is executed 6 times with an average run-time of 10 seconds, the edges connecting $v_4$ to $v_6$ and $v_5$ to $v_6$ have the attributes $\langle6, 5\rangle$).

It is important to note that based our definition of a task in Section \ref{sec-introduction} if the experiment database contains meta-data about multiple tasks, the constructed graph will contain one connected component for every task.
In the next sections, we assume that the experiment database contains information about 1 task, although, all the methods described can be applied to multiple tasks as well.
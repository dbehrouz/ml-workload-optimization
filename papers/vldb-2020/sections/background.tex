\section{Background} \label{sec-background}
In this section, we first provide an example use case and outline its deficiencies.
Next, we describe experiment databases and how they can improve execution of machine learning wokrloads.

\subsection{Motivational Example}\label{subsec-motivational-example}
The data science competition platform, Kaggle, enables organizations to host data science challenges.
Users participate in the challenges, write solutions, and submit these solutions.
To arrive at their final solutions, participants utilize the infrastructure provided by Kaggle to write data analysis and machine learning workloads either in the form of R and Python scripts or Jupyter notebooks and execute them on the Kaggle's platform.
Users can also make their workload publicly available to other users.
As a result, the entire community works together to find high-quality solutions.

Kaggle utilizes docker containers to provide isolated computational environments (called Kaggle kernels).
Each kernel has limited CPU, GPU, disk space, and memory (i.e., 4 CPU cores, 17 GB of RAM, 5 GB of disk space, and a maximum of 9 hours of execution time. GPU kernels have 2 CPU cores and 14 GB of RAM\footnote{https://www.kaggle.com/docs/kernels}).
In busy times, this results in users to be placed in queues (especially for GPU-enabled machines) until resources become available.
Figure \ref{example-use-case} shows the infrastructure of Kaggle.
Kaggle groups kernels by competition.
When users request to execute a kernel, if there are enough resources available, their kernel is executed.
However, when enough resources are not available, the kernel is inserted into a queue and only executed when existing kernels are completely executed.

\begin{figure}
\centering
\includegraphics[width=\columnwidth]{../images/example-use-case}
\caption{Kaggle Infrastructure}
\label{example-use-case}
\end{figure}

Every user that participates in a Kaggle competition, has the same goal, which is to solve the task described by the competition organizer.
Typically, the task is to design a machine learning workload containing a series of exploratory data analysis steps followed by a model training step to train a machine learning model, which aims to maximize a quality metric on an evaluation dataset.
For example, in the \textit{Titanic: Machine Learning from Disaster} competition in Kaggle\footnote{https://www.kaggle.com/c/titanic}, the task is to create a machine learning pipeline and train a classification model on the Titanic training dataset that can predict if a traveler survived the Titanic disaster, with the goal of maximizing the prediction accuracy on a separate test dataset.
When solving the same tasks, users tend to utilize the same type of operations.
Figure \ref{fig-titanic-script-hierarchy} shows the most popular kernels and their relationship for the Titanic competition.
\begin{figure}
\centering
\includegraphics[width=\columnwidth]{../images/kaggle-titanic-scripts-graph}
\caption{The fork hierarchy of some of the popular kernels in Kaggle's Titanic competition}
\label{fig-titanic-script-hierarchy}
\end{figure}
For example, in the kernel \textbf{Best Working Classifier}, the author cites the kernel \textbf{Exploring Survival} as his/her inspiration.
The numbers show how many times other users copy a kernel into their own workspace.
The most popular kernels for the Titanic competition have been copied a total of 44,434 times.
This demonstrates that many of the executed workloads share exact or similar operations.
By using isolated docker containers, the Kaggle platform cannot detect such similarities and must re-execute every operation regardless of how many time it is executed before.

\subsection{Experiment Database}
Experiment databases include data and meta-data of different data analytics and machine learning experiments executed over time \cite{miao2018provdb, vanschoren2014openml, schelter2017automatically, vartak2016m}.
They include different information about datasets, data processing pipelines, machine learning models, execution of machine learning training, and quality of the models. 
Moreover, experiment databases can also store the artifacts generated during the execution of a workload, such as datasets, intermediate data (resulting from applying data transformation operations), and machine learning models and their parameters.
 
Experiment databases can help in designing a better future workload.
For example, users can query the database to find the answer to the following questions, what type of data transformations and model training operations are executed on a dataset and what is the accuracy of the final models.
As a result, users can avoid executing data transformations or model training operations that do not result in high-quality models.
Moreover, experiment databases enable reproducibility and validation of results.
For example, users can query information about the environment and list of operations in a specific workload.
As a result, users can re-execute the workload and compare the results.
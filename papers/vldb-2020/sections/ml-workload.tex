\section{ML Workload Optimizations} \label{sec-ml-workloads}
In this section, we first describe the common types of operations in machine learning workloads.
Then, we discuss how to capture and store the operations in the experiment graph.
Lastly, we discuss how do we utilize the experiment graph to optimize new workloads.

\subsection{Operations in ML Workloads}
We assume the main units of work are data frame (e.g., pandas, R dataframes, and Spark dataframes) like objects that contain one or many columns, where all the data items in one column are of the same data type.
\hladd{Regardless of the type of the workload (interactive or script), we can divide the operations in the ML workloads into 3 categories.}
\begin{table}
\centering
\begin{tabular}{ll}
\hline
	   Feature Extraction & Feature Selection\\ \hline
        feature hasher & variance threshold  \\
        one hot encoding & select k best \\
        count vectorizer& select percentile \\ 
        tfidf transformer & recursive feature elimination \\
        hashing vectorizer & select from model \\
        extract\_patch\_2d &  \\
        \hline
\end{tabular}
\caption{List of feature extraction and feature selection operations}\label{feature-engineering-operations}
\end{table}

\subsubsection{Data and Feature Engineering}
This group of operations typically belongs to three categories, i.e., simple data transformations and aggregations, feature selection, and feature extraction.
All of these operations, receive one or multiple columns of a dataset and return another dataset as result. 
While different data processing tools may provide specialized data transformation and aggregation operations for data frame objects, most of them provide the same or similar operations such as map, reduce, group by, concatenation, and join. 
In Table \ref{feature-engineering-operations}, we show a list of the most common feature extraction and feature selection operations.

\subsubsection{Model Training}
Model training operations are a group of operations that receive a dataset and return a machine learning model.
The result of model training operations can either be used in other data and feature engineering operations (e.g., applying PCA to reduce the number of dimensions of the data) or can be used to perform prediction (for classification and regression tasks) on unseen data.

\subsubsection{Hyperparameter Tuning}
Before training a machine learning model, one has to set the hyperparameters of the model to appropriate values.
Typically, the best values for the hyperparameters of a model vary across different datasets.
The goal of hyperparameter tuning operations is to find the set of hyperparameters that yield the model with the best performance on a dataset.
A hyperparameter tuning operation is defined by a budget, a search method, and a search space.
The budget specifies how many models with different hyperparameter values, within the specified search space, should be trained and the search method specifies what search strategy to incorporate.
The most common search methods are grid search, random search, and Bayesian hyperparameter search \cite{bergstra2012random,snoek2012practical}.


\subsection{Experiment Graph Representation}\label{sub-graph-construction}
\hladd{A machine learning workload can be represented using a directed acyclic graph (DAG).
In the DAG, vertices represent the artifacts, i.e., raw or preprocessed data (represented by data frame objects) and machine learning models resulting from feature engineering and model training operations and edges represent the operations in the workload.
Each workload DAG has one or more root vertices representing the raw datasets that the workload operates on and have no incoming edges.
The collection of all the DAGs of previously executed machine learning workloads forms a rooted graph (with potentially multiple root vertices) which we refer to as the \textit{experiment graph}.}
More formally, we represent the experiment graph by $G(V, E)$.
$V=\{v_i\}, i = 1, \cdots, n$ is the set of all the artifacts in all the workload DAGs.
$E=\{e_i\}, i = 1, \cdots, m$ is the set of all the executed operations in the workload DAGs.
A directed edge $e$ from $v_i$ to $v_j$ in $G(V, E)$ indicates that the artifact $v_j$ is derived from the artifact $v_i$ by applying the operation in $e$.
Every vertex $v$ has the attributes $\langle f, s \rangle$ (accessed by $v.f$ and $v.s$) which represent the frequency, i.e., number of different workloads an artifact appears in, and storage size of the artifact.
Every edge $e$ has the attribute $\langle t \rangle$ (accessed by $e.t$) which represents the run-time (in seconds) of the operation.

\hladd{Inside each vertex, we store the meta-data of the artifact.
Depending on how \textit{useful} an artifact is, we may also store the actual underlying data inside the artifact (Section \ref{sec-materialization})
If the artifact is a raw or a preprocessed dataset, then its meta-data includes the name, type, and total size of each column of the data and its underlying data is represented by the dataframe object (i.e., pandas dataframe \cite{mckinney-proc-scipy-2010}). 
If the artifact is a machine learning model, its meta-data includes the name, type, hyperparameters, and the error metric of the model and its underlying data is consist of the model weights.
Each edge contains the meta-data of the operation it represents, such as the function name, training algorithm, hyperparameters, and in some cases even the source code of the operation.
To uniquely identify an edge, we utilize a hash function which receives as input the operation and its hyperparameters (if it has any).
Since the experiment graph is rooted, we assign a hash value to every vertex which is computed in the following way:
\[
    h(v)= 
\begin{cases}
    id,& \text{if } v \text{ is root}\\
    h\Big(\sum\limits_{e \in in\_edge(v)} (h(e.source) + h(e) ) \Big)  ,              & \text{otherwise}.
\end{cases}
\]
where $in\_edge(v)$ returns the edges with destination $v$. 
Intuitively, the hash of a root vertex is its unique identifier (location on disk or download URL) and the hashes of other vertices are derived by combining the hashes of their direct predecessors and edges which connect them to their predecessors.}

After a machine learning workload is executed, we update the experiment graph by adding the new artifacts and operations.
If any of the artifacts already exist in the graph, their frequency is updated.

Figure \ref{fig-experiment-graph}a shows an example graph constructed from the code in Listing \ref{listing-experiment-graph}.

\begin{lstlisting}[language=Python, caption=Example script,captionpos=b,label = {listing-experiment-graph}]
import numpy as np
import pandas as pd

from sklearn import svm
from sklearn.feature_selection import SelectKBest
from sklearn.feature_extraction.text import CountVectorizer

train = pd.read_csv('../input/train.csv') 
print train.columns # [ad_desc,ts,u_id,price,y]
vectorizer = CountVectorizer()
count_vectorized = vectorizer.fit_transform(train['ad_desc'])
selector =  SelectKBest(k=2)
top_features = selector.fit_transform(train[['ts','u_id','price']], 
				      train['y'])
top_features # print the content of the data frame			     
X = pd.concat([count_vectorized,top_features], axis = 1)
model = svm.SVC().fit(X, train['y'])
\end{lstlisting}

\begin{figure}
\begin{subfigure}[b]{0.4\linewidth}
\centering
\includegraphics[width=0.8\linewidth]{../images/tikz-standalone/example-graph}
\caption{}
\end{subfigure}%
\begin{subfigure}[b]{0.6\linewidth}
\begin{tabular}{lcl}
\hline
operation & label &  hash \\
\hline
project(ad..) & $\langle 2s\rangle$ &p1 \\
project(ts, ..) & $\langle 6s\rangle$ & p2\\
project(y) & $\langle 2s\rangle$ & p3\\
vectorizer.f\_t & $\langle 40s\rangle$ & vf1 \\
selector.f\_t & $\langle 60s\rangle$ & s1 \\
concat & $\langle 10s\rangle$ & c1 \\
merge & $\langle 0s\rangle$ & m\\
svm.fit & $\langle 100s\rangle$ & f\\
\hline
\end{tabular}
\caption{}
\end{subfigure}
\caption{Experiment graph constructed from the Listing \ref{listing-experiment-graph} (a) and the hash of the operations in the scripts (b)}
\label{fig-experiment-graph}
\end{figure}
Table \ref{fig-experiment-graph}b shows both the label of every edge operation, i.e., time, and the hash of the operations and their hyperparameters.
Since at the time of the execution of the script, the experiment graph is empty, all the artifacts (vertices) have a frequency of 1.
In order to represent operations which process multiple input artifacts, e.g., concat and svm.fit operations in Listing \ref{listing-experiment-graph}, we proceed as follows.
First, we merge the vertices representing the artifacts into a single vertex using a merge operator.
The merge operator is a logical operator which does not incur a cost, i.e., it has a run-time of 0 seconds.
The merged vertex is also a logical vertex with no actual attributes which only contains the vertex ids of the merged vertices.
Then, we draw an edge from the merged vertex which represents the actual operation.
For example, in Figure \ref{fig-experiment-graph}a, before applying the concatenation operation, we merge $v_4$ and $v_5$ into $v_6$, then we apply the concatenation operation (c1).
\hladd{Furthermore, when computing the hash of a merged vertex, we take the merge order into account.
For example, the operation svm.fit has $X$ (represented by $v_7$) as first argument and train['y'] (represented by $v_3$) as its second argument.
When computing hash of $v_9$, we combine the predecessors in the same order, i.e., $h(v_9) = h(h(v_7) + m + h(v_3) + m)$. }
 
\subsubsection{Notations and Terms}\label{notations-terms}
Apart from the experiment graph, $G(V, E)$, we need to define several other terms and notations.

\textbf{Task.} A \textit{task} defines the goal of the machine learning workloads. 
A task clearly defines what are the initial datasets which users should analyze and train machine learning models on. 
We refer to the set of initial datasets as $roots$ since during the graph representation, they become the root vertices of the graph, i.e., they have no incoming edges.
A task also specifies the evaluation function for the machine learning workloads.
An example of a task is to train a classification model on the training dataset $D_1$ which maximizes the F1 score on the evaluation dataset $D_2$, where $roots = \{D_1, D_2\}$.

\textbf{Workload graph.} Given a task, a \textit{workload graph} is the (directed acyclic) graph representation of a user-defined script with the goal of analyzing the datasets in $roots$ of the task and training machine learning models which maximize the evaluation function of the task.
A workload graph is rooted at one or all of the vertices representing the root datasets of the task.

\textbf{Terminal model.} A \textit{terminal model} is a vertex inside a workload graph which represents a fully trained machine learning model.
The quality of the terminal model can be measured by the evaluation function of the task.
A workload graph may contain $0$ (workloads for only performing exploratory data analysis) or multiple terminal models.

\hldel{\textbf{Pipeline subgraph.} Any unique subgraph inside a workload graph which starts at one or multiple root vertices and ends with one and only one terminal model is a \textit{pipeline subgraph}.
A workload graph may contain $0$ or multiple pipeline subgraphs.
A vertex (artifact) may belong to multiple pipelines, e.g., a preprocessed dataset which a user utilizes to train multiple machine learning models with different training algorithms.}

\subsection{Workload Optimizer for Kaggle Use Case}
By utilizing an experiment graph, we design a workload optimizer framework.
We integrate the workload optimizer into collaborative data science platforms to improve the execution of the workloads.
Here, we describe the integration process of the workload optimizer for optimizing and executing kernels in the Kaggle use case (our motivating example).
Furthermore, we map the components of the Kaggle use case and our workload optimizer framework to the terms defined in Section \ref{notations-terms}.

Figure \ref{improved-use-case} shows the process of workload optimization in the Kaggle Infrastructure.
Each competition in Kaggle corresponds to a task.
The competitions define the set of datasets (roots) and the evaluation function for assessing machine learning models. 
Given a task, the workload optimization process is as follows.
First, we parse a kernel and construct the \textit{workload graph} (\circledtext{1}).
Then, an \textbf{optimizer} component receives the workload graph and utilizes the existing experiment graph to look for optimization opportunities, namely, reusing the existing operations, warmstarting the model training \hldel{, and for workloads which contain a hyperparameter tuning operation, faster hyperparameter search} (\circledtext{2}).
The result of the optimization is another workload graph which contains precomputed artifacts, warmstarted models \hldel{, and proper search space and/or an initialized Bayesian hyperparameter search process}.
After executing the optimized workload, we return the result to the user (\circledtext{3}).
Depending on the number of \textit{pipeline subgraphs} in the workload, the result of the execution is one or multiple \textit{terminal models} which the user can choose to submit as their solution to the competition.
After the execution, we update the experiment graph using the original workload graph (\circledtext{4}).
Finally, to ensure that we can store the experiment graph given our storage budget, we execute our materialization algorithms to decide what artifacts to materialize (\circledtext{5}).

\begin{figure}
\centering
\includegraphics[width=\columnwidth]{../images/kaggle-workload-optimizer}
\caption{Improving the execution of Kaggle Kernels with Workload Optimizer}
\label{improved-use-case}
\end{figure}

In the experiment graph, each task corresponds to one connected component rooted at vertices representing the tasks root datasets.
Since each task has unique roots and workloads are not allowed to analyze roots belonging to multiple tasks, it is impossible for two tasks to be connected.
In Figure \ref{improved-use-case}, the experiment graph has three connected components representing the three competitions, A, B, and C which have 2, 2, and 1 root datasets, depicted as hollow vertices, respectively.
When optimizing a new Kaggle kernel from competition A, the optimizer component (Step \circledtext{2}) only look for optimization opportunities in the connected component A of the experiment graph.
Similarly, the scope of the materializer (Step \circledtext{5}) is also limited to individual competitions.
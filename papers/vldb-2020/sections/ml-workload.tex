\section{ML Workload Optimizations} \label{sec-ml-workloads}
In this section, we first describe the common types of operations in machine learning workloads.
Then, we discuss how to capture and store the operations in the experiment graph.
Lastly, we discuss how do we utilize the experiment graph to optimize new workloads.

\subsection{Operations in ML Workloads}
We assume the main units of work are dataframe (e.g., Pandas, R Data Frames, and Spark DataFrames) like objects that contain one or many columns, where all the data items in one column are of the same data type.
\hl{We divide the operations in the ML workloads into 3 categories.}
\begin{table}
\centering
\begin{tabular}{ll}
\hline
	   Feature Extraction & Feature Selection\\ \hline
        feature hasher & variance threshold  \\
        one hot encoding & select k best \\
        count vectorizer& select percentile \\ 
        tfidf transformer & recursive feature elimination \\
        hashing vectorizer & select from model \\
        extract\_patch\_2d &  \\
        \hline
\end{tabular}
\caption{List of feature extraction and feature selection operations}\label{feature-engineering-operations}
\end{table}

\subsubsection{Data and Feature Engineering}
This group of operations typically belongs to three categories, i.e., simple data transformations and aggregations, feature selection, and feature extraction.
All of these operations, receive one or multiple columns of a dataset and return another dataset as result. 
While different data processing tools may provide specialized data transformation and aggregation operations for dataframe objects, most of them provide the same or similar operations such as map, reduce, group by, concatenation, and join. 
In Table \ref{feature-engineering-operations}, we show a list of the most common feature extraction and feature selection operations.

\subsubsection{Model Training}
Model training operations are a group of operations that receive a dataset (or one or multiple columns of a dataset) and return a machine learning model.
The result of model training operations can either be used in other data and feature engineering operations (e.g., applying PCA to reduce the number of dimensions of the data) or can be used to perform prediction (for classification and regression tasks) on unseen data.

\subsubsection{Hyperparameter Tuning}
Before training a machine learning model, one has to set the hyperparameters of the model to appropriate values.
Typically, the best values for the hyperparameters of a model vary across different datasets.
The goal of hyperparameter tuning operations is to find the machine learning models with the best performance.
A hyperparameter tuning operation is defined by a budget and a search method.
The budget specifies how many models with different hyperparameter values the operation should train and the search method specifies what search strategy should be incorporated.
We limit our focus to popular search methods, namely, grid search, random search, and Bayesian hyperparameter search \cite{bergstra2012random,snoek2012practical}.

\subsection{Experiment Graph Representation}\label{sub-graph-construction}
To efficiently apply our optimizations, we utilize a graph data structure, called the experiment graph, to store the meta-data and artifacts of the machine learning workloads.
Let $\mathcal{V}=\{v_i\}, i = 1, \cdots, n$ be a collection of artifacts that exist in the workload.
Each artifact is either a raw dataset, a pre-processed dataset resulting from a feature engineering operation, or a model resulting from a model training operation.
Let $\mathcal{E}=\{e_i\}, i = 1, \cdots, m$ be a collection of executed operations that exist in the workload.
A directed edge $e$ from $v_i$ to $v_j$ in $\mathcal{G}(\mathcal{V},\mathcal{E})$ indicates that the artifact $v_j$ is (fully or partially) derived from the artifact $v_i$ by applying the operation in $e$.
Every vertex $v$ has the attribute $\langle s \rangle$ (accessed by $v.s$), which represents the storage size of artifact when materialized.
Every edge $e$ has the attributes $\langle f, t\rangle$ (accessed by $e.f$ and $e.t$), where $f$ represents the frequency of the operation (the number of times the operation has been executed) and $t$ represents the average run-time (in seconds) of the operation.
Each vertex contains meta-data about the artifacts, such as the name and type of the columns for datasets and name, size, the value of parameters and hyperparameters, and the error metric of the models.
Each edge contains the meta-data about the operation, such as the function name, training algorithm, hyperparameters, and in some cases even the source code of the operation.
When a new machine learning workload is executed, we extend the graph to capture the new operations and artifacts.
If an operation already exists in the graph, we update the frequency and average run-time attributes.
Otherwise, we add a new edge and vertex to the experiment graph, representing the new operation and the artifact.
Figure \ref{fig-experiment-graph}a shows an example graph constructed from the code in Listing \ref{listing-experiment-graph}.
To uniquely identify an edge, we utilize a hash function that receives as input the operation and its hyperparameters (if it has any).

\begin{lstlisting}[language=Python, caption=Example script,captionpos=b,label = {listing-experiment-graph}]
import numpy as np
import pandas as pd

from sklearn import svm
from sklearn.feature_selection import SelectKBest
from sklearn.feature_extraction.text import CountVectorizer

train = pd.read_csv('../input/train.csv') 
print train.columns # [ad_desc,ts,u_id,price,y]
vectorizer = CountVectorizer()
count_vectorized = vectorizer.fit_transform(train['ad_desc'])
selector =  SelectKBest(k=2)
top_features = selector.fit_transform(train[['ts','u_id','price']], 
				      train['y'])
X = pd.concat([count_vectorized,top_features], axis = 1)
model = svm.SVC()
model.fit(X, train['y'])
\end{lstlisting}

\begin{figure}
\begin{subfigure}[b]{0.4\linewidth}
\centering
\documentclass{standalone}
\usepackage{tikz}
\usetikzlibrary{graphdrawing, graphs, quotes, positioning,arrows, backgrounds, math, calc, shapes, positioning}
\usegdlibrary{trees}
\begin{document}
\begin{tikzpicture}
%\draw[help lines]  (-2,0) grid (6,6);
\tikzstyle{every node}=[inner sep=0.02cm]
\node (train) [ellipse, draw] at (2,6) {$train$};
% layer 1
\node (ad) [ellipse, draw, below left = of train] {$ad\_desc$};
\node (forselection) [ellipse, draw, below = of train] {$t\_subset$};
\node (y) [ellipse, draw, below right = of train] {$y$};
% layer 2
\node (cv) [ellipse, draw, below = of ad] {$cnt\_vect$};
\node(sk) [ellipse, draw, below = of forselection] {$top\_features$};
% layer 3
%\node (merged1) [circle, draw] at (1.5, 3.5) {$v_6$};
\node (cvsk) [ellipse, draw, below = of sk] {$X$};
% layer 4
%\node(merged2) [circle, draw] at (3, 2.8) {$v_8$};
% layer 5
\node(model) [ellipse, draw, below right = of cvsk]  {$model$};

\graph [grow down,edge quotes ={inner sep=1pt}, edges ={thick},radius=.2cm, nodes={circle, draw,font =\small}]{
(train) [label=train]
-> [anchor=east, align=center,"p1"] (ad)
-> [anchor=east,align=center,"vf1"] (cv)
%-> [anchor=east,align=center,"m"](merged1) 
-> [anchor=east,align=center,"c1"](cvsk) 
%-> [anchor=south, align=center,"m"](merged2) ;
-> [anchor=east,auto=false,align=center,"f"] (model);

(train) 
-> [anchor=east,align=center, "p2"] (forselection)
-> [anchor=east,auto=false,align=center,"s1"] (sk)
-> [anchor=east,align=center,"c1"](cvsk) ;

(train) 
-> [anchor=west, align=center,"p3"]   (y)
%-> [anchor=west, align=center,"m"](merged2) 
-> [anchor=east,auto=false,align=center,"f"] (model);
};
\end{tikzpicture}
\end{document}
\caption{}
\end{subfigure}%
\begin{subfigure}[b]{0.6\linewidth}
\begin{tabular}{lcl}
\hline
operation & label &  hash \\
\hline
project(ad..) & $\langle 2, 2\rangle$ &p1 \\
project(ts, ..) & $\langle 1, 6\rangle$ & p2\\
project(y) & $\langle 1, 2\rangle$ & p3\\
vectorizer.f\_t & $\langle 2, 40\rangle$ & v1 \\
selector.f\_t & $\langle 1, 60\rangle$ & s1 \\
concat & $\langle 1, 10\rangle$ & c1 \\
merge & $\langle 1, 0 \rangle$ & m\\
model.fit & $\langle 1, 100\rangle$ & f\\
\hline
\end{tabular}
\caption{}
\end{subfigure}
\caption{Experiment graph constructed from the Listing \ref{listing-experiment-graph} (a) and the hash of the operations in the scripts (b)}
\label{fig-experiment-graph}
\end{figure}
Table \ref{fig-experiment-graph}b shows both the label of every edge operation, i.e., frequency and time, and the hash of the operations and their hyperparameters.
We assume the operations project(ad..) and vectorizer.f\_t already exist in the experiment graph.
Therefore, when adding the new operations, we must update their frequency to 2.
In order to represent operations that process multiple input artifacts, e.g., concat and model.fit operations in Listing \ref{listing-experiment-graph}, we proceed as follows.
First, we merge the vertices representing the artifacts into a single vertex using a merge operator.
The merge operator is a logical operator which does not incur a cost, i.e., it has run-time of 0 seconds.
The merged vertex is also a logical vertex with no actual attributes which only contains the vertex id of the merged vertices.
Then, we draw an edge from the merged vertex which represents the actual operation.
For example, in Figure \ref{fig-experiment-graph}a, before applying the concatenation operation, we merge $v_4$ and $v_5$ into $v_6$, then we apply the concatenation operation (c1).
This is a critical step for the materialization algorithms and optimization strategies.

%It is important to note that based our definition of a task in Section \ref{sec-introduction}, the workloads belong to multiple different tasks, the constructed graph will contain one connected component for every task.
%In the next sections, we assume that the experiment graph contains information about 1 task, although, all the methods described can be applied to multiple tasks as well.
\subsection{Workload Optimizer for Kaggle Use Case}
By utilizing an experiment graph, we can improve the execution of the kernels in Kaggle.
This results in faster execution of the kernels and less time spent in the queue.

Figure \ref{improved-use-case} shows how we utilize the experiment graph in the Kaggle Infrastructure. 
The workflow is as follows.
First, we parse a workload (i.e., a kernel) and construct the workload graph.
Then, an \textbf{optimizer} component receives the workload graph and utilizes the existing experiment graph and look for optimization opportunities, namely, reusing the existing operations, warm starting the model training, and for workloads that contain a hyperparameter tuning operation, an improved tuning.
The result of the optimization is another workload graph which contains precomputed artifacts, warmstarted models, and proper search space and/or an initialized Bayesian hyperparameter search process.
After executing the optimized workload, we return the result to the user.
After the execution, we update the experiment graph using original workload graph.
Finally, to ensure the we can store the experiment graph given our storage limit, we run our materialization algorithms to decide what artifacts to materialize and what artifacts to remove.

\begin{figure}
\centering
\includegraphics[width=\columnwidth]{../images/kaggle-workload-optimizer}
\caption{Improving the execution of Kaggle Kernels with Workload Optimizer}
\label{improved-use-case}
\end{figure}

It is important to note that we restrict the scope of our optimizations and materialization algorithms to one machine learning task. 
In Kaggle, each competition defines the machine learning task, i.e., one or multiple raw training datasets, a validation data set, and an evaluation function which computes the quality/error rate of the model on the validation dataset.
While the experiment graph contains artifacts from all the tasks, each task corresponds to a unique connected component of the experiment graph.
All the artifacts in a task's connected component are reachable from the raw training datasets defined in the task.
When optimizing a new workload (i.e., a Kaggle kernel), we only utilize the connected component the task the workload is trying to solve (i.e., the Kaggle competition) and do not utilize the rest of the graph for optimization.


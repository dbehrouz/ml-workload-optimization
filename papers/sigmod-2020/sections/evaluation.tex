\section{Evaluation} \label{sec-evaluation} 
In this section, we evaluate the performance of our collaborative optimizer.
We focus on investigating the effect of our materialization and reuse algorithms on the execution cost of workloads in collaborative environments.
We first describe the setup, i.e., the hardware specification and the experiment workloads.
Then, we show the run-time improvement of our optimizer.
Finally, we investigate the effect of the individual contributions, i.e., materialization and reuse algorithms, on the run-time and storage cost.
\begin{table*}[h]
\begin{tabular}{lp{0.84\textwidth}rr}
\hline
\textbf{$ID$} & \textbf{$Description$}& \textbf{$N$}& \textbf{$S$}   \\
\hline
1 &  \parbox[t]{0.84\textwidth}{\linespread{0.5}\selectfont \small A real script titled 'start-here-a-gentle-introduction'. It includes several feature engineering operations before training logistic regression, random forest, and gradient boosted tree models.} & 397 & 14.5\\[0.4cm]

2 &   \parbox[t]{0.84\textwidth}{\linespread{0.5}\selectfont \small A real script titled 'introduction-to-manual-feature-engineering'. It joins multiple datasets, preprocesses the datasets to generate features, and trains gradient boosted tree models on the generated features.} & 406 & 25\\[0.4cm]

3 &   \parbox[t]{0.84\textwidth}{\linespread{0.5}\selectfont \small A real script titled 'introduction-to-manual-feature-engineering-p2'. It is similar to Workload 2, with the resulting preprocessed datasets having more features.} & 146 & 83.5\\[0.15cm]

4 & \parbox[t]{0.84\textwidth}{\linespread{0.5}\selectfont \small A modified version of Workload 1 submitted by the Kaggle user "crldata". It trains a gradient boosted tree with a different set of hyperparameters.} & 280 & 10\\[0.4cm]

5 & \parbox[t]{0.84\textwidth}{\linespread{0.5}\selectfont \small A modified version of Workload 1 submitted by the Kaggle user "taozhongxiao". It performs random and grid search for gradient boosted tree model using generated features of Workload 1.} & 402 & 13.8\\[0.4cm]

6 & \parbox[t]{0.84\textwidth}{\linespread{0.5}\selectfont \small A custom script based on Workloads 2 and 4. It trains a gradient boosted tree on the generated features of Workload 2.} & 121 & 21\\[0.15cm]

7 & \parbox[t]{0.84\textwidth}{\linespread{0.5}\selectfont \small A custom script based on Workload 3 and 4. It trains a gradient boosted tree on the generated features of Workload 3.} & 145 & 83\\[0.15cm]

8 & \parbox[t]{0.84\textwidth}{\linespread{0.5}\selectfont \small A custom script that joins the features of Workloads 1 and 2. Then, similar to Workload 4, it trains a gradient boosted tree on the joined dataset.} & 341 & 21.1\\
\hline
\end{tabular}
\caption{Description of Kaggle workloads. $N$ is number of the artifacts and $S$ is the total size of the artifacts in GB.}
\label{kaggle-workload}
\end{table*}
\subsection{Setup}
We execute all the experiments on a server running Linux Ubuntu with 128 GB of RAM.
We implement a prototype of our system in python 2.7.12.
In the prototype, we implement EG using python's NetworkX 2.2 \cite{hagberg2008exploring}.
We run every experiment 3 times and report the average results with error bars.
We evaluate our proposed optimizations on two different sets of workloads\footnote{We plan to release the entire codebase, i.e., the optimizer system and experiment workload scripts}.

\textbf{Kaggle workloads.} 
In the first set of workloads, we recreate the collaborative environment of the use case in Section \ref{sec-background}.
We use eight workloads, which generate a total of 130 GB of artifacts.
We introduced three workloads in the use case.
We retrieve two more workloads from the same Kaggle competition.
We also design three workloads based on the existing ones.
Table \ref{kaggle-workload} shows details of the workloads we utilize in our experiments.
There are 9 source datasets with a total size of 2.5 GB\footnote{https://www.kaggle.com/c/home-credit-default-risk/data}.
Unless specified otherwise, we use storage-aware materialization with a budget of 16 GB (approximately 10\% of the combined size of the artifacts) and $\alpha=0.5$.
For all the experiments, we execute the workloads in order (from 1 to 8) and report the results.
We use the Kaggle workloads to investigate the effect of end-to-end optimization, storage-aware and heuristics-based materializations, and our reuse algorithm.

\textbf{OpenML workloads.} In the OpenML workloads, we retrieve all the scikit-learn pipelines for Task 31 from the OpenML platform\footnote{https://www.openml.org/t/31}, i.e., classifying customers as good or bad credit risks using the German Credit data from the UCI repository \cite{asuncion2007uci}.
Task 31 is the most popular task with the highest number of ML pipeline runs.
For every task, OpenML stores a log of all the runs.
We extracted the first 2000 runs of Task 31 from the logs.
The dataset is small, and the total size of the artifacts after executing the 2000 runs is 1.5 GB.
For all the experiments, we execute the 2000 runs in order and report the result.
We use the OpenML workloads to show the effects of the quality ratio (i.e., $\alpha$ in the utility function for artifact materialization) and model warmstarting on workload execution in collaborative environments.
Unless specified otherwise,  we use storage-aware materialization with a budget of 100 MB and $\alpha=0.5$.
\subsection{End-to-end Optimization}
In this experiment, we evaluate the impact of our collaborative optimizer on the Kaggle workload.
In our motivating, we describe three workloads (Workloads 1-3 of Table \ref{kaggle-workload}).
Kaggle reports users have copied and modified these workloads a total of 7000 times.
Therefore, at the very least, users execute these workloads 7000 times.
Figures \ref{exp-reuse-kaggle-same-workload}(a)-(c) show the result of repeating the execution of each workload twice.
Before the first run, EG is empty; therefore, both the baseline (KG) and our collaborative optimizer (CO) must execute all the operation in the workloads.
In Workload 1, the run-time of CO is slightly larger than KG in the first run.
Workload 1 executes two alignment operations.
An alignment operation receives two datasets, removes all the columns that do not appear in both datasets, and returns the resulting two datasets.
In CO, we need to measure the precise compute-cost of every artifact.
This is not possible for operations that return multiple artifacts.
Thus, we re-implemented the alignment operation, which is less optimized than the baseline implementation.
In Workloads 2 and 3, CO outperforms KG even in the first.
Both Workloads 2 and 3 contain many redundant operations.
The local pruning step in CO identifies the redundancies and only execute such operations once.
In the second run of the workloads, CO reduces the run-time by one order of magnitude for workloads 2 and 3.
Workload 1 executes an external visualization command that computes a bivariate kernel density estimate, which incurs a large overhead.
Since our collaborative optimizer does not materialize such external information, it must re-execute the visualization command; thus, resulting in a smaller run-time reduction.
\begin{figure}[h]
\begin{subfigure}[b]{0.33\linewidth}
\centering
 \resizebox{\columnwidth}{!}{%
\input{../images/experiment-results/kaggle_home_credit/execution_time/repetition/start_here_a_gentle_introduction.pgf}%
}
\caption{Workload 1}
\end{subfigure}%
\begin{subfigure}[b]{0.33\linewidth}
\centering
 \resizebox{\columnwidth}{!}{%
%% Creator: Matplotlib, PGF backend
%%
%% To include the figure in your LaTeX document, write
%%   \input{<filename>.pgf}
%%
%% Make sure the required packages are loaded in your preamble
%%   \usepackage{pgf}
%%
%% Figures using additional raster images can only be included by \input if
%% they are in the same directory as the main LaTeX file. For loading figures
%% from other directories you can use the `import` package
%%   \usepackage{import}
%% and then include the figures with
%%   \import{<path to file>}{<filename>.pgf}
%%
%% Matplotlib used the following preamble
%%   \usepackage{fontspec}
%%   \setmonofont{Andale Mono}
%%
\begingroup%
\makeatletter%
\begin{pgfpicture}%
\pgfpathrectangle{\pgfpointorigin}{\pgfqpoint{4.964458in}{5.268083in}}%
\pgfusepath{use as bounding box, clip}%
\begin{pgfscope}%
\pgfsetbuttcap%
\pgfsetmiterjoin%
\definecolor{currentfill}{rgb}{1.000000,1.000000,1.000000}%
\pgfsetfillcolor{currentfill}%
\pgfsetlinewidth{0.000000pt}%
\definecolor{currentstroke}{rgb}{1.000000,1.000000,1.000000}%
\pgfsetstrokecolor{currentstroke}%
\pgfsetdash{}{0pt}%
\pgfpathmoveto{\pgfqpoint{0.000000in}{0.000000in}}%
\pgfpathlineto{\pgfqpoint{4.964458in}{0.000000in}}%
\pgfpathlineto{\pgfqpoint{4.964458in}{5.268083in}}%
\pgfpathlineto{\pgfqpoint{0.000000in}{5.268083in}}%
\pgfpathclose%
\pgfusepath{fill}%
\end{pgfscope}%
\begin{pgfscope}%
\pgfsetbuttcap%
\pgfsetmiterjoin%
\definecolor{currentfill}{rgb}{1.000000,1.000000,1.000000}%
\pgfsetfillcolor{currentfill}%
\pgfsetlinewidth{0.000000pt}%
\definecolor{currentstroke}{rgb}{0.000000,0.000000,0.000000}%
\pgfsetstrokecolor{currentstroke}%
\pgfsetstrokeopacity{0.000000}%
\pgfsetdash{}{0pt}%
\pgfpathmoveto{\pgfqpoint{1.535194in}{1.233139in}}%
\pgfpathlineto{\pgfqpoint{4.499430in}{1.233139in}}%
\pgfpathlineto{\pgfqpoint{4.499430in}{4.674805in}}%
\pgfpathlineto{\pgfqpoint{1.535194in}{4.674805in}}%
\pgfpathclose%
\pgfusepath{fill}%
\end{pgfscope}%
\begin{pgfscope}%
\definecolor{textcolor}{rgb}{0.150000,0.150000,0.150000}%
\pgfsetstrokecolor{textcolor}%
\pgfsetfillcolor{textcolor}%
\pgftext[x=2.276253in,y=1.069250in,,top]{\color{textcolor}\rmfamily\fontsize{36.000000}{43.200000}\selectfont 1}%
\end{pgfscope}%
\begin{pgfscope}%
\definecolor{textcolor}{rgb}{0.150000,0.150000,0.150000}%
\pgfsetstrokecolor{textcolor}%
\pgfsetfillcolor{textcolor}%
\pgftext[x=3.758371in,y=1.069250in,,top]{\color{textcolor}\rmfamily\fontsize{36.000000}{43.200000}\selectfont 2}%
\end{pgfscope}%
\begin{pgfscope}%
\definecolor{textcolor}{rgb}{0.150000,0.150000,0.150000}%
\pgfsetstrokecolor{textcolor}%
\pgfsetfillcolor{textcolor}%
\pgftext[x=3.017312in,y=0.569194in,,top]{\color{textcolor}\rmfamily\fontsize{38.000000}{45.600000}\selectfont Run}%
\end{pgfscope}%
\begin{pgfscope}%
\pgfpathrectangle{\pgfqpoint{1.535194in}{1.233139in}}{\pgfqpoint{2.964236in}{3.441667in}} %
\pgfusepath{clip}%
\pgfsetroundcap%
\pgfsetroundjoin%
\pgfsetlinewidth{0.803000pt}%
\definecolor{currentstroke}{rgb}{0.800000,0.800000,0.800000}%
\pgfsetstrokecolor{currentstroke}%
\pgfsetdash{}{0pt}%
\pgfpathmoveto{\pgfqpoint{1.535194in}{1.233139in}}%
\pgfpathlineto{\pgfqpoint{4.499430in}{1.233139in}}%
\pgfusepath{stroke}%
\end{pgfscope}%
\begin{pgfscope}%
\definecolor{textcolor}{rgb}{0.150000,0.150000,0.150000}%
\pgfsetstrokecolor{textcolor}%
\pgfsetfillcolor{textcolor}%
\pgftext[x=1.141805in,y=1.059639in,left,base]{\color{textcolor}\rmfamily\fontsize{36.000000}{43.200000}\selectfont 0}%
\end{pgfscope}%
\begin{pgfscope}%
\pgfpathrectangle{\pgfqpoint{1.535194in}{1.233139in}}{\pgfqpoint{2.964236in}{3.441667in}} %
\pgfusepath{clip}%
\pgfsetroundcap%
\pgfsetroundjoin%
\pgfsetlinewidth{0.803000pt}%
\definecolor{currentstroke}{rgb}{0.800000,0.800000,0.800000}%
\pgfsetstrokecolor{currentstroke}%
\pgfsetdash{}{0pt}%
\pgfpathmoveto{\pgfqpoint{1.535194in}{3.381485in}}%
\pgfpathlineto{\pgfqpoint{4.499430in}{3.381485in}}%
\pgfusepath{stroke}%
\end{pgfscope}%
\begin{pgfscope}%
\definecolor{textcolor}{rgb}{0.150000,0.150000,0.150000}%
\pgfsetstrokecolor{textcolor}%
\pgfsetfillcolor{textcolor}%
\pgftext[x=0.682805in,y=3.207985in,left,base]{\color{textcolor}\rmfamily\fontsize{36.000000}{43.200000}\selectfont 500}%
\end{pgfscope}%
\begin{pgfscope}%
\definecolor{textcolor}{rgb}{0.150000,0.150000,0.150000}%
\pgfsetstrokecolor{textcolor}%
\pgfsetfillcolor{textcolor}%
\pgftext[x=0.627250in,y=2.953972in,,bottom,rotate=90.000000]{\color{textcolor}\rmfamily\fontsize{38.000000}{45.600000}\selectfont Run Time (s)}%
\end{pgfscope}%
\begin{pgfscope}%
\pgfpathrectangle{\pgfqpoint{1.535194in}{1.233139in}}{\pgfqpoint{2.964236in}{3.441667in}} %
\pgfusepath{clip}%
\pgfsetbuttcap%
\pgfsetmiterjoin%
\definecolor{currentfill}{rgb}{0.798529,0.536765,0.389706}%
\pgfsetfillcolor{currentfill}%
\pgfsetlinewidth{0.803000pt}%
\definecolor{currentstroke}{rgb}{0.827451,0.827451,0.827451}%
\pgfsetstrokecolor{currentstroke}%
\pgfsetdash{}{0pt}%
\pgfpathmoveto{\pgfqpoint{1.683406in}{1.233139in}}%
\pgfpathlineto{\pgfqpoint{2.276253in}{1.233139in}}%
\pgfpathlineto{\pgfqpoint{2.276253in}{4.460613in}}%
\pgfpathlineto{\pgfqpoint{1.683406in}{4.460613in}}%
\pgfpathclose%
\pgfusepath{stroke,fill}%
\end{pgfscope}%
\begin{pgfscope}%
\pgfsetbuttcap%
\pgfsetmiterjoin%
\definecolor{currentfill}{rgb}{0.798529,0.536765,0.389706}%
\pgfsetfillcolor{currentfill}%
\pgfsetlinewidth{0.803000pt}%
\definecolor{currentstroke}{rgb}{0.827451,0.827451,0.827451}%
\pgfsetstrokecolor{currentstroke}%
\pgfsetdash{}{0pt}%
\pgfpathrectangle{\pgfqpoint{1.535194in}{1.233139in}}{\pgfqpoint{2.964236in}{3.441667in}} %
\pgfusepath{clip}%
\pgfpathmoveto{\pgfqpoint{1.683406in}{1.233139in}}%
\pgfpathlineto{\pgfqpoint{2.276253in}{1.233139in}}%
\pgfpathlineto{\pgfqpoint{2.276253in}{4.460613in}}%
\pgfpathlineto{\pgfqpoint{1.683406in}{4.460613in}}%
\pgfpathclose%
\pgfusepath{clip}%
\pgfsys@defobject{currentpattern}{\pgfqpoint{0in}{0in}}{\pgfqpoint{1in}{1in}}{%
\begin{pgfscope}%
\pgfpathrectangle{\pgfqpoint{0in}{0in}}{\pgfqpoint{1in}{1in}}%
\pgfusepath{clip}%
\pgfpathmoveto{\pgfqpoint{-0.500000in}{0.500000in}}%
\pgfpathlineto{\pgfqpoint{0.500000in}{1.500000in}}%
\pgfpathmoveto{\pgfqpoint{-0.333333in}{0.333333in}}%
\pgfpathlineto{\pgfqpoint{0.666667in}{1.333333in}}%
\pgfpathmoveto{\pgfqpoint{-0.166667in}{0.166667in}}%
\pgfpathlineto{\pgfqpoint{0.833333in}{1.166667in}}%
\pgfpathmoveto{\pgfqpoint{0.000000in}{0.000000in}}%
\pgfpathlineto{\pgfqpoint{1.000000in}{1.000000in}}%
\pgfpathmoveto{\pgfqpoint{0.166667in}{-0.166667in}}%
\pgfpathlineto{\pgfqpoint{1.166667in}{0.833333in}}%
\pgfpathmoveto{\pgfqpoint{0.333333in}{-0.333333in}}%
\pgfpathlineto{\pgfqpoint{1.333333in}{0.666667in}}%
\pgfpathmoveto{\pgfqpoint{0.500000in}{-0.500000in}}%
\pgfpathlineto{\pgfqpoint{1.500000in}{0.500000in}}%
\pgfusepath{stroke}%
\end{pgfscope}%
}%
\pgfsys@transformshift{1.683406in}{1.233139in}%
\pgfsys@useobject{currentpattern}{}%
\pgfsys@transformshift{1in}{0in}%
\pgfsys@transformshift{-1in}{0in}%
\pgfsys@transformshift{0in}{1in}%
\pgfsys@useobject{currentpattern}{}%
\pgfsys@transformshift{1in}{0in}%
\pgfsys@transformshift{-1in}{0in}%
\pgfsys@transformshift{0in}{1in}%
\pgfsys@useobject{currentpattern}{}%
\pgfsys@transformshift{1in}{0in}%
\pgfsys@transformshift{-1in}{0in}%
\pgfsys@transformshift{0in}{1in}%
\pgfsys@useobject{currentpattern}{}%
\pgfsys@transformshift{1in}{0in}%
\pgfsys@transformshift{-1in}{0in}%
\pgfsys@transformshift{0in}{1in}%
\end{pgfscope}%
\begin{pgfscope}%
\pgfpathrectangle{\pgfqpoint{1.535194in}{1.233139in}}{\pgfqpoint{2.964236in}{3.441667in}} %
\pgfusepath{clip}%
\pgfsetbuttcap%
\pgfsetmiterjoin%
\definecolor{currentfill}{rgb}{0.798529,0.536765,0.389706}%
\pgfsetfillcolor{currentfill}%
\pgfsetlinewidth{0.803000pt}%
\definecolor{currentstroke}{rgb}{0.827451,0.827451,0.827451}%
\pgfsetstrokecolor{currentstroke}%
\pgfsetdash{}{0pt}%
\pgfpathmoveto{\pgfqpoint{3.165524in}{1.233139in}}%
\pgfpathlineto{\pgfqpoint{3.758371in}{1.233139in}}%
\pgfpathlineto{\pgfqpoint{3.758371in}{4.504637in}}%
\pgfpathlineto{\pgfqpoint{3.165524in}{4.504637in}}%
\pgfpathclose%
\pgfusepath{stroke,fill}%
\end{pgfscope}%
\begin{pgfscope}%
\pgfsetbuttcap%
\pgfsetmiterjoin%
\definecolor{currentfill}{rgb}{0.798529,0.536765,0.389706}%
\pgfsetfillcolor{currentfill}%
\pgfsetlinewidth{0.803000pt}%
\definecolor{currentstroke}{rgb}{0.827451,0.827451,0.827451}%
\pgfsetstrokecolor{currentstroke}%
\pgfsetdash{}{0pt}%
\pgfpathrectangle{\pgfqpoint{1.535194in}{1.233139in}}{\pgfqpoint{2.964236in}{3.441667in}} %
\pgfusepath{clip}%
\pgfpathmoveto{\pgfqpoint{3.165524in}{1.233139in}}%
\pgfpathlineto{\pgfqpoint{3.758371in}{1.233139in}}%
\pgfpathlineto{\pgfqpoint{3.758371in}{4.504637in}}%
\pgfpathlineto{\pgfqpoint{3.165524in}{4.504637in}}%
\pgfpathclose%
\pgfusepath{clip}%
\pgfsys@defobject{currentpattern}{\pgfqpoint{0in}{0in}}{\pgfqpoint{1in}{1in}}{%
\begin{pgfscope}%
\pgfpathrectangle{\pgfqpoint{0in}{0in}}{\pgfqpoint{1in}{1in}}%
\pgfusepath{clip}%
\pgfpathmoveto{\pgfqpoint{-0.500000in}{0.500000in}}%
\pgfpathlineto{\pgfqpoint{0.500000in}{1.500000in}}%
\pgfpathmoveto{\pgfqpoint{-0.333333in}{0.333333in}}%
\pgfpathlineto{\pgfqpoint{0.666667in}{1.333333in}}%
\pgfpathmoveto{\pgfqpoint{-0.166667in}{0.166667in}}%
\pgfpathlineto{\pgfqpoint{0.833333in}{1.166667in}}%
\pgfpathmoveto{\pgfqpoint{0.000000in}{0.000000in}}%
\pgfpathlineto{\pgfqpoint{1.000000in}{1.000000in}}%
\pgfpathmoveto{\pgfqpoint{0.166667in}{-0.166667in}}%
\pgfpathlineto{\pgfqpoint{1.166667in}{0.833333in}}%
\pgfpathmoveto{\pgfqpoint{0.333333in}{-0.333333in}}%
\pgfpathlineto{\pgfqpoint{1.333333in}{0.666667in}}%
\pgfpathmoveto{\pgfqpoint{0.500000in}{-0.500000in}}%
\pgfpathlineto{\pgfqpoint{1.500000in}{0.500000in}}%
\pgfusepath{stroke}%
\end{pgfscope}%
}%
\pgfsys@transformshift{3.165524in}{1.233139in}%
\pgfsys@useobject{currentpattern}{}%
\pgfsys@transformshift{1in}{0in}%
\pgfsys@transformshift{-1in}{0in}%
\pgfsys@transformshift{0in}{1in}%
\pgfsys@useobject{currentpattern}{}%
\pgfsys@transformshift{1in}{0in}%
\pgfsys@transformshift{-1in}{0in}%
\pgfsys@transformshift{0in}{1in}%
\pgfsys@useobject{currentpattern}{}%
\pgfsys@transformshift{1in}{0in}%
\pgfsys@transformshift{-1in}{0in}%
\pgfsys@transformshift{0in}{1in}%
\pgfsys@useobject{currentpattern}{}%
\pgfsys@transformshift{1in}{0in}%
\pgfsys@transformshift{-1in}{0in}%
\pgfsys@transformshift{0in}{1in}%
\end{pgfscope}%
\begin{pgfscope}%
\pgfpathrectangle{\pgfqpoint{1.535194in}{1.233139in}}{\pgfqpoint{2.964236in}{3.441667in}} %
\pgfusepath{clip}%
\pgfsetbuttcap%
\pgfsetmiterjoin%
\definecolor{currentfill}{rgb}{0.347059,0.458824,0.641176}%
\pgfsetfillcolor{currentfill}%
\pgfsetlinewidth{0.803000pt}%
\definecolor{currentstroke}{rgb}{0.827451,0.827451,0.827451}%
\pgfsetstrokecolor{currentstroke}%
\pgfsetdash{}{0pt}%
\pgfpathmoveto{\pgfqpoint{2.276253in}{1.233139in}}%
\pgfpathlineto{\pgfqpoint{2.869100in}{1.233139in}}%
\pgfpathlineto{\pgfqpoint{2.869100in}{4.333691in}}%
\pgfpathlineto{\pgfqpoint{2.276253in}{4.333691in}}%
\pgfpathclose%
\pgfusepath{stroke,fill}%
\end{pgfscope}%
\begin{pgfscope}%
\pgfsetbuttcap%
\pgfsetmiterjoin%
\definecolor{currentfill}{rgb}{0.347059,0.458824,0.641176}%
\pgfsetfillcolor{currentfill}%
\pgfsetlinewidth{0.803000pt}%
\definecolor{currentstroke}{rgb}{0.827451,0.827451,0.827451}%
\pgfsetstrokecolor{currentstroke}%
\pgfsetdash{}{0pt}%
\pgfpathrectangle{\pgfqpoint{1.535194in}{1.233139in}}{\pgfqpoint{2.964236in}{3.441667in}} %
\pgfusepath{clip}%
\pgfpathmoveto{\pgfqpoint{2.276253in}{1.233139in}}%
\pgfpathlineto{\pgfqpoint{2.869100in}{1.233139in}}%
\pgfpathlineto{\pgfqpoint{2.869100in}{4.333691in}}%
\pgfpathlineto{\pgfqpoint{2.276253in}{4.333691in}}%
\pgfpathclose%
\pgfusepath{clip}%
\pgfsys@defobject{currentpattern}{\pgfqpoint{0in}{0in}}{\pgfqpoint{1in}{1in}}{%
\begin{pgfscope}%
\pgfpathrectangle{\pgfqpoint{0in}{0in}}{\pgfqpoint{1in}{1in}}%
\pgfusepath{clip}%
\pgfpathmoveto{\pgfqpoint{-0.500000in}{0.500000in}}%
\pgfpathlineto{\pgfqpoint{0.500000in}{1.500000in}}%
\pgfpathmoveto{\pgfqpoint{-0.333333in}{0.333333in}}%
\pgfpathlineto{\pgfqpoint{0.666667in}{1.333333in}}%
\pgfpathmoveto{\pgfqpoint{-0.166667in}{0.166667in}}%
\pgfpathlineto{\pgfqpoint{0.833333in}{1.166667in}}%
\pgfpathmoveto{\pgfqpoint{0.000000in}{0.000000in}}%
\pgfpathlineto{\pgfqpoint{1.000000in}{1.000000in}}%
\pgfpathmoveto{\pgfqpoint{0.166667in}{-0.166667in}}%
\pgfpathlineto{\pgfqpoint{1.166667in}{0.833333in}}%
\pgfpathmoveto{\pgfqpoint{0.333333in}{-0.333333in}}%
\pgfpathlineto{\pgfqpoint{1.333333in}{0.666667in}}%
\pgfpathmoveto{\pgfqpoint{0.500000in}{-0.500000in}}%
\pgfpathlineto{\pgfqpoint{1.500000in}{0.500000in}}%
\pgfpathmoveto{\pgfqpoint{-0.500000in}{0.500000in}}%
\pgfpathlineto{\pgfqpoint{0.500000in}{-0.500000in}}%
\pgfpathmoveto{\pgfqpoint{-0.333333in}{0.666667in}}%
\pgfpathlineto{\pgfqpoint{0.666667in}{-0.333333in}}%
\pgfpathmoveto{\pgfqpoint{-0.166667in}{0.833333in}}%
\pgfpathlineto{\pgfqpoint{0.833333in}{-0.166667in}}%
\pgfpathmoveto{\pgfqpoint{0.000000in}{1.000000in}}%
\pgfpathlineto{\pgfqpoint{1.000000in}{0.000000in}}%
\pgfpathmoveto{\pgfqpoint{0.166667in}{1.166667in}}%
\pgfpathlineto{\pgfqpoint{1.166667in}{0.166667in}}%
\pgfpathmoveto{\pgfqpoint{0.333333in}{1.333333in}}%
\pgfpathlineto{\pgfqpoint{1.333333in}{0.333333in}}%
\pgfpathmoveto{\pgfqpoint{0.500000in}{1.500000in}}%
\pgfpathlineto{\pgfqpoint{1.500000in}{0.500000in}}%
\pgfusepath{stroke}%
\end{pgfscope}%
}%
\pgfsys@transformshift{2.276253in}{1.233139in}%
\pgfsys@useobject{currentpattern}{}%
\pgfsys@transformshift{1in}{0in}%
\pgfsys@transformshift{-1in}{0in}%
\pgfsys@transformshift{0in}{1in}%
\pgfsys@useobject{currentpattern}{}%
\pgfsys@transformshift{1in}{0in}%
\pgfsys@transformshift{-1in}{0in}%
\pgfsys@transformshift{0in}{1in}%
\pgfsys@useobject{currentpattern}{}%
\pgfsys@transformshift{1in}{0in}%
\pgfsys@transformshift{-1in}{0in}%
\pgfsys@transformshift{0in}{1in}%
\pgfsys@useobject{currentpattern}{}%
\pgfsys@transformshift{1in}{0in}%
\pgfsys@transformshift{-1in}{0in}%
\pgfsys@transformshift{0in}{1in}%
\end{pgfscope}%
\begin{pgfscope}%
\pgfpathrectangle{\pgfqpoint{1.535194in}{1.233139in}}{\pgfqpoint{2.964236in}{3.441667in}} %
\pgfusepath{clip}%
\pgfsetbuttcap%
\pgfsetmiterjoin%
\definecolor{currentfill}{rgb}{0.347059,0.458824,0.641176}%
\pgfsetfillcolor{currentfill}%
\pgfsetlinewidth{0.803000pt}%
\definecolor{currentstroke}{rgb}{0.827451,0.827451,0.827451}%
\pgfsetstrokecolor{currentstroke}%
\pgfsetdash{}{0pt}%
\pgfpathmoveto{\pgfqpoint{3.758371in}{1.233139in}}%
\pgfpathlineto{\pgfqpoint{4.351218in}{1.233139in}}%
\pgfpathlineto{\pgfqpoint{4.351218in}{1.267312in}}%
\pgfpathlineto{\pgfqpoint{3.758371in}{1.267312in}}%
\pgfpathclose%
\pgfusepath{stroke,fill}%
\end{pgfscope}%
\begin{pgfscope}%
\pgfsetbuttcap%
\pgfsetmiterjoin%
\definecolor{currentfill}{rgb}{0.347059,0.458824,0.641176}%
\pgfsetfillcolor{currentfill}%
\pgfsetlinewidth{0.803000pt}%
\definecolor{currentstroke}{rgb}{0.827451,0.827451,0.827451}%
\pgfsetstrokecolor{currentstroke}%
\pgfsetdash{}{0pt}%
\pgfpathrectangle{\pgfqpoint{1.535194in}{1.233139in}}{\pgfqpoint{2.964236in}{3.441667in}} %
\pgfusepath{clip}%
\pgfpathmoveto{\pgfqpoint{3.758371in}{1.233139in}}%
\pgfpathlineto{\pgfqpoint{4.351218in}{1.233139in}}%
\pgfpathlineto{\pgfqpoint{4.351218in}{1.267312in}}%
\pgfpathlineto{\pgfqpoint{3.758371in}{1.267312in}}%
\pgfpathclose%
\pgfusepath{clip}%
\pgfsys@defobject{currentpattern}{\pgfqpoint{0in}{0in}}{\pgfqpoint{1in}{1in}}{%
\begin{pgfscope}%
\pgfpathrectangle{\pgfqpoint{0in}{0in}}{\pgfqpoint{1in}{1in}}%
\pgfusepath{clip}%
\pgfpathmoveto{\pgfqpoint{-0.500000in}{0.500000in}}%
\pgfpathlineto{\pgfqpoint{0.500000in}{1.500000in}}%
\pgfpathmoveto{\pgfqpoint{-0.333333in}{0.333333in}}%
\pgfpathlineto{\pgfqpoint{0.666667in}{1.333333in}}%
\pgfpathmoveto{\pgfqpoint{-0.166667in}{0.166667in}}%
\pgfpathlineto{\pgfqpoint{0.833333in}{1.166667in}}%
\pgfpathmoveto{\pgfqpoint{0.000000in}{0.000000in}}%
\pgfpathlineto{\pgfqpoint{1.000000in}{1.000000in}}%
\pgfpathmoveto{\pgfqpoint{0.166667in}{-0.166667in}}%
\pgfpathlineto{\pgfqpoint{1.166667in}{0.833333in}}%
\pgfpathmoveto{\pgfqpoint{0.333333in}{-0.333333in}}%
\pgfpathlineto{\pgfqpoint{1.333333in}{0.666667in}}%
\pgfpathmoveto{\pgfqpoint{0.500000in}{-0.500000in}}%
\pgfpathlineto{\pgfqpoint{1.500000in}{0.500000in}}%
\pgfpathmoveto{\pgfqpoint{-0.500000in}{0.500000in}}%
\pgfpathlineto{\pgfqpoint{0.500000in}{-0.500000in}}%
\pgfpathmoveto{\pgfqpoint{-0.333333in}{0.666667in}}%
\pgfpathlineto{\pgfqpoint{0.666667in}{-0.333333in}}%
\pgfpathmoveto{\pgfqpoint{-0.166667in}{0.833333in}}%
\pgfpathlineto{\pgfqpoint{0.833333in}{-0.166667in}}%
\pgfpathmoveto{\pgfqpoint{0.000000in}{1.000000in}}%
\pgfpathlineto{\pgfqpoint{1.000000in}{0.000000in}}%
\pgfpathmoveto{\pgfqpoint{0.166667in}{1.166667in}}%
\pgfpathlineto{\pgfqpoint{1.166667in}{0.166667in}}%
\pgfpathmoveto{\pgfqpoint{0.333333in}{1.333333in}}%
\pgfpathlineto{\pgfqpoint{1.333333in}{0.333333in}}%
\pgfpathmoveto{\pgfqpoint{0.500000in}{1.500000in}}%
\pgfpathlineto{\pgfqpoint{1.500000in}{0.500000in}}%
\pgfusepath{stroke}%
\end{pgfscope}%
}%
\pgfsys@transformshift{3.758371in}{1.233139in}%
\pgfsys@useobject{currentpattern}{}%
\pgfsys@transformshift{1in}{0in}%
\pgfsys@transformshift{-1in}{0in}%
\pgfsys@transformshift{0in}{1in}%
\end{pgfscope}%
\begin{pgfscope}%
\pgfpathrectangle{\pgfqpoint{1.535194in}{1.233139in}}{\pgfqpoint{2.964236in}{3.441667in}} %
\pgfusepath{clip}%
\pgfsetroundcap%
\pgfsetroundjoin%
\pgfsetlinewidth{2.168100pt}%
\definecolor{currentstroke}{rgb}{0.260000,0.260000,0.260000}%
\pgfsetstrokecolor{currentstroke}%
\pgfsetdash{}{0pt}%
\pgfpathmoveto{\pgfqpoint{1.979830in}{4.427255in}}%
\pgfpathlineto{\pgfqpoint{1.979830in}{4.481699in}}%
\pgfusepath{stroke}%
\end{pgfscope}%
\begin{pgfscope}%
\pgfpathrectangle{\pgfqpoint{1.535194in}{1.233139in}}{\pgfqpoint{2.964236in}{3.441667in}} %
\pgfusepath{clip}%
\pgfsetroundcap%
\pgfsetroundjoin%
\pgfsetlinewidth{2.168100pt}%
\definecolor{currentstroke}{rgb}{0.260000,0.260000,0.260000}%
\pgfsetstrokecolor{currentstroke}%
\pgfsetdash{}{0pt}%
\pgfpathmoveto{\pgfqpoint{3.461948in}{4.495418in}}%
\pgfpathlineto{\pgfqpoint{3.461948in}{4.510916in}}%
\pgfusepath{stroke}%
\end{pgfscope}%
\begin{pgfscope}%
\pgfpathrectangle{\pgfqpoint{1.535194in}{1.233139in}}{\pgfqpoint{2.964236in}{3.441667in}} %
\pgfusepath{clip}%
\pgfsetroundcap%
\pgfsetroundjoin%
\pgfsetlinewidth{2.168100pt}%
\definecolor{currentstroke}{rgb}{0.260000,0.260000,0.260000}%
\pgfsetstrokecolor{currentstroke}%
\pgfsetdash{}{0pt}%
\pgfpathmoveto{\pgfqpoint{2.572677in}{4.292714in}}%
\pgfpathlineto{\pgfqpoint{2.572677in}{4.374333in}}%
\pgfusepath{stroke}%
\end{pgfscope}%
\begin{pgfscope}%
\pgfpathrectangle{\pgfqpoint{1.535194in}{1.233139in}}{\pgfqpoint{2.964236in}{3.441667in}} %
\pgfusepath{clip}%
\pgfsetroundcap%
\pgfsetroundjoin%
\pgfsetlinewidth{2.168100pt}%
\definecolor{currentstroke}{rgb}{0.260000,0.260000,0.260000}%
\pgfsetstrokecolor{currentstroke}%
\pgfsetdash{}{0pt}%
\pgfpathmoveto{\pgfqpoint{4.054795in}{1.266525in}}%
\pgfpathlineto{\pgfqpoint{4.054795in}{1.267745in}}%
\pgfusepath{stroke}%
\end{pgfscope}%
\begin{pgfscope}%
\pgfsetrectcap%
\pgfsetmiterjoin%
\pgfsetlinewidth{1.003750pt}%
\definecolor{currentstroke}{rgb}{0.800000,0.800000,0.800000}%
\pgfsetstrokecolor{currentstroke}%
\pgfsetdash{}{0pt}%
\pgfpathmoveto{\pgfqpoint{1.535194in}{1.233139in}}%
\pgfpathlineto{\pgfqpoint{1.535194in}{4.674805in}}%
\pgfusepath{stroke}%
\end{pgfscope}%
\begin{pgfscope}%
\pgfsetrectcap%
\pgfsetmiterjoin%
\pgfsetlinewidth{1.003750pt}%
\definecolor{currentstroke}{rgb}{0.800000,0.800000,0.800000}%
\pgfsetstrokecolor{currentstroke}%
\pgfsetdash{}{0pt}%
\pgfpathmoveto{\pgfqpoint{1.535194in}{1.233139in}}%
\pgfpathlineto{\pgfqpoint{4.499430in}{1.233139in}}%
\pgfusepath{stroke}%
\end{pgfscope}%
\begin{pgfscope}%
\pgfsetbuttcap%
\pgfsetmiterjoin%
\definecolor{currentfill}{rgb}{0.798529,0.536765,0.389706}%
\pgfsetfillcolor{currentfill}%
\pgfsetlinewidth{0.803000pt}%
\definecolor{currentstroke}{rgb}{0.827451,0.827451,0.827451}%
\pgfsetstrokecolor{currentstroke}%
\pgfsetdash{}{0pt}%
\pgfpathmoveto{\pgfqpoint{1.621764in}{4.587527in}}%
\pgfpathlineto{\pgfqpoint{2.052319in}{4.587527in}}%
\pgfpathlineto{\pgfqpoint{2.052319in}{4.956972in}}%
\pgfpathlineto{\pgfqpoint{1.621764in}{4.956972in}}%
\pgfpathclose%
\pgfusepath{stroke,fill}%
\end{pgfscope}%
\begin{pgfscope}%
\pgfsetbuttcap%
\pgfsetmiterjoin%
\definecolor{currentfill}{rgb}{0.798529,0.536765,0.389706}%
\pgfsetfillcolor{currentfill}%
\pgfsetlinewidth{0.803000pt}%
\definecolor{currentstroke}{rgb}{0.827451,0.827451,0.827451}%
\pgfsetstrokecolor{currentstroke}%
\pgfsetdash{}{0pt}%
\pgfpathmoveto{\pgfqpoint{1.621764in}{4.587527in}}%
\pgfpathlineto{\pgfqpoint{2.052319in}{4.587527in}}%
\pgfpathlineto{\pgfqpoint{2.052319in}{4.956972in}}%
\pgfpathlineto{\pgfqpoint{1.621764in}{4.956972in}}%
\pgfpathclose%
\pgfusepath{clip}%
\pgfsys@defobject{currentpattern}{\pgfqpoint{0in}{0in}}{\pgfqpoint{1in}{1in}}{%
\begin{pgfscope}%
\pgfpathrectangle{\pgfqpoint{0in}{0in}}{\pgfqpoint{1in}{1in}}%
\pgfusepath{clip}%
\pgfpathmoveto{\pgfqpoint{-0.500000in}{0.500000in}}%
\pgfpathlineto{\pgfqpoint{0.500000in}{1.500000in}}%
\pgfpathmoveto{\pgfqpoint{-0.333333in}{0.333333in}}%
\pgfpathlineto{\pgfqpoint{0.666667in}{1.333333in}}%
\pgfpathmoveto{\pgfqpoint{-0.166667in}{0.166667in}}%
\pgfpathlineto{\pgfqpoint{0.833333in}{1.166667in}}%
\pgfpathmoveto{\pgfqpoint{0.000000in}{0.000000in}}%
\pgfpathlineto{\pgfqpoint{1.000000in}{1.000000in}}%
\pgfpathmoveto{\pgfqpoint{0.166667in}{-0.166667in}}%
\pgfpathlineto{\pgfqpoint{1.166667in}{0.833333in}}%
\pgfpathmoveto{\pgfqpoint{0.333333in}{-0.333333in}}%
\pgfpathlineto{\pgfqpoint{1.333333in}{0.666667in}}%
\pgfpathmoveto{\pgfqpoint{0.500000in}{-0.500000in}}%
\pgfpathlineto{\pgfqpoint{1.500000in}{0.500000in}}%
\pgfusepath{stroke}%
\end{pgfscope}%
}%
\pgfsys@transformshift{1.621764in}{4.587527in}%
\pgfsys@useobject{currentpattern}{}%
\pgfsys@transformshift{1in}{0in}%
\pgfsys@transformshift{-1in}{0in}%
\pgfsys@transformshift{0in}{1in}%
\end{pgfscope}%
\begin{pgfscope}%
\definecolor{textcolor}{rgb}{0.150000,0.150000,0.150000}%
\pgfsetstrokecolor{textcolor}%
\pgfsetfillcolor{textcolor}%
\pgftext[x=2.266208in,y=4.587527in,left,base]{\color{textcolor}\rmfamily\fontsize{38.000000}{45.600000}\selectfont DE}%
\end{pgfscope}%
\begin{pgfscope}%
\pgfsetbuttcap%
\pgfsetmiterjoin%
\definecolor{currentfill}{rgb}{0.347059,0.458824,0.641176}%
\pgfsetfillcolor{currentfill}%
\pgfsetlinewidth{0.803000pt}%
\definecolor{currentstroke}{rgb}{0.827451,0.827451,0.827451}%
\pgfsetstrokecolor{currentstroke}%
\pgfsetdash{}{0pt}%
\pgfpathmoveto{\pgfqpoint{3.276347in}{4.587527in}}%
\pgfpathlineto{\pgfqpoint{3.706902in}{4.587527in}}%
\pgfpathlineto{\pgfqpoint{3.706902in}{4.956972in}}%
\pgfpathlineto{\pgfqpoint{3.276347in}{4.956972in}}%
\pgfpathclose%
\pgfusepath{stroke,fill}%
\end{pgfscope}%
\begin{pgfscope}%
\pgfsetbuttcap%
\pgfsetmiterjoin%
\definecolor{currentfill}{rgb}{0.347059,0.458824,0.641176}%
\pgfsetfillcolor{currentfill}%
\pgfsetlinewidth{0.803000pt}%
\definecolor{currentstroke}{rgb}{0.827451,0.827451,0.827451}%
\pgfsetstrokecolor{currentstroke}%
\pgfsetdash{}{0pt}%
\pgfpathmoveto{\pgfqpoint{3.276347in}{4.587527in}}%
\pgfpathlineto{\pgfqpoint{3.706902in}{4.587527in}}%
\pgfpathlineto{\pgfqpoint{3.706902in}{4.956972in}}%
\pgfpathlineto{\pgfqpoint{3.276347in}{4.956972in}}%
\pgfpathclose%
\pgfusepath{clip}%
\pgfsys@defobject{currentpattern}{\pgfqpoint{0in}{0in}}{\pgfqpoint{1in}{1in}}{%
\begin{pgfscope}%
\pgfpathrectangle{\pgfqpoint{0in}{0in}}{\pgfqpoint{1in}{1in}}%
\pgfusepath{clip}%
\pgfpathmoveto{\pgfqpoint{-0.500000in}{0.500000in}}%
\pgfpathlineto{\pgfqpoint{0.500000in}{1.500000in}}%
\pgfpathmoveto{\pgfqpoint{-0.333333in}{0.333333in}}%
\pgfpathlineto{\pgfqpoint{0.666667in}{1.333333in}}%
\pgfpathmoveto{\pgfqpoint{-0.166667in}{0.166667in}}%
\pgfpathlineto{\pgfqpoint{0.833333in}{1.166667in}}%
\pgfpathmoveto{\pgfqpoint{0.000000in}{0.000000in}}%
\pgfpathlineto{\pgfqpoint{1.000000in}{1.000000in}}%
\pgfpathmoveto{\pgfqpoint{0.166667in}{-0.166667in}}%
\pgfpathlineto{\pgfqpoint{1.166667in}{0.833333in}}%
\pgfpathmoveto{\pgfqpoint{0.333333in}{-0.333333in}}%
\pgfpathlineto{\pgfqpoint{1.333333in}{0.666667in}}%
\pgfpathmoveto{\pgfqpoint{0.500000in}{-0.500000in}}%
\pgfpathlineto{\pgfqpoint{1.500000in}{0.500000in}}%
\pgfpathmoveto{\pgfqpoint{-0.500000in}{0.500000in}}%
\pgfpathlineto{\pgfqpoint{0.500000in}{-0.500000in}}%
\pgfpathmoveto{\pgfqpoint{-0.333333in}{0.666667in}}%
\pgfpathlineto{\pgfqpoint{0.666667in}{-0.333333in}}%
\pgfpathmoveto{\pgfqpoint{-0.166667in}{0.833333in}}%
\pgfpathlineto{\pgfqpoint{0.833333in}{-0.166667in}}%
\pgfpathmoveto{\pgfqpoint{0.000000in}{1.000000in}}%
\pgfpathlineto{\pgfqpoint{1.000000in}{0.000000in}}%
\pgfpathmoveto{\pgfqpoint{0.166667in}{1.166667in}}%
\pgfpathlineto{\pgfqpoint{1.166667in}{0.166667in}}%
\pgfpathmoveto{\pgfqpoint{0.333333in}{1.333333in}}%
\pgfpathlineto{\pgfqpoint{1.333333in}{0.333333in}}%
\pgfpathmoveto{\pgfqpoint{0.500000in}{1.500000in}}%
\pgfpathlineto{\pgfqpoint{1.500000in}{0.500000in}}%
\pgfusepath{stroke}%
\end{pgfscope}%
}%
\pgfsys@transformshift{3.276347in}{4.587527in}%
\pgfsys@useobject{currentpattern}{}%
\pgfsys@transformshift{1in}{0in}%
\pgfsys@transformshift{-1in}{0in}%
\pgfsys@transformshift{0in}{1in}%
\end{pgfscope}%
\begin{pgfscope}%
\definecolor{textcolor}{rgb}{0.150000,0.150000,0.150000}%
\pgfsetstrokecolor{textcolor}%
\pgfsetfillcolor{textcolor}%
\pgftext[x=3.920791in,y=4.587527in,left,base]{\color{textcolor}\rmfamily\fontsize{38.000000}{45.600000}\selectfont CO}%
\end{pgfscope}%
\end{pgfpicture}%
\makeatother%
\endgroup%
%
}
\caption{Workload 2}
\end{subfigure}%
\begin{subfigure}[b]{0.33\linewidth}
\centering
 \resizebox{\columnwidth}{!}{%
%% Creator: Matplotlib, PGF backend
%%
%% To include the figure in your LaTeX document, write
%%   \input{<filename>.pgf}
%%
%% Make sure the required packages are loaded in your preamble
%%   \usepackage{pgf}
%%
%% Figures using additional raster images can only be included by \input if
%% they are in the same directory as the main LaTeX file. For loading figures
%% from other directories you can use the `import` package
%%   \usepackage{import}
%% and then include the figures with
%%   \import{<path to file>}{<filename>.pgf}
%%
%% Matplotlib used the following preamble
%%   \usepackage{fontspec}
%%   \setmonofont{Andale Mono}
%%
\begingroup%
\makeatletter%
\begin{pgfpicture}%
\pgfpathrectangle{\pgfpointorigin}{\pgfqpoint{5.068958in}{5.268083in}}%
\pgfusepath{use as bounding box, clip}%
\begin{pgfscope}%
\pgfsetbuttcap%
\pgfsetmiterjoin%
\definecolor{currentfill}{rgb}{1.000000,1.000000,1.000000}%
\pgfsetfillcolor{currentfill}%
\pgfsetlinewidth{0.000000pt}%
\definecolor{currentstroke}{rgb}{1.000000,1.000000,1.000000}%
\pgfsetstrokecolor{currentstroke}%
\pgfsetdash{}{0pt}%
\pgfpathmoveto{\pgfqpoint{0.000000in}{0.000000in}}%
\pgfpathlineto{\pgfqpoint{5.068958in}{0.000000in}}%
\pgfpathlineto{\pgfqpoint{5.068958in}{5.268083in}}%
\pgfpathlineto{\pgfqpoint{0.000000in}{5.268083in}}%
\pgfpathclose%
\pgfusepath{fill}%
\end{pgfscope}%
\begin{pgfscope}%
\pgfsetbuttcap%
\pgfsetmiterjoin%
\definecolor{currentfill}{rgb}{1.000000,1.000000,1.000000}%
\pgfsetfillcolor{currentfill}%
\pgfsetlinewidth{0.000000pt}%
\definecolor{currentstroke}{rgb}{0.000000,0.000000,0.000000}%
\pgfsetstrokecolor{currentstroke}%
\pgfsetstrokeopacity{0.000000}%
\pgfsetdash{}{0pt}%
\pgfpathmoveto{\pgfqpoint{1.764694in}{1.233139in}}%
\pgfpathlineto{\pgfqpoint{4.416430in}{1.233139in}}%
\pgfpathlineto{\pgfqpoint{4.416430in}{4.674805in}}%
\pgfpathlineto{\pgfqpoint{1.764694in}{4.674805in}}%
\pgfpathclose%
\pgfusepath{fill}%
\end{pgfscope}%
\begin{pgfscope}%
\definecolor{textcolor}{rgb}{0.150000,0.150000,0.150000}%
\pgfsetstrokecolor{textcolor}%
\pgfsetfillcolor{textcolor}%
\pgftext[x=2.427628in,y=1.069250in,,top]{\color{textcolor}\rmfamily\fontsize{36.000000}{43.200000}\selectfont 1}%
\end{pgfscope}%
\begin{pgfscope}%
\definecolor{textcolor}{rgb}{0.150000,0.150000,0.150000}%
\pgfsetstrokecolor{textcolor}%
\pgfsetfillcolor{textcolor}%
\pgftext[x=3.753496in,y=1.069250in,,top]{\color{textcolor}\rmfamily\fontsize{36.000000}{43.200000}\selectfont 2}%
\end{pgfscope}%
\begin{pgfscope}%
\definecolor{textcolor}{rgb}{0.150000,0.150000,0.150000}%
\pgfsetstrokecolor{textcolor}%
\pgfsetfillcolor{textcolor}%
\pgftext[x=3.090562in,y=0.569194in,,top]{\color{textcolor}\rmfamily\fontsize{38.000000}{45.600000}\selectfont Run}%
\end{pgfscope}%
\begin{pgfscope}%
\pgfpathrectangle{\pgfqpoint{1.764694in}{1.233139in}}{\pgfqpoint{2.651736in}{3.441667in}} %
\pgfusepath{clip}%
\pgfsetroundcap%
\pgfsetroundjoin%
\pgfsetlinewidth{0.803000pt}%
\definecolor{currentstroke}{rgb}{0.800000,0.800000,0.800000}%
\pgfsetstrokecolor{currentstroke}%
\pgfsetdash{}{0pt}%
\pgfpathmoveto{\pgfqpoint{1.764694in}{1.233139in}}%
\pgfpathlineto{\pgfqpoint{4.416430in}{1.233139in}}%
\pgfusepath{stroke}%
\end{pgfscope}%
\begin{pgfscope}%
\definecolor{textcolor}{rgb}{0.150000,0.150000,0.150000}%
\pgfsetstrokecolor{textcolor}%
\pgfsetfillcolor{textcolor}%
\pgftext[x=1.371305in,y=1.059639in,left,base]{\color{textcolor}\rmfamily\fontsize{36.000000}{43.200000}\selectfont 0}%
\end{pgfscope}%
\begin{pgfscope}%
\pgfpathrectangle{\pgfqpoint{1.764694in}{1.233139in}}{\pgfqpoint{2.651736in}{3.441667in}} %
\pgfusepath{clip}%
\pgfsetroundcap%
\pgfsetroundjoin%
\pgfsetlinewidth{0.803000pt}%
\definecolor{currentstroke}{rgb}{0.800000,0.800000,0.800000}%
\pgfsetstrokecolor{currentstroke}%
\pgfsetdash{}{0pt}%
\pgfpathmoveto{\pgfqpoint{1.764694in}{2.532223in}}%
\pgfpathlineto{\pgfqpoint{4.416430in}{2.532223in}}%
\pgfusepath{stroke}%
\end{pgfscope}%
\begin{pgfscope}%
\definecolor{textcolor}{rgb}{0.150000,0.150000,0.150000}%
\pgfsetstrokecolor{textcolor}%
\pgfsetfillcolor{textcolor}%
\pgftext[x=0.912305in,y=2.358723in,left,base]{\color{textcolor}\rmfamily\fontsize{36.000000}{43.200000}\selectfont 500}%
\end{pgfscope}%
\begin{pgfscope}%
\pgfpathrectangle{\pgfqpoint{1.764694in}{1.233139in}}{\pgfqpoint{2.651736in}{3.441667in}} %
\pgfusepath{clip}%
\pgfsetroundcap%
\pgfsetroundjoin%
\pgfsetlinewidth{0.803000pt}%
\definecolor{currentstroke}{rgb}{0.800000,0.800000,0.800000}%
\pgfsetstrokecolor{currentstroke}%
\pgfsetdash{}{0pt}%
\pgfpathmoveto{\pgfqpoint{1.764694in}{3.831307in}}%
\pgfpathlineto{\pgfqpoint{4.416430in}{3.831307in}}%
\pgfusepath{stroke}%
\end{pgfscope}%
\begin{pgfscope}%
\definecolor{textcolor}{rgb}{0.150000,0.150000,0.150000}%
\pgfsetstrokecolor{textcolor}%
\pgfsetfillcolor{textcolor}%
\pgftext[x=0.682805in,y=3.657807in,left,base]{\color{textcolor}\rmfamily\fontsize{36.000000}{43.200000}\selectfont 1000}%
\end{pgfscope}%
\begin{pgfscope}%
\definecolor{textcolor}{rgb}{0.150000,0.150000,0.150000}%
\pgfsetstrokecolor{textcolor}%
\pgfsetfillcolor{textcolor}%
\pgftext[x=0.627250in,y=2.953972in,,bottom,rotate=90.000000]{\color{textcolor}\rmfamily\fontsize{38.000000}{45.600000}\selectfont Run Time (s)}%
\end{pgfscope}%
\begin{pgfscope}%
\pgfpathrectangle{\pgfqpoint{1.764694in}{1.233139in}}{\pgfqpoint{2.651736in}{3.441667in}} %
\pgfusepath{clip}%
\pgfsetbuttcap%
\pgfsetmiterjoin%
\definecolor{currentfill}{rgb}{0.798529,0.536765,0.389706}%
\pgfsetfillcolor{currentfill}%
\pgfsetlinewidth{0.803000pt}%
\definecolor{currentstroke}{rgb}{0.827451,0.827451,0.827451}%
\pgfsetstrokecolor{currentstroke}%
\pgfsetdash{}{0pt}%
\pgfpathmoveto{\pgfqpoint{1.897281in}{1.233139in}}%
\pgfpathlineto{\pgfqpoint{2.427628in}{1.233139in}}%
\pgfpathlineto{\pgfqpoint{2.427628in}{4.246158in}}%
\pgfpathlineto{\pgfqpoint{1.897281in}{4.246158in}}%
\pgfpathclose%
\pgfusepath{stroke,fill}%
\end{pgfscope}%
\begin{pgfscope}%
\pgfsetbuttcap%
\pgfsetmiterjoin%
\definecolor{currentfill}{rgb}{0.798529,0.536765,0.389706}%
\pgfsetfillcolor{currentfill}%
\pgfsetlinewidth{0.803000pt}%
\definecolor{currentstroke}{rgb}{0.827451,0.827451,0.827451}%
\pgfsetstrokecolor{currentstroke}%
\pgfsetdash{}{0pt}%
\pgfpathrectangle{\pgfqpoint{1.764694in}{1.233139in}}{\pgfqpoint{2.651736in}{3.441667in}} %
\pgfusepath{clip}%
\pgfpathmoveto{\pgfqpoint{1.897281in}{1.233139in}}%
\pgfpathlineto{\pgfqpoint{2.427628in}{1.233139in}}%
\pgfpathlineto{\pgfqpoint{2.427628in}{4.246158in}}%
\pgfpathlineto{\pgfqpoint{1.897281in}{4.246158in}}%
\pgfpathclose%
\pgfusepath{clip}%
\pgfsys@defobject{currentpattern}{\pgfqpoint{0in}{0in}}{\pgfqpoint{1in}{1in}}{%
\begin{pgfscope}%
\pgfpathrectangle{\pgfqpoint{0in}{0in}}{\pgfqpoint{1in}{1in}}%
\pgfusepath{clip}%
\pgfpathmoveto{\pgfqpoint{-0.500000in}{0.500000in}}%
\pgfpathlineto{\pgfqpoint{0.500000in}{1.500000in}}%
\pgfpathmoveto{\pgfqpoint{-0.333333in}{0.333333in}}%
\pgfpathlineto{\pgfqpoint{0.666667in}{1.333333in}}%
\pgfpathmoveto{\pgfqpoint{-0.166667in}{0.166667in}}%
\pgfpathlineto{\pgfqpoint{0.833333in}{1.166667in}}%
\pgfpathmoveto{\pgfqpoint{0.000000in}{0.000000in}}%
\pgfpathlineto{\pgfqpoint{1.000000in}{1.000000in}}%
\pgfpathmoveto{\pgfqpoint{0.166667in}{-0.166667in}}%
\pgfpathlineto{\pgfqpoint{1.166667in}{0.833333in}}%
\pgfpathmoveto{\pgfqpoint{0.333333in}{-0.333333in}}%
\pgfpathlineto{\pgfqpoint{1.333333in}{0.666667in}}%
\pgfpathmoveto{\pgfqpoint{0.500000in}{-0.500000in}}%
\pgfpathlineto{\pgfqpoint{1.500000in}{0.500000in}}%
\pgfusepath{stroke}%
\end{pgfscope}%
}%
\pgfsys@transformshift{1.897281in}{1.233139in}%
\pgfsys@useobject{currentpattern}{}%
\pgfsys@transformshift{1in}{0in}%
\pgfsys@transformshift{-1in}{0in}%
\pgfsys@transformshift{0in}{1in}%
\pgfsys@useobject{currentpattern}{}%
\pgfsys@transformshift{1in}{0in}%
\pgfsys@transformshift{-1in}{0in}%
\pgfsys@transformshift{0in}{1in}%
\pgfsys@useobject{currentpattern}{}%
\pgfsys@transformshift{1in}{0in}%
\pgfsys@transformshift{-1in}{0in}%
\pgfsys@transformshift{0in}{1in}%
\pgfsys@useobject{currentpattern}{}%
\pgfsys@transformshift{1in}{0in}%
\pgfsys@transformshift{-1in}{0in}%
\pgfsys@transformshift{0in}{1in}%
\end{pgfscope}%
\begin{pgfscope}%
\pgfpathrectangle{\pgfqpoint{1.764694in}{1.233139in}}{\pgfqpoint{2.651736in}{3.441667in}} %
\pgfusepath{clip}%
\pgfsetbuttcap%
\pgfsetmiterjoin%
\definecolor{currentfill}{rgb}{0.798529,0.536765,0.389706}%
\pgfsetfillcolor{currentfill}%
\pgfsetlinewidth{0.803000pt}%
\definecolor{currentstroke}{rgb}{0.827451,0.827451,0.827451}%
\pgfsetstrokecolor{currentstroke}%
\pgfsetdash{}{0pt}%
\pgfpathmoveto{\pgfqpoint{3.223149in}{1.233139in}}%
\pgfpathlineto{\pgfqpoint{3.753496in}{1.233139in}}%
\pgfpathlineto{\pgfqpoint{3.753496in}{4.372438in}}%
\pgfpathlineto{\pgfqpoint{3.223149in}{4.372438in}}%
\pgfpathclose%
\pgfusepath{stroke,fill}%
\end{pgfscope}%
\begin{pgfscope}%
\pgfsetbuttcap%
\pgfsetmiterjoin%
\definecolor{currentfill}{rgb}{0.798529,0.536765,0.389706}%
\pgfsetfillcolor{currentfill}%
\pgfsetlinewidth{0.803000pt}%
\definecolor{currentstroke}{rgb}{0.827451,0.827451,0.827451}%
\pgfsetstrokecolor{currentstroke}%
\pgfsetdash{}{0pt}%
\pgfpathrectangle{\pgfqpoint{1.764694in}{1.233139in}}{\pgfqpoint{2.651736in}{3.441667in}} %
\pgfusepath{clip}%
\pgfpathmoveto{\pgfqpoint{3.223149in}{1.233139in}}%
\pgfpathlineto{\pgfqpoint{3.753496in}{1.233139in}}%
\pgfpathlineto{\pgfqpoint{3.753496in}{4.372438in}}%
\pgfpathlineto{\pgfqpoint{3.223149in}{4.372438in}}%
\pgfpathclose%
\pgfusepath{clip}%
\pgfsys@defobject{currentpattern}{\pgfqpoint{0in}{0in}}{\pgfqpoint{1in}{1in}}{%
\begin{pgfscope}%
\pgfpathrectangle{\pgfqpoint{0in}{0in}}{\pgfqpoint{1in}{1in}}%
\pgfusepath{clip}%
\pgfpathmoveto{\pgfqpoint{-0.500000in}{0.500000in}}%
\pgfpathlineto{\pgfqpoint{0.500000in}{1.500000in}}%
\pgfpathmoveto{\pgfqpoint{-0.333333in}{0.333333in}}%
\pgfpathlineto{\pgfqpoint{0.666667in}{1.333333in}}%
\pgfpathmoveto{\pgfqpoint{-0.166667in}{0.166667in}}%
\pgfpathlineto{\pgfqpoint{0.833333in}{1.166667in}}%
\pgfpathmoveto{\pgfqpoint{0.000000in}{0.000000in}}%
\pgfpathlineto{\pgfqpoint{1.000000in}{1.000000in}}%
\pgfpathmoveto{\pgfqpoint{0.166667in}{-0.166667in}}%
\pgfpathlineto{\pgfqpoint{1.166667in}{0.833333in}}%
\pgfpathmoveto{\pgfqpoint{0.333333in}{-0.333333in}}%
\pgfpathlineto{\pgfqpoint{1.333333in}{0.666667in}}%
\pgfpathmoveto{\pgfqpoint{0.500000in}{-0.500000in}}%
\pgfpathlineto{\pgfqpoint{1.500000in}{0.500000in}}%
\pgfusepath{stroke}%
\end{pgfscope}%
}%
\pgfsys@transformshift{3.223149in}{1.233139in}%
\pgfsys@useobject{currentpattern}{}%
\pgfsys@transformshift{1in}{0in}%
\pgfsys@transformshift{-1in}{0in}%
\pgfsys@transformshift{0in}{1in}%
\pgfsys@useobject{currentpattern}{}%
\pgfsys@transformshift{1in}{0in}%
\pgfsys@transformshift{-1in}{0in}%
\pgfsys@transformshift{0in}{1in}%
\pgfsys@useobject{currentpattern}{}%
\pgfsys@transformshift{1in}{0in}%
\pgfsys@transformshift{-1in}{0in}%
\pgfsys@transformshift{0in}{1in}%
\pgfsys@useobject{currentpattern}{}%
\pgfsys@transformshift{1in}{0in}%
\pgfsys@transformshift{-1in}{0in}%
\pgfsys@transformshift{0in}{1in}%
\end{pgfscope}%
\begin{pgfscope}%
\pgfpathrectangle{\pgfqpoint{1.764694in}{1.233139in}}{\pgfqpoint{2.651736in}{3.441667in}} %
\pgfusepath{clip}%
\pgfsetbuttcap%
\pgfsetmiterjoin%
\definecolor{currentfill}{rgb}{0.347059,0.458824,0.641176}%
\pgfsetfillcolor{currentfill}%
\pgfsetlinewidth{0.803000pt}%
\definecolor{currentstroke}{rgb}{0.827451,0.827451,0.827451}%
\pgfsetstrokecolor{currentstroke}%
\pgfsetdash{}{0pt}%
\pgfpathmoveto{\pgfqpoint{2.427628in}{1.233139in}}%
\pgfpathlineto{\pgfqpoint{2.957976in}{1.233139in}}%
\pgfpathlineto{\pgfqpoint{2.957976in}{4.103945in}}%
\pgfpathlineto{\pgfqpoint{2.427628in}{4.103945in}}%
\pgfpathclose%
\pgfusepath{stroke,fill}%
\end{pgfscope}%
\begin{pgfscope}%
\pgfsetbuttcap%
\pgfsetmiterjoin%
\definecolor{currentfill}{rgb}{0.347059,0.458824,0.641176}%
\pgfsetfillcolor{currentfill}%
\pgfsetlinewidth{0.803000pt}%
\definecolor{currentstroke}{rgb}{0.827451,0.827451,0.827451}%
\pgfsetstrokecolor{currentstroke}%
\pgfsetdash{}{0pt}%
\pgfpathrectangle{\pgfqpoint{1.764694in}{1.233139in}}{\pgfqpoint{2.651736in}{3.441667in}} %
\pgfusepath{clip}%
\pgfpathmoveto{\pgfqpoint{2.427628in}{1.233139in}}%
\pgfpathlineto{\pgfqpoint{2.957976in}{1.233139in}}%
\pgfpathlineto{\pgfqpoint{2.957976in}{4.103945in}}%
\pgfpathlineto{\pgfqpoint{2.427628in}{4.103945in}}%
\pgfpathclose%
\pgfusepath{clip}%
\pgfsys@defobject{currentpattern}{\pgfqpoint{0in}{0in}}{\pgfqpoint{1in}{1in}}{%
\begin{pgfscope}%
\pgfpathrectangle{\pgfqpoint{0in}{0in}}{\pgfqpoint{1in}{1in}}%
\pgfusepath{clip}%
\pgfpathmoveto{\pgfqpoint{-0.500000in}{0.500000in}}%
\pgfpathlineto{\pgfqpoint{0.500000in}{1.500000in}}%
\pgfpathmoveto{\pgfqpoint{-0.333333in}{0.333333in}}%
\pgfpathlineto{\pgfqpoint{0.666667in}{1.333333in}}%
\pgfpathmoveto{\pgfqpoint{-0.166667in}{0.166667in}}%
\pgfpathlineto{\pgfqpoint{0.833333in}{1.166667in}}%
\pgfpathmoveto{\pgfqpoint{0.000000in}{0.000000in}}%
\pgfpathlineto{\pgfqpoint{1.000000in}{1.000000in}}%
\pgfpathmoveto{\pgfqpoint{0.166667in}{-0.166667in}}%
\pgfpathlineto{\pgfqpoint{1.166667in}{0.833333in}}%
\pgfpathmoveto{\pgfqpoint{0.333333in}{-0.333333in}}%
\pgfpathlineto{\pgfqpoint{1.333333in}{0.666667in}}%
\pgfpathmoveto{\pgfqpoint{0.500000in}{-0.500000in}}%
\pgfpathlineto{\pgfqpoint{1.500000in}{0.500000in}}%
\pgfpathmoveto{\pgfqpoint{-0.500000in}{0.500000in}}%
\pgfpathlineto{\pgfqpoint{0.500000in}{-0.500000in}}%
\pgfpathmoveto{\pgfqpoint{-0.333333in}{0.666667in}}%
\pgfpathlineto{\pgfqpoint{0.666667in}{-0.333333in}}%
\pgfpathmoveto{\pgfqpoint{-0.166667in}{0.833333in}}%
\pgfpathlineto{\pgfqpoint{0.833333in}{-0.166667in}}%
\pgfpathmoveto{\pgfqpoint{0.000000in}{1.000000in}}%
\pgfpathlineto{\pgfqpoint{1.000000in}{0.000000in}}%
\pgfpathmoveto{\pgfqpoint{0.166667in}{1.166667in}}%
\pgfpathlineto{\pgfqpoint{1.166667in}{0.166667in}}%
\pgfpathmoveto{\pgfqpoint{0.333333in}{1.333333in}}%
\pgfpathlineto{\pgfqpoint{1.333333in}{0.333333in}}%
\pgfpathmoveto{\pgfqpoint{0.500000in}{1.500000in}}%
\pgfpathlineto{\pgfqpoint{1.500000in}{0.500000in}}%
\pgfusepath{stroke}%
\end{pgfscope}%
}%
\pgfsys@transformshift{2.427628in}{1.233139in}%
\pgfsys@useobject{currentpattern}{}%
\pgfsys@transformshift{1in}{0in}%
\pgfsys@transformshift{-1in}{0in}%
\pgfsys@transformshift{0in}{1in}%
\pgfsys@useobject{currentpattern}{}%
\pgfsys@transformshift{1in}{0in}%
\pgfsys@transformshift{-1in}{0in}%
\pgfsys@transformshift{0in}{1in}%
\pgfsys@useobject{currentpattern}{}%
\pgfsys@transformshift{1in}{0in}%
\pgfsys@transformshift{-1in}{0in}%
\pgfsys@transformshift{0in}{1in}%
\end{pgfscope}%
\begin{pgfscope}%
\pgfpathrectangle{\pgfqpoint{1.764694in}{1.233139in}}{\pgfqpoint{2.651736in}{3.441667in}} %
\pgfusepath{clip}%
\pgfsetbuttcap%
\pgfsetmiterjoin%
\definecolor{currentfill}{rgb}{0.347059,0.458824,0.641176}%
\pgfsetfillcolor{currentfill}%
\pgfsetlinewidth{0.803000pt}%
\definecolor{currentstroke}{rgb}{0.827451,0.827451,0.827451}%
\pgfsetstrokecolor{currentstroke}%
\pgfsetdash{}{0pt}%
\pgfpathmoveto{\pgfqpoint{3.753496in}{1.233139in}}%
\pgfpathlineto{\pgfqpoint{4.283844in}{1.233139in}}%
\pgfpathlineto{\pgfqpoint{4.283844in}{1.427924in}}%
\pgfpathlineto{\pgfqpoint{3.753496in}{1.427924in}}%
\pgfpathclose%
\pgfusepath{stroke,fill}%
\end{pgfscope}%
\begin{pgfscope}%
\pgfsetbuttcap%
\pgfsetmiterjoin%
\definecolor{currentfill}{rgb}{0.347059,0.458824,0.641176}%
\pgfsetfillcolor{currentfill}%
\pgfsetlinewidth{0.803000pt}%
\definecolor{currentstroke}{rgb}{0.827451,0.827451,0.827451}%
\pgfsetstrokecolor{currentstroke}%
\pgfsetdash{}{0pt}%
\pgfpathrectangle{\pgfqpoint{1.764694in}{1.233139in}}{\pgfqpoint{2.651736in}{3.441667in}} %
\pgfusepath{clip}%
\pgfpathmoveto{\pgfqpoint{3.753496in}{1.233139in}}%
\pgfpathlineto{\pgfqpoint{4.283844in}{1.233139in}}%
\pgfpathlineto{\pgfqpoint{4.283844in}{1.427924in}}%
\pgfpathlineto{\pgfqpoint{3.753496in}{1.427924in}}%
\pgfpathclose%
\pgfusepath{clip}%
\pgfsys@defobject{currentpattern}{\pgfqpoint{0in}{0in}}{\pgfqpoint{1in}{1in}}{%
\begin{pgfscope}%
\pgfpathrectangle{\pgfqpoint{0in}{0in}}{\pgfqpoint{1in}{1in}}%
\pgfusepath{clip}%
\pgfpathmoveto{\pgfqpoint{-0.500000in}{0.500000in}}%
\pgfpathlineto{\pgfqpoint{0.500000in}{1.500000in}}%
\pgfpathmoveto{\pgfqpoint{-0.333333in}{0.333333in}}%
\pgfpathlineto{\pgfqpoint{0.666667in}{1.333333in}}%
\pgfpathmoveto{\pgfqpoint{-0.166667in}{0.166667in}}%
\pgfpathlineto{\pgfqpoint{0.833333in}{1.166667in}}%
\pgfpathmoveto{\pgfqpoint{0.000000in}{0.000000in}}%
\pgfpathlineto{\pgfqpoint{1.000000in}{1.000000in}}%
\pgfpathmoveto{\pgfqpoint{0.166667in}{-0.166667in}}%
\pgfpathlineto{\pgfqpoint{1.166667in}{0.833333in}}%
\pgfpathmoveto{\pgfqpoint{0.333333in}{-0.333333in}}%
\pgfpathlineto{\pgfqpoint{1.333333in}{0.666667in}}%
\pgfpathmoveto{\pgfqpoint{0.500000in}{-0.500000in}}%
\pgfpathlineto{\pgfqpoint{1.500000in}{0.500000in}}%
\pgfpathmoveto{\pgfqpoint{-0.500000in}{0.500000in}}%
\pgfpathlineto{\pgfqpoint{0.500000in}{-0.500000in}}%
\pgfpathmoveto{\pgfqpoint{-0.333333in}{0.666667in}}%
\pgfpathlineto{\pgfqpoint{0.666667in}{-0.333333in}}%
\pgfpathmoveto{\pgfqpoint{-0.166667in}{0.833333in}}%
\pgfpathlineto{\pgfqpoint{0.833333in}{-0.166667in}}%
\pgfpathmoveto{\pgfqpoint{0.000000in}{1.000000in}}%
\pgfpathlineto{\pgfqpoint{1.000000in}{0.000000in}}%
\pgfpathmoveto{\pgfqpoint{0.166667in}{1.166667in}}%
\pgfpathlineto{\pgfqpoint{1.166667in}{0.166667in}}%
\pgfpathmoveto{\pgfqpoint{0.333333in}{1.333333in}}%
\pgfpathlineto{\pgfqpoint{1.333333in}{0.333333in}}%
\pgfpathmoveto{\pgfqpoint{0.500000in}{1.500000in}}%
\pgfpathlineto{\pgfqpoint{1.500000in}{0.500000in}}%
\pgfusepath{stroke}%
\end{pgfscope}%
}%
\pgfsys@transformshift{3.753496in}{1.233139in}%
\pgfsys@useobject{currentpattern}{}%
\pgfsys@transformshift{1in}{0in}%
\pgfsys@transformshift{-1in}{0in}%
\pgfsys@transformshift{0in}{1in}%
\end{pgfscope}%
\begin{pgfscope}%
\pgfpathrectangle{\pgfqpoint{1.764694in}{1.233139in}}{\pgfqpoint{2.651736in}{3.441667in}} %
\pgfusepath{clip}%
\pgfsetroundcap%
\pgfsetroundjoin%
\pgfsetlinewidth{2.168100pt}%
\definecolor{currentstroke}{rgb}{0.260000,0.260000,0.260000}%
\pgfsetstrokecolor{currentstroke}%
\pgfsetdash{}{0pt}%
\pgfpathmoveto{\pgfqpoint{2.162455in}{4.091738in}}%
\pgfpathlineto{\pgfqpoint{2.162455in}{4.353052in}}%
\pgfusepath{stroke}%
\end{pgfscope}%
\begin{pgfscope}%
\pgfpathrectangle{\pgfqpoint{1.764694in}{1.233139in}}{\pgfqpoint{2.651736in}{3.441667in}} %
\pgfusepath{clip}%
\pgfsetroundcap%
\pgfsetroundjoin%
\pgfsetlinewidth{2.168100pt}%
\definecolor{currentstroke}{rgb}{0.260000,0.260000,0.260000}%
\pgfsetstrokecolor{currentstroke}%
\pgfsetdash{}{0pt}%
\pgfpathmoveto{\pgfqpoint{3.488323in}{4.147363in}}%
\pgfpathlineto{\pgfqpoint{3.488323in}{4.510916in}}%
\pgfusepath{stroke}%
\end{pgfscope}%
\begin{pgfscope}%
\pgfpathrectangle{\pgfqpoint{1.764694in}{1.233139in}}{\pgfqpoint{2.651736in}{3.441667in}} %
\pgfusepath{clip}%
\pgfsetroundcap%
\pgfsetroundjoin%
\pgfsetlinewidth{2.168100pt}%
\definecolor{currentstroke}{rgb}{0.260000,0.260000,0.260000}%
\pgfsetstrokecolor{currentstroke}%
\pgfsetdash{}{0pt}%
\pgfpathmoveto{\pgfqpoint{2.692802in}{4.099214in}}%
\pgfpathlineto{\pgfqpoint{2.692802in}{4.109645in}}%
\pgfusepath{stroke}%
\end{pgfscope}%
\begin{pgfscope}%
\pgfpathrectangle{\pgfqpoint{1.764694in}{1.233139in}}{\pgfqpoint{2.651736in}{3.441667in}} %
\pgfusepath{clip}%
\pgfsetroundcap%
\pgfsetroundjoin%
\pgfsetlinewidth{2.168100pt}%
\definecolor{currentstroke}{rgb}{0.260000,0.260000,0.260000}%
\pgfsetstrokecolor{currentstroke}%
\pgfsetdash{}{0pt}%
\pgfpathmoveto{\pgfqpoint{4.018670in}{1.422578in}}%
\pgfpathlineto{\pgfqpoint{4.018670in}{1.438206in}}%
\pgfusepath{stroke}%
\end{pgfscope}%
\begin{pgfscope}%
\pgfsetrectcap%
\pgfsetmiterjoin%
\pgfsetlinewidth{1.003750pt}%
\definecolor{currentstroke}{rgb}{0.800000,0.800000,0.800000}%
\pgfsetstrokecolor{currentstroke}%
\pgfsetdash{}{0pt}%
\pgfpathmoveto{\pgfqpoint{1.764694in}{1.233139in}}%
\pgfpathlineto{\pgfqpoint{1.764694in}{4.674805in}}%
\pgfusepath{stroke}%
\end{pgfscope}%
\begin{pgfscope}%
\pgfsetrectcap%
\pgfsetmiterjoin%
\pgfsetlinewidth{1.003750pt}%
\definecolor{currentstroke}{rgb}{0.800000,0.800000,0.800000}%
\pgfsetstrokecolor{currentstroke}%
\pgfsetdash{}{0pt}%
\pgfpathmoveto{\pgfqpoint{1.764694in}{1.233139in}}%
\pgfpathlineto{\pgfqpoint{4.416430in}{1.233139in}}%
\pgfusepath{stroke}%
\end{pgfscope}%
\begin{pgfscope}%
\pgfsetbuttcap%
\pgfsetmiterjoin%
\definecolor{currentfill}{rgb}{0.798529,0.536765,0.389706}%
\pgfsetfillcolor{currentfill}%
\pgfsetlinewidth{0.803000pt}%
\definecolor{currentstroke}{rgb}{0.827451,0.827451,0.827451}%
\pgfsetstrokecolor{currentstroke}%
\pgfsetdash{}{0pt}%
\pgfpathmoveto{\pgfqpoint{1.726264in}{4.587527in}}%
\pgfpathlineto{\pgfqpoint{2.156819in}{4.587527in}}%
\pgfpathlineto{\pgfqpoint{2.156819in}{4.956972in}}%
\pgfpathlineto{\pgfqpoint{1.726264in}{4.956972in}}%
\pgfpathclose%
\pgfusepath{stroke,fill}%
\end{pgfscope}%
\begin{pgfscope}%
\pgfsetbuttcap%
\pgfsetmiterjoin%
\definecolor{currentfill}{rgb}{0.798529,0.536765,0.389706}%
\pgfsetfillcolor{currentfill}%
\pgfsetlinewidth{0.803000pt}%
\definecolor{currentstroke}{rgb}{0.827451,0.827451,0.827451}%
\pgfsetstrokecolor{currentstroke}%
\pgfsetdash{}{0pt}%
\pgfpathmoveto{\pgfqpoint{1.726264in}{4.587527in}}%
\pgfpathlineto{\pgfqpoint{2.156819in}{4.587527in}}%
\pgfpathlineto{\pgfqpoint{2.156819in}{4.956972in}}%
\pgfpathlineto{\pgfqpoint{1.726264in}{4.956972in}}%
\pgfpathclose%
\pgfusepath{clip}%
\pgfsys@defobject{currentpattern}{\pgfqpoint{0in}{0in}}{\pgfqpoint{1in}{1in}}{%
\begin{pgfscope}%
\pgfpathrectangle{\pgfqpoint{0in}{0in}}{\pgfqpoint{1in}{1in}}%
\pgfusepath{clip}%
\pgfpathmoveto{\pgfqpoint{-0.500000in}{0.500000in}}%
\pgfpathlineto{\pgfqpoint{0.500000in}{1.500000in}}%
\pgfpathmoveto{\pgfqpoint{-0.333333in}{0.333333in}}%
\pgfpathlineto{\pgfqpoint{0.666667in}{1.333333in}}%
\pgfpathmoveto{\pgfqpoint{-0.166667in}{0.166667in}}%
\pgfpathlineto{\pgfqpoint{0.833333in}{1.166667in}}%
\pgfpathmoveto{\pgfqpoint{0.000000in}{0.000000in}}%
\pgfpathlineto{\pgfqpoint{1.000000in}{1.000000in}}%
\pgfpathmoveto{\pgfqpoint{0.166667in}{-0.166667in}}%
\pgfpathlineto{\pgfqpoint{1.166667in}{0.833333in}}%
\pgfpathmoveto{\pgfqpoint{0.333333in}{-0.333333in}}%
\pgfpathlineto{\pgfqpoint{1.333333in}{0.666667in}}%
\pgfpathmoveto{\pgfqpoint{0.500000in}{-0.500000in}}%
\pgfpathlineto{\pgfqpoint{1.500000in}{0.500000in}}%
\pgfusepath{stroke}%
\end{pgfscope}%
}%
\pgfsys@transformshift{1.726264in}{4.587527in}%
\pgfsys@useobject{currentpattern}{}%
\pgfsys@transformshift{1in}{0in}%
\pgfsys@transformshift{-1in}{0in}%
\pgfsys@transformshift{0in}{1in}%
\end{pgfscope}%
\begin{pgfscope}%
\definecolor{textcolor}{rgb}{0.150000,0.150000,0.150000}%
\pgfsetstrokecolor{textcolor}%
\pgfsetfillcolor{textcolor}%
\pgftext[x=2.370708in,y=4.587527in,left,base]{\color{textcolor}\rmfamily\fontsize{38.000000}{45.600000}\selectfont DE}%
\end{pgfscope}%
\begin{pgfscope}%
\pgfsetbuttcap%
\pgfsetmiterjoin%
\definecolor{currentfill}{rgb}{0.347059,0.458824,0.641176}%
\pgfsetfillcolor{currentfill}%
\pgfsetlinewidth{0.803000pt}%
\definecolor{currentstroke}{rgb}{0.827451,0.827451,0.827451}%
\pgfsetstrokecolor{currentstroke}%
\pgfsetdash{}{0pt}%
\pgfpathmoveto{\pgfqpoint{3.380847in}{4.587527in}}%
\pgfpathlineto{\pgfqpoint{3.811403in}{4.587527in}}%
\pgfpathlineto{\pgfqpoint{3.811403in}{4.956972in}}%
\pgfpathlineto{\pgfqpoint{3.380847in}{4.956972in}}%
\pgfpathclose%
\pgfusepath{stroke,fill}%
\end{pgfscope}%
\begin{pgfscope}%
\pgfsetbuttcap%
\pgfsetmiterjoin%
\definecolor{currentfill}{rgb}{0.347059,0.458824,0.641176}%
\pgfsetfillcolor{currentfill}%
\pgfsetlinewidth{0.803000pt}%
\definecolor{currentstroke}{rgb}{0.827451,0.827451,0.827451}%
\pgfsetstrokecolor{currentstroke}%
\pgfsetdash{}{0pt}%
\pgfpathmoveto{\pgfqpoint{3.380847in}{4.587527in}}%
\pgfpathlineto{\pgfqpoint{3.811403in}{4.587527in}}%
\pgfpathlineto{\pgfqpoint{3.811403in}{4.956972in}}%
\pgfpathlineto{\pgfqpoint{3.380847in}{4.956972in}}%
\pgfpathclose%
\pgfusepath{clip}%
\pgfsys@defobject{currentpattern}{\pgfqpoint{0in}{0in}}{\pgfqpoint{1in}{1in}}{%
\begin{pgfscope}%
\pgfpathrectangle{\pgfqpoint{0in}{0in}}{\pgfqpoint{1in}{1in}}%
\pgfusepath{clip}%
\pgfpathmoveto{\pgfqpoint{-0.500000in}{0.500000in}}%
\pgfpathlineto{\pgfqpoint{0.500000in}{1.500000in}}%
\pgfpathmoveto{\pgfqpoint{-0.333333in}{0.333333in}}%
\pgfpathlineto{\pgfqpoint{0.666667in}{1.333333in}}%
\pgfpathmoveto{\pgfqpoint{-0.166667in}{0.166667in}}%
\pgfpathlineto{\pgfqpoint{0.833333in}{1.166667in}}%
\pgfpathmoveto{\pgfqpoint{0.000000in}{0.000000in}}%
\pgfpathlineto{\pgfqpoint{1.000000in}{1.000000in}}%
\pgfpathmoveto{\pgfqpoint{0.166667in}{-0.166667in}}%
\pgfpathlineto{\pgfqpoint{1.166667in}{0.833333in}}%
\pgfpathmoveto{\pgfqpoint{0.333333in}{-0.333333in}}%
\pgfpathlineto{\pgfqpoint{1.333333in}{0.666667in}}%
\pgfpathmoveto{\pgfqpoint{0.500000in}{-0.500000in}}%
\pgfpathlineto{\pgfqpoint{1.500000in}{0.500000in}}%
\pgfpathmoveto{\pgfqpoint{-0.500000in}{0.500000in}}%
\pgfpathlineto{\pgfqpoint{0.500000in}{-0.500000in}}%
\pgfpathmoveto{\pgfqpoint{-0.333333in}{0.666667in}}%
\pgfpathlineto{\pgfqpoint{0.666667in}{-0.333333in}}%
\pgfpathmoveto{\pgfqpoint{-0.166667in}{0.833333in}}%
\pgfpathlineto{\pgfqpoint{0.833333in}{-0.166667in}}%
\pgfpathmoveto{\pgfqpoint{0.000000in}{1.000000in}}%
\pgfpathlineto{\pgfqpoint{1.000000in}{0.000000in}}%
\pgfpathmoveto{\pgfqpoint{0.166667in}{1.166667in}}%
\pgfpathlineto{\pgfqpoint{1.166667in}{0.166667in}}%
\pgfpathmoveto{\pgfqpoint{0.333333in}{1.333333in}}%
\pgfpathlineto{\pgfqpoint{1.333333in}{0.333333in}}%
\pgfpathmoveto{\pgfqpoint{0.500000in}{1.500000in}}%
\pgfpathlineto{\pgfqpoint{1.500000in}{0.500000in}}%
\pgfusepath{stroke}%
\end{pgfscope}%
}%
\pgfsys@transformshift{3.380847in}{4.587527in}%
\pgfsys@useobject{currentpattern}{}%
\pgfsys@transformshift{1in}{0in}%
\pgfsys@transformshift{-1in}{0in}%
\pgfsys@transformshift{0in}{1in}%
\end{pgfscope}%
\begin{pgfscope}%
\definecolor{textcolor}{rgb}{0.150000,0.150000,0.150000}%
\pgfsetstrokecolor{textcolor}%
\pgfsetfillcolor{textcolor}%
\pgftext[x=4.025292in,y=4.587527in,left,base]{\color{textcolor}\rmfamily\fontsize{38.000000}{45.600000}\selectfont CO}%
\end{pgfscope}%
\end{pgfpicture}%
\makeatother%
\endgroup%
%
}
\caption{Workload 3}
\end{subfigure}
\begin{subfigure}[b]{\linewidth}
\centering
 \resizebox{\columnwidth}{!}{%
%% Creator: Matplotlib, PGF backend
%%
%% To include the figure in your LaTeX document, write
%%   \input{<filename>.pgf}
%%
%% Make sure the required packages are loaded in your preamble
%%   \usepackage{pgf}
%%
%% Figures using additional raster images can only be included by \input if
%% they are in the same directory as the main LaTeX file. For loading figures
%% from other directories you can use the `import` package
%%   \usepackage{import}
%% and then include the figures with
%%   \import{<path to file>}{<filename>.pgf}
%%
%% Matplotlib used the following preamble
%%   \usepackage{fontspec}
%%   \setmonofont{Andale Mono}
%%
\begingroup%
\makeatletter%
\begin{pgfpicture}%
\pgfpathrectangle{\pgfpointorigin}{\pgfqpoint{8.887494in}{3.778216in}}%
\pgfusepath{use as bounding box, clip}%
\begin{pgfscope}%
\pgfsetbuttcap%
\pgfsetmiterjoin%
\definecolor{currentfill}{rgb}{1.000000,1.000000,1.000000}%
\pgfsetfillcolor{currentfill}%
\pgfsetlinewidth{0.000000pt}%
\definecolor{currentstroke}{rgb}{1.000000,1.000000,1.000000}%
\pgfsetstrokecolor{currentstroke}%
\pgfsetdash{}{0pt}%
\pgfpathmoveto{\pgfqpoint{0.000000in}{0.000000in}}%
\pgfpathlineto{\pgfqpoint{8.887494in}{0.000000in}}%
\pgfpathlineto{\pgfqpoint{8.887494in}{3.778216in}}%
\pgfpathlineto{\pgfqpoint{0.000000in}{3.778216in}}%
\pgfpathclose%
\pgfusepath{fill}%
\end{pgfscope}%
\begin{pgfscope}%
\pgfsetbuttcap%
\pgfsetmiterjoin%
\definecolor{currentfill}{rgb}{1.000000,1.000000,1.000000}%
\pgfsetfillcolor{currentfill}%
\pgfsetlinewidth{0.000000pt}%
\definecolor{currentstroke}{rgb}{0.000000,0.000000,0.000000}%
\pgfsetstrokecolor{currentstroke}%
\pgfsetstrokeopacity{0.000000}%
\pgfsetdash{}{0pt}%
\pgfpathmoveto{\pgfqpoint{0.970827in}{0.917049in}}%
\pgfpathlineto{\pgfqpoint{8.720828in}{0.917049in}}%
\pgfpathlineto{\pgfqpoint{8.720828in}{3.182049in}}%
\pgfpathlineto{\pgfqpoint{0.970827in}{3.182049in}}%
\pgfpathclose%
\pgfusepath{fill}%
\end{pgfscope}%
\begin{pgfscope}%
\pgfpathrectangle{\pgfqpoint{0.970827in}{0.917049in}}{\pgfqpoint{7.750000in}{2.265000in}} %
\pgfusepath{clip}%
\pgfsetroundcap%
\pgfsetroundjoin%
\pgfsetlinewidth{0.803000pt}%
\definecolor{currentstroke}{rgb}{0.800000,0.800000,0.800000}%
\pgfsetstrokecolor{currentstroke}%
\pgfsetdash{}{0pt}%
\pgfpathmoveto{\pgfqpoint{1.323100in}{0.917049in}}%
\pgfpathlineto{\pgfqpoint{1.323100in}{3.182049in}}%
\pgfusepath{stroke}%
\end{pgfscope}%
\begin{pgfscope}%
\definecolor{textcolor}{rgb}{0.150000,0.150000,0.150000}%
\pgfsetstrokecolor{textcolor}%
\pgfsetfillcolor{textcolor}%
\pgftext[x=1.323100in,y=0.753161in,,top]{\color{textcolor}\rmfamily\fontsize{26.400000}{31.680000}\selectfont 1}%
\end{pgfscope}%
\begin{pgfscope}%
\pgfpathrectangle{\pgfqpoint{0.970827in}{0.917049in}}{\pgfqpoint{7.750000in}{2.265000in}} %
\pgfusepath{clip}%
\pgfsetroundcap%
\pgfsetroundjoin%
\pgfsetlinewidth{0.803000pt}%
\definecolor{currentstroke}{rgb}{0.800000,0.800000,0.800000}%
\pgfsetstrokecolor{currentstroke}%
\pgfsetdash{}{0pt}%
\pgfpathmoveto{\pgfqpoint{2.329594in}{0.917049in}}%
\pgfpathlineto{\pgfqpoint{2.329594in}{3.182049in}}%
\pgfusepath{stroke}%
\end{pgfscope}%
\begin{pgfscope}%
\definecolor{textcolor}{rgb}{0.150000,0.150000,0.150000}%
\pgfsetstrokecolor{textcolor}%
\pgfsetfillcolor{textcolor}%
\pgftext[x=2.329594in,y=0.753161in,,top]{\color{textcolor}\rmfamily\fontsize{26.400000}{31.680000}\selectfont 2}%
\end{pgfscope}%
\begin{pgfscope}%
\pgfpathrectangle{\pgfqpoint{0.970827in}{0.917049in}}{\pgfqpoint{7.750000in}{2.265000in}} %
\pgfusepath{clip}%
\pgfsetroundcap%
\pgfsetroundjoin%
\pgfsetlinewidth{0.803000pt}%
\definecolor{currentstroke}{rgb}{0.800000,0.800000,0.800000}%
\pgfsetstrokecolor{currentstroke}%
\pgfsetdash{}{0pt}%
\pgfpathmoveto{\pgfqpoint{3.336087in}{0.917049in}}%
\pgfpathlineto{\pgfqpoint{3.336087in}{3.182049in}}%
\pgfusepath{stroke}%
\end{pgfscope}%
\begin{pgfscope}%
\definecolor{textcolor}{rgb}{0.150000,0.150000,0.150000}%
\pgfsetstrokecolor{textcolor}%
\pgfsetfillcolor{textcolor}%
\pgftext[x=3.336087in,y=0.753161in,,top]{\color{textcolor}\rmfamily\fontsize{26.400000}{31.680000}\selectfont 3}%
\end{pgfscope}%
\begin{pgfscope}%
\pgfpathrectangle{\pgfqpoint{0.970827in}{0.917049in}}{\pgfqpoint{7.750000in}{2.265000in}} %
\pgfusepath{clip}%
\pgfsetroundcap%
\pgfsetroundjoin%
\pgfsetlinewidth{0.803000pt}%
\definecolor{currentstroke}{rgb}{0.800000,0.800000,0.800000}%
\pgfsetstrokecolor{currentstroke}%
\pgfsetdash{}{0pt}%
\pgfpathmoveto{\pgfqpoint{4.342581in}{0.917049in}}%
\pgfpathlineto{\pgfqpoint{4.342581in}{3.182049in}}%
\pgfusepath{stroke}%
\end{pgfscope}%
\begin{pgfscope}%
\definecolor{textcolor}{rgb}{0.150000,0.150000,0.150000}%
\pgfsetstrokecolor{textcolor}%
\pgfsetfillcolor{textcolor}%
\pgftext[x=4.342581in,y=0.753161in,,top]{\color{textcolor}\rmfamily\fontsize{26.400000}{31.680000}\selectfont 4}%
\end{pgfscope}%
\begin{pgfscope}%
\pgfpathrectangle{\pgfqpoint{0.970827in}{0.917049in}}{\pgfqpoint{7.750000in}{2.265000in}} %
\pgfusepath{clip}%
\pgfsetroundcap%
\pgfsetroundjoin%
\pgfsetlinewidth{0.803000pt}%
\definecolor{currentstroke}{rgb}{0.800000,0.800000,0.800000}%
\pgfsetstrokecolor{currentstroke}%
\pgfsetdash{}{0pt}%
\pgfpathmoveto{\pgfqpoint{5.349074in}{0.917049in}}%
\pgfpathlineto{\pgfqpoint{5.349074in}{3.182049in}}%
\pgfusepath{stroke}%
\end{pgfscope}%
\begin{pgfscope}%
\definecolor{textcolor}{rgb}{0.150000,0.150000,0.150000}%
\pgfsetstrokecolor{textcolor}%
\pgfsetfillcolor{textcolor}%
\pgftext[x=5.349074in,y=0.753161in,,top]{\color{textcolor}\rmfamily\fontsize{26.400000}{31.680000}\selectfont 5}%
\end{pgfscope}%
\begin{pgfscope}%
\pgfpathrectangle{\pgfqpoint{0.970827in}{0.917049in}}{\pgfqpoint{7.750000in}{2.265000in}} %
\pgfusepath{clip}%
\pgfsetroundcap%
\pgfsetroundjoin%
\pgfsetlinewidth{0.803000pt}%
\definecolor{currentstroke}{rgb}{0.800000,0.800000,0.800000}%
\pgfsetstrokecolor{currentstroke}%
\pgfsetdash{}{0pt}%
\pgfpathmoveto{\pgfqpoint{6.355568in}{0.917049in}}%
\pgfpathlineto{\pgfqpoint{6.355568in}{3.182049in}}%
\pgfusepath{stroke}%
\end{pgfscope}%
\begin{pgfscope}%
\definecolor{textcolor}{rgb}{0.150000,0.150000,0.150000}%
\pgfsetstrokecolor{textcolor}%
\pgfsetfillcolor{textcolor}%
\pgftext[x=6.355568in,y=0.753161in,,top]{\color{textcolor}\rmfamily\fontsize{26.400000}{31.680000}\selectfont 6}%
\end{pgfscope}%
\begin{pgfscope}%
\pgfpathrectangle{\pgfqpoint{0.970827in}{0.917049in}}{\pgfqpoint{7.750000in}{2.265000in}} %
\pgfusepath{clip}%
\pgfsetroundcap%
\pgfsetroundjoin%
\pgfsetlinewidth{0.803000pt}%
\definecolor{currentstroke}{rgb}{0.800000,0.800000,0.800000}%
\pgfsetstrokecolor{currentstroke}%
\pgfsetdash{}{0pt}%
\pgfpathmoveto{\pgfqpoint{7.362061in}{0.917049in}}%
\pgfpathlineto{\pgfqpoint{7.362061in}{3.182049in}}%
\pgfusepath{stroke}%
\end{pgfscope}%
\begin{pgfscope}%
\definecolor{textcolor}{rgb}{0.150000,0.150000,0.150000}%
\pgfsetstrokecolor{textcolor}%
\pgfsetfillcolor{textcolor}%
\pgftext[x=7.362061in,y=0.753161in,,top]{\color{textcolor}\rmfamily\fontsize{26.400000}{31.680000}\selectfont 7}%
\end{pgfscope}%
\begin{pgfscope}%
\pgfpathrectangle{\pgfqpoint{0.970827in}{0.917049in}}{\pgfqpoint{7.750000in}{2.265000in}} %
\pgfusepath{clip}%
\pgfsetroundcap%
\pgfsetroundjoin%
\pgfsetlinewidth{0.803000pt}%
\definecolor{currentstroke}{rgb}{0.800000,0.800000,0.800000}%
\pgfsetstrokecolor{currentstroke}%
\pgfsetdash{}{0pt}%
\pgfpathmoveto{\pgfqpoint{8.368555in}{0.917049in}}%
\pgfpathlineto{\pgfqpoint{8.368555in}{3.182049in}}%
\pgfusepath{stroke}%
\end{pgfscope}%
\begin{pgfscope}%
\definecolor{textcolor}{rgb}{0.150000,0.150000,0.150000}%
\pgfsetstrokecolor{textcolor}%
\pgfsetfillcolor{textcolor}%
\pgftext[x=8.368555in,y=0.753161in,,top]{\color{textcolor}\rmfamily\fontsize{26.400000}{31.680000}\selectfont 8}%
\end{pgfscope}%
\begin{pgfscope}%
\definecolor{textcolor}{rgb}{0.150000,0.150000,0.150000}%
\pgfsetstrokecolor{textcolor}%
\pgfsetfillcolor{textcolor}%
\pgftext[x=4.845827in,y=0.371639in,,top]{\color{textcolor}\rmfamily\fontsize{22.000000}{26.400000}\selectfont Workload}%
\end{pgfscope}%
\begin{pgfscope}%
\pgfpathrectangle{\pgfqpoint{0.970827in}{0.917049in}}{\pgfqpoint{7.750000in}{2.265000in}} %
\pgfusepath{clip}%
\pgfsetroundcap%
\pgfsetroundjoin%
\pgfsetlinewidth{0.803000pt}%
\definecolor{currentstroke}{rgb}{0.800000,0.800000,0.800000}%
\pgfsetstrokecolor{currentstroke}%
\pgfsetdash{}{0pt}%
\pgfpathmoveto{\pgfqpoint{0.970827in}{0.917049in}}%
\pgfpathlineto{\pgfqpoint{8.720828in}{0.917049in}}%
\pgfusepath{stroke}%
\end{pgfscope}%
\begin{pgfscope}%
\definecolor{textcolor}{rgb}{0.150000,0.150000,0.150000}%
\pgfsetstrokecolor{textcolor}%
\pgfsetfillcolor{textcolor}%
\pgftext[x=0.638639in,y=0.789816in,left,base]{\color{textcolor}\rmfamily\fontsize{26.400000}{31.680000}\selectfont 0}%
\end{pgfscope}%
\begin{pgfscope}%
\pgfpathrectangle{\pgfqpoint{0.970827in}{0.917049in}}{\pgfqpoint{7.750000in}{2.265000in}} %
\pgfusepath{clip}%
\pgfsetroundcap%
\pgfsetroundjoin%
\pgfsetlinewidth{0.803000pt}%
\definecolor{currentstroke}{rgb}{0.800000,0.800000,0.800000}%
\pgfsetstrokecolor{currentstroke}%
\pgfsetdash{}{0pt}%
\pgfpathmoveto{\pgfqpoint{0.970827in}{2.049549in}}%
\pgfpathlineto{\pgfqpoint{8.720828in}{2.049549in}}%
\pgfusepath{stroke}%
\end{pgfscope}%
\begin{pgfscope}%
\definecolor{textcolor}{rgb}{0.150000,0.150000,0.150000}%
\pgfsetstrokecolor{textcolor}%
\pgfsetfillcolor{textcolor}%
\pgftext[x=0.460805in,y=1.922316in,left,base]{\color{textcolor}\rmfamily\fontsize{26.400000}{31.680000}\selectfont 1k}%
\end{pgfscope}%
\begin{pgfscope}%
\pgfpathrectangle{\pgfqpoint{0.970827in}{0.917049in}}{\pgfqpoint{7.750000in}{2.265000in}} %
\pgfusepath{clip}%
\pgfsetroundcap%
\pgfsetroundjoin%
\pgfsetlinewidth{0.803000pt}%
\definecolor{currentstroke}{rgb}{0.800000,0.800000,0.800000}%
\pgfsetstrokecolor{currentstroke}%
\pgfsetdash{}{0pt}%
\pgfpathmoveto{\pgfqpoint{0.970827in}{3.182049in}}%
\pgfpathlineto{\pgfqpoint{8.720828in}{3.182049in}}%
\pgfusepath{stroke}%
\end{pgfscope}%
\begin{pgfscope}%
\definecolor{textcolor}{rgb}{0.150000,0.150000,0.150000}%
\pgfsetstrokecolor{textcolor}%
\pgfsetfillcolor{textcolor}%
\pgftext[x=0.460805in,y=3.054816in,left,base]{\color{textcolor}\rmfamily\fontsize{26.400000}{31.680000}\selectfont 2k}%
\end{pgfscope}%
\begin{pgfscope}%
\definecolor{textcolor}{rgb}{0.150000,0.150000,0.150000}%
\pgfsetstrokecolor{textcolor}%
\pgfsetfillcolor{textcolor}%
\pgftext[x=0.405250in,y=2.049549in,,bottom,rotate=90.000000]{\color{textcolor}\rmfamily\fontsize{22.000000}{26.400000}\selectfont Cumulative Run Time (s)}%
\end{pgfscope}%
\begin{pgfscope}%
\pgfpathrectangle{\pgfqpoint{0.970827in}{0.917049in}}{\pgfqpoint{7.750000in}{2.265000in}} %
\pgfusepath{clip}%
\pgfsetbuttcap%
\pgfsetroundjoin%
\definecolor{currentfill}{rgb}{0.298039,0.447059,0.690196}%
\pgfsetfillcolor{currentfill}%
\pgfsetfillopacity{0.200000}%
\pgfsetlinewidth{0.803000pt}%
\definecolor{currentstroke}{rgb}{0.298039,0.447059,0.690196}%
\pgfsetstrokecolor{currentstroke}%
\pgfsetstrokeopacity{0.200000}%
\pgfsetdash{}{0pt}%
\pgfpathmoveto{\pgfqpoint{1.323100in}{1.144789in}}%
\pgfpathlineto{\pgfqpoint{1.323100in}{1.142805in}}%
\pgfpathlineto{\pgfqpoint{2.329594in}{1.337373in}}%
\pgfpathlineto{\pgfqpoint{3.336087in}{1.662518in}}%
\pgfpathlineto{\pgfqpoint{4.342581in}{1.727351in}}%
\pgfpathlineto{\pgfqpoint{5.349074in}{1.777113in}}%
\pgfpathlineto{\pgfqpoint{6.355568in}{1.803907in}}%
\pgfpathlineto{\pgfqpoint{7.362061in}{1.828708in}}%
\pgfpathlineto{\pgfqpoint{8.368555in}{1.899170in}}%
\pgfpathlineto{\pgfqpoint{8.368555in}{1.926724in}}%
\pgfpathlineto{\pgfqpoint{8.368555in}{1.926724in}}%
\pgfpathlineto{\pgfqpoint{7.362061in}{1.856800in}}%
\pgfpathlineto{\pgfqpoint{6.355568in}{1.809984in}}%
\pgfpathlineto{\pgfqpoint{5.349074in}{1.784201in}}%
\pgfpathlineto{\pgfqpoint{4.342581in}{1.730547in}}%
\pgfpathlineto{\pgfqpoint{3.336087in}{1.667423in}}%
\pgfpathlineto{\pgfqpoint{2.329594in}{1.338300in}}%
\pgfpathlineto{\pgfqpoint{1.323100in}{1.144789in}}%
\pgfpathclose%
\pgfusepath{stroke,fill}%
\end{pgfscope}%
\begin{pgfscope}%
\pgfpathrectangle{\pgfqpoint{0.970827in}{0.917049in}}{\pgfqpoint{7.750000in}{2.265000in}} %
\pgfusepath{clip}%
\pgfsetbuttcap%
\pgfsetroundjoin%
\definecolor{currentfill}{rgb}{0.866667,0.517647,0.321569}%
\pgfsetfillcolor{currentfill}%
\pgfsetfillopacity{0.200000}%
\pgfsetlinewidth{0.803000pt}%
\definecolor{currentstroke}{rgb}{0.866667,0.517647,0.321569}%
\pgfsetstrokecolor{currentstroke}%
\pgfsetstrokeopacity{0.200000}%
\pgfsetdash{}{0pt}%
\pgfpathmoveto{\pgfqpoint{1.323100in}{1.134397in}}%
\pgfpathlineto{\pgfqpoint{1.323100in}{1.132277in}}%
\pgfpathlineto{\pgfqpoint{2.329594in}{1.359844in}}%
\pgfpathlineto{\pgfqpoint{3.336087in}{1.764130in}}%
\pgfpathlineto{\pgfqpoint{4.342581in}{1.943740in}}%
\pgfpathlineto{\pgfqpoint{5.349074in}{2.067041in}}%
\pgfpathlineto{\pgfqpoint{6.355568in}{2.245476in}}%
\pgfpathlineto{\pgfqpoint{7.362061in}{2.638406in}}%
\pgfpathlineto{\pgfqpoint{8.368555in}{2.930629in}}%
\pgfpathlineto{\pgfqpoint{8.368555in}{2.947915in}}%
\pgfpathlineto{\pgfqpoint{8.368555in}{2.947915in}}%
\pgfpathlineto{\pgfqpoint{7.362061in}{2.657878in}}%
\pgfpathlineto{\pgfqpoint{6.355568in}{2.262720in}}%
\pgfpathlineto{\pgfqpoint{5.349074in}{2.084099in}}%
\pgfpathlineto{\pgfqpoint{4.342581in}{1.953815in}}%
\pgfpathlineto{\pgfqpoint{3.336087in}{1.776360in}}%
\pgfpathlineto{\pgfqpoint{2.329594in}{1.364670in}}%
\pgfpathlineto{\pgfqpoint{1.323100in}{1.134397in}}%
\pgfpathclose%
\pgfusepath{stroke,fill}%
\end{pgfscope}%
\begin{pgfscope}%
\pgfpathrectangle{\pgfqpoint{0.970827in}{0.917049in}}{\pgfqpoint{7.750000in}{2.265000in}} %
\pgfusepath{clip}%
\pgfsetbuttcap%
\pgfsetroundjoin%
\pgfsetlinewidth{3.011250pt}%
\definecolor{currentstroke}{rgb}{0.298039,0.447059,0.690196}%
\pgfsetstrokecolor{currentstroke}%
\pgfsetdash{{3.000000pt}{0.000000pt}}{0.000000pt}%
\pgfpathmoveto{\pgfqpoint{1.323100in}{1.143755in}}%
\pgfpathlineto{\pgfqpoint{2.329594in}{1.337852in}}%
\pgfpathlineto{\pgfqpoint{3.336087in}{1.665208in}}%
\pgfpathlineto{\pgfqpoint{4.342581in}{1.728975in}}%
\pgfpathlineto{\pgfqpoint{5.349074in}{1.781277in}}%
\pgfpathlineto{\pgfqpoint{6.355568in}{1.807699in}}%
\pgfpathlineto{\pgfqpoint{7.362061in}{1.841603in}}%
\pgfpathlineto{\pgfqpoint{8.368555in}{1.912593in}}%
\pgfusepath{stroke}%
\end{pgfscope}%
\begin{pgfscope}%
\pgfpathrectangle{\pgfqpoint{0.970827in}{0.917049in}}{\pgfqpoint{7.750000in}{2.265000in}} %
\pgfusepath{clip}%
\pgfsetbuttcap%
\pgfsetroundjoin%
\definecolor{currentfill}{rgb}{0.298039,0.447059,0.690196}%
\pgfsetfillcolor{currentfill}%
\pgfsetlinewidth{0.752812pt}%
\definecolor{currentstroke}{rgb}{1.000000,1.000000,1.000000}%
\pgfsetstrokecolor{currentstroke}%
\pgfsetdash{}{0pt}%
\pgfsys@defobject{currentmarker}{\pgfqpoint{-0.104167in}{-0.104167in}}{\pgfqpoint{0.104167in}{0.104167in}}{%
\pgfpathmoveto{\pgfqpoint{0.000000in}{-0.104167in}}%
\pgfpathcurveto{\pgfqpoint{0.027625in}{-0.104167in}}{\pgfqpoint{0.054123in}{-0.093191in}}{\pgfqpoint{0.073657in}{-0.073657in}}%
\pgfpathcurveto{\pgfqpoint{0.093191in}{-0.054123in}}{\pgfqpoint{0.104167in}{-0.027625in}}{\pgfqpoint{0.104167in}{0.000000in}}%
\pgfpathcurveto{\pgfqpoint{0.104167in}{0.027625in}}{\pgfqpoint{0.093191in}{0.054123in}}{\pgfqpoint{0.073657in}{0.073657in}}%
\pgfpathcurveto{\pgfqpoint{0.054123in}{0.093191in}}{\pgfqpoint{0.027625in}{0.104167in}}{\pgfqpoint{0.000000in}{0.104167in}}%
\pgfpathcurveto{\pgfqpoint{-0.027625in}{0.104167in}}{\pgfqpoint{-0.054123in}{0.093191in}}{\pgfqpoint{-0.073657in}{0.073657in}}%
\pgfpathcurveto{\pgfqpoint{-0.093191in}{0.054123in}}{\pgfqpoint{-0.104167in}{0.027625in}}{\pgfqpoint{-0.104167in}{0.000000in}}%
\pgfpathcurveto{\pgfqpoint{-0.104167in}{-0.027625in}}{\pgfqpoint{-0.093191in}{-0.054123in}}{\pgfqpoint{-0.073657in}{-0.073657in}}%
\pgfpathcurveto{\pgfqpoint{-0.054123in}{-0.093191in}}{\pgfqpoint{-0.027625in}{-0.104167in}}{\pgfqpoint{0.000000in}{-0.104167in}}%
\pgfpathclose%
\pgfusepath{stroke,fill}%
}%
\begin{pgfscope}%
\pgfsys@transformshift{1.323100in}{1.143755in}%
\pgfsys@useobject{currentmarker}{}%
\end{pgfscope}%
\begin{pgfscope}%
\pgfsys@transformshift{2.329594in}{1.337852in}%
\pgfsys@useobject{currentmarker}{}%
\end{pgfscope}%
\begin{pgfscope}%
\pgfsys@transformshift{3.336087in}{1.665208in}%
\pgfsys@useobject{currentmarker}{}%
\end{pgfscope}%
\begin{pgfscope}%
\pgfsys@transformshift{4.342581in}{1.728975in}%
\pgfsys@useobject{currentmarker}{}%
\end{pgfscope}%
\begin{pgfscope}%
\pgfsys@transformshift{5.349074in}{1.781277in}%
\pgfsys@useobject{currentmarker}{}%
\end{pgfscope}%
\begin{pgfscope}%
\pgfsys@transformshift{6.355568in}{1.807699in}%
\pgfsys@useobject{currentmarker}{}%
\end{pgfscope}%
\begin{pgfscope}%
\pgfsys@transformshift{7.362061in}{1.841603in}%
\pgfsys@useobject{currentmarker}{}%
\end{pgfscope}%
\begin{pgfscope}%
\pgfsys@transformshift{8.368555in}{1.912593in}%
\pgfsys@useobject{currentmarker}{}%
\end{pgfscope}%
\end{pgfscope}%
\begin{pgfscope}%
\pgfpathrectangle{\pgfqpoint{0.970827in}{0.917049in}}{\pgfqpoint{7.750000in}{2.265000in}} %
\pgfusepath{clip}%
\pgfsetbuttcap%
\pgfsetroundjoin%
\pgfsetlinewidth{3.011250pt}%
\definecolor{currentstroke}{rgb}{0.866667,0.517647,0.321569}%
\pgfsetstrokecolor{currentstroke}%
\pgfsetdash{{3.000000pt}{3.000000pt}}{0.000000pt}%
\pgfpathmoveto{\pgfqpoint{1.323100in}{1.133511in}}%
\pgfpathlineto{\pgfqpoint{2.329594in}{1.362479in}}%
\pgfpathlineto{\pgfqpoint{3.336087in}{1.769893in}}%
\pgfpathlineto{\pgfqpoint{4.342581in}{1.949548in}}%
\pgfpathlineto{\pgfqpoint{5.349074in}{2.075956in}}%
\pgfpathlineto{\pgfqpoint{6.355568in}{2.255243in}}%
\pgfpathlineto{\pgfqpoint{7.362061in}{2.651155in}}%
\pgfpathlineto{\pgfqpoint{8.368555in}{2.941970in}}%
\pgfusepath{stroke}%
\end{pgfscope}%
\begin{pgfscope}%
\pgfpathrectangle{\pgfqpoint{0.970827in}{0.917049in}}{\pgfqpoint{7.750000in}{2.265000in}} %
\pgfusepath{clip}%
\pgfsetbuttcap%
\pgfsetmiterjoin%
\definecolor{currentfill}{rgb}{0.866667,0.517647,0.321569}%
\pgfsetfillcolor{currentfill}%
\pgfsetlinewidth{0.752812pt}%
\definecolor{currentstroke}{rgb}{1.000000,1.000000,1.000000}%
\pgfsetstrokecolor{currentstroke}%
\pgfsetdash{}{0pt}%
\pgfsys@defobject{currentmarker}{\pgfqpoint{-0.104167in}{-0.104167in}}{\pgfqpoint{0.104167in}{0.104167in}}{%
\pgfpathmoveto{\pgfqpoint{0.000000in}{0.104167in}}%
\pgfpathlineto{\pgfqpoint{-0.104167in}{-0.104167in}}%
\pgfpathlineto{\pgfqpoint{0.104167in}{-0.104167in}}%
\pgfpathclose%
\pgfusepath{stroke,fill}%
}%
\begin{pgfscope}%
\pgfsys@transformshift{1.323100in}{1.133511in}%
\pgfsys@useobject{currentmarker}{}%
\end{pgfscope}%
\begin{pgfscope}%
\pgfsys@transformshift{2.329594in}{1.362479in}%
\pgfsys@useobject{currentmarker}{}%
\end{pgfscope}%
\begin{pgfscope}%
\pgfsys@transformshift{3.336087in}{1.769893in}%
\pgfsys@useobject{currentmarker}{}%
\end{pgfscope}%
\begin{pgfscope}%
\pgfsys@transformshift{4.342581in}{1.949548in}%
\pgfsys@useobject{currentmarker}{}%
\end{pgfscope}%
\begin{pgfscope}%
\pgfsys@transformshift{5.349074in}{2.075956in}%
\pgfsys@useobject{currentmarker}{}%
\end{pgfscope}%
\begin{pgfscope}%
\pgfsys@transformshift{6.355568in}{2.255243in}%
\pgfsys@useobject{currentmarker}{}%
\end{pgfscope}%
\begin{pgfscope}%
\pgfsys@transformshift{7.362061in}{2.651155in}%
\pgfsys@useobject{currentmarker}{}%
\end{pgfscope}%
\begin{pgfscope}%
\pgfsys@transformshift{8.368555in}{2.941970in}%
\pgfsys@useobject{currentmarker}{}%
\end{pgfscope}%
\end{pgfscope}%
\begin{pgfscope}%
\pgfsetrectcap%
\pgfsetmiterjoin%
\pgfsetlinewidth{1.003750pt}%
\definecolor{currentstroke}{rgb}{0.800000,0.800000,0.800000}%
\pgfsetstrokecolor{currentstroke}%
\pgfsetdash{}{0pt}%
\pgfpathmoveto{\pgfqpoint{0.970827in}{0.917049in}}%
\pgfpathlineto{\pgfqpoint{0.970827in}{3.182049in}}%
\pgfusepath{stroke}%
\end{pgfscope}%
\begin{pgfscope}%
\pgfsetrectcap%
\pgfsetmiterjoin%
\pgfsetlinewidth{1.003750pt}%
\definecolor{currentstroke}{rgb}{0.800000,0.800000,0.800000}%
\pgfsetstrokecolor{currentstroke}%
\pgfsetdash{}{0pt}%
\pgfpathmoveto{\pgfqpoint{8.720828in}{0.917049in}}%
\pgfpathlineto{\pgfqpoint{8.720828in}{3.182049in}}%
\pgfusepath{stroke}%
\end{pgfscope}%
\begin{pgfscope}%
\pgfsetrectcap%
\pgfsetmiterjoin%
\pgfsetlinewidth{1.003750pt}%
\definecolor{currentstroke}{rgb}{0.800000,0.800000,0.800000}%
\pgfsetstrokecolor{currentstroke}%
\pgfsetdash{}{0pt}%
\pgfpathmoveto{\pgfqpoint{0.970827in}{0.917049in}}%
\pgfpathlineto{\pgfqpoint{8.720828in}{0.917049in}}%
\pgfusepath{stroke}%
\end{pgfscope}%
\begin{pgfscope}%
\pgfsetrectcap%
\pgfsetmiterjoin%
\pgfsetlinewidth{1.003750pt}%
\definecolor{currentstroke}{rgb}{0.800000,0.800000,0.800000}%
\pgfsetstrokecolor{currentstroke}%
\pgfsetdash{}{0pt}%
\pgfpathmoveto{\pgfqpoint{0.970827in}{3.182049in}}%
\pgfpathlineto{\pgfqpoint{8.720828in}{3.182049in}}%
\pgfusepath{stroke}%
\end{pgfscope}%
\begin{pgfscope}%
\pgfsetroundcap%
\pgfsetroundjoin%
\pgfsetlinewidth{3.011250pt}%
\definecolor{currentstroke}{rgb}{1.000000,1.000000,1.000000}%
\pgfsetstrokecolor{currentstroke}%
\pgfsetdash{}{0pt}%
\pgfpathmoveto{\pgfqpoint{1.418861in}{3.403216in}}%
\pgfpathlineto{\pgfqpoint{2.152194in}{3.403216in}}%
\pgfusepath{stroke}%
\end{pgfscope}%
\begin{pgfscope}%
\pgfsetbuttcap%
\pgfsetroundjoin%
\pgfsetlinewidth{3.011250pt}%
\definecolor{currentstroke}{rgb}{0.298039,0.447059,0.690196}%
\pgfsetstrokecolor{currentstroke}%
\pgfsetdash{{3.000000pt}{0.000000pt}}{0.000000pt}%
\pgfpathmoveto{\pgfqpoint{1.418861in}{2.891716in}}%
\pgfpathlineto{\pgfqpoint{2.152194in}{2.891716in}}%
\pgfusepath{stroke}%
\end{pgfscope}%
\begin{pgfscope}%
\pgfsetbuttcap%
\pgfsetroundjoin%
\definecolor{currentfill}{rgb}{0.298039,0.447059,0.690196}%
\pgfsetfillcolor{currentfill}%
\pgfsetlinewidth{0.000000pt}%
\definecolor{currentstroke}{rgb}{0.298039,0.447059,0.690196}%
\pgfsetstrokecolor{currentstroke}%
\pgfsetdash{}{0pt}%
\pgfsys@defobject{currentmarker}{\pgfqpoint{-0.104167in}{-0.104167in}}{\pgfqpoint{0.104167in}{0.104167in}}{%
\pgfpathmoveto{\pgfqpoint{0.000000in}{-0.104167in}}%
\pgfpathcurveto{\pgfqpoint{0.027625in}{-0.104167in}}{\pgfqpoint{0.054123in}{-0.093191in}}{\pgfqpoint{0.073657in}{-0.073657in}}%
\pgfpathcurveto{\pgfqpoint{0.093191in}{-0.054123in}}{\pgfqpoint{0.104167in}{-0.027625in}}{\pgfqpoint{0.104167in}{0.000000in}}%
\pgfpathcurveto{\pgfqpoint{0.104167in}{0.027625in}}{\pgfqpoint{0.093191in}{0.054123in}}{\pgfqpoint{0.073657in}{0.073657in}}%
\pgfpathcurveto{\pgfqpoint{0.054123in}{0.093191in}}{\pgfqpoint{0.027625in}{0.104167in}}{\pgfqpoint{0.000000in}{0.104167in}}%
\pgfpathcurveto{\pgfqpoint{-0.027625in}{0.104167in}}{\pgfqpoint{-0.054123in}{0.093191in}}{\pgfqpoint{-0.073657in}{0.073657in}}%
\pgfpathcurveto{\pgfqpoint{-0.093191in}{0.054123in}}{\pgfqpoint{-0.104167in}{0.027625in}}{\pgfqpoint{-0.104167in}{0.000000in}}%
\pgfpathcurveto{\pgfqpoint{-0.104167in}{-0.027625in}}{\pgfqpoint{-0.093191in}{-0.054123in}}{\pgfqpoint{-0.073657in}{-0.073657in}}%
\pgfpathcurveto{\pgfqpoint{-0.054123in}{-0.093191in}}{\pgfqpoint{-0.027625in}{-0.104167in}}{\pgfqpoint{0.000000in}{-0.104167in}}%
\pgfpathclose%
\pgfusepath{fill}%
}%
\begin{pgfscope}%
\pgfsys@transformshift{1.785527in}{2.891716in}%
\pgfsys@useobject{currentmarker}{}%
\end{pgfscope}%
\end{pgfscope}%
\begin{pgfscope}%
\definecolor{textcolor}{rgb}{0.150000,0.150000,0.150000}%
\pgfsetstrokecolor{textcolor}%
\pgfsetfillcolor{textcolor}%
\pgftext[x=2.445527in,y=2.763383in,left,base]{\color{textcolor}\rmfamily\fontsize{26.400000}{31.680000}\selectfont CO}%
\end{pgfscope}%
\begin{pgfscope}%
\pgfsetbuttcap%
\pgfsetroundjoin%
\pgfsetlinewidth{3.011250pt}%
\definecolor{currentstroke}{rgb}{0.866667,0.517647,0.321569}%
\pgfsetstrokecolor{currentstroke}%
\pgfsetdash{{3.000000pt}{3.000000pt}}{0.000000pt}%
\pgfpathmoveto{\pgfqpoint{1.418861in}{2.380217in}}%
\pgfpathlineto{\pgfqpoint{2.152194in}{2.380217in}}%
\pgfusepath{stroke}%
\end{pgfscope}%
\begin{pgfscope}%
\pgfsetbuttcap%
\pgfsetmiterjoin%
\definecolor{currentfill}{rgb}{0.866667,0.517647,0.321569}%
\pgfsetfillcolor{currentfill}%
\pgfsetlinewidth{0.000000pt}%
\definecolor{currentstroke}{rgb}{0.866667,0.517647,0.321569}%
\pgfsetstrokecolor{currentstroke}%
\pgfsetdash{}{0pt}%
\pgfsys@defobject{currentmarker}{\pgfqpoint{-0.104167in}{-0.104167in}}{\pgfqpoint{0.104167in}{0.104167in}}{%
\pgfpathmoveto{\pgfqpoint{0.000000in}{0.104167in}}%
\pgfpathlineto{\pgfqpoint{-0.104167in}{-0.104167in}}%
\pgfpathlineto{\pgfqpoint{0.104167in}{-0.104167in}}%
\pgfpathclose%
\pgfusepath{fill}%
}%
\begin{pgfscope}%
\pgfsys@transformshift{1.785527in}{2.380217in}%
\pgfsys@useobject{currentmarker}{}%
\end{pgfscope}%
\end{pgfscope}%
\begin{pgfscope}%
\definecolor{textcolor}{rgb}{0.150000,0.150000,0.150000}%
\pgfsetstrokecolor{textcolor}%
\pgfsetfillcolor{textcolor}%
\pgftext[x=2.445527in,y=2.251883in,left,base]{\color{textcolor}\rmfamily\fontsize{26.400000}{31.680000}\selectfont KG}%
\end{pgfscope}%
\end{pgfpicture}%
\makeatother%
\endgroup%
%
}
\caption{Execution of Kaggle workloads in sequence}
\end{subfigure}
\caption{Run-time of Kaggle workloads. KG = Default Kaggle, CO = Collaborative Optimizer.}
\label{exp-reuse-kaggle-same-workload}
\end{figure}

Figure \ref{exp-reuse-kaggle-same-workload}(d) shows the cumulative run-time of executing the workloads of Table \ref{kaggle-workload} consecutively.
%This corresponds to a real scenario, where after some scripts in a collaborative environment gain popularity, i.e., Workloads 1-3 (after 7000 , other users modify and improve the scripts, i.e., Workloads 4-8.
Workloads 4-8 operate on the artifacts generated in Workloads 1-3; thus, instead of recomputing those artifacts, our collaborative optimizer reuses the existing artifacts.
As a result, the cumulative run-time of running the 8 workloads decreases by 50\%.
The experiment shows that optimizing even a single execution of each workload has an impact on the total run-time.
In a real collaborative environment, there are hundreds of more modified scripts and possibly thousands of repeated execution of such scripts, resulting in 1000s of hours of reduction in the cumulative run-time.
\subsection{Materialization}
In this set of experiments, we investigate the impact of our materialization algorithms on storage and run-time.

\begin{figure}[t]
\begin{subfigure}[b]{0.5\linewidth}
\centering
 \resizebox{\columnwidth}{!}{%
\input{../images/experiment-results/kaggle_home_credit/materialization/size-8.pgf}%
}
\caption{Budget = 8 GB}
\end{subfigure}%
\begin{subfigure}[b]{0.5\linewidth}
\centering
 \resizebox{\columnwidth}{!}{%
%% Creator: Matplotlib, PGF backend
%%
%% To include the figure in your LaTeX document, write
%%   \input{<filename>.pgf}
%%
%% Make sure the required packages are loaded in your preamble
%%   \usepackage{pgf}
%%
%% Figures using additional raster images can only be included by \input if
%% they are in the same directory as the main LaTeX file. For loading figures
%% from other directories you can use the `import` package
%%   \usepackage{import}
%% and then include the figures with
%%   \import{<path to file>}{<filename>.pgf}
%%
%% Matplotlib used the following preamble
%%   \usepackage[utf8x]{inputenc}
%%   \usepackage[T1]{fontenc}
%%
\begingroup%
\makeatletter%
\begin{pgfpicture}%
\pgfpathrectangle{\pgfpointorigin}{\pgfqpoint{7.974532in}{4.884138in}}%
\pgfusepath{use as bounding box, clip}%
\begin{pgfscope}%
\pgfsetbuttcap%
\pgfsetmiterjoin%
\definecolor{currentfill}{rgb}{1.000000,1.000000,1.000000}%
\pgfsetfillcolor{currentfill}%
\pgfsetlinewidth{0.000000pt}%
\definecolor{currentstroke}{rgb}{1.000000,1.000000,1.000000}%
\pgfsetstrokecolor{currentstroke}%
\pgfsetdash{}{0pt}%
\pgfpathmoveto{\pgfqpoint{0.000000in}{0.000000in}}%
\pgfpathlineto{\pgfqpoint{7.974532in}{0.000000in}}%
\pgfpathlineto{\pgfqpoint{7.974532in}{4.884138in}}%
\pgfpathlineto{\pgfqpoint{0.000000in}{4.884138in}}%
\pgfpathclose%
\pgfusepath{fill}%
\end{pgfscope}%
\begin{pgfscope}%
\pgfsetbuttcap%
\pgfsetmiterjoin%
\definecolor{currentfill}{rgb}{1.000000,1.000000,1.000000}%
\pgfsetfillcolor{currentfill}%
\pgfsetlinewidth{0.000000pt}%
\definecolor{currentstroke}{rgb}{0.000000,0.000000,0.000000}%
\pgfsetstrokecolor{currentstroke}%
\pgfsetstrokeopacity{0.000000}%
\pgfsetdash{}{0pt}%
\pgfpathmoveto{\pgfqpoint{1.307384in}{1.215049in}}%
\pgfpathlineto{\pgfqpoint{7.507384in}{1.215049in}}%
\pgfpathlineto{\pgfqpoint{7.507384in}{4.235049in}}%
\pgfpathlineto{\pgfqpoint{1.307384in}{4.235049in}}%
\pgfpathclose%
\pgfusepath{fill}%
\end{pgfscope}%
\begin{pgfscope}%
\pgfpathrectangle{\pgfqpoint{1.307384in}{1.215049in}}{\pgfqpoint{6.200000in}{3.020000in}} %
\pgfusepath{clip}%
\pgfsetroundcap%
\pgfsetroundjoin%
\pgfsetlinewidth{1.003750pt}%
\definecolor{currentstroke}{rgb}{0.500000,0.500000,0.500000}%
\pgfsetstrokecolor{currentstroke}%
\pgfsetdash{}{0pt}%
\pgfpathmoveto{\pgfqpoint{1.589202in}{1.215049in}}%
\pgfpathlineto{\pgfqpoint{1.589202in}{4.235049in}}%
\pgfusepath{stroke}%
\end{pgfscope}%
\begin{pgfscope}%
\definecolor{textcolor}{rgb}{0.150000,0.150000,0.150000}%
\pgfsetstrokecolor{textcolor}%
\pgfsetfillcolor{textcolor}%
\pgftext[x=1.589202in,y=1.034493in,,top]{\color{textcolor}\rmfamily\fontsize{34.000000}{40.800000}\selectfont 1}%
\end{pgfscope}%
\begin{pgfscope}%
\pgfpathrectangle{\pgfqpoint{1.307384in}{1.215049in}}{\pgfqpoint{6.200000in}{3.020000in}} %
\pgfusepath{clip}%
\pgfsetroundcap%
\pgfsetroundjoin%
\pgfsetlinewidth{1.003750pt}%
\definecolor{currentstroke}{rgb}{0.500000,0.500000,0.500000}%
\pgfsetstrokecolor{currentstroke}%
\pgfsetdash{}{0pt}%
\pgfpathmoveto{\pgfqpoint{2.394397in}{1.215049in}}%
\pgfpathlineto{\pgfqpoint{2.394397in}{4.235049in}}%
\pgfusepath{stroke}%
\end{pgfscope}%
\begin{pgfscope}%
\definecolor{textcolor}{rgb}{0.150000,0.150000,0.150000}%
\pgfsetstrokecolor{textcolor}%
\pgfsetfillcolor{textcolor}%
\pgftext[x=2.394397in,y=1.034493in,,top]{\color{textcolor}\rmfamily\fontsize{34.000000}{40.800000}\selectfont 2}%
\end{pgfscope}%
\begin{pgfscope}%
\pgfpathrectangle{\pgfqpoint{1.307384in}{1.215049in}}{\pgfqpoint{6.200000in}{3.020000in}} %
\pgfusepath{clip}%
\pgfsetroundcap%
\pgfsetroundjoin%
\pgfsetlinewidth{1.003750pt}%
\definecolor{currentstroke}{rgb}{0.500000,0.500000,0.500000}%
\pgfsetstrokecolor{currentstroke}%
\pgfsetdash{}{0pt}%
\pgfpathmoveto{\pgfqpoint{3.199592in}{1.215049in}}%
\pgfpathlineto{\pgfqpoint{3.199592in}{4.235049in}}%
\pgfusepath{stroke}%
\end{pgfscope}%
\begin{pgfscope}%
\definecolor{textcolor}{rgb}{0.150000,0.150000,0.150000}%
\pgfsetstrokecolor{textcolor}%
\pgfsetfillcolor{textcolor}%
\pgftext[x=3.199592in,y=1.034493in,,top]{\color{textcolor}\rmfamily\fontsize{34.000000}{40.800000}\selectfont 3}%
\end{pgfscope}%
\begin{pgfscope}%
\pgfpathrectangle{\pgfqpoint{1.307384in}{1.215049in}}{\pgfqpoint{6.200000in}{3.020000in}} %
\pgfusepath{clip}%
\pgfsetroundcap%
\pgfsetroundjoin%
\pgfsetlinewidth{1.003750pt}%
\definecolor{currentstroke}{rgb}{0.500000,0.500000,0.500000}%
\pgfsetstrokecolor{currentstroke}%
\pgfsetdash{}{0pt}%
\pgfpathmoveto{\pgfqpoint{4.004786in}{1.215049in}}%
\pgfpathlineto{\pgfqpoint{4.004786in}{4.235049in}}%
\pgfusepath{stroke}%
\end{pgfscope}%
\begin{pgfscope}%
\definecolor{textcolor}{rgb}{0.150000,0.150000,0.150000}%
\pgfsetstrokecolor{textcolor}%
\pgfsetfillcolor{textcolor}%
\pgftext[x=4.004786in,y=1.034493in,,top]{\color{textcolor}\rmfamily\fontsize{34.000000}{40.800000}\selectfont 4}%
\end{pgfscope}%
\begin{pgfscope}%
\pgfpathrectangle{\pgfqpoint{1.307384in}{1.215049in}}{\pgfqpoint{6.200000in}{3.020000in}} %
\pgfusepath{clip}%
\pgfsetroundcap%
\pgfsetroundjoin%
\pgfsetlinewidth{1.003750pt}%
\definecolor{currentstroke}{rgb}{0.500000,0.500000,0.500000}%
\pgfsetstrokecolor{currentstroke}%
\pgfsetdash{}{0pt}%
\pgfpathmoveto{\pgfqpoint{4.809981in}{1.215049in}}%
\pgfpathlineto{\pgfqpoint{4.809981in}{4.235049in}}%
\pgfusepath{stroke}%
\end{pgfscope}%
\begin{pgfscope}%
\definecolor{textcolor}{rgb}{0.150000,0.150000,0.150000}%
\pgfsetstrokecolor{textcolor}%
\pgfsetfillcolor{textcolor}%
\pgftext[x=4.809981in,y=1.034493in,,top]{\color{textcolor}\rmfamily\fontsize{34.000000}{40.800000}\selectfont 5}%
\end{pgfscope}%
\begin{pgfscope}%
\pgfpathrectangle{\pgfqpoint{1.307384in}{1.215049in}}{\pgfqpoint{6.200000in}{3.020000in}} %
\pgfusepath{clip}%
\pgfsetroundcap%
\pgfsetroundjoin%
\pgfsetlinewidth{1.003750pt}%
\definecolor{currentstroke}{rgb}{0.500000,0.500000,0.500000}%
\pgfsetstrokecolor{currentstroke}%
\pgfsetdash{}{0pt}%
\pgfpathmoveto{\pgfqpoint{5.615176in}{1.215049in}}%
\pgfpathlineto{\pgfqpoint{5.615176in}{4.235049in}}%
\pgfusepath{stroke}%
\end{pgfscope}%
\begin{pgfscope}%
\definecolor{textcolor}{rgb}{0.150000,0.150000,0.150000}%
\pgfsetstrokecolor{textcolor}%
\pgfsetfillcolor{textcolor}%
\pgftext[x=5.615176in,y=1.034493in,,top]{\color{textcolor}\rmfamily\fontsize{34.000000}{40.800000}\selectfont 6}%
\end{pgfscope}%
\begin{pgfscope}%
\pgfpathrectangle{\pgfqpoint{1.307384in}{1.215049in}}{\pgfqpoint{6.200000in}{3.020000in}} %
\pgfusepath{clip}%
\pgfsetroundcap%
\pgfsetroundjoin%
\pgfsetlinewidth{1.003750pt}%
\definecolor{currentstroke}{rgb}{0.500000,0.500000,0.500000}%
\pgfsetstrokecolor{currentstroke}%
\pgfsetdash{}{0pt}%
\pgfpathmoveto{\pgfqpoint{6.420371in}{1.215049in}}%
\pgfpathlineto{\pgfqpoint{6.420371in}{4.235049in}}%
\pgfusepath{stroke}%
\end{pgfscope}%
\begin{pgfscope}%
\definecolor{textcolor}{rgb}{0.150000,0.150000,0.150000}%
\pgfsetstrokecolor{textcolor}%
\pgfsetfillcolor{textcolor}%
\pgftext[x=6.420371in,y=1.034493in,,top]{\color{textcolor}\rmfamily\fontsize{34.000000}{40.800000}\selectfont 7}%
\end{pgfscope}%
\begin{pgfscope}%
\pgfpathrectangle{\pgfqpoint{1.307384in}{1.215049in}}{\pgfqpoint{6.200000in}{3.020000in}} %
\pgfusepath{clip}%
\pgfsetroundcap%
\pgfsetroundjoin%
\pgfsetlinewidth{1.003750pt}%
\definecolor{currentstroke}{rgb}{0.500000,0.500000,0.500000}%
\pgfsetstrokecolor{currentstroke}%
\pgfsetdash{}{0pt}%
\pgfpathmoveto{\pgfqpoint{7.225566in}{1.215049in}}%
\pgfpathlineto{\pgfqpoint{7.225566in}{4.235049in}}%
\pgfusepath{stroke}%
\end{pgfscope}%
\begin{pgfscope}%
\definecolor{textcolor}{rgb}{0.150000,0.150000,0.150000}%
\pgfsetstrokecolor{textcolor}%
\pgfsetfillcolor{textcolor}%
\pgftext[x=7.225566in,y=1.034493in,,top]{\color{textcolor}\rmfamily\fontsize{34.000000}{40.800000}\selectfont 8}%
\end{pgfscope}%
\begin{pgfscope}%
\definecolor{textcolor}{rgb}{0.150000,0.150000,0.150000}%
\pgfsetstrokecolor{textcolor}%
\pgfsetfillcolor{textcolor}%
\pgftext[x=4.407384in,y=0.539469in,,top]{\color{textcolor}\rmfamily\fontsize{40.000000}{48.000000}\selectfont Workload}%
\end{pgfscope}%
\begin{pgfscope}%
\pgfpathrectangle{\pgfqpoint{1.307384in}{1.215049in}}{\pgfqpoint{6.200000in}{3.020000in}} %
\pgfusepath{clip}%
\pgfsetroundcap%
\pgfsetroundjoin%
\pgfsetlinewidth{1.003750pt}%
\definecolor{currentstroke}{rgb}{0.500000,0.500000,0.500000}%
\pgfsetstrokecolor{currentstroke}%
\pgfsetdash{}{0pt}%
\pgfpathmoveto{\pgfqpoint{1.307384in}{1.215049in}}%
\pgfpathlineto{\pgfqpoint{7.507384in}{1.215049in}}%
\pgfusepath{stroke}%
\end{pgfscope}%
\begin{pgfscope}%
\definecolor{textcolor}{rgb}{0.150000,0.150000,0.150000}%
\pgfsetstrokecolor{textcolor}%
\pgfsetfillcolor{textcolor}%
\pgftext[x=0.968350in,y=1.043716in,left,base]{\color{textcolor}\rmfamily\fontsize{34.000000}{40.800000}\selectfont \(\displaystyle 0\)}%
\end{pgfscope}%
\begin{pgfscope}%
\pgfpathrectangle{\pgfqpoint{1.307384in}{1.215049in}}{\pgfqpoint{6.200000in}{3.020000in}} %
\pgfusepath{clip}%
\pgfsetroundcap%
\pgfsetroundjoin%
\pgfsetlinewidth{1.003750pt}%
\definecolor{currentstroke}{rgb}{0.500000,0.500000,0.500000}%
\pgfsetstrokecolor{currentstroke}%
\pgfsetdash{}{0pt}%
\pgfpathmoveto{\pgfqpoint{1.307384in}{1.754335in}}%
\pgfpathlineto{\pgfqpoint{7.507384in}{1.754335in}}%
\pgfusepath{stroke}%
\end{pgfscope}%
\begin{pgfscope}%
\definecolor{textcolor}{rgb}{0.150000,0.150000,0.150000}%
\pgfsetstrokecolor{textcolor}%
\pgfsetfillcolor{textcolor}%
\pgftext[x=0.809872in,y=1.583001in,left,base]{\color{textcolor}\rmfamily\fontsize{34.000000}{40.800000}\selectfont \(\displaystyle 25\)}%
\end{pgfscope}%
\begin{pgfscope}%
\pgfpathrectangle{\pgfqpoint{1.307384in}{1.215049in}}{\pgfqpoint{6.200000in}{3.020000in}} %
\pgfusepath{clip}%
\pgfsetroundcap%
\pgfsetroundjoin%
\pgfsetlinewidth{1.003750pt}%
\definecolor{currentstroke}{rgb}{0.500000,0.500000,0.500000}%
\pgfsetstrokecolor{currentstroke}%
\pgfsetdash{}{0pt}%
\pgfpathmoveto{\pgfqpoint{1.307384in}{2.293620in}}%
\pgfpathlineto{\pgfqpoint{7.507384in}{2.293620in}}%
\pgfusepath{stroke}%
\end{pgfscope}%
\begin{pgfscope}%
\definecolor{textcolor}{rgb}{0.150000,0.150000,0.150000}%
\pgfsetstrokecolor{textcolor}%
\pgfsetfillcolor{textcolor}%
\pgftext[x=0.809872in,y=2.122287in,left,base]{\color{textcolor}\rmfamily\fontsize{34.000000}{40.800000}\selectfont \(\displaystyle 50\)}%
\end{pgfscope}%
\begin{pgfscope}%
\pgfpathrectangle{\pgfqpoint{1.307384in}{1.215049in}}{\pgfqpoint{6.200000in}{3.020000in}} %
\pgfusepath{clip}%
\pgfsetroundcap%
\pgfsetroundjoin%
\pgfsetlinewidth{1.003750pt}%
\definecolor{currentstroke}{rgb}{0.500000,0.500000,0.500000}%
\pgfsetstrokecolor{currentstroke}%
\pgfsetdash{}{0pt}%
\pgfpathmoveto{\pgfqpoint{1.307384in}{2.832906in}}%
\pgfpathlineto{\pgfqpoint{7.507384in}{2.832906in}}%
\pgfusepath{stroke}%
\end{pgfscope}%
\begin{pgfscope}%
\definecolor{textcolor}{rgb}{0.150000,0.150000,0.150000}%
\pgfsetstrokecolor{textcolor}%
\pgfsetfillcolor{textcolor}%
\pgftext[x=0.809872in,y=2.661573in,left,base]{\color{textcolor}\rmfamily\fontsize{34.000000}{40.800000}\selectfont \(\displaystyle 75\)}%
\end{pgfscope}%
\begin{pgfscope}%
\pgfpathrectangle{\pgfqpoint{1.307384in}{1.215049in}}{\pgfqpoint{6.200000in}{3.020000in}} %
\pgfusepath{clip}%
\pgfsetroundcap%
\pgfsetroundjoin%
\pgfsetlinewidth{1.003750pt}%
\definecolor{currentstroke}{rgb}{0.500000,0.500000,0.500000}%
\pgfsetstrokecolor{currentstroke}%
\pgfsetdash{}{0pt}%
\pgfpathmoveto{\pgfqpoint{1.307384in}{3.372192in}}%
\pgfpathlineto{\pgfqpoint{7.507384in}{3.372192in}}%
\pgfusepath{stroke}%
\end{pgfscope}%
\begin{pgfscope}%
\definecolor{textcolor}{rgb}{0.150000,0.150000,0.150000}%
\pgfsetstrokecolor{textcolor}%
\pgfsetfillcolor{textcolor}%
\pgftext[x=0.651394in,y=3.200859in,left,base]{\color{textcolor}\rmfamily\fontsize{34.000000}{40.800000}\selectfont \(\displaystyle 100\)}%
\end{pgfscope}%
\begin{pgfscope}%
\pgfpathrectangle{\pgfqpoint{1.307384in}{1.215049in}}{\pgfqpoint{6.200000in}{3.020000in}} %
\pgfusepath{clip}%
\pgfsetroundcap%
\pgfsetroundjoin%
\pgfsetlinewidth{1.003750pt}%
\definecolor{currentstroke}{rgb}{0.500000,0.500000,0.500000}%
\pgfsetstrokecolor{currentstroke}%
\pgfsetdash{}{0pt}%
\pgfpathmoveto{\pgfqpoint{1.307384in}{3.911477in}}%
\pgfpathlineto{\pgfqpoint{7.507384in}{3.911477in}}%
\pgfusepath{stroke}%
\end{pgfscope}%
\begin{pgfscope}%
\definecolor{textcolor}{rgb}{0.150000,0.150000,0.150000}%
\pgfsetstrokecolor{textcolor}%
\pgfsetfillcolor{textcolor}%
\pgftext[x=0.651394in,y=3.740144in,left,base]{\color{textcolor}\rmfamily\fontsize{34.000000}{40.800000}\selectfont \(\displaystyle 125\)}%
\end{pgfscope}%
\begin{pgfscope}%
\definecolor{textcolor}{rgb}{0.150000,0.150000,0.150000}%
\pgfsetstrokecolor{textcolor}%
\pgfsetfillcolor{textcolor}%
\pgftext[x=0.595839in,y=2.725049in,,bottom,rotate=90.000000]{\color{textcolor}\rmfamily\fontsize{40.000000}{48.000000}\selectfont Size (GB)}%
\end{pgfscope}%
\begin{pgfscope}%
\pgfpathrectangle{\pgfqpoint{1.307384in}{1.215049in}}{\pgfqpoint{6.200000in}{3.020000in}} %
\pgfusepath{clip}%
\pgfsetbuttcap%
\pgfsetroundjoin%
\definecolor{currentfill}{rgb}{0.298039,0.447059,0.690196}%
\pgfsetfillcolor{currentfill}%
\pgfsetfillopacity{0.200000}%
\pgfsetlinewidth{1.003750pt}%
\definecolor{currentstroke}{rgb}{0.298039,0.447059,0.690196}%
\pgfsetstrokecolor{currentstroke}%
\pgfsetstrokeopacity{0.200000}%
\pgfsetdash{}{0pt}%
\pgfpathmoveto{\pgfqpoint{1.589202in}{1.528456in}}%
\pgfpathlineto{\pgfqpoint{1.589202in}{1.528456in}}%
\pgfpathlineto{\pgfqpoint{2.394397in}{2.045252in}}%
\pgfpathlineto{\pgfqpoint{3.199592in}{2.355195in}}%
\pgfpathlineto{\pgfqpoint{4.004786in}{2.480319in}}%
\pgfpathlineto{\pgfqpoint{4.809981in}{2.602371in}}%
\pgfpathlineto{\pgfqpoint{5.615176in}{2.861339in}}%
\pgfpathlineto{\pgfqpoint{6.420371in}{3.248992in}}%
\pgfpathlineto{\pgfqpoint{7.225566in}{3.283720in}}%
\pgfpathlineto{\pgfqpoint{7.225566in}{3.325435in}}%
\pgfpathlineto{\pgfqpoint{7.225566in}{3.325435in}}%
\pgfpathlineto{\pgfqpoint{6.420371in}{3.290708in}}%
\pgfpathlineto{\pgfqpoint{5.615176in}{2.886451in}}%
\pgfpathlineto{\pgfqpoint{4.809981in}{2.633246in}}%
\pgfpathlineto{\pgfqpoint{4.004786in}{2.511193in}}%
\pgfpathlineto{\pgfqpoint{3.199592in}{2.386070in}}%
\pgfpathlineto{\pgfqpoint{2.394397in}{2.045528in}}%
\pgfpathlineto{\pgfqpoint{1.589202in}{1.528456in}}%
\pgfpathclose%
\pgfusepath{stroke,fill}%
\end{pgfscope}%
\begin{pgfscope}%
\pgfpathrectangle{\pgfqpoint{1.307384in}{1.215049in}}{\pgfqpoint{6.200000in}{3.020000in}} %
\pgfusepath{clip}%
\pgfsetbuttcap%
\pgfsetroundjoin%
\definecolor{currentfill}{rgb}{0.866667,0.517647,0.321569}%
\pgfsetfillcolor{currentfill}%
\pgfsetfillopacity{0.200000}%
\pgfsetlinewidth{1.003750pt}%
\definecolor{currentstroke}{rgb}{0.866667,0.517647,0.321569}%
\pgfsetstrokecolor{currentstroke}%
\pgfsetstrokeopacity{0.200000}%
\pgfsetdash{}{0pt}%
\pgfpathmoveto{\pgfqpoint{1.589202in}{1.528456in}}%
\pgfpathlineto{\pgfqpoint{1.589202in}{1.528456in}}%
\pgfpathlineto{\pgfqpoint{2.394397in}{1.614056in}}%
\pgfpathlineto{\pgfqpoint{3.199592in}{1.704411in}}%
\pgfpathlineto{\pgfqpoint{4.004786in}{1.704411in}}%
\pgfpathlineto{\pgfqpoint{4.809981in}{1.704410in}}%
\pgfpathlineto{\pgfqpoint{5.615176in}{1.704411in}}%
\pgfpathlineto{\pgfqpoint{6.420371in}{1.704165in}}%
\pgfpathlineto{\pgfqpoint{7.225566in}{1.704411in}}%
\pgfpathlineto{\pgfqpoint{7.225566in}{1.704411in}}%
\pgfpathlineto{\pgfqpoint{7.225566in}{1.704411in}}%
\pgfpathlineto{\pgfqpoint{6.420371in}{1.704266in}}%
\pgfpathlineto{\pgfqpoint{5.615176in}{1.704411in}}%
\pgfpathlineto{\pgfqpoint{4.809981in}{1.704411in}}%
\pgfpathlineto{\pgfqpoint{4.004786in}{1.704411in}}%
\pgfpathlineto{\pgfqpoint{3.199592in}{1.704411in}}%
\pgfpathlineto{\pgfqpoint{2.394397in}{1.614056in}}%
\pgfpathlineto{\pgfqpoint{1.589202in}{1.528456in}}%
\pgfpathclose%
\pgfusepath{stroke,fill}%
\end{pgfscope}%
\begin{pgfscope}%
\pgfpathrectangle{\pgfqpoint{1.307384in}{1.215049in}}{\pgfqpoint{6.200000in}{3.020000in}} %
\pgfusepath{clip}%
\pgfsetbuttcap%
\pgfsetroundjoin%
\definecolor{currentfill}{rgb}{0.333333,0.658824,0.407843}%
\pgfsetfillcolor{currentfill}%
\pgfsetfillopacity{0.200000}%
\pgfsetlinewidth{1.003750pt}%
\definecolor{currentstroke}{rgb}{0.333333,0.658824,0.407843}%
\pgfsetstrokecolor{currentstroke}%
\pgfsetstrokeopacity{0.200000}%
\pgfsetdash{}{0pt}%
\pgfpathmoveto{\pgfqpoint{1.589202in}{1.510416in}}%
\pgfpathlineto{\pgfqpoint{1.589202in}{1.510416in}}%
\pgfpathlineto{\pgfqpoint{2.394397in}{1.594156in}}%
\pgfpathlineto{\pgfqpoint{3.199592in}{1.605531in}}%
\pgfpathlineto{\pgfqpoint{4.004786in}{1.605531in}}%
\pgfpathlineto{\pgfqpoint{4.809981in}{1.605531in}}%
\pgfpathlineto{\pgfqpoint{5.615176in}{1.567572in}}%
\pgfpathlineto{\pgfqpoint{6.420371in}{1.625569in}}%
\pgfpathlineto{\pgfqpoint{7.225566in}{1.647065in}}%
\pgfpathlineto{\pgfqpoint{7.225566in}{1.647065in}}%
\pgfpathlineto{\pgfqpoint{7.225566in}{1.647065in}}%
\pgfpathlineto{\pgfqpoint{6.420371in}{1.625569in}}%
\pgfpathlineto{\pgfqpoint{5.615176in}{1.567572in}}%
\pgfpathlineto{\pgfqpoint{4.809981in}{1.605531in}}%
\pgfpathlineto{\pgfqpoint{4.004786in}{1.605531in}}%
\pgfpathlineto{\pgfqpoint{3.199592in}{1.605531in}}%
\pgfpathlineto{\pgfqpoint{2.394397in}{1.594156in}}%
\pgfpathlineto{\pgfqpoint{1.589202in}{1.510416in}}%
\pgfpathclose%
\pgfusepath{stroke,fill}%
\end{pgfscope}%
\begin{pgfscope}%
\pgfpathrectangle{\pgfqpoint{1.307384in}{1.215049in}}{\pgfqpoint{6.200000in}{3.020000in}} %
\pgfusepath{clip}%
\pgfsetbuttcap%
\pgfsetroundjoin%
\definecolor{currentfill}{rgb}{0.768627,0.305882,0.321569}%
\pgfsetfillcolor{currentfill}%
\pgfsetfillopacity{0.200000}%
\pgfsetlinewidth{1.003750pt}%
\definecolor{currentstroke}{rgb}{0.768627,0.305882,0.321569}%
\pgfsetstrokecolor{currentstroke}%
\pgfsetstrokeopacity{0.200000}%
\pgfsetdash{}{0pt}%
\pgfpathmoveto{\pgfqpoint{1.589202in}{1.528456in}}%
\pgfpathlineto{\pgfqpoint{1.589202in}{1.528456in}}%
\pgfpathlineto{\pgfqpoint{2.394397in}{2.055351in}}%
\pgfpathlineto{\pgfqpoint{3.199592in}{3.845298in}}%
\pgfpathlineto{\pgfqpoint{4.004786in}{3.845319in}}%
\pgfpathlineto{\pgfqpoint{4.809981in}{3.967372in}}%
\pgfpathlineto{\pgfqpoint{5.615176in}{3.967441in}}%
\pgfpathlineto{\pgfqpoint{6.420371in}{3.967441in}}%
\pgfpathlineto{\pgfqpoint{7.225566in}{3.977407in}}%
\pgfpathlineto{\pgfqpoint{7.225566in}{3.977407in}}%
\pgfpathlineto{\pgfqpoint{7.225566in}{3.977407in}}%
\pgfpathlineto{\pgfqpoint{6.420371in}{3.967441in}}%
\pgfpathlineto{\pgfqpoint{5.615176in}{3.967441in}}%
\pgfpathlineto{\pgfqpoint{4.809981in}{3.967372in}}%
\pgfpathlineto{\pgfqpoint{4.004786in}{3.845319in}}%
\pgfpathlineto{\pgfqpoint{3.199592in}{3.845298in}}%
\pgfpathlineto{\pgfqpoint{2.394397in}{2.055351in}}%
\pgfpathlineto{\pgfqpoint{1.589202in}{1.528456in}}%
\pgfpathclose%
\pgfusepath{stroke,fill}%
\end{pgfscope}%
\begin{pgfscope}%
\pgfpathrectangle{\pgfqpoint{1.307384in}{1.215049in}}{\pgfqpoint{6.200000in}{3.020000in}} %
\pgfusepath{clip}%
\pgfsetbuttcap%
\pgfsetroundjoin%
\pgfsetlinewidth{8.030000pt}%
\definecolor{currentstroke}{rgb}{0.298039,0.447059,0.690196}%
\pgfsetstrokecolor{currentstroke}%
\pgfsetdash{{8.000000pt}{0.000000pt}}{0.000000pt}%
\pgfpathmoveto{\pgfqpoint{1.589202in}{1.528456in}}%
\pgfpathlineto{\pgfqpoint{2.394397in}{2.045436in}}%
\pgfpathlineto{\pgfqpoint{3.199592in}{2.372609in}}%
\pgfpathlineto{\pgfqpoint{4.004786in}{2.497733in}}%
\pgfpathlineto{\pgfqpoint{4.809981in}{2.619785in}}%
\pgfpathlineto{\pgfqpoint{5.615176in}{2.871680in}}%
\pgfpathlineto{\pgfqpoint{6.420371in}{3.264868in}}%
\pgfpathlineto{\pgfqpoint{7.225566in}{3.299596in}}%
\pgfusepath{stroke}%
\end{pgfscope}%
\begin{pgfscope}%
\pgfpathrectangle{\pgfqpoint{1.307384in}{1.215049in}}{\pgfqpoint{6.200000in}{3.020000in}} %
\pgfusepath{clip}%
\pgfsetbuttcap%
\pgfsetroundjoin%
\definecolor{currentfill}{rgb}{0.298039,0.447059,0.690196}%
\pgfsetfillcolor{currentfill}%
\pgfsetlinewidth{0.752812pt}%
\definecolor{currentstroke}{rgb}{1.000000,1.000000,1.000000}%
\pgfsetstrokecolor{currentstroke}%
\pgfsetdash{}{0pt}%
\pgfsys@defobject{currentmarker}{\pgfqpoint{-0.104167in}{-0.104167in}}{\pgfqpoint{0.104167in}{0.104167in}}{%
\pgfpathmoveto{\pgfqpoint{0.000000in}{-0.104167in}}%
\pgfpathcurveto{\pgfqpoint{0.027625in}{-0.104167in}}{\pgfqpoint{0.054123in}{-0.093191in}}{\pgfqpoint{0.073657in}{-0.073657in}}%
\pgfpathcurveto{\pgfqpoint{0.093191in}{-0.054123in}}{\pgfqpoint{0.104167in}{-0.027625in}}{\pgfqpoint{0.104167in}{0.000000in}}%
\pgfpathcurveto{\pgfqpoint{0.104167in}{0.027625in}}{\pgfqpoint{0.093191in}{0.054123in}}{\pgfqpoint{0.073657in}{0.073657in}}%
\pgfpathcurveto{\pgfqpoint{0.054123in}{0.093191in}}{\pgfqpoint{0.027625in}{0.104167in}}{\pgfqpoint{0.000000in}{0.104167in}}%
\pgfpathcurveto{\pgfqpoint{-0.027625in}{0.104167in}}{\pgfqpoint{-0.054123in}{0.093191in}}{\pgfqpoint{-0.073657in}{0.073657in}}%
\pgfpathcurveto{\pgfqpoint{-0.093191in}{0.054123in}}{\pgfqpoint{-0.104167in}{0.027625in}}{\pgfqpoint{-0.104167in}{0.000000in}}%
\pgfpathcurveto{\pgfqpoint{-0.104167in}{-0.027625in}}{\pgfqpoint{-0.093191in}{-0.054123in}}{\pgfqpoint{-0.073657in}{-0.073657in}}%
\pgfpathcurveto{\pgfqpoint{-0.054123in}{-0.093191in}}{\pgfqpoint{-0.027625in}{-0.104167in}}{\pgfqpoint{0.000000in}{-0.104167in}}%
\pgfpathclose%
\pgfusepath{stroke,fill}%
}%
\begin{pgfscope}%
\pgfsys@transformshift{1.589202in}{1.528456in}%
\pgfsys@useobject{currentmarker}{}%
\end{pgfscope}%
\begin{pgfscope}%
\pgfsys@transformshift{2.394397in}{2.045436in}%
\pgfsys@useobject{currentmarker}{}%
\end{pgfscope}%
\begin{pgfscope}%
\pgfsys@transformshift{3.199592in}{2.372609in}%
\pgfsys@useobject{currentmarker}{}%
\end{pgfscope}%
\begin{pgfscope}%
\pgfsys@transformshift{4.004786in}{2.497733in}%
\pgfsys@useobject{currentmarker}{}%
\end{pgfscope}%
\begin{pgfscope}%
\pgfsys@transformshift{4.809981in}{2.619785in}%
\pgfsys@useobject{currentmarker}{}%
\end{pgfscope}%
\begin{pgfscope}%
\pgfsys@transformshift{5.615176in}{2.871680in}%
\pgfsys@useobject{currentmarker}{}%
\end{pgfscope}%
\begin{pgfscope}%
\pgfsys@transformshift{6.420371in}{3.264868in}%
\pgfsys@useobject{currentmarker}{}%
\end{pgfscope}%
\begin{pgfscope}%
\pgfsys@transformshift{7.225566in}{3.299596in}%
\pgfsys@useobject{currentmarker}{}%
\end{pgfscope}%
\end{pgfscope}%
\begin{pgfscope}%
\pgfpathrectangle{\pgfqpoint{1.307384in}{1.215049in}}{\pgfqpoint{6.200000in}{3.020000in}} %
\pgfusepath{clip}%
\pgfsetbuttcap%
\pgfsetroundjoin%
\pgfsetlinewidth{8.030000pt}%
\definecolor{currentstroke}{rgb}{0.866667,0.517647,0.321569}%
\pgfsetstrokecolor{currentstroke}%
\pgfsetdash{{24.000000pt}{8.000000pt}}{0.000000pt}%
\pgfpathmoveto{\pgfqpoint{1.589202in}{1.528456in}}%
\pgfpathlineto{\pgfqpoint{2.394397in}{1.614056in}}%
\pgfpathlineto{\pgfqpoint{3.199592in}{1.704411in}}%
\pgfpathlineto{\pgfqpoint{4.004786in}{1.704411in}}%
\pgfpathlineto{\pgfqpoint{4.809981in}{1.704411in}}%
\pgfpathlineto{\pgfqpoint{5.615176in}{1.704411in}}%
\pgfpathlineto{\pgfqpoint{6.420371in}{1.704232in}}%
\pgfpathlineto{\pgfqpoint{7.225566in}{1.704411in}}%
\pgfusepath{stroke}%
\end{pgfscope}%
\begin{pgfscope}%
\pgfpathrectangle{\pgfqpoint{1.307384in}{1.215049in}}{\pgfqpoint{6.200000in}{3.020000in}} %
\pgfusepath{clip}%
\pgfsetbuttcap%
\pgfsetmiterjoin%
\definecolor{currentfill}{rgb}{0.866667,0.517647,0.321569}%
\pgfsetfillcolor{currentfill}%
\pgfsetlinewidth{0.752812pt}%
\definecolor{currentstroke}{rgb}{1.000000,1.000000,1.000000}%
\pgfsetstrokecolor{currentstroke}%
\pgfsetdash{}{0pt}%
\pgfsys@defobject{currentmarker}{\pgfqpoint{-0.104167in}{-0.104167in}}{\pgfqpoint{0.104167in}{0.104167in}}{%
\pgfpathmoveto{\pgfqpoint{0.000000in}{0.104167in}}%
\pgfpathlineto{\pgfqpoint{-0.104167in}{-0.104167in}}%
\pgfpathlineto{\pgfqpoint{0.104167in}{-0.104167in}}%
\pgfpathclose%
\pgfusepath{stroke,fill}%
}%
\begin{pgfscope}%
\pgfsys@transformshift{1.589202in}{1.528456in}%
\pgfsys@useobject{currentmarker}{}%
\end{pgfscope}%
\begin{pgfscope}%
\pgfsys@transformshift{2.394397in}{1.614056in}%
\pgfsys@useobject{currentmarker}{}%
\end{pgfscope}%
\begin{pgfscope}%
\pgfsys@transformshift{3.199592in}{1.704411in}%
\pgfsys@useobject{currentmarker}{}%
\end{pgfscope}%
\begin{pgfscope}%
\pgfsys@transformshift{4.004786in}{1.704411in}%
\pgfsys@useobject{currentmarker}{}%
\end{pgfscope}%
\begin{pgfscope}%
\pgfsys@transformshift{4.809981in}{1.704411in}%
\pgfsys@useobject{currentmarker}{}%
\end{pgfscope}%
\begin{pgfscope}%
\pgfsys@transformshift{5.615176in}{1.704411in}%
\pgfsys@useobject{currentmarker}{}%
\end{pgfscope}%
\begin{pgfscope}%
\pgfsys@transformshift{6.420371in}{1.704232in}%
\pgfsys@useobject{currentmarker}{}%
\end{pgfscope}%
\begin{pgfscope}%
\pgfsys@transformshift{7.225566in}{1.704411in}%
\pgfsys@useobject{currentmarker}{}%
\end{pgfscope}%
\end{pgfscope}%
\begin{pgfscope}%
\pgfpathrectangle{\pgfqpoint{1.307384in}{1.215049in}}{\pgfqpoint{6.200000in}{3.020000in}} %
\pgfusepath{clip}%
\pgfsetbuttcap%
\pgfsetroundjoin%
\pgfsetlinewidth{8.030000pt}%
\definecolor{currentstroke}{rgb}{0.333333,0.658824,0.407843}%
\pgfsetstrokecolor{currentstroke}%
\pgfsetdash{{16.000000pt}{16.000000pt}}{0.000000pt}%
\pgfpathmoveto{\pgfqpoint{1.589202in}{1.510416in}}%
\pgfpathlineto{\pgfqpoint{2.394397in}{1.594156in}}%
\pgfpathlineto{\pgfqpoint{3.199592in}{1.605531in}}%
\pgfpathlineto{\pgfqpoint{4.004786in}{1.605531in}}%
\pgfpathlineto{\pgfqpoint{4.809981in}{1.605531in}}%
\pgfpathlineto{\pgfqpoint{5.615176in}{1.567572in}}%
\pgfpathlineto{\pgfqpoint{6.420371in}{1.625569in}}%
\pgfpathlineto{\pgfqpoint{7.225566in}{1.647065in}}%
\pgfusepath{stroke}%
\end{pgfscope}%
\begin{pgfscope}%
\pgfpathrectangle{\pgfqpoint{1.307384in}{1.215049in}}{\pgfqpoint{6.200000in}{3.020000in}} %
\pgfusepath{clip}%
\pgfsetbuttcap%
\pgfsetmiterjoin%
\definecolor{currentfill}{rgb}{0.333333,0.658824,0.407843}%
\pgfsetfillcolor{currentfill}%
\pgfsetlinewidth{0.752812pt}%
\definecolor{currentstroke}{rgb}{1.000000,1.000000,1.000000}%
\pgfsetstrokecolor{currentstroke}%
\pgfsetdash{}{0pt}%
\pgfsys@defobject{currentmarker}{\pgfqpoint{-0.104167in}{-0.104167in}}{\pgfqpoint{0.104167in}{0.104167in}}{%
\pgfpathmoveto{\pgfqpoint{-0.052083in}{-0.104167in}}%
\pgfpathlineto{\pgfqpoint{0.000000in}{-0.052083in}}%
\pgfpathlineto{\pgfqpoint{0.052083in}{-0.104167in}}%
\pgfpathlineto{\pgfqpoint{0.104167in}{-0.052083in}}%
\pgfpathlineto{\pgfqpoint{0.052083in}{0.000000in}}%
\pgfpathlineto{\pgfqpoint{0.104167in}{0.052083in}}%
\pgfpathlineto{\pgfqpoint{0.052083in}{0.104167in}}%
\pgfpathlineto{\pgfqpoint{0.000000in}{0.052083in}}%
\pgfpathlineto{\pgfqpoint{-0.052083in}{0.104167in}}%
\pgfpathlineto{\pgfqpoint{-0.104167in}{0.052083in}}%
\pgfpathlineto{\pgfqpoint{-0.052083in}{0.000000in}}%
\pgfpathlineto{\pgfqpoint{-0.104167in}{-0.052083in}}%
\pgfpathclose%
\pgfusepath{stroke,fill}%
}%
\begin{pgfscope}%
\pgfsys@transformshift{1.589202in}{1.510416in}%
\pgfsys@useobject{currentmarker}{}%
\end{pgfscope}%
\begin{pgfscope}%
\pgfsys@transformshift{2.394397in}{1.594156in}%
\pgfsys@useobject{currentmarker}{}%
\end{pgfscope}%
\begin{pgfscope}%
\pgfsys@transformshift{3.199592in}{1.605531in}%
\pgfsys@useobject{currentmarker}{}%
\end{pgfscope}%
\begin{pgfscope}%
\pgfsys@transformshift{4.004786in}{1.605531in}%
\pgfsys@useobject{currentmarker}{}%
\end{pgfscope}%
\begin{pgfscope}%
\pgfsys@transformshift{4.809981in}{1.605531in}%
\pgfsys@useobject{currentmarker}{}%
\end{pgfscope}%
\begin{pgfscope}%
\pgfsys@transformshift{5.615176in}{1.567572in}%
\pgfsys@useobject{currentmarker}{}%
\end{pgfscope}%
\begin{pgfscope}%
\pgfsys@transformshift{6.420371in}{1.625569in}%
\pgfsys@useobject{currentmarker}{}%
\end{pgfscope}%
\begin{pgfscope}%
\pgfsys@transformshift{7.225566in}{1.647065in}%
\pgfsys@useobject{currentmarker}{}%
\end{pgfscope}%
\end{pgfscope}%
\begin{pgfscope}%
\pgfpathrectangle{\pgfqpoint{1.307384in}{1.215049in}}{\pgfqpoint{6.200000in}{3.020000in}} %
\pgfusepath{clip}%
\pgfsetbuttcap%
\pgfsetroundjoin%
\pgfsetlinewidth{8.030000pt}%
\definecolor{currentstroke}{rgb}{0.768627,0.305882,0.321569}%
\pgfsetstrokecolor{currentstroke}%
\pgfsetdash{{8.000000pt}{8.000000pt}}{0.000000pt}%
\pgfpathmoveto{\pgfqpoint{1.589202in}{1.528456in}}%
\pgfpathlineto{\pgfqpoint{2.394397in}{2.055351in}}%
\pgfpathlineto{\pgfqpoint{3.199592in}{3.845298in}}%
\pgfpathlineto{\pgfqpoint{4.004786in}{3.845319in}}%
\pgfpathlineto{\pgfqpoint{4.809981in}{3.967372in}}%
\pgfpathlineto{\pgfqpoint{5.615176in}{3.967441in}}%
\pgfpathlineto{\pgfqpoint{6.420371in}{3.967441in}}%
\pgfpathlineto{\pgfqpoint{7.225566in}{3.977407in}}%
\pgfusepath{stroke}%
\end{pgfscope}%
\begin{pgfscope}%
\pgfpathrectangle{\pgfqpoint{1.307384in}{1.215049in}}{\pgfqpoint{6.200000in}{3.020000in}} %
\pgfusepath{clip}%
\pgfsetbuttcap%
\pgfsetmiterjoin%
\definecolor{currentfill}{rgb}{0.768627,0.305882,0.321569}%
\pgfsetfillcolor{currentfill}%
\pgfsetlinewidth{0.752812pt}%
\definecolor{currentstroke}{rgb}{1.000000,1.000000,1.000000}%
\pgfsetstrokecolor{currentstroke}%
\pgfsetdash{}{0pt}%
\pgfsys@defobject{currentmarker}{\pgfqpoint{-0.104167in}{-0.104167in}}{\pgfqpoint{0.104167in}{0.104167in}}{%
\pgfpathmoveto{\pgfqpoint{-0.000000in}{-0.104167in}}%
\pgfpathlineto{\pgfqpoint{0.104167in}{0.104167in}}%
\pgfpathlineto{\pgfqpoint{-0.104167in}{0.104167in}}%
\pgfpathclose%
\pgfusepath{stroke,fill}%
}%
\begin{pgfscope}%
\pgfsys@transformshift{1.589202in}{1.528456in}%
\pgfsys@useobject{currentmarker}{}%
\end{pgfscope}%
\begin{pgfscope}%
\pgfsys@transformshift{2.394397in}{2.055351in}%
\pgfsys@useobject{currentmarker}{}%
\end{pgfscope}%
\begin{pgfscope}%
\pgfsys@transformshift{3.199592in}{3.845298in}%
\pgfsys@useobject{currentmarker}{}%
\end{pgfscope}%
\begin{pgfscope}%
\pgfsys@transformshift{4.004786in}{3.845319in}%
\pgfsys@useobject{currentmarker}{}%
\end{pgfscope}%
\begin{pgfscope}%
\pgfsys@transformshift{4.809981in}{3.967372in}%
\pgfsys@useobject{currentmarker}{}%
\end{pgfscope}%
\begin{pgfscope}%
\pgfsys@transformshift{5.615176in}{3.967441in}%
\pgfsys@useobject{currentmarker}{}%
\end{pgfscope}%
\begin{pgfscope}%
\pgfsys@transformshift{6.420371in}{3.967441in}%
\pgfsys@useobject{currentmarker}{}%
\end{pgfscope}%
\begin{pgfscope}%
\pgfsys@transformshift{7.225566in}{3.977407in}%
\pgfsys@useobject{currentmarker}{}%
\end{pgfscope}%
\end{pgfscope}%
\begin{pgfscope}%
\pgfsetrectcap%
\pgfsetmiterjoin%
\pgfsetlinewidth{1.254687pt}%
\definecolor{currentstroke}{rgb}{0.800000,0.800000,0.800000}%
\pgfsetstrokecolor{currentstroke}%
\pgfsetdash{}{0pt}%
\pgfpathmoveto{\pgfqpoint{1.307384in}{1.215049in}}%
\pgfpathlineto{\pgfqpoint{1.307384in}{4.235049in}}%
\pgfusepath{stroke}%
\end{pgfscope}%
\begin{pgfscope}%
\pgfsetrectcap%
\pgfsetmiterjoin%
\pgfsetlinewidth{1.254687pt}%
\definecolor{currentstroke}{rgb}{0.800000,0.800000,0.800000}%
\pgfsetstrokecolor{currentstroke}%
\pgfsetdash{}{0pt}%
\pgfpathmoveto{\pgfqpoint{7.507384in}{1.215049in}}%
\pgfpathlineto{\pgfqpoint{7.507384in}{4.235049in}}%
\pgfusepath{stroke}%
\end{pgfscope}%
\begin{pgfscope}%
\pgfsetrectcap%
\pgfsetmiterjoin%
\pgfsetlinewidth{1.254687pt}%
\definecolor{currentstroke}{rgb}{0.800000,0.800000,0.800000}%
\pgfsetstrokecolor{currentstroke}%
\pgfsetdash{}{0pt}%
\pgfpathmoveto{\pgfqpoint{1.307384in}{1.215049in}}%
\pgfpathlineto{\pgfqpoint{7.507384in}{1.215049in}}%
\pgfusepath{stroke}%
\end{pgfscope}%
\begin{pgfscope}%
\pgfsetrectcap%
\pgfsetmiterjoin%
\pgfsetlinewidth{1.254687pt}%
\definecolor{currentstroke}{rgb}{0.800000,0.800000,0.800000}%
\pgfsetstrokecolor{currentstroke}%
\pgfsetdash{}{0pt}%
\pgfpathmoveto{\pgfqpoint{1.307384in}{4.235049in}}%
\pgfpathlineto{\pgfqpoint{7.507384in}{4.235049in}}%
\pgfusepath{stroke}%
\end{pgfscope}%
\begin{pgfscope}%
\pgfsetbuttcap%
\pgfsetroundjoin%
\pgfsetlinewidth{6.022500pt}%
\definecolor{currentstroke}{rgb}{0.298039,0.447059,0.690196}%
\pgfsetstrokecolor{currentstroke}%
\pgfsetdash{{6.000000pt}{0.000000pt}}{0.000000pt}%
\pgfpathmoveto{\pgfqpoint{1.129125in}{4.417860in}}%
\pgfpathlineto{\pgfqpoint{1.837458in}{4.417860in}}%
\pgfusepath{stroke}%
\end{pgfscope}%
\begin{pgfscope}%
\pgfsetbuttcap%
\pgfsetroundjoin%
\definecolor{currentfill}{rgb}{0.298039,0.447059,0.690196}%
\pgfsetfillcolor{currentfill}%
\pgfsetlinewidth{0.000000pt}%
\definecolor{currentstroke}{rgb}{0.298039,0.447059,0.690196}%
\pgfsetstrokecolor{currentstroke}%
\pgfsetdash{}{0pt}%
\pgfsys@defobject{currentmarker}{\pgfqpoint{-0.104167in}{-0.104167in}}{\pgfqpoint{0.104167in}{0.104167in}}{%
\pgfpathmoveto{\pgfqpoint{0.000000in}{-0.104167in}}%
\pgfpathcurveto{\pgfqpoint{0.027625in}{-0.104167in}}{\pgfqpoint{0.054123in}{-0.093191in}}{\pgfqpoint{0.073657in}{-0.073657in}}%
\pgfpathcurveto{\pgfqpoint{0.093191in}{-0.054123in}}{\pgfqpoint{0.104167in}{-0.027625in}}{\pgfqpoint{0.104167in}{0.000000in}}%
\pgfpathcurveto{\pgfqpoint{0.104167in}{0.027625in}}{\pgfqpoint{0.093191in}{0.054123in}}{\pgfqpoint{0.073657in}{0.073657in}}%
\pgfpathcurveto{\pgfqpoint{0.054123in}{0.093191in}}{\pgfqpoint{0.027625in}{0.104167in}}{\pgfqpoint{0.000000in}{0.104167in}}%
\pgfpathcurveto{\pgfqpoint{-0.027625in}{0.104167in}}{\pgfqpoint{-0.054123in}{0.093191in}}{\pgfqpoint{-0.073657in}{0.073657in}}%
\pgfpathcurveto{\pgfqpoint{-0.093191in}{0.054123in}}{\pgfqpoint{-0.104167in}{0.027625in}}{\pgfqpoint{-0.104167in}{0.000000in}}%
\pgfpathcurveto{\pgfqpoint{-0.104167in}{-0.027625in}}{\pgfqpoint{-0.093191in}{-0.054123in}}{\pgfqpoint{-0.073657in}{-0.073657in}}%
\pgfpathcurveto{\pgfqpoint{-0.054123in}{-0.093191in}}{\pgfqpoint{-0.027625in}{-0.104167in}}{\pgfqpoint{0.000000in}{-0.104167in}}%
\pgfpathclose%
\pgfusepath{fill}%
}%
\begin{pgfscope}%
\pgfsys@transformshift{1.483291in}{4.417860in}%
\pgfsys@useobject{currentmarker}{}%
\end{pgfscope}%
\end{pgfscope}%
\begin{pgfscope}%
\definecolor{textcolor}{rgb}{0.150000,0.150000,0.150000}%
\pgfsetstrokecolor{textcolor}%
\pgfsetfillcolor{textcolor}%
\pgftext[x=1.884680in,y=4.252583in,left,base]{\color{textcolor}\rmfamily\fontsize{34.000000}{40.800000}\selectfont SA}%
\end{pgfscope}%
\begin{pgfscope}%
\pgfsetbuttcap%
\pgfsetroundjoin%
\pgfsetlinewidth{6.022500pt}%
\definecolor{currentstroke}{rgb}{0.866667,0.517647,0.321569}%
\pgfsetstrokecolor{currentstroke}%
\pgfsetdash{{18.000000pt}{6.000000pt}}{0.000000pt}%
\pgfpathmoveto{\pgfqpoint{2.702113in}{4.417860in}}%
\pgfpathlineto{\pgfqpoint{3.410446in}{4.417860in}}%
\pgfusepath{stroke}%
\end{pgfscope}%
\begin{pgfscope}%
\pgfsetbuttcap%
\pgfsetmiterjoin%
\definecolor{currentfill}{rgb}{0.866667,0.517647,0.321569}%
\pgfsetfillcolor{currentfill}%
\pgfsetlinewidth{0.000000pt}%
\definecolor{currentstroke}{rgb}{0.866667,0.517647,0.321569}%
\pgfsetstrokecolor{currentstroke}%
\pgfsetdash{}{0pt}%
\pgfsys@defobject{currentmarker}{\pgfqpoint{-0.104167in}{-0.104167in}}{\pgfqpoint{0.104167in}{0.104167in}}{%
\pgfpathmoveto{\pgfqpoint{0.000000in}{0.104167in}}%
\pgfpathlineto{\pgfqpoint{-0.104167in}{-0.104167in}}%
\pgfpathlineto{\pgfqpoint{0.104167in}{-0.104167in}}%
\pgfpathclose%
\pgfusepath{fill}%
}%
\begin{pgfscope}%
\pgfsys@transformshift{3.056279in}{4.417860in}%
\pgfsys@useobject{currentmarker}{}%
\end{pgfscope}%
\end{pgfscope}%
\begin{pgfscope}%
\definecolor{textcolor}{rgb}{0.150000,0.150000,0.150000}%
\pgfsetstrokecolor{textcolor}%
\pgfsetfillcolor{textcolor}%
\pgftext[x=3.457668in,y=4.252583in,left,base]{\color{textcolor}\rmfamily\fontsize{34.000000}{40.800000}\selectfont HM}%
\end{pgfscope}%
\begin{pgfscope}%
\pgfsetbuttcap%
\pgfsetroundjoin%
\pgfsetlinewidth{6.022500pt}%
\definecolor{currentstroke}{rgb}{0.333333,0.658824,0.407843}%
\pgfsetstrokecolor{currentstroke}%
\pgfsetdash{{12.000000pt}{12.000000pt}}{0.000000pt}%
\pgfpathmoveto{\pgfqpoint{4.435854in}{4.417860in}}%
\pgfpathlineto{\pgfqpoint{5.144188in}{4.417860in}}%
\pgfusepath{stroke}%
\end{pgfscope}%
\begin{pgfscope}%
\pgfsetbuttcap%
\pgfsetmiterjoin%
\definecolor{currentfill}{rgb}{0.333333,0.658824,0.407843}%
\pgfsetfillcolor{currentfill}%
\pgfsetlinewidth{0.000000pt}%
\definecolor{currentstroke}{rgb}{0.333333,0.658824,0.407843}%
\pgfsetstrokecolor{currentstroke}%
\pgfsetdash{}{0pt}%
\pgfsys@defobject{currentmarker}{\pgfqpoint{-0.104167in}{-0.104167in}}{\pgfqpoint{0.104167in}{0.104167in}}{%
\pgfpathmoveto{\pgfqpoint{-0.052083in}{-0.104167in}}%
\pgfpathlineto{\pgfqpoint{0.000000in}{-0.052083in}}%
\pgfpathlineto{\pgfqpoint{0.052083in}{-0.104167in}}%
\pgfpathlineto{\pgfqpoint{0.104167in}{-0.052083in}}%
\pgfpathlineto{\pgfqpoint{0.052083in}{0.000000in}}%
\pgfpathlineto{\pgfqpoint{0.104167in}{0.052083in}}%
\pgfpathlineto{\pgfqpoint{0.052083in}{0.104167in}}%
\pgfpathlineto{\pgfqpoint{0.000000in}{0.052083in}}%
\pgfpathlineto{\pgfqpoint{-0.052083in}{0.104167in}}%
\pgfpathlineto{\pgfqpoint{-0.104167in}{0.052083in}}%
\pgfpathlineto{\pgfqpoint{-0.052083in}{0.000000in}}%
\pgfpathlineto{\pgfqpoint{-0.104167in}{-0.052083in}}%
\pgfpathclose%
\pgfusepath{fill}%
}%
\begin{pgfscope}%
\pgfsys@transformshift{4.790021in}{4.417860in}%
\pgfsys@useobject{currentmarker}{}%
\end{pgfscope}%
\end{pgfscope}%
\begin{pgfscope}%
\definecolor{textcolor}{rgb}{0.150000,0.150000,0.150000}%
\pgfsetstrokecolor{textcolor}%
\pgfsetfillcolor{textcolor}%
\pgftext[x=5.191410in,y=4.252583in,left,base]{\color{textcolor}\rmfamily\fontsize{34.000000}{40.800000}\selectfont HL}%
\end{pgfscope}%
\begin{pgfscope}%
\pgfsetbuttcap%
\pgfsetroundjoin%
\pgfsetlinewidth{6.022500pt}%
\definecolor{currentstroke}{rgb}{0.768627,0.305882,0.321569}%
\pgfsetstrokecolor{currentstroke}%
\pgfsetdash{{6.000000pt}{6.000000pt}}{0.000000pt}%
\pgfpathmoveto{\pgfqpoint{6.039668in}{4.417860in}}%
\pgfpathlineto{\pgfqpoint{6.748001in}{4.417860in}}%
\pgfusepath{stroke}%
\end{pgfscope}%
\begin{pgfscope}%
\pgfsetbuttcap%
\pgfsetmiterjoin%
\definecolor{currentfill}{rgb}{0.768627,0.305882,0.321569}%
\pgfsetfillcolor{currentfill}%
\pgfsetlinewidth{0.000000pt}%
\definecolor{currentstroke}{rgb}{0.768627,0.305882,0.321569}%
\pgfsetstrokecolor{currentstroke}%
\pgfsetdash{}{0pt}%
\pgfsys@defobject{currentmarker}{\pgfqpoint{-0.104167in}{-0.104167in}}{\pgfqpoint{0.104167in}{0.104167in}}{%
\pgfpathmoveto{\pgfqpoint{-0.000000in}{-0.104167in}}%
\pgfpathlineto{\pgfqpoint{0.104167in}{0.104167in}}%
\pgfpathlineto{\pgfqpoint{-0.104167in}{0.104167in}}%
\pgfpathclose%
\pgfusepath{fill}%
}%
\begin{pgfscope}%
\pgfsys@transformshift{6.393834in}{4.417860in}%
\pgfsys@useobject{currentmarker}{}%
\end{pgfscope}%
\end{pgfscope}%
\begin{pgfscope}%
\definecolor{textcolor}{rgb}{0.150000,0.150000,0.150000}%
\pgfsetstrokecolor{textcolor}%
\pgfsetfillcolor{textcolor}%
\pgftext[x=6.795223in,y=4.252583in,left,base]{\color{textcolor}\rmfamily\fontsize{34.000000}{40.800000}\selectfont ALL}%
\end{pgfscope}%
\end{pgfpicture}%
\makeatother%
\endgroup%
%
}
\caption{Budget = 16 GB}
\end{subfigure}
\begin{subfigure}[b]{0.5\linewidth}
\centering
 \resizebox{\columnwidth}{!}{%
%% Creator: Matplotlib, PGF backend
%%
%% To include the figure in your LaTeX document, write
%%   \input{<filename>.pgf}
%%
%% Make sure the required packages are loaded in your preamble
%%   \usepackage{pgf}
%%
%% Figures using additional raster images can only be included by \input if
%% they are in the same directory as the main LaTeX file. For loading figures
%% from other directories you can use the `import` package
%%   \usepackage{import}
%% and then include the figures with
%%   \import{<path to file>}{<filename>.pgf}
%%
%% Matplotlib used the following preamble
%%   \usepackage{fontspec}
%%   \setmonofont{Andale Mono}
%%
\begingroup%
\makeatletter%
\begin{pgfpicture}%
\pgfpathrectangle{\pgfpointorigin}{\pgfqpoint{5.576861in}{4.844250in}}%
\pgfusepath{use as bounding box, clip}%
\begin{pgfscope}%
\pgfsetbuttcap%
\pgfsetmiterjoin%
\definecolor{currentfill}{rgb}{1.000000,1.000000,1.000000}%
\pgfsetfillcolor{currentfill}%
\pgfsetlinewidth{0.000000pt}%
\definecolor{currentstroke}{rgb}{1.000000,1.000000,1.000000}%
\pgfsetstrokecolor{currentstroke}%
\pgfsetdash{}{0pt}%
\pgfpathmoveto{\pgfqpoint{0.000000in}{0.000000in}}%
\pgfpathlineto{\pgfqpoint{5.576861in}{0.000000in}}%
\pgfpathlineto{\pgfqpoint{5.576861in}{4.844250in}}%
\pgfpathlineto{\pgfqpoint{0.000000in}{4.844250in}}%
\pgfpathclose%
\pgfusepath{fill}%
\end{pgfscope}%
\begin{pgfscope}%
\pgfsetbuttcap%
\pgfsetmiterjoin%
\definecolor{currentfill}{rgb}{1.000000,1.000000,1.000000}%
\pgfsetfillcolor{currentfill}%
\pgfsetlinewidth{0.000000pt}%
\definecolor{currentstroke}{rgb}{0.000000,0.000000,0.000000}%
\pgfsetstrokecolor{currentstroke}%
\pgfsetstrokeopacity{0.000000}%
\pgfsetdash{}{0pt}%
\pgfpathmoveto{\pgfqpoint{1.535194in}{1.233139in}}%
\pgfpathlineto{\pgfqpoint{5.410194in}{1.233139in}}%
\pgfpathlineto{\pgfqpoint{5.410194in}{4.253139in}}%
\pgfpathlineto{\pgfqpoint{1.535194in}{4.253139in}}%
\pgfpathclose%
\pgfusepath{fill}%
\end{pgfscope}%
\begin{pgfscope}%
\pgfpathrectangle{\pgfqpoint{1.535194in}{1.233139in}}{\pgfqpoint{3.875000in}{3.020000in}} %
\pgfusepath{clip}%
\pgfsetroundcap%
\pgfsetroundjoin%
\pgfsetlinewidth{0.803000pt}%
\definecolor{currentstroke}{rgb}{0.800000,0.800000,0.800000}%
\pgfsetstrokecolor{currentstroke}%
\pgfsetdash{}{0pt}%
\pgfpathmoveto{\pgfqpoint{1.711331in}{1.233139in}}%
\pgfpathlineto{\pgfqpoint{1.711331in}{4.253139in}}%
\pgfusepath{stroke}%
\end{pgfscope}%
\begin{pgfscope}%
\definecolor{textcolor}{rgb}{0.150000,0.150000,0.150000}%
\pgfsetstrokecolor{textcolor}%
\pgfsetfillcolor{textcolor}%
\pgftext[x=1.711331in,y=1.069250in,,top]{\color{textcolor}\rmfamily\fontsize{36.000000}{43.200000}\selectfont 1}%
\end{pgfscope}%
\begin{pgfscope}%
\pgfpathrectangle{\pgfqpoint{1.535194in}{1.233139in}}{\pgfqpoint{3.875000in}{3.020000in}} %
\pgfusepath{clip}%
\pgfsetroundcap%
\pgfsetroundjoin%
\pgfsetlinewidth{0.803000pt}%
\definecolor{currentstroke}{rgb}{0.800000,0.800000,0.800000}%
\pgfsetstrokecolor{currentstroke}%
\pgfsetdash{}{0pt}%
\pgfpathmoveto{\pgfqpoint{2.592012in}{1.233139in}}%
\pgfpathlineto{\pgfqpoint{2.592012in}{4.253139in}}%
\pgfusepath{stroke}%
\end{pgfscope}%
\begin{pgfscope}%
\definecolor{textcolor}{rgb}{0.150000,0.150000,0.150000}%
\pgfsetstrokecolor{textcolor}%
\pgfsetfillcolor{textcolor}%
\pgftext[x=2.592012in,y=1.069250in,,top]{\color{textcolor}\rmfamily\fontsize{36.000000}{43.200000}\selectfont 2}%
\end{pgfscope}%
\begin{pgfscope}%
\pgfpathrectangle{\pgfqpoint{1.535194in}{1.233139in}}{\pgfqpoint{3.875000in}{3.020000in}} %
\pgfusepath{clip}%
\pgfsetroundcap%
\pgfsetroundjoin%
\pgfsetlinewidth{0.803000pt}%
\definecolor{currentstroke}{rgb}{0.800000,0.800000,0.800000}%
\pgfsetstrokecolor{currentstroke}%
\pgfsetdash{}{0pt}%
\pgfpathmoveto{\pgfqpoint{3.472694in}{1.233139in}}%
\pgfpathlineto{\pgfqpoint{3.472694in}{4.253139in}}%
\pgfusepath{stroke}%
\end{pgfscope}%
\begin{pgfscope}%
\definecolor{textcolor}{rgb}{0.150000,0.150000,0.150000}%
\pgfsetstrokecolor{textcolor}%
\pgfsetfillcolor{textcolor}%
\pgftext[x=3.472694in,y=1.069250in,,top]{\color{textcolor}\rmfamily\fontsize{36.000000}{43.200000}\selectfont 3}%
\end{pgfscope}%
\begin{pgfscope}%
\pgfpathrectangle{\pgfqpoint{1.535194in}{1.233139in}}{\pgfqpoint{3.875000in}{3.020000in}} %
\pgfusepath{clip}%
\pgfsetroundcap%
\pgfsetroundjoin%
\pgfsetlinewidth{0.803000pt}%
\definecolor{currentstroke}{rgb}{0.800000,0.800000,0.800000}%
\pgfsetstrokecolor{currentstroke}%
\pgfsetdash{}{0pt}%
\pgfpathmoveto{\pgfqpoint{4.353376in}{1.233139in}}%
\pgfpathlineto{\pgfqpoint{4.353376in}{4.253139in}}%
\pgfusepath{stroke}%
\end{pgfscope}%
\begin{pgfscope}%
\definecolor{textcolor}{rgb}{0.150000,0.150000,0.150000}%
\pgfsetstrokecolor{textcolor}%
\pgfsetfillcolor{textcolor}%
\pgftext[x=4.353376in,y=1.069250in,,top]{\color{textcolor}\rmfamily\fontsize{36.000000}{43.200000}\selectfont 4}%
\end{pgfscope}%
\begin{pgfscope}%
\pgfpathrectangle{\pgfqpoint{1.535194in}{1.233139in}}{\pgfqpoint{3.875000in}{3.020000in}} %
\pgfusepath{clip}%
\pgfsetroundcap%
\pgfsetroundjoin%
\pgfsetlinewidth{0.803000pt}%
\definecolor{currentstroke}{rgb}{0.800000,0.800000,0.800000}%
\pgfsetstrokecolor{currentstroke}%
\pgfsetdash{}{0pt}%
\pgfpathmoveto{\pgfqpoint{5.234058in}{1.233139in}}%
\pgfpathlineto{\pgfqpoint{5.234058in}{4.253139in}}%
\pgfusepath{stroke}%
\end{pgfscope}%
\begin{pgfscope}%
\definecolor{textcolor}{rgb}{0.150000,0.150000,0.150000}%
\pgfsetstrokecolor{textcolor}%
\pgfsetfillcolor{textcolor}%
\pgftext[x=5.234058in,y=1.069250in,,top]{\color{textcolor}\rmfamily\fontsize{36.000000}{43.200000}\selectfont 5}%
\end{pgfscope}%
\begin{pgfscope}%
\definecolor{textcolor}{rgb}{0.150000,0.150000,0.150000}%
\pgfsetstrokecolor{textcolor}%
\pgfsetfillcolor{textcolor}%
\pgftext[x=3.472694in,y=0.569194in,,top]{\color{textcolor}\rmfamily\fontsize{38.000000}{45.600000}\selectfont Workload}%
\end{pgfscope}%
\begin{pgfscope}%
\pgfpathrectangle{\pgfqpoint{1.535194in}{1.233139in}}{\pgfqpoint{3.875000in}{3.020000in}} %
\pgfusepath{clip}%
\pgfsetroundcap%
\pgfsetroundjoin%
\pgfsetlinewidth{0.803000pt}%
\definecolor{currentstroke}{rgb}{0.800000,0.800000,0.800000}%
\pgfsetstrokecolor{currentstroke}%
\pgfsetdash{}{0pt}%
\pgfpathmoveto{\pgfqpoint{1.535194in}{1.233139in}}%
\pgfpathlineto{\pgfqpoint{5.410194in}{1.233139in}}%
\pgfusepath{stroke}%
\end{pgfscope}%
\begin{pgfscope}%
\definecolor{textcolor}{rgb}{0.150000,0.150000,0.150000}%
\pgfsetstrokecolor{textcolor}%
\pgfsetfillcolor{textcolor}%
\pgftext[x=1.141805in,y=1.059639in,left,base]{\color{textcolor}\rmfamily\fontsize{36.000000}{43.200000}\selectfont 0}%
\end{pgfscope}%
\begin{pgfscope}%
\pgfpathrectangle{\pgfqpoint{1.535194in}{1.233139in}}{\pgfqpoint{3.875000in}{3.020000in}} %
\pgfusepath{clip}%
\pgfsetroundcap%
\pgfsetroundjoin%
\pgfsetlinewidth{0.803000pt}%
\definecolor{currentstroke}{rgb}{0.800000,0.800000,0.800000}%
\pgfsetstrokecolor{currentstroke}%
\pgfsetdash{}{0pt}%
\pgfpathmoveto{\pgfqpoint{1.535194in}{2.311710in}}%
\pgfpathlineto{\pgfqpoint{5.410194in}{2.311710in}}%
\pgfusepath{stroke}%
\end{pgfscope}%
\begin{pgfscope}%
\definecolor{textcolor}{rgb}{0.150000,0.150000,0.150000}%
\pgfsetstrokecolor{textcolor}%
\pgfsetfillcolor{textcolor}%
\pgftext[x=0.912305in,y=2.138210in,left,base]{\color{textcolor}\rmfamily\fontsize{36.000000}{43.200000}\selectfont 50}%
\end{pgfscope}%
\begin{pgfscope}%
\pgfpathrectangle{\pgfqpoint{1.535194in}{1.233139in}}{\pgfqpoint{3.875000in}{3.020000in}} %
\pgfusepath{clip}%
\pgfsetroundcap%
\pgfsetroundjoin%
\pgfsetlinewidth{0.803000pt}%
\definecolor{currentstroke}{rgb}{0.800000,0.800000,0.800000}%
\pgfsetstrokecolor{currentstroke}%
\pgfsetdash{}{0pt}%
\pgfpathmoveto{\pgfqpoint{1.535194in}{3.390281in}}%
\pgfpathlineto{\pgfqpoint{5.410194in}{3.390281in}}%
\pgfusepath{stroke}%
\end{pgfscope}%
\begin{pgfscope}%
\definecolor{textcolor}{rgb}{0.150000,0.150000,0.150000}%
\pgfsetstrokecolor{textcolor}%
\pgfsetfillcolor{textcolor}%
\pgftext[x=0.682805in,y=3.216781in,left,base]{\color{textcolor}\rmfamily\fontsize{36.000000}{43.200000}\selectfont 100}%
\end{pgfscope}%
\begin{pgfscope}%
\definecolor{textcolor}{rgb}{0.150000,0.150000,0.150000}%
\pgfsetstrokecolor{textcolor}%
\pgfsetfillcolor{textcolor}%
\pgftext[x=0.627250in,y=2.743139in,,bottom,rotate=90.000000]{\color{textcolor}\rmfamily\fontsize{38.000000}{45.600000}\selectfont Size (GB)}%
\end{pgfscope}%
\begin{pgfscope}%
\pgfpathrectangle{\pgfqpoint{1.535194in}{1.233139in}}{\pgfqpoint{3.875000in}{3.020000in}} %
\pgfusepath{clip}%
\pgfsetbuttcap%
\pgfsetroundjoin%
\pgfsetlinewidth{3.011250pt}%
\definecolor{currentstroke}{rgb}{0.866667,0.517647,0.321569}%
\pgfsetstrokecolor{currentstroke}%
\pgfsetdash{{3.000000pt}{0.000000pt}}{0.000000pt}%
\pgfpathmoveto{\pgfqpoint{1.711331in}{1.529100in}}%
\pgfpathlineto{\pgfqpoint{2.592012in}{1.923359in}}%
\pgfpathlineto{\pgfqpoint{3.472694in}{1.923209in}}%
\pgfpathlineto{\pgfqpoint{4.353376in}{1.923366in}}%
\pgfpathlineto{\pgfqpoint{5.234058in}{1.923405in}}%
\pgfusepath{stroke}%
\end{pgfscope}%
\begin{pgfscope}%
\pgfpathrectangle{\pgfqpoint{1.535194in}{1.233139in}}{\pgfqpoint{3.875000in}{3.020000in}} %
\pgfusepath{clip}%
\pgfsetbuttcap%
\pgfsetroundjoin%
\definecolor{currentfill}{rgb}{0.866667,0.517647,0.321569}%
\pgfsetfillcolor{currentfill}%
\pgfsetlinewidth{0.752812pt}%
\definecolor{currentstroke}{rgb}{1.000000,1.000000,1.000000}%
\pgfsetstrokecolor{currentstroke}%
\pgfsetdash{}{0pt}%
\pgfsys@defobject{currentmarker}{\pgfqpoint{-0.104167in}{-0.104167in}}{\pgfqpoint{0.104167in}{0.104167in}}{%
\pgfpathmoveto{\pgfqpoint{0.000000in}{-0.104167in}}%
\pgfpathcurveto{\pgfqpoint{0.027625in}{-0.104167in}}{\pgfqpoint{0.054123in}{-0.093191in}}{\pgfqpoint{0.073657in}{-0.073657in}}%
\pgfpathcurveto{\pgfqpoint{0.093191in}{-0.054123in}}{\pgfqpoint{0.104167in}{-0.027625in}}{\pgfqpoint{0.104167in}{0.000000in}}%
\pgfpathcurveto{\pgfqpoint{0.104167in}{0.027625in}}{\pgfqpoint{0.093191in}{0.054123in}}{\pgfqpoint{0.073657in}{0.073657in}}%
\pgfpathcurveto{\pgfqpoint{0.054123in}{0.093191in}}{\pgfqpoint{0.027625in}{0.104167in}}{\pgfqpoint{0.000000in}{0.104167in}}%
\pgfpathcurveto{\pgfqpoint{-0.027625in}{0.104167in}}{\pgfqpoint{-0.054123in}{0.093191in}}{\pgfqpoint{-0.073657in}{0.073657in}}%
\pgfpathcurveto{\pgfqpoint{-0.093191in}{0.054123in}}{\pgfqpoint{-0.104167in}{0.027625in}}{\pgfqpoint{-0.104167in}{0.000000in}}%
\pgfpathcurveto{\pgfqpoint{-0.104167in}{-0.027625in}}{\pgfqpoint{-0.093191in}{-0.054123in}}{\pgfqpoint{-0.073657in}{-0.073657in}}%
\pgfpathcurveto{\pgfqpoint{-0.054123in}{-0.093191in}}{\pgfqpoint{-0.027625in}{-0.104167in}}{\pgfqpoint{0.000000in}{-0.104167in}}%
\pgfpathclose%
\pgfusepath{stroke,fill}%
}%
\begin{pgfscope}%
\pgfsys@transformshift{1.711331in}{1.529100in}%
\pgfsys@useobject{currentmarker}{}%
\end{pgfscope}%
\begin{pgfscope}%
\pgfsys@transformshift{2.592012in}{1.923359in}%
\pgfsys@useobject{currentmarker}{}%
\end{pgfscope}%
\begin{pgfscope}%
\pgfsys@transformshift{3.472694in}{1.923209in}%
\pgfsys@useobject{currentmarker}{}%
\end{pgfscope}%
\begin{pgfscope}%
\pgfsys@transformshift{4.353376in}{1.923366in}%
\pgfsys@useobject{currentmarker}{}%
\end{pgfscope}%
\begin{pgfscope}%
\pgfsys@transformshift{5.234058in}{1.923405in}%
\pgfsys@useobject{currentmarker}{}%
\end{pgfscope}%
\end{pgfscope}%
\begin{pgfscope}%
\pgfpathrectangle{\pgfqpoint{1.535194in}{1.233139in}}{\pgfqpoint{3.875000in}{3.020000in}} %
\pgfusepath{clip}%
\pgfsetbuttcap%
\pgfsetroundjoin%
\pgfsetlinewidth{3.011250pt}%
\definecolor{currentstroke}{rgb}{0.298039,0.447059,0.690196}%
\pgfsetstrokecolor{currentstroke}%
\pgfsetdash{{6.000000pt}{6.000000pt}}{0.000000pt}%
\pgfpathmoveto{\pgfqpoint{1.711331in}{1.529100in}}%
\pgfpathlineto{\pgfqpoint{2.592012in}{2.055994in}}%
\pgfpathlineto{\pgfqpoint{3.472694in}{3.845944in}}%
\pgfpathlineto{\pgfqpoint{4.353376in}{3.850800in}}%
\pgfpathlineto{\pgfqpoint{5.234058in}{3.979227in}}%
\pgfusepath{stroke}%
\end{pgfscope}%
\begin{pgfscope}%
\pgfpathrectangle{\pgfqpoint{1.535194in}{1.233139in}}{\pgfqpoint{3.875000in}{3.020000in}} %
\pgfusepath{clip}%
\pgfsetbuttcap%
\pgfsetmiterjoin%
\definecolor{currentfill}{rgb}{0.298039,0.447059,0.690196}%
\pgfsetfillcolor{currentfill}%
\pgfsetlinewidth{0.752812pt}%
\definecolor{currentstroke}{rgb}{1.000000,1.000000,1.000000}%
\pgfsetstrokecolor{currentstroke}%
\pgfsetdash{}{0pt}%
\pgfsys@defobject{currentmarker}{\pgfqpoint{-0.104167in}{-0.104167in}}{\pgfqpoint{0.104167in}{0.104167in}}{%
\pgfpathmoveto{\pgfqpoint{0.000000in}{0.104167in}}%
\pgfpathlineto{\pgfqpoint{-0.104167in}{-0.104167in}}%
\pgfpathlineto{\pgfqpoint{0.104167in}{-0.104167in}}%
\pgfpathclose%
\pgfusepath{stroke,fill}%
}%
\begin{pgfscope}%
\pgfsys@transformshift{1.711331in}{1.529100in}%
\pgfsys@useobject{currentmarker}{}%
\end{pgfscope}%
\begin{pgfscope}%
\pgfsys@transformshift{2.592012in}{2.055994in}%
\pgfsys@useobject{currentmarker}{}%
\end{pgfscope}%
\begin{pgfscope}%
\pgfsys@transformshift{3.472694in}{3.845944in}%
\pgfsys@useobject{currentmarker}{}%
\end{pgfscope}%
\begin{pgfscope}%
\pgfsys@transformshift{4.353376in}{3.850800in}%
\pgfsys@useobject{currentmarker}{}%
\end{pgfscope}%
\begin{pgfscope}%
\pgfsys@transformshift{5.234058in}{3.979227in}%
\pgfsys@useobject{currentmarker}{}%
\end{pgfscope}%
\end{pgfscope}%
\begin{pgfscope}%
\pgfsetrectcap%
\pgfsetmiterjoin%
\pgfsetlinewidth{1.003750pt}%
\definecolor{currentstroke}{rgb}{0.800000,0.800000,0.800000}%
\pgfsetstrokecolor{currentstroke}%
\pgfsetdash{}{0pt}%
\pgfpathmoveto{\pgfqpoint{1.535194in}{1.233139in}}%
\pgfpathlineto{\pgfqpoint{1.535194in}{4.253139in}}%
\pgfusepath{stroke}%
\end{pgfscope}%
\begin{pgfscope}%
\pgfsetrectcap%
\pgfsetmiterjoin%
\pgfsetlinewidth{1.003750pt}%
\definecolor{currentstroke}{rgb}{0.800000,0.800000,0.800000}%
\pgfsetstrokecolor{currentstroke}%
\pgfsetdash{}{0pt}%
\pgfpathmoveto{\pgfqpoint{5.410194in}{1.233139in}}%
\pgfpathlineto{\pgfqpoint{5.410194in}{4.253139in}}%
\pgfusepath{stroke}%
\end{pgfscope}%
\begin{pgfscope}%
\pgfsetrectcap%
\pgfsetmiterjoin%
\pgfsetlinewidth{1.003750pt}%
\definecolor{currentstroke}{rgb}{0.800000,0.800000,0.800000}%
\pgfsetstrokecolor{currentstroke}%
\pgfsetdash{}{0pt}%
\pgfpathmoveto{\pgfqpoint{1.535194in}{1.233139in}}%
\pgfpathlineto{\pgfqpoint{5.410194in}{1.233139in}}%
\pgfusepath{stroke}%
\end{pgfscope}%
\begin{pgfscope}%
\pgfsetrectcap%
\pgfsetmiterjoin%
\pgfsetlinewidth{1.003750pt}%
\definecolor{currentstroke}{rgb}{0.800000,0.800000,0.800000}%
\pgfsetstrokecolor{currentstroke}%
\pgfsetdash{}{0pt}%
\pgfpathmoveto{\pgfqpoint{1.535194in}{4.253139in}}%
\pgfpathlineto{\pgfqpoint{5.410194in}{4.253139in}}%
\pgfusepath{stroke}%
\end{pgfscope}%
\begin{pgfscope}%
\pgfsetroundcap%
\pgfsetroundjoin%
\pgfsetlinewidth{3.011250pt}%
\definecolor{currentstroke}{rgb}{1.000000,1.000000,1.000000}%
\pgfsetstrokecolor{currentstroke}%
\pgfsetdash{}{0pt}%
\pgfpathmoveto{\pgfqpoint{1.700194in}{4.348416in}}%
\pgfpathlineto{\pgfqpoint{2.333527in}{4.348416in}}%
\pgfusepath{stroke}%
\end{pgfscope}%
\begin{pgfscope}%
\pgfsetbuttcap%
\pgfsetroundjoin%
\pgfsetlinewidth{3.011250pt}%
\definecolor{currentstroke}{rgb}{0.866667,0.517647,0.321569}%
\pgfsetstrokecolor{currentstroke}%
\pgfsetdash{{3.000000pt}{0.000000pt}}{0.000000pt}%
\pgfpathmoveto{\pgfqpoint{1.700194in}{3.928833in}}%
\pgfpathlineto{\pgfqpoint{2.333527in}{3.928833in}}%
\pgfusepath{stroke}%
\end{pgfscope}%
\begin{pgfscope}%
\pgfsetbuttcap%
\pgfsetroundjoin%
\definecolor{currentfill}{rgb}{0.866667,0.517647,0.321569}%
\pgfsetfillcolor{currentfill}%
\pgfsetlinewidth{0.000000pt}%
\definecolor{currentstroke}{rgb}{0.866667,0.517647,0.321569}%
\pgfsetstrokecolor{currentstroke}%
\pgfsetdash{}{0pt}%
\pgfsys@defobject{currentmarker}{\pgfqpoint{-0.104167in}{-0.104167in}}{\pgfqpoint{0.104167in}{0.104167in}}{%
\pgfpathmoveto{\pgfqpoint{0.000000in}{-0.104167in}}%
\pgfpathcurveto{\pgfqpoint{0.027625in}{-0.104167in}}{\pgfqpoint{0.054123in}{-0.093191in}}{\pgfqpoint{0.073657in}{-0.073657in}}%
\pgfpathcurveto{\pgfqpoint{0.093191in}{-0.054123in}}{\pgfqpoint{0.104167in}{-0.027625in}}{\pgfqpoint{0.104167in}{0.000000in}}%
\pgfpathcurveto{\pgfqpoint{0.104167in}{0.027625in}}{\pgfqpoint{0.093191in}{0.054123in}}{\pgfqpoint{0.073657in}{0.073657in}}%
\pgfpathcurveto{\pgfqpoint{0.054123in}{0.093191in}}{\pgfqpoint{0.027625in}{0.104167in}}{\pgfqpoint{0.000000in}{0.104167in}}%
\pgfpathcurveto{\pgfqpoint{-0.027625in}{0.104167in}}{\pgfqpoint{-0.054123in}{0.093191in}}{\pgfqpoint{-0.073657in}{0.073657in}}%
\pgfpathcurveto{\pgfqpoint{-0.093191in}{0.054123in}}{\pgfqpoint{-0.104167in}{0.027625in}}{\pgfqpoint{-0.104167in}{0.000000in}}%
\pgfpathcurveto{\pgfqpoint{-0.104167in}{-0.027625in}}{\pgfqpoint{-0.093191in}{-0.054123in}}{\pgfqpoint{-0.073657in}{-0.073657in}}%
\pgfpathcurveto{\pgfqpoint{-0.054123in}{-0.093191in}}{\pgfqpoint{-0.027625in}{-0.104167in}}{\pgfqpoint{0.000000in}{-0.104167in}}%
\pgfpathclose%
\pgfusepath{fill}%
}%
\begin{pgfscope}%
\pgfsys@transformshift{2.016861in}{3.928833in}%
\pgfsys@useobject{currentmarker}{}%
\end{pgfscope}%
\end{pgfscope}%
\begin{pgfscope}%
\definecolor{textcolor}{rgb}{0.150000,0.150000,0.150000}%
\pgfsetstrokecolor{textcolor}%
\pgfsetfillcolor{textcolor}%
\pgftext[x=2.386305in,y=3.744111in,left,base]{\color{textcolor}\rmfamily\fontsize{38.000000}{45.600000}\selectfont SM}%
\end{pgfscope}%
\begin{pgfscope}%
\pgfsetbuttcap%
\pgfsetroundjoin%
\pgfsetlinewidth{3.011250pt}%
\definecolor{currentstroke}{rgb}{0.298039,0.447059,0.690196}%
\pgfsetstrokecolor{currentstroke}%
\pgfsetdash{{6.000000pt}{6.000000pt}}{0.000000pt}%
\pgfpathmoveto{\pgfqpoint{1.700194in}{3.509250in}}%
\pgfpathlineto{\pgfqpoint{2.333527in}{3.509250in}}%
\pgfusepath{stroke}%
\end{pgfscope}%
\begin{pgfscope}%
\pgfsetbuttcap%
\pgfsetmiterjoin%
\definecolor{currentfill}{rgb}{0.298039,0.447059,0.690196}%
\pgfsetfillcolor{currentfill}%
\pgfsetlinewidth{0.000000pt}%
\definecolor{currentstroke}{rgb}{0.298039,0.447059,0.690196}%
\pgfsetstrokecolor{currentstroke}%
\pgfsetdash{}{0pt}%
\pgfsys@defobject{currentmarker}{\pgfqpoint{-0.104167in}{-0.104167in}}{\pgfqpoint{0.104167in}{0.104167in}}{%
\pgfpathmoveto{\pgfqpoint{0.000000in}{0.104167in}}%
\pgfpathlineto{\pgfqpoint{-0.104167in}{-0.104167in}}%
\pgfpathlineto{\pgfqpoint{0.104167in}{-0.104167in}}%
\pgfpathclose%
\pgfusepath{fill}%
}%
\begin{pgfscope}%
\pgfsys@transformshift{2.016861in}{3.509250in}%
\pgfsys@useobject{currentmarker}{}%
\end{pgfscope}%
\end{pgfscope}%
\begin{pgfscope}%
\definecolor{textcolor}{rgb}{0.150000,0.150000,0.150000}%
\pgfsetstrokecolor{textcolor}%
\pgfsetfillcolor{textcolor}%
\pgftext[x=2.386305in,y=3.324528in,left,base]{\color{textcolor}\rmfamily\fontsize{38.000000}{45.600000}\selectfont SA}%
\end{pgfscope}%
\end{pgfpicture}%
\makeatother%
\endgroup%
%
}
\caption{Budget = 32 GB}
\end{subfigure}%
\begin{subfigure}[b]{0.5\linewidth}
\centering
 \resizebox{\columnwidth}{!}{%
%% Creator: Matplotlib, PGF backend
%%
%% To include the figure in your LaTeX document, write
%%   \input{<filename>.pgf}
%%
%% Make sure the required packages are loaded in your preamble
%%   \usepackage{pgf}
%%
%% Figures using additional raster images can only be included by \input if
%% they are in the same directory as the main LaTeX file. For loading figures
%% from other directories you can use the `import` package
%%   \usepackage{import}
%% and then include the figures with
%%   \import{<path to file>}{<filename>.pgf}
%%
%% Matplotlib used the following preamble
%%   \usepackage{fontspec}
%%   \setmonofont{Andale Mono}
%%
\begingroup%
\makeatletter%
\begin{pgfpicture}%
\pgfpathrectangle{\pgfpointorigin}{\pgfqpoint{7.126861in}{4.844250in}}%
\pgfusepath{use as bounding box, clip}%
\begin{pgfscope}%
\pgfsetbuttcap%
\pgfsetmiterjoin%
\definecolor{currentfill}{rgb}{1.000000,1.000000,1.000000}%
\pgfsetfillcolor{currentfill}%
\pgfsetlinewidth{0.000000pt}%
\definecolor{currentstroke}{rgb}{1.000000,1.000000,1.000000}%
\pgfsetstrokecolor{currentstroke}%
\pgfsetdash{}{0pt}%
\pgfpathmoveto{\pgfqpoint{-0.000000in}{0.000000in}}%
\pgfpathlineto{\pgfqpoint{7.126861in}{0.000000in}}%
\pgfpathlineto{\pgfqpoint{7.126861in}{4.844250in}}%
\pgfpathlineto{\pgfqpoint{-0.000000in}{4.844250in}}%
\pgfpathclose%
\pgfusepath{fill}%
\end{pgfscope}%
\begin{pgfscope}%
\pgfsetbuttcap%
\pgfsetmiterjoin%
\definecolor{currentfill}{rgb}{1.000000,1.000000,1.000000}%
\pgfsetfillcolor{currentfill}%
\pgfsetlinewidth{0.000000pt}%
\definecolor{currentstroke}{rgb}{0.000000,0.000000,0.000000}%
\pgfsetstrokecolor{currentstroke}%
\pgfsetstrokeopacity{0.000000}%
\pgfsetdash{}{0pt}%
\pgfpathmoveto{\pgfqpoint{1.535194in}{1.233139in}}%
\pgfpathlineto{\pgfqpoint{6.960194in}{1.233139in}}%
\pgfpathlineto{\pgfqpoint{6.960194in}{4.253139in}}%
\pgfpathlineto{\pgfqpoint{1.535194in}{4.253139in}}%
\pgfpathclose%
\pgfusepath{fill}%
\end{pgfscope}%
\begin{pgfscope}%
\pgfpathrectangle{\pgfqpoint{1.535194in}{1.233139in}}{\pgfqpoint{5.425000in}{3.020000in}} %
\pgfusepath{clip}%
\pgfsetroundcap%
\pgfsetroundjoin%
\pgfsetlinewidth{0.803000pt}%
\definecolor{currentstroke}{rgb}{0.800000,0.800000,0.800000}%
\pgfsetstrokecolor{currentstroke}%
\pgfsetdash{}{0pt}%
\pgfpathmoveto{\pgfqpoint{1.781785in}{1.233139in}}%
\pgfpathlineto{\pgfqpoint{1.781785in}{4.253139in}}%
\pgfusepath{stroke}%
\end{pgfscope}%
\begin{pgfscope}%
\definecolor{textcolor}{rgb}{0.150000,0.150000,0.150000}%
\pgfsetstrokecolor{textcolor}%
\pgfsetfillcolor{textcolor}%
\pgftext[x=1.781785in,y=1.069250in,,top]{\color{textcolor}\rmfamily\fontsize{36.000000}{43.200000}\selectfont 1}%
\end{pgfscope}%
\begin{pgfscope}%
\pgfpathrectangle{\pgfqpoint{1.535194in}{1.233139in}}{\pgfqpoint{5.425000in}{3.020000in}} %
\pgfusepath{clip}%
\pgfsetroundcap%
\pgfsetroundjoin%
\pgfsetlinewidth{0.803000pt}%
\definecolor{currentstroke}{rgb}{0.800000,0.800000,0.800000}%
\pgfsetstrokecolor{currentstroke}%
\pgfsetdash{}{0pt}%
\pgfpathmoveto{\pgfqpoint{2.486331in}{1.233139in}}%
\pgfpathlineto{\pgfqpoint{2.486331in}{4.253139in}}%
\pgfusepath{stroke}%
\end{pgfscope}%
\begin{pgfscope}%
\definecolor{textcolor}{rgb}{0.150000,0.150000,0.150000}%
\pgfsetstrokecolor{textcolor}%
\pgfsetfillcolor{textcolor}%
\pgftext[x=2.486331in,y=1.069250in,,top]{\color{textcolor}\rmfamily\fontsize{36.000000}{43.200000}\selectfont 2}%
\end{pgfscope}%
\begin{pgfscope}%
\pgfpathrectangle{\pgfqpoint{1.535194in}{1.233139in}}{\pgfqpoint{5.425000in}{3.020000in}} %
\pgfusepath{clip}%
\pgfsetroundcap%
\pgfsetroundjoin%
\pgfsetlinewidth{0.803000pt}%
\definecolor{currentstroke}{rgb}{0.800000,0.800000,0.800000}%
\pgfsetstrokecolor{currentstroke}%
\pgfsetdash{}{0pt}%
\pgfpathmoveto{\pgfqpoint{3.190876in}{1.233139in}}%
\pgfpathlineto{\pgfqpoint{3.190876in}{4.253139in}}%
\pgfusepath{stroke}%
\end{pgfscope}%
\begin{pgfscope}%
\definecolor{textcolor}{rgb}{0.150000,0.150000,0.150000}%
\pgfsetstrokecolor{textcolor}%
\pgfsetfillcolor{textcolor}%
\pgftext[x=3.190876in,y=1.069250in,,top]{\color{textcolor}\rmfamily\fontsize{36.000000}{43.200000}\selectfont 3}%
\end{pgfscope}%
\begin{pgfscope}%
\pgfpathrectangle{\pgfqpoint{1.535194in}{1.233139in}}{\pgfqpoint{5.425000in}{3.020000in}} %
\pgfusepath{clip}%
\pgfsetroundcap%
\pgfsetroundjoin%
\pgfsetlinewidth{0.803000pt}%
\definecolor{currentstroke}{rgb}{0.800000,0.800000,0.800000}%
\pgfsetstrokecolor{currentstroke}%
\pgfsetdash{}{0pt}%
\pgfpathmoveto{\pgfqpoint{3.895421in}{1.233139in}}%
\pgfpathlineto{\pgfqpoint{3.895421in}{4.253139in}}%
\pgfusepath{stroke}%
\end{pgfscope}%
\begin{pgfscope}%
\definecolor{textcolor}{rgb}{0.150000,0.150000,0.150000}%
\pgfsetstrokecolor{textcolor}%
\pgfsetfillcolor{textcolor}%
\pgftext[x=3.895421in,y=1.069250in,,top]{\color{textcolor}\rmfamily\fontsize{36.000000}{43.200000}\selectfont 4}%
\end{pgfscope}%
\begin{pgfscope}%
\pgfpathrectangle{\pgfqpoint{1.535194in}{1.233139in}}{\pgfqpoint{5.425000in}{3.020000in}} %
\pgfusepath{clip}%
\pgfsetroundcap%
\pgfsetroundjoin%
\pgfsetlinewidth{0.803000pt}%
\definecolor{currentstroke}{rgb}{0.800000,0.800000,0.800000}%
\pgfsetstrokecolor{currentstroke}%
\pgfsetdash{}{0pt}%
\pgfpathmoveto{\pgfqpoint{4.599967in}{1.233139in}}%
\pgfpathlineto{\pgfqpoint{4.599967in}{4.253139in}}%
\pgfusepath{stroke}%
\end{pgfscope}%
\begin{pgfscope}%
\definecolor{textcolor}{rgb}{0.150000,0.150000,0.150000}%
\pgfsetstrokecolor{textcolor}%
\pgfsetfillcolor{textcolor}%
\pgftext[x=4.599967in,y=1.069250in,,top]{\color{textcolor}\rmfamily\fontsize{36.000000}{43.200000}\selectfont 5}%
\end{pgfscope}%
\begin{pgfscope}%
\pgfpathrectangle{\pgfqpoint{1.535194in}{1.233139in}}{\pgfqpoint{5.425000in}{3.020000in}} %
\pgfusepath{clip}%
\pgfsetroundcap%
\pgfsetroundjoin%
\pgfsetlinewidth{0.803000pt}%
\definecolor{currentstroke}{rgb}{0.800000,0.800000,0.800000}%
\pgfsetstrokecolor{currentstroke}%
\pgfsetdash{}{0pt}%
\pgfpathmoveto{\pgfqpoint{5.304512in}{1.233139in}}%
\pgfpathlineto{\pgfqpoint{5.304512in}{4.253139in}}%
\pgfusepath{stroke}%
\end{pgfscope}%
\begin{pgfscope}%
\definecolor{textcolor}{rgb}{0.150000,0.150000,0.150000}%
\pgfsetstrokecolor{textcolor}%
\pgfsetfillcolor{textcolor}%
\pgftext[x=5.304512in,y=1.069250in,,top]{\color{textcolor}\rmfamily\fontsize{36.000000}{43.200000}\selectfont 6}%
\end{pgfscope}%
\begin{pgfscope}%
\pgfpathrectangle{\pgfqpoint{1.535194in}{1.233139in}}{\pgfqpoint{5.425000in}{3.020000in}} %
\pgfusepath{clip}%
\pgfsetroundcap%
\pgfsetroundjoin%
\pgfsetlinewidth{0.803000pt}%
\definecolor{currentstroke}{rgb}{0.800000,0.800000,0.800000}%
\pgfsetstrokecolor{currentstroke}%
\pgfsetdash{}{0pt}%
\pgfpathmoveto{\pgfqpoint{6.009058in}{1.233139in}}%
\pgfpathlineto{\pgfqpoint{6.009058in}{4.253139in}}%
\pgfusepath{stroke}%
\end{pgfscope}%
\begin{pgfscope}%
\definecolor{textcolor}{rgb}{0.150000,0.150000,0.150000}%
\pgfsetstrokecolor{textcolor}%
\pgfsetfillcolor{textcolor}%
\pgftext[x=6.009058in,y=1.069250in,,top]{\color{textcolor}\rmfamily\fontsize{36.000000}{43.200000}\selectfont 7}%
\end{pgfscope}%
\begin{pgfscope}%
\pgfpathrectangle{\pgfqpoint{1.535194in}{1.233139in}}{\pgfqpoint{5.425000in}{3.020000in}} %
\pgfusepath{clip}%
\pgfsetroundcap%
\pgfsetroundjoin%
\pgfsetlinewidth{0.803000pt}%
\definecolor{currentstroke}{rgb}{0.800000,0.800000,0.800000}%
\pgfsetstrokecolor{currentstroke}%
\pgfsetdash{}{0pt}%
\pgfpathmoveto{\pgfqpoint{6.713603in}{1.233139in}}%
\pgfpathlineto{\pgfqpoint{6.713603in}{4.253139in}}%
\pgfusepath{stroke}%
\end{pgfscope}%
\begin{pgfscope}%
\definecolor{textcolor}{rgb}{0.150000,0.150000,0.150000}%
\pgfsetstrokecolor{textcolor}%
\pgfsetfillcolor{textcolor}%
\pgftext[x=6.713603in,y=1.069250in,,top]{\color{textcolor}\rmfamily\fontsize{36.000000}{43.200000}\selectfont 8}%
\end{pgfscope}%
\begin{pgfscope}%
\definecolor{textcolor}{rgb}{0.150000,0.150000,0.150000}%
\pgfsetstrokecolor{textcolor}%
\pgfsetfillcolor{textcolor}%
\pgftext[x=4.247694in,y=0.569194in,,top]{\color{textcolor}\rmfamily\fontsize{38.000000}{45.600000}\selectfont Workload}%
\end{pgfscope}%
\begin{pgfscope}%
\pgfpathrectangle{\pgfqpoint{1.535194in}{1.233139in}}{\pgfqpoint{5.425000in}{3.020000in}} %
\pgfusepath{clip}%
\pgfsetroundcap%
\pgfsetroundjoin%
\pgfsetlinewidth{0.803000pt}%
\definecolor{currentstroke}{rgb}{0.800000,0.800000,0.800000}%
\pgfsetstrokecolor{currentstroke}%
\pgfsetdash{}{0pt}%
\pgfpathmoveto{\pgfqpoint{1.535194in}{1.233139in}}%
\pgfpathlineto{\pgfqpoint{6.960194in}{1.233139in}}%
\pgfusepath{stroke}%
\end{pgfscope}%
\begin{pgfscope}%
\definecolor{textcolor}{rgb}{0.150000,0.150000,0.150000}%
\pgfsetstrokecolor{textcolor}%
\pgfsetfillcolor{textcolor}%
\pgftext[x=1.141805in,y=1.059639in,left,base]{\color{textcolor}\rmfamily\fontsize{36.000000}{43.200000}\selectfont 0}%
\end{pgfscope}%
\begin{pgfscope}%
\pgfpathrectangle{\pgfqpoint{1.535194in}{1.233139in}}{\pgfqpoint{5.425000in}{3.020000in}} %
\pgfusepath{clip}%
\pgfsetroundcap%
\pgfsetroundjoin%
\pgfsetlinewidth{0.803000pt}%
\definecolor{currentstroke}{rgb}{0.800000,0.800000,0.800000}%
\pgfsetstrokecolor{currentstroke}%
\pgfsetdash{}{0pt}%
\pgfpathmoveto{\pgfqpoint{1.535194in}{2.311710in}}%
\pgfpathlineto{\pgfqpoint{6.960194in}{2.311710in}}%
\pgfusepath{stroke}%
\end{pgfscope}%
\begin{pgfscope}%
\definecolor{textcolor}{rgb}{0.150000,0.150000,0.150000}%
\pgfsetstrokecolor{textcolor}%
\pgfsetfillcolor{textcolor}%
\pgftext[x=0.912305in,y=2.138210in,left,base]{\color{textcolor}\rmfamily\fontsize{36.000000}{43.200000}\selectfont 50}%
\end{pgfscope}%
\begin{pgfscope}%
\pgfpathrectangle{\pgfqpoint{1.535194in}{1.233139in}}{\pgfqpoint{5.425000in}{3.020000in}} %
\pgfusepath{clip}%
\pgfsetroundcap%
\pgfsetroundjoin%
\pgfsetlinewidth{0.803000pt}%
\definecolor{currentstroke}{rgb}{0.800000,0.800000,0.800000}%
\pgfsetstrokecolor{currentstroke}%
\pgfsetdash{}{0pt}%
\pgfpathmoveto{\pgfqpoint{1.535194in}{3.390281in}}%
\pgfpathlineto{\pgfqpoint{6.960194in}{3.390281in}}%
\pgfusepath{stroke}%
\end{pgfscope}%
\begin{pgfscope}%
\definecolor{textcolor}{rgb}{0.150000,0.150000,0.150000}%
\pgfsetstrokecolor{textcolor}%
\pgfsetfillcolor{textcolor}%
\pgftext[x=0.682805in,y=3.216781in,left,base]{\color{textcolor}\rmfamily\fontsize{36.000000}{43.200000}\selectfont 100}%
\end{pgfscope}%
\begin{pgfscope}%
\definecolor{textcolor}{rgb}{0.150000,0.150000,0.150000}%
\pgfsetstrokecolor{textcolor}%
\pgfsetfillcolor{textcolor}%
\pgftext[x=0.627250in,y=2.743139in,,bottom,rotate=90.000000]{\color{textcolor}\rmfamily\fontsize{38.000000}{45.600000}\selectfont Size (GB)}%
\end{pgfscope}%
\begin{pgfscope}%
\pgfpathrectangle{\pgfqpoint{1.535194in}{1.233139in}}{\pgfqpoint{5.425000in}{3.020000in}} %
\pgfusepath{clip}%
\pgfsetbuttcap%
\pgfsetroundjoin%
\pgfsetlinewidth{3.011250pt}%
\definecolor{currentstroke}{rgb}{0.866667,0.517647,0.321569}%
\pgfsetstrokecolor{currentstroke}%
\pgfsetdash{{3.000000pt}{0.000000pt}}{0.000000pt}%
\pgfpathmoveto{\pgfqpoint{1.781785in}{1.546546in}}%
\pgfpathlineto{\pgfqpoint{2.486331in}{2.063342in}}%
\pgfpathlineto{\pgfqpoint{3.190876in}{2.757665in}}%
\pgfpathlineto{\pgfqpoint{3.895421in}{2.757687in}}%
\pgfpathlineto{\pgfqpoint{4.599967in}{2.757719in}}%
\pgfpathlineto{\pgfqpoint{5.304512in}{2.757788in}}%
\pgfpathlineto{\pgfqpoint{6.009058in}{2.757910in}}%
\pgfpathlineto{\pgfqpoint{6.713603in}{2.757922in}}%
\pgfusepath{stroke}%
\end{pgfscope}%
\begin{pgfscope}%
\pgfpathrectangle{\pgfqpoint{1.535194in}{1.233139in}}{\pgfqpoint{5.425000in}{3.020000in}} %
\pgfusepath{clip}%
\pgfsetbuttcap%
\pgfsetroundjoin%
\definecolor{currentfill}{rgb}{0.866667,0.517647,0.321569}%
\pgfsetfillcolor{currentfill}%
\pgfsetlinewidth{0.752812pt}%
\definecolor{currentstroke}{rgb}{1.000000,1.000000,1.000000}%
\pgfsetstrokecolor{currentstroke}%
\pgfsetdash{}{0pt}%
\pgfsys@defobject{currentmarker}{\pgfqpoint{-0.104167in}{-0.104167in}}{\pgfqpoint{0.104167in}{0.104167in}}{%
\pgfpathmoveto{\pgfqpoint{0.000000in}{-0.104167in}}%
\pgfpathcurveto{\pgfqpoint{0.027625in}{-0.104167in}}{\pgfqpoint{0.054123in}{-0.093191in}}{\pgfqpoint{0.073657in}{-0.073657in}}%
\pgfpathcurveto{\pgfqpoint{0.093191in}{-0.054123in}}{\pgfqpoint{0.104167in}{-0.027625in}}{\pgfqpoint{0.104167in}{0.000000in}}%
\pgfpathcurveto{\pgfqpoint{0.104167in}{0.027625in}}{\pgfqpoint{0.093191in}{0.054123in}}{\pgfqpoint{0.073657in}{0.073657in}}%
\pgfpathcurveto{\pgfqpoint{0.054123in}{0.093191in}}{\pgfqpoint{0.027625in}{0.104167in}}{\pgfqpoint{0.000000in}{0.104167in}}%
\pgfpathcurveto{\pgfqpoint{-0.027625in}{0.104167in}}{\pgfqpoint{-0.054123in}{0.093191in}}{\pgfqpoint{-0.073657in}{0.073657in}}%
\pgfpathcurveto{\pgfqpoint{-0.093191in}{0.054123in}}{\pgfqpoint{-0.104167in}{0.027625in}}{\pgfqpoint{-0.104167in}{0.000000in}}%
\pgfpathcurveto{\pgfqpoint{-0.104167in}{-0.027625in}}{\pgfqpoint{-0.093191in}{-0.054123in}}{\pgfqpoint{-0.073657in}{-0.073657in}}%
\pgfpathcurveto{\pgfqpoint{-0.054123in}{-0.093191in}}{\pgfqpoint{-0.027625in}{-0.104167in}}{\pgfqpoint{0.000000in}{-0.104167in}}%
\pgfpathclose%
\pgfusepath{stroke,fill}%
}%
\begin{pgfscope}%
\pgfsys@transformshift{1.781785in}{1.546546in}%
\pgfsys@useobject{currentmarker}{}%
\end{pgfscope}%
\begin{pgfscope}%
\pgfsys@transformshift{2.486331in}{2.063342in}%
\pgfsys@useobject{currentmarker}{}%
\end{pgfscope}%
\begin{pgfscope}%
\pgfsys@transformshift{3.190876in}{2.757665in}%
\pgfsys@useobject{currentmarker}{}%
\end{pgfscope}%
\begin{pgfscope}%
\pgfsys@transformshift{3.895421in}{2.757687in}%
\pgfsys@useobject{currentmarker}{}%
\end{pgfscope}%
\begin{pgfscope}%
\pgfsys@transformshift{4.599967in}{2.757719in}%
\pgfsys@useobject{currentmarker}{}%
\end{pgfscope}%
\begin{pgfscope}%
\pgfsys@transformshift{5.304512in}{2.757788in}%
\pgfsys@useobject{currentmarker}{}%
\end{pgfscope}%
\begin{pgfscope}%
\pgfsys@transformshift{6.009058in}{2.757910in}%
\pgfsys@useobject{currentmarker}{}%
\end{pgfscope}%
\begin{pgfscope}%
\pgfsys@transformshift{6.713603in}{2.757922in}%
\pgfsys@useobject{currentmarker}{}%
\end{pgfscope}%
\end{pgfscope}%
\begin{pgfscope}%
\pgfpathrectangle{\pgfqpoint{1.535194in}{1.233139in}}{\pgfqpoint{5.425000in}{3.020000in}} %
\pgfusepath{clip}%
\pgfsetbuttcap%
\pgfsetroundjoin%
\pgfsetlinewidth{3.011250pt}%
\definecolor{currentstroke}{rgb}{0.298039,0.447059,0.690196}%
\pgfsetstrokecolor{currentstroke}%
\pgfsetdash{{6.000000pt}{6.000000pt}}{0.000000pt}%
\pgfpathmoveto{\pgfqpoint{1.781785in}{1.546546in}}%
\pgfpathlineto{\pgfqpoint{2.486331in}{2.063342in}}%
\pgfpathlineto{\pgfqpoint{3.190876in}{3.758523in}}%
\pgfpathlineto{\pgfqpoint{3.895421in}{3.758544in}}%
\pgfpathlineto{\pgfqpoint{4.599967in}{3.880597in}}%
\pgfpathlineto{\pgfqpoint{5.304512in}{3.880666in}}%
\pgfpathlineto{\pgfqpoint{6.009058in}{3.880666in}}%
\pgfpathlineto{\pgfqpoint{6.713603in}{3.899904in}}%
\pgfusepath{stroke}%
\end{pgfscope}%
\begin{pgfscope}%
\pgfpathrectangle{\pgfqpoint{1.535194in}{1.233139in}}{\pgfqpoint{5.425000in}{3.020000in}} %
\pgfusepath{clip}%
\pgfsetbuttcap%
\pgfsetmiterjoin%
\definecolor{currentfill}{rgb}{0.298039,0.447059,0.690196}%
\pgfsetfillcolor{currentfill}%
\pgfsetlinewidth{0.752812pt}%
\definecolor{currentstroke}{rgb}{1.000000,1.000000,1.000000}%
\pgfsetstrokecolor{currentstroke}%
\pgfsetdash{}{0pt}%
\pgfsys@defobject{currentmarker}{\pgfqpoint{-0.104167in}{-0.104167in}}{\pgfqpoint{0.104167in}{0.104167in}}{%
\pgfpathmoveto{\pgfqpoint{0.000000in}{0.104167in}}%
\pgfpathlineto{\pgfqpoint{-0.104167in}{-0.104167in}}%
\pgfpathlineto{\pgfqpoint{0.104167in}{-0.104167in}}%
\pgfpathclose%
\pgfusepath{stroke,fill}%
}%
\begin{pgfscope}%
\pgfsys@transformshift{1.781785in}{1.546546in}%
\pgfsys@useobject{currentmarker}{}%
\end{pgfscope}%
\begin{pgfscope}%
\pgfsys@transformshift{2.486331in}{2.063342in}%
\pgfsys@useobject{currentmarker}{}%
\end{pgfscope}%
\begin{pgfscope}%
\pgfsys@transformshift{3.190876in}{3.758523in}%
\pgfsys@useobject{currentmarker}{}%
\end{pgfscope}%
\begin{pgfscope}%
\pgfsys@transformshift{3.895421in}{3.758544in}%
\pgfsys@useobject{currentmarker}{}%
\end{pgfscope}%
\begin{pgfscope}%
\pgfsys@transformshift{4.599967in}{3.880597in}%
\pgfsys@useobject{currentmarker}{}%
\end{pgfscope}%
\begin{pgfscope}%
\pgfsys@transformshift{5.304512in}{3.880666in}%
\pgfsys@useobject{currentmarker}{}%
\end{pgfscope}%
\begin{pgfscope}%
\pgfsys@transformshift{6.009058in}{3.880666in}%
\pgfsys@useobject{currentmarker}{}%
\end{pgfscope}%
\begin{pgfscope}%
\pgfsys@transformshift{6.713603in}{3.899904in}%
\pgfsys@useobject{currentmarker}{}%
\end{pgfscope}%
\end{pgfscope}%
\begin{pgfscope}%
\pgfsetrectcap%
\pgfsetmiterjoin%
\pgfsetlinewidth{1.003750pt}%
\definecolor{currentstroke}{rgb}{0.800000,0.800000,0.800000}%
\pgfsetstrokecolor{currentstroke}%
\pgfsetdash{}{0pt}%
\pgfpathmoveto{\pgfqpoint{1.535194in}{1.233139in}}%
\pgfpathlineto{\pgfqpoint{1.535194in}{4.253139in}}%
\pgfusepath{stroke}%
\end{pgfscope}%
\begin{pgfscope}%
\pgfsetrectcap%
\pgfsetmiterjoin%
\pgfsetlinewidth{1.003750pt}%
\definecolor{currentstroke}{rgb}{0.800000,0.800000,0.800000}%
\pgfsetstrokecolor{currentstroke}%
\pgfsetdash{}{0pt}%
\pgfpathmoveto{\pgfqpoint{6.960194in}{1.233139in}}%
\pgfpathlineto{\pgfqpoint{6.960194in}{4.253139in}}%
\pgfusepath{stroke}%
\end{pgfscope}%
\begin{pgfscope}%
\pgfsetrectcap%
\pgfsetmiterjoin%
\pgfsetlinewidth{1.003750pt}%
\definecolor{currentstroke}{rgb}{0.800000,0.800000,0.800000}%
\pgfsetstrokecolor{currentstroke}%
\pgfsetdash{}{0pt}%
\pgfpathmoveto{\pgfqpoint{1.535194in}{1.233139in}}%
\pgfpathlineto{\pgfqpoint{6.960194in}{1.233139in}}%
\pgfusepath{stroke}%
\end{pgfscope}%
\begin{pgfscope}%
\pgfsetrectcap%
\pgfsetmiterjoin%
\pgfsetlinewidth{1.003750pt}%
\definecolor{currentstroke}{rgb}{0.800000,0.800000,0.800000}%
\pgfsetstrokecolor{currentstroke}%
\pgfsetdash{}{0pt}%
\pgfpathmoveto{\pgfqpoint{1.535194in}{4.253139in}}%
\pgfpathlineto{\pgfqpoint{6.960194in}{4.253139in}}%
\pgfusepath{stroke}%
\end{pgfscope}%
\begin{pgfscope}%
\pgfsetroundcap%
\pgfsetroundjoin%
\pgfsetlinewidth{3.011250pt}%
\definecolor{currentstroke}{rgb}{1.000000,1.000000,1.000000}%
\pgfsetstrokecolor{currentstroke}%
\pgfsetdash{}{0pt}%
\pgfpathmoveto{\pgfqpoint{1.576194in}{4.348416in}}%
\pgfpathlineto{\pgfqpoint{2.209528in}{4.348416in}}%
\pgfusepath{stroke}%
\end{pgfscope}%
\begin{pgfscope}%
\pgfsetbuttcap%
\pgfsetroundjoin%
\pgfsetlinewidth{3.011250pt}%
\definecolor{currentstroke}{rgb}{0.866667,0.517647,0.321569}%
\pgfsetstrokecolor{currentstroke}%
\pgfsetdash{{3.000000pt}{0.000000pt}}{0.000000pt}%
\pgfpathmoveto{\pgfqpoint{1.576194in}{3.928833in}}%
\pgfpathlineto{\pgfqpoint{2.209528in}{3.928833in}}%
\pgfusepath{stroke}%
\end{pgfscope}%
\begin{pgfscope}%
\pgfsetbuttcap%
\pgfsetroundjoin%
\definecolor{currentfill}{rgb}{0.866667,0.517647,0.321569}%
\pgfsetfillcolor{currentfill}%
\pgfsetlinewidth{0.000000pt}%
\definecolor{currentstroke}{rgb}{0.866667,0.517647,0.321569}%
\pgfsetstrokecolor{currentstroke}%
\pgfsetdash{}{0pt}%
\pgfsys@defobject{currentmarker}{\pgfqpoint{-0.104167in}{-0.104167in}}{\pgfqpoint{0.104167in}{0.104167in}}{%
\pgfpathmoveto{\pgfqpoint{0.000000in}{-0.104167in}}%
\pgfpathcurveto{\pgfqpoint{0.027625in}{-0.104167in}}{\pgfqpoint{0.054123in}{-0.093191in}}{\pgfqpoint{0.073657in}{-0.073657in}}%
\pgfpathcurveto{\pgfqpoint{0.093191in}{-0.054123in}}{\pgfqpoint{0.104167in}{-0.027625in}}{\pgfqpoint{0.104167in}{0.000000in}}%
\pgfpathcurveto{\pgfqpoint{0.104167in}{0.027625in}}{\pgfqpoint{0.093191in}{0.054123in}}{\pgfqpoint{0.073657in}{0.073657in}}%
\pgfpathcurveto{\pgfqpoint{0.054123in}{0.093191in}}{\pgfqpoint{0.027625in}{0.104167in}}{\pgfqpoint{0.000000in}{0.104167in}}%
\pgfpathcurveto{\pgfqpoint{-0.027625in}{0.104167in}}{\pgfqpoint{-0.054123in}{0.093191in}}{\pgfqpoint{-0.073657in}{0.073657in}}%
\pgfpathcurveto{\pgfqpoint{-0.093191in}{0.054123in}}{\pgfqpoint{-0.104167in}{0.027625in}}{\pgfqpoint{-0.104167in}{0.000000in}}%
\pgfpathcurveto{\pgfqpoint{-0.104167in}{-0.027625in}}{\pgfqpoint{-0.093191in}{-0.054123in}}{\pgfqpoint{-0.073657in}{-0.073657in}}%
\pgfpathcurveto{\pgfqpoint{-0.054123in}{-0.093191in}}{\pgfqpoint{-0.027625in}{-0.104167in}}{\pgfqpoint{0.000000in}{-0.104167in}}%
\pgfpathclose%
\pgfusepath{fill}%
}%
\begin{pgfscope}%
\pgfsys@transformshift{1.892861in}{3.928833in}%
\pgfsys@useobject{currentmarker}{}%
\end{pgfscope}%
\end{pgfscope}%
\begin{pgfscope}%
\definecolor{textcolor}{rgb}{0.150000,0.150000,0.150000}%
\pgfsetstrokecolor{textcolor}%
\pgfsetfillcolor{textcolor}%
\pgftext[x=2.262305in,y=3.744111in,left,base]{\color{textcolor}\rmfamily\fontsize{38.000000}{45.600000}\selectfont SM}%
\end{pgfscope}%
\begin{pgfscope}%
\pgfsetbuttcap%
\pgfsetroundjoin%
\pgfsetlinewidth{3.011250pt}%
\definecolor{currentstroke}{rgb}{0.298039,0.447059,0.690196}%
\pgfsetstrokecolor{currentstroke}%
\pgfsetdash{{6.000000pt}{6.000000pt}}{0.000000pt}%
\pgfpathmoveto{\pgfqpoint{1.576194in}{3.509250in}}%
\pgfpathlineto{\pgfqpoint{2.209528in}{3.509250in}}%
\pgfusepath{stroke}%
\end{pgfscope}%
\begin{pgfscope}%
\pgfsetbuttcap%
\pgfsetmiterjoin%
\definecolor{currentfill}{rgb}{0.298039,0.447059,0.690196}%
\pgfsetfillcolor{currentfill}%
\pgfsetlinewidth{0.000000pt}%
\definecolor{currentstroke}{rgb}{0.298039,0.447059,0.690196}%
\pgfsetstrokecolor{currentstroke}%
\pgfsetdash{}{0pt}%
\pgfsys@defobject{currentmarker}{\pgfqpoint{-0.104167in}{-0.104167in}}{\pgfqpoint{0.104167in}{0.104167in}}{%
\pgfpathmoveto{\pgfqpoint{0.000000in}{0.104167in}}%
\pgfpathlineto{\pgfqpoint{-0.104167in}{-0.104167in}}%
\pgfpathlineto{\pgfqpoint{0.104167in}{-0.104167in}}%
\pgfpathclose%
\pgfusepath{fill}%
}%
\begin{pgfscope}%
\pgfsys@transformshift{1.892861in}{3.509250in}%
\pgfsys@useobject{currentmarker}{}%
\end{pgfscope}%
\end{pgfscope}%
\begin{pgfscope}%
\definecolor{textcolor}{rgb}{0.150000,0.150000,0.150000}%
\pgfsetstrokecolor{textcolor}%
\pgfsetfillcolor{textcolor}%
\pgftext[x=2.262305in,y=3.324528in,left,base]{\color{textcolor}\rmfamily\fontsize{38.000000}{45.600000}\selectfont SA}%
\end{pgfscope}%
\end{pgfpicture}%
\makeatother%
\endgroup%
%
}
\caption{Budget = 64 GB}
\end{subfigure}
\caption{Real size of the stored artifact for storage-aware (SA) and heuristics-based (HM) materializations}
\label{exp-sa-vs-simple-size}
\end{figure}
\textbf{Effect of Materialization on Storage.}
Typically, in a real collaborative environment, deciding on a reasonable materialization budget requires knowledge of the expected size of the artifacts, number of the users in a collaborative environment, and rate of incoming workloads.
The goal of this experiment is to show that even with a small budget, our materialization algorithms, particularly our storage-aware algorithm, store a large portion of the artifacts that reappear in future workloads.
We hypothesize that there is considerable overlap between columns of different dataset artifacts in ML workloads.
Therefore, the actual total size of the artifacts our storage-aware algorithm materializes is larger than the specified budget.
We run the workloads of Table \ref{kaggle-workload} under different materialization budgets using both our heuristics-based and storage-aware materialization algorithms.
Figures \ref{exp-sa-vs-simple-size}(a)-(d) show the real size of the stored artifacts under different materialization budgets for the heuristics-based (HM) and storage-aware (SA) algorithms.
Furthermore, to show the total size of the materialized artifacts, we also implement an all materializer (represented by ALL in the figure).
The all-materializer materializes every artifact in EG.
In HM, the maximum real size is always equal to the budget.
However, in SA, we observe that the real size of the stored artifacts reaches up to 8 times the budget.
With a materialization budget of 8 GB and 16 GB, SA materializes nearly 50\% and 80\% of all the artifacts.
When the materialization budget is more than 16 GB, SA materializes nearly all the artifacts.
This indicates that there is considerable overlap between the artifacts of ML workloads.
By deduplicating the artifacts, our storage-aware materialization can materialize more artifacts.
Note that when an artifact with a high utility value has no overlap with other artifacts, SA still prioritizes it over other artifacts.
As a result, it is likely that when materializing an artifact that has no overlap with other artifacts, the total size of the materialized data decreases.
Figure \ref{exp-sa-vs-simple-size}(a) shows such an example.
After executing Workload 2, SA materializes several artifacts that overlap with each other.
However, in Workload 3, SA materializes a new artifact with a high utility, which represents a large dataset with many features (i.e., 1133 columns and around 3 GB).
Since the new artifact is large, SA removes many of the existing artifacts.
As a result, the total size of the materialized artifacts decreases after executing Workload 3.

\textbf{Effect of Materialization on Run-time.}
The goal of this experiment is to show that SA has a small run-time with a low materialization budget.
Figure \ref{run-time-vs-mat} shows the total run-time of all the workloads under different budgets for heuristics-based and storage-aware algorithms.
We also plot the total run-time for the all-materializer.
Even with a materialization budget of 8 GB, SA has a comparable performance to the scenario where all the artifacts are materialized (i.e., difference in run-time is 100 seconds).
On the contrary, under a small materialization budget ($\leq 16$), HM performs on average 50\% worse than SA.
For larger materialization budgets ($\geq 16$), HM performs better, however, its performance is still around 40\% worse than SA.
This is mainly because many of the preprocessed dataset artifacts are large, e.g., in Workload 3, some artifacts are more than 3 GB.
Most of these artifacts differ with each other in only a few columns.
However, HM is unable to exploit this similarity and chooses not to materialize any of the large artifacts.
Recomputing these artifacts is costly, which results in a larger total run-time for HM under a smaller budget.
\begin{figure}[t]
\centering
 \resizebox{\columnwidth}{!}{%
%% Creator: Matplotlib, PGF backend
%%
%% To include the figure in your LaTeX document, write
%%   \input{<filename>.pgf}
%%
%% Make sure the required packages are loaded in your preamble
%%   \usepackage{pgf}
%%
%% Figures using additional raster images can only be included by \input if
%% they are in the same directory as the main LaTeX file. For loading figures
%% from other directories you can use the `import` package
%%   \usepackage{import}
%% and then include the figures with
%%   \import{<path to file>}{<filename>.pgf}
%%
%% Matplotlib used the following preamble
%%   \usepackage{fontspec}
%%   \setmonofont{Andale Mono}
%%
\begingroup%
\makeatletter%
\begin{pgfpicture}%
\pgfpathrectangle{\pgfpointorigin}{\pgfqpoint{7.769500in}{5.538722in}}%
\pgfusepath{use as bounding box, clip}%
\begin{pgfscope}%
\pgfsetbuttcap%
\pgfsetmiterjoin%
\definecolor{currentfill}{rgb}{1.000000,1.000000,1.000000}%
\pgfsetfillcolor{currentfill}%
\pgfsetlinewidth{0.000000pt}%
\definecolor{currentstroke}{rgb}{1.000000,1.000000,1.000000}%
\pgfsetstrokecolor{currentstroke}%
\pgfsetdash{}{0pt}%
\pgfpathmoveto{\pgfqpoint{0.000000in}{0.000000in}}%
\pgfpathlineto{\pgfqpoint{7.769500in}{0.000000in}}%
\pgfpathlineto{\pgfqpoint{7.769500in}{5.538722in}}%
\pgfpathlineto{\pgfqpoint{0.000000in}{5.538722in}}%
\pgfpathclose%
\pgfusepath{fill}%
\end{pgfscope}%
\begin{pgfscope}%
\pgfsetbuttcap%
\pgfsetmiterjoin%
\definecolor{currentfill}{rgb}{1.000000,1.000000,1.000000}%
\pgfsetfillcolor{currentfill}%
\pgfsetlinewidth{0.000000pt}%
\definecolor{currentstroke}{rgb}{0.000000,0.000000,0.000000}%
\pgfsetstrokecolor{currentstroke}%
\pgfsetstrokeopacity{0.000000}%
\pgfsetdash{}{0pt}%
\pgfpathmoveto{\pgfqpoint{1.584777in}{1.109166in}}%
\pgfpathlineto{\pgfqpoint{7.602833in}{1.109166in}}%
\pgfpathlineto{\pgfqpoint{7.602833in}{5.009166in}}%
\pgfpathlineto{\pgfqpoint{1.584777in}{5.009166in}}%
\pgfpathclose%
\pgfusepath{fill}%
\end{pgfscope}%
\begin{pgfscope}%
\definecolor{textcolor}{rgb}{0.150000,0.150000,0.150000}%
\pgfsetstrokecolor{textcolor}%
\pgfsetfillcolor{textcolor}%
\pgftext[x=2.631396in,y=0.945277in,,top]{\color{textcolor}\rmfamily\fontsize{28.000000}{33.600000}\selectfont 8}%
\end{pgfscope}%
\begin{pgfscope}%
\definecolor{textcolor}{rgb}{0.150000,0.150000,0.150000}%
\pgfsetstrokecolor{textcolor}%
\pgfsetfillcolor{textcolor}%
\pgftext[x=3.939669in,y=0.945277in,,top]{\color{textcolor}\rmfamily\fontsize{28.000000}{33.600000}\selectfont 16}%
\end{pgfscope}%
\begin{pgfscope}%
\definecolor{textcolor}{rgb}{0.150000,0.150000,0.150000}%
\pgfsetstrokecolor{textcolor}%
\pgfsetfillcolor{textcolor}%
\pgftext[x=5.247942in,y=0.945277in,,top]{\color{textcolor}\rmfamily\fontsize{28.000000}{33.600000}\selectfont 32}%
\end{pgfscope}%
\begin{pgfscope}%
\definecolor{textcolor}{rgb}{0.150000,0.150000,0.150000}%
\pgfsetstrokecolor{textcolor}%
\pgfsetfillcolor{textcolor}%
\pgftext[x=6.556215in,y=0.945277in,,top]{\color{textcolor}\rmfamily\fontsize{28.000000}{33.600000}\selectfont 64}%
\end{pgfscope}%
\begin{pgfscope}%
\definecolor{textcolor}{rgb}{0.150000,0.150000,0.150000}%
\pgfsetstrokecolor{textcolor}%
\pgfsetfillcolor{textcolor}%
\pgftext[x=4.593805in,y=0.544000in,,top]{\color{textcolor}\rmfamily\fontsize{32.000000}{38.400000}\selectfont Budget (GB)}%
\end{pgfscope}%
\begin{pgfscope}%
\pgfpathrectangle{\pgfqpoint{1.584777in}{1.109166in}}{\pgfqpoint{6.018056in}{3.900000in}} %
\pgfusepath{clip}%
\pgfsetroundcap%
\pgfsetroundjoin%
\pgfsetlinewidth{0.803000pt}%
\definecolor{currentstroke}{rgb}{0.800000,0.800000,0.800000}%
\pgfsetstrokecolor{currentstroke}%
\pgfsetdash{}{0pt}%
\pgfpathmoveto{\pgfqpoint{1.584777in}{1.109166in}}%
\pgfpathlineto{\pgfqpoint{7.602833in}{1.109166in}}%
\pgfusepath{stroke}%
\end{pgfscope}%
\begin{pgfscope}%
\definecolor{textcolor}{rgb}{0.150000,0.150000,0.150000}%
\pgfsetstrokecolor{textcolor}%
\pgfsetfillcolor{textcolor}%
\pgftext[x=1.242389in,y=0.974222in,left,base]{\color{textcolor}\rmfamily\fontsize{28.000000}{33.600000}\selectfont 0}%
\end{pgfscope}%
\begin{pgfscope}%
\pgfpathrectangle{\pgfqpoint{1.584777in}{1.109166in}}{\pgfqpoint{6.018056in}{3.900000in}} %
\pgfusepath{clip}%
\pgfsetroundcap%
\pgfsetroundjoin%
\pgfsetlinewidth{0.803000pt}%
\definecolor{currentstroke}{rgb}{0.800000,0.800000,0.800000}%
\pgfsetstrokecolor{currentstroke}%
\pgfsetdash{}{0pt}%
\pgfpathmoveto{\pgfqpoint{1.584777in}{1.666309in}}%
\pgfpathlineto{\pgfqpoint{7.602833in}{1.666309in}}%
\pgfusepath{stroke}%
\end{pgfscope}%
\begin{pgfscope}%
\definecolor{textcolor}{rgb}{0.150000,0.150000,0.150000}%
\pgfsetstrokecolor{textcolor}%
\pgfsetfillcolor{textcolor}%
\pgftext[x=0.885389in,y=1.531365in,left,base]{\color{textcolor}\rmfamily\fontsize{28.000000}{33.600000}\selectfont 250}%
\end{pgfscope}%
\begin{pgfscope}%
\pgfpathrectangle{\pgfqpoint{1.584777in}{1.109166in}}{\pgfqpoint{6.018056in}{3.900000in}} %
\pgfusepath{clip}%
\pgfsetroundcap%
\pgfsetroundjoin%
\pgfsetlinewidth{0.803000pt}%
\definecolor{currentstroke}{rgb}{0.800000,0.800000,0.800000}%
\pgfsetstrokecolor{currentstroke}%
\pgfsetdash{}{0pt}%
\pgfpathmoveto{\pgfqpoint{1.584777in}{2.223452in}}%
\pgfpathlineto{\pgfqpoint{7.602833in}{2.223452in}}%
\pgfusepath{stroke}%
\end{pgfscope}%
\begin{pgfscope}%
\definecolor{textcolor}{rgb}{0.150000,0.150000,0.150000}%
\pgfsetstrokecolor{textcolor}%
\pgfsetfillcolor{textcolor}%
\pgftext[x=0.885389in,y=2.088508in,left,base]{\color{textcolor}\rmfamily\fontsize{28.000000}{33.600000}\selectfont 500}%
\end{pgfscope}%
\begin{pgfscope}%
\pgfpathrectangle{\pgfqpoint{1.584777in}{1.109166in}}{\pgfqpoint{6.018056in}{3.900000in}} %
\pgfusepath{clip}%
\pgfsetroundcap%
\pgfsetroundjoin%
\pgfsetlinewidth{0.803000pt}%
\definecolor{currentstroke}{rgb}{0.800000,0.800000,0.800000}%
\pgfsetstrokecolor{currentstroke}%
\pgfsetdash{}{0pt}%
\pgfpathmoveto{\pgfqpoint{1.584777in}{2.780595in}}%
\pgfpathlineto{\pgfqpoint{7.602833in}{2.780595in}}%
\pgfusepath{stroke}%
\end{pgfscope}%
\begin{pgfscope}%
\definecolor{textcolor}{rgb}{0.150000,0.150000,0.150000}%
\pgfsetstrokecolor{textcolor}%
\pgfsetfillcolor{textcolor}%
\pgftext[x=0.885389in,y=2.645650in,left,base]{\color{textcolor}\rmfamily\fontsize{28.000000}{33.600000}\selectfont 750}%
\end{pgfscope}%
\begin{pgfscope}%
\pgfpathrectangle{\pgfqpoint{1.584777in}{1.109166in}}{\pgfqpoint{6.018056in}{3.900000in}} %
\pgfusepath{clip}%
\pgfsetroundcap%
\pgfsetroundjoin%
\pgfsetlinewidth{0.803000pt}%
\definecolor{currentstroke}{rgb}{0.800000,0.800000,0.800000}%
\pgfsetstrokecolor{currentstroke}%
\pgfsetdash{}{0pt}%
\pgfpathmoveto{\pgfqpoint{1.584777in}{3.337738in}}%
\pgfpathlineto{\pgfqpoint{7.602833in}{3.337738in}}%
\pgfusepath{stroke}%
\end{pgfscope}%
\begin{pgfscope}%
\definecolor{textcolor}{rgb}{0.150000,0.150000,0.150000}%
\pgfsetstrokecolor{textcolor}%
\pgfsetfillcolor{textcolor}%
\pgftext[x=1.053778in,y=3.202793in,left,base]{\color{textcolor}\rmfamily\fontsize{28.000000}{33.600000}\selectfont 1k}%
\end{pgfscope}%
\begin{pgfscope}%
\pgfpathrectangle{\pgfqpoint{1.584777in}{1.109166in}}{\pgfqpoint{6.018056in}{3.900000in}} %
\pgfusepath{clip}%
\pgfsetroundcap%
\pgfsetroundjoin%
\pgfsetlinewidth{0.803000pt}%
\definecolor{currentstroke}{rgb}{0.800000,0.800000,0.800000}%
\pgfsetstrokecolor{currentstroke}%
\pgfsetdash{}{0pt}%
\pgfpathmoveto{\pgfqpoint{1.584777in}{3.894881in}}%
\pgfpathlineto{\pgfqpoint{7.602833in}{3.894881in}}%
\pgfusepath{stroke}%
\end{pgfscope}%
\begin{pgfscope}%
\definecolor{textcolor}{rgb}{0.150000,0.150000,0.150000}%
\pgfsetstrokecolor{textcolor}%
\pgfsetfillcolor{textcolor}%
\pgftext[x=0.599555in,y=3.759936in,left,base]{\color{textcolor}\rmfamily\fontsize{28.000000}{33.600000}\selectfont 1.25k}%
\end{pgfscope}%
\begin{pgfscope}%
\pgfpathrectangle{\pgfqpoint{1.584777in}{1.109166in}}{\pgfqpoint{6.018056in}{3.900000in}} %
\pgfusepath{clip}%
\pgfsetroundcap%
\pgfsetroundjoin%
\pgfsetlinewidth{0.803000pt}%
\definecolor{currentstroke}{rgb}{0.800000,0.800000,0.800000}%
\pgfsetstrokecolor{currentstroke}%
\pgfsetdash{}{0pt}%
\pgfpathmoveto{\pgfqpoint{1.584777in}{4.452023in}}%
\pgfpathlineto{\pgfqpoint{7.602833in}{4.452023in}}%
\pgfusepath{stroke}%
\end{pgfscope}%
\begin{pgfscope}%
\definecolor{textcolor}{rgb}{0.150000,0.150000,0.150000}%
\pgfsetstrokecolor{textcolor}%
\pgfsetfillcolor{textcolor}%
\pgftext[x=0.778055in,y=4.317079in,left,base]{\color{textcolor}\rmfamily\fontsize{28.000000}{33.600000}\selectfont 1.5k}%
\end{pgfscope}%
\begin{pgfscope}%
\pgfpathrectangle{\pgfqpoint{1.584777in}{1.109166in}}{\pgfqpoint{6.018056in}{3.900000in}} %
\pgfusepath{clip}%
\pgfsetroundcap%
\pgfsetroundjoin%
\pgfsetlinewidth{0.803000pt}%
\definecolor{currentstroke}{rgb}{0.800000,0.800000,0.800000}%
\pgfsetstrokecolor{currentstroke}%
\pgfsetdash{}{0pt}%
\pgfpathmoveto{\pgfqpoint{1.584777in}{5.009166in}}%
\pgfpathlineto{\pgfqpoint{7.602833in}{5.009166in}}%
\pgfusepath{stroke}%
\end{pgfscope}%
\begin{pgfscope}%
\definecolor{textcolor}{rgb}{0.150000,0.150000,0.150000}%
\pgfsetstrokecolor{textcolor}%
\pgfsetfillcolor{textcolor}%
\pgftext[x=0.599555in,y=4.874222in,left,base]{\color{textcolor}\rmfamily\fontsize{28.000000}{33.600000}\selectfont 1.75k}%
\end{pgfscope}%
\begin{pgfscope}%
\definecolor{textcolor}{rgb}{0.150000,0.150000,0.150000}%
\pgfsetstrokecolor{textcolor}%
\pgfsetfillcolor{textcolor}%
\pgftext[x=0.544000in,y=3.059166in,,bottom,rotate=90.000000]{\color{textcolor}\rmfamily\fontsize{32.000000}{38.400000}\selectfont Total Run Time (s)}%
\end{pgfscope}%
\begin{pgfscope}%
\pgfpathrectangle{\pgfqpoint{1.584777in}{1.109166in}}{\pgfqpoint{6.018056in}{3.900000in}} %
\pgfusepath{clip}%
\pgfsetbuttcap%
\pgfsetmiterjoin%
\definecolor{currentfill}{rgb}{0.347059,0.458824,0.641176}%
\pgfsetfillcolor{currentfill}%
\pgfsetlinewidth{0.803000pt}%
\definecolor{currentstroke}{rgb}{0.000000,0.000000,0.000000}%
\pgfsetstrokecolor{currentstroke}%
\pgfsetdash{}{0pt}%
\pgfpathmoveto{\pgfqpoint{2.108087in}{1.109166in}}%
\pgfpathlineto{\pgfqpoint{2.631396in}{1.109166in}}%
\pgfpathlineto{\pgfqpoint{2.631396in}{3.270798in}}%
\pgfpathlineto{\pgfqpoint{2.108087in}{3.270798in}}%
\pgfpathclose%
\pgfusepath{stroke,fill}%
\end{pgfscope}%
\begin{pgfscope}%
\pgfsetbuttcap%
\pgfsetmiterjoin%
\definecolor{currentfill}{rgb}{0.347059,0.458824,0.641176}%
\pgfsetfillcolor{currentfill}%
\pgfsetlinewidth{0.803000pt}%
\definecolor{currentstroke}{rgb}{0.000000,0.000000,0.000000}%
\pgfsetstrokecolor{currentstroke}%
\pgfsetdash{}{0pt}%
\pgfpathrectangle{\pgfqpoint{1.584777in}{1.109166in}}{\pgfqpoint{6.018056in}{3.900000in}} %
\pgfusepath{clip}%
\pgfpathmoveto{\pgfqpoint{2.108087in}{1.109166in}}%
\pgfpathlineto{\pgfqpoint{2.631396in}{1.109166in}}%
\pgfpathlineto{\pgfqpoint{2.631396in}{3.270798in}}%
\pgfpathlineto{\pgfqpoint{2.108087in}{3.270798in}}%
\pgfpathclose%
\pgfusepath{clip}%
\pgfsys@defobject{currentpattern}{\pgfqpoint{0in}{0in}}{\pgfqpoint{1in}{1in}}{%
\begin{pgfscope}%
\pgfpathrectangle{\pgfqpoint{0in}{0in}}{\pgfqpoint{1in}{1in}}%
\pgfusepath{clip}%
\pgfpathmoveto{\pgfqpoint{-0.500000in}{0.500000in}}%
\pgfpathlineto{\pgfqpoint{0.500000in}{1.500000in}}%
\pgfpathmoveto{\pgfqpoint{-0.333333in}{0.333333in}}%
\pgfpathlineto{\pgfqpoint{0.666667in}{1.333333in}}%
\pgfpathmoveto{\pgfqpoint{-0.166667in}{0.166667in}}%
\pgfpathlineto{\pgfqpoint{0.833333in}{1.166667in}}%
\pgfpathmoveto{\pgfqpoint{0.000000in}{0.000000in}}%
\pgfpathlineto{\pgfqpoint{1.000000in}{1.000000in}}%
\pgfpathmoveto{\pgfqpoint{0.166667in}{-0.166667in}}%
\pgfpathlineto{\pgfqpoint{1.166667in}{0.833333in}}%
\pgfpathmoveto{\pgfqpoint{0.333333in}{-0.333333in}}%
\pgfpathlineto{\pgfqpoint{1.333333in}{0.666667in}}%
\pgfpathmoveto{\pgfqpoint{0.500000in}{-0.500000in}}%
\pgfpathlineto{\pgfqpoint{1.500000in}{0.500000in}}%
\pgfusepath{stroke}%
\end{pgfscope}%
}%
\pgfsys@transformshift{2.108087in}{1.109166in}%
\pgfsys@useobject{currentpattern}{}%
\pgfsys@transformshift{1in}{0in}%
\pgfsys@transformshift{-1in}{0in}%
\pgfsys@transformshift{0in}{1in}%
\pgfsys@useobject{currentpattern}{}%
\pgfsys@transformshift{1in}{0in}%
\pgfsys@transformshift{-1in}{0in}%
\pgfsys@transformshift{0in}{1in}%
\pgfsys@useobject{currentpattern}{}%
\pgfsys@transformshift{1in}{0in}%
\pgfsys@transformshift{-1in}{0in}%
\pgfsys@transformshift{0in}{1in}%
\end{pgfscope}%
\begin{pgfscope}%
\pgfpathrectangle{\pgfqpoint{1.584777in}{1.109166in}}{\pgfqpoint{6.018056in}{3.900000in}} %
\pgfusepath{clip}%
\pgfsetbuttcap%
\pgfsetmiterjoin%
\definecolor{currentfill}{rgb}{0.347059,0.458824,0.641176}%
\pgfsetfillcolor{currentfill}%
\pgfsetlinewidth{0.803000pt}%
\definecolor{currentstroke}{rgb}{0.000000,0.000000,0.000000}%
\pgfsetstrokecolor{currentstroke}%
\pgfsetdash{}{0pt}%
\pgfpathmoveto{\pgfqpoint{3.416360in}{1.109166in}}%
\pgfpathlineto{\pgfqpoint{3.939669in}{1.109166in}}%
\pgfpathlineto{\pgfqpoint{3.939669in}{3.210967in}}%
\pgfpathlineto{\pgfqpoint{3.416360in}{3.210967in}}%
\pgfpathclose%
\pgfusepath{stroke,fill}%
\end{pgfscope}%
\begin{pgfscope}%
\pgfsetbuttcap%
\pgfsetmiterjoin%
\definecolor{currentfill}{rgb}{0.347059,0.458824,0.641176}%
\pgfsetfillcolor{currentfill}%
\pgfsetlinewidth{0.803000pt}%
\definecolor{currentstroke}{rgb}{0.000000,0.000000,0.000000}%
\pgfsetstrokecolor{currentstroke}%
\pgfsetdash{}{0pt}%
\pgfpathrectangle{\pgfqpoint{1.584777in}{1.109166in}}{\pgfqpoint{6.018056in}{3.900000in}} %
\pgfusepath{clip}%
\pgfpathmoveto{\pgfqpoint{3.416360in}{1.109166in}}%
\pgfpathlineto{\pgfqpoint{3.939669in}{1.109166in}}%
\pgfpathlineto{\pgfqpoint{3.939669in}{3.210967in}}%
\pgfpathlineto{\pgfqpoint{3.416360in}{3.210967in}}%
\pgfpathclose%
\pgfusepath{clip}%
\pgfsys@defobject{currentpattern}{\pgfqpoint{0in}{0in}}{\pgfqpoint{1in}{1in}}{%
\begin{pgfscope}%
\pgfpathrectangle{\pgfqpoint{0in}{0in}}{\pgfqpoint{1in}{1in}}%
\pgfusepath{clip}%
\pgfpathmoveto{\pgfqpoint{-0.500000in}{0.500000in}}%
\pgfpathlineto{\pgfqpoint{0.500000in}{1.500000in}}%
\pgfpathmoveto{\pgfqpoint{-0.333333in}{0.333333in}}%
\pgfpathlineto{\pgfqpoint{0.666667in}{1.333333in}}%
\pgfpathmoveto{\pgfqpoint{-0.166667in}{0.166667in}}%
\pgfpathlineto{\pgfqpoint{0.833333in}{1.166667in}}%
\pgfpathmoveto{\pgfqpoint{0.000000in}{0.000000in}}%
\pgfpathlineto{\pgfqpoint{1.000000in}{1.000000in}}%
\pgfpathmoveto{\pgfqpoint{0.166667in}{-0.166667in}}%
\pgfpathlineto{\pgfqpoint{1.166667in}{0.833333in}}%
\pgfpathmoveto{\pgfqpoint{0.333333in}{-0.333333in}}%
\pgfpathlineto{\pgfqpoint{1.333333in}{0.666667in}}%
\pgfpathmoveto{\pgfqpoint{0.500000in}{-0.500000in}}%
\pgfpathlineto{\pgfqpoint{1.500000in}{0.500000in}}%
\pgfusepath{stroke}%
\end{pgfscope}%
}%
\pgfsys@transformshift{3.416360in}{1.109166in}%
\pgfsys@useobject{currentpattern}{}%
\pgfsys@transformshift{1in}{0in}%
\pgfsys@transformshift{-1in}{0in}%
\pgfsys@transformshift{0in}{1in}%
\pgfsys@useobject{currentpattern}{}%
\pgfsys@transformshift{1in}{0in}%
\pgfsys@transformshift{-1in}{0in}%
\pgfsys@transformshift{0in}{1in}%
\pgfsys@useobject{currentpattern}{}%
\pgfsys@transformshift{1in}{0in}%
\pgfsys@transformshift{-1in}{0in}%
\pgfsys@transformshift{0in}{1in}%
\end{pgfscope}%
\begin{pgfscope}%
\pgfpathrectangle{\pgfqpoint{1.584777in}{1.109166in}}{\pgfqpoint{6.018056in}{3.900000in}} %
\pgfusepath{clip}%
\pgfsetbuttcap%
\pgfsetmiterjoin%
\definecolor{currentfill}{rgb}{0.347059,0.458824,0.641176}%
\pgfsetfillcolor{currentfill}%
\pgfsetlinewidth{0.803000pt}%
\definecolor{currentstroke}{rgb}{0.000000,0.000000,0.000000}%
\pgfsetstrokecolor{currentstroke}%
\pgfsetdash{}{0pt}%
\pgfpathmoveto{\pgfqpoint{4.724633in}{1.109166in}}%
\pgfpathlineto{\pgfqpoint{5.247942in}{1.109166in}}%
\pgfpathlineto{\pgfqpoint{5.247942in}{3.134253in}}%
\pgfpathlineto{\pgfqpoint{4.724633in}{3.134253in}}%
\pgfpathclose%
\pgfusepath{stroke,fill}%
\end{pgfscope}%
\begin{pgfscope}%
\pgfsetbuttcap%
\pgfsetmiterjoin%
\definecolor{currentfill}{rgb}{0.347059,0.458824,0.641176}%
\pgfsetfillcolor{currentfill}%
\pgfsetlinewidth{0.803000pt}%
\definecolor{currentstroke}{rgb}{0.000000,0.000000,0.000000}%
\pgfsetstrokecolor{currentstroke}%
\pgfsetdash{}{0pt}%
\pgfpathrectangle{\pgfqpoint{1.584777in}{1.109166in}}{\pgfqpoint{6.018056in}{3.900000in}} %
\pgfusepath{clip}%
\pgfpathmoveto{\pgfqpoint{4.724633in}{1.109166in}}%
\pgfpathlineto{\pgfqpoint{5.247942in}{1.109166in}}%
\pgfpathlineto{\pgfqpoint{5.247942in}{3.134253in}}%
\pgfpathlineto{\pgfqpoint{4.724633in}{3.134253in}}%
\pgfpathclose%
\pgfusepath{clip}%
\pgfsys@defobject{currentpattern}{\pgfqpoint{0in}{0in}}{\pgfqpoint{1in}{1in}}{%
\begin{pgfscope}%
\pgfpathrectangle{\pgfqpoint{0in}{0in}}{\pgfqpoint{1in}{1in}}%
\pgfusepath{clip}%
\pgfpathmoveto{\pgfqpoint{-0.500000in}{0.500000in}}%
\pgfpathlineto{\pgfqpoint{0.500000in}{1.500000in}}%
\pgfpathmoveto{\pgfqpoint{-0.333333in}{0.333333in}}%
\pgfpathlineto{\pgfqpoint{0.666667in}{1.333333in}}%
\pgfpathmoveto{\pgfqpoint{-0.166667in}{0.166667in}}%
\pgfpathlineto{\pgfqpoint{0.833333in}{1.166667in}}%
\pgfpathmoveto{\pgfqpoint{0.000000in}{0.000000in}}%
\pgfpathlineto{\pgfqpoint{1.000000in}{1.000000in}}%
\pgfpathmoveto{\pgfqpoint{0.166667in}{-0.166667in}}%
\pgfpathlineto{\pgfqpoint{1.166667in}{0.833333in}}%
\pgfpathmoveto{\pgfqpoint{0.333333in}{-0.333333in}}%
\pgfpathlineto{\pgfqpoint{1.333333in}{0.666667in}}%
\pgfpathmoveto{\pgfqpoint{0.500000in}{-0.500000in}}%
\pgfpathlineto{\pgfqpoint{1.500000in}{0.500000in}}%
\pgfusepath{stroke}%
\end{pgfscope}%
}%
\pgfsys@transformshift{4.724633in}{1.109166in}%
\pgfsys@useobject{currentpattern}{}%
\pgfsys@transformshift{1in}{0in}%
\pgfsys@transformshift{-1in}{0in}%
\pgfsys@transformshift{0in}{1in}%
\pgfsys@useobject{currentpattern}{}%
\pgfsys@transformshift{1in}{0in}%
\pgfsys@transformshift{-1in}{0in}%
\pgfsys@transformshift{0in}{1in}%
\pgfsys@useobject{currentpattern}{}%
\pgfsys@transformshift{1in}{0in}%
\pgfsys@transformshift{-1in}{0in}%
\pgfsys@transformshift{0in}{1in}%
\end{pgfscope}%
\begin{pgfscope}%
\pgfpathrectangle{\pgfqpoint{1.584777in}{1.109166in}}{\pgfqpoint{6.018056in}{3.900000in}} %
\pgfusepath{clip}%
\pgfsetbuttcap%
\pgfsetmiterjoin%
\definecolor{currentfill}{rgb}{0.347059,0.458824,0.641176}%
\pgfsetfillcolor{currentfill}%
\pgfsetlinewidth{0.803000pt}%
\definecolor{currentstroke}{rgb}{0.000000,0.000000,0.000000}%
\pgfsetstrokecolor{currentstroke}%
\pgfsetdash{}{0pt}%
\pgfpathmoveto{\pgfqpoint{6.032906in}{1.109166in}}%
\pgfpathlineto{\pgfqpoint{6.556215in}{1.109166in}}%
\pgfpathlineto{\pgfqpoint{6.556215in}{3.106925in}}%
\pgfpathlineto{\pgfqpoint{6.032906in}{3.106925in}}%
\pgfpathclose%
\pgfusepath{stroke,fill}%
\end{pgfscope}%
\begin{pgfscope}%
\pgfsetbuttcap%
\pgfsetmiterjoin%
\definecolor{currentfill}{rgb}{0.347059,0.458824,0.641176}%
\pgfsetfillcolor{currentfill}%
\pgfsetlinewidth{0.803000pt}%
\definecolor{currentstroke}{rgb}{0.000000,0.000000,0.000000}%
\pgfsetstrokecolor{currentstroke}%
\pgfsetdash{}{0pt}%
\pgfpathrectangle{\pgfqpoint{1.584777in}{1.109166in}}{\pgfqpoint{6.018056in}{3.900000in}} %
\pgfusepath{clip}%
\pgfpathmoveto{\pgfqpoint{6.032906in}{1.109166in}}%
\pgfpathlineto{\pgfqpoint{6.556215in}{1.109166in}}%
\pgfpathlineto{\pgfqpoint{6.556215in}{3.106925in}}%
\pgfpathlineto{\pgfqpoint{6.032906in}{3.106925in}}%
\pgfpathclose%
\pgfusepath{clip}%
\pgfsys@defobject{currentpattern}{\pgfqpoint{0in}{0in}}{\pgfqpoint{1in}{1in}}{%
\begin{pgfscope}%
\pgfpathrectangle{\pgfqpoint{0in}{0in}}{\pgfqpoint{1in}{1in}}%
\pgfusepath{clip}%
\pgfpathmoveto{\pgfqpoint{-0.500000in}{0.500000in}}%
\pgfpathlineto{\pgfqpoint{0.500000in}{1.500000in}}%
\pgfpathmoveto{\pgfqpoint{-0.333333in}{0.333333in}}%
\pgfpathlineto{\pgfqpoint{0.666667in}{1.333333in}}%
\pgfpathmoveto{\pgfqpoint{-0.166667in}{0.166667in}}%
\pgfpathlineto{\pgfqpoint{0.833333in}{1.166667in}}%
\pgfpathmoveto{\pgfqpoint{0.000000in}{0.000000in}}%
\pgfpathlineto{\pgfqpoint{1.000000in}{1.000000in}}%
\pgfpathmoveto{\pgfqpoint{0.166667in}{-0.166667in}}%
\pgfpathlineto{\pgfqpoint{1.166667in}{0.833333in}}%
\pgfpathmoveto{\pgfqpoint{0.333333in}{-0.333333in}}%
\pgfpathlineto{\pgfqpoint{1.333333in}{0.666667in}}%
\pgfpathmoveto{\pgfqpoint{0.500000in}{-0.500000in}}%
\pgfpathlineto{\pgfqpoint{1.500000in}{0.500000in}}%
\pgfusepath{stroke}%
\end{pgfscope}%
}%
\pgfsys@transformshift{6.032906in}{1.109166in}%
\pgfsys@useobject{currentpattern}{}%
\pgfsys@transformshift{1in}{0in}%
\pgfsys@transformshift{-1in}{0in}%
\pgfsys@transformshift{0in}{1in}%
\pgfsys@useobject{currentpattern}{}%
\pgfsys@transformshift{1in}{0in}%
\pgfsys@transformshift{-1in}{0in}%
\pgfsys@transformshift{0in}{1in}%
\end{pgfscope}%
\begin{pgfscope}%
\pgfpathrectangle{\pgfqpoint{1.584777in}{1.109166in}}{\pgfqpoint{6.018056in}{3.900000in}} %
\pgfusepath{clip}%
\pgfsetbuttcap%
\pgfsetmiterjoin%
\definecolor{currentfill}{rgb}{0.798529,0.536765,0.389706}%
\pgfsetfillcolor{currentfill}%
\pgfsetlinewidth{0.803000pt}%
\definecolor{currentstroke}{rgb}{0.000000,0.000000,0.000000}%
\pgfsetstrokecolor{currentstroke}%
\pgfsetdash{}{0pt}%
\pgfpathmoveto{\pgfqpoint{2.631396in}{1.109166in}}%
\pgfpathlineto{\pgfqpoint{3.154705in}{1.109166in}}%
\pgfpathlineto{\pgfqpoint{3.154705in}{4.713932in}}%
\pgfpathlineto{\pgfqpoint{2.631396in}{4.713932in}}%
\pgfpathclose%
\pgfusepath{stroke,fill}%
\end{pgfscope}%
\begin{pgfscope}%
\pgfsetbuttcap%
\pgfsetmiterjoin%
\definecolor{currentfill}{rgb}{0.798529,0.536765,0.389706}%
\pgfsetfillcolor{currentfill}%
\pgfsetlinewidth{0.803000pt}%
\definecolor{currentstroke}{rgb}{0.000000,0.000000,0.000000}%
\pgfsetstrokecolor{currentstroke}%
\pgfsetdash{}{0pt}%
\pgfpathrectangle{\pgfqpoint{1.584777in}{1.109166in}}{\pgfqpoint{6.018056in}{3.900000in}} %
\pgfusepath{clip}%
\pgfpathmoveto{\pgfqpoint{2.631396in}{1.109166in}}%
\pgfpathlineto{\pgfqpoint{3.154705in}{1.109166in}}%
\pgfpathlineto{\pgfqpoint{3.154705in}{4.713932in}}%
\pgfpathlineto{\pgfqpoint{2.631396in}{4.713932in}}%
\pgfpathclose%
\pgfusepath{clip}%
\pgfsys@defobject{currentpattern}{\pgfqpoint{0in}{0in}}{\pgfqpoint{1in}{1in}}{%
\begin{pgfscope}%
\pgfpathrectangle{\pgfqpoint{0in}{0in}}{\pgfqpoint{1in}{1in}}%
\pgfusepath{clip}%
\pgfpathmoveto{\pgfqpoint{-0.500000in}{0.500000in}}%
\pgfpathlineto{\pgfqpoint{0.500000in}{1.500000in}}%
\pgfpathmoveto{\pgfqpoint{-0.333333in}{0.333333in}}%
\pgfpathlineto{\pgfqpoint{0.666667in}{1.333333in}}%
\pgfpathmoveto{\pgfqpoint{-0.166667in}{0.166667in}}%
\pgfpathlineto{\pgfqpoint{0.833333in}{1.166667in}}%
\pgfpathmoveto{\pgfqpoint{0.000000in}{0.000000in}}%
\pgfpathlineto{\pgfqpoint{1.000000in}{1.000000in}}%
\pgfpathmoveto{\pgfqpoint{0.166667in}{-0.166667in}}%
\pgfpathlineto{\pgfqpoint{1.166667in}{0.833333in}}%
\pgfpathmoveto{\pgfqpoint{0.333333in}{-0.333333in}}%
\pgfpathlineto{\pgfqpoint{1.333333in}{0.666667in}}%
\pgfpathmoveto{\pgfqpoint{0.500000in}{-0.500000in}}%
\pgfpathlineto{\pgfqpoint{1.500000in}{0.500000in}}%
\pgfpathmoveto{\pgfqpoint{-0.500000in}{0.500000in}}%
\pgfpathlineto{\pgfqpoint{0.500000in}{-0.500000in}}%
\pgfpathmoveto{\pgfqpoint{-0.333333in}{0.666667in}}%
\pgfpathlineto{\pgfqpoint{0.666667in}{-0.333333in}}%
\pgfpathmoveto{\pgfqpoint{-0.166667in}{0.833333in}}%
\pgfpathlineto{\pgfqpoint{0.833333in}{-0.166667in}}%
\pgfpathmoveto{\pgfqpoint{0.000000in}{1.000000in}}%
\pgfpathlineto{\pgfqpoint{1.000000in}{0.000000in}}%
\pgfpathmoveto{\pgfqpoint{0.166667in}{1.166667in}}%
\pgfpathlineto{\pgfqpoint{1.166667in}{0.166667in}}%
\pgfpathmoveto{\pgfqpoint{0.333333in}{1.333333in}}%
\pgfpathlineto{\pgfqpoint{1.333333in}{0.333333in}}%
\pgfpathmoveto{\pgfqpoint{0.500000in}{1.500000in}}%
\pgfpathlineto{\pgfqpoint{1.500000in}{0.500000in}}%
\pgfusepath{stroke}%
\end{pgfscope}%
}%
\pgfsys@transformshift{2.631396in}{1.109166in}%
\pgfsys@useobject{currentpattern}{}%
\pgfsys@transformshift{1in}{0in}%
\pgfsys@transformshift{-1in}{0in}%
\pgfsys@transformshift{0in}{1in}%
\pgfsys@useobject{currentpattern}{}%
\pgfsys@transformshift{1in}{0in}%
\pgfsys@transformshift{-1in}{0in}%
\pgfsys@transformshift{0in}{1in}%
\pgfsys@useobject{currentpattern}{}%
\pgfsys@transformshift{1in}{0in}%
\pgfsys@transformshift{-1in}{0in}%
\pgfsys@transformshift{0in}{1in}%
\pgfsys@useobject{currentpattern}{}%
\pgfsys@transformshift{1in}{0in}%
\pgfsys@transformshift{-1in}{0in}%
\pgfsys@transformshift{0in}{1in}%
\end{pgfscope}%
\begin{pgfscope}%
\pgfpathrectangle{\pgfqpoint{1.584777in}{1.109166in}}{\pgfqpoint{6.018056in}{3.900000in}} %
\pgfusepath{clip}%
\pgfsetbuttcap%
\pgfsetmiterjoin%
\definecolor{currentfill}{rgb}{0.798529,0.536765,0.389706}%
\pgfsetfillcolor{currentfill}%
\pgfsetlinewidth{0.803000pt}%
\definecolor{currentstroke}{rgb}{0.000000,0.000000,0.000000}%
\pgfsetstrokecolor{currentstroke}%
\pgfsetdash{}{0pt}%
\pgfpathmoveto{\pgfqpoint{3.939669in}{1.109166in}}%
\pgfpathlineto{\pgfqpoint{4.462978in}{1.109166in}}%
\pgfpathlineto{\pgfqpoint{4.462978in}{4.309457in}}%
\pgfpathlineto{\pgfqpoint{3.939669in}{4.309457in}}%
\pgfpathclose%
\pgfusepath{stroke,fill}%
\end{pgfscope}%
\begin{pgfscope}%
\pgfsetbuttcap%
\pgfsetmiterjoin%
\definecolor{currentfill}{rgb}{0.798529,0.536765,0.389706}%
\pgfsetfillcolor{currentfill}%
\pgfsetlinewidth{0.803000pt}%
\definecolor{currentstroke}{rgb}{0.000000,0.000000,0.000000}%
\pgfsetstrokecolor{currentstroke}%
\pgfsetdash{}{0pt}%
\pgfpathrectangle{\pgfqpoint{1.584777in}{1.109166in}}{\pgfqpoint{6.018056in}{3.900000in}} %
\pgfusepath{clip}%
\pgfpathmoveto{\pgfqpoint{3.939669in}{1.109166in}}%
\pgfpathlineto{\pgfqpoint{4.462978in}{1.109166in}}%
\pgfpathlineto{\pgfqpoint{4.462978in}{4.309457in}}%
\pgfpathlineto{\pgfqpoint{3.939669in}{4.309457in}}%
\pgfpathclose%
\pgfusepath{clip}%
\pgfsys@defobject{currentpattern}{\pgfqpoint{0in}{0in}}{\pgfqpoint{1in}{1in}}{%
\begin{pgfscope}%
\pgfpathrectangle{\pgfqpoint{0in}{0in}}{\pgfqpoint{1in}{1in}}%
\pgfusepath{clip}%
\pgfpathmoveto{\pgfqpoint{-0.500000in}{0.500000in}}%
\pgfpathlineto{\pgfqpoint{0.500000in}{1.500000in}}%
\pgfpathmoveto{\pgfqpoint{-0.333333in}{0.333333in}}%
\pgfpathlineto{\pgfqpoint{0.666667in}{1.333333in}}%
\pgfpathmoveto{\pgfqpoint{-0.166667in}{0.166667in}}%
\pgfpathlineto{\pgfqpoint{0.833333in}{1.166667in}}%
\pgfpathmoveto{\pgfqpoint{0.000000in}{0.000000in}}%
\pgfpathlineto{\pgfqpoint{1.000000in}{1.000000in}}%
\pgfpathmoveto{\pgfqpoint{0.166667in}{-0.166667in}}%
\pgfpathlineto{\pgfqpoint{1.166667in}{0.833333in}}%
\pgfpathmoveto{\pgfqpoint{0.333333in}{-0.333333in}}%
\pgfpathlineto{\pgfqpoint{1.333333in}{0.666667in}}%
\pgfpathmoveto{\pgfqpoint{0.500000in}{-0.500000in}}%
\pgfpathlineto{\pgfqpoint{1.500000in}{0.500000in}}%
\pgfpathmoveto{\pgfqpoint{-0.500000in}{0.500000in}}%
\pgfpathlineto{\pgfqpoint{0.500000in}{-0.500000in}}%
\pgfpathmoveto{\pgfqpoint{-0.333333in}{0.666667in}}%
\pgfpathlineto{\pgfqpoint{0.666667in}{-0.333333in}}%
\pgfpathmoveto{\pgfqpoint{-0.166667in}{0.833333in}}%
\pgfpathlineto{\pgfqpoint{0.833333in}{-0.166667in}}%
\pgfpathmoveto{\pgfqpoint{0.000000in}{1.000000in}}%
\pgfpathlineto{\pgfqpoint{1.000000in}{0.000000in}}%
\pgfpathmoveto{\pgfqpoint{0.166667in}{1.166667in}}%
\pgfpathlineto{\pgfqpoint{1.166667in}{0.166667in}}%
\pgfpathmoveto{\pgfqpoint{0.333333in}{1.333333in}}%
\pgfpathlineto{\pgfqpoint{1.333333in}{0.333333in}}%
\pgfpathmoveto{\pgfqpoint{0.500000in}{1.500000in}}%
\pgfpathlineto{\pgfqpoint{1.500000in}{0.500000in}}%
\pgfusepath{stroke}%
\end{pgfscope}%
}%
\pgfsys@transformshift{3.939669in}{1.109166in}%
\pgfsys@useobject{currentpattern}{}%
\pgfsys@transformshift{1in}{0in}%
\pgfsys@transformshift{-1in}{0in}%
\pgfsys@transformshift{0in}{1in}%
\pgfsys@useobject{currentpattern}{}%
\pgfsys@transformshift{1in}{0in}%
\pgfsys@transformshift{-1in}{0in}%
\pgfsys@transformshift{0in}{1in}%
\pgfsys@useobject{currentpattern}{}%
\pgfsys@transformshift{1in}{0in}%
\pgfsys@transformshift{-1in}{0in}%
\pgfsys@transformshift{0in}{1in}%
\pgfsys@useobject{currentpattern}{}%
\pgfsys@transformshift{1in}{0in}%
\pgfsys@transformshift{-1in}{0in}%
\pgfsys@transformshift{0in}{1in}%
\end{pgfscope}%
\begin{pgfscope}%
\pgfpathrectangle{\pgfqpoint{1.584777in}{1.109166in}}{\pgfqpoint{6.018056in}{3.900000in}} %
\pgfusepath{clip}%
\pgfsetbuttcap%
\pgfsetmiterjoin%
\definecolor{currentfill}{rgb}{0.798529,0.536765,0.389706}%
\pgfsetfillcolor{currentfill}%
\pgfsetlinewidth{0.803000pt}%
\definecolor{currentstroke}{rgb}{0.000000,0.000000,0.000000}%
\pgfsetstrokecolor{currentstroke}%
\pgfsetdash{}{0pt}%
\pgfpathmoveto{\pgfqpoint{5.247942in}{1.109166in}}%
\pgfpathlineto{\pgfqpoint{5.771251in}{1.109166in}}%
\pgfpathlineto{\pgfqpoint{5.771251in}{3.301951in}}%
\pgfpathlineto{\pgfqpoint{5.247942in}{3.301951in}}%
\pgfpathclose%
\pgfusepath{stroke,fill}%
\end{pgfscope}%
\begin{pgfscope}%
\pgfsetbuttcap%
\pgfsetmiterjoin%
\definecolor{currentfill}{rgb}{0.798529,0.536765,0.389706}%
\pgfsetfillcolor{currentfill}%
\pgfsetlinewidth{0.803000pt}%
\definecolor{currentstroke}{rgb}{0.000000,0.000000,0.000000}%
\pgfsetstrokecolor{currentstroke}%
\pgfsetdash{}{0pt}%
\pgfpathrectangle{\pgfqpoint{1.584777in}{1.109166in}}{\pgfqpoint{6.018056in}{3.900000in}} %
\pgfusepath{clip}%
\pgfpathmoveto{\pgfqpoint{5.247942in}{1.109166in}}%
\pgfpathlineto{\pgfqpoint{5.771251in}{1.109166in}}%
\pgfpathlineto{\pgfqpoint{5.771251in}{3.301951in}}%
\pgfpathlineto{\pgfqpoint{5.247942in}{3.301951in}}%
\pgfpathclose%
\pgfusepath{clip}%
\pgfsys@defobject{currentpattern}{\pgfqpoint{0in}{0in}}{\pgfqpoint{1in}{1in}}{%
\begin{pgfscope}%
\pgfpathrectangle{\pgfqpoint{0in}{0in}}{\pgfqpoint{1in}{1in}}%
\pgfusepath{clip}%
\pgfpathmoveto{\pgfqpoint{-0.500000in}{0.500000in}}%
\pgfpathlineto{\pgfqpoint{0.500000in}{1.500000in}}%
\pgfpathmoveto{\pgfqpoint{-0.333333in}{0.333333in}}%
\pgfpathlineto{\pgfqpoint{0.666667in}{1.333333in}}%
\pgfpathmoveto{\pgfqpoint{-0.166667in}{0.166667in}}%
\pgfpathlineto{\pgfqpoint{0.833333in}{1.166667in}}%
\pgfpathmoveto{\pgfqpoint{0.000000in}{0.000000in}}%
\pgfpathlineto{\pgfqpoint{1.000000in}{1.000000in}}%
\pgfpathmoveto{\pgfqpoint{0.166667in}{-0.166667in}}%
\pgfpathlineto{\pgfqpoint{1.166667in}{0.833333in}}%
\pgfpathmoveto{\pgfqpoint{0.333333in}{-0.333333in}}%
\pgfpathlineto{\pgfqpoint{1.333333in}{0.666667in}}%
\pgfpathmoveto{\pgfqpoint{0.500000in}{-0.500000in}}%
\pgfpathlineto{\pgfqpoint{1.500000in}{0.500000in}}%
\pgfpathmoveto{\pgfqpoint{-0.500000in}{0.500000in}}%
\pgfpathlineto{\pgfqpoint{0.500000in}{-0.500000in}}%
\pgfpathmoveto{\pgfqpoint{-0.333333in}{0.666667in}}%
\pgfpathlineto{\pgfqpoint{0.666667in}{-0.333333in}}%
\pgfpathmoveto{\pgfqpoint{-0.166667in}{0.833333in}}%
\pgfpathlineto{\pgfqpoint{0.833333in}{-0.166667in}}%
\pgfpathmoveto{\pgfqpoint{0.000000in}{1.000000in}}%
\pgfpathlineto{\pgfqpoint{1.000000in}{0.000000in}}%
\pgfpathmoveto{\pgfqpoint{0.166667in}{1.166667in}}%
\pgfpathlineto{\pgfqpoint{1.166667in}{0.166667in}}%
\pgfpathmoveto{\pgfqpoint{0.333333in}{1.333333in}}%
\pgfpathlineto{\pgfqpoint{1.333333in}{0.333333in}}%
\pgfpathmoveto{\pgfqpoint{0.500000in}{1.500000in}}%
\pgfpathlineto{\pgfqpoint{1.500000in}{0.500000in}}%
\pgfusepath{stroke}%
\end{pgfscope}%
}%
\pgfsys@transformshift{5.247942in}{1.109166in}%
\pgfsys@useobject{currentpattern}{}%
\pgfsys@transformshift{1in}{0in}%
\pgfsys@transformshift{-1in}{0in}%
\pgfsys@transformshift{0in}{1in}%
\pgfsys@useobject{currentpattern}{}%
\pgfsys@transformshift{1in}{0in}%
\pgfsys@transformshift{-1in}{0in}%
\pgfsys@transformshift{0in}{1in}%
\pgfsys@useobject{currentpattern}{}%
\pgfsys@transformshift{1in}{0in}%
\pgfsys@transformshift{-1in}{0in}%
\pgfsys@transformshift{0in}{1in}%
\end{pgfscope}%
\begin{pgfscope}%
\pgfpathrectangle{\pgfqpoint{1.584777in}{1.109166in}}{\pgfqpoint{6.018056in}{3.900000in}} %
\pgfusepath{clip}%
\pgfsetbuttcap%
\pgfsetmiterjoin%
\definecolor{currentfill}{rgb}{0.798529,0.536765,0.389706}%
\pgfsetfillcolor{currentfill}%
\pgfsetlinewidth{0.803000pt}%
\definecolor{currentstroke}{rgb}{0.000000,0.000000,0.000000}%
\pgfsetstrokecolor{currentstroke}%
\pgfsetdash{}{0pt}%
\pgfpathmoveto{\pgfqpoint{6.556215in}{1.109166in}}%
\pgfpathlineto{\pgfqpoint{7.079524in}{1.109166in}}%
\pgfpathlineto{\pgfqpoint{7.079524in}{3.183124in}}%
\pgfpathlineto{\pgfqpoint{6.556215in}{3.183124in}}%
\pgfpathclose%
\pgfusepath{stroke,fill}%
\end{pgfscope}%
\begin{pgfscope}%
\pgfsetbuttcap%
\pgfsetmiterjoin%
\definecolor{currentfill}{rgb}{0.798529,0.536765,0.389706}%
\pgfsetfillcolor{currentfill}%
\pgfsetlinewidth{0.803000pt}%
\definecolor{currentstroke}{rgb}{0.000000,0.000000,0.000000}%
\pgfsetstrokecolor{currentstroke}%
\pgfsetdash{}{0pt}%
\pgfpathrectangle{\pgfqpoint{1.584777in}{1.109166in}}{\pgfqpoint{6.018056in}{3.900000in}} %
\pgfusepath{clip}%
\pgfpathmoveto{\pgfqpoint{6.556215in}{1.109166in}}%
\pgfpathlineto{\pgfqpoint{7.079524in}{1.109166in}}%
\pgfpathlineto{\pgfqpoint{7.079524in}{3.183124in}}%
\pgfpathlineto{\pgfqpoint{6.556215in}{3.183124in}}%
\pgfpathclose%
\pgfusepath{clip}%
\pgfsys@defobject{currentpattern}{\pgfqpoint{0in}{0in}}{\pgfqpoint{1in}{1in}}{%
\begin{pgfscope}%
\pgfpathrectangle{\pgfqpoint{0in}{0in}}{\pgfqpoint{1in}{1in}}%
\pgfusepath{clip}%
\pgfpathmoveto{\pgfqpoint{-0.500000in}{0.500000in}}%
\pgfpathlineto{\pgfqpoint{0.500000in}{1.500000in}}%
\pgfpathmoveto{\pgfqpoint{-0.333333in}{0.333333in}}%
\pgfpathlineto{\pgfqpoint{0.666667in}{1.333333in}}%
\pgfpathmoveto{\pgfqpoint{-0.166667in}{0.166667in}}%
\pgfpathlineto{\pgfqpoint{0.833333in}{1.166667in}}%
\pgfpathmoveto{\pgfqpoint{0.000000in}{0.000000in}}%
\pgfpathlineto{\pgfqpoint{1.000000in}{1.000000in}}%
\pgfpathmoveto{\pgfqpoint{0.166667in}{-0.166667in}}%
\pgfpathlineto{\pgfqpoint{1.166667in}{0.833333in}}%
\pgfpathmoveto{\pgfqpoint{0.333333in}{-0.333333in}}%
\pgfpathlineto{\pgfqpoint{1.333333in}{0.666667in}}%
\pgfpathmoveto{\pgfqpoint{0.500000in}{-0.500000in}}%
\pgfpathlineto{\pgfqpoint{1.500000in}{0.500000in}}%
\pgfpathmoveto{\pgfqpoint{-0.500000in}{0.500000in}}%
\pgfpathlineto{\pgfqpoint{0.500000in}{-0.500000in}}%
\pgfpathmoveto{\pgfqpoint{-0.333333in}{0.666667in}}%
\pgfpathlineto{\pgfqpoint{0.666667in}{-0.333333in}}%
\pgfpathmoveto{\pgfqpoint{-0.166667in}{0.833333in}}%
\pgfpathlineto{\pgfqpoint{0.833333in}{-0.166667in}}%
\pgfpathmoveto{\pgfqpoint{0.000000in}{1.000000in}}%
\pgfpathlineto{\pgfqpoint{1.000000in}{0.000000in}}%
\pgfpathmoveto{\pgfqpoint{0.166667in}{1.166667in}}%
\pgfpathlineto{\pgfqpoint{1.166667in}{0.166667in}}%
\pgfpathmoveto{\pgfqpoint{0.333333in}{1.333333in}}%
\pgfpathlineto{\pgfqpoint{1.333333in}{0.333333in}}%
\pgfpathmoveto{\pgfqpoint{0.500000in}{1.500000in}}%
\pgfpathlineto{\pgfqpoint{1.500000in}{0.500000in}}%
\pgfusepath{stroke}%
\end{pgfscope}%
}%
\pgfsys@transformshift{6.556215in}{1.109166in}%
\pgfsys@useobject{currentpattern}{}%
\pgfsys@transformshift{1in}{0in}%
\pgfsys@transformshift{-1in}{0in}%
\pgfsys@transformshift{0in}{1in}%
\pgfsys@useobject{currentpattern}{}%
\pgfsys@transformshift{1in}{0in}%
\pgfsys@transformshift{-1in}{0in}%
\pgfsys@transformshift{0in}{1in}%
\pgfsys@useobject{currentpattern}{}%
\pgfsys@transformshift{1in}{0in}%
\pgfsys@transformshift{-1in}{0in}%
\pgfsys@transformshift{0in}{1in}%
\end{pgfscope}%
\begin{pgfscope}%
\pgfpathrectangle{\pgfqpoint{1.584777in}{1.109166in}}{\pgfqpoint{6.018056in}{3.900000in}} %
\pgfusepath{clip}%
\pgfsetroundcap%
\pgfsetroundjoin%
\pgfsetlinewidth{3.011250pt}%
\definecolor{currentstroke}{rgb}{0.000000,0.000000,0.000000}%
\pgfsetstrokecolor{currentstroke}%
\pgfsetdash{}{0pt}%
\pgfusepath{stroke}%
\end{pgfscope}%
\begin{pgfscope}%
\pgfpathrectangle{\pgfqpoint{1.584777in}{1.109166in}}{\pgfqpoint{6.018056in}{3.900000in}} %
\pgfusepath{clip}%
\pgfsetroundcap%
\pgfsetroundjoin%
\pgfsetlinewidth{3.011250pt}%
\definecolor{currentstroke}{rgb}{0.000000,0.000000,0.000000}%
\pgfsetstrokecolor{currentstroke}%
\pgfsetdash{}{0pt}%
\pgfusepath{stroke}%
\end{pgfscope}%
\begin{pgfscope}%
\pgfpathrectangle{\pgfqpoint{1.584777in}{1.109166in}}{\pgfqpoint{6.018056in}{3.900000in}} %
\pgfusepath{clip}%
\pgfsetroundcap%
\pgfsetroundjoin%
\pgfsetlinewidth{3.011250pt}%
\definecolor{currentstroke}{rgb}{0.000000,0.000000,0.000000}%
\pgfsetstrokecolor{currentstroke}%
\pgfsetdash{}{0pt}%
\pgfusepath{stroke}%
\end{pgfscope}%
\begin{pgfscope}%
\pgfpathrectangle{\pgfqpoint{1.584777in}{1.109166in}}{\pgfqpoint{6.018056in}{3.900000in}} %
\pgfusepath{clip}%
\pgfsetroundcap%
\pgfsetroundjoin%
\pgfsetlinewidth{3.011250pt}%
\definecolor{currentstroke}{rgb}{0.000000,0.000000,0.000000}%
\pgfsetstrokecolor{currentstroke}%
\pgfsetdash{}{0pt}%
\pgfusepath{stroke}%
\end{pgfscope}%
\begin{pgfscope}%
\pgfpathrectangle{\pgfqpoint{1.584777in}{1.109166in}}{\pgfqpoint{6.018056in}{3.900000in}} %
\pgfusepath{clip}%
\pgfsetroundcap%
\pgfsetroundjoin%
\pgfsetlinewidth{3.011250pt}%
\definecolor{currentstroke}{rgb}{0.000000,0.000000,0.000000}%
\pgfsetstrokecolor{currentstroke}%
\pgfsetdash{}{0pt}%
\pgfusepath{stroke}%
\end{pgfscope}%
\begin{pgfscope}%
\pgfpathrectangle{\pgfqpoint{1.584777in}{1.109166in}}{\pgfqpoint{6.018056in}{3.900000in}} %
\pgfusepath{clip}%
\pgfsetroundcap%
\pgfsetroundjoin%
\pgfsetlinewidth{3.011250pt}%
\definecolor{currentstroke}{rgb}{0.000000,0.000000,0.000000}%
\pgfsetstrokecolor{currentstroke}%
\pgfsetdash{}{0pt}%
\pgfusepath{stroke}%
\end{pgfscope}%
\begin{pgfscope}%
\pgfpathrectangle{\pgfqpoint{1.584777in}{1.109166in}}{\pgfqpoint{6.018056in}{3.900000in}} %
\pgfusepath{clip}%
\pgfsetroundcap%
\pgfsetroundjoin%
\pgfsetlinewidth{3.011250pt}%
\definecolor{currentstroke}{rgb}{0.000000,0.000000,0.000000}%
\pgfsetstrokecolor{currentstroke}%
\pgfsetdash{}{0pt}%
\pgfusepath{stroke}%
\end{pgfscope}%
\begin{pgfscope}%
\pgfpathrectangle{\pgfqpoint{1.584777in}{1.109166in}}{\pgfqpoint{6.018056in}{3.900000in}} %
\pgfusepath{clip}%
\pgfsetroundcap%
\pgfsetroundjoin%
\pgfsetlinewidth{3.011250pt}%
\definecolor{currentstroke}{rgb}{0.000000,0.000000,0.000000}%
\pgfsetstrokecolor{currentstroke}%
\pgfsetdash{}{0pt}%
\pgfusepath{stroke}%
\end{pgfscope}%
\begin{pgfscope}%
\pgfpathrectangle{\pgfqpoint{1.584777in}{1.109166in}}{\pgfqpoint{6.018056in}{3.900000in}} %
\pgfusepath{clip}%
\pgfsetbuttcap%
\pgfsetroundjoin%
\pgfsetlinewidth{4.015000pt}%
\definecolor{currentstroke}{rgb}{0.333333,0.658824,0.407843}%
\pgfsetstrokecolor{currentstroke}%
\pgfsetdash{{8.000000pt}{4.000000pt}}{0.000000pt}%
\pgfpathmoveto{\pgfqpoint{1.584777in}{3.080004in}}%
\pgfpathlineto{\pgfqpoint{7.602833in}{3.080004in}}%
\pgfusepath{stroke}%
\end{pgfscope}%
\begin{pgfscope}%
\pgfsetrectcap%
\pgfsetmiterjoin%
\pgfsetlinewidth{1.003750pt}%
\definecolor{currentstroke}{rgb}{0.800000,0.800000,0.800000}%
\pgfsetstrokecolor{currentstroke}%
\pgfsetdash{}{0pt}%
\pgfpathmoveto{\pgfqpoint{1.584777in}{1.109166in}}%
\pgfpathlineto{\pgfqpoint{1.584777in}{5.009166in}}%
\pgfusepath{stroke}%
\end{pgfscope}%
\begin{pgfscope}%
\pgfsetrectcap%
\pgfsetmiterjoin%
\pgfsetlinewidth{1.003750pt}%
\definecolor{currentstroke}{rgb}{0.800000,0.800000,0.800000}%
\pgfsetstrokecolor{currentstroke}%
\pgfsetdash{}{0pt}%
\pgfpathmoveto{\pgfqpoint{1.584777in}{1.109166in}}%
\pgfpathlineto{\pgfqpoint{7.602833in}{1.109166in}}%
\pgfusepath{stroke}%
\end{pgfscope}%
\begin{pgfscope}%
\pgfsetbuttcap%
\pgfsetmiterjoin%
\definecolor{currentfill}{rgb}{0.347059,0.458824,0.641176}%
\pgfsetfillcolor{currentfill}%
\pgfsetlinewidth{0.803000pt}%
\definecolor{currentstroke}{rgb}{0.000000,0.000000,0.000000}%
\pgfsetstrokecolor{currentstroke}%
\pgfsetdash{}{0pt}%
\pgfpathmoveto{\pgfqpoint{2.514805in}{5.010944in}}%
\pgfpathlineto{\pgfqpoint{3.137027in}{5.010944in}}%
\pgfpathlineto{\pgfqpoint{3.137027in}{5.283166in}}%
\pgfpathlineto{\pgfqpoint{2.514805in}{5.283166in}}%
\pgfpathclose%
\pgfusepath{stroke,fill}%
\end{pgfscope}%
\begin{pgfscope}%
\pgfsetbuttcap%
\pgfsetmiterjoin%
\definecolor{currentfill}{rgb}{0.347059,0.458824,0.641176}%
\pgfsetfillcolor{currentfill}%
\pgfsetlinewidth{0.803000pt}%
\definecolor{currentstroke}{rgb}{0.000000,0.000000,0.000000}%
\pgfsetstrokecolor{currentstroke}%
\pgfsetdash{}{0pt}%
\pgfpathmoveto{\pgfqpoint{2.514805in}{5.010944in}}%
\pgfpathlineto{\pgfqpoint{3.137027in}{5.010944in}}%
\pgfpathlineto{\pgfqpoint{3.137027in}{5.283166in}}%
\pgfpathlineto{\pgfqpoint{2.514805in}{5.283166in}}%
\pgfpathclose%
\pgfusepath{clip}%
\pgfsys@defobject{currentpattern}{\pgfqpoint{0in}{0in}}{\pgfqpoint{1in}{1in}}{%
\begin{pgfscope}%
\pgfpathrectangle{\pgfqpoint{0in}{0in}}{\pgfqpoint{1in}{1in}}%
\pgfusepath{clip}%
\pgfpathmoveto{\pgfqpoint{-0.500000in}{0.500000in}}%
\pgfpathlineto{\pgfqpoint{0.500000in}{1.500000in}}%
\pgfpathmoveto{\pgfqpoint{-0.333333in}{0.333333in}}%
\pgfpathlineto{\pgfqpoint{0.666667in}{1.333333in}}%
\pgfpathmoveto{\pgfqpoint{-0.166667in}{0.166667in}}%
\pgfpathlineto{\pgfqpoint{0.833333in}{1.166667in}}%
\pgfpathmoveto{\pgfqpoint{0.000000in}{0.000000in}}%
\pgfpathlineto{\pgfqpoint{1.000000in}{1.000000in}}%
\pgfpathmoveto{\pgfqpoint{0.166667in}{-0.166667in}}%
\pgfpathlineto{\pgfqpoint{1.166667in}{0.833333in}}%
\pgfpathmoveto{\pgfqpoint{0.333333in}{-0.333333in}}%
\pgfpathlineto{\pgfqpoint{1.333333in}{0.666667in}}%
\pgfpathmoveto{\pgfqpoint{0.500000in}{-0.500000in}}%
\pgfpathlineto{\pgfqpoint{1.500000in}{0.500000in}}%
\pgfusepath{stroke}%
\end{pgfscope}%
}%
\pgfsys@transformshift{2.514805in}{5.010944in}%
\pgfsys@useobject{currentpattern}{}%
\pgfsys@transformshift{1in}{0in}%
\pgfsys@transformshift{-1in}{0in}%
\pgfsys@transformshift{0in}{1in}%
\end{pgfscope}%
\begin{pgfscope}%
\definecolor{textcolor}{rgb}{0.150000,0.150000,0.150000}%
\pgfsetstrokecolor{textcolor}%
\pgfsetfillcolor{textcolor}%
\pgftext[x=3.175916in,y=5.010944in,left,base]{\color{textcolor}\rmfamily\fontsize{28.000000}{33.600000}\selectfont SA}%
\end{pgfscope}%
\begin{pgfscope}%
\pgfsetbuttcap%
\pgfsetmiterjoin%
\definecolor{currentfill}{rgb}{0.798529,0.536765,0.389706}%
\pgfsetfillcolor{currentfill}%
\pgfsetlinewidth{0.803000pt}%
\definecolor{currentstroke}{rgb}{0.000000,0.000000,0.000000}%
\pgfsetstrokecolor{currentstroke}%
\pgfsetdash{}{0pt}%
\pgfpathmoveto{\pgfqpoint{3.838583in}{5.010944in}}%
\pgfpathlineto{\pgfqpoint{4.460805in}{5.010944in}}%
\pgfpathlineto{\pgfqpoint{4.460805in}{5.283166in}}%
\pgfpathlineto{\pgfqpoint{3.838583in}{5.283166in}}%
\pgfpathclose%
\pgfusepath{stroke,fill}%
\end{pgfscope}%
\begin{pgfscope}%
\pgfsetbuttcap%
\pgfsetmiterjoin%
\definecolor{currentfill}{rgb}{0.798529,0.536765,0.389706}%
\pgfsetfillcolor{currentfill}%
\pgfsetlinewidth{0.803000pt}%
\definecolor{currentstroke}{rgb}{0.000000,0.000000,0.000000}%
\pgfsetstrokecolor{currentstroke}%
\pgfsetdash{}{0pt}%
\pgfpathmoveto{\pgfqpoint{3.838583in}{5.010944in}}%
\pgfpathlineto{\pgfqpoint{4.460805in}{5.010944in}}%
\pgfpathlineto{\pgfqpoint{4.460805in}{5.283166in}}%
\pgfpathlineto{\pgfqpoint{3.838583in}{5.283166in}}%
\pgfpathclose%
\pgfusepath{clip}%
\pgfsys@defobject{currentpattern}{\pgfqpoint{0in}{0in}}{\pgfqpoint{1in}{1in}}{%
\begin{pgfscope}%
\pgfpathrectangle{\pgfqpoint{0in}{0in}}{\pgfqpoint{1in}{1in}}%
\pgfusepath{clip}%
\pgfpathmoveto{\pgfqpoint{-0.500000in}{0.500000in}}%
\pgfpathlineto{\pgfqpoint{0.500000in}{1.500000in}}%
\pgfpathmoveto{\pgfqpoint{-0.333333in}{0.333333in}}%
\pgfpathlineto{\pgfqpoint{0.666667in}{1.333333in}}%
\pgfpathmoveto{\pgfqpoint{-0.166667in}{0.166667in}}%
\pgfpathlineto{\pgfqpoint{0.833333in}{1.166667in}}%
\pgfpathmoveto{\pgfqpoint{0.000000in}{0.000000in}}%
\pgfpathlineto{\pgfqpoint{1.000000in}{1.000000in}}%
\pgfpathmoveto{\pgfqpoint{0.166667in}{-0.166667in}}%
\pgfpathlineto{\pgfqpoint{1.166667in}{0.833333in}}%
\pgfpathmoveto{\pgfqpoint{0.333333in}{-0.333333in}}%
\pgfpathlineto{\pgfqpoint{1.333333in}{0.666667in}}%
\pgfpathmoveto{\pgfqpoint{0.500000in}{-0.500000in}}%
\pgfpathlineto{\pgfqpoint{1.500000in}{0.500000in}}%
\pgfpathmoveto{\pgfqpoint{-0.500000in}{0.500000in}}%
\pgfpathlineto{\pgfqpoint{0.500000in}{-0.500000in}}%
\pgfpathmoveto{\pgfqpoint{-0.333333in}{0.666667in}}%
\pgfpathlineto{\pgfqpoint{0.666667in}{-0.333333in}}%
\pgfpathmoveto{\pgfqpoint{-0.166667in}{0.833333in}}%
\pgfpathlineto{\pgfqpoint{0.833333in}{-0.166667in}}%
\pgfpathmoveto{\pgfqpoint{0.000000in}{1.000000in}}%
\pgfpathlineto{\pgfqpoint{1.000000in}{0.000000in}}%
\pgfpathmoveto{\pgfqpoint{0.166667in}{1.166667in}}%
\pgfpathlineto{\pgfqpoint{1.166667in}{0.166667in}}%
\pgfpathmoveto{\pgfqpoint{0.333333in}{1.333333in}}%
\pgfpathlineto{\pgfqpoint{1.333333in}{0.333333in}}%
\pgfpathmoveto{\pgfqpoint{0.500000in}{1.500000in}}%
\pgfpathlineto{\pgfqpoint{1.500000in}{0.500000in}}%
\pgfusepath{stroke}%
\end{pgfscope}%
}%
\pgfsys@transformshift{3.838583in}{5.010944in}%
\pgfsys@useobject{currentpattern}{}%
\pgfsys@transformshift{1in}{0in}%
\pgfsys@transformshift{-1in}{0in}%
\pgfsys@transformshift{0in}{1in}%
\end{pgfscope}%
\begin{pgfscope}%
\definecolor{textcolor}{rgb}{0.150000,0.150000,0.150000}%
\pgfsetstrokecolor{textcolor}%
\pgfsetfillcolor{textcolor}%
\pgftext[x=4.499694in,y=5.010944in,left,base]{\color{textcolor}\rmfamily\fontsize{28.000000}{33.600000}\selectfont HM}%
\end{pgfscope}%
\begin{pgfscope}%
\pgfsetbuttcap%
\pgfsetroundjoin%
\pgfsetlinewidth{4.015000pt}%
\definecolor{currentstroke}{rgb}{0.333333,0.658824,0.407843}%
\pgfsetstrokecolor{currentstroke}%
\pgfsetdash{{8.000000pt}{4.000000pt}}{0.000000pt}%
\pgfpathmoveto{\pgfqpoint{5.294194in}{5.147055in}}%
\pgfpathlineto{\pgfqpoint{5.916416in}{5.147055in}}%
\pgfusepath{stroke}%
\end{pgfscope}%
\begin{pgfscope}%
\definecolor{textcolor}{rgb}{0.150000,0.150000,0.150000}%
\pgfsetstrokecolor{textcolor}%
\pgfsetfillcolor{textcolor}%
\pgftext[x=5.955305in,y=5.010944in,left,base]{\color{textcolor}\rmfamily\fontsize{28.000000}{33.600000}\selectfont ALL}%
\end{pgfscope}%
\end{pgfpicture}%
\makeatother%
\endgroup%
%
}
\caption{Total run-time of the workloads with different materialization algorithms and budgets}
\label{run-time-vs-mat}
\end{figure}

\textbf{Model Materialization. }
One goal of our materialization algorithms is to materialize high-quality models as soon as they appear in a new workload.
In this experiment, we run a model-benchmarking scenario, where users compare the score of their models with the score of the best performing model in the collaborative environment.
The model-benchmarking scenario represents a common real-world use-case where every user is trying to improve the best current model.
We use the OpenML workloads for the model-benchmarking scenario.
The implementation of the scenario is as follows.
First, We execute every OpenML workload one by one and keep track of the best performing workload.
Then, for every workload, as a post-processing step, we compare the score of the model with the score of the best performing model so far.
We compare the cumulative run-time of running the model-benchmarking scenario using our collaborative optimizer (CO) with default configuration (i.e., storage-aware materializer with budget 100 MB and $\alpha=0.5$) and the OpenML baseline (OML).
Figure \ref{exp-model-materialization}(a) shows the cumulative run-time of executing the model-benchmarking scenario for CO and OML (i.e., the combined run-time of the current and the best workload).
In OML, for every workload, we have to re-run the pipeline of the best performing model.
As a result, OML has a larger cumulative run-time when compared with CO.
Figure \ref{exp-model-materialization}(b) explains the reason for such a large difference between OML and CO.
When CO encounters a model that performs better than all the existing models, the materialization algorithm immediately materializes the pipeline and the model.
As a result, we reuse the model from EG instead of re-running all the operations to train the model.
In Figure \ref{exp-model-materialization}(b), we observe that reusing the best pipeline for all the 2000 OpenML workloads has an overhead of 65 seconds.
In comparison, re-executing the best pipeline for OML has an overhead of 2000 seconds.

\begin{figure}[t]
\begin{subfigure}[b]{0.5\linewidth}
\centering
 \resizebox{\columnwidth}{!}{%
%% Creator: Matplotlib, PGF backend
%%
%% To include the figure in your LaTeX document, write
%%   \input{<filename>.pgf}
%%
%% Make sure the required packages are loaded in your preamble
%%   \usepackage{pgf}
%%
%% Figures using additional raster images can only be included by \input if
%% they are in the same directory as the main LaTeX file. For loading figures
%% from other directories you can use the `import` package
%%   \usepackage{import}
%% and then include the figures with
%%   \import{<path to file>}{<filename>.pgf}
%%
%% Matplotlib used the following preamble
%%   \usepackage{fontspec}
%%   \setmonofont{Andale Mono}
%%
\begingroup%
\makeatletter%
\begin{pgfpicture}%
\pgfpathrectangle{\pgfpointorigin}{\pgfqpoint{8.039987in}{5.001044in}}%
\pgfusepath{use as bounding box, clip}%
\begin{pgfscope}%
\pgfsetbuttcap%
\pgfsetmiterjoin%
\definecolor{currentfill}{rgb}{1.000000,1.000000,1.000000}%
\pgfsetfillcolor{currentfill}%
\pgfsetlinewidth{0.000000pt}%
\definecolor{currentstroke}{rgb}{1.000000,1.000000,1.000000}%
\pgfsetstrokecolor{currentstroke}%
\pgfsetdash{}{0pt}%
\pgfpathmoveto{\pgfqpoint{0.000000in}{0.000000in}}%
\pgfpathlineto{\pgfqpoint{8.039987in}{0.000000in}}%
\pgfpathlineto{\pgfqpoint{8.039987in}{5.001044in}}%
\pgfpathlineto{\pgfqpoint{0.000000in}{5.001044in}}%
\pgfpathclose%
\pgfusepath{fill}%
\end{pgfscope}%
\begin{pgfscope}%
\pgfsetbuttcap%
\pgfsetmiterjoin%
\definecolor{currentfill}{rgb}{1.000000,1.000000,1.000000}%
\pgfsetfillcolor{currentfill}%
\pgfsetlinewidth{0.000000pt}%
\definecolor{currentstroke}{rgb}{0.000000,0.000000,0.000000}%
\pgfsetstrokecolor{currentstroke}%
\pgfsetstrokeopacity{0.000000}%
\pgfsetdash{}{0pt}%
\pgfpathmoveto{\pgfqpoint{2.302578in}{1.285444in}}%
\pgfpathlineto{\pgfqpoint{7.727578in}{1.285444in}}%
\pgfpathlineto{\pgfqpoint{7.727578in}{4.305444in}}%
\pgfpathlineto{\pgfqpoint{2.302578in}{4.305444in}}%
\pgfpathclose%
\pgfusepath{fill}%
\end{pgfscope}%
\begin{pgfscope}%
\pgfpathrectangle{\pgfqpoint{2.302578in}{1.285444in}}{\pgfqpoint{5.425000in}{3.020000in}} %
\pgfusepath{clip}%
\pgfsetroundcap%
\pgfsetroundjoin%
\pgfsetlinewidth{0.803000pt}%
\definecolor{currentstroke}{rgb}{0.800000,0.800000,0.800000}%
\pgfsetstrokecolor{currentstroke}%
\pgfsetdash{}{0pt}%
\pgfpathmoveto{\pgfqpoint{2.546701in}{1.285444in}}%
\pgfpathlineto{\pgfqpoint{2.546701in}{4.305444in}}%
\pgfusepath{stroke}%
\end{pgfscope}%
\begin{pgfscope}%
\definecolor{textcolor}{rgb}{0.150000,0.150000,0.150000}%
\pgfsetstrokecolor{textcolor}%
\pgfsetfillcolor{textcolor}%
\pgftext[x=2.546701in,y=1.121555in,,top]{\color{textcolor}\rmfamily\fontsize{36.000000}{43.200000}\selectfont 0}%
\end{pgfscope}%
\begin{pgfscope}%
\pgfpathrectangle{\pgfqpoint{2.302578in}{1.285444in}}{\pgfqpoint{5.425000in}{3.020000in}} %
\pgfusepath{clip}%
\pgfsetroundcap%
\pgfsetroundjoin%
\pgfsetlinewidth{0.803000pt}%
\definecolor{currentstroke}{rgb}{0.800000,0.800000,0.800000}%
\pgfsetstrokecolor{currentstroke}%
\pgfsetdash{}{0pt}%
\pgfpathmoveto{\pgfqpoint{5.013844in}{1.285444in}}%
\pgfpathlineto{\pgfqpoint{5.013844in}{4.305444in}}%
\pgfusepath{stroke}%
\end{pgfscope}%
\begin{pgfscope}%
\definecolor{textcolor}{rgb}{0.150000,0.150000,0.150000}%
\pgfsetstrokecolor{textcolor}%
\pgfsetfillcolor{textcolor}%
\pgftext[x=5.013844in,y=1.121555in,,top]{\color{textcolor}\rmfamily\fontsize{36.000000}{43.200000}\selectfont 1000}%
\end{pgfscope}%
\begin{pgfscope}%
\pgfpathrectangle{\pgfqpoint{2.302578in}{1.285444in}}{\pgfqpoint{5.425000in}{3.020000in}} %
\pgfusepath{clip}%
\pgfsetroundcap%
\pgfsetroundjoin%
\pgfsetlinewidth{0.803000pt}%
\definecolor{currentstroke}{rgb}{0.800000,0.800000,0.800000}%
\pgfsetstrokecolor{currentstroke}%
\pgfsetdash{}{0pt}%
\pgfpathmoveto{\pgfqpoint{7.480987in}{1.285444in}}%
\pgfpathlineto{\pgfqpoint{7.480987in}{4.305444in}}%
\pgfusepath{stroke}%
\end{pgfscope}%
\begin{pgfscope}%
\definecolor{textcolor}{rgb}{0.150000,0.150000,0.150000}%
\pgfsetstrokecolor{textcolor}%
\pgfsetfillcolor{textcolor}%
\pgftext[x=7.480987in,y=1.121555in,,top]{\color{textcolor}\rmfamily\fontsize{36.000000}{43.200000}\selectfont 2000}%
\end{pgfscope}%
\begin{pgfscope}%
\definecolor{textcolor}{rgb}{0.150000,0.150000,0.150000}%
\pgfsetstrokecolor{textcolor}%
\pgfsetfillcolor{textcolor}%
\pgftext[x=5.015078in,y=0.621500in,,top]{\color{textcolor}\rmfamily\fontsize{42.000000}{50.400000}\selectfont OpenML Workload}%
\end{pgfscope}%
\begin{pgfscope}%
\pgfpathrectangle{\pgfqpoint{2.302578in}{1.285444in}}{\pgfqpoint{5.425000in}{3.020000in}} %
\pgfusepath{clip}%
\pgfsetroundcap%
\pgfsetroundjoin%
\pgfsetlinewidth{0.803000pt}%
\definecolor{currentstroke}{rgb}{0.800000,0.800000,0.800000}%
\pgfsetstrokecolor{currentstroke}%
\pgfsetdash{}{0pt}%
\pgfpathmoveto{\pgfqpoint{2.302578in}{1.285444in}}%
\pgfpathlineto{\pgfqpoint{7.727578in}{1.285444in}}%
\pgfusepath{stroke}%
\end{pgfscope}%
\begin{pgfscope}%
\definecolor{textcolor}{rgb}{0.150000,0.150000,0.150000}%
\pgfsetstrokecolor{textcolor}%
\pgfsetfillcolor{textcolor}%
\pgftext[x=1.909189in,y=1.111944in,left,base]{\color{textcolor}\rmfamily\fontsize{36.000000}{43.200000}\selectfont 0}%
\end{pgfscope}%
\begin{pgfscope}%
\pgfpathrectangle{\pgfqpoint{2.302578in}{1.285444in}}{\pgfqpoint{5.425000in}{3.020000in}} %
\pgfusepath{clip}%
\pgfsetroundcap%
\pgfsetroundjoin%
\pgfsetlinewidth{0.803000pt}%
\definecolor{currentstroke}{rgb}{0.800000,0.800000,0.800000}%
\pgfsetstrokecolor{currentstroke}%
\pgfsetdash{}{0pt}%
\pgfpathmoveto{\pgfqpoint{2.302578in}{1.806134in}}%
\pgfpathlineto{\pgfqpoint{7.727578in}{1.806134in}}%
\pgfusepath{stroke}%
\end{pgfscope}%
\begin{pgfscope}%
\definecolor{textcolor}{rgb}{0.150000,0.150000,0.150000}%
\pgfsetstrokecolor{textcolor}%
\pgfsetfillcolor{textcolor}%
\pgftext[x=1.450189in,y=1.632634in,left,base]{\color{textcolor}\rmfamily\fontsize{36.000000}{43.200000}\selectfont 500}%
\end{pgfscope}%
\begin{pgfscope}%
\pgfpathrectangle{\pgfqpoint{2.302578in}{1.285444in}}{\pgfqpoint{5.425000in}{3.020000in}} %
\pgfusepath{clip}%
\pgfsetroundcap%
\pgfsetroundjoin%
\pgfsetlinewidth{0.803000pt}%
\definecolor{currentstroke}{rgb}{0.800000,0.800000,0.800000}%
\pgfsetstrokecolor{currentstroke}%
\pgfsetdash{}{0pt}%
\pgfpathmoveto{\pgfqpoint{2.302578in}{2.326823in}}%
\pgfpathlineto{\pgfqpoint{7.727578in}{2.326823in}}%
\pgfusepath{stroke}%
\end{pgfscope}%
\begin{pgfscope}%
\definecolor{textcolor}{rgb}{0.150000,0.150000,0.150000}%
\pgfsetstrokecolor{textcolor}%
\pgfsetfillcolor{textcolor}%
\pgftext[x=1.666689in,y=2.153323in,left,base]{\color{textcolor}\rmfamily\fontsize{36.000000}{43.200000}\selectfont 1k}%
\end{pgfscope}%
\begin{pgfscope}%
\pgfpathrectangle{\pgfqpoint{2.302578in}{1.285444in}}{\pgfqpoint{5.425000in}{3.020000in}} %
\pgfusepath{clip}%
\pgfsetroundcap%
\pgfsetroundjoin%
\pgfsetlinewidth{0.803000pt}%
\definecolor{currentstroke}{rgb}{0.800000,0.800000,0.800000}%
\pgfsetstrokecolor{currentstroke}%
\pgfsetdash{}{0pt}%
\pgfpathmoveto{\pgfqpoint{2.302578in}{2.847513in}}%
\pgfpathlineto{\pgfqpoint{7.727578in}{2.847513in}}%
\pgfusepath{stroke}%
\end{pgfscope}%
\begin{pgfscope}%
\definecolor{textcolor}{rgb}{0.150000,0.150000,0.150000}%
\pgfsetstrokecolor{textcolor}%
\pgfsetfillcolor{textcolor}%
\pgftext[x=1.312189in,y=2.674013in,left,base]{\color{textcolor}\rmfamily\fontsize{36.000000}{43.200000}\selectfont 1.5k}%
\end{pgfscope}%
\begin{pgfscope}%
\pgfpathrectangle{\pgfqpoint{2.302578in}{1.285444in}}{\pgfqpoint{5.425000in}{3.020000in}} %
\pgfusepath{clip}%
\pgfsetroundcap%
\pgfsetroundjoin%
\pgfsetlinewidth{0.803000pt}%
\definecolor{currentstroke}{rgb}{0.800000,0.800000,0.800000}%
\pgfsetstrokecolor{currentstroke}%
\pgfsetdash{}{0pt}%
\pgfpathmoveto{\pgfqpoint{2.302578in}{3.368203in}}%
\pgfpathlineto{\pgfqpoint{7.727578in}{3.368203in}}%
\pgfusepath{stroke}%
\end{pgfscope}%
\begin{pgfscope}%
\definecolor{textcolor}{rgb}{0.150000,0.150000,0.150000}%
\pgfsetstrokecolor{textcolor}%
\pgfsetfillcolor{textcolor}%
\pgftext[x=1.666689in,y=3.194703in,left,base]{\color{textcolor}\rmfamily\fontsize{36.000000}{43.200000}\selectfont 2k}%
\end{pgfscope}%
\begin{pgfscope}%
\pgfpathrectangle{\pgfqpoint{2.302578in}{1.285444in}}{\pgfqpoint{5.425000in}{3.020000in}} %
\pgfusepath{clip}%
\pgfsetroundcap%
\pgfsetroundjoin%
\pgfsetlinewidth{0.803000pt}%
\definecolor{currentstroke}{rgb}{0.800000,0.800000,0.800000}%
\pgfsetstrokecolor{currentstroke}%
\pgfsetdash{}{0pt}%
\pgfpathmoveto{\pgfqpoint{2.302578in}{3.888892in}}%
\pgfpathlineto{\pgfqpoint{7.727578in}{3.888892in}}%
\pgfusepath{stroke}%
\end{pgfscope}%
\begin{pgfscope}%
\definecolor{textcolor}{rgb}{0.150000,0.150000,0.150000}%
\pgfsetstrokecolor{textcolor}%
\pgfsetfillcolor{textcolor}%
\pgftext[x=1.312189in,y=3.715392in,left,base]{\color{textcolor}\rmfamily\fontsize{36.000000}{43.200000}\selectfont 2.5k}%
\end{pgfscope}%
\begin{pgfscope}%
\definecolor{textcolor}{rgb}{0.150000,0.150000,0.150000}%
\pgfsetstrokecolor{textcolor}%
\pgfsetfillcolor{textcolor}%
\pgftext[x=0.472750in,y=1.361611in,left,base,rotate=90.000000]{\color{textcolor}\rmfamily\fontsize{42.000000}{50.400000}\selectfont Cumulative }%
\end{pgfscope}%
\begin{pgfscope}%
\definecolor{textcolor}{rgb}{0.150000,0.150000,0.150000}%
\pgfsetstrokecolor{textcolor}%
\pgfsetfillcolor{textcolor}%
\pgftext[x=1.110800in,y=1.207902in,left,base,rotate=90.000000]{\color{textcolor}\rmfamily\fontsize{42.000000}{50.400000}\selectfont Run Time (s)}%
\end{pgfscope}%
\begin{pgfscope}%
\pgfpathrectangle{\pgfqpoint{2.302578in}{1.285444in}}{\pgfqpoint{5.425000in}{3.020000in}} %
\pgfusepath{clip}%
\pgfsetbuttcap%
\pgfsetroundjoin%
\pgfsetlinewidth{3.011250pt}%
\definecolor{currentstroke}{rgb}{0.298039,0.447059,0.690196}%
\pgfsetstrokecolor{currentstroke}%
\pgfsetdash{{3.000000pt}{0.000000pt}}{0.000000pt}%
\pgfpathmoveto{\pgfqpoint{2.549169in}{1.287083in}}%
\pgfpathlineto{\pgfqpoint{2.556570in}{1.291719in}}%
\pgfpathlineto{\pgfqpoint{2.559037in}{1.312155in}}%
\pgfpathlineto{\pgfqpoint{2.588643in}{1.330736in}}%
\pgfpathlineto{\pgfqpoint{2.593577in}{1.332853in}}%
\pgfpathlineto{\pgfqpoint{2.605913in}{1.343989in}}%
\pgfpathlineto{\pgfqpoint{2.608380in}{1.344040in}}%
\pgfpathlineto{\pgfqpoint{2.620716in}{1.353818in}}%
\pgfpathlineto{\pgfqpoint{2.625650in}{1.364138in}}%
\pgfpathlineto{\pgfqpoint{2.633051in}{1.369849in}}%
\pgfpathlineto{\pgfqpoint{3.471880in}{1.386911in}}%
\pgfpathlineto{\pgfqpoint{3.489150in}{1.388507in}}%
\pgfpathlineto{\pgfqpoint{3.491617in}{1.402249in}}%
\pgfpathlineto{\pgfqpoint{3.688988in}{1.436040in}}%
\pgfpathlineto{\pgfqpoint{3.735864in}{1.442797in}}%
\pgfpathlineto{\pgfqpoint{3.812346in}{1.454905in}}%
\pgfpathlineto{\pgfqpoint{3.829616in}{1.457643in}}%
\pgfpathlineto{\pgfqpoint{3.874024in}{1.463970in}}%
\pgfpathlineto{\pgfqpoint{3.920900in}{1.471805in}}%
\pgfpathlineto{\pgfqpoint{3.999848in}{1.483666in}}%
\pgfpathlineto{\pgfqpoint{4.017118in}{1.486244in}}%
\pgfpathlineto{\pgfqpoint{4.076330in}{1.495179in}}%
\pgfpathlineto{\pgfqpoint{4.105936in}{1.500435in}}%
\pgfpathlineto{\pgfqpoint{4.140476in}{1.506326in}}%
\pgfpathlineto{\pgfqpoint{4.271234in}{1.526224in}}%
\pgfpathlineto{\pgfqpoint{5.191478in}{1.666478in}}%
\pgfpathlineto{\pgfqpoint{5.233420in}{1.673203in}}%
\pgfpathlineto{\pgfqpoint{5.749053in}{1.752883in}}%
\pgfpathlineto{\pgfqpoint{5.983431in}{1.787665in}}%
\pgfpathlineto{\pgfqpoint{7.051704in}{1.948431in}}%
\pgfpathlineto{\pgfqpoint{7.088711in}{1.954188in}}%
\pgfpathlineto{\pgfqpoint{7.135587in}{1.961775in}}%
\pgfpathlineto{\pgfqpoint{7.157791in}{1.965440in}}%
\pgfpathlineto{\pgfqpoint{7.179995in}{1.968675in}}%
\pgfpathlineto{\pgfqpoint{7.182462in}{2.016829in}}%
\pgfpathlineto{\pgfqpoint{7.202200in}{2.020000in}}%
\pgfpathlineto{\pgfqpoint{7.480987in}{2.062569in}}%
\pgfpathlineto{\pgfqpoint{7.480987in}{2.062569in}}%
\pgfusepath{stroke}%
\end{pgfscope}%
\begin{pgfscope}%
\pgfpathrectangle{\pgfqpoint{2.302578in}{1.285444in}}{\pgfqpoint{5.425000in}{3.020000in}} %
\pgfusepath{clip}%
\pgfsetbuttcap%
\pgfsetroundjoin%
\pgfsetlinewidth{3.011250pt}%
\definecolor{currentstroke}{rgb}{0.866667,0.517647,0.321569}%
\pgfsetstrokecolor{currentstroke}%
\pgfsetdash{{9.000000pt}{3.000000pt}}{0.000000pt}%
\pgfpathmoveto{\pgfqpoint{2.549169in}{1.287192in}}%
\pgfpathlineto{\pgfqpoint{2.556570in}{1.296704in}}%
\pgfpathlineto{\pgfqpoint{2.559037in}{1.313268in}}%
\pgfpathlineto{\pgfqpoint{2.583709in}{1.345900in}}%
\pgfpathlineto{\pgfqpoint{2.598511in}{1.361542in}}%
\pgfpathlineto{\pgfqpoint{2.605913in}{1.371309in}}%
\pgfpathlineto{\pgfqpoint{2.615781in}{1.385101in}}%
\pgfpathlineto{\pgfqpoint{2.620716in}{1.390675in}}%
\pgfpathlineto{\pgfqpoint{2.625650in}{1.403103in}}%
\pgfpathlineto{\pgfqpoint{2.635519in}{1.413199in}}%
\pgfpathlineto{\pgfqpoint{2.761343in}{1.471322in}}%
\pgfpathlineto{\pgfqpoint{3.143750in}{1.648714in}}%
\pgfpathlineto{\pgfqpoint{3.220231in}{1.684441in}}%
\pgfpathlineto{\pgfqpoint{3.264640in}{1.705039in}}%
\pgfpathlineto{\pgfqpoint{3.484216in}{1.807645in}}%
\pgfpathlineto{\pgfqpoint{3.489150in}{1.810520in}}%
\pgfpathlineto{\pgfqpoint{3.491617in}{1.825411in}}%
\pgfpathlineto{\pgfqpoint{3.676653in}{1.938656in}}%
\pgfpathlineto{\pgfqpoint{3.753134in}{1.983500in}}%
\pgfpathlineto{\pgfqpoint{3.802477in}{2.013689in}}%
\pgfpathlineto{\pgfqpoint{3.901163in}{2.072416in}}%
\pgfpathlineto{\pgfqpoint{3.940637in}{2.095820in}}%
\pgfpathlineto{\pgfqpoint{4.786867in}{2.599721in}}%
\pgfpathlineto{\pgfqpoint{4.823874in}{2.621584in}}%
\pgfpathlineto{\pgfqpoint{5.073056in}{2.770309in}}%
\pgfpathlineto{\pgfqpoint{5.112530in}{2.793455in}}%
\pgfpathlineto{\pgfqpoint{5.191478in}{2.841489in}}%
\pgfpathlineto{\pgfqpoint{5.233420in}{2.866875in}}%
\pgfpathlineto{\pgfqpoint{5.554148in}{3.059480in}}%
\pgfpathlineto{\pgfqpoint{5.625695in}{3.101641in}}%
\pgfpathlineto{\pgfqpoint{5.645433in}{3.113334in}}%
\pgfpathlineto{\pgfqpoint{5.721914in}{3.160262in}}%
\pgfpathlineto{\pgfqpoint{5.776191in}{3.191432in}}%
\pgfpathlineto{\pgfqpoint{5.869943in}{3.246403in}}%
\pgfpathlineto{\pgfqpoint{5.877344in}{3.251392in}}%
\pgfpathlineto{\pgfqpoint{5.904483in}{3.267705in}}%
\pgfpathlineto{\pgfqpoint{5.931621in}{3.283488in}}%
\pgfpathlineto{\pgfqpoint{5.946424in}{3.292240in}}%
\pgfpathlineto{\pgfqpoint{6.030307in}{3.341766in}}%
\pgfpathlineto{\pgfqpoint{6.099387in}{3.382458in}}%
\pgfpathlineto{\pgfqpoint{7.044303in}{3.943813in}}%
\pgfpathlineto{\pgfqpoint{7.071441in}{3.960151in}}%
\pgfpathlineto{\pgfqpoint{7.105981in}{3.977304in}}%
\pgfpathlineto{\pgfqpoint{7.115850in}{3.979916in}}%
\pgfpathlineto{\pgfqpoint{7.170127in}{3.993593in}}%
\pgfpathlineto{\pgfqpoint{7.179995in}{3.996194in}}%
\pgfpathlineto{\pgfqpoint{7.182462in}{4.044557in}}%
\pgfpathlineto{\pgfqpoint{7.207134in}{4.051096in}}%
\pgfpathlineto{\pgfqpoint{7.384768in}{4.094459in}}%
\pgfpathlineto{\pgfqpoint{7.424242in}{4.104204in}}%
\pgfpathlineto{\pgfqpoint{7.480987in}{4.118990in}}%
\pgfpathlineto{\pgfqpoint{7.480987in}{4.118990in}}%
\pgfusepath{stroke}%
\end{pgfscope}%
\begin{pgfscope}%
\pgfsetrectcap%
\pgfsetmiterjoin%
\pgfsetlinewidth{1.003750pt}%
\definecolor{currentstroke}{rgb}{0.800000,0.800000,0.800000}%
\pgfsetstrokecolor{currentstroke}%
\pgfsetdash{}{0pt}%
\pgfpathmoveto{\pgfqpoint{2.302578in}{1.285444in}}%
\pgfpathlineto{\pgfqpoint{2.302578in}{4.305444in}}%
\pgfusepath{stroke}%
\end{pgfscope}%
\begin{pgfscope}%
\pgfsetrectcap%
\pgfsetmiterjoin%
\pgfsetlinewidth{1.003750pt}%
\definecolor{currentstroke}{rgb}{0.800000,0.800000,0.800000}%
\pgfsetstrokecolor{currentstroke}%
\pgfsetdash{}{0pt}%
\pgfpathmoveto{\pgfqpoint{7.727578in}{1.285444in}}%
\pgfpathlineto{\pgfqpoint{7.727578in}{4.305444in}}%
\pgfusepath{stroke}%
\end{pgfscope}%
\begin{pgfscope}%
\pgfsetrectcap%
\pgfsetmiterjoin%
\pgfsetlinewidth{1.003750pt}%
\definecolor{currentstroke}{rgb}{0.800000,0.800000,0.800000}%
\pgfsetstrokecolor{currentstroke}%
\pgfsetdash{}{0pt}%
\pgfpathmoveto{\pgfqpoint{2.302578in}{1.285444in}}%
\pgfpathlineto{\pgfqpoint{7.727578in}{1.285444in}}%
\pgfusepath{stroke}%
\end{pgfscope}%
\begin{pgfscope}%
\pgfsetrectcap%
\pgfsetmiterjoin%
\pgfsetlinewidth{1.003750pt}%
\definecolor{currentstroke}{rgb}{0.800000,0.800000,0.800000}%
\pgfsetstrokecolor{currentstroke}%
\pgfsetdash{}{0pt}%
\pgfpathmoveto{\pgfqpoint{2.302578in}{4.305444in}}%
\pgfpathlineto{\pgfqpoint{7.727578in}{4.305444in}}%
\pgfusepath{stroke}%
\end{pgfscope}%
\begin{pgfscope}%
\pgfsetbuttcap%
\pgfsetroundjoin%
\pgfsetlinewidth{4.015000pt}%
\definecolor{currentstroke}{rgb}{0.298039,0.447059,0.690196}%
\pgfsetstrokecolor{currentstroke}%
\pgfsetdash{{4.000000pt}{0.000000pt}}{0.000000pt}%
\pgfpathmoveto{\pgfqpoint{3.281578in}{4.526044in}}%
\pgfpathlineto{\pgfqpoint{4.031578in}{4.526044in}}%
\pgfusepath{stroke}%
\end{pgfscope}%
\begin{pgfscope}%
\definecolor{textcolor}{rgb}{0.150000,0.150000,0.150000}%
\pgfsetstrokecolor{textcolor}%
\pgfsetfillcolor{textcolor}%
\pgftext[x=4.081578in,y=4.351044in,left,base]{\color{textcolor}\rmfamily\fontsize{36.000000}{43.200000}\selectfont CO}%
\end{pgfscope}%
\begin{pgfscope}%
\pgfsetbuttcap%
\pgfsetroundjoin%
\pgfsetlinewidth{4.015000pt}%
\definecolor{currentstroke}{rgb}{0.866667,0.517647,0.321569}%
\pgfsetstrokecolor{currentstroke}%
\pgfsetdash{{12.000000pt}{4.000000pt}}{0.000000pt}%
\pgfpathmoveto{\pgfqpoint{4.875578in}{4.526044in}}%
\pgfpathlineto{\pgfqpoint{5.625578in}{4.526044in}}%
\pgfusepath{stroke}%
\end{pgfscope}%
\begin{pgfscope}%
\definecolor{textcolor}{rgb}{0.150000,0.150000,0.150000}%
\pgfsetstrokecolor{textcolor}%
\pgfsetfillcolor{textcolor}%
\pgftext[x=5.675578in,y=4.351044in,left,base]{\color{textcolor}\rmfamily\fontsize{36.000000}{43.200000}\selectfont OML}%
\end{pgfscope}%
\end{pgfpicture}%
\makeatother%
\endgroup%
%
}
\caption{Combined run-time}
\end{subfigure}%
\begin{subfigure}[b]{0.5\linewidth}
\centering
 \resizebox{\columnwidth}{!}{%
%% Creator: Matplotlib, PGF backend
%%
%% To include the figure in your LaTeX document, write
%%   \input{<filename>.pgf}
%%
%% Make sure the required packages are loaded in your preamble
%%   \usepackage{pgf}
%%
%% Figures using additional raster images can only be included by \input if
%% they are in the same directory as the main LaTeX file. For loading figures
%% from other directories you can use the `import` package
%%   \usepackage{import}
%% and then include the figures with
%%   \import{<path to file>}{<filename>.pgf}
%%
%% Matplotlib used the following preamble
%%   \usepackage{fontspec}
%%   \setmonofont{Andale Mono}
%%
\begingroup%
\makeatletter%
\begin{pgfpicture}%
\pgfpathrectangle{\pgfpointorigin}{\pgfqpoint{8.039987in}{5.001044in}}%
\pgfusepath{use as bounding box, clip}%
\begin{pgfscope}%
\pgfsetbuttcap%
\pgfsetmiterjoin%
\definecolor{currentfill}{rgb}{1.000000,1.000000,1.000000}%
\pgfsetfillcolor{currentfill}%
\pgfsetlinewidth{0.000000pt}%
\definecolor{currentstroke}{rgb}{1.000000,1.000000,1.000000}%
\pgfsetstrokecolor{currentstroke}%
\pgfsetdash{}{0pt}%
\pgfpathmoveto{\pgfqpoint{0.000000in}{0.000000in}}%
\pgfpathlineto{\pgfqpoint{8.039987in}{0.000000in}}%
\pgfpathlineto{\pgfqpoint{8.039987in}{5.001044in}}%
\pgfpathlineto{\pgfqpoint{0.000000in}{5.001044in}}%
\pgfpathclose%
\pgfusepath{fill}%
\end{pgfscope}%
\begin{pgfscope}%
\pgfsetbuttcap%
\pgfsetmiterjoin%
\definecolor{currentfill}{rgb}{1.000000,1.000000,1.000000}%
\pgfsetfillcolor{currentfill}%
\pgfsetlinewidth{0.000000pt}%
\definecolor{currentstroke}{rgb}{0.000000,0.000000,0.000000}%
\pgfsetstrokecolor{currentstroke}%
\pgfsetstrokeopacity{0.000000}%
\pgfsetdash{}{0pt}%
\pgfpathmoveto{\pgfqpoint{2.302578in}{1.285444in}}%
\pgfpathlineto{\pgfqpoint{7.727578in}{1.285444in}}%
\pgfpathlineto{\pgfqpoint{7.727578in}{4.305444in}}%
\pgfpathlineto{\pgfqpoint{2.302578in}{4.305444in}}%
\pgfpathclose%
\pgfusepath{fill}%
\end{pgfscope}%
\begin{pgfscope}%
\pgfpathrectangle{\pgfqpoint{2.302578in}{1.285444in}}{\pgfqpoint{5.425000in}{3.020000in}} %
\pgfusepath{clip}%
\pgfsetroundcap%
\pgfsetroundjoin%
\pgfsetlinewidth{0.803000pt}%
\definecolor{currentstroke}{rgb}{0.800000,0.800000,0.800000}%
\pgfsetstrokecolor{currentstroke}%
\pgfsetdash{}{0pt}%
\pgfpathmoveto{\pgfqpoint{2.546701in}{1.285444in}}%
\pgfpathlineto{\pgfqpoint{2.546701in}{4.305444in}}%
\pgfusepath{stroke}%
\end{pgfscope}%
\begin{pgfscope}%
\definecolor{textcolor}{rgb}{0.150000,0.150000,0.150000}%
\pgfsetstrokecolor{textcolor}%
\pgfsetfillcolor{textcolor}%
\pgftext[x=2.546701in,y=1.121555in,,top]{\color{textcolor}\rmfamily\fontsize{36.000000}{43.200000}\selectfont 0}%
\end{pgfscope}%
\begin{pgfscope}%
\pgfpathrectangle{\pgfqpoint{2.302578in}{1.285444in}}{\pgfqpoint{5.425000in}{3.020000in}} %
\pgfusepath{clip}%
\pgfsetroundcap%
\pgfsetroundjoin%
\pgfsetlinewidth{0.803000pt}%
\definecolor{currentstroke}{rgb}{0.800000,0.800000,0.800000}%
\pgfsetstrokecolor{currentstroke}%
\pgfsetdash{}{0pt}%
\pgfpathmoveto{\pgfqpoint{5.013844in}{1.285444in}}%
\pgfpathlineto{\pgfqpoint{5.013844in}{4.305444in}}%
\pgfusepath{stroke}%
\end{pgfscope}%
\begin{pgfscope}%
\definecolor{textcolor}{rgb}{0.150000,0.150000,0.150000}%
\pgfsetstrokecolor{textcolor}%
\pgfsetfillcolor{textcolor}%
\pgftext[x=5.013844in,y=1.121555in,,top]{\color{textcolor}\rmfamily\fontsize{36.000000}{43.200000}\selectfont 1000}%
\end{pgfscope}%
\begin{pgfscope}%
\pgfpathrectangle{\pgfqpoint{2.302578in}{1.285444in}}{\pgfqpoint{5.425000in}{3.020000in}} %
\pgfusepath{clip}%
\pgfsetroundcap%
\pgfsetroundjoin%
\pgfsetlinewidth{0.803000pt}%
\definecolor{currentstroke}{rgb}{0.800000,0.800000,0.800000}%
\pgfsetstrokecolor{currentstroke}%
\pgfsetdash{}{0pt}%
\pgfpathmoveto{\pgfqpoint{7.480987in}{1.285444in}}%
\pgfpathlineto{\pgfqpoint{7.480987in}{4.305444in}}%
\pgfusepath{stroke}%
\end{pgfscope}%
\begin{pgfscope}%
\definecolor{textcolor}{rgb}{0.150000,0.150000,0.150000}%
\pgfsetstrokecolor{textcolor}%
\pgfsetfillcolor{textcolor}%
\pgftext[x=7.480987in,y=1.121555in,,top]{\color{textcolor}\rmfamily\fontsize{36.000000}{43.200000}\selectfont 2000}%
\end{pgfscope}%
\begin{pgfscope}%
\definecolor{textcolor}{rgb}{0.150000,0.150000,0.150000}%
\pgfsetstrokecolor{textcolor}%
\pgfsetfillcolor{textcolor}%
\pgftext[x=5.015078in,y=0.621500in,,top]{\color{textcolor}\rmfamily\fontsize{42.000000}{50.400000}\selectfont OpenML Workload}%
\end{pgfscope}%
\begin{pgfscope}%
\pgfpathrectangle{\pgfqpoint{2.302578in}{1.285444in}}{\pgfqpoint{5.425000in}{3.020000in}} %
\pgfusepath{clip}%
\pgfsetroundcap%
\pgfsetroundjoin%
\pgfsetlinewidth{0.803000pt}%
\definecolor{currentstroke}{rgb}{0.800000,0.800000,0.800000}%
\pgfsetstrokecolor{currentstroke}%
\pgfsetdash{}{0pt}%
\pgfpathmoveto{\pgfqpoint{2.302578in}{1.285444in}}%
\pgfpathlineto{\pgfqpoint{7.727578in}{1.285444in}}%
\pgfusepath{stroke}%
\end{pgfscope}%
\begin{pgfscope}%
\definecolor{textcolor}{rgb}{0.150000,0.150000,0.150000}%
\pgfsetstrokecolor{textcolor}%
\pgfsetfillcolor{textcolor}%
\pgftext[x=1.909189in,y=1.111944in,left,base]{\color{textcolor}\rmfamily\fontsize{36.000000}{43.200000}\selectfont 0}%
\end{pgfscope}%
\begin{pgfscope}%
\pgfpathrectangle{\pgfqpoint{2.302578in}{1.285444in}}{\pgfqpoint{5.425000in}{3.020000in}} %
\pgfusepath{clip}%
\pgfsetroundcap%
\pgfsetroundjoin%
\pgfsetlinewidth{0.803000pt}%
\definecolor{currentstroke}{rgb}{0.800000,0.800000,0.800000}%
\pgfsetstrokecolor{currentstroke}%
\pgfsetdash{}{0pt}%
\pgfpathmoveto{\pgfqpoint{2.302578in}{1.806134in}}%
\pgfpathlineto{\pgfqpoint{7.727578in}{1.806134in}}%
\pgfusepath{stroke}%
\end{pgfscope}%
\begin{pgfscope}%
\definecolor{textcolor}{rgb}{0.150000,0.150000,0.150000}%
\pgfsetstrokecolor{textcolor}%
\pgfsetfillcolor{textcolor}%
\pgftext[x=1.450189in,y=1.632634in,left,base]{\color{textcolor}\rmfamily\fontsize{36.000000}{43.200000}\selectfont 500}%
\end{pgfscope}%
\begin{pgfscope}%
\pgfpathrectangle{\pgfqpoint{2.302578in}{1.285444in}}{\pgfqpoint{5.425000in}{3.020000in}} %
\pgfusepath{clip}%
\pgfsetroundcap%
\pgfsetroundjoin%
\pgfsetlinewidth{0.803000pt}%
\definecolor{currentstroke}{rgb}{0.800000,0.800000,0.800000}%
\pgfsetstrokecolor{currentstroke}%
\pgfsetdash{}{0pt}%
\pgfpathmoveto{\pgfqpoint{2.302578in}{2.326823in}}%
\pgfpathlineto{\pgfqpoint{7.727578in}{2.326823in}}%
\pgfusepath{stroke}%
\end{pgfscope}%
\begin{pgfscope}%
\definecolor{textcolor}{rgb}{0.150000,0.150000,0.150000}%
\pgfsetstrokecolor{textcolor}%
\pgfsetfillcolor{textcolor}%
\pgftext[x=1.666689in,y=2.153323in,left,base]{\color{textcolor}\rmfamily\fontsize{36.000000}{43.200000}\selectfont 1k}%
\end{pgfscope}%
\begin{pgfscope}%
\pgfpathrectangle{\pgfqpoint{2.302578in}{1.285444in}}{\pgfqpoint{5.425000in}{3.020000in}} %
\pgfusepath{clip}%
\pgfsetroundcap%
\pgfsetroundjoin%
\pgfsetlinewidth{0.803000pt}%
\definecolor{currentstroke}{rgb}{0.800000,0.800000,0.800000}%
\pgfsetstrokecolor{currentstroke}%
\pgfsetdash{}{0pt}%
\pgfpathmoveto{\pgfqpoint{2.302578in}{2.847513in}}%
\pgfpathlineto{\pgfqpoint{7.727578in}{2.847513in}}%
\pgfusepath{stroke}%
\end{pgfscope}%
\begin{pgfscope}%
\definecolor{textcolor}{rgb}{0.150000,0.150000,0.150000}%
\pgfsetstrokecolor{textcolor}%
\pgfsetfillcolor{textcolor}%
\pgftext[x=1.312189in,y=2.674013in,left,base]{\color{textcolor}\rmfamily\fontsize{36.000000}{43.200000}\selectfont 1.5k}%
\end{pgfscope}%
\begin{pgfscope}%
\pgfpathrectangle{\pgfqpoint{2.302578in}{1.285444in}}{\pgfqpoint{5.425000in}{3.020000in}} %
\pgfusepath{clip}%
\pgfsetroundcap%
\pgfsetroundjoin%
\pgfsetlinewidth{0.803000pt}%
\definecolor{currentstroke}{rgb}{0.800000,0.800000,0.800000}%
\pgfsetstrokecolor{currentstroke}%
\pgfsetdash{}{0pt}%
\pgfpathmoveto{\pgfqpoint{2.302578in}{3.368203in}}%
\pgfpathlineto{\pgfqpoint{7.727578in}{3.368203in}}%
\pgfusepath{stroke}%
\end{pgfscope}%
\begin{pgfscope}%
\definecolor{textcolor}{rgb}{0.150000,0.150000,0.150000}%
\pgfsetstrokecolor{textcolor}%
\pgfsetfillcolor{textcolor}%
\pgftext[x=1.666689in,y=3.194703in,left,base]{\color{textcolor}\rmfamily\fontsize{36.000000}{43.200000}\selectfont 2k}%
\end{pgfscope}%
\begin{pgfscope}%
\pgfpathrectangle{\pgfqpoint{2.302578in}{1.285444in}}{\pgfqpoint{5.425000in}{3.020000in}} %
\pgfusepath{clip}%
\pgfsetroundcap%
\pgfsetroundjoin%
\pgfsetlinewidth{0.803000pt}%
\definecolor{currentstroke}{rgb}{0.800000,0.800000,0.800000}%
\pgfsetstrokecolor{currentstroke}%
\pgfsetdash{}{0pt}%
\pgfpathmoveto{\pgfqpoint{2.302578in}{3.888892in}}%
\pgfpathlineto{\pgfqpoint{7.727578in}{3.888892in}}%
\pgfusepath{stroke}%
\end{pgfscope}%
\begin{pgfscope}%
\definecolor{textcolor}{rgb}{0.150000,0.150000,0.150000}%
\pgfsetstrokecolor{textcolor}%
\pgfsetfillcolor{textcolor}%
\pgftext[x=1.312189in,y=3.715392in,left,base]{\color{textcolor}\rmfamily\fontsize{36.000000}{43.200000}\selectfont 2.5k}%
\end{pgfscope}%
\begin{pgfscope}%
\definecolor{textcolor}{rgb}{0.150000,0.150000,0.150000}%
\pgfsetstrokecolor{textcolor}%
\pgfsetfillcolor{textcolor}%
\pgftext[x=0.472750in,y=1.361611in,left,base,rotate=90.000000]{\color{textcolor}\rmfamily\fontsize{42.000000}{50.400000}\selectfont Cumulative }%
\end{pgfscope}%
\begin{pgfscope}%
\definecolor{textcolor}{rgb}{0.150000,0.150000,0.150000}%
\pgfsetstrokecolor{textcolor}%
\pgfsetfillcolor{textcolor}%
\pgftext[x=1.110800in,y=1.207902in,left,base,rotate=90.000000]{\color{textcolor}\rmfamily\fontsize{42.000000}{50.400000}\selectfont Run Time (s)}%
\end{pgfscope}%
\begin{pgfscope}%
\pgfpathrectangle{\pgfqpoint{2.302578in}{1.285444in}}{\pgfqpoint{5.425000in}{3.020000in}} %
\pgfusepath{clip}%
\pgfsetbuttcap%
\pgfsetroundjoin%
\pgfsetlinewidth{3.011250pt}%
\definecolor{currentstroke}{rgb}{0.298039,0.447059,0.690196}%
\pgfsetstrokecolor{currentstroke}%
\pgfsetdash{{3.000000pt}{0.000000pt}}{0.000000pt}%
\pgfpathmoveto{\pgfqpoint{2.549169in}{1.287083in}}%
\pgfpathlineto{\pgfqpoint{2.677460in}{1.289529in}}%
\pgfpathlineto{\pgfqpoint{7.480987in}{1.353997in}}%
\pgfpathlineto{\pgfqpoint{7.480987in}{1.353997in}}%
\pgfusepath{stroke}%
\end{pgfscope}%
\begin{pgfscope}%
\pgfpathrectangle{\pgfqpoint{2.302578in}{1.285444in}}{\pgfqpoint{5.425000in}{3.020000in}} %
\pgfusepath{clip}%
\pgfsetbuttcap%
\pgfsetroundjoin%
\pgfsetlinewidth{3.011250pt}%
\definecolor{currentstroke}{rgb}{0.866667,0.517647,0.321569}%
\pgfsetstrokecolor{currentstroke}%
\pgfsetdash{{9.000000pt}{3.000000pt}}{0.000000pt}%
\pgfpathmoveto{\pgfqpoint{2.549169in}{1.287192in}}%
\pgfpathlineto{\pgfqpoint{2.600979in}{1.319141in}}%
\pgfpathlineto{\pgfqpoint{4.796736in}{2.302386in}}%
\pgfpathlineto{\pgfqpoint{5.124866in}{2.448575in}}%
\pgfpathlineto{\pgfqpoint{6.059913in}{2.865653in}}%
\pgfpathlineto{\pgfqpoint{6.200540in}{2.928620in}}%
\pgfpathlineto{\pgfqpoint{6.242481in}{2.947426in}}%
\pgfpathlineto{\pgfqpoint{6.370773in}{3.004253in}}%
\pgfpathlineto{\pgfqpoint{6.861734in}{3.223840in}}%
\pgfpathlineto{\pgfqpoint{7.017164in}{3.293342in}}%
\pgfpathlineto{\pgfqpoint{7.098580in}{3.328976in}}%
\pgfpathlineto{\pgfqpoint{7.480987in}{3.370665in}}%
\pgfpathlineto{\pgfqpoint{7.480987in}{3.370665in}}%
\pgfusepath{stroke}%
\end{pgfscope}%
\begin{pgfscope}%
\pgfsetrectcap%
\pgfsetmiterjoin%
\pgfsetlinewidth{1.003750pt}%
\definecolor{currentstroke}{rgb}{0.800000,0.800000,0.800000}%
\pgfsetstrokecolor{currentstroke}%
\pgfsetdash{}{0pt}%
\pgfpathmoveto{\pgfqpoint{2.302578in}{1.285444in}}%
\pgfpathlineto{\pgfqpoint{2.302578in}{4.305444in}}%
\pgfusepath{stroke}%
\end{pgfscope}%
\begin{pgfscope}%
\pgfsetrectcap%
\pgfsetmiterjoin%
\pgfsetlinewidth{1.003750pt}%
\definecolor{currentstroke}{rgb}{0.800000,0.800000,0.800000}%
\pgfsetstrokecolor{currentstroke}%
\pgfsetdash{}{0pt}%
\pgfpathmoveto{\pgfqpoint{7.727578in}{1.285444in}}%
\pgfpathlineto{\pgfqpoint{7.727578in}{4.305444in}}%
\pgfusepath{stroke}%
\end{pgfscope}%
\begin{pgfscope}%
\pgfsetrectcap%
\pgfsetmiterjoin%
\pgfsetlinewidth{1.003750pt}%
\definecolor{currentstroke}{rgb}{0.800000,0.800000,0.800000}%
\pgfsetstrokecolor{currentstroke}%
\pgfsetdash{}{0pt}%
\pgfpathmoveto{\pgfqpoint{2.302578in}{1.285444in}}%
\pgfpathlineto{\pgfqpoint{7.727578in}{1.285444in}}%
\pgfusepath{stroke}%
\end{pgfscope}%
\begin{pgfscope}%
\pgfsetrectcap%
\pgfsetmiterjoin%
\pgfsetlinewidth{1.003750pt}%
\definecolor{currentstroke}{rgb}{0.800000,0.800000,0.800000}%
\pgfsetstrokecolor{currentstroke}%
\pgfsetdash{}{0pt}%
\pgfpathmoveto{\pgfqpoint{2.302578in}{4.305444in}}%
\pgfpathlineto{\pgfqpoint{7.727578in}{4.305444in}}%
\pgfusepath{stroke}%
\end{pgfscope}%
\begin{pgfscope}%
\pgfsetbuttcap%
\pgfsetroundjoin%
\pgfsetlinewidth{4.015000pt}%
\definecolor{currentstroke}{rgb}{0.298039,0.447059,0.690196}%
\pgfsetstrokecolor{currentstroke}%
\pgfsetdash{{4.000000pt}{0.000000pt}}{0.000000pt}%
\pgfpathmoveto{\pgfqpoint{3.281578in}{4.526044in}}%
\pgfpathlineto{\pgfqpoint{4.031578in}{4.526044in}}%
\pgfusepath{stroke}%
\end{pgfscope}%
\begin{pgfscope}%
\definecolor{textcolor}{rgb}{0.150000,0.150000,0.150000}%
\pgfsetstrokecolor{textcolor}%
\pgfsetfillcolor{textcolor}%
\pgftext[x=4.081578in,y=4.351044in,left,base]{\color{textcolor}\rmfamily\fontsize{36.000000}{43.200000}\selectfont CO}%
\end{pgfscope}%
\begin{pgfscope}%
\pgfsetbuttcap%
\pgfsetroundjoin%
\pgfsetlinewidth{4.015000pt}%
\definecolor{currentstroke}{rgb}{0.866667,0.517647,0.321569}%
\pgfsetstrokecolor{currentstroke}%
\pgfsetdash{{12.000000pt}{4.000000pt}}{0.000000pt}%
\pgfpathmoveto{\pgfqpoint{4.875578in}{4.526044in}}%
\pgfpathlineto{\pgfqpoint{5.625578in}{4.526044in}}%
\pgfusepath{stroke}%
\end{pgfscope}%
\begin{pgfscope}%
\definecolor{textcolor}{rgb}{0.150000,0.150000,0.150000}%
\pgfsetstrokecolor{textcolor}%
\pgfsetfillcolor{textcolor}%
\pgftext[x=5.675578in,y=4.351044in,left,base]{\color{textcolor}\rmfamily\fontsize{36.000000}{43.200000}\selectfont OML}%
\end{pgfscope}%
\end{pgfpicture}%
\makeatother%
\endgroup%
%
}
\caption{Best pipeline run-time}
\end{subfigure}
\begin{subfigure}[b]{\linewidth}
\centering
 \resizebox{\columnwidth}{!}{%
%% Creator: Matplotlib, PGF backend
%%
%% To include the figure in your LaTeX document, write
%%   \input{<filename>.pgf}
%%
%% Make sure the required packages are loaded in your preamble
%%   \usepackage{pgf}
%%
%% Figures using additional raster images can only be included by \input if
%% they are in the same directory as the main LaTeX file. For loading figures
%% from other directories you can use the `import` package
%%   \usepackage{import}
%% and then include the figures with
%%   \import{<path to file>}{<filename>.pgf}
%%
%% Matplotlib used the following preamble
%%   \usepackage{fontspec}
%%   \setmonofont{Andale Mono}
%%
\begingroup%
\makeatletter%
\begin{pgfpicture}%
\pgfpathrectangle{\pgfpointorigin}{\pgfqpoint{10.512444in}{4.241111in}}%
\pgfusepath{use as bounding box, clip}%
\begin{pgfscope}%
\pgfsetbuttcap%
\pgfsetmiterjoin%
\definecolor{currentfill}{rgb}{1.000000,1.000000,1.000000}%
\pgfsetfillcolor{currentfill}%
\pgfsetlinewidth{0.000000pt}%
\definecolor{currentstroke}{rgb}{1.000000,1.000000,1.000000}%
\pgfsetstrokecolor{currentstroke}%
\pgfsetdash{}{0pt}%
\pgfpathmoveto{\pgfqpoint{0.000000in}{0.000000in}}%
\pgfpathlineto{\pgfqpoint{10.512444in}{0.000000in}}%
\pgfpathlineto{\pgfqpoint{10.512444in}{4.241111in}}%
\pgfpathlineto{\pgfqpoint{0.000000in}{4.241111in}}%
\pgfpathclose%
\pgfusepath{fill}%
\end{pgfscope}%
\begin{pgfscope}%
\pgfsetbuttcap%
\pgfsetmiterjoin%
\definecolor{currentfill}{rgb}{1.000000,1.000000,1.000000}%
\pgfsetfillcolor{currentfill}%
\pgfsetlinewidth{0.000000pt}%
\definecolor{currentstroke}{rgb}{0.000000,0.000000,0.000000}%
\pgfsetstrokecolor{currentstroke}%
\pgfsetstrokeopacity{0.000000}%
\pgfsetdash{}{0pt}%
\pgfpathmoveto{\pgfqpoint{1.623736in}{1.000625in}}%
\pgfpathlineto{\pgfqpoint{8.598736in}{1.000625in}}%
\pgfpathlineto{\pgfqpoint{8.598736in}{4.020625in}}%
\pgfpathlineto{\pgfqpoint{1.623736in}{4.020625in}}%
\pgfpathclose%
\pgfusepath{fill}%
\end{pgfscope}%
\begin{pgfscope}%
\pgfpathrectangle{\pgfqpoint{1.623736in}{1.000625in}}{\pgfqpoint{6.975000in}{3.020000in}} %
\pgfusepath{clip}%
\pgfsetroundcap%
\pgfsetroundjoin%
\pgfsetlinewidth{0.803000pt}%
\definecolor{currentstroke}{rgb}{0.800000,0.800000,0.800000}%
\pgfsetstrokecolor{currentstroke}%
\pgfsetdash{}{0pt}%
\pgfpathmoveto{\pgfqpoint{1.937609in}{1.000625in}}%
\pgfpathlineto{\pgfqpoint{1.937609in}{4.020625in}}%
\pgfusepath{stroke}%
\end{pgfscope}%
\begin{pgfscope}%
\definecolor{textcolor}{rgb}{0.150000,0.150000,0.150000}%
\pgfsetstrokecolor{textcolor}%
\pgfsetfillcolor{textcolor}%
\pgftext[x=1.937609in,y=0.836736in,,top]{\color{textcolor}\rmfamily\fontsize{25.000000}{30.000000}\selectfont 0}%
\end{pgfscope}%
\begin{pgfscope}%
\pgfpathrectangle{\pgfqpoint{1.623736in}{1.000625in}}{\pgfqpoint{6.975000in}{3.020000in}} %
\pgfusepath{clip}%
\pgfsetroundcap%
\pgfsetroundjoin%
\pgfsetlinewidth{0.803000pt}%
\definecolor{currentstroke}{rgb}{0.800000,0.800000,0.800000}%
\pgfsetstrokecolor{currentstroke}%
\pgfsetdash{}{0pt}%
\pgfpathmoveto{\pgfqpoint{3.523630in}{1.000625in}}%
\pgfpathlineto{\pgfqpoint{3.523630in}{4.020625in}}%
\pgfusepath{stroke}%
\end{pgfscope}%
\begin{pgfscope}%
\definecolor{textcolor}{rgb}{0.150000,0.150000,0.150000}%
\pgfsetstrokecolor{textcolor}%
\pgfsetfillcolor{textcolor}%
\pgftext[x=3.523630in,y=0.836736in,,top]{\color{textcolor}\rmfamily\fontsize{25.000000}{30.000000}\selectfont 500}%
\end{pgfscope}%
\begin{pgfscope}%
\pgfpathrectangle{\pgfqpoint{1.623736in}{1.000625in}}{\pgfqpoint{6.975000in}{3.020000in}} %
\pgfusepath{clip}%
\pgfsetroundcap%
\pgfsetroundjoin%
\pgfsetlinewidth{0.803000pt}%
\definecolor{currentstroke}{rgb}{0.800000,0.800000,0.800000}%
\pgfsetstrokecolor{currentstroke}%
\pgfsetdash{}{0pt}%
\pgfpathmoveto{\pgfqpoint{5.109650in}{1.000625in}}%
\pgfpathlineto{\pgfqpoint{5.109650in}{4.020625in}}%
\pgfusepath{stroke}%
\end{pgfscope}%
\begin{pgfscope}%
\definecolor{textcolor}{rgb}{0.150000,0.150000,0.150000}%
\pgfsetstrokecolor{textcolor}%
\pgfsetfillcolor{textcolor}%
\pgftext[x=5.109650in,y=0.836736in,,top]{\color{textcolor}\rmfamily\fontsize{25.000000}{30.000000}\selectfont 1000}%
\end{pgfscope}%
\begin{pgfscope}%
\pgfpathrectangle{\pgfqpoint{1.623736in}{1.000625in}}{\pgfqpoint{6.975000in}{3.020000in}} %
\pgfusepath{clip}%
\pgfsetroundcap%
\pgfsetroundjoin%
\pgfsetlinewidth{0.803000pt}%
\definecolor{currentstroke}{rgb}{0.800000,0.800000,0.800000}%
\pgfsetstrokecolor{currentstroke}%
\pgfsetdash{}{0pt}%
\pgfpathmoveto{\pgfqpoint{6.695670in}{1.000625in}}%
\pgfpathlineto{\pgfqpoint{6.695670in}{4.020625in}}%
\pgfusepath{stroke}%
\end{pgfscope}%
\begin{pgfscope}%
\definecolor{textcolor}{rgb}{0.150000,0.150000,0.150000}%
\pgfsetstrokecolor{textcolor}%
\pgfsetfillcolor{textcolor}%
\pgftext[x=6.695670in,y=0.836736in,,top]{\color{textcolor}\rmfamily\fontsize{25.000000}{30.000000}\selectfont 1500}%
\end{pgfscope}%
\begin{pgfscope}%
\pgfpathrectangle{\pgfqpoint{1.623736in}{1.000625in}}{\pgfqpoint{6.975000in}{3.020000in}} %
\pgfusepath{clip}%
\pgfsetroundcap%
\pgfsetroundjoin%
\pgfsetlinewidth{0.803000pt}%
\definecolor{currentstroke}{rgb}{0.800000,0.800000,0.800000}%
\pgfsetstrokecolor{currentstroke}%
\pgfsetdash{}{0pt}%
\pgfpathmoveto{\pgfqpoint{8.281690in}{1.000625in}}%
\pgfpathlineto{\pgfqpoint{8.281690in}{4.020625in}}%
\pgfusepath{stroke}%
\end{pgfscope}%
\begin{pgfscope}%
\definecolor{textcolor}{rgb}{0.150000,0.150000,0.150000}%
\pgfsetstrokecolor{textcolor}%
\pgfsetfillcolor{textcolor}%
\pgftext[x=8.281690in,y=0.836736in,,top]{\color{textcolor}\rmfamily\fontsize{25.000000}{30.000000}\selectfont 2000}%
\end{pgfscope}%
\begin{pgfscope}%
\definecolor{textcolor}{rgb}{0.150000,0.150000,0.150000}%
\pgfsetstrokecolor{textcolor}%
\pgfsetfillcolor{textcolor}%
\pgftext[x=5.111236in,y=0.472500in,,top]{\color{textcolor}\rmfamily\fontsize{30.000000}{36.000000}\selectfont OpenML Workload}%
\end{pgfscope}%
\begin{pgfscope}%
\pgfpathrectangle{\pgfqpoint{1.623736in}{1.000625in}}{\pgfqpoint{6.975000in}{3.020000in}} %
\pgfusepath{clip}%
\pgfsetroundcap%
\pgfsetroundjoin%
\pgfsetlinewidth{0.803000pt}%
\definecolor{currentstroke}{rgb}{0.800000,0.800000,0.800000}%
\pgfsetstrokecolor{currentstroke}%
\pgfsetdash{}{0pt}%
\pgfpathmoveto{\pgfqpoint{1.623736in}{1.126818in}}%
\pgfpathlineto{\pgfqpoint{8.598736in}{1.126818in}}%
\pgfusepath{stroke}%
\end{pgfscope}%
\begin{pgfscope}%
\definecolor{textcolor}{rgb}{0.150000,0.150000,0.150000}%
\pgfsetstrokecolor{textcolor}%
\pgfsetfillcolor{textcolor}%
\pgftext[x=1.300472in,y=1.006332in,left,base]{\color{textcolor}\rmfamily\fontsize{25.000000}{30.000000}\selectfont 0}%
\end{pgfscope}%
\begin{pgfscope}%
\pgfpathrectangle{\pgfqpoint{1.623736in}{1.000625in}}{\pgfqpoint{6.975000in}{3.020000in}} %
\pgfusepath{clip}%
\pgfsetroundcap%
\pgfsetroundjoin%
\pgfsetlinewidth{0.803000pt}%
\definecolor{currentstroke}{rgb}{0.800000,0.800000,0.800000}%
\pgfsetstrokecolor{currentstroke}%
\pgfsetdash{}{0pt}%
\pgfpathmoveto{\pgfqpoint{1.623736in}{1.850270in}}%
\pgfpathlineto{\pgfqpoint{8.598736in}{1.850270in}}%
\pgfusepath{stroke}%
\end{pgfscope}%
\begin{pgfscope}%
\definecolor{textcolor}{rgb}{0.150000,0.150000,0.150000}%
\pgfsetstrokecolor{textcolor}%
\pgfsetfillcolor{textcolor}%
\pgftext[x=1.141097in,y=1.729784in,left,base]{\color{textcolor}\rmfamily\fontsize{25.000000}{30.000000}\selectfont 50}%
\end{pgfscope}%
\begin{pgfscope}%
\pgfpathrectangle{\pgfqpoint{1.623736in}{1.000625in}}{\pgfqpoint{6.975000in}{3.020000in}} %
\pgfusepath{clip}%
\pgfsetroundcap%
\pgfsetroundjoin%
\pgfsetlinewidth{0.803000pt}%
\definecolor{currentstroke}{rgb}{0.800000,0.800000,0.800000}%
\pgfsetstrokecolor{currentstroke}%
\pgfsetdash{}{0pt}%
\pgfpathmoveto{\pgfqpoint{1.623736in}{2.573721in}}%
\pgfpathlineto{\pgfqpoint{8.598736in}{2.573721in}}%
\pgfusepath{stroke}%
\end{pgfscope}%
\begin{pgfscope}%
\definecolor{textcolor}{rgb}{0.150000,0.150000,0.150000}%
\pgfsetstrokecolor{textcolor}%
\pgfsetfillcolor{textcolor}%
\pgftext[x=0.981722in,y=2.453235in,left,base]{\color{textcolor}\rmfamily\fontsize{25.000000}{30.000000}\selectfont 100}%
\end{pgfscope}%
\begin{pgfscope}%
\pgfpathrectangle{\pgfqpoint{1.623736in}{1.000625in}}{\pgfqpoint{6.975000in}{3.020000in}} %
\pgfusepath{clip}%
\pgfsetroundcap%
\pgfsetroundjoin%
\pgfsetlinewidth{0.803000pt}%
\definecolor{currentstroke}{rgb}{0.800000,0.800000,0.800000}%
\pgfsetstrokecolor{currentstroke}%
\pgfsetdash{}{0pt}%
\pgfpathmoveto{\pgfqpoint{1.623736in}{3.297173in}}%
\pgfpathlineto{\pgfqpoint{8.598736in}{3.297173in}}%
\pgfusepath{stroke}%
\end{pgfscope}%
\begin{pgfscope}%
\definecolor{textcolor}{rgb}{0.150000,0.150000,0.150000}%
\pgfsetstrokecolor{textcolor}%
\pgfsetfillcolor{textcolor}%
\pgftext[x=0.981722in,y=3.176687in,left,base]{\color{textcolor}\rmfamily\fontsize{25.000000}{30.000000}\selectfont 150}%
\end{pgfscope}%
\begin{pgfscope}%
\pgfpathrectangle{\pgfqpoint{1.623736in}{1.000625in}}{\pgfqpoint{6.975000in}{3.020000in}} %
\pgfusepath{clip}%
\pgfsetroundcap%
\pgfsetroundjoin%
\pgfsetlinewidth{0.803000pt}%
\definecolor{currentstroke}{rgb}{0.800000,0.800000,0.800000}%
\pgfsetstrokecolor{currentstroke}%
\pgfsetdash{}{0pt}%
\pgfpathmoveto{\pgfqpoint{1.623736in}{4.020625in}}%
\pgfpathlineto{\pgfqpoint{8.598736in}{4.020625in}}%
\pgfusepath{stroke}%
\end{pgfscope}%
\begin{pgfscope}%
\definecolor{textcolor}{rgb}{0.150000,0.150000,0.150000}%
\pgfsetstrokecolor{textcolor}%
\pgfsetfillcolor{textcolor}%
\pgftext[x=0.981722in,y=3.900138in,left,base]{\color{textcolor}\rmfamily\fontsize{25.000000}{30.000000}\selectfont 200}%
\end{pgfscope}%
\begin{pgfscope}%
\definecolor{textcolor}{rgb}{0.150000,0.150000,0.150000}%
\pgfsetstrokecolor{textcolor}%
\pgfsetfillcolor{textcolor}%
\pgftext[x=0.366250in,y=1.262708in,left,base,rotate=90.000000]{\color{textcolor}\rmfamily\fontsize{30.000000}{36.000000}\selectfont Δ Cumulative }%
\end{pgfscope}%
\begin{pgfscope}%
\definecolor{textcolor}{rgb}{0.150000,0.150000,0.150000}%
\pgfsetstrokecolor{textcolor}%
\pgfsetfillcolor{textcolor}%
\pgftext[x=0.822000in,y=1.376666in,left,base,rotate=90.000000]{\color{textcolor}\rmfamily\fontsize{30.000000}{36.000000}\selectfont Run Time (s)}%
\end{pgfscope}%
\begin{pgfscope}%
\pgfpathrectangle{\pgfqpoint{1.623736in}{1.000625in}}{\pgfqpoint{6.975000in}{3.020000in}} %
\pgfusepath{clip}%
\pgfsetbuttcap%
\pgfsetroundjoin%
\definecolor{currentfill}{rgb}{0.866667,0.517647,0.321569}%
\pgfsetfillcolor{currentfill}%
\pgfsetfillopacity{0.200000}%
\pgfsetlinewidth{0.803000pt}%
\definecolor{currentstroke}{rgb}{0.866667,0.517647,0.321569}%
\pgfsetstrokecolor{currentstroke}%
\pgfsetstrokeopacity{0.200000}%
\pgfsetdash{}{0pt}%
\pgfpathmoveto{\pgfqpoint{1.940781in}{1.129465in}}%
\pgfpathlineto{\pgfqpoint{1.940781in}{1.126115in}}%
\pgfpathlineto{\pgfqpoint{1.943953in}{1.125967in}}%
\pgfpathlineto{\pgfqpoint{1.947125in}{1.126079in}}%
\pgfpathlineto{\pgfqpoint{1.950297in}{1.142332in}}%
\pgfpathlineto{\pgfqpoint{1.953469in}{1.166491in}}%
\pgfpathlineto{\pgfqpoint{1.956641in}{1.186411in}}%
\pgfpathlineto{\pgfqpoint{1.959814in}{1.206265in}}%
\pgfpathlineto{\pgfqpoint{1.962986in}{1.226705in}}%
\pgfpathlineto{\pgfqpoint{1.966158in}{1.249386in}}%
\pgfpathlineto{\pgfqpoint{1.969330in}{1.269295in}}%
\pgfpathlineto{\pgfqpoint{1.972502in}{1.287660in}}%
\pgfpathlineto{\pgfqpoint{1.975674in}{1.306299in}}%
\pgfpathlineto{\pgfqpoint{1.978846in}{1.323682in}}%
\pgfpathlineto{\pgfqpoint{1.982018in}{1.342862in}}%
\pgfpathlineto{\pgfqpoint{1.985190in}{1.361900in}}%
\pgfpathlineto{\pgfqpoint{1.988362in}{1.374433in}}%
\pgfpathlineto{\pgfqpoint{1.991534in}{1.385673in}}%
\pgfpathlineto{\pgfqpoint{1.994706in}{1.397443in}}%
\pgfpathlineto{\pgfqpoint{1.997878in}{1.409170in}}%
\pgfpathlineto{\pgfqpoint{2.001050in}{1.420537in}}%
\pgfpathlineto{\pgfqpoint{2.004222in}{1.432130in}}%
\pgfpathlineto{\pgfqpoint{2.007394in}{1.442727in}}%
\pgfpathlineto{\pgfqpoint{2.010566in}{1.454214in}}%
\pgfpathlineto{\pgfqpoint{2.013738in}{1.465992in}}%
\pgfpathlineto{\pgfqpoint{2.016910in}{1.477980in}}%
\pgfpathlineto{\pgfqpoint{2.020082in}{1.489115in}}%
\pgfpathlineto{\pgfqpoint{2.023254in}{1.499862in}}%
\pgfpathlineto{\pgfqpoint{2.026426in}{1.510146in}}%
\pgfpathlineto{\pgfqpoint{2.029598in}{1.521413in}}%
\pgfpathlineto{\pgfqpoint{2.032770in}{1.532048in}}%
\pgfpathlineto{\pgfqpoint{2.035943in}{1.542777in}}%
\pgfpathlineto{\pgfqpoint{2.039115in}{1.554252in}}%
\pgfpathlineto{\pgfqpoint{2.042287in}{1.565396in}}%
\pgfpathlineto{\pgfqpoint{2.045459in}{1.576867in}}%
\pgfpathlineto{\pgfqpoint{2.048631in}{1.588803in}}%
\pgfpathlineto{\pgfqpoint{2.051803in}{1.600461in}}%
\pgfpathlineto{\pgfqpoint{2.054975in}{1.611999in}}%
\pgfpathlineto{\pgfqpoint{2.058147in}{1.623413in}}%
\pgfpathlineto{\pgfqpoint{2.061319in}{1.635179in}}%
\pgfpathlineto{\pgfqpoint{2.064491in}{1.646739in}}%
\pgfpathlineto{\pgfqpoint{2.067663in}{1.658386in}}%
\pgfpathlineto{\pgfqpoint{2.070835in}{1.669816in}}%
\pgfpathlineto{\pgfqpoint{2.074007in}{1.681352in}}%
\pgfpathlineto{\pgfqpoint{2.077179in}{1.693439in}}%
\pgfpathlineto{\pgfqpoint{2.080351in}{1.705190in}}%
\pgfpathlineto{\pgfqpoint{2.083523in}{1.716808in}}%
\pgfpathlineto{\pgfqpoint{2.086695in}{1.728704in}}%
\pgfpathlineto{\pgfqpoint{2.089867in}{1.740764in}}%
\pgfpathlineto{\pgfqpoint{2.093039in}{1.752583in}}%
\pgfpathlineto{\pgfqpoint{2.096211in}{1.764569in}}%
\pgfpathlineto{\pgfqpoint{2.099383in}{1.775935in}}%
\pgfpathlineto{\pgfqpoint{2.102555in}{1.787685in}}%
\pgfpathlineto{\pgfqpoint{2.105727in}{1.800773in}}%
\pgfpathlineto{\pgfqpoint{2.108899in}{1.812412in}}%
\pgfpathlineto{\pgfqpoint{2.112071in}{1.824774in}}%
\pgfpathlineto{\pgfqpoint{2.115244in}{1.836620in}}%
\pgfpathlineto{\pgfqpoint{2.118416in}{1.847886in}}%
\pgfpathlineto{\pgfqpoint{2.121588in}{1.859627in}}%
\pgfpathlineto{\pgfqpoint{2.124760in}{1.871499in}}%
\pgfpathlineto{\pgfqpoint{2.127932in}{1.883250in}}%
\pgfpathlineto{\pgfqpoint{2.131104in}{1.895105in}}%
\pgfpathlineto{\pgfqpoint{2.134276in}{1.906771in}}%
\pgfpathlineto{\pgfqpoint{2.137448in}{1.918595in}}%
\pgfpathlineto{\pgfqpoint{2.140620in}{1.930285in}}%
\pgfpathlineto{\pgfqpoint{2.143792in}{1.941802in}}%
\pgfpathlineto{\pgfqpoint{2.146964in}{1.952990in}}%
\pgfpathlineto{\pgfqpoint{2.150136in}{1.964553in}}%
\pgfpathlineto{\pgfqpoint{2.153308in}{1.976533in}}%
\pgfpathlineto{\pgfqpoint{2.156480in}{1.988043in}}%
\pgfpathlineto{\pgfqpoint{2.159652in}{1.999755in}}%
\pgfpathlineto{\pgfqpoint{2.162824in}{2.011119in}}%
\pgfpathlineto{\pgfqpoint{2.165996in}{2.022800in}}%
\pgfpathlineto{\pgfqpoint{2.169168in}{2.034947in}}%
\pgfpathlineto{\pgfqpoint{2.172340in}{2.046462in}}%
\pgfpathlineto{\pgfqpoint{2.175512in}{2.057952in}}%
\pgfpathlineto{\pgfqpoint{2.178684in}{2.069672in}}%
\pgfpathlineto{\pgfqpoint{2.181856in}{2.081421in}}%
\pgfpathlineto{\pgfqpoint{2.185028in}{2.093199in}}%
\pgfpathlineto{\pgfqpoint{2.188200in}{2.104711in}}%
\pgfpathlineto{\pgfqpoint{2.191372in}{2.116263in}}%
\pgfpathlineto{\pgfqpoint{2.194545in}{2.128432in}}%
\pgfpathlineto{\pgfqpoint{2.197717in}{2.139789in}}%
\pgfpathlineto{\pgfqpoint{2.200889in}{2.151538in}}%
\pgfpathlineto{\pgfqpoint{2.204061in}{2.163109in}}%
\pgfpathlineto{\pgfqpoint{2.207233in}{2.174796in}}%
\pgfpathlineto{\pgfqpoint{2.210405in}{2.186737in}}%
\pgfpathlineto{\pgfqpoint{2.213577in}{2.198261in}}%
\pgfpathlineto{\pgfqpoint{2.216749in}{2.210012in}}%
\pgfpathlineto{\pgfqpoint{2.219921in}{2.221705in}}%
\pgfpathlineto{\pgfqpoint{2.223093in}{2.233406in}}%
\pgfpathlineto{\pgfqpoint{2.226265in}{2.245038in}}%
\pgfpathlineto{\pgfqpoint{2.229437in}{2.256425in}}%
\pgfpathlineto{\pgfqpoint{2.232609in}{2.268408in}}%
\pgfpathlineto{\pgfqpoint{2.235781in}{2.279966in}}%
\pgfpathlineto{\pgfqpoint{2.238953in}{2.291540in}}%
\pgfpathlineto{\pgfqpoint{2.242125in}{2.303256in}}%
\pgfpathlineto{\pgfqpoint{2.245297in}{2.314903in}}%
\pgfpathlineto{\pgfqpoint{2.248469in}{2.326562in}}%
\pgfpathlineto{\pgfqpoint{2.251641in}{2.338141in}}%
\pgfpathlineto{\pgfqpoint{2.254813in}{2.349598in}}%
\pgfpathlineto{\pgfqpoint{2.257985in}{2.361469in}}%
\pgfpathlineto{\pgfqpoint{2.261157in}{2.373395in}}%
\pgfpathlineto{\pgfqpoint{2.264329in}{2.384987in}}%
\pgfpathlineto{\pgfqpoint{2.267501in}{2.397050in}}%
\pgfpathlineto{\pgfqpoint{2.270674in}{2.408819in}}%
\pgfpathlineto{\pgfqpoint{2.273846in}{2.420182in}}%
\pgfpathlineto{\pgfqpoint{2.277018in}{2.431652in}}%
\pgfpathlineto{\pgfqpoint{2.280190in}{2.443360in}}%
\pgfpathlineto{\pgfqpoint{2.283362in}{2.455374in}}%
\pgfpathlineto{\pgfqpoint{2.286534in}{2.467155in}}%
\pgfpathlineto{\pgfqpoint{2.289706in}{2.478538in}}%
\pgfpathlineto{\pgfqpoint{2.292878in}{2.490104in}}%
\pgfpathlineto{\pgfqpoint{2.296050in}{2.501444in}}%
\pgfpathlineto{\pgfqpoint{2.299222in}{2.512760in}}%
\pgfpathlineto{\pgfqpoint{2.302394in}{2.524282in}}%
\pgfpathlineto{\pgfqpoint{2.305566in}{2.535923in}}%
\pgfpathlineto{\pgfqpoint{2.308738in}{2.547155in}}%
\pgfpathlineto{\pgfqpoint{2.311910in}{2.558716in}}%
\pgfpathlineto{\pgfqpoint{2.315082in}{2.570207in}}%
\pgfpathlineto{\pgfqpoint{2.318254in}{2.581598in}}%
\pgfpathlineto{\pgfqpoint{2.321426in}{2.593541in}}%
\pgfpathlineto{\pgfqpoint{2.324598in}{2.605096in}}%
\pgfpathlineto{\pgfqpoint{2.327770in}{2.616377in}}%
\pgfpathlineto{\pgfqpoint{2.330942in}{2.628096in}}%
\pgfpathlineto{\pgfqpoint{2.334114in}{2.639640in}}%
\pgfpathlineto{\pgfqpoint{2.337286in}{2.651338in}}%
\pgfpathlineto{\pgfqpoint{2.340458in}{2.662774in}}%
\pgfpathlineto{\pgfqpoint{2.343630in}{2.673908in}}%
\pgfpathlineto{\pgfqpoint{2.346802in}{2.686076in}}%
\pgfpathlineto{\pgfqpoint{2.349975in}{2.697537in}}%
\pgfpathlineto{\pgfqpoint{2.353147in}{2.708841in}}%
\pgfpathlineto{\pgfqpoint{2.356319in}{2.720670in}}%
\pgfpathlineto{\pgfqpoint{2.359491in}{2.732288in}}%
\pgfpathlineto{\pgfqpoint{2.362663in}{2.743884in}}%
\pgfpathlineto{\pgfqpoint{2.365835in}{2.755434in}}%
\pgfpathlineto{\pgfqpoint{2.369007in}{2.767297in}}%
\pgfpathlineto{\pgfqpoint{2.372179in}{2.779153in}}%
\pgfpathlineto{\pgfqpoint{2.375351in}{2.791127in}}%
\pgfpathlineto{\pgfqpoint{2.378523in}{2.803187in}}%
\pgfpathlineto{\pgfqpoint{2.381695in}{2.814963in}}%
\pgfpathlineto{\pgfqpoint{2.384867in}{2.826435in}}%
\pgfpathlineto{\pgfqpoint{2.388039in}{2.838303in}}%
\pgfpathlineto{\pgfqpoint{2.391211in}{2.849849in}}%
\pgfpathlineto{\pgfqpoint{2.394383in}{2.861277in}}%
\pgfpathlineto{\pgfqpoint{2.397555in}{2.873198in}}%
\pgfpathlineto{\pgfqpoint{2.400727in}{2.885288in}}%
\pgfpathlineto{\pgfqpoint{2.403899in}{2.897313in}}%
\pgfpathlineto{\pgfqpoint{2.407071in}{2.909026in}}%
\pgfpathlineto{\pgfqpoint{2.410243in}{2.920223in}}%
\pgfpathlineto{\pgfqpoint{2.413415in}{2.932257in}}%
\pgfpathlineto{\pgfqpoint{2.416587in}{2.943767in}}%
\pgfpathlineto{\pgfqpoint{2.419759in}{2.955389in}}%
\pgfpathlineto{\pgfqpoint{2.422931in}{2.967107in}}%
\pgfpathlineto{\pgfqpoint{2.426103in}{2.978833in}}%
\pgfpathlineto{\pgfqpoint{2.429276in}{2.990432in}}%
\pgfpathlineto{\pgfqpoint{2.432448in}{3.001811in}}%
\pgfpathlineto{\pgfqpoint{2.435620in}{3.013159in}}%
\pgfpathlineto{\pgfqpoint{2.438792in}{3.024181in}}%
\pgfpathlineto{\pgfqpoint{2.441964in}{3.035880in}}%
\pgfpathlineto{\pgfqpoint{2.445136in}{3.047715in}}%
\pgfpathlineto{\pgfqpoint{2.448308in}{3.059266in}}%
\pgfpathlineto{\pgfqpoint{2.451480in}{3.069688in}}%
\pgfpathlineto{\pgfqpoint{2.454652in}{3.068011in}}%
\pgfpathlineto{\pgfqpoint{2.457824in}{3.065727in}}%
\pgfpathlineto{\pgfqpoint{2.460996in}{3.063097in}}%
\pgfpathlineto{\pgfqpoint{2.464168in}{3.063069in}}%
\pgfpathlineto{\pgfqpoint{2.467340in}{3.063338in}}%
\pgfpathlineto{\pgfqpoint{2.470512in}{3.063639in}}%
\pgfpathlineto{\pgfqpoint{2.473684in}{3.063857in}}%
\pgfpathlineto{\pgfqpoint{2.476856in}{3.063836in}}%
\pgfpathlineto{\pgfqpoint{2.480028in}{3.064007in}}%
\pgfpathlineto{\pgfqpoint{2.483200in}{3.064003in}}%
\pgfpathlineto{\pgfqpoint{2.486372in}{3.064046in}}%
\pgfpathlineto{\pgfqpoint{2.489544in}{3.064190in}}%
\pgfpathlineto{\pgfqpoint{2.492716in}{3.064017in}}%
\pgfpathlineto{\pgfqpoint{2.495888in}{3.064069in}}%
\pgfpathlineto{\pgfqpoint{2.499060in}{3.064191in}}%
\pgfpathlineto{\pgfqpoint{2.502232in}{3.064366in}}%
\pgfpathlineto{\pgfqpoint{2.505405in}{3.064354in}}%
\pgfpathlineto{\pgfqpoint{2.508577in}{3.064350in}}%
\pgfpathlineto{\pgfqpoint{2.511749in}{3.064312in}}%
\pgfpathlineto{\pgfqpoint{2.514921in}{3.064372in}}%
\pgfpathlineto{\pgfqpoint{2.518093in}{3.064389in}}%
\pgfpathlineto{\pgfqpoint{2.521265in}{3.064401in}}%
\pgfpathlineto{\pgfqpoint{2.524437in}{3.064168in}}%
\pgfpathlineto{\pgfqpoint{2.527609in}{3.064207in}}%
\pgfpathlineto{\pgfqpoint{2.530781in}{3.064179in}}%
\pgfpathlineto{\pgfqpoint{2.533953in}{3.064136in}}%
\pgfpathlineto{\pgfqpoint{2.537125in}{3.063937in}}%
\pgfpathlineto{\pgfqpoint{2.540297in}{3.063792in}}%
\pgfpathlineto{\pgfqpoint{2.543469in}{3.063805in}}%
\pgfpathlineto{\pgfqpoint{2.546641in}{3.063915in}}%
\pgfpathlineto{\pgfqpoint{2.549813in}{3.063925in}}%
\pgfpathlineto{\pgfqpoint{2.552985in}{3.063942in}}%
\pgfpathlineto{\pgfqpoint{2.556157in}{3.064057in}}%
\pgfpathlineto{\pgfqpoint{2.559329in}{3.064224in}}%
\pgfpathlineto{\pgfqpoint{2.562501in}{3.064592in}}%
\pgfpathlineto{\pgfqpoint{2.565673in}{3.064503in}}%
\pgfpathlineto{\pgfqpoint{2.568845in}{3.065024in}}%
\pgfpathlineto{\pgfqpoint{2.572017in}{3.065083in}}%
\pgfpathlineto{\pgfqpoint{2.575189in}{3.065066in}}%
\pgfpathlineto{\pgfqpoint{2.578361in}{3.065125in}}%
\pgfpathlineto{\pgfqpoint{2.581533in}{3.065204in}}%
\pgfpathlineto{\pgfqpoint{2.584706in}{3.065210in}}%
\pgfpathlineto{\pgfqpoint{2.587878in}{3.065227in}}%
\pgfpathlineto{\pgfqpoint{2.591050in}{3.065360in}}%
\pgfpathlineto{\pgfqpoint{2.594222in}{3.065414in}}%
\pgfpathlineto{\pgfqpoint{2.597394in}{3.065188in}}%
\pgfpathlineto{\pgfqpoint{2.600566in}{3.065407in}}%
\pgfpathlineto{\pgfqpoint{2.603738in}{3.065393in}}%
\pgfpathlineto{\pgfqpoint{2.606910in}{3.065430in}}%
\pgfpathlineto{\pgfqpoint{2.610082in}{3.065375in}}%
\pgfpathlineto{\pgfqpoint{2.613254in}{3.065391in}}%
\pgfpathlineto{\pgfqpoint{2.616426in}{3.065320in}}%
\pgfpathlineto{\pgfqpoint{2.619598in}{3.065224in}}%
\pgfpathlineto{\pgfqpoint{2.622770in}{3.065174in}}%
\pgfpathlineto{\pgfqpoint{2.625942in}{3.065458in}}%
\pgfpathlineto{\pgfqpoint{2.629114in}{3.065228in}}%
\pgfpathlineto{\pgfqpoint{2.632286in}{3.065108in}}%
\pgfpathlineto{\pgfqpoint{2.635458in}{3.065176in}}%
\pgfpathlineto{\pgfqpoint{2.638630in}{3.065173in}}%
\pgfpathlineto{\pgfqpoint{2.641802in}{3.065165in}}%
\pgfpathlineto{\pgfqpoint{2.644974in}{3.065074in}}%
\pgfpathlineto{\pgfqpoint{2.648146in}{3.065256in}}%
\pgfpathlineto{\pgfqpoint{2.651318in}{3.065076in}}%
\pgfpathlineto{\pgfqpoint{2.654490in}{3.065150in}}%
\pgfpathlineto{\pgfqpoint{2.657662in}{3.065202in}}%
\pgfpathlineto{\pgfqpoint{2.660834in}{3.065194in}}%
\pgfpathlineto{\pgfqpoint{2.664007in}{3.065239in}}%
\pgfpathlineto{\pgfqpoint{2.667179in}{3.065440in}}%
\pgfpathlineto{\pgfqpoint{2.670351in}{3.065474in}}%
\pgfpathlineto{\pgfqpoint{2.673523in}{3.065417in}}%
\pgfpathlineto{\pgfqpoint{2.676695in}{3.065435in}}%
\pgfpathlineto{\pgfqpoint{2.679867in}{3.065644in}}%
\pgfpathlineto{\pgfqpoint{2.683039in}{3.065936in}}%
\pgfpathlineto{\pgfqpoint{2.686211in}{3.066008in}}%
\pgfpathlineto{\pgfqpoint{2.689383in}{3.066033in}}%
\pgfpathlineto{\pgfqpoint{2.692555in}{3.066046in}}%
\pgfpathlineto{\pgfqpoint{2.695727in}{3.066113in}}%
\pgfpathlineto{\pgfqpoint{2.698899in}{3.066138in}}%
\pgfpathlineto{\pgfqpoint{2.702071in}{3.066038in}}%
\pgfpathlineto{\pgfqpoint{2.705243in}{3.066075in}}%
\pgfpathlineto{\pgfqpoint{2.708415in}{3.066292in}}%
\pgfpathlineto{\pgfqpoint{2.711587in}{3.066334in}}%
\pgfpathlineto{\pgfqpoint{2.714759in}{3.066231in}}%
\pgfpathlineto{\pgfqpoint{2.717931in}{3.066336in}}%
\pgfpathlineto{\pgfqpoint{2.721103in}{3.066258in}}%
\pgfpathlineto{\pgfqpoint{2.724275in}{3.066278in}}%
\pgfpathlineto{\pgfqpoint{2.727447in}{3.066244in}}%
\pgfpathlineto{\pgfqpoint{2.730619in}{3.066259in}}%
\pgfpathlineto{\pgfqpoint{2.733791in}{3.066077in}}%
\pgfpathlineto{\pgfqpoint{2.736963in}{3.065966in}}%
\pgfpathlineto{\pgfqpoint{2.740136in}{3.065771in}}%
\pgfpathlineto{\pgfqpoint{2.743308in}{3.065735in}}%
\pgfpathlineto{\pgfqpoint{2.746480in}{3.065603in}}%
\pgfpathlineto{\pgfqpoint{2.749652in}{3.065347in}}%
\pgfpathlineto{\pgfqpoint{2.752824in}{3.065472in}}%
\pgfpathlineto{\pgfqpoint{2.755996in}{3.065367in}}%
\pgfpathlineto{\pgfqpoint{2.759168in}{3.065157in}}%
\pgfpathlineto{\pgfqpoint{2.762340in}{3.064866in}}%
\pgfpathlineto{\pgfqpoint{2.765512in}{3.064735in}}%
\pgfpathlineto{\pgfqpoint{2.768684in}{3.064793in}}%
\pgfpathlineto{\pgfqpoint{2.771856in}{3.064993in}}%
\pgfpathlineto{\pgfqpoint{2.775028in}{3.064967in}}%
\pgfpathlineto{\pgfqpoint{2.778200in}{3.064918in}}%
\pgfpathlineto{\pgfqpoint{2.781372in}{3.065012in}}%
\pgfpathlineto{\pgfqpoint{2.784544in}{3.064971in}}%
\pgfpathlineto{\pgfqpoint{2.787716in}{3.065033in}}%
\pgfpathlineto{\pgfqpoint{2.790888in}{3.064958in}}%
\pgfpathlineto{\pgfqpoint{2.794060in}{3.064988in}}%
\pgfpathlineto{\pgfqpoint{2.797232in}{3.064934in}}%
\pgfpathlineto{\pgfqpoint{2.800404in}{3.065069in}}%
\pgfpathlineto{\pgfqpoint{2.803576in}{3.064878in}}%
\pgfpathlineto{\pgfqpoint{2.806748in}{3.064869in}}%
\pgfpathlineto{\pgfqpoint{2.809920in}{3.065125in}}%
\pgfpathlineto{\pgfqpoint{2.813092in}{3.065438in}}%
\pgfpathlineto{\pgfqpoint{2.816264in}{3.065378in}}%
\pgfpathlineto{\pgfqpoint{2.819437in}{3.065193in}}%
\pgfpathlineto{\pgfqpoint{2.822609in}{3.065259in}}%
\pgfpathlineto{\pgfqpoint{2.825781in}{3.065543in}}%
\pgfpathlineto{\pgfqpoint{2.828953in}{3.065681in}}%
\pgfpathlineto{\pgfqpoint{2.832125in}{3.065582in}}%
\pgfpathlineto{\pgfqpoint{2.835297in}{3.065437in}}%
\pgfpathlineto{\pgfqpoint{2.838469in}{3.065629in}}%
\pgfpathlineto{\pgfqpoint{2.841641in}{3.065756in}}%
\pgfpathlineto{\pgfqpoint{2.844813in}{3.065724in}}%
\pgfpathlineto{\pgfqpoint{2.847985in}{3.065538in}}%
\pgfpathlineto{\pgfqpoint{2.851157in}{3.065346in}}%
\pgfpathlineto{\pgfqpoint{2.854329in}{3.065237in}}%
\pgfpathlineto{\pgfqpoint{2.857501in}{3.065096in}}%
\pgfpathlineto{\pgfqpoint{2.860673in}{3.065253in}}%
\pgfpathlineto{\pgfqpoint{2.863845in}{3.065327in}}%
\pgfpathlineto{\pgfqpoint{2.867017in}{3.065241in}}%
\pgfpathlineto{\pgfqpoint{2.870189in}{3.065290in}}%
\pgfpathlineto{\pgfqpoint{2.873361in}{3.065286in}}%
\pgfpathlineto{\pgfqpoint{2.876533in}{3.065265in}}%
\pgfpathlineto{\pgfqpoint{2.879705in}{3.065209in}}%
\pgfpathlineto{\pgfqpoint{2.882877in}{3.065276in}}%
\pgfpathlineto{\pgfqpoint{2.886049in}{3.065198in}}%
\pgfpathlineto{\pgfqpoint{2.889221in}{3.065222in}}%
\pgfpathlineto{\pgfqpoint{2.892393in}{3.065137in}}%
\pgfpathlineto{\pgfqpoint{2.895565in}{3.065159in}}%
\pgfpathlineto{\pgfqpoint{2.898738in}{3.065096in}}%
\pgfpathlineto{\pgfqpoint{2.901910in}{3.065122in}}%
\pgfpathlineto{\pgfqpoint{2.905082in}{3.064973in}}%
\pgfpathlineto{\pgfqpoint{2.908254in}{3.064969in}}%
\pgfpathlineto{\pgfqpoint{2.911426in}{3.064926in}}%
\pgfpathlineto{\pgfqpoint{2.914598in}{3.064774in}}%
\pgfpathlineto{\pgfqpoint{2.917770in}{3.064847in}}%
\pgfpathlineto{\pgfqpoint{2.920942in}{3.064957in}}%
\pgfpathlineto{\pgfqpoint{2.924114in}{3.064940in}}%
\pgfpathlineto{\pgfqpoint{2.927286in}{3.065053in}}%
\pgfpathlineto{\pgfqpoint{2.930458in}{3.065022in}}%
\pgfpathlineto{\pgfqpoint{2.933630in}{3.064924in}}%
\pgfpathlineto{\pgfqpoint{2.936802in}{3.064793in}}%
\pgfpathlineto{\pgfqpoint{2.939974in}{3.064993in}}%
\pgfpathlineto{\pgfqpoint{2.943146in}{3.064995in}}%
\pgfpathlineto{\pgfqpoint{2.946318in}{3.064814in}}%
\pgfpathlineto{\pgfqpoint{2.949490in}{3.064464in}}%
\pgfpathlineto{\pgfqpoint{2.952662in}{3.064356in}}%
\pgfpathlineto{\pgfqpoint{2.955834in}{3.064432in}}%
\pgfpathlineto{\pgfqpoint{2.959006in}{3.063951in}}%
\pgfpathlineto{\pgfqpoint{2.962178in}{3.063914in}}%
\pgfpathlineto{\pgfqpoint{2.965350in}{3.063808in}}%
\pgfpathlineto{\pgfqpoint{2.968522in}{3.063743in}}%
\pgfpathlineto{\pgfqpoint{2.971694in}{3.063630in}}%
\pgfpathlineto{\pgfqpoint{2.974867in}{3.063581in}}%
\pgfpathlineto{\pgfqpoint{2.978039in}{3.063649in}}%
\pgfpathlineto{\pgfqpoint{2.981211in}{3.063654in}}%
\pgfpathlineto{\pgfqpoint{2.984383in}{3.063589in}}%
\pgfpathlineto{\pgfqpoint{2.987555in}{3.063688in}}%
\pgfpathlineto{\pgfqpoint{2.990727in}{3.063658in}}%
\pgfpathlineto{\pgfqpoint{2.993899in}{3.063665in}}%
\pgfpathlineto{\pgfqpoint{2.997071in}{3.063733in}}%
\pgfpathlineto{\pgfqpoint{3.000243in}{3.063997in}}%
\pgfpathlineto{\pgfqpoint{3.003415in}{3.064005in}}%
\pgfpathlineto{\pgfqpoint{3.006587in}{3.063982in}}%
\pgfpathlineto{\pgfqpoint{3.009759in}{3.063916in}}%
\pgfpathlineto{\pgfqpoint{3.012931in}{3.063936in}}%
\pgfpathlineto{\pgfqpoint{3.016103in}{3.063909in}}%
\pgfpathlineto{\pgfqpoint{3.019275in}{3.063631in}}%
\pgfpathlineto{\pgfqpoint{3.022447in}{3.064022in}}%
\pgfpathlineto{\pgfqpoint{3.025619in}{3.064207in}}%
\pgfpathlineto{\pgfqpoint{3.028791in}{3.064188in}}%
\pgfpathlineto{\pgfqpoint{3.031963in}{3.064168in}}%
\pgfpathlineto{\pgfqpoint{3.035135in}{3.064208in}}%
\pgfpathlineto{\pgfqpoint{3.038307in}{3.064230in}}%
\pgfpathlineto{\pgfqpoint{3.041479in}{3.064333in}}%
\pgfpathlineto{\pgfqpoint{3.044651in}{3.064402in}}%
\pgfpathlineto{\pgfqpoint{3.047823in}{3.064609in}}%
\pgfpathlineto{\pgfqpoint{3.050995in}{3.064979in}}%
\pgfpathlineto{\pgfqpoint{3.054168in}{3.064959in}}%
\pgfpathlineto{\pgfqpoint{3.057340in}{3.064830in}}%
\pgfpathlineto{\pgfqpoint{3.060512in}{3.064888in}}%
\pgfpathlineto{\pgfqpoint{3.063684in}{3.064798in}}%
\pgfpathlineto{\pgfqpoint{3.066856in}{3.064707in}}%
\pgfpathlineto{\pgfqpoint{3.070028in}{3.064401in}}%
\pgfpathlineto{\pgfqpoint{3.073200in}{3.064326in}}%
\pgfpathlineto{\pgfqpoint{3.076372in}{3.064375in}}%
\pgfpathlineto{\pgfqpoint{3.079544in}{3.064338in}}%
\pgfpathlineto{\pgfqpoint{3.082716in}{3.064429in}}%
\pgfpathlineto{\pgfqpoint{3.085888in}{3.064540in}}%
\pgfpathlineto{\pgfqpoint{3.089060in}{3.064611in}}%
\pgfpathlineto{\pgfqpoint{3.092232in}{3.064455in}}%
\pgfpathlineto{\pgfqpoint{3.095404in}{3.064444in}}%
\pgfpathlineto{\pgfqpoint{3.098576in}{3.064587in}}%
\pgfpathlineto{\pgfqpoint{3.101748in}{3.064610in}}%
\pgfpathlineto{\pgfqpoint{3.104920in}{3.064249in}}%
\pgfpathlineto{\pgfqpoint{3.108092in}{3.064161in}}%
\pgfpathlineto{\pgfqpoint{3.111264in}{3.064281in}}%
\pgfpathlineto{\pgfqpoint{3.114436in}{3.064289in}}%
\pgfpathlineto{\pgfqpoint{3.117608in}{3.064062in}}%
\pgfpathlineto{\pgfqpoint{3.120780in}{3.064028in}}%
\pgfpathlineto{\pgfqpoint{3.123952in}{3.064121in}}%
\pgfpathlineto{\pgfqpoint{3.127124in}{3.063982in}}%
\pgfpathlineto{\pgfqpoint{3.130297in}{3.063839in}}%
\pgfpathlineto{\pgfqpoint{3.133469in}{3.063904in}}%
\pgfpathlineto{\pgfqpoint{3.136641in}{3.064094in}}%
\pgfpathlineto{\pgfqpoint{3.139813in}{3.064098in}}%
\pgfpathlineto{\pgfqpoint{3.142985in}{3.063920in}}%
\pgfpathlineto{\pgfqpoint{3.146157in}{3.063564in}}%
\pgfpathlineto{\pgfqpoint{3.149329in}{3.063836in}}%
\pgfpathlineto{\pgfqpoint{3.152501in}{3.066090in}}%
\pgfpathlineto{\pgfqpoint{3.155673in}{3.066147in}}%
\pgfpathlineto{\pgfqpoint{3.158845in}{3.066232in}}%
\pgfpathlineto{\pgfqpoint{3.162017in}{3.066090in}}%
\pgfpathlineto{\pgfqpoint{3.165189in}{3.066043in}}%
\pgfpathlineto{\pgfqpoint{3.168361in}{3.065241in}}%
\pgfpathlineto{\pgfqpoint{3.171533in}{3.064985in}}%
\pgfpathlineto{\pgfqpoint{3.174705in}{3.064193in}}%
\pgfpathlineto{\pgfqpoint{3.177877in}{3.064663in}}%
\pgfpathlineto{\pgfqpoint{3.181049in}{3.065000in}}%
\pgfpathlineto{\pgfqpoint{3.184221in}{3.065078in}}%
\pgfpathlineto{\pgfqpoint{3.187393in}{3.065805in}}%
\pgfpathlineto{\pgfqpoint{3.190565in}{3.065735in}}%
\pgfpathlineto{\pgfqpoint{3.193737in}{3.065649in}}%
\pgfpathlineto{\pgfqpoint{3.196909in}{3.065265in}}%
\pgfpathlineto{\pgfqpoint{3.200081in}{3.065418in}}%
\pgfpathlineto{\pgfqpoint{3.203253in}{3.064902in}}%
\pgfpathlineto{\pgfqpoint{3.206425in}{3.064695in}}%
\pgfpathlineto{\pgfqpoint{3.209598in}{3.065255in}}%
\pgfpathlineto{\pgfqpoint{3.212770in}{3.066096in}}%
\pgfpathlineto{\pgfqpoint{3.215942in}{3.066062in}}%
\pgfpathlineto{\pgfqpoint{3.219114in}{3.066343in}}%
\pgfpathlineto{\pgfqpoint{3.222286in}{3.066292in}}%
\pgfpathlineto{\pgfqpoint{3.225458in}{3.066502in}}%
\pgfpathlineto{\pgfqpoint{3.228630in}{3.066324in}}%
\pgfpathlineto{\pgfqpoint{3.231802in}{3.066419in}}%
\pgfpathlineto{\pgfqpoint{3.234974in}{3.065991in}}%
\pgfpathlineto{\pgfqpoint{3.238146in}{3.065647in}}%
\pgfpathlineto{\pgfqpoint{3.241318in}{3.065971in}}%
\pgfpathlineto{\pgfqpoint{3.244490in}{3.064577in}}%
\pgfpathlineto{\pgfqpoint{3.247662in}{3.064629in}}%
\pgfpathlineto{\pgfqpoint{3.250834in}{3.064915in}}%
\pgfpathlineto{\pgfqpoint{3.254006in}{3.065210in}}%
\pgfpathlineto{\pgfqpoint{3.257178in}{3.065108in}}%
\pgfpathlineto{\pgfqpoint{3.260350in}{3.064864in}}%
\pgfpathlineto{\pgfqpoint{3.263522in}{3.064833in}}%
\pgfpathlineto{\pgfqpoint{3.266694in}{3.064653in}}%
\pgfpathlineto{\pgfqpoint{3.269866in}{3.064882in}}%
\pgfpathlineto{\pgfqpoint{3.273038in}{3.064721in}}%
\pgfpathlineto{\pgfqpoint{3.276210in}{3.064846in}}%
\pgfpathlineto{\pgfqpoint{3.279382in}{3.064880in}}%
\pgfpathlineto{\pgfqpoint{3.282554in}{3.065449in}}%
\pgfpathlineto{\pgfqpoint{3.285726in}{3.065288in}}%
\pgfpathlineto{\pgfqpoint{3.288899in}{3.065734in}}%
\pgfpathlineto{\pgfqpoint{3.292071in}{3.065410in}}%
\pgfpathlineto{\pgfqpoint{3.295243in}{3.064751in}}%
\pgfpathlineto{\pgfqpoint{3.298415in}{3.064777in}}%
\pgfpathlineto{\pgfqpoint{3.301587in}{3.064322in}}%
\pgfpathlineto{\pgfqpoint{3.304759in}{3.063431in}}%
\pgfpathlineto{\pgfqpoint{3.307931in}{3.063256in}}%
\pgfpathlineto{\pgfqpoint{3.311103in}{3.062934in}}%
\pgfpathlineto{\pgfqpoint{3.314275in}{3.063438in}}%
\pgfpathlineto{\pgfqpoint{3.317447in}{3.063528in}}%
\pgfpathlineto{\pgfqpoint{3.320619in}{3.063789in}}%
\pgfpathlineto{\pgfqpoint{3.323791in}{3.063434in}}%
\pgfpathlineto{\pgfqpoint{3.326963in}{3.063436in}}%
\pgfpathlineto{\pgfqpoint{3.330135in}{3.063216in}}%
\pgfpathlineto{\pgfqpoint{3.333307in}{3.063163in}}%
\pgfpathlineto{\pgfqpoint{3.336479in}{3.063386in}}%
\pgfpathlineto{\pgfqpoint{3.339651in}{3.063598in}}%
\pgfpathlineto{\pgfqpoint{3.342823in}{3.063913in}}%
\pgfpathlineto{\pgfqpoint{3.345995in}{3.064032in}}%
\pgfpathlineto{\pgfqpoint{3.349167in}{3.063271in}}%
\pgfpathlineto{\pgfqpoint{3.352339in}{3.063114in}}%
\pgfpathlineto{\pgfqpoint{3.355511in}{3.062962in}}%
\pgfpathlineto{\pgfqpoint{3.358683in}{3.063049in}}%
\pgfpathlineto{\pgfqpoint{3.361855in}{3.062999in}}%
\pgfpathlineto{\pgfqpoint{3.365028in}{3.063072in}}%
\pgfpathlineto{\pgfqpoint{3.368200in}{3.063200in}}%
\pgfpathlineto{\pgfqpoint{3.371372in}{3.062917in}}%
\pgfpathlineto{\pgfqpoint{3.374544in}{3.063282in}}%
\pgfpathlineto{\pgfqpoint{3.377716in}{3.063509in}}%
\pgfpathlineto{\pgfqpoint{3.380888in}{3.063477in}}%
\pgfpathlineto{\pgfqpoint{3.384060in}{3.063771in}}%
\pgfpathlineto{\pgfqpoint{3.387232in}{3.063440in}}%
\pgfpathlineto{\pgfqpoint{3.390404in}{3.063073in}}%
\pgfpathlineto{\pgfqpoint{3.393576in}{3.063184in}}%
\pgfpathlineto{\pgfqpoint{3.396748in}{3.062939in}}%
\pgfpathlineto{\pgfqpoint{3.399920in}{3.063066in}}%
\pgfpathlineto{\pgfqpoint{3.403092in}{3.063036in}}%
\pgfpathlineto{\pgfqpoint{3.406264in}{3.063217in}}%
\pgfpathlineto{\pgfqpoint{3.409436in}{3.063573in}}%
\pgfpathlineto{\pgfqpoint{3.412608in}{3.063830in}}%
\pgfpathlineto{\pgfqpoint{3.415780in}{3.064256in}}%
\pgfpathlineto{\pgfqpoint{3.418952in}{3.064684in}}%
\pgfpathlineto{\pgfqpoint{3.422124in}{3.064742in}}%
\pgfpathlineto{\pgfqpoint{3.425296in}{3.064546in}}%
\pgfpathlineto{\pgfqpoint{3.428468in}{3.064633in}}%
\pgfpathlineto{\pgfqpoint{3.431640in}{3.064682in}}%
\pgfpathlineto{\pgfqpoint{3.434812in}{3.064567in}}%
\pgfpathlineto{\pgfqpoint{3.437984in}{3.064640in}}%
\pgfpathlineto{\pgfqpoint{3.441156in}{3.065114in}}%
\pgfpathlineto{\pgfqpoint{3.444329in}{3.065052in}}%
\pgfpathlineto{\pgfqpoint{3.447501in}{3.065015in}}%
\pgfpathlineto{\pgfqpoint{3.450673in}{3.065148in}}%
\pgfpathlineto{\pgfqpoint{3.453845in}{3.065307in}}%
\pgfpathlineto{\pgfqpoint{3.457017in}{3.065294in}}%
\pgfpathlineto{\pgfqpoint{3.460189in}{3.065090in}}%
\pgfpathlineto{\pgfqpoint{3.463361in}{3.064837in}}%
\pgfpathlineto{\pgfqpoint{3.466533in}{3.064130in}}%
\pgfpathlineto{\pgfqpoint{3.469705in}{3.063863in}}%
\pgfpathlineto{\pgfqpoint{3.472877in}{3.063802in}}%
\pgfpathlineto{\pgfqpoint{3.476049in}{3.063640in}}%
\pgfpathlineto{\pgfqpoint{3.479221in}{3.063606in}}%
\pgfpathlineto{\pgfqpoint{3.482393in}{3.063816in}}%
\pgfpathlineto{\pgfqpoint{3.485565in}{3.064286in}}%
\pgfpathlineto{\pgfqpoint{3.488737in}{3.064164in}}%
\pgfpathlineto{\pgfqpoint{3.491909in}{3.064173in}}%
\pgfpathlineto{\pgfqpoint{3.495081in}{3.064623in}}%
\pgfpathlineto{\pgfqpoint{3.498253in}{3.064941in}}%
\pgfpathlineto{\pgfqpoint{3.501425in}{3.064450in}}%
\pgfpathlineto{\pgfqpoint{3.504597in}{3.064189in}}%
\pgfpathlineto{\pgfqpoint{3.507769in}{3.064543in}}%
\pgfpathlineto{\pgfqpoint{3.510941in}{3.064562in}}%
\pgfpathlineto{\pgfqpoint{3.514113in}{3.064174in}}%
\pgfpathlineto{\pgfqpoint{3.517285in}{3.064093in}}%
\pgfpathlineto{\pgfqpoint{3.520457in}{3.063654in}}%
\pgfpathlineto{\pgfqpoint{3.523630in}{3.064233in}}%
\pgfpathlineto{\pgfqpoint{3.526802in}{3.064180in}}%
\pgfpathlineto{\pgfqpoint{3.529974in}{3.064358in}}%
\pgfpathlineto{\pgfqpoint{3.533146in}{3.064412in}}%
\pgfpathlineto{\pgfqpoint{3.536318in}{3.064111in}}%
\pgfpathlineto{\pgfqpoint{3.539490in}{3.063891in}}%
\pgfpathlineto{\pgfqpoint{3.542662in}{3.063492in}}%
\pgfpathlineto{\pgfqpoint{3.545834in}{3.063358in}}%
\pgfpathlineto{\pgfqpoint{3.549006in}{3.063429in}}%
\pgfpathlineto{\pgfqpoint{3.552178in}{3.062954in}}%
\pgfpathlineto{\pgfqpoint{3.555350in}{3.062550in}}%
\pgfpathlineto{\pgfqpoint{3.558522in}{3.062390in}}%
\pgfpathlineto{\pgfqpoint{3.561694in}{3.062218in}}%
\pgfpathlineto{\pgfqpoint{3.564866in}{3.061974in}}%
\pgfpathlineto{\pgfqpoint{3.568038in}{3.061811in}}%
\pgfpathlineto{\pgfqpoint{3.571210in}{3.062259in}}%
\pgfpathlineto{\pgfqpoint{3.574382in}{3.061982in}}%
\pgfpathlineto{\pgfqpoint{3.577554in}{3.062149in}}%
\pgfpathlineto{\pgfqpoint{3.580726in}{3.061737in}}%
\pgfpathlineto{\pgfqpoint{3.583898in}{3.061696in}}%
\pgfpathlineto{\pgfqpoint{3.587070in}{3.061861in}}%
\pgfpathlineto{\pgfqpoint{3.590242in}{3.061705in}}%
\pgfpathlineto{\pgfqpoint{3.593414in}{3.061714in}}%
\pgfpathlineto{\pgfqpoint{3.596586in}{3.061987in}}%
\pgfpathlineto{\pgfqpoint{3.599759in}{3.062244in}}%
\pgfpathlineto{\pgfqpoint{3.602931in}{3.062090in}}%
\pgfpathlineto{\pgfqpoint{3.606103in}{3.062096in}}%
\pgfpathlineto{\pgfqpoint{3.609275in}{3.061904in}}%
\pgfpathlineto{\pgfqpoint{3.612447in}{3.062005in}}%
\pgfpathlineto{\pgfqpoint{3.615619in}{3.062251in}}%
\pgfpathlineto{\pgfqpoint{3.618791in}{3.062221in}}%
\pgfpathlineto{\pgfqpoint{3.621963in}{3.062021in}}%
\pgfpathlineto{\pgfqpoint{3.625135in}{3.062459in}}%
\pgfpathlineto{\pgfqpoint{3.628307in}{3.062369in}}%
\pgfpathlineto{\pgfqpoint{3.631479in}{3.062690in}}%
\pgfpathlineto{\pgfqpoint{3.634651in}{3.062438in}}%
\pgfpathlineto{\pgfqpoint{3.637823in}{3.062548in}}%
\pgfpathlineto{\pgfqpoint{3.640995in}{3.062492in}}%
\pgfpathlineto{\pgfqpoint{3.644167in}{3.062461in}}%
\pgfpathlineto{\pgfqpoint{3.647339in}{3.062530in}}%
\pgfpathlineto{\pgfqpoint{3.650511in}{3.062761in}}%
\pgfpathlineto{\pgfqpoint{3.653683in}{3.062278in}}%
\pgfpathlineto{\pgfqpoint{3.656855in}{3.061667in}}%
\pgfpathlineto{\pgfqpoint{3.660027in}{3.061308in}}%
\pgfpathlineto{\pgfqpoint{3.663199in}{3.060718in}}%
\pgfpathlineto{\pgfqpoint{3.666371in}{3.060336in}}%
\pgfpathlineto{\pgfqpoint{3.669543in}{3.060210in}}%
\pgfpathlineto{\pgfqpoint{3.672715in}{3.060484in}}%
\pgfpathlineto{\pgfqpoint{3.675887in}{3.060420in}}%
\pgfpathlineto{\pgfqpoint{3.679060in}{3.060492in}}%
\pgfpathlineto{\pgfqpoint{3.682232in}{3.061021in}}%
\pgfpathlineto{\pgfqpoint{3.685404in}{3.061256in}}%
\pgfpathlineto{\pgfqpoint{3.688576in}{3.061044in}}%
\pgfpathlineto{\pgfqpoint{3.691748in}{3.060557in}}%
\pgfpathlineto{\pgfqpoint{3.694920in}{3.060338in}}%
\pgfpathlineto{\pgfqpoint{3.698092in}{3.060070in}}%
\pgfpathlineto{\pgfqpoint{3.701264in}{3.060294in}}%
\pgfpathlineto{\pgfqpoint{3.704436in}{3.060457in}}%
\pgfpathlineto{\pgfqpoint{3.707608in}{3.060300in}}%
\pgfpathlineto{\pgfqpoint{3.710780in}{3.059760in}}%
\pgfpathlineto{\pgfqpoint{3.713952in}{3.059493in}}%
\pgfpathlineto{\pgfqpoint{3.717124in}{3.059325in}}%
\pgfpathlineto{\pgfqpoint{3.720296in}{3.059057in}}%
\pgfpathlineto{\pgfqpoint{3.723468in}{3.059040in}}%
\pgfpathlineto{\pgfqpoint{3.726640in}{3.058795in}}%
\pgfpathlineto{\pgfqpoint{3.729812in}{3.058713in}}%
\pgfpathlineto{\pgfqpoint{3.732984in}{3.058681in}}%
\pgfpathlineto{\pgfqpoint{3.736156in}{3.058967in}}%
\pgfpathlineto{\pgfqpoint{3.739328in}{3.058644in}}%
\pgfpathlineto{\pgfqpoint{3.742500in}{3.058191in}}%
\pgfpathlineto{\pgfqpoint{3.745672in}{3.058096in}}%
\pgfpathlineto{\pgfqpoint{3.748844in}{3.057949in}}%
\pgfpathlineto{\pgfqpoint{3.752016in}{3.057537in}}%
\pgfpathlineto{\pgfqpoint{3.755188in}{3.057243in}}%
\pgfpathlineto{\pgfqpoint{3.758361in}{3.056590in}}%
\pgfpathlineto{\pgfqpoint{3.761533in}{3.056921in}}%
\pgfpathlineto{\pgfqpoint{3.764705in}{3.056846in}}%
\pgfpathlineto{\pgfqpoint{3.767877in}{3.056359in}}%
\pgfpathlineto{\pgfqpoint{3.771049in}{3.055580in}}%
\pgfpathlineto{\pgfqpoint{3.774221in}{3.055284in}}%
\pgfpathlineto{\pgfqpoint{3.777393in}{3.055353in}}%
\pgfpathlineto{\pgfqpoint{3.780565in}{3.055182in}}%
\pgfpathlineto{\pgfqpoint{3.783737in}{3.054766in}}%
\pgfpathlineto{\pgfqpoint{3.786909in}{3.054823in}}%
\pgfpathlineto{\pgfqpoint{3.790081in}{3.055090in}}%
\pgfpathlineto{\pgfqpoint{3.793253in}{3.054999in}}%
\pgfpathlineto{\pgfqpoint{3.796425in}{3.054834in}}%
\pgfpathlineto{\pgfqpoint{3.799597in}{3.054509in}}%
\pgfpathlineto{\pgfqpoint{3.802769in}{3.055047in}}%
\pgfpathlineto{\pgfqpoint{3.805941in}{3.054964in}}%
\pgfpathlineto{\pgfqpoint{3.809113in}{3.054748in}}%
\pgfpathlineto{\pgfqpoint{3.812285in}{3.055042in}}%
\pgfpathlineto{\pgfqpoint{3.815457in}{3.055161in}}%
\pgfpathlineto{\pgfqpoint{3.818629in}{3.054859in}}%
\pgfpathlineto{\pgfqpoint{3.821801in}{3.055028in}}%
\pgfpathlineto{\pgfqpoint{3.824973in}{3.054917in}}%
\pgfpathlineto{\pgfqpoint{3.828145in}{3.055166in}}%
\pgfpathlineto{\pgfqpoint{3.831317in}{3.055119in}}%
\pgfpathlineto{\pgfqpoint{3.834490in}{3.055297in}}%
\pgfpathlineto{\pgfqpoint{3.837662in}{3.054872in}}%
\pgfpathlineto{\pgfqpoint{3.840834in}{3.054585in}}%
\pgfpathlineto{\pgfqpoint{3.844006in}{3.054737in}}%
\pgfpathlineto{\pgfqpoint{3.847178in}{3.054642in}}%
\pgfpathlineto{\pgfqpoint{3.850350in}{3.054419in}}%
\pgfpathlineto{\pgfqpoint{3.853522in}{3.054258in}}%
\pgfpathlineto{\pgfqpoint{3.856694in}{3.054533in}}%
\pgfpathlineto{\pgfqpoint{3.859866in}{3.054489in}}%
\pgfpathlineto{\pgfqpoint{3.863038in}{3.054221in}}%
\pgfpathlineto{\pgfqpoint{3.866210in}{3.053986in}}%
\pgfpathlineto{\pgfqpoint{3.869382in}{3.053931in}}%
\pgfpathlineto{\pgfqpoint{3.872554in}{3.054085in}}%
\pgfpathlineto{\pgfqpoint{3.875726in}{3.054273in}}%
\pgfpathlineto{\pgfqpoint{3.878898in}{3.054438in}}%
\pgfpathlineto{\pgfqpoint{3.882070in}{3.054493in}}%
\pgfpathlineto{\pgfqpoint{3.885242in}{3.053901in}}%
\pgfpathlineto{\pgfqpoint{3.888414in}{3.053848in}}%
\pgfpathlineto{\pgfqpoint{3.891586in}{3.053815in}}%
\pgfpathlineto{\pgfqpoint{3.894758in}{3.053654in}}%
\pgfpathlineto{\pgfqpoint{3.897930in}{3.053646in}}%
\pgfpathlineto{\pgfqpoint{3.901102in}{3.053266in}}%
\pgfpathlineto{\pgfqpoint{3.904274in}{3.053205in}}%
\pgfpathlineto{\pgfqpoint{3.907446in}{3.053326in}}%
\pgfpathlineto{\pgfqpoint{3.910618in}{3.053417in}}%
\pgfpathlineto{\pgfqpoint{3.913791in}{3.053024in}}%
\pgfpathlineto{\pgfqpoint{3.916963in}{3.053134in}}%
\pgfpathlineto{\pgfqpoint{3.920135in}{3.053025in}}%
\pgfpathlineto{\pgfqpoint{3.923307in}{3.053064in}}%
\pgfpathlineto{\pgfqpoint{3.926479in}{3.052983in}}%
\pgfpathlineto{\pgfqpoint{3.929651in}{3.053233in}}%
\pgfpathlineto{\pgfqpoint{3.932823in}{3.053470in}}%
\pgfpathlineto{\pgfqpoint{3.935995in}{3.053441in}}%
\pgfpathlineto{\pgfqpoint{3.939167in}{3.053371in}}%
\pgfpathlineto{\pgfqpoint{3.942339in}{3.053413in}}%
\pgfpathlineto{\pgfqpoint{3.945511in}{3.053519in}}%
\pgfpathlineto{\pgfqpoint{3.948683in}{3.053554in}}%
\pgfpathlineto{\pgfqpoint{3.951855in}{3.053243in}}%
\pgfpathlineto{\pgfqpoint{3.955027in}{3.052922in}}%
\pgfpathlineto{\pgfqpoint{3.958199in}{3.053237in}}%
\pgfpathlineto{\pgfqpoint{3.961371in}{3.053286in}}%
\pgfpathlineto{\pgfqpoint{3.964543in}{3.053856in}}%
\pgfpathlineto{\pgfqpoint{3.967715in}{3.053815in}}%
\pgfpathlineto{\pgfqpoint{3.970887in}{3.053698in}}%
\pgfpathlineto{\pgfqpoint{3.974059in}{3.053585in}}%
\pgfpathlineto{\pgfqpoint{3.977231in}{3.053369in}}%
\pgfpathlineto{\pgfqpoint{3.980403in}{3.052978in}}%
\pgfpathlineto{\pgfqpoint{3.983575in}{3.052751in}}%
\pgfpathlineto{\pgfqpoint{3.986747in}{3.052595in}}%
\pgfpathlineto{\pgfqpoint{3.989919in}{3.052543in}}%
\pgfpathlineto{\pgfqpoint{3.993092in}{3.052746in}}%
\pgfpathlineto{\pgfqpoint{3.996264in}{3.052640in}}%
\pgfpathlineto{\pgfqpoint{3.999436in}{3.052758in}}%
\pgfpathlineto{\pgfqpoint{4.002608in}{3.052908in}}%
\pgfpathlineto{\pgfqpoint{4.005780in}{3.052724in}}%
\pgfpathlineto{\pgfqpoint{4.008952in}{3.052656in}}%
\pgfpathlineto{\pgfqpoint{4.012124in}{3.053142in}}%
\pgfpathlineto{\pgfqpoint{4.015296in}{3.053489in}}%
\pgfpathlineto{\pgfqpoint{4.018468in}{3.053723in}}%
\pgfpathlineto{\pgfqpoint{4.021640in}{3.053918in}}%
\pgfpathlineto{\pgfqpoint{4.024812in}{3.054116in}}%
\pgfpathlineto{\pgfqpoint{4.027984in}{3.054450in}}%
\pgfpathlineto{\pgfqpoint{4.031156in}{3.054610in}}%
\pgfpathlineto{\pgfqpoint{4.034328in}{3.054719in}}%
\pgfpathlineto{\pgfqpoint{4.037500in}{3.054199in}}%
\pgfpathlineto{\pgfqpoint{4.040672in}{3.054029in}}%
\pgfpathlineto{\pgfqpoint{4.043844in}{3.054020in}}%
\pgfpathlineto{\pgfqpoint{4.047016in}{3.054068in}}%
\pgfpathlineto{\pgfqpoint{4.050188in}{3.054082in}}%
\pgfpathlineto{\pgfqpoint{4.053360in}{3.053661in}}%
\pgfpathlineto{\pgfqpoint{4.056532in}{3.053370in}}%
\pgfpathlineto{\pgfqpoint{4.059704in}{3.053443in}}%
\pgfpathlineto{\pgfqpoint{4.062876in}{3.053666in}}%
\pgfpathlineto{\pgfqpoint{4.066048in}{3.053909in}}%
\pgfpathlineto{\pgfqpoint{4.069221in}{3.054193in}}%
\pgfpathlineto{\pgfqpoint{4.072393in}{3.054402in}}%
\pgfpathlineto{\pgfqpoint{4.075565in}{3.054523in}}%
\pgfpathlineto{\pgfqpoint{4.078737in}{3.054707in}}%
\pgfpathlineto{\pgfqpoint{4.081909in}{3.054567in}}%
\pgfpathlineto{\pgfqpoint{4.085081in}{3.054412in}}%
\pgfpathlineto{\pgfqpoint{4.088253in}{3.054125in}}%
\pgfpathlineto{\pgfqpoint{4.091425in}{3.053277in}}%
\pgfpathlineto{\pgfqpoint{4.094597in}{3.053295in}}%
\pgfpathlineto{\pgfqpoint{4.097769in}{3.053192in}}%
\pgfpathlineto{\pgfqpoint{4.100941in}{3.053396in}}%
\pgfpathlineto{\pgfqpoint{4.104113in}{3.053278in}}%
\pgfpathlineto{\pgfqpoint{4.107285in}{3.053402in}}%
\pgfpathlineto{\pgfqpoint{4.110457in}{3.053068in}}%
\pgfpathlineto{\pgfqpoint{4.113629in}{3.052991in}}%
\pgfpathlineto{\pgfqpoint{4.116801in}{3.053018in}}%
\pgfpathlineto{\pgfqpoint{4.119973in}{3.053098in}}%
\pgfpathlineto{\pgfqpoint{4.123145in}{3.053113in}}%
\pgfpathlineto{\pgfqpoint{4.126317in}{3.053183in}}%
\pgfpathlineto{\pgfqpoint{4.129489in}{3.053264in}}%
\pgfpathlineto{\pgfqpoint{4.132661in}{3.053317in}}%
\pgfpathlineto{\pgfqpoint{4.135833in}{3.053100in}}%
\pgfpathlineto{\pgfqpoint{4.139005in}{3.053222in}}%
\pgfpathlineto{\pgfqpoint{4.142177in}{3.053239in}}%
\pgfpathlineto{\pgfqpoint{4.145349in}{3.053252in}}%
\pgfpathlineto{\pgfqpoint{4.148522in}{3.052906in}}%
\pgfpathlineto{\pgfqpoint{4.151694in}{3.052623in}}%
\pgfpathlineto{\pgfqpoint{4.154866in}{3.052462in}}%
\pgfpathlineto{\pgfqpoint{4.158038in}{3.052425in}}%
\pgfpathlineto{\pgfqpoint{4.161210in}{3.052294in}}%
\pgfpathlineto{\pgfqpoint{4.164382in}{3.052327in}}%
\pgfpathlineto{\pgfqpoint{4.167554in}{3.052361in}}%
\pgfpathlineto{\pgfqpoint{4.170726in}{3.052333in}}%
\pgfpathlineto{\pgfqpoint{4.173898in}{3.052030in}}%
\pgfpathlineto{\pgfqpoint{4.177070in}{3.052063in}}%
\pgfpathlineto{\pgfqpoint{4.180242in}{3.051818in}}%
\pgfpathlineto{\pgfqpoint{4.183414in}{3.051834in}}%
\pgfpathlineto{\pgfqpoint{4.186586in}{3.051578in}}%
\pgfpathlineto{\pgfqpoint{4.189758in}{3.051048in}}%
\pgfpathlineto{\pgfqpoint{4.192930in}{3.051009in}}%
\pgfpathlineto{\pgfqpoint{4.196102in}{3.051153in}}%
\pgfpathlineto{\pgfqpoint{4.199274in}{3.051262in}}%
\pgfpathlineto{\pgfqpoint{4.202446in}{3.051044in}}%
\pgfpathlineto{\pgfqpoint{4.205618in}{3.050802in}}%
\pgfpathlineto{\pgfqpoint{4.208790in}{3.050244in}}%
\pgfpathlineto{\pgfqpoint{4.211962in}{3.049204in}}%
\pgfpathlineto{\pgfqpoint{4.215134in}{3.048931in}}%
\pgfpathlineto{\pgfqpoint{4.218306in}{3.049084in}}%
\pgfpathlineto{\pgfqpoint{4.221478in}{3.049006in}}%
\pgfpathlineto{\pgfqpoint{4.224650in}{3.048955in}}%
\pgfpathlineto{\pgfqpoint{4.227823in}{3.048583in}}%
\pgfpathlineto{\pgfqpoint{4.230995in}{3.048775in}}%
\pgfpathlineto{\pgfqpoint{4.234167in}{3.049106in}}%
\pgfpathlineto{\pgfqpoint{4.237339in}{3.048798in}}%
\pgfpathlineto{\pgfqpoint{4.240511in}{3.048596in}}%
\pgfpathlineto{\pgfqpoint{4.243683in}{3.048290in}}%
\pgfpathlineto{\pgfqpoint{4.246855in}{3.048095in}}%
\pgfpathlineto{\pgfqpoint{4.250027in}{3.047922in}}%
\pgfpathlineto{\pgfqpoint{4.253199in}{3.047627in}}%
\pgfpathlineto{\pgfqpoint{4.256371in}{3.047769in}}%
\pgfpathlineto{\pgfqpoint{4.259543in}{3.047477in}}%
\pgfpathlineto{\pgfqpoint{4.262715in}{3.047561in}}%
\pgfpathlineto{\pgfqpoint{4.265887in}{3.047593in}}%
\pgfpathlineto{\pgfqpoint{4.269059in}{3.047695in}}%
\pgfpathlineto{\pgfqpoint{4.272231in}{3.047672in}}%
\pgfpathlineto{\pgfqpoint{4.275403in}{3.047515in}}%
\pgfpathlineto{\pgfqpoint{4.278575in}{3.047727in}}%
\pgfpathlineto{\pgfqpoint{4.281747in}{3.047633in}}%
\pgfpathlineto{\pgfqpoint{4.284919in}{3.048015in}}%
\pgfpathlineto{\pgfqpoint{4.288091in}{3.048049in}}%
\pgfpathlineto{\pgfqpoint{4.291263in}{3.047696in}}%
\pgfpathlineto{\pgfqpoint{4.294435in}{3.047320in}}%
\pgfpathlineto{\pgfqpoint{4.297607in}{3.047266in}}%
\pgfpathlineto{\pgfqpoint{4.300779in}{3.047270in}}%
\pgfpathlineto{\pgfqpoint{4.303952in}{3.047550in}}%
\pgfpathlineto{\pgfqpoint{4.307124in}{3.047533in}}%
\pgfpathlineto{\pgfqpoint{4.310296in}{3.047867in}}%
\pgfpathlineto{\pgfqpoint{4.313468in}{3.047978in}}%
\pgfpathlineto{\pgfqpoint{4.316640in}{3.047871in}}%
\pgfpathlineto{\pgfqpoint{4.319812in}{3.047900in}}%
\pgfpathlineto{\pgfqpoint{4.322984in}{3.047953in}}%
\pgfpathlineto{\pgfqpoint{4.326156in}{3.048304in}}%
\pgfpathlineto{\pgfqpoint{4.329328in}{3.048425in}}%
\pgfpathlineto{\pgfqpoint{4.332500in}{3.048515in}}%
\pgfpathlineto{\pgfqpoint{4.335672in}{3.048396in}}%
\pgfpathlineto{\pgfqpoint{4.338844in}{3.048458in}}%
\pgfpathlineto{\pgfqpoint{4.342016in}{3.048333in}}%
\pgfpathlineto{\pgfqpoint{4.345188in}{3.047722in}}%
\pgfpathlineto{\pgfqpoint{4.348360in}{3.047337in}}%
\pgfpathlineto{\pgfqpoint{4.351532in}{3.046927in}}%
\pgfpathlineto{\pgfqpoint{4.354704in}{3.046733in}}%
\pgfpathlineto{\pgfqpoint{4.357876in}{3.046070in}}%
\pgfpathlineto{\pgfqpoint{4.361048in}{3.045838in}}%
\pgfpathlineto{\pgfqpoint{4.364220in}{3.045712in}}%
\pgfpathlineto{\pgfqpoint{4.367392in}{3.045724in}}%
\pgfpathlineto{\pgfqpoint{4.370564in}{3.045099in}}%
\pgfpathlineto{\pgfqpoint{4.373736in}{3.045091in}}%
\pgfpathlineto{\pgfqpoint{4.376908in}{3.045008in}}%
\pgfpathlineto{\pgfqpoint{4.380080in}{3.045020in}}%
\pgfpathlineto{\pgfqpoint{4.383253in}{3.044991in}}%
\pgfpathlineto{\pgfqpoint{4.386425in}{3.044902in}}%
\pgfpathlineto{\pgfqpoint{4.389597in}{3.045022in}}%
\pgfpathlineto{\pgfqpoint{4.392769in}{3.045226in}}%
\pgfpathlineto{\pgfqpoint{4.395941in}{3.045264in}}%
\pgfpathlineto{\pgfqpoint{4.399113in}{3.045516in}}%
\pgfpathlineto{\pgfqpoint{4.402285in}{3.045747in}}%
\pgfpathlineto{\pgfqpoint{4.405457in}{3.045525in}}%
\pgfpathlineto{\pgfqpoint{4.408629in}{3.045628in}}%
\pgfpathlineto{\pgfqpoint{4.411801in}{3.045522in}}%
\pgfpathlineto{\pgfqpoint{4.414973in}{3.045493in}}%
\pgfpathlineto{\pgfqpoint{4.418145in}{3.045443in}}%
\pgfpathlineto{\pgfqpoint{4.421317in}{3.045450in}}%
\pgfpathlineto{\pgfqpoint{4.424489in}{3.045226in}}%
\pgfpathlineto{\pgfqpoint{4.427661in}{3.045288in}}%
\pgfpathlineto{\pgfqpoint{4.430833in}{3.044555in}}%
\pgfpathlineto{\pgfqpoint{4.434005in}{3.044691in}}%
\pgfpathlineto{\pgfqpoint{4.437177in}{3.044484in}}%
\pgfpathlineto{\pgfqpoint{4.440349in}{3.044486in}}%
\pgfpathlineto{\pgfqpoint{4.443521in}{3.044217in}}%
\pgfpathlineto{\pgfqpoint{4.446693in}{3.043737in}}%
\pgfpathlineto{\pgfqpoint{4.449865in}{3.043254in}}%
\pgfpathlineto{\pgfqpoint{4.453037in}{3.043127in}}%
\pgfpathlineto{\pgfqpoint{4.456209in}{3.043507in}}%
\pgfpathlineto{\pgfqpoint{4.459381in}{3.043409in}}%
\pgfpathlineto{\pgfqpoint{4.462554in}{3.043019in}}%
\pgfpathlineto{\pgfqpoint{4.465726in}{3.042923in}}%
\pgfpathlineto{\pgfqpoint{4.468898in}{3.042954in}}%
\pgfpathlineto{\pgfqpoint{4.472070in}{3.043063in}}%
\pgfpathlineto{\pgfqpoint{4.475242in}{3.042584in}}%
\pgfpathlineto{\pgfqpoint{4.478414in}{3.042718in}}%
\pgfpathlineto{\pgfqpoint{4.481586in}{3.042431in}}%
\pgfpathlineto{\pgfqpoint{4.484758in}{3.042848in}}%
\pgfpathlineto{\pgfqpoint{4.487930in}{3.042741in}}%
\pgfpathlineto{\pgfqpoint{4.491102in}{3.042706in}}%
\pgfpathlineto{\pgfqpoint{4.494274in}{3.042549in}}%
\pgfpathlineto{\pgfqpoint{4.497446in}{3.042166in}}%
\pgfpathlineto{\pgfqpoint{4.500618in}{3.042245in}}%
\pgfpathlineto{\pgfqpoint{4.503790in}{3.042150in}}%
\pgfpathlineto{\pgfqpoint{4.506962in}{3.042176in}}%
\pgfpathlineto{\pgfqpoint{4.510134in}{3.041890in}}%
\pgfpathlineto{\pgfqpoint{4.513306in}{3.041848in}}%
\pgfpathlineto{\pgfqpoint{4.516478in}{3.041547in}}%
\pgfpathlineto{\pgfqpoint{4.519650in}{3.041718in}}%
\pgfpathlineto{\pgfqpoint{4.522822in}{3.041880in}}%
\pgfpathlineto{\pgfqpoint{4.525994in}{3.041787in}}%
\pgfpathlineto{\pgfqpoint{4.529166in}{3.041569in}}%
\pgfpathlineto{\pgfqpoint{4.532338in}{3.041514in}}%
\pgfpathlineto{\pgfqpoint{4.535510in}{3.041464in}}%
\pgfpathlineto{\pgfqpoint{4.538683in}{3.041246in}}%
\pgfpathlineto{\pgfqpoint{4.541855in}{3.040932in}}%
\pgfpathlineto{\pgfqpoint{4.545027in}{3.040756in}}%
\pgfpathlineto{\pgfqpoint{4.548199in}{3.040679in}}%
\pgfpathlineto{\pgfqpoint{4.551371in}{3.040395in}}%
\pgfpathlineto{\pgfqpoint{4.554543in}{3.040466in}}%
\pgfpathlineto{\pgfqpoint{4.557715in}{3.040266in}}%
\pgfpathlineto{\pgfqpoint{4.560887in}{3.040334in}}%
\pgfpathlineto{\pgfqpoint{4.564059in}{3.040489in}}%
\pgfpathlineto{\pgfqpoint{4.567231in}{3.040422in}}%
\pgfpathlineto{\pgfqpoint{4.570403in}{3.040292in}}%
\pgfpathlineto{\pgfqpoint{4.573575in}{3.040732in}}%
\pgfpathlineto{\pgfqpoint{4.576747in}{3.040584in}}%
\pgfpathlineto{\pgfqpoint{4.579919in}{3.040288in}}%
\pgfpathlineto{\pgfqpoint{4.583091in}{3.040433in}}%
\pgfpathlineto{\pgfqpoint{4.586263in}{3.040210in}}%
\pgfpathlineto{\pgfqpoint{4.589435in}{3.040345in}}%
\pgfpathlineto{\pgfqpoint{4.592607in}{3.040060in}}%
\pgfpathlineto{\pgfqpoint{4.595779in}{3.040029in}}%
\pgfpathlineto{\pgfqpoint{4.598951in}{3.039756in}}%
\pgfpathlineto{\pgfqpoint{4.602123in}{3.039662in}}%
\pgfpathlineto{\pgfqpoint{4.605295in}{3.039171in}}%
\pgfpathlineto{\pgfqpoint{4.608467in}{3.039688in}}%
\pgfpathlineto{\pgfqpoint{4.611639in}{3.039733in}}%
\pgfpathlineto{\pgfqpoint{4.614811in}{3.039582in}}%
\pgfpathlineto{\pgfqpoint{4.617984in}{3.039109in}}%
\pgfpathlineto{\pgfqpoint{4.621156in}{3.038946in}}%
\pgfpathlineto{\pgfqpoint{4.624328in}{3.038703in}}%
\pgfpathlineto{\pgfqpoint{4.627500in}{3.038734in}}%
\pgfpathlineto{\pgfqpoint{4.630672in}{3.038307in}}%
\pgfpathlineto{\pgfqpoint{4.633844in}{3.038060in}}%
\pgfpathlineto{\pgfqpoint{4.637016in}{3.037971in}}%
\pgfpathlineto{\pgfqpoint{4.640188in}{3.037458in}}%
\pgfpathlineto{\pgfqpoint{4.643360in}{3.037071in}}%
\pgfpathlineto{\pgfqpoint{4.646532in}{3.037082in}}%
\pgfpathlineto{\pgfqpoint{4.649704in}{3.036884in}}%
\pgfpathlineto{\pgfqpoint{4.652876in}{3.037094in}}%
\pgfpathlineto{\pgfqpoint{4.656048in}{3.036881in}}%
\pgfpathlineto{\pgfqpoint{4.659220in}{3.036905in}}%
\pgfpathlineto{\pgfqpoint{4.662392in}{3.036945in}}%
\pgfpathlineto{\pgfqpoint{4.665564in}{3.037073in}}%
\pgfpathlineto{\pgfqpoint{4.668736in}{3.037038in}}%
\pgfpathlineto{\pgfqpoint{4.671908in}{3.037332in}}%
\pgfpathlineto{\pgfqpoint{4.675080in}{3.037825in}}%
\pgfpathlineto{\pgfqpoint{4.678252in}{3.038007in}}%
\pgfpathlineto{\pgfqpoint{4.681424in}{3.037859in}}%
\pgfpathlineto{\pgfqpoint{4.684596in}{3.037508in}}%
\pgfpathlineto{\pgfqpoint{4.687768in}{3.037572in}}%
\pgfpathlineto{\pgfqpoint{4.690940in}{3.037673in}}%
\pgfpathlineto{\pgfqpoint{4.694112in}{3.037053in}}%
\pgfpathlineto{\pgfqpoint{4.697285in}{3.036864in}}%
\pgfpathlineto{\pgfqpoint{4.700457in}{3.036929in}}%
\pgfpathlineto{\pgfqpoint{4.703629in}{3.036794in}}%
\pgfpathlineto{\pgfqpoint{4.706801in}{3.036465in}}%
\pgfpathlineto{\pgfqpoint{4.709973in}{3.036133in}}%
\pgfpathlineto{\pgfqpoint{4.713145in}{3.036026in}}%
\pgfpathlineto{\pgfqpoint{4.716317in}{3.035892in}}%
\pgfpathlineto{\pgfqpoint{4.719489in}{3.035420in}}%
\pgfpathlineto{\pgfqpoint{4.722661in}{3.035637in}}%
\pgfpathlineto{\pgfqpoint{4.725833in}{3.035569in}}%
\pgfpathlineto{\pgfqpoint{4.729005in}{3.034682in}}%
\pgfpathlineto{\pgfqpoint{4.732177in}{3.034689in}}%
\pgfpathlineto{\pgfqpoint{4.735349in}{3.034777in}}%
\pgfpathlineto{\pgfqpoint{4.738521in}{3.034620in}}%
\pgfpathlineto{\pgfqpoint{4.741693in}{3.034267in}}%
\pgfpathlineto{\pgfqpoint{4.744865in}{3.034719in}}%
\pgfpathlineto{\pgfqpoint{4.748037in}{3.035012in}}%
\pgfpathlineto{\pgfqpoint{4.751209in}{3.035093in}}%
\pgfpathlineto{\pgfqpoint{4.754381in}{3.035053in}}%
\pgfpathlineto{\pgfqpoint{4.757553in}{3.035297in}}%
\pgfpathlineto{\pgfqpoint{4.760725in}{3.035589in}}%
\pgfpathlineto{\pgfqpoint{4.763897in}{3.036107in}}%
\pgfpathlineto{\pgfqpoint{4.767069in}{3.035605in}}%
\pgfpathlineto{\pgfqpoint{4.770241in}{3.035443in}}%
\pgfpathlineto{\pgfqpoint{4.773414in}{3.035255in}}%
\pgfpathlineto{\pgfqpoint{4.776586in}{3.035044in}}%
\pgfpathlineto{\pgfqpoint{4.779758in}{3.034817in}}%
\pgfpathlineto{\pgfqpoint{4.782930in}{3.034044in}}%
\pgfpathlineto{\pgfqpoint{4.786102in}{3.034627in}}%
\pgfpathlineto{\pgfqpoint{4.789274in}{3.034656in}}%
\pgfpathlineto{\pgfqpoint{4.792446in}{3.034881in}}%
\pgfpathlineto{\pgfqpoint{4.795618in}{3.034323in}}%
\pgfpathlineto{\pgfqpoint{4.798790in}{3.034300in}}%
\pgfpathlineto{\pgfqpoint{4.801962in}{3.034150in}}%
\pgfpathlineto{\pgfqpoint{4.805134in}{3.034129in}}%
\pgfpathlineto{\pgfqpoint{4.808306in}{3.034181in}}%
\pgfpathlineto{\pgfqpoint{4.811478in}{3.034215in}}%
\pgfpathlineto{\pgfqpoint{4.814650in}{3.034250in}}%
\pgfpathlineto{\pgfqpoint{4.817822in}{3.034160in}}%
\pgfpathlineto{\pgfqpoint{4.820994in}{3.033924in}}%
\pgfpathlineto{\pgfqpoint{4.824166in}{3.033712in}}%
\pgfpathlineto{\pgfqpoint{4.827338in}{3.033678in}}%
\pgfpathlineto{\pgfqpoint{4.830510in}{3.033571in}}%
\pgfpathlineto{\pgfqpoint{4.833682in}{3.033657in}}%
\pgfpathlineto{\pgfqpoint{4.836854in}{3.033586in}}%
\pgfpathlineto{\pgfqpoint{4.840026in}{3.033430in}}%
\pgfpathlineto{\pgfqpoint{4.843198in}{3.033528in}}%
\pgfpathlineto{\pgfqpoint{4.846370in}{3.033740in}}%
\pgfpathlineto{\pgfqpoint{4.849542in}{3.033661in}}%
\pgfpathlineto{\pgfqpoint{4.852715in}{3.033750in}}%
\pgfpathlineto{\pgfqpoint{4.855887in}{3.033688in}}%
\pgfpathlineto{\pgfqpoint{4.859059in}{3.033560in}}%
\pgfpathlineto{\pgfqpoint{4.862231in}{3.033613in}}%
\pgfpathlineto{\pgfqpoint{4.865403in}{3.033725in}}%
\pgfpathlineto{\pgfqpoint{4.868575in}{3.033575in}}%
\pgfpathlineto{\pgfqpoint{4.871747in}{3.034001in}}%
\pgfpathlineto{\pgfqpoint{4.874919in}{3.034527in}}%
\pgfpathlineto{\pgfqpoint{4.878091in}{3.034626in}}%
\pgfpathlineto{\pgfqpoint{4.881263in}{3.034759in}}%
\pgfpathlineto{\pgfqpoint{4.884435in}{3.034227in}}%
\pgfpathlineto{\pgfqpoint{4.887607in}{3.034112in}}%
\pgfpathlineto{\pgfqpoint{4.890779in}{3.034086in}}%
\pgfpathlineto{\pgfqpoint{4.893951in}{3.033689in}}%
\pgfpathlineto{\pgfqpoint{4.897123in}{3.033438in}}%
\pgfpathlineto{\pgfqpoint{4.900295in}{3.033518in}}%
\pgfpathlineto{\pgfqpoint{4.903467in}{3.033611in}}%
\pgfpathlineto{\pgfqpoint{4.906639in}{3.033439in}}%
\pgfpathlineto{\pgfqpoint{4.909811in}{3.033168in}}%
\pgfpathlineto{\pgfqpoint{4.912983in}{3.033406in}}%
\pgfpathlineto{\pgfqpoint{4.916155in}{3.033736in}}%
\pgfpathlineto{\pgfqpoint{4.919327in}{3.033928in}}%
\pgfpathlineto{\pgfqpoint{4.922499in}{3.033907in}}%
\pgfpathlineto{\pgfqpoint{4.925671in}{3.033831in}}%
\pgfpathlineto{\pgfqpoint{4.928844in}{3.034011in}}%
\pgfpathlineto{\pgfqpoint{4.932016in}{3.033803in}}%
\pgfpathlineto{\pgfqpoint{4.935188in}{3.033929in}}%
\pgfpathlineto{\pgfqpoint{4.938360in}{3.033954in}}%
\pgfpathlineto{\pgfqpoint{4.941532in}{3.034145in}}%
\pgfpathlineto{\pgfqpoint{4.944704in}{3.034370in}}%
\pgfpathlineto{\pgfqpoint{4.947876in}{3.034356in}}%
\pgfpathlineto{\pgfqpoint{4.951048in}{3.034358in}}%
\pgfpathlineto{\pgfqpoint{4.954220in}{3.034178in}}%
\pgfpathlineto{\pgfqpoint{4.957392in}{3.033782in}}%
\pgfpathlineto{\pgfqpoint{4.960564in}{3.033850in}}%
\pgfpathlineto{\pgfqpoint{4.963736in}{3.033581in}}%
\pgfpathlineto{\pgfqpoint{4.966908in}{3.033452in}}%
\pgfpathlineto{\pgfqpoint{4.970080in}{3.033519in}}%
\pgfpathlineto{\pgfqpoint{4.973252in}{3.033419in}}%
\pgfpathlineto{\pgfqpoint{4.976424in}{3.033400in}}%
\pgfpathlineto{\pgfqpoint{4.979596in}{3.033799in}}%
\pgfpathlineto{\pgfqpoint{4.982768in}{3.033885in}}%
\pgfpathlineto{\pgfqpoint{4.985940in}{3.033900in}}%
\pgfpathlineto{\pgfqpoint{4.989112in}{3.033735in}}%
\pgfpathlineto{\pgfqpoint{4.992284in}{3.033177in}}%
\pgfpathlineto{\pgfqpoint{4.995456in}{3.033298in}}%
\pgfpathlineto{\pgfqpoint{4.998628in}{3.033474in}}%
\pgfpathlineto{\pgfqpoint{5.001800in}{3.033583in}}%
\pgfpathlineto{\pgfqpoint{5.004972in}{3.033944in}}%
\pgfpathlineto{\pgfqpoint{5.008145in}{3.033173in}}%
\pgfpathlineto{\pgfqpoint{5.011317in}{3.032660in}}%
\pgfpathlineto{\pgfqpoint{5.014489in}{3.031991in}}%
\pgfpathlineto{\pgfqpoint{5.017661in}{3.031234in}}%
\pgfpathlineto{\pgfqpoint{5.020833in}{3.031049in}}%
\pgfpathlineto{\pgfqpoint{5.024005in}{3.030709in}}%
\pgfpathlineto{\pgfqpoint{5.027177in}{3.030333in}}%
\pgfpathlineto{\pgfqpoint{5.030349in}{3.030116in}}%
\pgfpathlineto{\pgfqpoint{5.033521in}{3.029841in}}%
\pgfpathlineto{\pgfqpoint{5.036693in}{3.029510in}}%
\pgfpathlineto{\pgfqpoint{5.039865in}{3.029501in}}%
\pgfpathlineto{\pgfqpoint{5.043037in}{3.029147in}}%
\pgfpathlineto{\pgfqpoint{5.046209in}{3.029145in}}%
\pgfpathlineto{\pgfqpoint{5.049381in}{3.028795in}}%
\pgfpathlineto{\pgfqpoint{5.052553in}{3.028682in}}%
\pgfpathlineto{\pgfqpoint{5.055725in}{3.028558in}}%
\pgfpathlineto{\pgfqpoint{5.058897in}{3.028527in}}%
\pgfpathlineto{\pgfqpoint{5.062069in}{3.028717in}}%
\pgfpathlineto{\pgfqpoint{5.065241in}{3.028552in}}%
\pgfpathlineto{\pgfqpoint{5.068413in}{3.029042in}}%
\pgfpathlineto{\pgfqpoint{5.071585in}{3.028836in}}%
\pgfpathlineto{\pgfqpoint{5.074757in}{3.028685in}}%
\pgfpathlineto{\pgfqpoint{5.077929in}{3.028448in}}%
\pgfpathlineto{\pgfqpoint{5.081101in}{3.028533in}}%
\pgfpathlineto{\pgfqpoint{5.084273in}{3.028154in}}%
\pgfpathlineto{\pgfqpoint{5.087446in}{3.028025in}}%
\pgfpathlineto{\pgfqpoint{5.090618in}{3.028140in}}%
\pgfpathlineto{\pgfqpoint{5.093790in}{3.028282in}}%
\pgfpathlineto{\pgfqpoint{5.096962in}{3.028217in}}%
\pgfpathlineto{\pgfqpoint{5.100134in}{3.028245in}}%
\pgfpathlineto{\pgfqpoint{5.103306in}{3.028091in}}%
\pgfpathlineto{\pgfqpoint{5.106478in}{3.028495in}}%
\pgfpathlineto{\pgfqpoint{5.109650in}{3.028518in}}%
\pgfpathlineto{\pgfqpoint{5.112822in}{3.028329in}}%
\pgfpathlineto{\pgfqpoint{5.115994in}{3.028351in}}%
\pgfpathlineto{\pgfqpoint{5.119166in}{3.028436in}}%
\pgfpathlineto{\pgfqpoint{5.122338in}{3.028250in}}%
\pgfpathlineto{\pgfqpoint{5.125510in}{3.028353in}}%
\pgfpathlineto{\pgfqpoint{5.128682in}{3.028476in}}%
\pgfpathlineto{\pgfqpoint{5.131854in}{3.028673in}}%
\pgfpathlineto{\pgfqpoint{5.135026in}{3.028481in}}%
\pgfpathlineto{\pgfqpoint{5.138198in}{3.028567in}}%
\pgfpathlineto{\pgfqpoint{5.141370in}{3.028781in}}%
\pgfpathlineto{\pgfqpoint{5.144542in}{3.028725in}}%
\pgfpathlineto{\pgfqpoint{5.147714in}{3.028494in}}%
\pgfpathlineto{\pgfqpoint{5.150886in}{3.028504in}}%
\pgfpathlineto{\pgfqpoint{5.154058in}{3.028830in}}%
\pgfpathlineto{\pgfqpoint{5.157230in}{3.029019in}}%
\pgfpathlineto{\pgfqpoint{5.160402in}{3.028885in}}%
\pgfpathlineto{\pgfqpoint{5.163575in}{3.028511in}}%
\pgfpathlineto{\pgfqpoint{5.166747in}{3.028706in}}%
\pgfpathlineto{\pgfqpoint{5.169919in}{3.028615in}}%
\pgfpathlineto{\pgfqpoint{5.173091in}{3.027979in}}%
\pgfpathlineto{\pgfqpoint{5.176263in}{3.027748in}}%
\pgfpathlineto{\pgfqpoint{5.179435in}{3.027817in}}%
\pgfpathlineto{\pgfqpoint{5.182607in}{3.027623in}}%
\pgfpathlineto{\pgfqpoint{5.185779in}{3.027878in}}%
\pgfpathlineto{\pgfqpoint{5.188951in}{3.028227in}}%
\pgfpathlineto{\pgfqpoint{5.192123in}{3.028333in}}%
\pgfpathlineto{\pgfqpoint{5.195295in}{3.027862in}}%
\pgfpathlineto{\pgfqpoint{5.198467in}{3.027425in}}%
\pgfpathlineto{\pgfqpoint{5.201639in}{3.027164in}}%
\pgfpathlineto{\pgfqpoint{5.204811in}{3.026819in}}%
\pgfpathlineto{\pgfqpoint{5.207983in}{3.026785in}}%
\pgfpathlineto{\pgfqpoint{5.211155in}{3.026788in}}%
\pgfpathlineto{\pgfqpoint{5.214327in}{3.026850in}}%
\pgfpathlineto{\pgfqpoint{5.217499in}{3.026675in}}%
\pgfpathlineto{\pgfqpoint{5.220671in}{3.026647in}}%
\pgfpathlineto{\pgfqpoint{5.223843in}{3.026481in}}%
\pgfpathlineto{\pgfqpoint{5.227015in}{3.026506in}}%
\pgfpathlineto{\pgfqpoint{5.230187in}{3.026595in}}%
\pgfpathlineto{\pgfqpoint{5.233359in}{3.026354in}}%
\pgfpathlineto{\pgfqpoint{5.236531in}{3.026229in}}%
\pgfpathlineto{\pgfqpoint{5.239703in}{3.026434in}}%
\pgfpathlineto{\pgfqpoint{5.242876in}{3.026714in}}%
\pgfpathlineto{\pgfqpoint{5.246048in}{3.026365in}}%
\pgfpathlineto{\pgfqpoint{5.249220in}{3.026056in}}%
\pgfpathlineto{\pgfqpoint{5.252392in}{3.026227in}}%
\pgfpathlineto{\pgfqpoint{5.255564in}{3.026032in}}%
\pgfpathlineto{\pgfqpoint{5.258736in}{3.025735in}}%
\pgfpathlineto{\pgfqpoint{5.261908in}{3.025545in}}%
\pgfpathlineto{\pgfqpoint{5.265080in}{3.025489in}}%
\pgfpathlineto{\pgfqpoint{5.268252in}{3.025593in}}%
\pgfpathlineto{\pgfqpoint{5.271424in}{3.025025in}}%
\pgfpathlineto{\pgfqpoint{5.274596in}{3.024738in}}%
\pgfpathlineto{\pgfqpoint{5.277768in}{3.024260in}}%
\pgfpathlineto{\pgfqpoint{5.280940in}{3.023750in}}%
\pgfpathlineto{\pgfqpoint{5.284112in}{3.024087in}}%
\pgfpathlineto{\pgfqpoint{5.287284in}{3.024203in}}%
\pgfpathlineto{\pgfqpoint{5.290456in}{3.024034in}}%
\pgfpathlineto{\pgfqpoint{5.293628in}{3.023590in}}%
\pgfpathlineto{\pgfqpoint{5.296800in}{3.023801in}}%
\pgfpathlineto{\pgfqpoint{5.299972in}{3.023577in}}%
\pgfpathlineto{\pgfqpoint{5.303144in}{3.023260in}}%
\pgfpathlineto{\pgfqpoint{5.306316in}{3.023172in}}%
\pgfpathlineto{\pgfqpoint{5.309488in}{3.023074in}}%
\pgfpathlineto{\pgfqpoint{5.312660in}{3.023117in}}%
\pgfpathlineto{\pgfqpoint{5.315832in}{3.023202in}}%
\pgfpathlineto{\pgfqpoint{5.319004in}{3.023264in}}%
\pgfpathlineto{\pgfqpoint{5.322177in}{3.023351in}}%
\pgfpathlineto{\pgfqpoint{5.325349in}{3.023242in}}%
\pgfpathlineto{\pgfqpoint{5.328521in}{3.023625in}}%
\pgfpathlineto{\pgfqpoint{5.331693in}{3.023182in}}%
\pgfpathlineto{\pgfqpoint{5.334865in}{3.023324in}}%
\pgfpathlineto{\pgfqpoint{5.338037in}{3.023814in}}%
\pgfpathlineto{\pgfqpoint{5.341209in}{3.023684in}}%
\pgfpathlineto{\pgfqpoint{5.344381in}{3.023554in}}%
\pgfpathlineto{\pgfqpoint{5.347553in}{3.023893in}}%
\pgfpathlineto{\pgfqpoint{5.350725in}{3.023965in}}%
\pgfpathlineto{\pgfqpoint{5.353897in}{3.023840in}}%
\pgfpathlineto{\pgfqpoint{5.357069in}{3.023694in}}%
\pgfpathlineto{\pgfqpoint{5.360241in}{3.023334in}}%
\pgfpathlineto{\pgfqpoint{5.363413in}{3.023258in}}%
\pgfpathlineto{\pgfqpoint{5.366585in}{3.022911in}}%
\pgfpathlineto{\pgfqpoint{5.369757in}{3.023328in}}%
\pgfpathlineto{\pgfqpoint{5.372929in}{3.023482in}}%
\pgfpathlineto{\pgfqpoint{5.376101in}{3.023514in}}%
\pgfpathlineto{\pgfqpoint{5.379273in}{3.023403in}}%
\pgfpathlineto{\pgfqpoint{5.382445in}{3.023060in}}%
\pgfpathlineto{\pgfqpoint{5.385617in}{3.023568in}}%
\pgfpathlineto{\pgfqpoint{5.388789in}{3.023146in}}%
\pgfpathlineto{\pgfqpoint{5.391961in}{3.023004in}}%
\pgfpathlineto{\pgfqpoint{5.395133in}{3.023074in}}%
\pgfpathlineto{\pgfqpoint{5.398306in}{3.023007in}}%
\pgfpathlineto{\pgfqpoint{5.401478in}{3.022982in}}%
\pgfpathlineto{\pgfqpoint{5.404650in}{3.023368in}}%
\pgfpathlineto{\pgfqpoint{5.407822in}{3.023112in}}%
\pgfpathlineto{\pgfqpoint{5.410994in}{3.022849in}}%
\pgfpathlineto{\pgfqpoint{5.414166in}{3.022600in}}%
\pgfpathlineto{\pgfqpoint{5.417338in}{3.022681in}}%
\pgfpathlineto{\pgfqpoint{5.420510in}{3.022971in}}%
\pgfpathlineto{\pgfqpoint{5.423682in}{3.022982in}}%
\pgfpathlineto{\pgfqpoint{5.426854in}{3.022586in}}%
\pgfpathlineto{\pgfqpoint{5.430026in}{3.022648in}}%
\pgfpathlineto{\pgfqpoint{5.433198in}{3.022460in}}%
\pgfpathlineto{\pgfqpoint{5.436370in}{3.022552in}}%
\pgfpathlineto{\pgfqpoint{5.439542in}{3.022465in}}%
\pgfpathlineto{\pgfqpoint{5.442714in}{3.022409in}}%
\pgfpathlineto{\pgfqpoint{5.445886in}{3.021858in}}%
\pgfpathlineto{\pgfqpoint{5.449058in}{3.021814in}}%
\pgfpathlineto{\pgfqpoint{5.452230in}{3.021814in}}%
\pgfpathlineto{\pgfqpoint{5.455402in}{3.021928in}}%
\pgfpathlineto{\pgfqpoint{5.458574in}{3.021856in}}%
\pgfpathlineto{\pgfqpoint{5.461746in}{3.021570in}}%
\pgfpathlineto{\pgfqpoint{5.464918in}{3.021825in}}%
\pgfpathlineto{\pgfqpoint{5.468090in}{3.022082in}}%
\pgfpathlineto{\pgfqpoint{5.471262in}{3.022035in}}%
\pgfpathlineto{\pgfqpoint{5.474434in}{3.021603in}}%
\pgfpathlineto{\pgfqpoint{5.477607in}{3.021117in}}%
\pgfpathlineto{\pgfqpoint{5.480779in}{3.020904in}}%
\pgfpathlineto{\pgfqpoint{5.483951in}{3.021133in}}%
\pgfpathlineto{\pgfqpoint{5.487123in}{3.021204in}}%
\pgfpathlineto{\pgfqpoint{5.490295in}{3.021088in}}%
\pgfpathlineto{\pgfqpoint{5.493467in}{3.020718in}}%
\pgfpathlineto{\pgfqpoint{5.496639in}{3.020689in}}%
\pgfpathlineto{\pgfqpoint{5.499811in}{3.020619in}}%
\pgfpathlineto{\pgfqpoint{5.502983in}{3.020562in}}%
\pgfpathlineto{\pgfqpoint{5.506155in}{3.020609in}}%
\pgfpathlineto{\pgfqpoint{5.509327in}{3.020314in}}%
\pgfpathlineto{\pgfqpoint{5.512499in}{3.020741in}}%
\pgfpathlineto{\pgfqpoint{5.515671in}{3.020760in}}%
\pgfpathlineto{\pgfqpoint{5.518843in}{3.019858in}}%
\pgfpathlineto{\pgfqpoint{5.522015in}{3.019711in}}%
\pgfpathlineto{\pgfqpoint{5.525187in}{3.019755in}}%
\pgfpathlineto{\pgfqpoint{5.528359in}{3.019728in}}%
\pgfpathlineto{\pgfqpoint{5.531531in}{3.020287in}}%
\pgfpathlineto{\pgfqpoint{5.534703in}{3.020255in}}%
\pgfpathlineto{\pgfqpoint{5.537875in}{3.020206in}}%
\pgfpathlineto{\pgfqpoint{5.541047in}{3.020318in}}%
\pgfpathlineto{\pgfqpoint{5.544219in}{3.020745in}}%
\pgfpathlineto{\pgfqpoint{5.547391in}{3.020174in}}%
\pgfpathlineto{\pgfqpoint{5.550563in}{3.020223in}}%
\pgfpathlineto{\pgfqpoint{5.553735in}{3.020359in}}%
\pgfpathlineto{\pgfqpoint{5.556908in}{3.020106in}}%
\pgfpathlineto{\pgfqpoint{5.560080in}{3.019726in}}%
\pgfpathlineto{\pgfqpoint{5.563252in}{3.019967in}}%
\pgfpathlineto{\pgfqpoint{5.566424in}{3.019795in}}%
\pgfpathlineto{\pgfqpoint{5.569596in}{3.019639in}}%
\pgfpathlineto{\pgfqpoint{5.572768in}{3.019504in}}%
\pgfpathlineto{\pgfqpoint{5.575940in}{3.019207in}}%
\pgfpathlineto{\pgfqpoint{5.579112in}{3.018696in}}%
\pgfpathlineto{\pgfqpoint{5.582284in}{3.018405in}}%
\pgfpathlineto{\pgfqpoint{5.585456in}{3.018516in}}%
\pgfpathlineto{\pgfqpoint{5.588628in}{3.019038in}}%
\pgfpathlineto{\pgfqpoint{5.591800in}{3.018863in}}%
\pgfpathlineto{\pgfqpoint{5.594972in}{3.018386in}}%
\pgfpathlineto{\pgfqpoint{5.598144in}{3.018198in}}%
\pgfpathlineto{\pgfqpoint{5.601316in}{3.017770in}}%
\pgfpathlineto{\pgfqpoint{5.604488in}{3.017571in}}%
\pgfpathlineto{\pgfqpoint{5.607660in}{3.017125in}}%
\pgfpathlineto{\pgfqpoint{5.610832in}{3.017139in}}%
\pgfpathlineto{\pgfqpoint{5.614004in}{3.017250in}}%
\pgfpathlineto{\pgfqpoint{5.617176in}{3.016943in}}%
\pgfpathlineto{\pgfqpoint{5.620348in}{3.016566in}}%
\pgfpathlineto{\pgfqpoint{5.623520in}{3.016034in}}%
\pgfpathlineto{\pgfqpoint{5.626692in}{3.015529in}}%
\pgfpathlineto{\pgfqpoint{5.629864in}{3.015424in}}%
\pgfpathlineto{\pgfqpoint{5.633037in}{3.015042in}}%
\pgfpathlineto{\pgfqpoint{5.636209in}{3.014499in}}%
\pgfpathlineto{\pgfqpoint{5.639381in}{3.014340in}}%
\pgfpathlineto{\pgfqpoint{5.642553in}{3.014638in}}%
\pgfpathlineto{\pgfqpoint{5.645725in}{3.014385in}}%
\pgfpathlineto{\pgfqpoint{5.648897in}{3.014311in}}%
\pgfpathlineto{\pgfqpoint{5.652069in}{3.014108in}}%
\pgfpathlineto{\pgfqpoint{5.655241in}{3.013479in}}%
\pgfpathlineto{\pgfqpoint{5.658413in}{3.013177in}}%
\pgfpathlineto{\pgfqpoint{5.661585in}{3.013068in}}%
\pgfpathlineto{\pgfqpoint{5.664757in}{3.012490in}}%
\pgfpathlineto{\pgfqpoint{5.667929in}{3.012404in}}%
\pgfpathlineto{\pgfqpoint{5.671101in}{3.012238in}}%
\pgfpathlineto{\pgfqpoint{5.674273in}{3.012591in}}%
\pgfpathlineto{\pgfqpoint{5.677445in}{3.012534in}}%
\pgfpathlineto{\pgfqpoint{5.680617in}{3.012299in}}%
\pgfpathlineto{\pgfqpoint{5.683789in}{3.012373in}}%
\pgfpathlineto{\pgfqpoint{5.686961in}{3.012568in}}%
\pgfpathlineto{\pgfqpoint{5.690133in}{3.012655in}}%
\pgfpathlineto{\pgfqpoint{5.693305in}{3.012713in}}%
\pgfpathlineto{\pgfqpoint{5.696477in}{3.012527in}}%
\pgfpathlineto{\pgfqpoint{5.699649in}{3.012168in}}%
\pgfpathlineto{\pgfqpoint{5.702821in}{3.011772in}}%
\pgfpathlineto{\pgfqpoint{5.705993in}{3.011795in}}%
\pgfpathlineto{\pgfqpoint{5.709165in}{3.011456in}}%
\pgfpathlineto{\pgfqpoint{5.712338in}{3.010958in}}%
\pgfpathlineto{\pgfqpoint{5.715510in}{3.010453in}}%
\pgfpathlineto{\pgfqpoint{5.718682in}{3.010177in}}%
\pgfpathlineto{\pgfqpoint{5.721854in}{3.009754in}}%
\pgfpathlineto{\pgfqpoint{5.725026in}{3.008966in}}%
\pgfpathlineto{\pgfqpoint{5.728198in}{3.009204in}}%
\pgfpathlineto{\pgfqpoint{5.731370in}{3.009506in}}%
\pgfpathlineto{\pgfqpoint{5.734542in}{3.009660in}}%
\pgfpathlineto{\pgfqpoint{5.737714in}{3.009478in}}%
\pgfpathlineto{\pgfqpoint{5.740886in}{3.009462in}}%
\pgfpathlineto{\pgfqpoint{5.744058in}{3.009842in}}%
\pgfpathlineto{\pgfqpoint{5.747230in}{3.009278in}}%
\pgfpathlineto{\pgfqpoint{5.750402in}{3.009089in}}%
\pgfpathlineto{\pgfqpoint{5.753574in}{3.009531in}}%
\pgfpathlineto{\pgfqpoint{5.756746in}{3.009388in}}%
\pgfpathlineto{\pgfqpoint{5.759918in}{3.009733in}}%
\pgfpathlineto{\pgfqpoint{5.763090in}{3.010224in}}%
\pgfpathlineto{\pgfqpoint{5.766262in}{3.010747in}}%
\pgfpathlineto{\pgfqpoint{5.769434in}{3.011091in}}%
\pgfpathlineto{\pgfqpoint{5.772606in}{3.011191in}}%
\pgfpathlineto{\pgfqpoint{5.775778in}{3.010705in}}%
\pgfpathlineto{\pgfqpoint{5.778950in}{3.010417in}}%
\pgfpathlineto{\pgfqpoint{5.782122in}{3.009728in}}%
\pgfpathlineto{\pgfqpoint{5.785294in}{3.008940in}}%
\pgfpathlineto{\pgfqpoint{5.788466in}{3.008945in}}%
\pgfpathlineto{\pgfqpoint{5.791639in}{3.009269in}}%
\pgfpathlineto{\pgfqpoint{5.794811in}{3.009166in}}%
\pgfpathlineto{\pgfqpoint{5.797983in}{3.008882in}}%
\pgfpathlineto{\pgfqpoint{5.801155in}{3.009283in}}%
\pgfpathlineto{\pgfqpoint{5.804327in}{3.009107in}}%
\pgfpathlineto{\pgfqpoint{5.807499in}{3.009088in}}%
\pgfpathlineto{\pgfqpoint{5.810671in}{3.008616in}}%
\pgfpathlineto{\pgfqpoint{5.813843in}{3.008744in}}%
\pgfpathlineto{\pgfqpoint{5.817015in}{3.008586in}}%
\pgfpathlineto{\pgfqpoint{5.820187in}{3.008730in}}%
\pgfpathlineto{\pgfqpoint{5.823359in}{3.009123in}}%
\pgfpathlineto{\pgfqpoint{5.826531in}{3.009194in}}%
\pgfpathlineto{\pgfqpoint{5.829703in}{3.008560in}}%
\pgfpathlineto{\pgfqpoint{5.832875in}{3.008040in}}%
\pgfpathlineto{\pgfqpoint{5.836047in}{3.007226in}}%
\pgfpathlineto{\pgfqpoint{5.839219in}{3.007092in}}%
\pgfpathlineto{\pgfqpoint{5.842391in}{3.006991in}}%
\pgfpathlineto{\pgfqpoint{5.845563in}{3.007123in}}%
\pgfpathlineto{\pgfqpoint{5.848735in}{3.006988in}}%
\pgfpathlineto{\pgfqpoint{5.851907in}{3.006715in}}%
\pgfpathlineto{\pgfqpoint{5.855079in}{3.007063in}}%
\pgfpathlineto{\pgfqpoint{5.858251in}{3.007154in}}%
\pgfpathlineto{\pgfqpoint{5.861423in}{3.007192in}}%
\pgfpathlineto{\pgfqpoint{5.864595in}{3.007616in}}%
\pgfpathlineto{\pgfqpoint{5.867768in}{3.007201in}}%
\pgfpathlineto{\pgfqpoint{5.870940in}{3.007299in}}%
\pgfpathlineto{\pgfqpoint{5.874112in}{3.007066in}}%
\pgfpathlineto{\pgfqpoint{5.877284in}{3.006703in}}%
\pgfpathlineto{\pgfqpoint{5.880456in}{3.006389in}}%
\pgfpathlineto{\pgfqpoint{5.883628in}{3.006473in}}%
\pgfpathlineto{\pgfqpoint{5.886800in}{3.006491in}}%
\pgfpathlineto{\pgfqpoint{5.889972in}{3.006485in}}%
\pgfpathlineto{\pgfqpoint{5.893144in}{3.006105in}}%
\pgfpathlineto{\pgfqpoint{5.896316in}{3.005957in}}%
\pgfpathlineto{\pgfqpoint{5.899488in}{3.005999in}}%
\pgfpathlineto{\pgfqpoint{5.902660in}{3.006054in}}%
\pgfpathlineto{\pgfqpoint{5.905832in}{3.006457in}}%
\pgfpathlineto{\pgfqpoint{5.909004in}{3.006335in}}%
\pgfpathlineto{\pgfqpoint{5.912176in}{3.006510in}}%
\pgfpathlineto{\pgfqpoint{5.915348in}{3.006635in}}%
\pgfpathlineto{\pgfqpoint{5.918520in}{3.006581in}}%
\pgfpathlineto{\pgfqpoint{5.921692in}{3.006426in}}%
\pgfpathlineto{\pgfqpoint{5.924864in}{3.006257in}}%
\pgfpathlineto{\pgfqpoint{5.928036in}{3.005869in}}%
\pgfpathlineto{\pgfqpoint{5.931208in}{3.005732in}}%
\pgfpathlineto{\pgfqpoint{5.934380in}{3.005293in}}%
\pgfpathlineto{\pgfqpoint{5.937552in}{3.005183in}}%
\pgfpathlineto{\pgfqpoint{5.940724in}{3.005356in}}%
\pgfpathlineto{\pgfqpoint{5.943896in}{3.005592in}}%
\pgfpathlineto{\pgfqpoint{5.947069in}{3.005456in}}%
\pgfpathlineto{\pgfqpoint{5.950241in}{3.005153in}}%
\pgfpathlineto{\pgfqpoint{5.953413in}{3.005113in}}%
\pgfpathlineto{\pgfqpoint{5.956585in}{3.004801in}}%
\pgfpathlineto{\pgfqpoint{5.959757in}{3.004528in}}%
\pgfpathlineto{\pgfqpoint{5.962929in}{3.004295in}}%
\pgfpathlineto{\pgfqpoint{5.966101in}{3.003862in}}%
\pgfpathlineto{\pgfqpoint{5.969273in}{3.003922in}}%
\pgfpathlineto{\pgfqpoint{5.972445in}{3.003660in}}%
\pgfpathlineto{\pgfqpoint{5.975617in}{3.003601in}}%
\pgfpathlineto{\pgfqpoint{5.978789in}{3.003045in}}%
\pgfpathlineto{\pgfqpoint{5.981961in}{3.003072in}}%
\pgfpathlineto{\pgfqpoint{5.985133in}{3.003165in}}%
\pgfpathlineto{\pgfqpoint{5.988305in}{3.003311in}}%
\pgfpathlineto{\pgfqpoint{5.991477in}{3.002701in}}%
\pgfpathlineto{\pgfqpoint{5.994649in}{3.002227in}}%
\pgfpathlineto{\pgfqpoint{5.997821in}{3.001519in}}%
\pgfpathlineto{\pgfqpoint{6.000993in}{3.001755in}}%
\pgfpathlineto{\pgfqpoint{6.004165in}{3.002077in}}%
\pgfpathlineto{\pgfqpoint{6.007337in}{3.002175in}}%
\pgfpathlineto{\pgfqpoint{6.010509in}{3.002654in}}%
\pgfpathlineto{\pgfqpoint{6.013681in}{3.002706in}}%
\pgfpathlineto{\pgfqpoint{6.016853in}{3.002993in}}%
\pgfpathlineto{\pgfqpoint{6.020025in}{3.002986in}}%
\pgfpathlineto{\pgfqpoint{6.023197in}{3.003083in}}%
\pgfpathlineto{\pgfqpoint{6.026370in}{3.002910in}}%
\pgfpathlineto{\pgfqpoint{6.029542in}{3.002887in}}%
\pgfpathlineto{\pgfqpoint{6.032714in}{3.002780in}}%
\pgfpathlineto{\pgfqpoint{6.035886in}{3.003120in}}%
\pgfpathlineto{\pgfqpoint{6.039058in}{3.003292in}}%
\pgfpathlineto{\pgfqpoint{6.042230in}{3.002846in}}%
\pgfpathlineto{\pgfqpoint{6.045402in}{3.003285in}}%
\pgfpathlineto{\pgfqpoint{6.048574in}{3.002799in}}%
\pgfpathlineto{\pgfqpoint{6.051746in}{3.002589in}}%
\pgfpathlineto{\pgfqpoint{6.054918in}{3.002800in}}%
\pgfpathlineto{\pgfqpoint{6.058090in}{3.002451in}}%
\pgfpathlineto{\pgfqpoint{6.061262in}{3.001985in}}%
\pgfpathlineto{\pgfqpoint{6.064434in}{3.001657in}}%
\pgfpathlineto{\pgfqpoint{6.067606in}{3.001837in}}%
\pgfpathlineto{\pgfqpoint{6.070778in}{3.001609in}}%
\pgfpathlineto{\pgfqpoint{6.073950in}{3.001639in}}%
\pgfpathlineto{\pgfqpoint{6.077122in}{3.001667in}}%
\pgfpathlineto{\pgfqpoint{6.080294in}{3.001553in}}%
\pgfpathlineto{\pgfqpoint{6.083466in}{3.001665in}}%
\pgfpathlineto{\pgfqpoint{6.086638in}{3.001397in}}%
\pgfpathlineto{\pgfqpoint{6.089810in}{3.001043in}}%
\pgfpathlineto{\pgfqpoint{6.092982in}{3.000825in}}%
\pgfpathlineto{\pgfqpoint{6.096154in}{3.000994in}}%
\pgfpathlineto{\pgfqpoint{6.099326in}{3.001119in}}%
\pgfpathlineto{\pgfqpoint{6.102499in}{3.001377in}}%
\pgfpathlineto{\pgfqpoint{6.105671in}{3.001196in}}%
\pgfpathlineto{\pgfqpoint{6.108843in}{3.000810in}}%
\pgfpathlineto{\pgfqpoint{6.112015in}{3.000679in}}%
\pgfpathlineto{\pgfqpoint{6.115187in}{3.000532in}}%
\pgfpathlineto{\pgfqpoint{6.118359in}{3.000613in}}%
\pgfpathlineto{\pgfqpoint{6.121531in}{3.000134in}}%
\pgfpathlineto{\pgfqpoint{6.124703in}{3.000075in}}%
\pgfpathlineto{\pgfqpoint{6.127875in}{2.999798in}}%
\pgfpathlineto{\pgfqpoint{6.131047in}{3.000113in}}%
\pgfpathlineto{\pgfqpoint{6.134219in}{2.999940in}}%
\pgfpathlineto{\pgfqpoint{6.137391in}{3.000119in}}%
\pgfpathlineto{\pgfqpoint{6.140563in}{2.999801in}}%
\pgfpathlineto{\pgfqpoint{6.143735in}{2.999974in}}%
\pgfpathlineto{\pgfqpoint{6.146907in}{2.999081in}}%
\pgfpathlineto{\pgfqpoint{6.150079in}{2.999156in}}%
\pgfpathlineto{\pgfqpoint{6.153251in}{2.998837in}}%
\pgfpathlineto{\pgfqpoint{6.156423in}{2.999543in}}%
\pgfpathlineto{\pgfqpoint{6.159595in}{2.999947in}}%
\pgfpathlineto{\pgfqpoint{6.162767in}{3.000145in}}%
\pgfpathlineto{\pgfqpoint{6.165939in}{3.000101in}}%
\pgfpathlineto{\pgfqpoint{6.169111in}{2.999760in}}%
\pgfpathlineto{\pgfqpoint{6.172283in}{2.999733in}}%
\pgfpathlineto{\pgfqpoint{6.175455in}{2.999830in}}%
\pgfpathlineto{\pgfqpoint{6.178627in}{2.999727in}}%
\pgfpathlineto{\pgfqpoint{6.181800in}{2.999795in}}%
\pgfpathlineto{\pgfqpoint{6.184972in}{2.999378in}}%
\pgfpathlineto{\pgfqpoint{6.188144in}{2.999344in}}%
\pgfpathlineto{\pgfqpoint{6.191316in}{2.999300in}}%
\pgfpathlineto{\pgfqpoint{6.194488in}{2.998708in}}%
\pgfpathlineto{\pgfqpoint{6.197660in}{2.998924in}}%
\pgfpathlineto{\pgfqpoint{6.200832in}{2.998944in}}%
\pgfpathlineto{\pgfqpoint{6.204004in}{2.999112in}}%
\pgfpathlineto{\pgfqpoint{6.207176in}{2.998779in}}%
\pgfpathlineto{\pgfqpoint{6.210348in}{2.998593in}}%
\pgfpathlineto{\pgfqpoint{6.213520in}{2.998078in}}%
\pgfpathlineto{\pgfqpoint{6.216692in}{2.998103in}}%
\pgfpathlineto{\pgfqpoint{6.219864in}{2.998174in}}%
\pgfpathlineto{\pgfqpoint{6.223036in}{2.998303in}}%
\pgfpathlineto{\pgfqpoint{6.226208in}{2.998305in}}%
\pgfpathlineto{\pgfqpoint{6.229380in}{2.998156in}}%
\pgfpathlineto{\pgfqpoint{6.232552in}{2.998203in}}%
\pgfpathlineto{\pgfqpoint{6.235724in}{2.998224in}}%
\pgfpathlineto{\pgfqpoint{6.238896in}{2.998154in}}%
\pgfpathlineto{\pgfqpoint{6.242068in}{2.997837in}}%
\pgfpathlineto{\pgfqpoint{6.245240in}{2.997834in}}%
\pgfpathlineto{\pgfqpoint{6.248412in}{2.997672in}}%
\pgfpathlineto{\pgfqpoint{6.251584in}{2.997409in}}%
\pgfpathlineto{\pgfqpoint{6.254756in}{2.997426in}}%
\pgfpathlineto{\pgfqpoint{6.257928in}{2.997452in}}%
\pgfpathlineto{\pgfqpoint{6.261101in}{2.997602in}}%
\pgfpathlineto{\pgfqpoint{6.264273in}{2.997664in}}%
\pgfpathlineto{\pgfqpoint{6.267445in}{2.997683in}}%
\pgfpathlineto{\pgfqpoint{6.270617in}{2.997250in}}%
\pgfpathlineto{\pgfqpoint{6.273789in}{2.996856in}}%
\pgfpathlineto{\pgfqpoint{6.276961in}{2.996852in}}%
\pgfpathlineto{\pgfqpoint{6.280133in}{2.996696in}}%
\pgfpathlineto{\pgfqpoint{6.283305in}{2.996544in}}%
\pgfpathlineto{\pgfqpoint{6.286477in}{2.996317in}}%
\pgfpathlineto{\pgfqpoint{6.289649in}{2.996423in}}%
\pgfpathlineto{\pgfqpoint{6.292821in}{2.996683in}}%
\pgfpathlineto{\pgfqpoint{6.295993in}{2.996540in}}%
\pgfpathlineto{\pgfqpoint{6.299165in}{2.996427in}}%
\pgfpathlineto{\pgfqpoint{6.302337in}{2.996394in}}%
\pgfpathlineto{\pgfqpoint{6.305509in}{2.996018in}}%
\pgfpathlineto{\pgfqpoint{6.308681in}{2.995601in}}%
\pgfpathlineto{\pgfqpoint{6.311853in}{2.995308in}}%
\pgfpathlineto{\pgfqpoint{6.315025in}{2.995296in}}%
\pgfpathlineto{\pgfqpoint{6.318197in}{2.995374in}}%
\pgfpathlineto{\pgfqpoint{6.321369in}{2.995199in}}%
\pgfpathlineto{\pgfqpoint{6.324541in}{2.995242in}}%
\pgfpathlineto{\pgfqpoint{6.327713in}{2.995372in}}%
\pgfpathlineto{\pgfqpoint{6.330885in}{2.995258in}}%
\pgfpathlineto{\pgfqpoint{6.334057in}{2.995121in}}%
\pgfpathlineto{\pgfqpoint{6.337230in}{2.995234in}}%
\pgfpathlineto{\pgfqpoint{6.340402in}{2.995302in}}%
\pgfpathlineto{\pgfqpoint{6.343574in}{2.995162in}}%
\pgfpathlineto{\pgfqpoint{6.346746in}{2.995060in}}%
\pgfpathlineto{\pgfqpoint{6.349918in}{2.995226in}}%
\pgfpathlineto{\pgfqpoint{6.353090in}{2.995491in}}%
\pgfpathlineto{\pgfqpoint{6.356262in}{2.995968in}}%
\pgfpathlineto{\pgfqpoint{6.359434in}{2.995675in}}%
\pgfpathlineto{\pgfqpoint{6.362606in}{2.995624in}}%
\pgfpathlineto{\pgfqpoint{6.365778in}{2.995590in}}%
\pgfpathlineto{\pgfqpoint{6.368950in}{2.995721in}}%
\pgfpathlineto{\pgfqpoint{6.372122in}{2.995424in}}%
\pgfpathlineto{\pgfqpoint{6.375294in}{2.995459in}}%
\pgfpathlineto{\pgfqpoint{6.378466in}{2.995368in}}%
\pgfpathlineto{\pgfqpoint{6.381638in}{2.995778in}}%
\pgfpathlineto{\pgfqpoint{6.384810in}{2.995599in}}%
\pgfpathlineto{\pgfqpoint{6.387982in}{2.995652in}}%
\pgfpathlineto{\pgfqpoint{6.391154in}{2.995495in}}%
\pgfpathlineto{\pgfqpoint{6.394326in}{2.995041in}}%
\pgfpathlineto{\pgfqpoint{6.397498in}{2.994893in}}%
\pgfpathlineto{\pgfqpoint{6.400670in}{2.994142in}}%
\pgfpathlineto{\pgfqpoint{6.403842in}{2.993908in}}%
\pgfpathlineto{\pgfqpoint{6.407014in}{2.993660in}}%
\pgfpathlineto{\pgfqpoint{6.410186in}{2.993632in}}%
\pgfpathlineto{\pgfqpoint{6.413358in}{2.993148in}}%
\pgfpathlineto{\pgfqpoint{6.416531in}{2.993260in}}%
\pgfpathlineto{\pgfqpoint{6.419703in}{2.992827in}}%
\pgfpathlineto{\pgfqpoint{6.422875in}{2.992878in}}%
\pgfpathlineto{\pgfqpoint{6.426047in}{2.993414in}}%
\pgfpathlineto{\pgfqpoint{6.429219in}{2.993555in}}%
\pgfpathlineto{\pgfqpoint{6.432391in}{2.993401in}}%
\pgfpathlineto{\pgfqpoint{6.435563in}{2.993252in}}%
\pgfpathlineto{\pgfqpoint{6.438735in}{2.992911in}}%
\pgfpathlineto{\pgfqpoint{6.441907in}{2.992688in}}%
\pgfpathlineto{\pgfqpoint{6.445079in}{2.992367in}}%
\pgfpathlineto{\pgfqpoint{6.448251in}{2.992233in}}%
\pgfpathlineto{\pgfqpoint{6.451423in}{2.991980in}}%
\pgfpathlineto{\pgfqpoint{6.454595in}{2.992204in}}%
\pgfpathlineto{\pgfqpoint{6.457767in}{2.992228in}}%
\pgfpathlineto{\pgfqpoint{6.460939in}{2.991978in}}%
\pgfpathlineto{\pgfqpoint{6.464111in}{2.992278in}}%
\pgfpathlineto{\pgfqpoint{6.467283in}{2.992121in}}%
\pgfpathlineto{\pgfqpoint{6.470455in}{2.992192in}}%
\pgfpathlineto{\pgfqpoint{6.473627in}{2.992168in}}%
\pgfpathlineto{\pgfqpoint{6.476799in}{2.991966in}}%
\pgfpathlineto{\pgfqpoint{6.479971in}{2.991748in}}%
\pgfpathlineto{\pgfqpoint{6.483143in}{2.991769in}}%
\pgfpathlineto{\pgfqpoint{6.486315in}{2.991981in}}%
\pgfpathlineto{\pgfqpoint{6.489487in}{2.991976in}}%
\pgfpathlineto{\pgfqpoint{6.492659in}{2.992160in}}%
\pgfpathlineto{\pgfqpoint{6.495832in}{2.992225in}}%
\pgfpathlineto{\pgfqpoint{6.499004in}{2.992095in}}%
\pgfpathlineto{\pgfqpoint{6.502176in}{2.991934in}}%
\pgfpathlineto{\pgfqpoint{6.505348in}{2.991788in}}%
\pgfpathlineto{\pgfqpoint{6.508520in}{2.991512in}}%
\pgfpathlineto{\pgfqpoint{6.511692in}{2.991203in}}%
\pgfpathlineto{\pgfqpoint{6.514864in}{2.991139in}}%
\pgfpathlineto{\pgfqpoint{6.518036in}{2.990791in}}%
\pgfpathlineto{\pgfqpoint{6.521208in}{2.990501in}}%
\pgfpathlineto{\pgfqpoint{6.524380in}{2.989718in}}%
\pgfpathlineto{\pgfqpoint{6.527552in}{2.989206in}}%
\pgfpathlineto{\pgfqpoint{6.530724in}{2.988956in}}%
\pgfpathlineto{\pgfqpoint{6.533896in}{2.989259in}}%
\pgfpathlineto{\pgfqpoint{6.537068in}{2.989510in}}%
\pgfpathlineto{\pgfqpoint{6.540240in}{2.989278in}}%
\pgfpathlineto{\pgfqpoint{6.543412in}{2.989294in}}%
\pgfpathlineto{\pgfqpoint{6.546584in}{2.988535in}}%
\pgfpathlineto{\pgfqpoint{6.549756in}{2.988115in}}%
\pgfpathlineto{\pgfqpoint{6.552928in}{2.988140in}}%
\pgfpathlineto{\pgfqpoint{6.556100in}{2.988081in}}%
\pgfpathlineto{\pgfqpoint{6.559272in}{2.988220in}}%
\pgfpathlineto{\pgfqpoint{6.562444in}{2.988053in}}%
\pgfpathlineto{\pgfqpoint{6.565616in}{2.988360in}}%
\pgfpathlineto{\pgfqpoint{6.568788in}{2.988401in}}%
\pgfpathlineto{\pgfqpoint{6.571961in}{2.988303in}}%
\pgfpathlineto{\pgfqpoint{6.575133in}{2.988285in}}%
\pgfpathlineto{\pgfqpoint{6.578305in}{2.988049in}}%
\pgfpathlineto{\pgfqpoint{6.581477in}{2.987972in}}%
\pgfpathlineto{\pgfqpoint{6.584649in}{2.988211in}}%
\pgfpathlineto{\pgfqpoint{6.587821in}{2.988196in}}%
\pgfpathlineto{\pgfqpoint{6.590993in}{2.987969in}}%
\pgfpathlineto{\pgfqpoint{6.594165in}{2.988012in}}%
\pgfpathlineto{\pgfqpoint{6.597337in}{2.988120in}}%
\pgfpathlineto{\pgfqpoint{6.600509in}{2.987381in}}%
\pgfpathlineto{\pgfqpoint{6.603681in}{2.987379in}}%
\pgfpathlineto{\pgfqpoint{6.606853in}{2.987320in}}%
\pgfpathlineto{\pgfqpoint{6.610025in}{2.986665in}}%
\pgfpathlineto{\pgfqpoint{6.613197in}{2.985734in}}%
\pgfpathlineto{\pgfqpoint{6.616369in}{2.985649in}}%
\pgfpathlineto{\pgfqpoint{6.619541in}{2.986125in}}%
\pgfpathlineto{\pgfqpoint{6.622713in}{2.986337in}}%
\pgfpathlineto{\pgfqpoint{6.625885in}{2.986334in}}%
\pgfpathlineto{\pgfqpoint{6.629057in}{2.986242in}}%
\pgfpathlineto{\pgfqpoint{6.632229in}{2.986291in}}%
\pgfpathlineto{\pgfqpoint{6.635401in}{2.986188in}}%
\pgfpathlineto{\pgfqpoint{6.638573in}{2.985825in}}%
\pgfpathlineto{\pgfqpoint{6.641745in}{2.985887in}}%
\pgfpathlineto{\pgfqpoint{6.644917in}{2.985846in}}%
\pgfpathlineto{\pgfqpoint{6.648089in}{2.985972in}}%
\pgfpathlineto{\pgfqpoint{6.651262in}{2.985881in}}%
\pgfpathlineto{\pgfqpoint{6.654434in}{2.985968in}}%
\pgfpathlineto{\pgfqpoint{6.657606in}{2.985747in}}%
\pgfpathlineto{\pgfqpoint{6.660778in}{2.985415in}}%
\pgfpathlineto{\pgfqpoint{6.663950in}{2.985391in}}%
\pgfpathlineto{\pgfqpoint{6.667122in}{2.985206in}}%
\pgfpathlineto{\pgfqpoint{6.670294in}{2.985231in}}%
\pgfpathlineto{\pgfqpoint{6.673466in}{2.985261in}}%
\pgfpathlineto{\pgfqpoint{6.676638in}{2.985496in}}%
\pgfpathlineto{\pgfqpoint{6.679810in}{2.985326in}}%
\pgfpathlineto{\pgfqpoint{6.682982in}{2.985494in}}%
\pgfpathlineto{\pgfqpoint{6.686154in}{2.985226in}}%
\pgfpathlineto{\pgfqpoint{6.689326in}{2.984711in}}%
\pgfpathlineto{\pgfqpoint{6.692498in}{2.984922in}}%
\pgfpathlineto{\pgfqpoint{6.695670in}{2.984875in}}%
\pgfpathlineto{\pgfqpoint{6.698842in}{2.984740in}}%
\pgfpathlineto{\pgfqpoint{6.702014in}{2.984867in}}%
\pgfpathlineto{\pgfqpoint{6.705186in}{2.984881in}}%
\pgfpathlineto{\pgfqpoint{6.708358in}{2.985118in}}%
\pgfpathlineto{\pgfqpoint{6.711530in}{2.984502in}}%
\pgfpathlineto{\pgfqpoint{6.714702in}{2.984567in}}%
\pgfpathlineto{\pgfqpoint{6.717874in}{2.984392in}}%
\pgfpathlineto{\pgfqpoint{6.721046in}{2.984198in}}%
\pgfpathlineto{\pgfqpoint{6.724218in}{2.983978in}}%
\pgfpathlineto{\pgfqpoint{6.727391in}{2.983748in}}%
\pgfpathlineto{\pgfqpoint{6.730563in}{2.983716in}}%
\pgfpathlineto{\pgfqpoint{6.733735in}{2.983587in}}%
\pgfpathlineto{\pgfqpoint{6.736907in}{2.983921in}}%
\pgfpathlineto{\pgfqpoint{6.740079in}{2.983689in}}%
\pgfpathlineto{\pgfqpoint{6.743251in}{2.983492in}}%
\pgfpathlineto{\pgfqpoint{6.746423in}{2.982916in}}%
\pgfpathlineto{\pgfqpoint{6.749595in}{2.982654in}}%
\pgfpathlineto{\pgfqpoint{6.752767in}{2.982398in}}%
\pgfpathlineto{\pgfqpoint{6.755939in}{2.982519in}}%
\pgfpathlineto{\pgfqpoint{6.759111in}{2.982778in}}%
\pgfpathlineto{\pgfqpoint{6.762283in}{2.983126in}}%
\pgfpathlineto{\pgfqpoint{6.765455in}{2.983130in}}%
\pgfpathlineto{\pgfqpoint{6.768627in}{2.983110in}}%
\pgfpathlineto{\pgfqpoint{6.771799in}{2.983247in}}%
\pgfpathlineto{\pgfqpoint{6.774971in}{2.983074in}}%
\pgfpathlineto{\pgfqpoint{6.778143in}{2.983182in}}%
\pgfpathlineto{\pgfqpoint{6.781315in}{2.983106in}}%
\pgfpathlineto{\pgfqpoint{6.784487in}{2.983145in}}%
\pgfpathlineto{\pgfqpoint{6.787659in}{2.983398in}}%
\pgfpathlineto{\pgfqpoint{6.790831in}{2.983563in}}%
\pgfpathlineto{\pgfqpoint{6.794003in}{2.983913in}}%
\pgfpathlineto{\pgfqpoint{6.797175in}{2.983997in}}%
\pgfpathlineto{\pgfqpoint{6.800347in}{2.983721in}}%
\pgfpathlineto{\pgfqpoint{6.803519in}{2.983469in}}%
\pgfpathlineto{\pgfqpoint{6.806692in}{2.983689in}}%
\pgfpathlineto{\pgfqpoint{6.809864in}{2.983743in}}%
\pgfpathlineto{\pgfqpoint{6.813036in}{2.983374in}}%
\pgfpathlineto{\pgfqpoint{6.816208in}{2.983109in}}%
\pgfpathlineto{\pgfqpoint{6.819380in}{2.982922in}}%
\pgfpathlineto{\pgfqpoint{6.822552in}{2.982974in}}%
\pgfpathlineto{\pgfqpoint{6.825724in}{2.982916in}}%
\pgfpathlineto{\pgfqpoint{6.828896in}{2.982652in}}%
\pgfpathlineto{\pgfqpoint{6.832068in}{2.982474in}}%
\pgfpathlineto{\pgfqpoint{6.835240in}{2.982390in}}%
\pgfpathlineto{\pgfqpoint{6.838412in}{2.982874in}}%
\pgfpathlineto{\pgfqpoint{6.841584in}{2.982856in}}%
\pgfpathlineto{\pgfqpoint{6.844756in}{2.982619in}}%
\pgfpathlineto{\pgfqpoint{6.847928in}{2.982250in}}%
\pgfpathlineto{\pgfqpoint{6.851100in}{2.982265in}}%
\pgfpathlineto{\pgfqpoint{6.854272in}{2.982277in}}%
\pgfpathlineto{\pgfqpoint{6.857444in}{2.982264in}}%
\pgfpathlineto{\pgfqpoint{6.860616in}{2.982003in}}%
\pgfpathlineto{\pgfqpoint{6.863788in}{2.981807in}}%
\pgfpathlineto{\pgfqpoint{6.866960in}{2.981877in}}%
\pgfpathlineto{\pgfqpoint{6.870132in}{2.981646in}}%
\pgfpathlineto{\pgfqpoint{6.873304in}{2.981361in}}%
\pgfpathlineto{\pgfqpoint{6.876476in}{2.980882in}}%
\pgfpathlineto{\pgfqpoint{6.879648in}{2.981201in}}%
\pgfpathlineto{\pgfqpoint{6.882820in}{2.980871in}}%
\pgfpathlineto{\pgfqpoint{6.885993in}{2.980734in}}%
\pgfpathlineto{\pgfqpoint{6.889165in}{2.980566in}}%
\pgfpathlineto{\pgfqpoint{6.892337in}{2.980643in}}%
\pgfpathlineto{\pgfqpoint{6.895509in}{2.980503in}}%
\pgfpathlineto{\pgfqpoint{6.898681in}{2.980912in}}%
\pgfpathlineto{\pgfqpoint{6.901853in}{2.981085in}}%
\pgfpathlineto{\pgfqpoint{6.905025in}{2.980476in}}%
\pgfpathlineto{\pgfqpoint{6.908197in}{2.980545in}}%
\pgfpathlineto{\pgfqpoint{6.911369in}{2.980054in}}%
\pgfpathlineto{\pgfqpoint{6.914541in}{2.979965in}}%
\pgfpathlineto{\pgfqpoint{6.917713in}{2.980083in}}%
\pgfpathlineto{\pgfqpoint{6.920885in}{2.979774in}}%
\pgfpathlineto{\pgfqpoint{6.924057in}{2.979736in}}%
\pgfpathlineto{\pgfqpoint{6.927229in}{2.979440in}}%
\pgfpathlineto{\pgfqpoint{6.930401in}{2.979567in}}%
\pgfpathlineto{\pgfqpoint{6.933573in}{2.979977in}}%
\pgfpathlineto{\pgfqpoint{6.936745in}{2.979609in}}%
\pgfpathlineto{\pgfqpoint{6.939917in}{2.979563in}}%
\pgfpathlineto{\pgfqpoint{6.943089in}{2.979498in}}%
\pgfpathlineto{\pgfqpoint{6.946261in}{2.978860in}}%
\pgfpathlineto{\pgfqpoint{6.949433in}{2.978688in}}%
\pgfpathlineto{\pgfqpoint{6.952605in}{2.978634in}}%
\pgfpathlineto{\pgfqpoint{6.955777in}{2.978504in}}%
\pgfpathlineto{\pgfqpoint{6.958949in}{2.978855in}}%
\pgfpathlineto{\pgfqpoint{6.962122in}{2.978903in}}%
\pgfpathlineto{\pgfqpoint{6.965294in}{2.978932in}}%
\pgfpathlineto{\pgfqpoint{6.968466in}{2.978570in}}%
\pgfpathlineto{\pgfqpoint{6.971638in}{2.978728in}}%
\pgfpathlineto{\pgfqpoint{6.974810in}{2.978421in}}%
\pgfpathlineto{\pgfqpoint{6.977982in}{2.978468in}}%
\pgfpathlineto{\pgfqpoint{6.981154in}{2.978340in}}%
\pgfpathlineto{\pgfqpoint{6.984326in}{2.978351in}}%
\pgfpathlineto{\pgfqpoint{6.987498in}{2.977976in}}%
\pgfpathlineto{\pgfqpoint{6.990670in}{2.977673in}}%
\pgfpathlineto{\pgfqpoint{6.993842in}{2.977199in}}%
\pgfpathlineto{\pgfqpoint{6.997014in}{2.977091in}}%
\pgfpathlineto{\pgfqpoint{7.000186in}{2.977069in}}%
\pgfpathlineto{\pgfqpoint{7.003358in}{2.977119in}}%
\pgfpathlineto{\pgfqpoint{7.006530in}{2.976997in}}%
\pgfpathlineto{\pgfqpoint{7.009702in}{2.976941in}}%
\pgfpathlineto{\pgfqpoint{7.012874in}{2.976595in}}%
\pgfpathlineto{\pgfqpoint{7.016046in}{2.976185in}}%
\pgfpathlineto{\pgfqpoint{7.019218in}{2.976044in}}%
\pgfpathlineto{\pgfqpoint{7.022390in}{2.976113in}}%
\pgfpathlineto{\pgfqpoint{7.025562in}{2.976129in}}%
\pgfpathlineto{\pgfqpoint{7.028734in}{2.976485in}}%
\pgfpathlineto{\pgfqpoint{7.031906in}{2.976377in}}%
\pgfpathlineto{\pgfqpoint{7.035078in}{2.976631in}}%
\pgfpathlineto{\pgfqpoint{7.038250in}{2.976361in}}%
\pgfpathlineto{\pgfqpoint{7.041423in}{2.976631in}}%
\pgfpathlineto{\pgfqpoint{7.044595in}{2.976285in}}%
\pgfpathlineto{\pgfqpoint{7.047767in}{2.976343in}}%
\pgfpathlineto{\pgfqpoint{7.050939in}{2.976292in}}%
\pgfpathlineto{\pgfqpoint{7.054111in}{2.976026in}}%
\pgfpathlineto{\pgfqpoint{7.057283in}{2.976279in}}%
\pgfpathlineto{\pgfqpoint{7.060455in}{2.976462in}}%
\pgfpathlineto{\pgfqpoint{7.063627in}{2.976850in}}%
\pgfpathlineto{\pgfqpoint{7.066799in}{2.976453in}}%
\pgfpathlineto{\pgfqpoint{7.069971in}{2.976308in}}%
\pgfpathlineto{\pgfqpoint{7.073143in}{2.976136in}}%
\pgfpathlineto{\pgfqpoint{7.076315in}{2.976091in}}%
\pgfpathlineto{\pgfqpoint{7.079487in}{2.976033in}}%
\pgfpathlineto{\pgfqpoint{7.082659in}{2.975795in}}%
\pgfpathlineto{\pgfqpoint{7.085831in}{2.975719in}}%
\pgfpathlineto{\pgfqpoint{7.089003in}{2.975778in}}%
\pgfpathlineto{\pgfqpoint{7.092175in}{2.975436in}}%
\pgfpathlineto{\pgfqpoint{7.095347in}{2.974891in}}%
\pgfpathlineto{\pgfqpoint{7.098519in}{2.974395in}}%
\pgfpathlineto{\pgfqpoint{7.101691in}{2.974353in}}%
\pgfpathlineto{\pgfqpoint{7.104863in}{2.974320in}}%
\pgfpathlineto{\pgfqpoint{7.108035in}{2.974528in}}%
\pgfpathlineto{\pgfqpoint{7.111207in}{2.974173in}}%
\pgfpathlineto{\pgfqpoint{7.114379in}{2.974337in}}%
\pgfpathlineto{\pgfqpoint{7.117551in}{2.974588in}}%
\pgfpathlineto{\pgfqpoint{7.120724in}{2.974272in}}%
\pgfpathlineto{\pgfqpoint{7.123896in}{2.974722in}}%
\pgfpathlineto{\pgfqpoint{7.127068in}{2.974480in}}%
\pgfpathlineto{\pgfqpoint{7.130240in}{2.974314in}}%
\pgfpathlineto{\pgfqpoint{7.133412in}{2.974041in}}%
\pgfpathlineto{\pgfqpoint{7.136584in}{2.973384in}}%
\pgfpathlineto{\pgfqpoint{7.139756in}{2.973351in}}%
\pgfpathlineto{\pgfqpoint{7.142928in}{2.973367in}}%
\pgfpathlineto{\pgfqpoint{7.146100in}{2.973127in}}%
\pgfpathlineto{\pgfqpoint{7.149272in}{2.973053in}}%
\pgfpathlineto{\pgfqpoint{7.152444in}{2.973207in}}%
\pgfpathlineto{\pgfqpoint{7.155616in}{2.973064in}}%
\pgfpathlineto{\pgfqpoint{7.158788in}{2.973002in}}%
\pgfpathlineto{\pgfqpoint{7.161960in}{2.972874in}}%
\pgfpathlineto{\pgfqpoint{7.165132in}{2.972679in}}%
\pgfpathlineto{\pgfqpoint{7.168304in}{2.972464in}}%
\pgfpathlineto{\pgfqpoint{7.171476in}{2.971999in}}%
\pgfpathlineto{\pgfqpoint{7.174648in}{2.971865in}}%
\pgfpathlineto{\pgfqpoint{7.177820in}{2.971634in}}%
\pgfpathlineto{\pgfqpoint{7.180992in}{2.971396in}}%
\pgfpathlineto{\pgfqpoint{7.184164in}{2.971124in}}%
\pgfpathlineto{\pgfqpoint{7.187336in}{2.970959in}}%
\pgfpathlineto{\pgfqpoint{7.190508in}{2.971038in}}%
\pgfpathlineto{\pgfqpoint{7.193680in}{2.971293in}}%
\pgfpathlineto{\pgfqpoint{7.196853in}{2.970948in}}%
\pgfpathlineto{\pgfqpoint{7.200025in}{2.971217in}}%
\pgfpathlineto{\pgfqpoint{7.203197in}{2.971274in}}%
\pgfpathlineto{\pgfqpoint{7.206369in}{2.971731in}}%
\pgfpathlineto{\pgfqpoint{7.209541in}{2.971947in}}%
\pgfpathlineto{\pgfqpoint{7.212713in}{2.971158in}}%
\pgfpathlineto{\pgfqpoint{7.215885in}{2.970711in}}%
\pgfpathlineto{\pgfqpoint{7.219057in}{2.970736in}}%
\pgfpathlineto{\pgfqpoint{7.222229in}{2.970610in}}%
\pgfpathlineto{\pgfqpoint{7.225401in}{2.970456in}}%
\pgfpathlineto{\pgfqpoint{7.228573in}{2.970444in}}%
\pgfpathlineto{\pgfqpoint{7.231745in}{2.970530in}}%
\pgfpathlineto{\pgfqpoint{7.234917in}{2.970207in}}%
\pgfpathlineto{\pgfqpoint{7.238089in}{2.970455in}}%
\pgfpathlineto{\pgfqpoint{7.241261in}{2.970570in}}%
\pgfpathlineto{\pgfqpoint{7.244433in}{2.970494in}}%
\pgfpathlineto{\pgfqpoint{7.247605in}{2.970073in}}%
\pgfpathlineto{\pgfqpoint{7.250777in}{2.970045in}}%
\pgfpathlineto{\pgfqpoint{7.253949in}{2.970422in}}%
\pgfpathlineto{\pgfqpoint{7.257121in}{2.970273in}}%
\pgfpathlineto{\pgfqpoint{7.260293in}{2.970341in}}%
\pgfpathlineto{\pgfqpoint{7.263465in}{2.970337in}}%
\pgfpathlineto{\pgfqpoint{7.266637in}{2.970056in}}%
\pgfpathlineto{\pgfqpoint{7.269809in}{2.970322in}}%
\pgfpathlineto{\pgfqpoint{7.272981in}{2.970234in}}%
\pgfpathlineto{\pgfqpoint{7.276154in}{2.970399in}}%
\pgfpathlineto{\pgfqpoint{7.279326in}{2.970589in}}%
\pgfpathlineto{\pgfqpoint{7.282498in}{2.971009in}}%
\pgfpathlineto{\pgfqpoint{7.285670in}{2.971030in}}%
\pgfpathlineto{\pgfqpoint{7.288842in}{2.970851in}}%
\pgfpathlineto{\pgfqpoint{7.292014in}{2.971067in}}%
\pgfpathlineto{\pgfqpoint{7.295186in}{2.970891in}}%
\pgfpathlineto{\pgfqpoint{7.298358in}{2.971048in}}%
\pgfpathlineto{\pgfqpoint{7.301530in}{2.970932in}}%
\pgfpathlineto{\pgfqpoint{7.304702in}{2.970820in}}%
\pgfpathlineto{\pgfqpoint{7.307874in}{2.969760in}}%
\pgfpathlineto{\pgfqpoint{7.311046in}{2.969218in}}%
\pgfpathlineto{\pgfqpoint{7.314218in}{2.968944in}}%
\pgfpathlineto{\pgfqpoint{7.317390in}{2.968459in}}%
\pgfpathlineto{\pgfqpoint{7.320562in}{2.968312in}}%
\pgfpathlineto{\pgfqpoint{7.323734in}{2.968220in}}%
\pgfpathlineto{\pgfqpoint{7.326906in}{2.968006in}}%
\pgfpathlineto{\pgfqpoint{7.330078in}{2.967422in}}%
\pgfpathlineto{\pgfqpoint{7.333250in}{2.967232in}}%
\pgfpathlineto{\pgfqpoint{7.336422in}{2.967209in}}%
\pgfpathlineto{\pgfqpoint{7.339594in}{2.967014in}}%
\pgfpathlineto{\pgfqpoint{7.342766in}{2.966368in}}%
\pgfpathlineto{\pgfqpoint{7.345938in}{2.966344in}}%
\pgfpathlineto{\pgfqpoint{7.349110in}{2.966322in}}%
\pgfpathlineto{\pgfqpoint{7.352282in}{2.966387in}}%
\pgfpathlineto{\pgfqpoint{7.355455in}{2.966329in}}%
\pgfpathlineto{\pgfqpoint{7.358627in}{2.966682in}}%
\pgfpathlineto{\pgfqpoint{7.361799in}{2.966604in}}%
\pgfpathlineto{\pgfqpoint{7.364971in}{2.966878in}}%
\pgfpathlineto{\pgfqpoint{7.368143in}{2.966532in}}%
\pgfpathlineto{\pgfqpoint{7.371315in}{2.966567in}}%
\pgfpathlineto{\pgfqpoint{7.374487in}{2.966473in}}%
\pgfpathlineto{\pgfqpoint{7.377659in}{2.965880in}}%
\pgfpathlineto{\pgfqpoint{7.380831in}{2.966264in}}%
\pgfpathlineto{\pgfqpoint{7.384003in}{2.965937in}}%
\pgfpathlineto{\pgfqpoint{7.387175in}{2.965694in}}%
\pgfpathlineto{\pgfqpoint{7.390347in}{2.965931in}}%
\pgfpathlineto{\pgfqpoint{7.393519in}{2.965756in}}%
\pgfpathlineto{\pgfqpoint{7.396691in}{2.965600in}}%
\pgfpathlineto{\pgfqpoint{7.399863in}{2.965296in}}%
\pgfpathlineto{\pgfqpoint{7.403035in}{2.964894in}}%
\pgfpathlineto{\pgfqpoint{7.406207in}{2.964877in}}%
\pgfpathlineto{\pgfqpoint{7.409379in}{2.965247in}}%
\pgfpathlineto{\pgfqpoint{7.412551in}{2.964650in}}%
\pgfpathlineto{\pgfqpoint{7.415723in}{2.964401in}}%
\pgfpathlineto{\pgfqpoint{7.418895in}{2.964359in}}%
\pgfpathlineto{\pgfqpoint{7.422067in}{2.964327in}}%
\pgfpathlineto{\pgfqpoint{7.425239in}{2.964259in}}%
\pgfpathlineto{\pgfqpoint{7.428411in}{2.964274in}}%
\pgfpathlineto{\pgfqpoint{7.431584in}{2.964359in}}%
\pgfpathlineto{\pgfqpoint{7.434756in}{2.964350in}}%
\pgfpathlineto{\pgfqpoint{7.437928in}{2.964004in}}%
\pgfpathlineto{\pgfqpoint{7.441100in}{2.964485in}}%
\pgfpathlineto{\pgfqpoint{7.444272in}{2.964632in}}%
\pgfpathlineto{\pgfqpoint{7.447444in}{2.964501in}}%
\pgfpathlineto{\pgfqpoint{7.450616in}{2.964771in}}%
\pgfpathlineto{\pgfqpoint{7.453788in}{2.964703in}}%
\pgfpathlineto{\pgfqpoint{7.456960in}{2.964485in}}%
\pgfpathlineto{\pgfqpoint{7.460132in}{2.964412in}}%
\pgfpathlineto{\pgfqpoint{7.463304in}{2.964470in}}%
\pgfpathlineto{\pgfqpoint{7.466476in}{2.964528in}}%
\pgfpathlineto{\pgfqpoint{7.469648in}{2.964376in}}%
\pgfpathlineto{\pgfqpoint{7.472820in}{2.964638in}}%
\pgfpathlineto{\pgfqpoint{7.475992in}{2.964511in}}%
\pgfpathlineto{\pgfqpoint{7.479164in}{2.964519in}}%
\pgfpathlineto{\pgfqpoint{7.482336in}{2.964488in}}%
\pgfpathlineto{\pgfqpoint{7.485508in}{2.964609in}}%
\pgfpathlineto{\pgfqpoint{7.488680in}{2.964251in}}%
\pgfpathlineto{\pgfqpoint{7.491852in}{2.963893in}}%
\pgfpathlineto{\pgfqpoint{7.495024in}{2.964121in}}%
\pgfpathlineto{\pgfqpoint{7.498196in}{2.963751in}}%
\pgfpathlineto{\pgfqpoint{7.501368in}{2.963610in}}%
\pgfpathlineto{\pgfqpoint{7.504540in}{2.963950in}}%
\pgfpathlineto{\pgfqpoint{7.507712in}{2.963799in}}%
\pgfpathlineto{\pgfqpoint{7.510885in}{2.963725in}}%
\pgfpathlineto{\pgfqpoint{7.514057in}{2.963377in}}%
\pgfpathlineto{\pgfqpoint{7.517229in}{2.962636in}}%
\pgfpathlineto{\pgfqpoint{7.520401in}{2.962005in}}%
\pgfpathlineto{\pgfqpoint{7.523573in}{2.961963in}}%
\pgfpathlineto{\pgfqpoint{7.526745in}{2.961402in}}%
\pgfpathlineto{\pgfqpoint{7.529917in}{2.961261in}}%
\pgfpathlineto{\pgfqpoint{7.533089in}{2.960676in}}%
\pgfpathlineto{\pgfqpoint{7.536261in}{2.960255in}}%
\pgfpathlineto{\pgfqpoint{7.539433in}{2.960806in}}%
\pgfpathlineto{\pgfqpoint{7.542605in}{2.960768in}}%
\pgfpathlineto{\pgfqpoint{7.545777in}{2.960561in}}%
\pgfpathlineto{\pgfqpoint{7.548949in}{2.960465in}}%
\pgfpathlineto{\pgfqpoint{7.552121in}{2.960385in}}%
\pgfpathlineto{\pgfqpoint{7.555293in}{2.960803in}}%
\pgfpathlineto{\pgfqpoint{7.558465in}{2.960733in}}%
\pgfpathlineto{\pgfqpoint{7.561637in}{2.960981in}}%
\pgfpathlineto{\pgfqpoint{7.564809in}{2.960760in}}%
\pgfpathlineto{\pgfqpoint{7.567981in}{2.960475in}}%
\pgfpathlineto{\pgfqpoint{7.571153in}{2.960247in}}%
\pgfpathlineto{\pgfqpoint{7.574325in}{2.960186in}}%
\pgfpathlineto{\pgfqpoint{7.577497in}{2.960400in}}%
\pgfpathlineto{\pgfqpoint{7.580669in}{2.960747in}}%
\pgfpathlineto{\pgfqpoint{7.583841in}{2.960679in}}%
\pgfpathlineto{\pgfqpoint{7.587013in}{2.960525in}}%
\pgfpathlineto{\pgfqpoint{7.590186in}{2.960771in}}%
\pgfpathlineto{\pgfqpoint{7.593358in}{2.960595in}}%
\pgfpathlineto{\pgfqpoint{7.596530in}{2.960004in}}%
\pgfpathlineto{\pgfqpoint{7.599702in}{2.959886in}}%
\pgfpathlineto{\pgfqpoint{7.602874in}{2.959879in}}%
\pgfpathlineto{\pgfqpoint{7.606046in}{2.959860in}}%
\pgfpathlineto{\pgfqpoint{7.609218in}{2.959509in}}%
\pgfpathlineto{\pgfqpoint{7.612390in}{2.959761in}}%
\pgfpathlineto{\pgfqpoint{7.615562in}{2.959628in}}%
\pgfpathlineto{\pgfqpoint{7.618734in}{2.959123in}}%
\pgfpathlineto{\pgfqpoint{7.621906in}{2.959231in}}%
\pgfpathlineto{\pgfqpoint{7.625078in}{2.959423in}}%
\pgfpathlineto{\pgfqpoint{7.628250in}{2.959220in}}%
\pgfpathlineto{\pgfqpoint{7.631422in}{2.959153in}}%
\pgfpathlineto{\pgfqpoint{7.634594in}{2.959200in}}%
\pgfpathlineto{\pgfqpoint{7.637766in}{2.958845in}}%
\pgfpathlineto{\pgfqpoint{7.640938in}{2.958837in}}%
\pgfpathlineto{\pgfqpoint{7.644110in}{2.958876in}}%
\pgfpathlineto{\pgfqpoint{7.647282in}{2.958823in}}%
\pgfpathlineto{\pgfqpoint{7.650454in}{2.959103in}}%
\pgfpathlineto{\pgfqpoint{7.653626in}{2.959024in}}%
\pgfpathlineto{\pgfqpoint{7.656798in}{2.958914in}}%
\pgfpathlineto{\pgfqpoint{7.659970in}{2.958859in}}%
\pgfpathlineto{\pgfqpoint{7.663142in}{2.958631in}}%
\pgfpathlineto{\pgfqpoint{7.666315in}{2.958834in}}%
\pgfpathlineto{\pgfqpoint{7.669487in}{2.958801in}}%
\pgfpathlineto{\pgfqpoint{7.672659in}{2.958905in}}%
\pgfpathlineto{\pgfqpoint{7.675831in}{2.959163in}}%
\pgfpathlineto{\pgfqpoint{7.679003in}{2.959507in}}%
\pgfpathlineto{\pgfqpoint{7.682175in}{2.959501in}}%
\pgfpathlineto{\pgfqpoint{7.685347in}{2.959452in}}%
\pgfpathlineto{\pgfqpoint{7.688519in}{2.959446in}}%
\pgfpathlineto{\pgfqpoint{7.691691in}{2.959476in}}%
\pgfpathlineto{\pgfqpoint{7.694863in}{2.959452in}}%
\pgfpathlineto{\pgfqpoint{7.698035in}{2.959546in}}%
\pgfpathlineto{\pgfqpoint{7.701207in}{2.959759in}}%
\pgfpathlineto{\pgfqpoint{7.704379in}{2.959590in}}%
\pgfpathlineto{\pgfqpoint{7.707551in}{2.959008in}}%
\pgfpathlineto{\pgfqpoint{7.710723in}{2.958401in}}%
\pgfpathlineto{\pgfqpoint{7.713895in}{2.958732in}}%
\pgfpathlineto{\pgfqpoint{7.717067in}{2.959069in}}%
\pgfpathlineto{\pgfqpoint{7.720239in}{2.959215in}}%
\pgfpathlineto{\pgfqpoint{7.723411in}{2.959129in}}%
\pgfpathlineto{\pgfqpoint{7.726583in}{2.959044in}}%
\pgfpathlineto{\pgfqpoint{7.729755in}{2.959101in}}%
\pgfpathlineto{\pgfqpoint{7.732927in}{2.958598in}}%
\pgfpathlineto{\pgfqpoint{7.736099in}{2.958465in}}%
\pgfpathlineto{\pgfqpoint{7.739271in}{2.958020in}}%
\pgfpathlineto{\pgfqpoint{7.742443in}{2.957137in}}%
\pgfpathlineto{\pgfqpoint{7.745616in}{2.956691in}}%
\pgfpathlineto{\pgfqpoint{7.748788in}{2.956764in}}%
\pgfpathlineto{\pgfqpoint{7.751960in}{2.956476in}}%
\pgfpathlineto{\pgfqpoint{7.755132in}{2.956134in}}%
\pgfpathlineto{\pgfqpoint{7.758304in}{2.955866in}}%
\pgfpathlineto{\pgfqpoint{7.761476in}{2.956343in}}%
\pgfpathlineto{\pgfqpoint{7.764648in}{2.956258in}}%
\pgfpathlineto{\pgfqpoint{7.767820in}{2.956701in}}%
\pgfpathlineto{\pgfqpoint{7.770992in}{2.956675in}}%
\pgfpathlineto{\pgfqpoint{7.774164in}{2.956580in}}%
\pgfpathlineto{\pgfqpoint{7.777336in}{2.956539in}}%
\pgfpathlineto{\pgfqpoint{7.780508in}{2.956658in}}%
\pgfpathlineto{\pgfqpoint{7.783680in}{2.956809in}}%
\pgfpathlineto{\pgfqpoint{7.786852in}{2.956425in}}%
\pgfpathlineto{\pgfqpoint{7.790024in}{2.959371in}}%
\pgfpathlineto{\pgfqpoint{7.793196in}{2.962462in}}%
\pgfpathlineto{\pgfqpoint{7.796368in}{2.965568in}}%
\pgfpathlineto{\pgfqpoint{7.799540in}{2.968670in}}%
\pgfpathlineto{\pgfqpoint{7.802712in}{2.971686in}}%
\pgfpathlineto{\pgfqpoint{7.805884in}{2.974809in}}%
\pgfpathlineto{\pgfqpoint{7.809056in}{2.977827in}}%
\pgfpathlineto{\pgfqpoint{7.812228in}{2.980900in}}%
\pgfpathlineto{\pgfqpoint{7.815400in}{2.983990in}}%
\pgfpathlineto{\pgfqpoint{7.818572in}{2.987073in}}%
\pgfpathlineto{\pgfqpoint{7.821744in}{2.990217in}}%
\pgfpathlineto{\pgfqpoint{7.824917in}{2.993349in}}%
\pgfpathlineto{\pgfqpoint{7.828089in}{2.996322in}}%
\pgfpathlineto{\pgfqpoint{7.831261in}{2.999359in}}%
\pgfpathlineto{\pgfqpoint{7.834433in}{3.002423in}}%
\pgfpathlineto{\pgfqpoint{7.837605in}{3.005520in}}%
\pgfpathlineto{\pgfqpoint{7.840777in}{3.008575in}}%
\pgfpathlineto{\pgfqpoint{7.843949in}{3.011638in}}%
\pgfpathlineto{\pgfqpoint{7.847121in}{3.014726in}}%
\pgfpathlineto{\pgfqpoint{7.850293in}{3.017754in}}%
\pgfpathlineto{\pgfqpoint{7.853465in}{3.020837in}}%
\pgfpathlineto{\pgfqpoint{7.856637in}{3.023901in}}%
\pgfpathlineto{\pgfqpoint{7.859809in}{3.026978in}}%
\pgfpathlineto{\pgfqpoint{7.862981in}{3.030082in}}%
\pgfpathlineto{\pgfqpoint{7.866153in}{3.033223in}}%
\pgfpathlineto{\pgfqpoint{7.869325in}{3.036289in}}%
\pgfpathlineto{\pgfqpoint{7.872497in}{3.039383in}}%
\pgfpathlineto{\pgfqpoint{7.875669in}{3.042463in}}%
\pgfpathlineto{\pgfqpoint{7.878841in}{3.045559in}}%
\pgfpathlineto{\pgfqpoint{7.882013in}{3.048677in}}%
\pgfpathlineto{\pgfqpoint{7.885185in}{3.051696in}}%
\pgfpathlineto{\pgfqpoint{7.888357in}{3.054778in}}%
\pgfpathlineto{\pgfqpoint{7.891529in}{3.057718in}}%
\pgfpathlineto{\pgfqpoint{7.894701in}{3.060779in}}%
\pgfpathlineto{\pgfqpoint{7.897873in}{3.063570in}}%
\pgfpathlineto{\pgfqpoint{7.901046in}{3.066901in}}%
\pgfpathlineto{\pgfqpoint{7.904218in}{3.070195in}}%
\pgfpathlineto{\pgfqpoint{7.907390in}{3.073222in}}%
\pgfpathlineto{\pgfqpoint{7.910562in}{3.076330in}}%
\pgfpathlineto{\pgfqpoint{7.913734in}{3.079400in}}%
\pgfpathlineto{\pgfqpoint{7.916906in}{3.082503in}}%
\pgfpathlineto{\pgfqpoint{7.920078in}{3.085580in}}%
\pgfpathlineto{\pgfqpoint{7.923250in}{3.088691in}}%
\pgfpathlineto{\pgfqpoint{7.926422in}{3.091786in}}%
\pgfpathlineto{\pgfqpoint{7.929594in}{3.094819in}}%
\pgfpathlineto{\pgfqpoint{7.932766in}{3.097890in}}%
\pgfpathlineto{\pgfqpoint{7.935938in}{3.100982in}}%
\pgfpathlineto{\pgfqpoint{7.939110in}{3.104075in}}%
\pgfpathlineto{\pgfqpoint{7.942282in}{3.107145in}}%
\pgfpathlineto{\pgfqpoint{7.945454in}{3.110250in}}%
\pgfpathlineto{\pgfqpoint{7.948626in}{3.113323in}}%
\pgfpathlineto{\pgfqpoint{7.951798in}{3.116310in}}%
\pgfpathlineto{\pgfqpoint{7.954970in}{3.119396in}}%
\pgfpathlineto{\pgfqpoint{7.958142in}{3.122459in}}%
\pgfpathlineto{\pgfqpoint{7.961314in}{3.125574in}}%
\pgfpathlineto{\pgfqpoint{7.964486in}{3.128728in}}%
\pgfpathlineto{\pgfqpoint{7.967658in}{3.131819in}}%
\pgfpathlineto{\pgfqpoint{7.970830in}{3.134905in}}%
\pgfpathlineto{\pgfqpoint{7.974002in}{3.137993in}}%
\pgfpathlineto{\pgfqpoint{7.977174in}{3.141124in}}%
\pgfpathlineto{\pgfqpoint{7.980347in}{3.144211in}}%
\pgfpathlineto{\pgfqpoint{7.983519in}{3.147263in}}%
\pgfpathlineto{\pgfqpoint{7.986691in}{3.150354in}}%
\pgfpathlineto{\pgfqpoint{7.989863in}{3.153427in}}%
\pgfpathlineto{\pgfqpoint{7.993035in}{3.156497in}}%
\pgfpathlineto{\pgfqpoint{7.996207in}{3.159588in}}%
\pgfpathlineto{\pgfqpoint{7.999379in}{3.162432in}}%
\pgfpathlineto{\pgfqpoint{8.002551in}{3.165264in}}%
\pgfpathlineto{\pgfqpoint{8.005723in}{3.168360in}}%
\pgfpathlineto{\pgfqpoint{8.008895in}{3.171395in}}%
\pgfpathlineto{\pgfqpoint{8.012067in}{3.174466in}}%
\pgfpathlineto{\pgfqpoint{8.015239in}{3.177505in}}%
\pgfpathlineto{\pgfqpoint{8.018411in}{3.180584in}}%
\pgfpathlineto{\pgfqpoint{8.021583in}{3.183649in}}%
\pgfpathlineto{\pgfqpoint{8.024755in}{3.186730in}}%
\pgfpathlineto{\pgfqpoint{8.027927in}{3.189745in}}%
\pgfpathlineto{\pgfqpoint{8.031099in}{3.192784in}}%
\pgfpathlineto{\pgfqpoint{8.034271in}{3.195860in}}%
\pgfpathlineto{\pgfqpoint{8.037443in}{3.198920in}}%
\pgfpathlineto{\pgfqpoint{8.040615in}{3.202016in}}%
\pgfpathlineto{\pgfqpoint{8.043787in}{3.205130in}}%
\pgfpathlineto{\pgfqpoint{8.046959in}{3.208201in}}%
\pgfpathlineto{\pgfqpoint{8.050131in}{3.211351in}}%
\pgfpathlineto{\pgfqpoint{8.053303in}{3.214429in}}%
\pgfpathlineto{\pgfqpoint{8.056475in}{3.217493in}}%
\pgfpathlineto{\pgfqpoint{8.059648in}{3.220539in}}%
\pgfpathlineto{\pgfqpoint{8.062820in}{3.223608in}}%
\pgfpathlineto{\pgfqpoint{8.065992in}{3.226722in}}%
\pgfpathlineto{\pgfqpoint{8.069164in}{3.229842in}}%
\pgfpathlineto{\pgfqpoint{8.072336in}{3.232946in}}%
\pgfpathlineto{\pgfqpoint{8.075508in}{3.235969in}}%
\pgfpathlineto{\pgfqpoint{8.078680in}{3.239061in}}%
\pgfpathlineto{\pgfqpoint{8.081852in}{3.242205in}}%
\pgfpathlineto{\pgfqpoint{8.085024in}{3.245339in}}%
\pgfpathlineto{\pgfqpoint{8.088196in}{3.248471in}}%
\pgfpathlineto{\pgfqpoint{8.091368in}{3.251512in}}%
\pgfpathlineto{\pgfqpoint{8.094540in}{3.254637in}}%
\pgfpathlineto{\pgfqpoint{8.097712in}{3.257749in}}%
\pgfpathlineto{\pgfqpoint{8.100884in}{3.260935in}}%
\pgfpathlineto{\pgfqpoint{8.104056in}{3.263998in}}%
\pgfpathlineto{\pgfqpoint{8.107228in}{3.267033in}}%
\pgfpathlineto{\pgfqpoint{8.110400in}{3.270118in}}%
\pgfpathlineto{\pgfqpoint{8.113572in}{3.273206in}}%
\pgfpathlineto{\pgfqpoint{8.116744in}{3.276327in}}%
\pgfpathlineto{\pgfqpoint{8.119916in}{3.279391in}}%
\pgfpathlineto{\pgfqpoint{8.123088in}{3.282510in}}%
\pgfpathlineto{\pgfqpoint{8.126260in}{3.285599in}}%
\pgfpathlineto{\pgfqpoint{8.129432in}{3.288835in}}%
\pgfpathlineto{\pgfqpoint{8.132604in}{3.291853in}}%
\pgfpathlineto{\pgfqpoint{8.135777in}{3.294929in}}%
\pgfpathlineto{\pgfqpoint{8.138949in}{3.297992in}}%
\pgfpathlineto{\pgfqpoint{8.142121in}{3.301215in}}%
\pgfpathlineto{\pgfqpoint{8.145293in}{3.304315in}}%
\pgfpathlineto{\pgfqpoint{8.148465in}{3.307418in}}%
\pgfpathlineto{\pgfqpoint{8.151637in}{3.310511in}}%
\pgfpathlineto{\pgfqpoint{8.154809in}{3.313603in}}%
\pgfpathlineto{\pgfqpoint{8.157981in}{3.316554in}}%
\pgfpathlineto{\pgfqpoint{8.161153in}{3.319473in}}%
\pgfpathlineto{\pgfqpoint{8.164325in}{3.322353in}}%
\pgfpathlineto{\pgfqpoint{8.167497in}{3.325281in}}%
\pgfpathlineto{\pgfqpoint{8.170669in}{3.328194in}}%
\pgfpathlineto{\pgfqpoint{8.173841in}{3.331184in}}%
\pgfpathlineto{\pgfqpoint{8.177013in}{3.334140in}}%
\pgfpathlineto{\pgfqpoint{8.180185in}{3.336931in}}%
\pgfpathlineto{\pgfqpoint{8.183357in}{3.339889in}}%
\pgfpathlineto{\pgfqpoint{8.186529in}{3.342964in}}%
\pgfpathlineto{\pgfqpoint{8.189701in}{3.346045in}}%
\pgfpathlineto{\pgfqpoint{8.192873in}{3.349131in}}%
\pgfpathlineto{\pgfqpoint{8.196045in}{3.352175in}}%
\pgfpathlineto{\pgfqpoint{8.199217in}{3.355326in}}%
\pgfpathlineto{\pgfqpoint{8.202389in}{3.358431in}}%
\pgfpathlineto{\pgfqpoint{8.205561in}{3.361501in}}%
\pgfpathlineto{\pgfqpoint{8.208733in}{3.364669in}}%
\pgfpathlineto{\pgfqpoint{8.211905in}{3.367783in}}%
\pgfpathlineto{\pgfqpoint{8.215078in}{3.370922in}}%
\pgfpathlineto{\pgfqpoint{8.218250in}{3.374072in}}%
\pgfpathlineto{\pgfqpoint{8.221422in}{3.377135in}}%
\pgfpathlineto{\pgfqpoint{8.224594in}{3.380274in}}%
\pgfpathlineto{\pgfqpoint{8.227766in}{3.383359in}}%
\pgfpathlineto{\pgfqpoint{8.230938in}{3.386305in}}%
\pgfpathlineto{\pgfqpoint{8.234110in}{3.389389in}}%
\pgfpathlineto{\pgfqpoint{8.237282in}{3.392430in}}%
\pgfpathlineto{\pgfqpoint{8.240454in}{3.395513in}}%
\pgfpathlineto{\pgfqpoint{8.243626in}{3.398538in}}%
\pgfpathlineto{\pgfqpoint{8.246798in}{3.401618in}}%
\pgfpathlineto{\pgfqpoint{8.249970in}{3.404659in}}%
\pgfpathlineto{\pgfqpoint{8.253142in}{3.407688in}}%
\pgfpathlineto{\pgfqpoint{8.256314in}{3.410634in}}%
\pgfpathlineto{\pgfqpoint{8.259486in}{3.413717in}}%
\pgfpathlineto{\pgfqpoint{8.262658in}{3.416812in}}%
\pgfpathlineto{\pgfqpoint{8.265830in}{3.419910in}}%
\pgfpathlineto{\pgfqpoint{8.269002in}{3.423019in}}%
\pgfpathlineto{\pgfqpoint{8.272174in}{3.426080in}}%
\pgfpathlineto{\pgfqpoint{8.275346in}{3.429082in}}%
\pgfpathlineto{\pgfqpoint{8.278518in}{3.432189in}}%
\pgfpathlineto{\pgfqpoint{8.281690in}{3.435289in}}%
\pgfpathlineto{\pgfqpoint{8.281690in}{3.629790in}}%
\pgfpathlineto{\pgfqpoint{8.281690in}{3.629790in}}%
\pgfpathlineto{\pgfqpoint{8.278518in}{3.626685in}}%
\pgfpathlineto{\pgfqpoint{8.275346in}{3.623532in}}%
\pgfpathlineto{\pgfqpoint{8.272174in}{3.620431in}}%
\pgfpathlineto{\pgfqpoint{8.269002in}{3.617335in}}%
\pgfpathlineto{\pgfqpoint{8.265830in}{3.614293in}}%
\pgfpathlineto{\pgfqpoint{8.262658in}{3.611176in}}%
\pgfpathlineto{\pgfqpoint{8.259486in}{3.608072in}}%
\pgfpathlineto{\pgfqpoint{8.256314in}{3.604966in}}%
\pgfpathlineto{\pgfqpoint{8.253142in}{3.601815in}}%
\pgfpathlineto{\pgfqpoint{8.249970in}{3.598723in}}%
\pgfpathlineto{\pgfqpoint{8.246798in}{3.595612in}}%
\pgfpathlineto{\pgfqpoint{8.243626in}{3.592510in}}%
\pgfpathlineto{\pgfqpoint{8.240454in}{3.589426in}}%
\pgfpathlineto{\pgfqpoint{8.237282in}{3.586325in}}%
\pgfpathlineto{\pgfqpoint{8.234110in}{3.583214in}}%
\pgfpathlineto{\pgfqpoint{8.230938in}{3.580072in}}%
\pgfpathlineto{\pgfqpoint{8.227766in}{3.576949in}}%
\pgfpathlineto{\pgfqpoint{8.224594in}{3.573875in}}%
\pgfpathlineto{\pgfqpoint{8.221422in}{3.570772in}}%
\pgfpathlineto{\pgfqpoint{8.218250in}{3.567674in}}%
\pgfpathlineto{\pgfqpoint{8.215078in}{3.564601in}}%
\pgfpathlineto{\pgfqpoint{8.211905in}{3.561497in}}%
\pgfpathlineto{\pgfqpoint{8.208733in}{3.558357in}}%
\pgfpathlineto{\pgfqpoint{8.205561in}{3.555285in}}%
\pgfpathlineto{\pgfqpoint{8.202389in}{3.552193in}}%
\pgfpathlineto{\pgfqpoint{8.199217in}{3.549079in}}%
\pgfpathlineto{\pgfqpoint{8.196045in}{3.545975in}}%
\pgfpathlineto{\pgfqpoint{8.192873in}{3.542857in}}%
\pgfpathlineto{\pgfqpoint{8.189701in}{3.539756in}}%
\pgfpathlineto{\pgfqpoint{8.186529in}{3.536637in}}%
\pgfpathlineto{\pgfqpoint{8.183357in}{3.533475in}}%
\pgfpathlineto{\pgfqpoint{8.180185in}{3.529770in}}%
\pgfpathlineto{\pgfqpoint{8.177013in}{3.525914in}}%
\pgfpathlineto{\pgfqpoint{8.173841in}{3.522306in}}%
\pgfpathlineto{\pgfqpoint{8.170669in}{3.518791in}}%
\pgfpathlineto{\pgfqpoint{8.167497in}{3.515075in}}%
\pgfpathlineto{\pgfqpoint{8.164325in}{3.511363in}}%
\pgfpathlineto{\pgfqpoint{8.161153in}{3.507475in}}%
\pgfpathlineto{\pgfqpoint{8.157981in}{3.503633in}}%
\pgfpathlineto{\pgfqpoint{8.154809in}{3.500032in}}%
\pgfpathlineto{\pgfqpoint{8.151637in}{3.496913in}}%
\pgfpathlineto{\pgfqpoint{8.148465in}{3.493796in}}%
\pgfpathlineto{\pgfqpoint{8.145293in}{3.490684in}}%
\pgfpathlineto{\pgfqpoint{8.142121in}{3.487604in}}%
\pgfpathlineto{\pgfqpoint{8.138949in}{3.484519in}}%
\pgfpathlineto{\pgfqpoint{8.135777in}{3.481433in}}%
\pgfpathlineto{\pgfqpoint{8.132604in}{3.478359in}}%
\pgfpathlineto{\pgfqpoint{8.129432in}{3.475253in}}%
\pgfpathlineto{\pgfqpoint{8.126260in}{3.472506in}}%
\pgfpathlineto{\pgfqpoint{8.123088in}{3.469381in}}%
\pgfpathlineto{\pgfqpoint{8.119916in}{3.466277in}}%
\pgfpathlineto{\pgfqpoint{8.116744in}{3.463172in}}%
\pgfpathlineto{\pgfqpoint{8.113572in}{3.460074in}}%
\pgfpathlineto{\pgfqpoint{8.110400in}{3.456961in}}%
\pgfpathlineto{\pgfqpoint{8.107228in}{3.453850in}}%
\pgfpathlineto{\pgfqpoint{8.104056in}{3.450719in}}%
\pgfpathlineto{\pgfqpoint{8.100884in}{3.447601in}}%
\pgfpathlineto{\pgfqpoint{8.097712in}{3.444477in}}%
\pgfpathlineto{\pgfqpoint{8.094540in}{3.441407in}}%
\pgfpathlineto{\pgfqpoint{8.091368in}{3.438252in}}%
\pgfpathlineto{\pgfqpoint{8.088196in}{3.435159in}}%
\pgfpathlineto{\pgfqpoint{8.085024in}{3.432061in}}%
\pgfpathlineto{\pgfqpoint{8.081852in}{3.428944in}}%
\pgfpathlineto{\pgfqpoint{8.078680in}{3.425858in}}%
\pgfpathlineto{\pgfqpoint{8.075508in}{3.422736in}}%
\pgfpathlineto{\pgfqpoint{8.072336in}{3.419638in}}%
\pgfpathlineto{\pgfqpoint{8.069164in}{3.416567in}}%
\pgfpathlineto{\pgfqpoint{8.065992in}{3.413469in}}%
\pgfpathlineto{\pgfqpoint{8.062820in}{3.410328in}}%
\pgfpathlineto{\pgfqpoint{8.059648in}{3.407203in}}%
\pgfpathlineto{\pgfqpoint{8.056475in}{3.404092in}}%
\pgfpathlineto{\pgfqpoint{8.053303in}{3.400975in}}%
\pgfpathlineto{\pgfqpoint{8.050131in}{3.397877in}}%
\pgfpathlineto{\pgfqpoint{8.046959in}{3.394786in}}%
\pgfpathlineto{\pgfqpoint{8.043787in}{3.391719in}}%
\pgfpathlineto{\pgfqpoint{8.040615in}{3.388624in}}%
\pgfpathlineto{\pgfqpoint{8.037443in}{3.385513in}}%
\pgfpathlineto{\pgfqpoint{8.034271in}{3.382388in}}%
\pgfpathlineto{\pgfqpoint{8.031099in}{3.379259in}}%
\pgfpathlineto{\pgfqpoint{8.027927in}{3.376168in}}%
\pgfpathlineto{\pgfqpoint{8.024755in}{3.373056in}}%
\pgfpathlineto{\pgfqpoint{8.021583in}{3.369985in}}%
\pgfpathlineto{\pgfqpoint{8.018411in}{3.366920in}}%
\pgfpathlineto{\pgfqpoint{8.015239in}{3.363811in}}%
\pgfpathlineto{\pgfqpoint{8.012067in}{3.360718in}}%
\pgfpathlineto{\pgfqpoint{8.008895in}{3.357612in}}%
\pgfpathlineto{\pgfqpoint{8.005723in}{3.354518in}}%
\pgfpathlineto{\pgfqpoint{8.002551in}{3.351425in}}%
\pgfpathlineto{\pgfqpoint{7.999379in}{3.348265in}}%
\pgfpathlineto{\pgfqpoint{7.996207in}{3.345316in}}%
\pgfpathlineto{\pgfqpoint{7.993035in}{3.342228in}}%
\pgfpathlineto{\pgfqpoint{7.989863in}{3.339110in}}%
\pgfpathlineto{\pgfqpoint{7.986691in}{3.336043in}}%
\pgfpathlineto{\pgfqpoint{7.983519in}{3.332957in}}%
\pgfpathlineto{\pgfqpoint{7.980347in}{3.329877in}}%
\pgfpathlineto{\pgfqpoint{7.977174in}{3.326738in}}%
\pgfpathlineto{\pgfqpoint{7.974002in}{3.323639in}}%
\pgfpathlineto{\pgfqpoint{7.970830in}{3.320531in}}%
\pgfpathlineto{\pgfqpoint{7.967658in}{3.317425in}}%
\pgfpathlineto{\pgfqpoint{7.964486in}{3.314299in}}%
\pgfpathlineto{\pgfqpoint{7.961314in}{3.311134in}}%
\pgfpathlineto{\pgfqpoint{7.958142in}{3.308061in}}%
\pgfpathlineto{\pgfqpoint{7.954970in}{3.304977in}}%
\pgfpathlineto{\pgfqpoint{7.951798in}{3.301909in}}%
\pgfpathlineto{\pgfqpoint{7.948626in}{3.298246in}}%
\pgfpathlineto{\pgfqpoint{7.945454in}{3.295150in}}%
\pgfpathlineto{\pgfqpoint{7.942282in}{3.292038in}}%
\pgfpathlineto{\pgfqpoint{7.939110in}{3.288904in}}%
\pgfpathlineto{\pgfqpoint{7.935938in}{3.285833in}}%
\pgfpathlineto{\pgfqpoint{7.932766in}{3.282762in}}%
\pgfpathlineto{\pgfqpoint{7.929594in}{3.279659in}}%
\pgfpathlineto{\pgfqpoint{7.926422in}{3.276550in}}%
\pgfpathlineto{\pgfqpoint{7.923250in}{3.273510in}}%
\pgfpathlineto{\pgfqpoint{7.920078in}{3.270441in}}%
\pgfpathlineto{\pgfqpoint{7.916906in}{3.267306in}}%
\pgfpathlineto{\pgfqpoint{7.913734in}{3.264234in}}%
\pgfpathlineto{\pgfqpoint{7.910562in}{3.261177in}}%
\pgfpathlineto{\pgfqpoint{7.907390in}{3.258087in}}%
\pgfpathlineto{\pgfqpoint{7.904218in}{3.254996in}}%
\pgfpathlineto{\pgfqpoint{7.901046in}{3.251832in}}%
\pgfpathlineto{\pgfqpoint{7.897873in}{3.248635in}}%
\pgfpathlineto{\pgfqpoint{7.894701in}{3.244618in}}%
\pgfpathlineto{\pgfqpoint{7.891529in}{3.241536in}}%
\pgfpathlineto{\pgfqpoint{7.888357in}{3.238429in}}%
\pgfpathlineto{\pgfqpoint{7.885185in}{3.235413in}}%
\pgfpathlineto{\pgfqpoint{7.882013in}{3.232287in}}%
\pgfpathlineto{\pgfqpoint{7.878841in}{3.229244in}}%
\pgfpathlineto{\pgfqpoint{7.875669in}{3.226166in}}%
\pgfpathlineto{\pgfqpoint{7.872497in}{3.223092in}}%
\pgfpathlineto{\pgfqpoint{7.869325in}{3.220006in}}%
\pgfpathlineto{\pgfqpoint{7.866153in}{3.216936in}}%
\pgfpathlineto{\pgfqpoint{7.862981in}{3.213767in}}%
\pgfpathlineto{\pgfqpoint{7.859809in}{3.210674in}}%
\pgfpathlineto{\pgfqpoint{7.856637in}{3.207515in}}%
\pgfpathlineto{\pgfqpoint{7.853465in}{3.204429in}}%
\pgfpathlineto{\pgfqpoint{7.850293in}{3.201334in}}%
\pgfpathlineto{\pgfqpoint{7.847121in}{3.198247in}}%
\pgfpathlineto{\pgfqpoint{7.843949in}{3.195136in}}%
\pgfpathlineto{\pgfqpoint{7.840777in}{3.192018in}}%
\pgfpathlineto{\pgfqpoint{7.837605in}{3.188966in}}%
\pgfpathlineto{\pgfqpoint{7.834433in}{3.185856in}}%
\pgfpathlineto{\pgfqpoint{7.831261in}{3.182747in}}%
\pgfpathlineto{\pgfqpoint{7.828089in}{3.179699in}}%
\pgfpathlineto{\pgfqpoint{7.824917in}{3.176677in}}%
\pgfpathlineto{\pgfqpoint{7.821744in}{3.173546in}}%
\pgfpathlineto{\pgfqpoint{7.818572in}{3.170405in}}%
\pgfpathlineto{\pgfqpoint{7.815400in}{3.167311in}}%
\pgfpathlineto{\pgfqpoint{7.812228in}{3.164223in}}%
\pgfpathlineto{\pgfqpoint{7.809056in}{3.161142in}}%
\pgfpathlineto{\pgfqpoint{7.805884in}{3.158041in}}%
\pgfpathlineto{\pgfqpoint{7.802712in}{3.154929in}}%
\pgfpathlineto{\pgfqpoint{7.799540in}{3.151812in}}%
\pgfpathlineto{\pgfqpoint{7.796368in}{3.148710in}}%
\pgfpathlineto{\pgfqpoint{7.793196in}{3.145577in}}%
\pgfpathlineto{\pgfqpoint{7.790024in}{3.142442in}}%
\pgfpathlineto{\pgfqpoint{7.786852in}{3.139362in}}%
\pgfpathlineto{\pgfqpoint{7.783680in}{3.139212in}}%
\pgfpathlineto{\pgfqpoint{7.780508in}{3.139628in}}%
\pgfpathlineto{\pgfqpoint{7.777336in}{3.139564in}}%
\pgfpathlineto{\pgfqpoint{7.774164in}{3.139839in}}%
\pgfpathlineto{\pgfqpoint{7.770992in}{3.139746in}}%
\pgfpathlineto{\pgfqpoint{7.767820in}{3.139527in}}%
\pgfpathlineto{\pgfqpoint{7.764648in}{3.139563in}}%
\pgfpathlineto{\pgfqpoint{7.761476in}{3.139494in}}%
\pgfpathlineto{\pgfqpoint{7.758304in}{3.139470in}}%
\pgfpathlineto{\pgfqpoint{7.755132in}{3.139253in}}%
\pgfpathlineto{\pgfqpoint{7.751960in}{3.139379in}}%
\pgfpathlineto{\pgfqpoint{7.748788in}{3.139948in}}%
\pgfpathlineto{\pgfqpoint{7.745616in}{3.140405in}}%
\pgfpathlineto{\pgfqpoint{7.742443in}{3.140271in}}%
\pgfpathlineto{\pgfqpoint{7.739271in}{3.140402in}}%
\pgfpathlineto{\pgfqpoint{7.736099in}{3.140698in}}%
\pgfpathlineto{\pgfqpoint{7.732927in}{3.140751in}}%
\pgfpathlineto{\pgfqpoint{7.729755in}{3.140747in}}%
\pgfpathlineto{\pgfqpoint{7.726583in}{3.140963in}}%
\pgfpathlineto{\pgfqpoint{7.723411in}{3.140782in}}%
\pgfpathlineto{\pgfqpoint{7.720239in}{3.140932in}}%
\pgfpathlineto{\pgfqpoint{7.717067in}{3.141019in}}%
\pgfpathlineto{\pgfqpoint{7.713895in}{3.140860in}}%
\pgfpathlineto{\pgfqpoint{7.710723in}{3.140854in}}%
\pgfpathlineto{\pgfqpoint{7.707551in}{3.140967in}}%
\pgfpathlineto{\pgfqpoint{7.704379in}{3.140842in}}%
\pgfpathlineto{\pgfqpoint{7.701207in}{3.141104in}}%
\pgfpathlineto{\pgfqpoint{7.698035in}{3.141533in}}%
\pgfpathlineto{\pgfqpoint{7.694863in}{3.141290in}}%
\pgfpathlineto{\pgfqpoint{7.691691in}{3.140987in}}%
\pgfpathlineto{\pgfqpoint{7.688519in}{3.140830in}}%
\pgfpathlineto{\pgfqpoint{7.685347in}{3.140561in}}%
\pgfpathlineto{\pgfqpoint{7.682175in}{3.140435in}}%
\pgfpathlineto{\pgfqpoint{7.679003in}{3.140101in}}%
\pgfpathlineto{\pgfqpoint{7.675831in}{3.140036in}}%
\pgfpathlineto{\pgfqpoint{7.672659in}{3.140018in}}%
\pgfpathlineto{\pgfqpoint{7.669487in}{3.140120in}}%
\pgfpathlineto{\pgfqpoint{7.666315in}{3.140243in}}%
\pgfpathlineto{\pgfqpoint{7.663142in}{3.140268in}}%
\pgfpathlineto{\pgfqpoint{7.659970in}{3.140106in}}%
\pgfpathlineto{\pgfqpoint{7.656798in}{3.140325in}}%
\pgfpathlineto{\pgfqpoint{7.653626in}{3.140355in}}%
\pgfpathlineto{\pgfqpoint{7.650454in}{3.140452in}}%
\pgfpathlineto{\pgfqpoint{7.647282in}{3.140331in}}%
\pgfpathlineto{\pgfqpoint{7.644110in}{3.140105in}}%
\pgfpathlineto{\pgfqpoint{7.640938in}{3.140172in}}%
\pgfpathlineto{\pgfqpoint{7.637766in}{3.140191in}}%
\pgfpathlineto{\pgfqpoint{7.634594in}{3.140253in}}%
\pgfpathlineto{\pgfqpoint{7.631422in}{3.140421in}}%
\pgfpathlineto{\pgfqpoint{7.628250in}{3.140318in}}%
\pgfpathlineto{\pgfqpoint{7.625078in}{3.140285in}}%
\pgfpathlineto{\pgfqpoint{7.621906in}{3.140479in}}%
\pgfpathlineto{\pgfqpoint{7.618734in}{3.140389in}}%
\pgfpathlineto{\pgfqpoint{7.615562in}{3.140238in}}%
\pgfpathlineto{\pgfqpoint{7.612390in}{3.140328in}}%
\pgfpathlineto{\pgfqpoint{7.609218in}{3.140508in}}%
\pgfpathlineto{\pgfqpoint{7.606046in}{3.140357in}}%
\pgfpathlineto{\pgfqpoint{7.602874in}{3.140299in}}%
\pgfpathlineto{\pgfqpoint{7.599702in}{3.140429in}}%
\pgfpathlineto{\pgfqpoint{7.596530in}{3.140331in}}%
\pgfpathlineto{\pgfqpoint{7.593358in}{3.139980in}}%
\pgfpathlineto{\pgfqpoint{7.590186in}{3.139919in}}%
\pgfpathlineto{\pgfqpoint{7.587013in}{3.139929in}}%
\pgfpathlineto{\pgfqpoint{7.583841in}{3.140102in}}%
\pgfpathlineto{\pgfqpoint{7.580669in}{3.140146in}}%
\pgfpathlineto{\pgfqpoint{7.577497in}{3.139974in}}%
\pgfpathlineto{\pgfqpoint{7.574325in}{3.139859in}}%
\pgfpathlineto{\pgfqpoint{7.571153in}{3.139968in}}%
\pgfpathlineto{\pgfqpoint{7.567981in}{3.139825in}}%
\pgfpathlineto{\pgfqpoint{7.564809in}{3.139466in}}%
\pgfpathlineto{\pgfqpoint{7.561637in}{3.139134in}}%
\pgfpathlineto{\pgfqpoint{7.558465in}{3.139400in}}%
\pgfpathlineto{\pgfqpoint{7.555293in}{3.139567in}}%
\pgfpathlineto{\pgfqpoint{7.552121in}{3.139128in}}%
\pgfpathlineto{\pgfqpoint{7.548949in}{3.139050in}}%
\pgfpathlineto{\pgfqpoint{7.545777in}{3.139130in}}%
\pgfpathlineto{\pgfqpoint{7.542605in}{3.139007in}}%
\pgfpathlineto{\pgfqpoint{7.539433in}{3.138770in}}%
\pgfpathlineto{\pgfqpoint{7.536261in}{3.139413in}}%
\pgfpathlineto{\pgfqpoint{7.533089in}{3.139546in}}%
\pgfpathlineto{\pgfqpoint{7.529917in}{3.139802in}}%
\pgfpathlineto{\pgfqpoint{7.526745in}{3.139683in}}%
\pgfpathlineto{\pgfqpoint{7.523573in}{3.139138in}}%
\pgfpathlineto{\pgfqpoint{7.520401in}{3.139165in}}%
\pgfpathlineto{\pgfqpoint{7.517229in}{3.139107in}}%
\pgfpathlineto{\pgfqpoint{7.514057in}{3.139148in}}%
\pgfpathlineto{\pgfqpoint{7.510885in}{3.139228in}}%
\pgfpathlineto{\pgfqpoint{7.507712in}{3.138862in}}%
\pgfpathlineto{\pgfqpoint{7.504540in}{3.139059in}}%
\pgfpathlineto{\pgfqpoint{7.501368in}{3.139000in}}%
\pgfpathlineto{\pgfqpoint{7.498196in}{3.139249in}}%
\pgfpathlineto{\pgfqpoint{7.495024in}{3.139158in}}%
\pgfpathlineto{\pgfqpoint{7.491852in}{3.139104in}}%
\pgfpathlineto{\pgfqpoint{7.488680in}{3.138824in}}%
\pgfpathlineto{\pgfqpoint{7.485508in}{3.138695in}}%
\pgfpathlineto{\pgfqpoint{7.482336in}{3.138861in}}%
\pgfpathlineto{\pgfqpoint{7.479164in}{3.139105in}}%
\pgfpathlineto{\pgfqpoint{7.475992in}{3.139214in}}%
\pgfpathlineto{\pgfqpoint{7.472820in}{3.138997in}}%
\pgfpathlineto{\pgfqpoint{7.469648in}{3.139134in}}%
\pgfpathlineto{\pgfqpoint{7.466476in}{3.139152in}}%
\pgfpathlineto{\pgfqpoint{7.463304in}{3.139085in}}%
\pgfpathlineto{\pgfqpoint{7.460132in}{3.139190in}}%
\pgfpathlineto{\pgfqpoint{7.456960in}{3.139236in}}%
\pgfpathlineto{\pgfqpoint{7.453788in}{3.139406in}}%
\pgfpathlineto{\pgfqpoint{7.450616in}{3.139286in}}%
\pgfpathlineto{\pgfqpoint{7.447444in}{3.139344in}}%
\pgfpathlineto{\pgfqpoint{7.444272in}{3.139306in}}%
\pgfpathlineto{\pgfqpoint{7.441100in}{3.139023in}}%
\pgfpathlineto{\pgfqpoint{7.437928in}{3.139221in}}%
\pgfpathlineto{\pgfqpoint{7.434756in}{3.139143in}}%
\pgfpathlineto{\pgfqpoint{7.431584in}{3.139008in}}%
\pgfpathlineto{\pgfqpoint{7.428411in}{3.138948in}}%
\pgfpathlineto{\pgfqpoint{7.425239in}{3.139026in}}%
\pgfpathlineto{\pgfqpoint{7.422067in}{3.139411in}}%
\pgfpathlineto{\pgfqpoint{7.418895in}{3.139643in}}%
\pgfpathlineto{\pgfqpoint{7.415723in}{3.139392in}}%
\pgfpathlineto{\pgfqpoint{7.412551in}{3.139452in}}%
\pgfpathlineto{\pgfqpoint{7.409379in}{3.139647in}}%
\pgfpathlineto{\pgfqpoint{7.406207in}{3.139879in}}%
\pgfpathlineto{\pgfqpoint{7.403035in}{3.140354in}}%
\pgfpathlineto{\pgfqpoint{7.399863in}{3.140465in}}%
\pgfpathlineto{\pgfqpoint{7.396691in}{3.140436in}}%
\pgfpathlineto{\pgfqpoint{7.393519in}{3.140543in}}%
\pgfpathlineto{\pgfqpoint{7.390347in}{3.140694in}}%
\pgfpathlineto{\pgfqpoint{7.387175in}{3.140980in}}%
\pgfpathlineto{\pgfqpoint{7.384003in}{3.140559in}}%
\pgfpathlineto{\pgfqpoint{7.380831in}{3.140575in}}%
\pgfpathlineto{\pgfqpoint{7.377659in}{3.140654in}}%
\pgfpathlineto{\pgfqpoint{7.374487in}{3.140533in}}%
\pgfpathlineto{\pgfqpoint{7.371315in}{3.140597in}}%
\pgfpathlineto{\pgfqpoint{7.368143in}{3.140330in}}%
\pgfpathlineto{\pgfqpoint{7.364971in}{3.140324in}}%
\pgfpathlineto{\pgfqpoint{7.361799in}{3.140117in}}%
\pgfpathlineto{\pgfqpoint{7.358627in}{3.139898in}}%
\pgfpathlineto{\pgfqpoint{7.355455in}{3.139885in}}%
\pgfpathlineto{\pgfqpoint{7.352282in}{3.139664in}}%
\pgfpathlineto{\pgfqpoint{7.349110in}{3.139139in}}%
\pgfpathlineto{\pgfqpoint{7.345938in}{3.139135in}}%
\pgfpathlineto{\pgfqpoint{7.342766in}{3.139538in}}%
\pgfpathlineto{\pgfqpoint{7.339594in}{3.139523in}}%
\pgfpathlineto{\pgfqpoint{7.336422in}{3.139439in}}%
\pgfpathlineto{\pgfqpoint{7.333250in}{3.139539in}}%
\pgfpathlineto{\pgfqpoint{7.330078in}{3.138845in}}%
\pgfpathlineto{\pgfqpoint{7.326906in}{3.138640in}}%
\pgfpathlineto{\pgfqpoint{7.323734in}{3.138631in}}%
\pgfpathlineto{\pgfqpoint{7.320562in}{3.138679in}}%
\pgfpathlineto{\pgfqpoint{7.317390in}{3.138351in}}%
\pgfpathlineto{\pgfqpoint{7.314218in}{3.138109in}}%
\pgfpathlineto{\pgfqpoint{7.311046in}{3.138050in}}%
\pgfpathlineto{\pgfqpoint{7.307874in}{3.138019in}}%
\pgfpathlineto{\pgfqpoint{7.304702in}{3.137872in}}%
\pgfpathlineto{\pgfqpoint{7.301530in}{3.137962in}}%
\pgfpathlineto{\pgfqpoint{7.298358in}{3.137815in}}%
\pgfpathlineto{\pgfqpoint{7.295186in}{3.137654in}}%
\pgfpathlineto{\pgfqpoint{7.292014in}{3.137845in}}%
\pgfpathlineto{\pgfqpoint{7.288842in}{3.137998in}}%
\pgfpathlineto{\pgfqpoint{7.285670in}{3.138188in}}%
\pgfpathlineto{\pgfqpoint{7.282498in}{3.138194in}}%
\pgfpathlineto{\pgfqpoint{7.279326in}{3.138347in}}%
\pgfpathlineto{\pgfqpoint{7.276154in}{3.138527in}}%
\pgfpathlineto{\pgfqpoint{7.272981in}{3.138605in}}%
\pgfpathlineto{\pgfqpoint{7.269809in}{3.138554in}}%
\pgfpathlineto{\pgfqpoint{7.266637in}{3.138775in}}%
\pgfpathlineto{\pgfqpoint{7.263465in}{3.139114in}}%
\pgfpathlineto{\pgfqpoint{7.260293in}{3.139137in}}%
\pgfpathlineto{\pgfqpoint{7.257121in}{3.139171in}}%
\pgfpathlineto{\pgfqpoint{7.253949in}{3.139082in}}%
\pgfpathlineto{\pgfqpoint{7.250777in}{3.139010in}}%
\pgfpathlineto{\pgfqpoint{7.247605in}{3.138694in}}%
\pgfpathlineto{\pgfqpoint{7.244433in}{3.138894in}}%
\pgfpathlineto{\pgfqpoint{7.241261in}{3.138761in}}%
\pgfpathlineto{\pgfqpoint{7.238089in}{3.138875in}}%
\pgfpathlineto{\pgfqpoint{7.234917in}{3.138824in}}%
\pgfpathlineto{\pgfqpoint{7.231745in}{3.139004in}}%
\pgfpathlineto{\pgfqpoint{7.228573in}{3.138883in}}%
\pgfpathlineto{\pgfqpoint{7.225401in}{3.138967in}}%
\pgfpathlineto{\pgfqpoint{7.222229in}{3.138737in}}%
\pgfpathlineto{\pgfqpoint{7.219057in}{3.138959in}}%
\pgfpathlineto{\pgfqpoint{7.215885in}{3.139204in}}%
\pgfpathlineto{\pgfqpoint{7.212713in}{3.139186in}}%
\pgfpathlineto{\pgfqpoint{7.209541in}{3.139103in}}%
\pgfpathlineto{\pgfqpoint{7.206369in}{3.138947in}}%
\pgfpathlineto{\pgfqpoint{7.203197in}{3.138942in}}%
\pgfpathlineto{\pgfqpoint{7.200025in}{3.139021in}}%
\pgfpathlineto{\pgfqpoint{7.196853in}{3.139251in}}%
\pgfpathlineto{\pgfqpoint{7.193680in}{3.139299in}}%
\pgfpathlineto{\pgfqpoint{7.190508in}{3.139289in}}%
\pgfpathlineto{\pgfqpoint{7.187336in}{3.139482in}}%
\pgfpathlineto{\pgfqpoint{7.184164in}{3.139294in}}%
\pgfpathlineto{\pgfqpoint{7.180992in}{3.139231in}}%
\pgfpathlineto{\pgfqpoint{7.177820in}{3.139298in}}%
\pgfpathlineto{\pgfqpoint{7.174648in}{3.139411in}}%
\pgfpathlineto{\pgfqpoint{7.171476in}{3.139261in}}%
\pgfpathlineto{\pgfqpoint{7.168304in}{3.139260in}}%
\pgfpathlineto{\pgfqpoint{7.165132in}{3.139439in}}%
\pgfpathlineto{\pgfqpoint{7.161960in}{3.139358in}}%
\pgfpathlineto{\pgfqpoint{7.158788in}{3.139194in}}%
\pgfpathlineto{\pgfqpoint{7.155616in}{3.139039in}}%
\pgfpathlineto{\pgfqpoint{7.152444in}{3.139113in}}%
\pgfpathlineto{\pgfqpoint{7.149272in}{3.139151in}}%
\pgfpathlineto{\pgfqpoint{7.146100in}{3.139350in}}%
\pgfpathlineto{\pgfqpoint{7.142928in}{3.139247in}}%
\pgfpathlineto{\pgfqpoint{7.139756in}{3.139255in}}%
\pgfpathlineto{\pgfqpoint{7.136584in}{3.139349in}}%
\pgfpathlineto{\pgfqpoint{7.133412in}{3.139130in}}%
\pgfpathlineto{\pgfqpoint{7.130240in}{3.138878in}}%
\pgfpathlineto{\pgfqpoint{7.127068in}{3.138752in}}%
\pgfpathlineto{\pgfqpoint{7.123896in}{3.138888in}}%
\pgfpathlineto{\pgfqpoint{7.120724in}{3.138859in}}%
\pgfpathlineto{\pgfqpoint{7.117551in}{3.139206in}}%
\pgfpathlineto{\pgfqpoint{7.114379in}{3.139251in}}%
\pgfpathlineto{\pgfqpoint{7.111207in}{3.139233in}}%
\pgfpathlineto{\pgfqpoint{7.108035in}{3.138900in}}%
\pgfpathlineto{\pgfqpoint{7.104863in}{3.139087in}}%
\pgfpathlineto{\pgfqpoint{7.101691in}{3.138790in}}%
\pgfpathlineto{\pgfqpoint{7.098519in}{3.138670in}}%
\pgfpathlineto{\pgfqpoint{7.095347in}{3.138751in}}%
\pgfpathlineto{\pgfqpoint{7.092175in}{3.138541in}}%
\pgfpathlineto{\pgfqpoint{7.089003in}{3.138637in}}%
\pgfpathlineto{\pgfqpoint{7.085831in}{3.138702in}}%
\pgfpathlineto{\pgfqpoint{7.082659in}{3.138799in}}%
\pgfpathlineto{\pgfqpoint{7.079487in}{3.138961in}}%
\pgfpathlineto{\pgfqpoint{7.076315in}{3.139058in}}%
\pgfpathlineto{\pgfqpoint{7.073143in}{3.138833in}}%
\pgfpathlineto{\pgfqpoint{7.069971in}{3.138805in}}%
\pgfpathlineto{\pgfqpoint{7.066799in}{3.138770in}}%
\pgfpathlineto{\pgfqpoint{7.063627in}{3.138681in}}%
\pgfpathlineto{\pgfqpoint{7.060455in}{3.139299in}}%
\pgfpathlineto{\pgfqpoint{7.057283in}{3.139465in}}%
\pgfpathlineto{\pgfqpoint{7.054111in}{3.139437in}}%
\pgfpathlineto{\pgfqpoint{7.050939in}{3.139828in}}%
\pgfpathlineto{\pgfqpoint{7.047767in}{3.140025in}}%
\pgfpathlineto{\pgfqpoint{7.044595in}{3.139986in}}%
\pgfpathlineto{\pgfqpoint{7.041423in}{3.139395in}}%
\pgfpathlineto{\pgfqpoint{7.038250in}{3.139934in}}%
\pgfpathlineto{\pgfqpoint{7.035078in}{3.139904in}}%
\pgfpathlineto{\pgfqpoint{7.031906in}{3.140038in}}%
\pgfpathlineto{\pgfqpoint{7.028734in}{3.139645in}}%
\pgfpathlineto{\pgfqpoint{7.025562in}{3.139739in}}%
\pgfpathlineto{\pgfqpoint{7.022390in}{3.139869in}}%
\pgfpathlineto{\pgfqpoint{7.019218in}{3.139457in}}%
\pgfpathlineto{\pgfqpoint{7.016046in}{3.139304in}}%
\pgfpathlineto{\pgfqpoint{7.012874in}{3.139375in}}%
\pgfpathlineto{\pgfqpoint{7.009702in}{3.139565in}}%
\pgfpathlineto{\pgfqpoint{7.006530in}{3.139688in}}%
\pgfpathlineto{\pgfqpoint{7.003358in}{3.139732in}}%
\pgfpathlineto{\pgfqpoint{7.000186in}{3.139844in}}%
\pgfpathlineto{\pgfqpoint{6.997014in}{3.139285in}}%
\pgfpathlineto{\pgfqpoint{6.993842in}{3.139271in}}%
\pgfpathlineto{\pgfqpoint{6.990670in}{3.139062in}}%
\pgfpathlineto{\pgfqpoint{6.987498in}{3.139092in}}%
\pgfpathlineto{\pgfqpoint{6.984326in}{3.139346in}}%
\pgfpathlineto{\pgfqpoint{6.981154in}{3.139420in}}%
\pgfpathlineto{\pgfqpoint{6.977982in}{3.139298in}}%
\pgfpathlineto{\pgfqpoint{6.974810in}{3.139413in}}%
\pgfpathlineto{\pgfqpoint{6.971638in}{3.139339in}}%
\pgfpathlineto{\pgfqpoint{6.968466in}{3.139679in}}%
\pgfpathlineto{\pgfqpoint{6.965294in}{3.139440in}}%
\pgfpathlineto{\pgfqpoint{6.962122in}{3.139164in}}%
\pgfpathlineto{\pgfqpoint{6.958949in}{3.139060in}}%
\pgfpathlineto{\pgfqpoint{6.955777in}{3.138580in}}%
\pgfpathlineto{\pgfqpoint{6.952605in}{3.138462in}}%
\pgfpathlineto{\pgfqpoint{6.949433in}{3.138449in}}%
\pgfpathlineto{\pgfqpoint{6.946261in}{3.138420in}}%
\pgfpathlineto{\pgfqpoint{6.943089in}{3.137977in}}%
\pgfpathlineto{\pgfqpoint{6.939917in}{3.137819in}}%
\pgfpathlineto{\pgfqpoint{6.936745in}{3.138043in}}%
\pgfpathlineto{\pgfqpoint{6.933573in}{3.138051in}}%
\pgfpathlineto{\pgfqpoint{6.930401in}{3.137956in}}%
\pgfpathlineto{\pgfqpoint{6.927229in}{3.138029in}}%
\pgfpathlineto{\pgfqpoint{6.924057in}{3.137729in}}%
\pgfpathlineto{\pgfqpoint{6.920885in}{3.137532in}}%
\pgfpathlineto{\pgfqpoint{6.917713in}{3.137688in}}%
\pgfpathlineto{\pgfqpoint{6.914541in}{3.137822in}}%
\pgfpathlineto{\pgfqpoint{6.911369in}{3.137737in}}%
\pgfpathlineto{\pgfqpoint{6.908197in}{3.137330in}}%
\pgfpathlineto{\pgfqpoint{6.905025in}{3.137197in}}%
\pgfpathlineto{\pgfqpoint{6.901853in}{3.136922in}}%
\pgfpathlineto{\pgfqpoint{6.898681in}{3.137312in}}%
\pgfpathlineto{\pgfqpoint{6.895509in}{3.137312in}}%
\pgfpathlineto{\pgfqpoint{6.892337in}{3.137589in}}%
\pgfpathlineto{\pgfqpoint{6.889165in}{3.137365in}}%
\pgfpathlineto{\pgfqpoint{6.885993in}{3.137360in}}%
\pgfpathlineto{\pgfqpoint{6.882820in}{3.137212in}}%
\pgfpathlineto{\pgfqpoint{6.879648in}{3.137287in}}%
\pgfpathlineto{\pgfqpoint{6.876476in}{3.137270in}}%
\pgfpathlineto{\pgfqpoint{6.873304in}{3.137402in}}%
\pgfpathlineto{\pgfqpoint{6.870132in}{3.137408in}}%
\pgfpathlineto{\pgfqpoint{6.866960in}{3.137294in}}%
\pgfpathlineto{\pgfqpoint{6.863788in}{3.137671in}}%
\pgfpathlineto{\pgfqpoint{6.860616in}{3.137932in}}%
\pgfpathlineto{\pgfqpoint{6.857444in}{3.138143in}}%
\pgfpathlineto{\pgfqpoint{6.854272in}{3.137998in}}%
\pgfpathlineto{\pgfqpoint{6.851100in}{3.137832in}}%
\pgfpathlineto{\pgfqpoint{6.847928in}{3.137880in}}%
\pgfpathlineto{\pgfqpoint{6.844756in}{3.137645in}}%
\pgfpathlineto{\pgfqpoint{6.841584in}{3.137529in}}%
\pgfpathlineto{\pgfqpoint{6.838412in}{3.137515in}}%
\pgfpathlineto{\pgfqpoint{6.835240in}{3.137420in}}%
\pgfpathlineto{\pgfqpoint{6.832068in}{3.137725in}}%
\pgfpathlineto{\pgfqpoint{6.828896in}{3.137590in}}%
\pgfpathlineto{\pgfqpoint{6.825724in}{3.137276in}}%
\pgfpathlineto{\pgfqpoint{6.822552in}{3.136971in}}%
\pgfpathlineto{\pgfqpoint{6.819380in}{3.136840in}}%
\pgfpathlineto{\pgfqpoint{6.816208in}{3.137435in}}%
\pgfpathlineto{\pgfqpoint{6.813036in}{3.137535in}}%
\pgfpathlineto{\pgfqpoint{6.809864in}{3.137790in}}%
\pgfpathlineto{\pgfqpoint{6.806692in}{3.137606in}}%
\pgfpathlineto{\pgfqpoint{6.803519in}{3.137634in}}%
\pgfpathlineto{\pgfqpoint{6.800347in}{3.137552in}}%
\pgfpathlineto{\pgfqpoint{6.797175in}{3.137454in}}%
\pgfpathlineto{\pgfqpoint{6.794003in}{3.136792in}}%
\pgfpathlineto{\pgfqpoint{6.790831in}{3.137211in}}%
\pgfpathlineto{\pgfqpoint{6.787659in}{3.136756in}}%
\pgfpathlineto{\pgfqpoint{6.784487in}{3.136649in}}%
\pgfpathlineto{\pgfqpoint{6.781315in}{3.136876in}}%
\pgfpathlineto{\pgfqpoint{6.778143in}{3.136493in}}%
\pgfpathlineto{\pgfqpoint{6.774971in}{3.136672in}}%
\pgfpathlineto{\pgfqpoint{6.771799in}{3.136431in}}%
\pgfpathlineto{\pgfqpoint{6.768627in}{3.136126in}}%
\pgfpathlineto{\pgfqpoint{6.765455in}{3.136053in}}%
\pgfpathlineto{\pgfqpoint{6.762283in}{3.135950in}}%
\pgfpathlineto{\pgfqpoint{6.759111in}{3.135915in}}%
\pgfpathlineto{\pgfqpoint{6.755939in}{3.136058in}}%
\pgfpathlineto{\pgfqpoint{6.752767in}{3.135894in}}%
\pgfpathlineto{\pgfqpoint{6.749595in}{3.135912in}}%
\pgfpathlineto{\pgfqpoint{6.746423in}{3.135883in}}%
\pgfpathlineto{\pgfqpoint{6.743251in}{3.135818in}}%
\pgfpathlineto{\pgfqpoint{6.740079in}{3.135875in}}%
\pgfpathlineto{\pgfqpoint{6.736907in}{3.135953in}}%
\pgfpathlineto{\pgfqpoint{6.733735in}{3.135543in}}%
\pgfpathlineto{\pgfqpoint{6.730563in}{3.135464in}}%
\pgfpathlineto{\pgfqpoint{6.727391in}{3.135297in}}%
\pgfpathlineto{\pgfqpoint{6.724218in}{3.135449in}}%
\pgfpathlineto{\pgfqpoint{6.721046in}{3.135425in}}%
\pgfpathlineto{\pgfqpoint{6.717874in}{3.135257in}}%
\pgfpathlineto{\pgfqpoint{6.714702in}{3.135176in}}%
\pgfpathlineto{\pgfqpoint{6.711530in}{3.135079in}}%
\pgfpathlineto{\pgfqpoint{6.708358in}{3.135234in}}%
\pgfpathlineto{\pgfqpoint{6.705186in}{3.134565in}}%
\pgfpathlineto{\pgfqpoint{6.702014in}{3.134660in}}%
\pgfpathlineto{\pgfqpoint{6.698842in}{3.134721in}}%
\pgfpathlineto{\pgfqpoint{6.695670in}{3.134946in}}%
\pgfpathlineto{\pgfqpoint{6.692498in}{3.134834in}}%
\pgfpathlineto{\pgfqpoint{6.689326in}{3.134661in}}%
\pgfpathlineto{\pgfqpoint{6.686154in}{3.134834in}}%
\pgfpathlineto{\pgfqpoint{6.682982in}{3.134977in}}%
\pgfpathlineto{\pgfqpoint{6.679810in}{3.135077in}}%
\pgfpathlineto{\pgfqpoint{6.676638in}{3.135292in}}%
\pgfpathlineto{\pgfqpoint{6.673466in}{3.135798in}}%
\pgfpathlineto{\pgfqpoint{6.670294in}{3.135904in}}%
\pgfpathlineto{\pgfqpoint{6.667122in}{3.136140in}}%
\pgfpathlineto{\pgfqpoint{6.663950in}{3.136415in}}%
\pgfpathlineto{\pgfqpoint{6.660778in}{3.136395in}}%
\pgfpathlineto{\pgfqpoint{6.657606in}{3.136410in}}%
\pgfpathlineto{\pgfqpoint{6.654434in}{3.136374in}}%
\pgfpathlineto{\pgfqpoint{6.651262in}{3.136596in}}%
\pgfpathlineto{\pgfqpoint{6.648089in}{3.136556in}}%
\pgfpathlineto{\pgfqpoint{6.644917in}{3.136793in}}%
\pgfpathlineto{\pgfqpoint{6.641745in}{3.136800in}}%
\pgfpathlineto{\pgfqpoint{6.638573in}{3.137041in}}%
\pgfpathlineto{\pgfqpoint{6.635401in}{3.137016in}}%
\pgfpathlineto{\pgfqpoint{6.632229in}{3.136933in}}%
\pgfpathlineto{\pgfqpoint{6.629057in}{3.136626in}}%
\pgfpathlineto{\pgfqpoint{6.625885in}{3.136968in}}%
\pgfpathlineto{\pgfqpoint{6.622713in}{3.136729in}}%
\pgfpathlineto{\pgfqpoint{6.619541in}{3.136738in}}%
\pgfpathlineto{\pgfqpoint{6.616369in}{3.137141in}}%
\pgfpathlineto{\pgfqpoint{6.613197in}{3.137183in}}%
\pgfpathlineto{\pgfqpoint{6.610025in}{3.137194in}}%
\pgfpathlineto{\pgfqpoint{6.606853in}{3.137378in}}%
\pgfpathlineto{\pgfqpoint{6.603681in}{3.137206in}}%
\pgfpathlineto{\pgfqpoint{6.600509in}{3.137046in}}%
\pgfpathlineto{\pgfqpoint{6.597337in}{3.137246in}}%
\pgfpathlineto{\pgfqpoint{6.594165in}{3.137268in}}%
\pgfpathlineto{\pgfqpoint{6.590993in}{3.137312in}}%
\pgfpathlineto{\pgfqpoint{6.587821in}{3.137349in}}%
\pgfpathlineto{\pgfqpoint{6.584649in}{3.137203in}}%
\pgfpathlineto{\pgfqpoint{6.581477in}{3.137161in}}%
\pgfpathlineto{\pgfqpoint{6.578305in}{3.137053in}}%
\pgfpathlineto{\pgfqpoint{6.575133in}{3.137006in}}%
\pgfpathlineto{\pgfqpoint{6.571961in}{3.137185in}}%
\pgfpathlineto{\pgfqpoint{6.568788in}{3.137096in}}%
\pgfpathlineto{\pgfqpoint{6.565616in}{3.137054in}}%
\pgfpathlineto{\pgfqpoint{6.562444in}{3.137099in}}%
\pgfpathlineto{\pgfqpoint{6.559272in}{3.137285in}}%
\pgfpathlineto{\pgfqpoint{6.556100in}{3.137562in}}%
\pgfpathlineto{\pgfqpoint{6.552928in}{3.137447in}}%
\pgfpathlineto{\pgfqpoint{6.549756in}{3.137626in}}%
\pgfpathlineto{\pgfqpoint{6.546584in}{3.137469in}}%
\pgfpathlineto{\pgfqpoint{6.543412in}{3.137501in}}%
\pgfpathlineto{\pgfqpoint{6.540240in}{3.137422in}}%
\pgfpathlineto{\pgfqpoint{6.537068in}{3.137773in}}%
\pgfpathlineto{\pgfqpoint{6.533896in}{3.137979in}}%
\pgfpathlineto{\pgfqpoint{6.530724in}{3.138361in}}%
\pgfpathlineto{\pgfqpoint{6.527552in}{3.138556in}}%
\pgfpathlineto{\pgfqpoint{6.524380in}{3.138394in}}%
\pgfpathlineto{\pgfqpoint{6.521208in}{3.138606in}}%
\pgfpathlineto{\pgfqpoint{6.518036in}{3.138655in}}%
\pgfpathlineto{\pgfqpoint{6.514864in}{3.138674in}}%
\pgfpathlineto{\pgfqpoint{6.511692in}{3.139405in}}%
\pgfpathlineto{\pgfqpoint{6.508520in}{3.139227in}}%
\pgfpathlineto{\pgfqpoint{6.505348in}{3.139252in}}%
\pgfpathlineto{\pgfqpoint{6.502176in}{3.139312in}}%
\pgfpathlineto{\pgfqpoint{6.499004in}{3.139178in}}%
\pgfpathlineto{\pgfqpoint{6.495832in}{3.139369in}}%
\pgfpathlineto{\pgfqpoint{6.492659in}{3.139666in}}%
\pgfpathlineto{\pgfqpoint{6.489487in}{3.139668in}}%
\pgfpathlineto{\pgfqpoint{6.486315in}{3.139856in}}%
\pgfpathlineto{\pgfqpoint{6.483143in}{3.140009in}}%
\pgfpathlineto{\pgfqpoint{6.479971in}{3.139940in}}%
\pgfpathlineto{\pgfqpoint{6.476799in}{3.140136in}}%
\pgfpathlineto{\pgfqpoint{6.473627in}{3.140223in}}%
\pgfpathlineto{\pgfqpoint{6.470455in}{3.140095in}}%
\pgfpathlineto{\pgfqpoint{6.467283in}{3.140358in}}%
\pgfpathlineto{\pgfqpoint{6.464111in}{3.140186in}}%
\pgfpathlineto{\pgfqpoint{6.460939in}{3.140328in}}%
\pgfpathlineto{\pgfqpoint{6.457767in}{3.139879in}}%
\pgfpathlineto{\pgfqpoint{6.454595in}{3.139818in}}%
\pgfpathlineto{\pgfqpoint{6.451423in}{3.140266in}}%
\pgfpathlineto{\pgfqpoint{6.448251in}{3.140045in}}%
\pgfpathlineto{\pgfqpoint{6.445079in}{3.140086in}}%
\pgfpathlineto{\pgfqpoint{6.441907in}{3.139999in}}%
\pgfpathlineto{\pgfqpoint{6.438735in}{3.139995in}}%
\pgfpathlineto{\pgfqpoint{6.435563in}{3.140097in}}%
\pgfpathlineto{\pgfqpoint{6.432391in}{3.140322in}}%
\pgfpathlineto{\pgfqpoint{6.429219in}{3.140438in}}%
\pgfpathlineto{\pgfqpoint{6.426047in}{3.140676in}}%
\pgfpathlineto{\pgfqpoint{6.422875in}{3.140556in}}%
\pgfpathlineto{\pgfqpoint{6.419703in}{3.140514in}}%
\pgfpathlineto{\pgfqpoint{6.416531in}{3.140322in}}%
\pgfpathlineto{\pgfqpoint{6.413358in}{3.140544in}}%
\pgfpathlineto{\pgfqpoint{6.410186in}{3.140773in}}%
\pgfpathlineto{\pgfqpoint{6.407014in}{3.140918in}}%
\pgfpathlineto{\pgfqpoint{6.403842in}{3.140955in}}%
\pgfpathlineto{\pgfqpoint{6.400670in}{3.140915in}}%
\pgfpathlineto{\pgfqpoint{6.397498in}{3.140914in}}%
\pgfpathlineto{\pgfqpoint{6.394326in}{3.140972in}}%
\pgfpathlineto{\pgfqpoint{6.391154in}{3.141189in}}%
\pgfpathlineto{\pgfqpoint{6.387982in}{3.141219in}}%
\pgfpathlineto{\pgfqpoint{6.384810in}{3.141249in}}%
\pgfpathlineto{\pgfqpoint{6.381638in}{3.141101in}}%
\pgfpathlineto{\pgfqpoint{6.378466in}{3.140910in}}%
\pgfpathlineto{\pgfqpoint{6.375294in}{3.140902in}}%
\pgfpathlineto{\pgfqpoint{6.372122in}{3.140984in}}%
\pgfpathlineto{\pgfqpoint{6.368950in}{3.140785in}}%
\pgfpathlineto{\pgfqpoint{6.365778in}{3.140790in}}%
\pgfpathlineto{\pgfqpoint{6.362606in}{3.140761in}}%
\pgfpathlineto{\pgfqpoint{6.359434in}{3.140851in}}%
\pgfpathlineto{\pgfqpoint{6.356262in}{3.141092in}}%
\pgfpathlineto{\pgfqpoint{6.353090in}{3.141092in}}%
\pgfpathlineto{\pgfqpoint{6.349918in}{3.141074in}}%
\pgfpathlineto{\pgfqpoint{6.346746in}{3.141074in}}%
\pgfpathlineto{\pgfqpoint{6.343574in}{3.140960in}}%
\pgfpathlineto{\pgfqpoint{6.340402in}{3.140801in}}%
\pgfpathlineto{\pgfqpoint{6.337230in}{3.140818in}}%
\pgfpathlineto{\pgfqpoint{6.334057in}{3.140730in}}%
\pgfpathlineto{\pgfqpoint{6.330885in}{3.140682in}}%
\pgfpathlineto{\pgfqpoint{6.327713in}{3.140536in}}%
\pgfpathlineto{\pgfqpoint{6.324541in}{3.140446in}}%
\pgfpathlineto{\pgfqpoint{6.321369in}{3.140204in}}%
\pgfpathlineto{\pgfqpoint{6.318197in}{3.140283in}}%
\pgfpathlineto{\pgfqpoint{6.315025in}{3.140117in}}%
\pgfpathlineto{\pgfqpoint{6.311853in}{3.139930in}}%
\pgfpathlineto{\pgfqpoint{6.308681in}{3.140194in}}%
\pgfpathlineto{\pgfqpoint{6.305509in}{3.140089in}}%
\pgfpathlineto{\pgfqpoint{6.302337in}{3.140116in}}%
\pgfpathlineto{\pgfqpoint{6.299165in}{3.140047in}}%
\pgfpathlineto{\pgfqpoint{6.295993in}{3.139927in}}%
\pgfpathlineto{\pgfqpoint{6.292821in}{3.139638in}}%
\pgfpathlineto{\pgfqpoint{6.289649in}{3.139429in}}%
\pgfpathlineto{\pgfqpoint{6.286477in}{3.139447in}}%
\pgfpathlineto{\pgfqpoint{6.283305in}{3.139613in}}%
\pgfpathlineto{\pgfqpoint{6.280133in}{3.139441in}}%
\pgfpathlineto{\pgfqpoint{6.276961in}{3.139433in}}%
\pgfpathlineto{\pgfqpoint{6.273789in}{3.139541in}}%
\pgfpathlineto{\pgfqpoint{6.270617in}{3.139562in}}%
\pgfpathlineto{\pgfqpoint{6.267445in}{3.139587in}}%
\pgfpathlineto{\pgfqpoint{6.264273in}{3.139739in}}%
\pgfpathlineto{\pgfqpoint{6.261101in}{3.139188in}}%
\pgfpathlineto{\pgfqpoint{6.257928in}{3.139026in}}%
\pgfpathlineto{\pgfqpoint{6.254756in}{3.138976in}}%
\pgfpathlineto{\pgfqpoint{6.251584in}{3.138886in}}%
\pgfpathlineto{\pgfqpoint{6.248412in}{3.138709in}}%
\pgfpathlineto{\pgfqpoint{6.245240in}{3.138633in}}%
\pgfpathlineto{\pgfqpoint{6.242068in}{3.138670in}}%
\pgfpathlineto{\pgfqpoint{6.238896in}{3.139108in}}%
\pgfpathlineto{\pgfqpoint{6.235724in}{3.138796in}}%
\pgfpathlineto{\pgfqpoint{6.232552in}{3.138683in}}%
\pgfpathlineto{\pgfqpoint{6.229380in}{3.138643in}}%
\pgfpathlineto{\pgfqpoint{6.226208in}{3.138499in}}%
\pgfpathlineto{\pgfqpoint{6.223036in}{3.138701in}}%
\pgfpathlineto{\pgfqpoint{6.219864in}{3.138732in}}%
\pgfpathlineto{\pgfqpoint{6.216692in}{3.138542in}}%
\pgfpathlineto{\pgfqpoint{6.213520in}{3.138916in}}%
\pgfpathlineto{\pgfqpoint{6.210348in}{3.139001in}}%
\pgfpathlineto{\pgfqpoint{6.207176in}{3.139044in}}%
\pgfpathlineto{\pgfqpoint{6.204004in}{3.138875in}}%
\pgfpathlineto{\pgfqpoint{6.200832in}{3.139130in}}%
\pgfpathlineto{\pgfqpoint{6.197660in}{3.139286in}}%
\pgfpathlineto{\pgfqpoint{6.194488in}{3.139040in}}%
\pgfpathlineto{\pgfqpoint{6.191316in}{3.138870in}}%
\pgfpathlineto{\pgfqpoint{6.188144in}{3.138802in}}%
\pgfpathlineto{\pgfqpoint{6.184972in}{3.138771in}}%
\pgfpathlineto{\pgfqpoint{6.181800in}{3.139045in}}%
\pgfpathlineto{\pgfqpoint{6.178627in}{3.139302in}}%
\pgfpathlineto{\pgfqpoint{6.175455in}{3.139534in}}%
\pgfpathlineto{\pgfqpoint{6.172283in}{3.139507in}}%
\pgfpathlineto{\pgfqpoint{6.169111in}{3.139470in}}%
\pgfpathlineto{\pgfqpoint{6.165939in}{3.139369in}}%
\pgfpathlineto{\pgfqpoint{6.162767in}{3.139484in}}%
\pgfpathlineto{\pgfqpoint{6.159595in}{3.139536in}}%
\pgfpathlineto{\pgfqpoint{6.156423in}{3.139376in}}%
\pgfpathlineto{\pgfqpoint{6.153251in}{3.139796in}}%
\pgfpathlineto{\pgfqpoint{6.150079in}{3.139999in}}%
\pgfpathlineto{\pgfqpoint{6.146907in}{3.139997in}}%
\pgfpathlineto{\pgfqpoint{6.143735in}{3.139697in}}%
\pgfpathlineto{\pgfqpoint{6.140563in}{3.139653in}}%
\pgfpathlineto{\pgfqpoint{6.137391in}{3.139341in}}%
\pgfpathlineto{\pgfqpoint{6.134219in}{3.139060in}}%
\pgfpathlineto{\pgfqpoint{6.131047in}{3.139029in}}%
\pgfpathlineto{\pgfqpoint{6.127875in}{3.139012in}}%
\pgfpathlineto{\pgfqpoint{6.124703in}{3.138791in}}%
\pgfpathlineto{\pgfqpoint{6.121531in}{3.138945in}}%
\pgfpathlineto{\pgfqpoint{6.118359in}{3.139153in}}%
\pgfpathlineto{\pgfqpoint{6.115187in}{3.139534in}}%
\pgfpathlineto{\pgfqpoint{6.112015in}{3.139495in}}%
\pgfpathlineto{\pgfqpoint{6.108843in}{3.139454in}}%
\pgfpathlineto{\pgfqpoint{6.105671in}{3.139327in}}%
\pgfpathlineto{\pgfqpoint{6.102499in}{3.139441in}}%
\pgfpathlineto{\pgfqpoint{6.099326in}{3.139553in}}%
\pgfpathlineto{\pgfqpoint{6.096154in}{3.139666in}}%
\pgfpathlineto{\pgfqpoint{6.092982in}{3.139524in}}%
\pgfpathlineto{\pgfqpoint{6.089810in}{3.139953in}}%
\pgfpathlineto{\pgfqpoint{6.086638in}{3.140044in}}%
\pgfpathlineto{\pgfqpoint{6.083466in}{3.139834in}}%
\pgfpathlineto{\pgfqpoint{6.080294in}{3.139985in}}%
\pgfpathlineto{\pgfqpoint{6.077122in}{3.139947in}}%
\pgfpathlineto{\pgfqpoint{6.073950in}{3.139938in}}%
\pgfpathlineto{\pgfqpoint{6.070778in}{3.139808in}}%
\pgfpathlineto{\pgfqpoint{6.067606in}{3.139844in}}%
\pgfpathlineto{\pgfqpoint{6.064434in}{3.139877in}}%
\pgfpathlineto{\pgfqpoint{6.061262in}{3.140053in}}%
\pgfpathlineto{\pgfqpoint{6.058090in}{3.139623in}}%
\pgfpathlineto{\pgfqpoint{6.054918in}{3.139422in}}%
\pgfpathlineto{\pgfqpoint{6.051746in}{3.139355in}}%
\pgfpathlineto{\pgfqpoint{6.048574in}{3.139491in}}%
\pgfpathlineto{\pgfqpoint{6.045402in}{3.139170in}}%
\pgfpathlineto{\pgfqpoint{6.042230in}{3.139334in}}%
\pgfpathlineto{\pgfqpoint{6.039058in}{3.139278in}}%
\pgfpathlineto{\pgfqpoint{6.035886in}{3.139401in}}%
\pgfpathlineto{\pgfqpoint{6.032714in}{3.139658in}}%
\pgfpathlineto{\pgfqpoint{6.029542in}{3.139429in}}%
\pgfpathlineto{\pgfqpoint{6.026370in}{3.139323in}}%
\pgfpathlineto{\pgfqpoint{6.023197in}{3.139445in}}%
\pgfpathlineto{\pgfqpoint{6.020025in}{3.139648in}}%
\pgfpathlineto{\pgfqpoint{6.016853in}{3.139699in}}%
\pgfpathlineto{\pgfqpoint{6.013681in}{3.139781in}}%
\pgfpathlineto{\pgfqpoint{6.010509in}{3.139806in}}%
\pgfpathlineto{\pgfqpoint{6.007337in}{3.140049in}}%
\pgfpathlineto{\pgfqpoint{6.004165in}{3.139825in}}%
\pgfpathlineto{\pgfqpoint{6.000993in}{3.139669in}}%
\pgfpathlineto{\pgfqpoint{5.997821in}{3.140054in}}%
\pgfpathlineto{\pgfqpoint{5.994649in}{3.140380in}}%
\pgfpathlineto{\pgfqpoint{5.991477in}{3.140500in}}%
\pgfpathlineto{\pgfqpoint{5.988305in}{3.139951in}}%
\pgfpathlineto{\pgfqpoint{5.985133in}{3.139932in}}%
\pgfpathlineto{\pgfqpoint{5.981961in}{3.139732in}}%
\pgfpathlineto{\pgfqpoint{5.978789in}{3.139581in}}%
\pgfpathlineto{\pgfqpoint{5.975617in}{3.139514in}}%
\pgfpathlineto{\pgfqpoint{5.972445in}{3.139621in}}%
\pgfpathlineto{\pgfqpoint{5.969273in}{3.139690in}}%
\pgfpathlineto{\pgfqpoint{5.966101in}{3.139740in}}%
\pgfpathlineto{\pgfqpoint{5.962929in}{3.139423in}}%
\pgfpathlineto{\pgfqpoint{5.959757in}{3.139388in}}%
\pgfpathlineto{\pgfqpoint{5.956585in}{3.139317in}}%
\pgfpathlineto{\pgfqpoint{5.953413in}{3.139177in}}%
\pgfpathlineto{\pgfqpoint{5.950241in}{3.139206in}}%
\pgfpathlineto{\pgfqpoint{5.947069in}{3.139242in}}%
\pgfpathlineto{\pgfqpoint{5.943896in}{3.139234in}}%
\pgfpathlineto{\pgfqpoint{5.940724in}{3.138931in}}%
\pgfpathlineto{\pgfqpoint{5.937552in}{3.139007in}}%
\pgfpathlineto{\pgfqpoint{5.934380in}{3.138935in}}%
\pgfpathlineto{\pgfqpoint{5.931208in}{3.138557in}}%
\pgfpathlineto{\pgfqpoint{5.928036in}{3.138145in}}%
\pgfpathlineto{\pgfqpoint{5.924864in}{3.137991in}}%
\pgfpathlineto{\pgfqpoint{5.921692in}{3.138293in}}%
\pgfpathlineto{\pgfqpoint{5.918520in}{3.138050in}}%
\pgfpathlineto{\pgfqpoint{5.915348in}{3.137881in}}%
\pgfpathlineto{\pgfqpoint{5.912176in}{3.137817in}}%
\pgfpathlineto{\pgfqpoint{5.909004in}{3.137990in}}%
\pgfpathlineto{\pgfqpoint{5.905832in}{3.137664in}}%
\pgfpathlineto{\pgfqpoint{5.902660in}{3.137676in}}%
\pgfpathlineto{\pgfqpoint{5.899488in}{3.137564in}}%
\pgfpathlineto{\pgfqpoint{5.896316in}{3.137229in}}%
\pgfpathlineto{\pgfqpoint{5.893144in}{3.137225in}}%
\pgfpathlineto{\pgfqpoint{5.889972in}{3.137447in}}%
\pgfpathlineto{\pgfqpoint{5.886800in}{3.137409in}}%
\pgfpathlineto{\pgfqpoint{5.883628in}{3.137982in}}%
\pgfpathlineto{\pgfqpoint{5.880456in}{3.138039in}}%
\pgfpathlineto{\pgfqpoint{5.877284in}{3.138579in}}%
\pgfpathlineto{\pgfqpoint{5.874112in}{3.138498in}}%
\pgfpathlineto{\pgfqpoint{5.870940in}{3.137986in}}%
\pgfpathlineto{\pgfqpoint{5.867768in}{3.137965in}}%
\pgfpathlineto{\pgfqpoint{5.864595in}{3.137745in}}%
\pgfpathlineto{\pgfqpoint{5.861423in}{3.137536in}}%
\pgfpathlineto{\pgfqpoint{5.858251in}{3.137453in}}%
\pgfpathlineto{\pgfqpoint{5.855079in}{3.137490in}}%
\pgfpathlineto{\pgfqpoint{5.851907in}{3.137544in}}%
\pgfpathlineto{\pgfqpoint{5.848735in}{3.137437in}}%
\pgfpathlineto{\pgfqpoint{5.845563in}{3.137521in}}%
\pgfpathlineto{\pgfqpoint{5.842391in}{3.137769in}}%
\pgfpathlineto{\pgfqpoint{5.839219in}{3.137962in}}%
\pgfpathlineto{\pgfqpoint{5.836047in}{3.137929in}}%
\pgfpathlineto{\pgfqpoint{5.832875in}{3.137861in}}%
\pgfpathlineto{\pgfqpoint{5.829703in}{3.137966in}}%
\pgfpathlineto{\pgfqpoint{5.826531in}{3.137904in}}%
\pgfpathlineto{\pgfqpoint{5.823359in}{3.137908in}}%
\pgfpathlineto{\pgfqpoint{5.820187in}{3.138320in}}%
\pgfpathlineto{\pgfqpoint{5.817015in}{3.138534in}}%
\pgfpathlineto{\pgfqpoint{5.813843in}{3.138557in}}%
\pgfpathlineto{\pgfqpoint{5.810671in}{3.138317in}}%
\pgfpathlineto{\pgfqpoint{5.807499in}{3.138414in}}%
\pgfpathlineto{\pgfqpoint{5.804327in}{3.138356in}}%
\pgfpathlineto{\pgfqpoint{5.801155in}{3.138478in}}%
\pgfpathlineto{\pgfqpoint{5.797983in}{3.138319in}}%
\pgfpathlineto{\pgfqpoint{5.794811in}{3.138555in}}%
\pgfpathlineto{\pgfqpoint{5.791639in}{3.138172in}}%
\pgfpathlineto{\pgfqpoint{5.788466in}{3.138143in}}%
\pgfpathlineto{\pgfqpoint{5.785294in}{3.138268in}}%
\pgfpathlineto{\pgfqpoint{5.782122in}{3.138338in}}%
\pgfpathlineto{\pgfqpoint{5.778950in}{3.138447in}}%
\pgfpathlineto{\pgfqpoint{5.775778in}{3.138429in}}%
\pgfpathlineto{\pgfqpoint{5.772606in}{3.138524in}}%
\pgfpathlineto{\pgfqpoint{5.769434in}{3.138899in}}%
\pgfpathlineto{\pgfqpoint{5.766262in}{3.138774in}}%
\pgfpathlineto{\pgfqpoint{5.763090in}{3.138626in}}%
\pgfpathlineto{\pgfqpoint{5.759918in}{3.138539in}}%
\pgfpathlineto{\pgfqpoint{5.756746in}{3.138374in}}%
\pgfpathlineto{\pgfqpoint{5.753574in}{3.138293in}}%
\pgfpathlineto{\pgfqpoint{5.750402in}{3.138775in}}%
\pgfpathlineto{\pgfqpoint{5.747230in}{3.138909in}}%
\pgfpathlineto{\pgfqpoint{5.744058in}{3.138357in}}%
\pgfpathlineto{\pgfqpoint{5.740886in}{3.138240in}}%
\pgfpathlineto{\pgfqpoint{5.737714in}{3.137917in}}%
\pgfpathlineto{\pgfqpoint{5.734542in}{3.138057in}}%
\pgfpathlineto{\pgfqpoint{5.731370in}{3.138129in}}%
\pgfpathlineto{\pgfqpoint{5.728198in}{3.138082in}}%
\pgfpathlineto{\pgfqpoint{5.725026in}{3.137962in}}%
\pgfpathlineto{\pgfqpoint{5.721854in}{3.138133in}}%
\pgfpathlineto{\pgfqpoint{5.718682in}{3.138126in}}%
\pgfpathlineto{\pgfqpoint{5.715510in}{3.138575in}}%
\pgfpathlineto{\pgfqpoint{5.712338in}{3.138422in}}%
\pgfpathlineto{\pgfqpoint{5.709165in}{3.137980in}}%
\pgfpathlineto{\pgfqpoint{5.705993in}{3.137641in}}%
\pgfpathlineto{\pgfqpoint{5.702821in}{3.137618in}}%
\pgfpathlineto{\pgfqpoint{5.699649in}{3.137546in}}%
\pgfpathlineto{\pgfqpoint{5.696477in}{3.137476in}}%
\pgfpathlineto{\pgfqpoint{5.693305in}{3.137070in}}%
\pgfpathlineto{\pgfqpoint{5.690133in}{3.136956in}}%
\pgfpathlineto{\pgfqpoint{5.686961in}{3.136964in}}%
\pgfpathlineto{\pgfqpoint{5.683789in}{3.136722in}}%
\pgfpathlineto{\pgfqpoint{5.680617in}{3.136874in}}%
\pgfpathlineto{\pgfqpoint{5.677445in}{3.136782in}}%
\pgfpathlineto{\pgfqpoint{5.674273in}{3.136798in}}%
\pgfpathlineto{\pgfqpoint{5.671101in}{3.136915in}}%
\pgfpathlineto{\pgfqpoint{5.667929in}{3.136949in}}%
\pgfpathlineto{\pgfqpoint{5.664757in}{3.136957in}}%
\pgfpathlineto{\pgfqpoint{5.661585in}{3.137182in}}%
\pgfpathlineto{\pgfqpoint{5.658413in}{3.137222in}}%
\pgfpathlineto{\pgfqpoint{5.655241in}{3.137285in}}%
\pgfpathlineto{\pgfqpoint{5.652069in}{3.137249in}}%
\pgfpathlineto{\pgfqpoint{5.648897in}{3.137344in}}%
\pgfpathlineto{\pgfqpoint{5.645725in}{3.137277in}}%
\pgfpathlineto{\pgfqpoint{5.642553in}{3.137214in}}%
\pgfpathlineto{\pgfqpoint{5.639381in}{3.137607in}}%
\pgfpathlineto{\pgfqpoint{5.636209in}{3.137552in}}%
\pgfpathlineto{\pgfqpoint{5.633037in}{3.137412in}}%
\pgfpathlineto{\pgfqpoint{5.629864in}{3.137162in}}%
\pgfpathlineto{\pgfqpoint{5.626692in}{3.136777in}}%
\pgfpathlineto{\pgfqpoint{5.623520in}{3.136560in}}%
\pgfpathlineto{\pgfqpoint{5.620348in}{3.136087in}}%
\pgfpathlineto{\pgfqpoint{5.617176in}{3.136303in}}%
\pgfpathlineto{\pgfqpoint{5.614004in}{3.136213in}}%
\pgfpathlineto{\pgfqpoint{5.610832in}{3.136146in}}%
\pgfpathlineto{\pgfqpoint{5.607660in}{3.136149in}}%
\pgfpathlineto{\pgfqpoint{5.604488in}{3.135958in}}%
\pgfpathlineto{\pgfqpoint{5.601316in}{3.135822in}}%
\pgfpathlineto{\pgfqpoint{5.598144in}{3.136307in}}%
\pgfpathlineto{\pgfqpoint{5.594972in}{3.136446in}}%
\pgfpathlineto{\pgfqpoint{5.591800in}{3.136762in}}%
\pgfpathlineto{\pgfqpoint{5.588628in}{3.136599in}}%
\pgfpathlineto{\pgfqpoint{5.585456in}{3.136630in}}%
\pgfpathlineto{\pgfqpoint{5.582284in}{3.136538in}}%
\pgfpathlineto{\pgfqpoint{5.579112in}{3.136779in}}%
\pgfpathlineto{\pgfqpoint{5.575940in}{3.136449in}}%
\pgfpathlineto{\pgfqpoint{5.572768in}{3.136340in}}%
\pgfpathlineto{\pgfqpoint{5.569596in}{3.136480in}}%
\pgfpathlineto{\pgfqpoint{5.566424in}{3.136092in}}%
\pgfpathlineto{\pgfqpoint{5.563252in}{3.136293in}}%
\pgfpathlineto{\pgfqpoint{5.560080in}{3.136625in}}%
\pgfpathlineto{\pgfqpoint{5.556908in}{3.136402in}}%
\pgfpathlineto{\pgfqpoint{5.553735in}{3.136284in}}%
\pgfpathlineto{\pgfqpoint{5.550563in}{3.136544in}}%
\pgfpathlineto{\pgfqpoint{5.547391in}{3.136478in}}%
\pgfpathlineto{\pgfqpoint{5.544219in}{3.136123in}}%
\pgfpathlineto{\pgfqpoint{5.541047in}{3.135606in}}%
\pgfpathlineto{\pgfqpoint{5.537875in}{3.135544in}}%
\pgfpathlineto{\pgfqpoint{5.534703in}{3.135827in}}%
\pgfpathlineto{\pgfqpoint{5.531531in}{3.136137in}}%
\pgfpathlineto{\pgfqpoint{5.528359in}{3.136617in}}%
\pgfpathlineto{\pgfqpoint{5.525187in}{3.137092in}}%
\pgfpathlineto{\pgfqpoint{5.522015in}{3.137193in}}%
\pgfpathlineto{\pgfqpoint{5.518843in}{3.137557in}}%
\pgfpathlineto{\pgfqpoint{5.515671in}{3.137456in}}%
\pgfpathlineto{\pgfqpoint{5.512499in}{3.137814in}}%
\pgfpathlineto{\pgfqpoint{5.509327in}{3.137848in}}%
\pgfpathlineto{\pgfqpoint{5.506155in}{3.137738in}}%
\pgfpathlineto{\pgfqpoint{5.502983in}{3.137781in}}%
\pgfpathlineto{\pgfqpoint{5.499811in}{3.137529in}}%
\pgfpathlineto{\pgfqpoint{5.496639in}{3.137598in}}%
\pgfpathlineto{\pgfqpoint{5.493467in}{3.137663in}}%
\pgfpathlineto{\pgfqpoint{5.490295in}{3.137466in}}%
\pgfpathlineto{\pgfqpoint{5.487123in}{3.137598in}}%
\pgfpathlineto{\pgfqpoint{5.483951in}{3.137392in}}%
\pgfpathlineto{\pgfqpoint{5.480779in}{3.137194in}}%
\pgfpathlineto{\pgfqpoint{5.477607in}{3.137123in}}%
\pgfpathlineto{\pgfqpoint{5.474434in}{3.137254in}}%
\pgfpathlineto{\pgfqpoint{5.471262in}{3.137325in}}%
\pgfpathlineto{\pgfqpoint{5.468090in}{3.137531in}}%
\pgfpathlineto{\pgfqpoint{5.464918in}{3.137484in}}%
\pgfpathlineto{\pgfqpoint{5.461746in}{3.137718in}}%
\pgfpathlineto{\pgfqpoint{5.458574in}{3.137722in}}%
\pgfpathlineto{\pgfqpoint{5.455402in}{3.137601in}}%
\pgfpathlineto{\pgfqpoint{5.452230in}{3.137596in}}%
\pgfpathlineto{\pgfqpoint{5.449058in}{3.137720in}}%
\pgfpathlineto{\pgfqpoint{5.445886in}{3.137543in}}%
\pgfpathlineto{\pgfqpoint{5.442714in}{3.137370in}}%
\pgfpathlineto{\pgfqpoint{5.439542in}{3.137384in}}%
\pgfpathlineto{\pgfqpoint{5.436370in}{3.137369in}}%
\pgfpathlineto{\pgfqpoint{5.433198in}{3.137531in}}%
\pgfpathlineto{\pgfqpoint{5.430026in}{3.136926in}}%
\pgfpathlineto{\pgfqpoint{5.426854in}{3.136533in}}%
\pgfpathlineto{\pgfqpoint{5.423682in}{3.136361in}}%
\pgfpathlineto{\pgfqpoint{5.420510in}{3.136497in}}%
\pgfpathlineto{\pgfqpoint{5.417338in}{3.136613in}}%
\pgfpathlineto{\pgfqpoint{5.414166in}{3.136893in}}%
\pgfpathlineto{\pgfqpoint{5.410994in}{3.137227in}}%
\pgfpathlineto{\pgfqpoint{5.407822in}{3.137347in}}%
\pgfpathlineto{\pgfqpoint{5.404650in}{3.136893in}}%
\pgfpathlineto{\pgfqpoint{5.401478in}{3.136874in}}%
\pgfpathlineto{\pgfqpoint{5.398306in}{3.137212in}}%
\pgfpathlineto{\pgfqpoint{5.395133in}{3.137011in}}%
\pgfpathlineto{\pgfqpoint{5.391961in}{3.137216in}}%
\pgfpathlineto{\pgfqpoint{5.388789in}{3.137084in}}%
\pgfpathlineto{\pgfqpoint{5.385617in}{3.136796in}}%
\pgfpathlineto{\pgfqpoint{5.382445in}{3.136753in}}%
\pgfpathlineto{\pgfqpoint{5.379273in}{3.137081in}}%
\pgfpathlineto{\pgfqpoint{5.376101in}{3.137201in}}%
\pgfpathlineto{\pgfqpoint{5.372929in}{3.136979in}}%
\pgfpathlineto{\pgfqpoint{5.369757in}{3.136979in}}%
\pgfpathlineto{\pgfqpoint{5.366585in}{3.136810in}}%
\pgfpathlineto{\pgfqpoint{5.363413in}{3.136590in}}%
\pgfpathlineto{\pgfqpoint{5.360241in}{3.136811in}}%
\pgfpathlineto{\pgfqpoint{5.357069in}{3.136663in}}%
\pgfpathlineto{\pgfqpoint{5.353897in}{3.136375in}}%
\pgfpathlineto{\pgfqpoint{5.350725in}{3.136342in}}%
\pgfpathlineto{\pgfqpoint{5.347553in}{3.136306in}}%
\pgfpathlineto{\pgfqpoint{5.344381in}{3.136546in}}%
\pgfpathlineto{\pgfqpoint{5.341209in}{3.136319in}}%
\pgfpathlineto{\pgfqpoint{5.338037in}{3.136278in}}%
\pgfpathlineto{\pgfqpoint{5.334865in}{3.136133in}}%
\pgfpathlineto{\pgfqpoint{5.331693in}{3.136693in}}%
\pgfpathlineto{\pgfqpoint{5.328521in}{3.136645in}}%
\pgfpathlineto{\pgfqpoint{5.325349in}{3.136626in}}%
\pgfpathlineto{\pgfqpoint{5.322177in}{3.136739in}}%
\pgfpathlineto{\pgfqpoint{5.319004in}{3.136882in}}%
\pgfpathlineto{\pgfqpoint{5.315832in}{3.136873in}}%
\pgfpathlineto{\pgfqpoint{5.312660in}{3.136718in}}%
\pgfpathlineto{\pgfqpoint{5.309488in}{3.136603in}}%
\pgfpathlineto{\pgfqpoint{5.306316in}{3.136054in}}%
\pgfpathlineto{\pgfqpoint{5.303144in}{3.135764in}}%
\pgfpathlineto{\pgfqpoint{5.299972in}{3.135466in}}%
\pgfpathlineto{\pgfqpoint{5.296800in}{3.135374in}}%
\pgfpathlineto{\pgfqpoint{5.293628in}{3.135903in}}%
\pgfpathlineto{\pgfqpoint{5.290456in}{3.135932in}}%
\pgfpathlineto{\pgfqpoint{5.287284in}{3.135934in}}%
\pgfpathlineto{\pgfqpoint{5.284112in}{3.136212in}}%
\pgfpathlineto{\pgfqpoint{5.280940in}{3.136634in}}%
\pgfpathlineto{\pgfqpoint{5.277768in}{3.136157in}}%
\pgfpathlineto{\pgfqpoint{5.274596in}{3.136209in}}%
\pgfpathlineto{\pgfqpoint{5.271424in}{3.136127in}}%
\pgfpathlineto{\pgfqpoint{5.268252in}{3.136253in}}%
\pgfpathlineto{\pgfqpoint{5.265080in}{3.136141in}}%
\pgfpathlineto{\pgfqpoint{5.261908in}{3.135681in}}%
\pgfpathlineto{\pgfqpoint{5.258736in}{3.135050in}}%
\pgfpathlineto{\pgfqpoint{5.255564in}{3.134690in}}%
\pgfpathlineto{\pgfqpoint{5.252392in}{3.133866in}}%
\pgfpathlineto{\pgfqpoint{5.249220in}{3.133375in}}%
\pgfpathlineto{\pgfqpoint{5.246048in}{3.133494in}}%
\pgfpathlineto{\pgfqpoint{5.242876in}{3.132881in}}%
\pgfpathlineto{\pgfqpoint{5.239703in}{3.132513in}}%
\pgfpathlineto{\pgfqpoint{5.236531in}{3.132137in}}%
\pgfpathlineto{\pgfqpoint{5.233359in}{3.131993in}}%
\pgfpathlineto{\pgfqpoint{5.230187in}{3.131879in}}%
\pgfpathlineto{\pgfqpoint{5.227015in}{3.131949in}}%
\pgfpathlineto{\pgfqpoint{5.223843in}{3.131766in}}%
\pgfpathlineto{\pgfqpoint{5.220671in}{3.132324in}}%
\pgfpathlineto{\pgfqpoint{5.217499in}{3.132467in}}%
\pgfpathlineto{\pgfqpoint{5.214327in}{3.132269in}}%
\pgfpathlineto{\pgfqpoint{5.211155in}{3.132374in}}%
\pgfpathlineto{\pgfqpoint{5.207983in}{3.133198in}}%
\pgfpathlineto{\pgfqpoint{5.204811in}{3.133197in}}%
\pgfpathlineto{\pgfqpoint{5.201639in}{3.133161in}}%
\pgfpathlineto{\pgfqpoint{5.198467in}{3.133672in}}%
\pgfpathlineto{\pgfqpoint{5.195295in}{3.133760in}}%
\pgfpathlineto{\pgfqpoint{5.192123in}{3.133921in}}%
\pgfpathlineto{\pgfqpoint{5.188951in}{3.134338in}}%
\pgfpathlineto{\pgfqpoint{5.185779in}{3.134489in}}%
\pgfpathlineto{\pgfqpoint{5.182607in}{3.134589in}}%
\pgfpathlineto{\pgfqpoint{5.179435in}{3.134712in}}%
\pgfpathlineto{\pgfqpoint{5.176263in}{3.134821in}}%
\pgfpathlineto{\pgfqpoint{5.173091in}{3.134814in}}%
\pgfpathlineto{\pgfqpoint{5.169919in}{3.135281in}}%
\pgfpathlineto{\pgfqpoint{5.166747in}{3.135256in}}%
\pgfpathlineto{\pgfqpoint{5.163575in}{3.135414in}}%
\pgfpathlineto{\pgfqpoint{5.160402in}{3.135589in}}%
\pgfpathlineto{\pgfqpoint{5.157230in}{3.135541in}}%
\pgfpathlineto{\pgfqpoint{5.154058in}{3.135567in}}%
\pgfpathlineto{\pgfqpoint{5.150886in}{3.135502in}}%
\pgfpathlineto{\pgfqpoint{5.147714in}{3.135490in}}%
\pgfpathlineto{\pgfqpoint{5.144542in}{3.135518in}}%
\pgfpathlineto{\pgfqpoint{5.141370in}{3.135488in}}%
\pgfpathlineto{\pgfqpoint{5.138198in}{3.135701in}}%
\pgfpathlineto{\pgfqpoint{5.135026in}{3.136135in}}%
\pgfpathlineto{\pgfqpoint{5.131854in}{3.136156in}}%
\pgfpathlineto{\pgfqpoint{5.128682in}{3.136278in}}%
\pgfpathlineto{\pgfqpoint{5.125510in}{3.136099in}}%
\pgfpathlineto{\pgfqpoint{5.122338in}{3.136165in}}%
\pgfpathlineto{\pgfqpoint{5.119166in}{3.136257in}}%
\pgfpathlineto{\pgfqpoint{5.115994in}{3.135857in}}%
\pgfpathlineto{\pgfqpoint{5.112822in}{3.135615in}}%
\pgfpathlineto{\pgfqpoint{5.109650in}{3.135790in}}%
\pgfpathlineto{\pgfqpoint{5.106478in}{3.135677in}}%
\pgfpathlineto{\pgfqpoint{5.103306in}{3.135641in}}%
\pgfpathlineto{\pgfqpoint{5.100134in}{3.135176in}}%
\pgfpathlineto{\pgfqpoint{5.096962in}{3.135241in}}%
\pgfpathlineto{\pgfqpoint{5.093790in}{3.135181in}}%
\pgfpathlineto{\pgfqpoint{5.090618in}{3.135108in}}%
\pgfpathlineto{\pgfqpoint{5.087446in}{3.135319in}}%
\pgfpathlineto{\pgfqpoint{5.084273in}{3.135190in}}%
\pgfpathlineto{\pgfqpoint{5.081101in}{3.135010in}}%
\pgfpathlineto{\pgfqpoint{5.077929in}{3.135197in}}%
\pgfpathlineto{\pgfqpoint{5.074757in}{3.135338in}}%
\pgfpathlineto{\pgfqpoint{5.071585in}{3.135565in}}%
\pgfpathlineto{\pgfqpoint{5.068413in}{3.135621in}}%
\pgfpathlineto{\pgfqpoint{5.065241in}{3.135673in}}%
\pgfpathlineto{\pgfqpoint{5.062069in}{3.135439in}}%
\pgfpathlineto{\pgfqpoint{5.058897in}{3.135498in}}%
\pgfpathlineto{\pgfqpoint{5.055725in}{3.135253in}}%
\pgfpathlineto{\pgfqpoint{5.052553in}{3.135354in}}%
\pgfpathlineto{\pgfqpoint{5.049381in}{3.135385in}}%
\pgfpathlineto{\pgfqpoint{5.046209in}{3.136063in}}%
\pgfpathlineto{\pgfqpoint{5.043037in}{3.136106in}}%
\pgfpathlineto{\pgfqpoint{5.039865in}{3.136159in}}%
\pgfpathlineto{\pgfqpoint{5.036693in}{3.136027in}}%
\pgfpathlineto{\pgfqpoint{5.033521in}{3.135968in}}%
\pgfpathlineto{\pgfqpoint{5.030349in}{3.135802in}}%
\pgfpathlineto{\pgfqpoint{5.027177in}{3.135935in}}%
\pgfpathlineto{\pgfqpoint{5.024005in}{3.135991in}}%
\pgfpathlineto{\pgfqpoint{5.020833in}{3.136399in}}%
\pgfpathlineto{\pgfqpoint{5.017661in}{3.136892in}}%
\pgfpathlineto{\pgfqpoint{5.014489in}{3.136800in}}%
\pgfpathlineto{\pgfqpoint{5.011317in}{3.136792in}}%
\pgfpathlineto{\pgfqpoint{5.008145in}{3.136598in}}%
\pgfpathlineto{\pgfqpoint{5.004972in}{3.136605in}}%
\pgfpathlineto{\pgfqpoint{5.001800in}{3.136912in}}%
\pgfpathlineto{\pgfqpoint{4.998628in}{3.136794in}}%
\pgfpathlineto{\pgfqpoint{4.995456in}{3.136809in}}%
\pgfpathlineto{\pgfqpoint{4.992284in}{3.136749in}}%
\pgfpathlineto{\pgfqpoint{4.989112in}{3.136212in}}%
\pgfpathlineto{\pgfqpoint{4.985940in}{3.135965in}}%
\pgfpathlineto{\pgfqpoint{4.982768in}{3.135898in}}%
\pgfpathlineto{\pgfqpoint{4.979596in}{3.135829in}}%
\pgfpathlineto{\pgfqpoint{4.976424in}{3.135353in}}%
\pgfpathlineto{\pgfqpoint{4.973252in}{3.135328in}}%
\pgfpathlineto{\pgfqpoint{4.970080in}{3.135470in}}%
\pgfpathlineto{\pgfqpoint{4.966908in}{3.135418in}}%
\pgfpathlineto{\pgfqpoint{4.963736in}{3.135241in}}%
\pgfpathlineto{\pgfqpoint{4.960564in}{3.135526in}}%
\pgfpathlineto{\pgfqpoint{4.957392in}{3.135812in}}%
\pgfpathlineto{\pgfqpoint{4.954220in}{3.135597in}}%
\pgfpathlineto{\pgfqpoint{4.951048in}{3.135684in}}%
\pgfpathlineto{\pgfqpoint{4.947876in}{3.135728in}}%
\pgfpathlineto{\pgfqpoint{4.944704in}{3.135898in}}%
\pgfpathlineto{\pgfqpoint{4.941532in}{3.135783in}}%
\pgfpathlineto{\pgfqpoint{4.938360in}{3.135799in}}%
\pgfpathlineto{\pgfqpoint{4.935188in}{3.135866in}}%
\pgfpathlineto{\pgfqpoint{4.932016in}{3.135947in}}%
\pgfpathlineto{\pgfqpoint{4.928844in}{3.135956in}}%
\pgfpathlineto{\pgfqpoint{4.925671in}{3.136087in}}%
\pgfpathlineto{\pgfqpoint{4.922499in}{3.135998in}}%
\pgfpathlineto{\pgfqpoint{4.919327in}{3.136080in}}%
\pgfpathlineto{\pgfqpoint{4.916155in}{3.136389in}}%
\pgfpathlineto{\pgfqpoint{4.912983in}{3.136530in}}%
\pgfpathlineto{\pgfqpoint{4.909811in}{3.136999in}}%
\pgfpathlineto{\pgfqpoint{4.906639in}{3.137191in}}%
\pgfpathlineto{\pgfqpoint{4.903467in}{3.137130in}}%
\pgfpathlineto{\pgfqpoint{4.900295in}{3.137081in}}%
\pgfpathlineto{\pgfqpoint{4.897123in}{3.137406in}}%
\pgfpathlineto{\pgfqpoint{4.893951in}{3.137329in}}%
\pgfpathlineto{\pgfqpoint{4.890779in}{3.137113in}}%
\pgfpathlineto{\pgfqpoint{4.887607in}{3.137202in}}%
\pgfpathlineto{\pgfqpoint{4.884435in}{3.137033in}}%
\pgfpathlineto{\pgfqpoint{4.881263in}{3.136920in}}%
\pgfpathlineto{\pgfqpoint{4.878091in}{3.137168in}}%
\pgfpathlineto{\pgfqpoint{4.874919in}{3.136969in}}%
\pgfpathlineto{\pgfqpoint{4.871747in}{3.137683in}}%
\pgfpathlineto{\pgfqpoint{4.868575in}{3.137785in}}%
\pgfpathlineto{\pgfqpoint{4.865403in}{3.137374in}}%
\pgfpathlineto{\pgfqpoint{4.862231in}{3.137401in}}%
\pgfpathlineto{\pgfqpoint{4.859059in}{3.137375in}}%
\pgfpathlineto{\pgfqpoint{4.855887in}{3.137417in}}%
\pgfpathlineto{\pgfqpoint{4.852715in}{3.137326in}}%
\pgfpathlineto{\pgfqpoint{4.849542in}{3.137488in}}%
\pgfpathlineto{\pgfqpoint{4.846370in}{3.137187in}}%
\pgfpathlineto{\pgfqpoint{4.843198in}{3.136873in}}%
\pgfpathlineto{\pgfqpoint{4.840026in}{3.136704in}}%
\pgfpathlineto{\pgfqpoint{4.836854in}{3.136747in}}%
\pgfpathlineto{\pgfqpoint{4.833682in}{3.137027in}}%
\pgfpathlineto{\pgfqpoint{4.830510in}{3.137043in}}%
\pgfpathlineto{\pgfqpoint{4.827338in}{3.137029in}}%
\pgfpathlineto{\pgfqpoint{4.824166in}{3.136888in}}%
\pgfpathlineto{\pgfqpoint{4.820994in}{3.137204in}}%
\pgfpathlineto{\pgfqpoint{4.817822in}{3.137314in}}%
\pgfpathlineto{\pgfqpoint{4.814650in}{3.137371in}}%
\pgfpathlineto{\pgfqpoint{4.811478in}{3.137123in}}%
\pgfpathlineto{\pgfqpoint{4.808306in}{3.137069in}}%
\pgfpathlineto{\pgfqpoint{4.805134in}{3.137227in}}%
\pgfpathlineto{\pgfqpoint{4.801962in}{3.137435in}}%
\pgfpathlineto{\pgfqpoint{4.798790in}{3.137462in}}%
\pgfpathlineto{\pgfqpoint{4.795618in}{3.137307in}}%
\pgfpathlineto{\pgfqpoint{4.792446in}{3.137249in}}%
\pgfpathlineto{\pgfqpoint{4.789274in}{3.137361in}}%
\pgfpathlineto{\pgfqpoint{4.786102in}{3.137319in}}%
\pgfpathlineto{\pgfqpoint{4.782930in}{3.137205in}}%
\pgfpathlineto{\pgfqpoint{4.779758in}{3.136964in}}%
\pgfpathlineto{\pgfqpoint{4.776586in}{3.137215in}}%
\pgfpathlineto{\pgfqpoint{4.773414in}{3.137313in}}%
\pgfpathlineto{\pgfqpoint{4.770241in}{3.137197in}}%
\pgfpathlineto{\pgfqpoint{4.767069in}{3.136800in}}%
\pgfpathlineto{\pgfqpoint{4.763897in}{3.136819in}}%
\pgfpathlineto{\pgfqpoint{4.760725in}{3.136874in}}%
\pgfpathlineto{\pgfqpoint{4.757553in}{3.136798in}}%
\pgfpathlineto{\pgfqpoint{4.754381in}{3.136707in}}%
\pgfpathlineto{\pgfqpoint{4.751209in}{3.136770in}}%
\pgfpathlineto{\pgfqpoint{4.748037in}{3.136600in}}%
\pgfpathlineto{\pgfqpoint{4.744865in}{3.136753in}}%
\pgfpathlineto{\pgfqpoint{4.741693in}{3.136965in}}%
\pgfpathlineto{\pgfqpoint{4.738521in}{3.136718in}}%
\pgfpathlineto{\pgfqpoint{4.735349in}{3.136521in}}%
\pgfpathlineto{\pgfqpoint{4.732177in}{3.136400in}}%
\pgfpathlineto{\pgfqpoint{4.729005in}{3.136656in}}%
\pgfpathlineto{\pgfqpoint{4.725833in}{3.136490in}}%
\pgfpathlineto{\pgfqpoint{4.722661in}{3.136132in}}%
\pgfpathlineto{\pgfqpoint{4.719489in}{3.136168in}}%
\pgfpathlineto{\pgfqpoint{4.716317in}{3.136212in}}%
\pgfpathlineto{\pgfqpoint{4.713145in}{3.136168in}}%
\pgfpathlineto{\pgfqpoint{4.709973in}{3.136157in}}%
\pgfpathlineto{\pgfqpoint{4.706801in}{3.135863in}}%
\pgfpathlineto{\pgfqpoint{4.703629in}{3.136036in}}%
\pgfpathlineto{\pgfqpoint{4.700457in}{3.136063in}}%
\pgfpathlineto{\pgfqpoint{4.697285in}{3.136054in}}%
\pgfpathlineto{\pgfqpoint{4.694112in}{3.136063in}}%
\pgfpathlineto{\pgfqpoint{4.690940in}{3.135621in}}%
\pgfpathlineto{\pgfqpoint{4.687768in}{3.136092in}}%
\pgfpathlineto{\pgfqpoint{4.684596in}{3.136161in}}%
\pgfpathlineto{\pgfqpoint{4.681424in}{3.136187in}}%
\pgfpathlineto{\pgfqpoint{4.678252in}{3.136107in}}%
\pgfpathlineto{\pgfqpoint{4.675080in}{3.136107in}}%
\pgfpathlineto{\pgfqpoint{4.671908in}{3.135982in}}%
\pgfpathlineto{\pgfqpoint{4.668736in}{3.136254in}}%
\pgfpathlineto{\pgfqpoint{4.665564in}{3.136115in}}%
\pgfpathlineto{\pgfqpoint{4.662392in}{3.136087in}}%
\pgfpathlineto{\pgfqpoint{4.659220in}{3.136136in}}%
\pgfpathlineto{\pgfqpoint{4.656048in}{3.135701in}}%
\pgfpathlineto{\pgfqpoint{4.652876in}{3.136125in}}%
\pgfpathlineto{\pgfqpoint{4.649704in}{3.136222in}}%
\pgfpathlineto{\pgfqpoint{4.646532in}{3.136194in}}%
\pgfpathlineto{\pgfqpoint{4.643360in}{3.136279in}}%
\pgfpathlineto{\pgfqpoint{4.640188in}{3.136280in}}%
\pgfpathlineto{\pgfqpoint{4.637016in}{3.136188in}}%
\pgfpathlineto{\pgfqpoint{4.633844in}{3.136329in}}%
\pgfpathlineto{\pgfqpoint{4.630672in}{3.136070in}}%
\pgfpathlineto{\pgfqpoint{4.627500in}{3.135953in}}%
\pgfpathlineto{\pgfqpoint{4.624328in}{3.136033in}}%
\pgfpathlineto{\pgfqpoint{4.621156in}{3.136024in}}%
\pgfpathlineto{\pgfqpoint{4.617984in}{3.136150in}}%
\pgfpathlineto{\pgfqpoint{4.614811in}{3.135638in}}%
\pgfpathlineto{\pgfqpoint{4.611639in}{3.135741in}}%
\pgfpathlineto{\pgfqpoint{4.608467in}{3.135724in}}%
\pgfpathlineto{\pgfqpoint{4.605295in}{3.135781in}}%
\pgfpathlineto{\pgfqpoint{4.602123in}{3.135707in}}%
\pgfpathlineto{\pgfqpoint{4.598951in}{3.135565in}}%
\pgfpathlineto{\pgfqpoint{4.595779in}{3.135593in}}%
\pgfpathlineto{\pgfqpoint{4.592607in}{3.134892in}}%
\pgfpathlineto{\pgfqpoint{4.589435in}{3.134861in}}%
\pgfpathlineto{\pgfqpoint{4.586263in}{3.134770in}}%
\pgfpathlineto{\pgfqpoint{4.583091in}{3.134929in}}%
\pgfpathlineto{\pgfqpoint{4.579919in}{3.135048in}}%
\pgfpathlineto{\pgfqpoint{4.576747in}{3.135219in}}%
\pgfpathlineto{\pgfqpoint{4.573575in}{3.135562in}}%
\pgfpathlineto{\pgfqpoint{4.570403in}{3.135385in}}%
\pgfpathlineto{\pgfqpoint{4.567231in}{3.135411in}}%
\pgfpathlineto{\pgfqpoint{4.564059in}{3.135758in}}%
\pgfpathlineto{\pgfqpoint{4.560887in}{3.135582in}}%
\pgfpathlineto{\pgfqpoint{4.557715in}{3.135536in}}%
\pgfpathlineto{\pgfqpoint{4.554543in}{3.135166in}}%
\pgfpathlineto{\pgfqpoint{4.551371in}{3.135192in}}%
\pgfpathlineto{\pgfqpoint{4.548199in}{3.135072in}}%
\pgfpathlineto{\pgfqpoint{4.545027in}{3.135002in}}%
\pgfpathlineto{\pgfqpoint{4.541855in}{3.135109in}}%
\pgfpathlineto{\pgfqpoint{4.538683in}{3.134948in}}%
\pgfpathlineto{\pgfqpoint{4.535510in}{3.134886in}}%
\pgfpathlineto{\pgfqpoint{4.532338in}{3.135186in}}%
\pgfpathlineto{\pgfqpoint{4.529166in}{3.135291in}}%
\pgfpathlineto{\pgfqpoint{4.525994in}{3.135061in}}%
\pgfpathlineto{\pgfqpoint{4.522822in}{3.135075in}}%
\pgfpathlineto{\pgfqpoint{4.519650in}{3.134810in}}%
\pgfpathlineto{\pgfqpoint{4.516478in}{3.134455in}}%
\pgfpathlineto{\pgfqpoint{4.513306in}{3.134084in}}%
\pgfpathlineto{\pgfqpoint{4.510134in}{3.134094in}}%
\pgfpathlineto{\pgfqpoint{4.506962in}{3.133995in}}%
\pgfpathlineto{\pgfqpoint{4.503790in}{3.133869in}}%
\pgfpathlineto{\pgfqpoint{4.500618in}{3.133446in}}%
\pgfpathlineto{\pgfqpoint{4.497446in}{3.133231in}}%
\pgfpathlineto{\pgfqpoint{4.494274in}{3.133260in}}%
\pgfpathlineto{\pgfqpoint{4.491102in}{3.133331in}}%
\pgfpathlineto{\pgfqpoint{4.487930in}{3.133441in}}%
\pgfpathlineto{\pgfqpoint{4.484758in}{3.133229in}}%
\pgfpathlineto{\pgfqpoint{4.481586in}{3.133126in}}%
\pgfpathlineto{\pgfqpoint{4.478414in}{3.132892in}}%
\pgfpathlineto{\pgfqpoint{4.475242in}{3.132512in}}%
\pgfpathlineto{\pgfqpoint{4.472070in}{3.132501in}}%
\pgfpathlineto{\pgfqpoint{4.468898in}{3.132505in}}%
\pgfpathlineto{\pgfqpoint{4.465726in}{3.132429in}}%
\pgfpathlineto{\pgfqpoint{4.462554in}{3.132322in}}%
\pgfpathlineto{\pgfqpoint{4.459381in}{3.132212in}}%
\pgfpathlineto{\pgfqpoint{4.456209in}{3.132193in}}%
\pgfpathlineto{\pgfqpoint{4.453037in}{3.132723in}}%
\pgfpathlineto{\pgfqpoint{4.449865in}{3.132966in}}%
\pgfpathlineto{\pgfqpoint{4.446693in}{3.132847in}}%
\pgfpathlineto{\pgfqpoint{4.443521in}{3.133196in}}%
\pgfpathlineto{\pgfqpoint{4.440349in}{3.133144in}}%
\pgfpathlineto{\pgfqpoint{4.437177in}{3.133042in}}%
\pgfpathlineto{\pgfqpoint{4.434005in}{3.133054in}}%
\pgfpathlineto{\pgfqpoint{4.430833in}{3.133507in}}%
\pgfpathlineto{\pgfqpoint{4.427661in}{3.133579in}}%
\pgfpathlineto{\pgfqpoint{4.424489in}{3.133674in}}%
\pgfpathlineto{\pgfqpoint{4.421317in}{3.133543in}}%
\pgfpathlineto{\pgfqpoint{4.418145in}{3.133629in}}%
\pgfpathlineto{\pgfqpoint{4.414973in}{3.133755in}}%
\pgfpathlineto{\pgfqpoint{4.411801in}{3.133561in}}%
\pgfpathlineto{\pgfqpoint{4.408629in}{3.133211in}}%
\pgfpathlineto{\pgfqpoint{4.405457in}{3.133161in}}%
\pgfpathlineto{\pgfqpoint{4.402285in}{3.132872in}}%
\pgfpathlineto{\pgfqpoint{4.399113in}{3.132682in}}%
\pgfpathlineto{\pgfqpoint{4.395941in}{3.132802in}}%
\pgfpathlineto{\pgfqpoint{4.392769in}{3.132495in}}%
\pgfpathlineto{\pgfqpoint{4.389597in}{3.132473in}}%
\pgfpathlineto{\pgfqpoint{4.386425in}{3.132181in}}%
\pgfpathlineto{\pgfqpoint{4.383253in}{3.131824in}}%
\pgfpathlineto{\pgfqpoint{4.380080in}{3.131675in}}%
\pgfpathlineto{\pgfqpoint{4.376908in}{3.131601in}}%
\pgfpathlineto{\pgfqpoint{4.373736in}{3.131370in}}%
\pgfpathlineto{\pgfqpoint{4.370564in}{3.131196in}}%
\pgfpathlineto{\pgfqpoint{4.367392in}{3.131237in}}%
\pgfpathlineto{\pgfqpoint{4.364220in}{3.131005in}}%
\pgfpathlineto{\pgfqpoint{4.361048in}{3.130969in}}%
\pgfpathlineto{\pgfqpoint{4.357876in}{3.130839in}}%
\pgfpathlineto{\pgfqpoint{4.354704in}{3.130886in}}%
\pgfpathlineto{\pgfqpoint{4.351532in}{3.130713in}}%
\pgfpathlineto{\pgfqpoint{4.348360in}{3.130619in}}%
\pgfpathlineto{\pgfqpoint{4.345188in}{3.129957in}}%
\pgfpathlineto{\pgfqpoint{4.342016in}{3.129925in}}%
\pgfpathlineto{\pgfqpoint{4.338844in}{3.130020in}}%
\pgfpathlineto{\pgfqpoint{4.335672in}{3.129851in}}%
\pgfpathlineto{\pgfqpoint{4.332500in}{3.129981in}}%
\pgfpathlineto{\pgfqpoint{4.329328in}{3.129987in}}%
\pgfpathlineto{\pgfqpoint{4.326156in}{3.130114in}}%
\pgfpathlineto{\pgfqpoint{4.322984in}{3.130272in}}%
\pgfpathlineto{\pgfqpoint{4.319812in}{3.130151in}}%
\pgfpathlineto{\pgfqpoint{4.316640in}{3.130279in}}%
\pgfpathlineto{\pgfqpoint{4.313468in}{3.130109in}}%
\pgfpathlineto{\pgfqpoint{4.310296in}{3.130060in}}%
\pgfpathlineto{\pgfqpoint{4.307124in}{3.130005in}}%
\pgfpathlineto{\pgfqpoint{4.303952in}{3.130082in}}%
\pgfpathlineto{\pgfqpoint{4.300779in}{3.130521in}}%
\pgfpathlineto{\pgfqpoint{4.297607in}{3.130572in}}%
\pgfpathlineto{\pgfqpoint{4.294435in}{3.130251in}}%
\pgfpathlineto{\pgfqpoint{4.291263in}{3.129935in}}%
\pgfpathlineto{\pgfqpoint{4.288091in}{3.130030in}}%
\pgfpathlineto{\pgfqpoint{4.284919in}{3.129913in}}%
\pgfpathlineto{\pgfqpoint{4.281747in}{3.129548in}}%
\pgfpathlineto{\pgfqpoint{4.278575in}{3.129106in}}%
\pgfpathlineto{\pgfqpoint{4.275403in}{3.129296in}}%
\pgfpathlineto{\pgfqpoint{4.272231in}{3.129389in}}%
\pgfpathlineto{\pgfqpoint{4.269059in}{3.129716in}}%
\pgfpathlineto{\pgfqpoint{4.265887in}{3.129873in}}%
\pgfpathlineto{\pgfqpoint{4.262715in}{3.129928in}}%
\pgfpathlineto{\pgfqpoint{4.259543in}{3.130051in}}%
\pgfpathlineto{\pgfqpoint{4.256371in}{3.130228in}}%
\pgfpathlineto{\pgfqpoint{4.253199in}{3.129984in}}%
\pgfpathlineto{\pgfqpoint{4.250027in}{3.130250in}}%
\pgfpathlineto{\pgfqpoint{4.246855in}{3.130214in}}%
\pgfpathlineto{\pgfqpoint{4.243683in}{3.130243in}}%
\pgfpathlineto{\pgfqpoint{4.240511in}{3.129964in}}%
\pgfpathlineto{\pgfqpoint{4.237339in}{3.129897in}}%
\pgfpathlineto{\pgfqpoint{4.234167in}{3.129792in}}%
\pgfpathlineto{\pgfqpoint{4.230995in}{3.129858in}}%
\pgfpathlineto{\pgfqpoint{4.227823in}{3.129713in}}%
\pgfpathlineto{\pgfqpoint{4.224650in}{3.129913in}}%
\pgfpathlineto{\pgfqpoint{4.221478in}{3.129805in}}%
\pgfpathlineto{\pgfqpoint{4.218306in}{3.129789in}}%
\pgfpathlineto{\pgfqpoint{4.215134in}{3.129704in}}%
\pgfpathlineto{\pgfqpoint{4.211962in}{3.129868in}}%
\pgfpathlineto{\pgfqpoint{4.208790in}{3.130069in}}%
\pgfpathlineto{\pgfqpoint{4.205618in}{3.129960in}}%
\pgfpathlineto{\pgfqpoint{4.202446in}{3.130017in}}%
\pgfpathlineto{\pgfqpoint{4.199274in}{3.129936in}}%
\pgfpathlineto{\pgfqpoint{4.196102in}{3.129880in}}%
\pgfpathlineto{\pgfqpoint{4.192930in}{3.129888in}}%
\pgfpathlineto{\pgfqpoint{4.189758in}{3.130200in}}%
\pgfpathlineto{\pgfqpoint{4.186586in}{3.130238in}}%
\pgfpathlineto{\pgfqpoint{4.183414in}{3.130422in}}%
\pgfpathlineto{\pgfqpoint{4.180242in}{3.130413in}}%
\pgfpathlineto{\pgfqpoint{4.177070in}{3.129825in}}%
\pgfpathlineto{\pgfqpoint{4.173898in}{3.129636in}}%
\pgfpathlineto{\pgfqpoint{4.170726in}{3.129736in}}%
\pgfpathlineto{\pgfqpoint{4.167554in}{3.129755in}}%
\pgfpathlineto{\pgfqpoint{4.164382in}{3.129610in}}%
\pgfpathlineto{\pgfqpoint{4.161210in}{3.129533in}}%
\pgfpathlineto{\pgfqpoint{4.158038in}{3.129557in}}%
\pgfpathlineto{\pgfqpoint{4.154866in}{3.129581in}}%
\pgfpathlineto{\pgfqpoint{4.151694in}{3.129413in}}%
\pgfpathlineto{\pgfqpoint{4.148522in}{3.129495in}}%
\pgfpathlineto{\pgfqpoint{4.145349in}{3.128862in}}%
\pgfpathlineto{\pgfqpoint{4.142177in}{3.128707in}}%
\pgfpathlineto{\pgfqpoint{4.139005in}{3.128528in}}%
\pgfpathlineto{\pgfqpoint{4.135833in}{3.128390in}}%
\pgfpathlineto{\pgfqpoint{4.132661in}{3.127884in}}%
\pgfpathlineto{\pgfqpoint{4.129489in}{3.127819in}}%
\pgfpathlineto{\pgfqpoint{4.126317in}{3.127716in}}%
\pgfpathlineto{\pgfqpoint{4.123145in}{3.127599in}}%
\pgfpathlineto{\pgfqpoint{4.119973in}{3.127390in}}%
\pgfpathlineto{\pgfqpoint{4.116801in}{3.127258in}}%
\pgfpathlineto{\pgfqpoint{4.113629in}{3.127090in}}%
\pgfpathlineto{\pgfqpoint{4.110457in}{3.126588in}}%
\pgfpathlineto{\pgfqpoint{4.107285in}{3.126691in}}%
\pgfpathlineto{\pgfqpoint{4.104113in}{3.126792in}}%
\pgfpathlineto{\pgfqpoint{4.100941in}{3.126745in}}%
\pgfpathlineto{\pgfqpoint{4.097769in}{3.126649in}}%
\pgfpathlineto{\pgfqpoint{4.094597in}{3.126853in}}%
\pgfpathlineto{\pgfqpoint{4.091425in}{3.126702in}}%
\pgfpathlineto{\pgfqpoint{4.088253in}{3.126768in}}%
\pgfpathlineto{\pgfqpoint{4.085081in}{3.126260in}}%
\pgfpathlineto{\pgfqpoint{4.081909in}{3.125931in}}%
\pgfpathlineto{\pgfqpoint{4.078737in}{3.125667in}}%
\pgfpathlineto{\pgfqpoint{4.075565in}{3.126092in}}%
\pgfpathlineto{\pgfqpoint{4.072393in}{3.126255in}}%
\pgfpathlineto{\pgfqpoint{4.069221in}{3.126257in}}%
\pgfpathlineto{\pgfqpoint{4.066048in}{3.126220in}}%
\pgfpathlineto{\pgfqpoint{4.062876in}{3.126370in}}%
\pgfpathlineto{\pgfqpoint{4.059704in}{3.126528in}}%
\pgfpathlineto{\pgfqpoint{4.056532in}{3.126606in}}%
\pgfpathlineto{\pgfqpoint{4.053360in}{3.126583in}}%
\pgfpathlineto{\pgfqpoint{4.050188in}{3.126831in}}%
\pgfpathlineto{\pgfqpoint{4.047016in}{3.126944in}}%
\pgfpathlineto{\pgfqpoint{4.043844in}{3.127233in}}%
\pgfpathlineto{\pgfqpoint{4.040672in}{3.127473in}}%
\pgfpathlineto{\pgfqpoint{4.037500in}{3.127374in}}%
\pgfpathlineto{\pgfqpoint{4.034328in}{3.127262in}}%
\pgfpathlineto{\pgfqpoint{4.031156in}{3.127247in}}%
\pgfpathlineto{\pgfqpoint{4.027984in}{3.126897in}}%
\pgfpathlineto{\pgfqpoint{4.024812in}{3.126922in}}%
\pgfpathlineto{\pgfqpoint{4.021640in}{3.126929in}}%
\pgfpathlineto{\pgfqpoint{4.018468in}{3.126991in}}%
\pgfpathlineto{\pgfqpoint{4.015296in}{3.127082in}}%
\pgfpathlineto{\pgfqpoint{4.012124in}{3.127057in}}%
\pgfpathlineto{\pgfqpoint{4.008952in}{3.127107in}}%
\pgfpathlineto{\pgfqpoint{4.005780in}{3.126991in}}%
\pgfpathlineto{\pgfqpoint{4.002608in}{3.126986in}}%
\pgfpathlineto{\pgfqpoint{3.999436in}{3.126721in}}%
\pgfpathlineto{\pgfqpoint{3.996264in}{3.126725in}}%
\pgfpathlineto{\pgfqpoint{3.993092in}{3.126736in}}%
\pgfpathlineto{\pgfqpoint{3.989919in}{3.126745in}}%
\pgfpathlineto{\pgfqpoint{3.986747in}{3.126731in}}%
\pgfpathlineto{\pgfqpoint{3.983575in}{3.126921in}}%
\pgfpathlineto{\pgfqpoint{3.980403in}{3.126800in}}%
\pgfpathlineto{\pgfqpoint{3.977231in}{3.126946in}}%
\pgfpathlineto{\pgfqpoint{3.974059in}{3.127054in}}%
\pgfpathlineto{\pgfqpoint{3.970887in}{3.126921in}}%
\pgfpathlineto{\pgfqpoint{3.967715in}{3.126876in}}%
\pgfpathlineto{\pgfqpoint{3.964543in}{3.127150in}}%
\pgfpathlineto{\pgfqpoint{3.961371in}{3.127649in}}%
\pgfpathlineto{\pgfqpoint{3.958199in}{3.127608in}}%
\pgfpathlineto{\pgfqpoint{3.955027in}{3.127783in}}%
\pgfpathlineto{\pgfqpoint{3.951855in}{3.128260in}}%
\pgfpathlineto{\pgfqpoint{3.948683in}{3.128346in}}%
\pgfpathlineto{\pgfqpoint{3.945511in}{3.128257in}}%
\pgfpathlineto{\pgfqpoint{3.942339in}{3.128287in}}%
\pgfpathlineto{\pgfqpoint{3.939167in}{3.128467in}}%
\pgfpathlineto{\pgfqpoint{3.935995in}{3.128173in}}%
\pgfpathlineto{\pgfqpoint{3.932823in}{3.128228in}}%
\pgfpathlineto{\pgfqpoint{3.929651in}{3.128790in}}%
\pgfpathlineto{\pgfqpoint{3.926479in}{3.129121in}}%
\pgfpathlineto{\pgfqpoint{3.923307in}{3.129207in}}%
\pgfpathlineto{\pgfqpoint{3.920135in}{3.129305in}}%
\pgfpathlineto{\pgfqpoint{3.916963in}{3.129395in}}%
\pgfpathlineto{\pgfqpoint{3.913791in}{3.129291in}}%
\pgfpathlineto{\pgfqpoint{3.910618in}{3.129557in}}%
\pgfpathlineto{\pgfqpoint{3.907446in}{3.129454in}}%
\pgfpathlineto{\pgfqpoint{3.904274in}{3.129453in}}%
\pgfpathlineto{\pgfqpoint{3.901102in}{3.129298in}}%
\pgfpathlineto{\pgfqpoint{3.897930in}{3.129290in}}%
\pgfpathlineto{\pgfqpoint{3.894758in}{3.129346in}}%
\pgfpathlineto{\pgfqpoint{3.891586in}{3.129275in}}%
\pgfpathlineto{\pgfqpoint{3.888414in}{3.129357in}}%
\pgfpathlineto{\pgfqpoint{3.885242in}{3.129438in}}%
\pgfpathlineto{\pgfqpoint{3.882070in}{3.129306in}}%
\pgfpathlineto{\pgfqpoint{3.878898in}{3.129708in}}%
\pgfpathlineto{\pgfqpoint{3.875726in}{3.130299in}}%
\pgfpathlineto{\pgfqpoint{3.872554in}{3.130644in}}%
\pgfpathlineto{\pgfqpoint{3.869382in}{3.131076in}}%
\pgfpathlineto{\pgfqpoint{3.866210in}{3.131405in}}%
\pgfpathlineto{\pgfqpoint{3.863038in}{3.131412in}}%
\pgfpathlineto{\pgfqpoint{3.859866in}{3.131525in}}%
\pgfpathlineto{\pgfqpoint{3.856694in}{3.131442in}}%
\pgfpathlineto{\pgfqpoint{3.853522in}{3.131433in}}%
\pgfpathlineto{\pgfqpoint{3.850350in}{3.131422in}}%
\pgfpathlineto{\pgfqpoint{3.847178in}{3.131694in}}%
\pgfpathlineto{\pgfqpoint{3.844006in}{3.131460in}}%
\pgfpathlineto{\pgfqpoint{3.840834in}{3.131650in}}%
\pgfpathlineto{\pgfqpoint{3.837662in}{3.131729in}}%
\pgfpathlineto{\pgfqpoint{3.834490in}{3.131725in}}%
\pgfpathlineto{\pgfqpoint{3.831317in}{3.132017in}}%
\pgfpathlineto{\pgfqpoint{3.828145in}{3.131977in}}%
\pgfpathlineto{\pgfqpoint{3.824973in}{3.132046in}}%
\pgfpathlineto{\pgfqpoint{3.821801in}{3.132106in}}%
\pgfpathlineto{\pgfqpoint{3.818629in}{3.132335in}}%
\pgfpathlineto{\pgfqpoint{3.815457in}{3.132065in}}%
\pgfpathlineto{\pgfqpoint{3.812285in}{3.131951in}}%
\pgfpathlineto{\pgfqpoint{3.809113in}{3.131994in}}%
\pgfpathlineto{\pgfqpoint{3.805941in}{3.131941in}}%
\pgfpathlineto{\pgfqpoint{3.802769in}{3.131821in}}%
\pgfpathlineto{\pgfqpoint{3.799597in}{3.131559in}}%
\pgfpathlineto{\pgfqpoint{3.796425in}{3.131482in}}%
\pgfpathlineto{\pgfqpoint{3.793253in}{3.131472in}}%
\pgfpathlineto{\pgfqpoint{3.790081in}{3.131469in}}%
\pgfpathlineto{\pgfqpoint{3.786909in}{3.131263in}}%
\pgfpathlineto{\pgfqpoint{3.783737in}{3.131240in}}%
\pgfpathlineto{\pgfqpoint{3.780565in}{3.131129in}}%
\pgfpathlineto{\pgfqpoint{3.777393in}{3.131167in}}%
\pgfpathlineto{\pgfqpoint{3.774221in}{3.131111in}}%
\pgfpathlineto{\pgfqpoint{3.771049in}{3.131013in}}%
\pgfpathlineto{\pgfqpoint{3.767877in}{3.130984in}}%
\pgfpathlineto{\pgfqpoint{3.764705in}{3.131215in}}%
\pgfpathlineto{\pgfqpoint{3.761533in}{3.131101in}}%
\pgfpathlineto{\pgfqpoint{3.758361in}{3.131384in}}%
\pgfpathlineto{\pgfqpoint{3.755188in}{3.131431in}}%
\pgfpathlineto{\pgfqpoint{3.752016in}{3.131381in}}%
\pgfpathlineto{\pgfqpoint{3.748844in}{3.131511in}}%
\pgfpathlineto{\pgfqpoint{3.745672in}{3.131919in}}%
\pgfpathlineto{\pgfqpoint{3.742500in}{3.131568in}}%
\pgfpathlineto{\pgfqpoint{3.739328in}{3.131502in}}%
\pgfpathlineto{\pgfqpoint{3.736156in}{3.131326in}}%
\pgfpathlineto{\pgfqpoint{3.732984in}{3.131292in}}%
\pgfpathlineto{\pgfqpoint{3.729812in}{3.131238in}}%
\pgfpathlineto{\pgfqpoint{3.726640in}{3.131485in}}%
\pgfpathlineto{\pgfqpoint{3.723468in}{3.131524in}}%
\pgfpathlineto{\pgfqpoint{3.720296in}{3.131712in}}%
\pgfpathlineto{\pgfqpoint{3.717124in}{3.131803in}}%
\pgfpathlineto{\pgfqpoint{3.713952in}{3.131836in}}%
\pgfpathlineto{\pgfqpoint{3.710780in}{3.131736in}}%
\pgfpathlineto{\pgfqpoint{3.707608in}{3.131620in}}%
\pgfpathlineto{\pgfqpoint{3.704436in}{3.131433in}}%
\pgfpathlineto{\pgfqpoint{3.701264in}{3.131516in}}%
\pgfpathlineto{\pgfqpoint{3.698092in}{3.131659in}}%
\pgfpathlineto{\pgfqpoint{3.694920in}{3.131733in}}%
\pgfpathlineto{\pgfqpoint{3.691748in}{3.131916in}}%
\pgfpathlineto{\pgfqpoint{3.688576in}{3.132400in}}%
\pgfpathlineto{\pgfqpoint{3.685404in}{3.132429in}}%
\pgfpathlineto{\pgfqpoint{3.682232in}{3.132333in}}%
\pgfpathlineto{\pgfqpoint{3.679060in}{3.132886in}}%
\pgfpathlineto{\pgfqpoint{3.675887in}{3.132994in}}%
\pgfpathlineto{\pgfqpoint{3.672715in}{3.132883in}}%
\pgfpathlineto{\pgfqpoint{3.669543in}{3.133187in}}%
\pgfpathlineto{\pgfqpoint{3.666371in}{3.134114in}}%
\pgfpathlineto{\pgfqpoint{3.663199in}{3.134190in}}%
\pgfpathlineto{\pgfqpoint{3.660027in}{3.134013in}}%
\pgfpathlineto{\pgfqpoint{3.656855in}{3.134055in}}%
\pgfpathlineto{\pgfqpoint{3.653683in}{3.134147in}}%
\pgfpathlineto{\pgfqpoint{3.650511in}{3.134195in}}%
\pgfpathlineto{\pgfqpoint{3.647339in}{3.134128in}}%
\pgfpathlineto{\pgfqpoint{3.644167in}{3.134330in}}%
\pgfpathlineto{\pgfqpoint{3.640995in}{3.133944in}}%
\pgfpathlineto{\pgfqpoint{3.637823in}{3.134122in}}%
\pgfpathlineto{\pgfqpoint{3.634651in}{3.134263in}}%
\pgfpathlineto{\pgfqpoint{3.631479in}{3.134545in}}%
\pgfpathlineto{\pgfqpoint{3.628307in}{3.134912in}}%
\pgfpathlineto{\pgfqpoint{3.625135in}{3.134948in}}%
\pgfpathlineto{\pgfqpoint{3.621963in}{3.134738in}}%
\pgfpathlineto{\pgfqpoint{3.618791in}{3.134892in}}%
\pgfpathlineto{\pgfqpoint{3.615619in}{3.134881in}}%
\pgfpathlineto{\pgfqpoint{3.612447in}{3.134892in}}%
\pgfpathlineto{\pgfqpoint{3.609275in}{3.134951in}}%
\pgfpathlineto{\pgfqpoint{3.606103in}{3.134774in}}%
\pgfpathlineto{\pgfqpoint{3.602931in}{3.134694in}}%
\pgfpathlineto{\pgfqpoint{3.599759in}{3.134640in}}%
\pgfpathlineto{\pgfqpoint{3.596586in}{3.134881in}}%
\pgfpathlineto{\pgfqpoint{3.593414in}{3.135114in}}%
\pgfpathlineto{\pgfqpoint{3.590242in}{3.134946in}}%
\pgfpathlineto{\pgfqpoint{3.587070in}{3.135004in}}%
\pgfpathlineto{\pgfqpoint{3.583898in}{3.135207in}}%
\pgfpathlineto{\pgfqpoint{3.580726in}{3.135249in}}%
\pgfpathlineto{\pgfqpoint{3.577554in}{3.135146in}}%
\pgfpathlineto{\pgfqpoint{3.574382in}{3.135291in}}%
\pgfpathlineto{\pgfqpoint{3.571210in}{3.135276in}}%
\pgfpathlineto{\pgfqpoint{3.568038in}{3.135627in}}%
\pgfpathlineto{\pgfqpoint{3.564866in}{3.135449in}}%
\pgfpathlineto{\pgfqpoint{3.561694in}{3.135721in}}%
\pgfpathlineto{\pgfqpoint{3.558522in}{3.135469in}}%
\pgfpathlineto{\pgfqpoint{3.555350in}{3.135075in}}%
\pgfpathlineto{\pgfqpoint{3.552178in}{3.134909in}}%
\pgfpathlineto{\pgfqpoint{3.549006in}{3.134897in}}%
\pgfpathlineto{\pgfqpoint{3.545834in}{3.134939in}}%
\pgfpathlineto{\pgfqpoint{3.542662in}{3.134815in}}%
\pgfpathlineto{\pgfqpoint{3.539490in}{3.135032in}}%
\pgfpathlineto{\pgfqpoint{3.536318in}{3.134808in}}%
\pgfpathlineto{\pgfqpoint{3.533146in}{3.134759in}}%
\pgfpathlineto{\pgfqpoint{3.529974in}{3.134645in}}%
\pgfpathlineto{\pgfqpoint{3.526802in}{3.134457in}}%
\pgfpathlineto{\pgfqpoint{3.523630in}{3.134435in}}%
\pgfpathlineto{\pgfqpoint{3.520457in}{3.134126in}}%
\pgfpathlineto{\pgfqpoint{3.517285in}{3.134489in}}%
\pgfpathlineto{\pgfqpoint{3.514113in}{3.134077in}}%
\pgfpathlineto{\pgfqpoint{3.510941in}{3.133922in}}%
\pgfpathlineto{\pgfqpoint{3.507769in}{3.133732in}}%
\pgfpathlineto{\pgfqpoint{3.504597in}{3.133659in}}%
\pgfpathlineto{\pgfqpoint{3.501425in}{3.133465in}}%
\pgfpathlineto{\pgfqpoint{3.498253in}{3.133539in}}%
\pgfpathlineto{\pgfqpoint{3.495081in}{3.133909in}}%
\pgfpathlineto{\pgfqpoint{3.491909in}{3.134035in}}%
\pgfpathlineto{\pgfqpoint{3.488737in}{3.133985in}}%
\pgfpathlineto{\pgfqpoint{3.485565in}{3.133457in}}%
\pgfpathlineto{\pgfqpoint{3.482393in}{3.133528in}}%
\pgfpathlineto{\pgfqpoint{3.479221in}{3.133509in}}%
\pgfpathlineto{\pgfqpoint{3.476049in}{3.133562in}}%
\pgfpathlineto{\pgfqpoint{3.472877in}{3.133204in}}%
\pgfpathlineto{\pgfqpoint{3.469705in}{3.133221in}}%
\pgfpathlineto{\pgfqpoint{3.466533in}{3.132820in}}%
\pgfpathlineto{\pgfqpoint{3.463361in}{3.133014in}}%
\pgfpathlineto{\pgfqpoint{3.460189in}{3.133299in}}%
\pgfpathlineto{\pgfqpoint{3.457017in}{3.133510in}}%
\pgfpathlineto{\pgfqpoint{3.453845in}{3.133555in}}%
\pgfpathlineto{\pgfqpoint{3.450673in}{3.133644in}}%
\pgfpathlineto{\pgfqpoint{3.447501in}{3.133625in}}%
\pgfpathlineto{\pgfqpoint{3.444329in}{3.133718in}}%
\pgfpathlineto{\pgfqpoint{3.441156in}{3.133317in}}%
\pgfpathlineto{\pgfqpoint{3.437984in}{3.133282in}}%
\pgfpathlineto{\pgfqpoint{3.434812in}{3.133306in}}%
\pgfpathlineto{\pgfqpoint{3.431640in}{3.133553in}}%
\pgfpathlineto{\pgfqpoint{3.428468in}{3.133547in}}%
\pgfpathlineto{\pgfqpoint{3.425296in}{3.133380in}}%
\pgfpathlineto{\pgfqpoint{3.422124in}{3.133043in}}%
\pgfpathlineto{\pgfqpoint{3.418952in}{3.132888in}}%
\pgfpathlineto{\pgfqpoint{3.415780in}{3.133030in}}%
\pgfpathlineto{\pgfqpoint{3.412608in}{3.133217in}}%
\pgfpathlineto{\pgfqpoint{3.409436in}{3.133227in}}%
\pgfpathlineto{\pgfqpoint{3.406264in}{3.133156in}}%
\pgfpathlineto{\pgfqpoint{3.403092in}{3.133270in}}%
\pgfpathlineto{\pgfqpoint{3.399920in}{3.133333in}}%
\pgfpathlineto{\pgfqpoint{3.396748in}{3.133159in}}%
\pgfpathlineto{\pgfqpoint{3.393576in}{3.133651in}}%
\pgfpathlineto{\pgfqpoint{3.390404in}{3.133564in}}%
\pgfpathlineto{\pgfqpoint{3.387232in}{3.133264in}}%
\pgfpathlineto{\pgfqpoint{3.384060in}{3.133210in}}%
\pgfpathlineto{\pgfqpoint{3.380888in}{3.133183in}}%
\pgfpathlineto{\pgfqpoint{3.377716in}{3.133088in}}%
\pgfpathlineto{\pgfqpoint{3.374544in}{3.133022in}}%
\pgfpathlineto{\pgfqpoint{3.371372in}{3.132714in}}%
\pgfpathlineto{\pgfqpoint{3.368200in}{3.132513in}}%
\pgfpathlineto{\pgfqpoint{3.365028in}{3.131666in}}%
\pgfpathlineto{\pgfqpoint{3.361855in}{3.131562in}}%
\pgfpathlineto{\pgfqpoint{3.358683in}{3.131571in}}%
\pgfpathlineto{\pgfqpoint{3.355511in}{3.131388in}}%
\pgfpathlineto{\pgfqpoint{3.352339in}{3.131398in}}%
\pgfpathlineto{\pgfqpoint{3.349167in}{3.131245in}}%
\pgfpathlineto{\pgfqpoint{3.345995in}{3.131311in}}%
\pgfpathlineto{\pgfqpoint{3.342823in}{3.131359in}}%
\pgfpathlineto{\pgfqpoint{3.339651in}{3.131473in}}%
\pgfpathlineto{\pgfqpoint{3.336479in}{3.131462in}}%
\pgfpathlineto{\pgfqpoint{3.333307in}{3.131686in}}%
\pgfpathlineto{\pgfqpoint{3.330135in}{3.131624in}}%
\pgfpathlineto{\pgfqpoint{3.326963in}{3.131389in}}%
\pgfpathlineto{\pgfqpoint{3.323791in}{3.131054in}}%
\pgfpathlineto{\pgfqpoint{3.320619in}{3.130940in}}%
\pgfpathlineto{\pgfqpoint{3.317447in}{3.131045in}}%
\pgfpathlineto{\pgfqpoint{3.314275in}{3.131316in}}%
\pgfpathlineto{\pgfqpoint{3.311103in}{3.131647in}}%
\pgfpathlineto{\pgfqpoint{3.307931in}{3.131444in}}%
\pgfpathlineto{\pgfqpoint{3.304759in}{3.131139in}}%
\pgfpathlineto{\pgfqpoint{3.301587in}{3.130600in}}%
\pgfpathlineto{\pgfqpoint{3.298415in}{3.130346in}}%
\pgfpathlineto{\pgfqpoint{3.295243in}{3.130735in}}%
\pgfpathlineto{\pgfqpoint{3.292071in}{3.130356in}}%
\pgfpathlineto{\pgfqpoint{3.288899in}{3.130148in}}%
\pgfpathlineto{\pgfqpoint{3.285726in}{3.130154in}}%
\pgfpathlineto{\pgfqpoint{3.282554in}{3.130284in}}%
\pgfpathlineto{\pgfqpoint{3.279382in}{3.130818in}}%
\pgfpathlineto{\pgfqpoint{3.276210in}{3.130807in}}%
\pgfpathlineto{\pgfqpoint{3.273038in}{3.130902in}}%
\pgfpathlineto{\pgfqpoint{3.269866in}{3.130882in}}%
\pgfpathlineto{\pgfqpoint{3.266694in}{3.131092in}}%
\pgfpathlineto{\pgfqpoint{3.263522in}{3.130954in}}%
\pgfpathlineto{\pgfqpoint{3.260350in}{3.131223in}}%
\pgfpathlineto{\pgfqpoint{3.257178in}{3.131212in}}%
\pgfpathlineto{\pgfqpoint{3.254006in}{3.131315in}}%
\pgfpathlineto{\pgfqpoint{3.250834in}{3.131947in}}%
\pgfpathlineto{\pgfqpoint{3.247662in}{3.132136in}}%
\pgfpathlineto{\pgfqpoint{3.244490in}{3.132207in}}%
\pgfpathlineto{\pgfqpoint{3.241318in}{3.132801in}}%
\pgfpathlineto{\pgfqpoint{3.238146in}{3.132694in}}%
\pgfpathlineto{\pgfqpoint{3.234974in}{3.132925in}}%
\pgfpathlineto{\pgfqpoint{3.231802in}{3.133116in}}%
\pgfpathlineto{\pgfqpoint{3.228630in}{3.132613in}}%
\pgfpathlineto{\pgfqpoint{3.225458in}{3.132646in}}%
\pgfpathlineto{\pgfqpoint{3.222286in}{3.132879in}}%
\pgfpathlineto{\pgfqpoint{3.219114in}{3.133224in}}%
\pgfpathlineto{\pgfqpoint{3.215942in}{3.133239in}}%
\pgfpathlineto{\pgfqpoint{3.212770in}{3.133242in}}%
\pgfpathlineto{\pgfqpoint{3.209598in}{3.133238in}}%
\pgfpathlineto{\pgfqpoint{3.206425in}{3.133961in}}%
\pgfpathlineto{\pgfqpoint{3.203253in}{3.134691in}}%
\pgfpathlineto{\pgfqpoint{3.200081in}{3.134511in}}%
\pgfpathlineto{\pgfqpoint{3.196909in}{3.134995in}}%
\pgfpathlineto{\pgfqpoint{3.193737in}{3.135053in}}%
\pgfpathlineto{\pgfqpoint{3.190565in}{3.135054in}}%
\pgfpathlineto{\pgfqpoint{3.187393in}{3.135135in}}%
\pgfpathlineto{\pgfqpoint{3.184221in}{3.134730in}}%
\pgfpathlineto{\pgfqpoint{3.181049in}{3.134800in}}%
\pgfpathlineto{\pgfqpoint{3.177877in}{3.134693in}}%
\pgfpathlineto{\pgfqpoint{3.174705in}{3.134789in}}%
\pgfpathlineto{\pgfqpoint{3.171533in}{3.134435in}}%
\pgfpathlineto{\pgfqpoint{3.168361in}{3.134314in}}%
\pgfpathlineto{\pgfqpoint{3.165189in}{3.133913in}}%
\pgfpathlineto{\pgfqpoint{3.162017in}{3.133954in}}%
\pgfpathlineto{\pgfqpoint{3.158845in}{3.133971in}}%
\pgfpathlineto{\pgfqpoint{3.155673in}{3.134029in}}%
\pgfpathlineto{\pgfqpoint{3.152501in}{3.133869in}}%
\pgfpathlineto{\pgfqpoint{3.149329in}{3.133766in}}%
\pgfpathlineto{\pgfqpoint{3.146157in}{3.133877in}}%
\pgfpathlineto{\pgfqpoint{3.142985in}{3.134285in}}%
\pgfpathlineto{\pgfqpoint{3.139813in}{3.134130in}}%
\pgfpathlineto{\pgfqpoint{3.136641in}{3.133949in}}%
\pgfpathlineto{\pgfqpoint{3.133469in}{3.134060in}}%
\pgfpathlineto{\pgfqpoint{3.130297in}{3.134044in}}%
\pgfpathlineto{\pgfqpoint{3.127124in}{3.133858in}}%
\pgfpathlineto{\pgfqpoint{3.123952in}{3.133844in}}%
\pgfpathlineto{\pgfqpoint{3.120780in}{3.133690in}}%
\pgfpathlineto{\pgfqpoint{3.117608in}{3.133335in}}%
\pgfpathlineto{\pgfqpoint{3.114436in}{3.133238in}}%
\pgfpathlineto{\pgfqpoint{3.111264in}{3.133464in}}%
\pgfpathlineto{\pgfqpoint{3.108092in}{3.133471in}}%
\pgfpathlineto{\pgfqpoint{3.104920in}{3.133701in}}%
\pgfpathlineto{\pgfqpoint{3.101748in}{3.133609in}}%
\pgfpathlineto{\pgfqpoint{3.098576in}{3.133587in}}%
\pgfpathlineto{\pgfqpoint{3.095404in}{3.133567in}}%
\pgfpathlineto{\pgfqpoint{3.092232in}{3.133761in}}%
\pgfpathlineto{\pgfqpoint{3.089060in}{3.133578in}}%
\pgfpathlineto{\pgfqpoint{3.085888in}{3.133565in}}%
\pgfpathlineto{\pgfqpoint{3.082716in}{3.133510in}}%
\pgfpathlineto{\pgfqpoint{3.079544in}{3.133393in}}%
\pgfpathlineto{\pgfqpoint{3.076372in}{3.133258in}}%
\pgfpathlineto{\pgfqpoint{3.073200in}{3.133432in}}%
\pgfpathlineto{\pgfqpoint{3.070028in}{3.133303in}}%
\pgfpathlineto{\pgfqpoint{3.066856in}{3.133096in}}%
\pgfpathlineto{\pgfqpoint{3.063684in}{3.133085in}}%
\pgfpathlineto{\pgfqpoint{3.060512in}{3.133177in}}%
\pgfpathlineto{\pgfqpoint{3.057340in}{3.132941in}}%
\pgfpathlineto{\pgfqpoint{3.054168in}{3.132847in}}%
\pgfpathlineto{\pgfqpoint{3.050995in}{3.132822in}}%
\pgfpathlineto{\pgfqpoint{3.047823in}{3.132968in}}%
\pgfpathlineto{\pgfqpoint{3.044651in}{3.133071in}}%
\pgfpathlineto{\pgfqpoint{3.041479in}{3.132996in}}%
\pgfpathlineto{\pgfqpoint{3.038307in}{3.133252in}}%
\pgfpathlineto{\pgfqpoint{3.035135in}{3.132984in}}%
\pgfpathlineto{\pgfqpoint{3.031963in}{3.132987in}}%
\pgfpathlineto{\pgfqpoint{3.028791in}{3.132863in}}%
\pgfpathlineto{\pgfqpoint{3.025619in}{3.132686in}}%
\pgfpathlineto{\pgfqpoint{3.022447in}{3.132748in}}%
\pgfpathlineto{\pgfqpoint{3.019275in}{3.133010in}}%
\pgfpathlineto{\pgfqpoint{3.016103in}{3.133177in}}%
\pgfpathlineto{\pgfqpoint{3.012931in}{3.133048in}}%
\pgfpathlineto{\pgfqpoint{3.009759in}{3.133118in}}%
\pgfpathlineto{\pgfqpoint{3.006587in}{3.133030in}}%
\pgfpathlineto{\pgfqpoint{3.003415in}{3.133139in}}%
\pgfpathlineto{\pgfqpoint{3.000243in}{3.133065in}}%
\pgfpathlineto{\pgfqpoint{2.997071in}{3.132951in}}%
\pgfpathlineto{\pgfqpoint{2.993899in}{3.133070in}}%
\pgfpathlineto{\pgfqpoint{2.990727in}{3.133177in}}%
\pgfpathlineto{\pgfqpoint{2.987555in}{3.133110in}}%
\pgfpathlineto{\pgfqpoint{2.984383in}{3.133045in}}%
\pgfpathlineto{\pgfqpoint{2.981211in}{3.132922in}}%
\pgfpathlineto{\pgfqpoint{2.978039in}{3.132693in}}%
\pgfpathlineto{\pgfqpoint{2.974867in}{3.132867in}}%
\pgfpathlineto{\pgfqpoint{2.971694in}{3.132816in}}%
\pgfpathlineto{\pgfqpoint{2.968522in}{3.132692in}}%
\pgfpathlineto{\pgfqpoint{2.965350in}{3.132583in}}%
\pgfpathlineto{\pgfqpoint{2.962178in}{3.132780in}}%
\pgfpathlineto{\pgfqpoint{2.959006in}{3.132802in}}%
\pgfpathlineto{\pgfqpoint{2.955834in}{3.131110in}}%
\pgfpathlineto{\pgfqpoint{2.952662in}{3.130886in}}%
\pgfpathlineto{\pgfqpoint{2.949490in}{3.130775in}}%
\pgfpathlineto{\pgfqpoint{2.946318in}{3.130810in}}%
\pgfpathlineto{\pgfqpoint{2.943146in}{3.130851in}}%
\pgfpathlineto{\pgfqpoint{2.939974in}{3.130802in}}%
\pgfpathlineto{\pgfqpoint{2.936802in}{3.130902in}}%
\pgfpathlineto{\pgfqpoint{2.933630in}{3.131034in}}%
\pgfpathlineto{\pgfqpoint{2.930458in}{3.131076in}}%
\pgfpathlineto{\pgfqpoint{2.927286in}{3.131087in}}%
\pgfpathlineto{\pgfqpoint{2.924114in}{3.130965in}}%
\pgfpathlineto{\pgfqpoint{2.920942in}{3.130834in}}%
\pgfpathlineto{\pgfqpoint{2.917770in}{3.130915in}}%
\pgfpathlineto{\pgfqpoint{2.914598in}{3.130868in}}%
\pgfpathlineto{\pgfqpoint{2.911426in}{3.130743in}}%
\pgfpathlineto{\pgfqpoint{2.908254in}{3.130605in}}%
\pgfpathlineto{\pgfqpoint{2.905082in}{3.130615in}}%
\pgfpathlineto{\pgfqpoint{2.901910in}{3.130551in}}%
\pgfpathlineto{\pgfqpoint{2.898738in}{3.130540in}}%
\pgfpathlineto{\pgfqpoint{2.895565in}{3.130452in}}%
\pgfpathlineto{\pgfqpoint{2.892393in}{3.130365in}}%
\pgfpathlineto{\pgfqpoint{2.889221in}{3.130191in}}%
\pgfpathlineto{\pgfqpoint{2.886049in}{3.129993in}}%
\pgfpathlineto{\pgfqpoint{2.882877in}{3.130124in}}%
\pgfpathlineto{\pgfqpoint{2.879705in}{3.130277in}}%
\pgfpathlineto{\pgfqpoint{2.876533in}{3.130140in}}%
\pgfpathlineto{\pgfqpoint{2.873361in}{3.129969in}}%
\pgfpathlineto{\pgfqpoint{2.870189in}{3.130105in}}%
\pgfpathlineto{\pgfqpoint{2.867017in}{3.129986in}}%
\pgfpathlineto{\pgfqpoint{2.863845in}{3.130100in}}%
\pgfpathlineto{\pgfqpoint{2.860673in}{3.130070in}}%
\pgfpathlineto{\pgfqpoint{2.857501in}{3.130042in}}%
\pgfpathlineto{\pgfqpoint{2.854329in}{3.129986in}}%
\pgfpathlineto{\pgfqpoint{2.851157in}{3.130021in}}%
\pgfpathlineto{\pgfqpoint{2.847985in}{3.130043in}}%
\pgfpathlineto{\pgfqpoint{2.844813in}{3.129947in}}%
\pgfpathlineto{\pgfqpoint{2.841641in}{3.129953in}}%
\pgfpathlineto{\pgfqpoint{2.838469in}{3.129983in}}%
\pgfpathlineto{\pgfqpoint{2.835297in}{3.129907in}}%
\pgfpathlineto{\pgfqpoint{2.832125in}{3.129709in}}%
\pgfpathlineto{\pgfqpoint{2.828953in}{3.129622in}}%
\pgfpathlineto{\pgfqpoint{2.825781in}{3.129708in}}%
\pgfpathlineto{\pgfqpoint{2.822609in}{3.129737in}}%
\pgfpathlineto{\pgfqpoint{2.819437in}{3.129774in}}%
\pgfpathlineto{\pgfqpoint{2.816264in}{3.129670in}}%
\pgfpathlineto{\pgfqpoint{2.813092in}{3.129689in}}%
\pgfpathlineto{\pgfqpoint{2.809920in}{3.129667in}}%
\pgfpathlineto{\pgfqpoint{2.806748in}{3.129836in}}%
\pgfpathlineto{\pgfqpoint{2.803576in}{3.129787in}}%
\pgfpathlineto{\pgfqpoint{2.800404in}{3.129727in}}%
\pgfpathlineto{\pgfqpoint{2.797232in}{3.129767in}}%
\pgfpathlineto{\pgfqpoint{2.794060in}{3.129729in}}%
\pgfpathlineto{\pgfqpoint{2.790888in}{3.129679in}}%
\pgfpathlineto{\pgfqpoint{2.787716in}{3.129790in}}%
\pgfpathlineto{\pgfqpoint{2.784544in}{3.129900in}}%
\pgfpathlineto{\pgfqpoint{2.781372in}{3.129851in}}%
\pgfpathlineto{\pgfqpoint{2.778200in}{3.130006in}}%
\pgfpathlineto{\pgfqpoint{2.775028in}{3.130115in}}%
\pgfpathlineto{\pgfqpoint{2.771856in}{3.130086in}}%
\pgfpathlineto{\pgfqpoint{2.768684in}{3.130065in}}%
\pgfpathlineto{\pgfqpoint{2.765512in}{3.129965in}}%
\pgfpathlineto{\pgfqpoint{2.762340in}{3.129914in}}%
\pgfpathlineto{\pgfqpoint{2.759168in}{3.129684in}}%
\pgfpathlineto{\pgfqpoint{2.755996in}{3.129697in}}%
\pgfpathlineto{\pgfqpoint{2.752824in}{3.129694in}}%
\pgfpathlineto{\pgfqpoint{2.749652in}{3.129538in}}%
\pgfpathlineto{\pgfqpoint{2.746480in}{3.129598in}}%
\pgfpathlineto{\pgfqpoint{2.743308in}{3.129539in}}%
\pgfpathlineto{\pgfqpoint{2.740136in}{3.129540in}}%
\pgfpathlineto{\pgfqpoint{2.736963in}{3.129584in}}%
\pgfpathlineto{\pgfqpoint{2.733791in}{3.129555in}}%
\pgfpathlineto{\pgfqpoint{2.730619in}{3.129609in}}%
\pgfpathlineto{\pgfqpoint{2.727447in}{3.129686in}}%
\pgfpathlineto{\pgfqpoint{2.724275in}{3.129664in}}%
\pgfpathlineto{\pgfqpoint{2.721103in}{3.129573in}}%
\pgfpathlineto{\pgfqpoint{2.717931in}{3.129722in}}%
\pgfpathlineto{\pgfqpoint{2.714759in}{3.129899in}}%
\pgfpathlineto{\pgfqpoint{2.711587in}{3.129677in}}%
\pgfpathlineto{\pgfqpoint{2.708415in}{3.129633in}}%
\pgfpathlineto{\pgfqpoint{2.705243in}{3.129625in}}%
\pgfpathlineto{\pgfqpoint{2.702071in}{3.129905in}}%
\pgfpathlineto{\pgfqpoint{2.698899in}{3.129892in}}%
\pgfpathlineto{\pgfqpoint{2.695727in}{3.129983in}}%
\pgfpathlineto{\pgfqpoint{2.692555in}{3.129960in}}%
\pgfpathlineto{\pgfqpoint{2.689383in}{3.129875in}}%
\pgfpathlineto{\pgfqpoint{2.686211in}{3.129737in}}%
\pgfpathlineto{\pgfqpoint{2.683039in}{3.129705in}}%
\pgfpathlineto{\pgfqpoint{2.679867in}{3.129773in}}%
\pgfpathlineto{\pgfqpoint{2.676695in}{3.129689in}}%
\pgfpathlineto{\pgfqpoint{2.673523in}{3.129727in}}%
\pgfpathlineto{\pgfqpoint{2.670351in}{3.129796in}}%
\pgfpathlineto{\pgfqpoint{2.667179in}{3.129608in}}%
\pgfpathlineto{\pgfqpoint{2.664007in}{3.129612in}}%
\pgfpathlineto{\pgfqpoint{2.660834in}{3.129643in}}%
\pgfpathlineto{\pgfqpoint{2.657662in}{3.129625in}}%
\pgfpathlineto{\pgfqpoint{2.654490in}{3.129560in}}%
\pgfpathlineto{\pgfqpoint{2.651318in}{3.129287in}}%
\pgfpathlineto{\pgfqpoint{2.648146in}{3.129265in}}%
\pgfpathlineto{\pgfqpoint{2.644974in}{3.129373in}}%
\pgfpathlineto{\pgfqpoint{2.641802in}{3.129390in}}%
\pgfpathlineto{\pgfqpoint{2.638630in}{3.129566in}}%
\pgfpathlineto{\pgfqpoint{2.635458in}{3.129115in}}%
\pgfpathlineto{\pgfqpoint{2.632286in}{3.129077in}}%
\pgfpathlineto{\pgfqpoint{2.629114in}{3.129238in}}%
\pgfpathlineto{\pgfqpoint{2.625942in}{3.129209in}}%
\pgfpathlineto{\pgfqpoint{2.622770in}{3.129264in}}%
\pgfpathlineto{\pgfqpoint{2.619598in}{3.129233in}}%
\pgfpathlineto{\pgfqpoint{2.616426in}{3.129310in}}%
\pgfpathlineto{\pgfqpoint{2.613254in}{3.129352in}}%
\pgfpathlineto{\pgfqpoint{2.610082in}{3.129334in}}%
\pgfpathlineto{\pgfqpoint{2.606910in}{3.129331in}}%
\pgfpathlineto{\pgfqpoint{2.603738in}{3.129511in}}%
\pgfpathlineto{\pgfqpoint{2.600566in}{3.129486in}}%
\pgfpathlineto{\pgfqpoint{2.597394in}{3.129440in}}%
\pgfpathlineto{\pgfqpoint{2.594222in}{3.129070in}}%
\pgfpathlineto{\pgfqpoint{2.591050in}{3.129213in}}%
\pgfpathlineto{\pgfqpoint{2.587878in}{3.129315in}}%
\pgfpathlineto{\pgfqpoint{2.584706in}{3.129137in}}%
\pgfpathlineto{\pgfqpoint{2.581533in}{3.129200in}}%
\pgfpathlineto{\pgfqpoint{2.578361in}{3.129171in}}%
\pgfpathlineto{\pgfqpoint{2.575189in}{3.128947in}}%
\pgfpathlineto{\pgfqpoint{2.572017in}{3.128878in}}%
\pgfpathlineto{\pgfqpoint{2.568845in}{3.128791in}}%
\pgfpathlineto{\pgfqpoint{2.565673in}{3.128589in}}%
\pgfpathlineto{\pgfqpoint{2.562501in}{3.128480in}}%
\pgfpathlineto{\pgfqpoint{2.559329in}{3.128450in}}%
\pgfpathlineto{\pgfqpoint{2.556157in}{3.128564in}}%
\pgfpathlineto{\pgfqpoint{2.552985in}{3.128717in}}%
\pgfpathlineto{\pgfqpoint{2.549813in}{3.128968in}}%
\pgfpathlineto{\pgfqpoint{2.546641in}{3.128902in}}%
\pgfpathlineto{\pgfqpoint{2.543469in}{3.129006in}}%
\pgfpathlineto{\pgfqpoint{2.540297in}{3.128925in}}%
\pgfpathlineto{\pgfqpoint{2.537125in}{3.128770in}}%
\pgfpathlineto{\pgfqpoint{2.533953in}{3.128807in}}%
\pgfpathlineto{\pgfqpoint{2.530781in}{3.128736in}}%
\pgfpathlineto{\pgfqpoint{2.527609in}{3.128770in}}%
\pgfpathlineto{\pgfqpoint{2.524437in}{3.128805in}}%
\pgfpathlineto{\pgfqpoint{2.521265in}{3.128898in}}%
\pgfpathlineto{\pgfqpoint{2.518093in}{3.128569in}}%
\pgfpathlineto{\pgfqpoint{2.514921in}{3.128711in}}%
\pgfpathlineto{\pgfqpoint{2.511749in}{3.128750in}}%
\pgfpathlineto{\pgfqpoint{2.508577in}{3.128676in}}%
\pgfpathlineto{\pgfqpoint{2.505405in}{3.128719in}}%
\pgfpathlineto{\pgfqpoint{2.502232in}{3.128604in}}%
\pgfpathlineto{\pgfqpoint{2.499060in}{3.128639in}}%
\pgfpathlineto{\pgfqpoint{2.495888in}{3.128636in}}%
\pgfpathlineto{\pgfqpoint{2.492716in}{3.128515in}}%
\pgfpathlineto{\pgfqpoint{2.489544in}{3.128422in}}%
\pgfpathlineto{\pgfqpoint{2.486372in}{3.128296in}}%
\pgfpathlineto{\pgfqpoint{2.483200in}{3.128286in}}%
\pgfpathlineto{\pgfqpoint{2.480028in}{3.128279in}}%
\pgfpathlineto{\pgfqpoint{2.476856in}{3.128142in}}%
\pgfpathlineto{\pgfqpoint{2.473684in}{3.128073in}}%
\pgfpathlineto{\pgfqpoint{2.470512in}{3.128076in}}%
\pgfpathlineto{\pgfqpoint{2.467340in}{3.128088in}}%
\pgfpathlineto{\pgfqpoint{2.464168in}{3.128031in}}%
\pgfpathlineto{\pgfqpoint{2.460996in}{3.128204in}}%
\pgfpathlineto{\pgfqpoint{2.457824in}{3.117443in}}%
\pgfpathlineto{\pgfqpoint{2.454652in}{3.106961in}}%
\pgfpathlineto{\pgfqpoint{2.451480in}{3.097150in}}%
\pgfpathlineto{\pgfqpoint{2.448308in}{3.091871in}}%
\pgfpathlineto{\pgfqpoint{2.445136in}{3.079781in}}%
\pgfpathlineto{\pgfqpoint{2.441964in}{3.067810in}}%
\pgfpathlineto{\pgfqpoint{2.438792in}{3.056231in}}%
\pgfpathlineto{\pgfqpoint{2.435620in}{3.044245in}}%
\pgfpathlineto{\pgfqpoint{2.432448in}{3.032030in}}%
\pgfpathlineto{\pgfqpoint{2.429276in}{3.020388in}}%
\pgfpathlineto{\pgfqpoint{2.426103in}{3.008354in}}%
\pgfpathlineto{\pgfqpoint{2.422931in}{2.996394in}}%
\pgfpathlineto{\pgfqpoint{2.419759in}{2.984819in}}%
\pgfpathlineto{\pgfqpoint{2.416587in}{2.972647in}}%
\pgfpathlineto{\pgfqpoint{2.413415in}{2.960587in}}%
\pgfpathlineto{\pgfqpoint{2.410243in}{2.948699in}}%
\pgfpathlineto{\pgfqpoint{2.407071in}{2.937280in}}%
\pgfpathlineto{\pgfqpoint{2.403899in}{2.925438in}}%
\pgfpathlineto{\pgfqpoint{2.400727in}{2.913419in}}%
\pgfpathlineto{\pgfqpoint{2.397555in}{2.902227in}}%
\pgfpathlineto{\pgfqpoint{2.394383in}{2.890454in}}%
\pgfpathlineto{\pgfqpoint{2.391211in}{2.879007in}}%
\pgfpathlineto{\pgfqpoint{2.388039in}{2.867644in}}%
\pgfpathlineto{\pgfqpoint{2.384867in}{2.856195in}}%
\pgfpathlineto{\pgfqpoint{2.381695in}{2.844272in}}%
\pgfpathlineto{\pgfqpoint{2.378523in}{2.832260in}}%
\pgfpathlineto{\pgfqpoint{2.375351in}{2.820266in}}%
\pgfpathlineto{\pgfqpoint{2.372179in}{2.808534in}}%
\pgfpathlineto{\pgfqpoint{2.369007in}{2.797052in}}%
\pgfpathlineto{\pgfqpoint{2.365835in}{2.785684in}}%
\pgfpathlineto{\pgfqpoint{2.362663in}{2.773485in}}%
\pgfpathlineto{\pgfqpoint{2.359491in}{2.761434in}}%
\pgfpathlineto{\pgfqpoint{2.356319in}{2.748909in}}%
\pgfpathlineto{\pgfqpoint{2.353147in}{2.735772in}}%
\pgfpathlineto{\pgfqpoint{2.349975in}{2.723831in}}%
\pgfpathlineto{\pgfqpoint{2.346802in}{2.712602in}}%
\pgfpathlineto{\pgfqpoint{2.343630in}{2.701148in}}%
\pgfpathlineto{\pgfqpoint{2.340458in}{2.689305in}}%
\pgfpathlineto{\pgfqpoint{2.337286in}{2.677131in}}%
\pgfpathlineto{\pgfqpoint{2.334114in}{2.665723in}}%
\pgfpathlineto{\pgfqpoint{2.330942in}{2.653659in}}%
\pgfpathlineto{\pgfqpoint{2.327770in}{2.641480in}}%
\pgfpathlineto{\pgfqpoint{2.324598in}{2.629483in}}%
\pgfpathlineto{\pgfqpoint{2.321426in}{2.617720in}}%
\pgfpathlineto{\pgfqpoint{2.318254in}{2.606018in}}%
\pgfpathlineto{\pgfqpoint{2.315082in}{2.594056in}}%
\pgfpathlineto{\pgfqpoint{2.311910in}{2.582698in}}%
\pgfpathlineto{\pgfqpoint{2.308738in}{2.570807in}}%
\pgfpathlineto{\pgfqpoint{2.305566in}{2.558871in}}%
\pgfpathlineto{\pgfqpoint{2.302394in}{2.547759in}}%
\pgfpathlineto{\pgfqpoint{2.299222in}{2.536074in}}%
\pgfpathlineto{\pgfqpoint{2.296050in}{2.524182in}}%
\pgfpathlineto{\pgfqpoint{2.292878in}{2.511990in}}%
\pgfpathlineto{\pgfqpoint{2.289706in}{2.500116in}}%
\pgfpathlineto{\pgfqpoint{2.286534in}{2.488261in}}%
\pgfpathlineto{\pgfqpoint{2.283362in}{2.476115in}}%
\pgfpathlineto{\pgfqpoint{2.280190in}{2.464291in}}%
\pgfpathlineto{\pgfqpoint{2.277018in}{2.452608in}}%
\pgfpathlineto{\pgfqpoint{2.273846in}{2.440405in}}%
\pgfpathlineto{\pgfqpoint{2.270674in}{2.428675in}}%
\pgfpathlineto{\pgfqpoint{2.267501in}{2.416911in}}%
\pgfpathlineto{\pgfqpoint{2.264329in}{2.405237in}}%
\pgfpathlineto{\pgfqpoint{2.261157in}{2.393390in}}%
\pgfpathlineto{\pgfqpoint{2.257985in}{2.381871in}}%
\pgfpathlineto{\pgfqpoint{2.254813in}{2.370259in}}%
\pgfpathlineto{\pgfqpoint{2.251641in}{2.358079in}}%
\pgfpathlineto{\pgfqpoint{2.248469in}{2.346304in}}%
\pgfpathlineto{\pgfqpoint{2.245297in}{2.334261in}}%
\pgfpathlineto{\pgfqpoint{2.242125in}{2.322757in}}%
\pgfpathlineto{\pgfqpoint{2.238953in}{2.311176in}}%
\pgfpathlineto{\pgfqpoint{2.235781in}{2.298977in}}%
\pgfpathlineto{\pgfqpoint{2.232609in}{2.286853in}}%
\pgfpathlineto{\pgfqpoint{2.229437in}{2.275559in}}%
\pgfpathlineto{\pgfqpoint{2.226265in}{2.263459in}}%
\pgfpathlineto{\pgfqpoint{2.223093in}{2.251556in}}%
\pgfpathlineto{\pgfqpoint{2.219921in}{2.239493in}}%
\pgfpathlineto{\pgfqpoint{2.216749in}{2.227737in}}%
\pgfpathlineto{\pgfqpoint{2.213577in}{2.215975in}}%
\pgfpathlineto{\pgfqpoint{2.210405in}{2.204002in}}%
\pgfpathlineto{\pgfqpoint{2.207233in}{2.192187in}}%
\pgfpathlineto{\pgfqpoint{2.204061in}{2.180539in}}%
\pgfpathlineto{\pgfqpoint{2.200889in}{2.168323in}}%
\pgfpathlineto{\pgfqpoint{2.197717in}{2.156715in}}%
\pgfpathlineto{\pgfqpoint{2.194545in}{2.145007in}}%
\pgfpathlineto{\pgfqpoint{2.191372in}{2.133346in}}%
\pgfpathlineto{\pgfqpoint{2.188200in}{2.121876in}}%
\pgfpathlineto{\pgfqpoint{2.185028in}{2.109889in}}%
\pgfpathlineto{\pgfqpoint{2.181856in}{2.098180in}}%
\pgfpathlineto{\pgfqpoint{2.178684in}{2.086255in}}%
\pgfpathlineto{\pgfqpoint{2.175512in}{2.074400in}}%
\pgfpathlineto{\pgfqpoint{2.172340in}{2.062315in}}%
\pgfpathlineto{\pgfqpoint{2.169168in}{2.050393in}}%
\pgfpathlineto{\pgfqpoint{2.165996in}{2.038922in}}%
\pgfpathlineto{\pgfqpoint{2.162824in}{2.027375in}}%
\pgfpathlineto{\pgfqpoint{2.159652in}{2.015183in}}%
\pgfpathlineto{\pgfqpoint{2.156480in}{2.003460in}}%
\pgfpathlineto{\pgfqpoint{2.153308in}{1.991705in}}%
\pgfpathlineto{\pgfqpoint{2.150136in}{1.980265in}}%
\pgfpathlineto{\pgfqpoint{2.146964in}{1.968537in}}%
\pgfpathlineto{\pgfqpoint{2.143792in}{1.956396in}}%
\pgfpathlineto{\pgfqpoint{2.140620in}{1.944558in}}%
\pgfpathlineto{\pgfqpoint{2.137448in}{1.933092in}}%
\pgfpathlineto{\pgfqpoint{2.134276in}{1.921301in}}%
\pgfpathlineto{\pgfqpoint{2.131104in}{1.909905in}}%
\pgfpathlineto{\pgfqpoint{2.127932in}{1.898444in}}%
\pgfpathlineto{\pgfqpoint{2.124760in}{1.886932in}}%
\pgfpathlineto{\pgfqpoint{2.121588in}{1.875393in}}%
\pgfpathlineto{\pgfqpoint{2.118416in}{1.863637in}}%
\pgfpathlineto{\pgfqpoint{2.115244in}{1.851374in}}%
\pgfpathlineto{\pgfqpoint{2.112071in}{1.839888in}}%
\pgfpathlineto{\pgfqpoint{2.108899in}{1.828565in}}%
\pgfpathlineto{\pgfqpoint{2.105727in}{1.816732in}}%
\pgfpathlineto{\pgfqpoint{2.102555in}{1.805076in}}%
\pgfpathlineto{\pgfqpoint{2.099383in}{1.792748in}}%
\pgfpathlineto{\pgfqpoint{2.096211in}{1.781187in}}%
\pgfpathlineto{\pgfqpoint{2.093039in}{1.769563in}}%
\pgfpathlineto{\pgfqpoint{2.089867in}{1.757681in}}%
\pgfpathlineto{\pgfqpoint{2.086695in}{1.746014in}}%
\pgfpathlineto{\pgfqpoint{2.083523in}{1.734094in}}%
\pgfpathlineto{\pgfqpoint{2.080351in}{1.722403in}}%
\pgfpathlineto{\pgfqpoint{2.077179in}{1.710854in}}%
\pgfpathlineto{\pgfqpoint{2.074007in}{1.699369in}}%
\pgfpathlineto{\pgfqpoint{2.070835in}{1.687596in}}%
\pgfpathlineto{\pgfqpoint{2.067663in}{1.675748in}}%
\pgfpathlineto{\pgfqpoint{2.064491in}{1.664179in}}%
\pgfpathlineto{\pgfqpoint{2.061319in}{1.652629in}}%
\pgfpathlineto{\pgfqpoint{2.058147in}{1.641049in}}%
\pgfpathlineto{\pgfqpoint{2.054975in}{1.629302in}}%
\pgfpathlineto{\pgfqpoint{2.051803in}{1.617666in}}%
\pgfpathlineto{\pgfqpoint{2.048631in}{1.606121in}}%
\pgfpathlineto{\pgfqpoint{2.045459in}{1.595210in}}%
\pgfpathlineto{\pgfqpoint{2.042287in}{1.584402in}}%
\pgfpathlineto{\pgfqpoint{2.039115in}{1.575357in}}%
\pgfpathlineto{\pgfqpoint{2.035943in}{1.564416in}}%
\pgfpathlineto{\pgfqpoint{2.032770in}{1.553541in}}%
\pgfpathlineto{\pgfqpoint{2.029598in}{1.541858in}}%
\pgfpathlineto{\pgfqpoint{2.026426in}{1.530753in}}%
\pgfpathlineto{\pgfqpoint{2.023254in}{1.519854in}}%
\pgfpathlineto{\pgfqpoint{2.020082in}{1.508170in}}%
\pgfpathlineto{\pgfqpoint{2.016910in}{1.496266in}}%
\pgfpathlineto{\pgfqpoint{2.013738in}{1.484850in}}%
\pgfpathlineto{\pgfqpoint{2.010566in}{1.474033in}}%
\pgfpathlineto{\pgfqpoint{2.007394in}{1.463028in}}%
\pgfpathlineto{\pgfqpoint{2.004222in}{1.451756in}}%
\pgfpathlineto{\pgfqpoint{2.001050in}{1.440168in}}%
\pgfpathlineto{\pgfqpoint{1.997878in}{1.428692in}}%
\pgfpathlineto{\pgfqpoint{1.994706in}{1.416812in}}%
\pgfpathlineto{\pgfqpoint{1.991534in}{1.405381in}}%
\pgfpathlineto{\pgfqpoint{1.988362in}{1.393975in}}%
\pgfpathlineto{\pgfqpoint{1.985190in}{1.382459in}}%
\pgfpathlineto{\pgfqpoint{1.982018in}{1.361666in}}%
\pgfpathlineto{\pgfqpoint{1.978846in}{1.340565in}}%
\pgfpathlineto{\pgfqpoint{1.975674in}{1.319427in}}%
\pgfpathlineto{\pgfqpoint{1.972502in}{1.297916in}}%
\pgfpathlineto{\pgfqpoint{1.969330in}{1.280118in}}%
\pgfpathlineto{\pgfqpoint{1.966158in}{1.259519in}}%
\pgfpathlineto{\pgfqpoint{1.962986in}{1.238224in}}%
\pgfpathlineto{\pgfqpoint{1.959814in}{1.218037in}}%
\pgfpathlineto{\pgfqpoint{1.956641in}{1.195051in}}%
\pgfpathlineto{\pgfqpoint{1.953469in}{1.175184in}}%
\pgfpathlineto{\pgfqpoint{1.950297in}{1.147042in}}%
\pgfpathlineto{\pgfqpoint{1.947125in}{1.129577in}}%
\pgfpathlineto{\pgfqpoint{1.943953in}{1.129994in}}%
\pgfpathlineto{\pgfqpoint{1.940781in}{1.129465in}}%
\pgfpathclose%
\pgfusepath{stroke,fill}%
\end{pgfscope}%
\begin{pgfscope}%
\pgfpathrectangle{\pgfqpoint{1.623736in}{1.000625in}}{\pgfqpoint{6.975000in}{3.020000in}} %
\pgfusepath{clip}%
\pgfsetbuttcap%
\pgfsetroundjoin%
\definecolor{currentfill}{rgb}{0.333333,0.658824,0.407843}%
\pgfsetfillcolor{currentfill}%
\pgfsetfillopacity{0.200000}%
\pgfsetlinewidth{0.803000pt}%
\definecolor{currentstroke}{rgb}{0.333333,0.658824,0.407843}%
\pgfsetstrokecolor{currentstroke}%
\pgfsetstrokeopacity{0.200000}%
\pgfsetdash{}{0pt}%
\pgfpathmoveto{\pgfqpoint{1.940781in}{1.127084in}}%
\pgfpathlineto{\pgfqpoint{1.940781in}{1.126747in}}%
\pgfpathlineto{\pgfqpoint{1.943953in}{1.126818in}}%
\pgfpathlineto{\pgfqpoint{1.947125in}{1.126818in}}%
\pgfpathlineto{\pgfqpoint{1.950297in}{1.140427in}}%
\pgfpathlineto{\pgfqpoint{1.953469in}{1.164155in}}%
\pgfpathlineto{\pgfqpoint{1.956641in}{1.183665in}}%
\pgfpathlineto{\pgfqpoint{1.959814in}{1.205879in}}%
\pgfpathlineto{\pgfqpoint{1.962986in}{1.226132in}}%
\pgfpathlineto{\pgfqpoint{1.966158in}{1.245268in}}%
\pgfpathlineto{\pgfqpoint{1.969330in}{1.260899in}}%
\pgfpathlineto{\pgfqpoint{1.972502in}{1.279950in}}%
\pgfpathlineto{\pgfqpoint{1.975674in}{1.299755in}}%
\pgfpathlineto{\pgfqpoint{1.978846in}{1.317494in}}%
\pgfpathlineto{\pgfqpoint{1.982018in}{1.338179in}}%
\pgfpathlineto{\pgfqpoint{1.985190in}{1.359440in}}%
\pgfpathlineto{\pgfqpoint{1.988362in}{1.372183in}}%
\pgfpathlineto{\pgfqpoint{1.991534in}{1.385965in}}%
\pgfpathlineto{\pgfqpoint{1.994706in}{1.397563in}}%
\pgfpathlineto{\pgfqpoint{1.997878in}{1.409857in}}%
\pgfpathlineto{\pgfqpoint{2.001050in}{1.421533in}}%
\pgfpathlineto{\pgfqpoint{2.004222in}{1.434329in}}%
\pgfpathlineto{\pgfqpoint{2.007394in}{1.445303in}}%
\pgfpathlineto{\pgfqpoint{2.010566in}{1.457347in}}%
\pgfpathlineto{\pgfqpoint{2.013738in}{1.469323in}}%
\pgfpathlineto{\pgfqpoint{2.016910in}{1.481874in}}%
\pgfpathlineto{\pgfqpoint{2.020082in}{1.492805in}}%
\pgfpathlineto{\pgfqpoint{2.023254in}{1.504589in}}%
\pgfpathlineto{\pgfqpoint{2.026426in}{1.515135in}}%
\pgfpathlineto{\pgfqpoint{2.029598in}{1.526769in}}%
\pgfpathlineto{\pgfqpoint{2.032770in}{1.537291in}}%
\pgfpathlineto{\pgfqpoint{2.035943in}{1.549347in}}%
\pgfpathlineto{\pgfqpoint{2.039115in}{1.561284in}}%
\pgfpathlineto{\pgfqpoint{2.042287in}{1.573535in}}%
\pgfpathlineto{\pgfqpoint{2.045459in}{1.584931in}}%
\pgfpathlineto{\pgfqpoint{2.048631in}{1.597182in}}%
\pgfpathlineto{\pgfqpoint{2.051803in}{1.608789in}}%
\pgfpathlineto{\pgfqpoint{2.054975in}{1.620228in}}%
\pgfpathlineto{\pgfqpoint{2.058147in}{1.631856in}}%
\pgfpathlineto{\pgfqpoint{2.061319in}{1.643593in}}%
\pgfpathlineto{\pgfqpoint{2.064491in}{1.655178in}}%
\pgfpathlineto{\pgfqpoint{2.067663in}{1.666991in}}%
\pgfpathlineto{\pgfqpoint{2.070835in}{1.678151in}}%
\pgfpathlineto{\pgfqpoint{2.074007in}{1.689709in}}%
\pgfpathlineto{\pgfqpoint{2.077179in}{1.701196in}}%
\pgfpathlineto{\pgfqpoint{2.080351in}{1.713158in}}%
\pgfpathlineto{\pgfqpoint{2.083523in}{1.724861in}}%
\pgfpathlineto{\pgfqpoint{2.086695in}{1.736035in}}%
\pgfpathlineto{\pgfqpoint{2.089867in}{1.747468in}}%
\pgfpathlineto{\pgfqpoint{2.093039in}{1.759208in}}%
\pgfpathlineto{\pgfqpoint{2.096211in}{1.770849in}}%
\pgfpathlineto{\pgfqpoint{2.099383in}{1.782817in}}%
\pgfpathlineto{\pgfqpoint{2.102555in}{1.794297in}}%
\pgfpathlineto{\pgfqpoint{2.105727in}{1.805925in}}%
\pgfpathlineto{\pgfqpoint{2.108899in}{1.817469in}}%
\pgfpathlineto{\pgfqpoint{2.112071in}{1.828816in}}%
\pgfpathlineto{\pgfqpoint{2.115244in}{1.840587in}}%
\pgfpathlineto{\pgfqpoint{2.118416in}{1.852155in}}%
\pgfpathlineto{\pgfqpoint{2.121588in}{1.863745in}}%
\pgfpathlineto{\pgfqpoint{2.124760in}{1.875145in}}%
\pgfpathlineto{\pgfqpoint{2.127932in}{1.886586in}}%
\pgfpathlineto{\pgfqpoint{2.131104in}{1.898502in}}%
\pgfpathlineto{\pgfqpoint{2.134276in}{1.909966in}}%
\pgfpathlineto{\pgfqpoint{2.137448in}{1.921515in}}%
\pgfpathlineto{\pgfqpoint{2.140620in}{1.933136in}}%
\pgfpathlineto{\pgfqpoint{2.143792in}{1.944840in}}%
\pgfpathlineto{\pgfqpoint{2.146964in}{1.956646in}}%
\pgfpathlineto{\pgfqpoint{2.150136in}{1.968148in}}%
\pgfpathlineto{\pgfqpoint{2.153308in}{1.979770in}}%
\pgfpathlineto{\pgfqpoint{2.156480in}{1.991086in}}%
\pgfpathlineto{\pgfqpoint{2.159652in}{2.002574in}}%
\pgfpathlineto{\pgfqpoint{2.162824in}{2.014289in}}%
\pgfpathlineto{\pgfqpoint{2.165996in}{2.026105in}}%
\pgfpathlineto{\pgfqpoint{2.169168in}{2.037907in}}%
\pgfpathlineto{\pgfqpoint{2.172340in}{2.049981in}}%
\pgfpathlineto{\pgfqpoint{2.175512in}{2.061060in}}%
\pgfpathlineto{\pgfqpoint{2.178684in}{2.072521in}}%
\pgfpathlineto{\pgfqpoint{2.181856in}{2.084235in}}%
\pgfpathlineto{\pgfqpoint{2.185028in}{2.095850in}}%
\pgfpathlineto{\pgfqpoint{2.188200in}{2.107665in}}%
\pgfpathlineto{\pgfqpoint{2.191372in}{2.119057in}}%
\pgfpathlineto{\pgfqpoint{2.194545in}{2.131135in}}%
\pgfpathlineto{\pgfqpoint{2.197717in}{2.142738in}}%
\pgfpathlineto{\pgfqpoint{2.200889in}{2.154279in}}%
\pgfpathlineto{\pgfqpoint{2.204061in}{2.165879in}}%
\pgfpathlineto{\pgfqpoint{2.207233in}{2.177295in}}%
\pgfpathlineto{\pgfqpoint{2.210405in}{2.189041in}}%
\pgfpathlineto{\pgfqpoint{2.213577in}{2.200568in}}%
\pgfpathlineto{\pgfqpoint{2.216749in}{2.212635in}}%
\pgfpathlineto{\pgfqpoint{2.219921in}{2.224069in}}%
\pgfpathlineto{\pgfqpoint{2.223093in}{2.236129in}}%
\pgfpathlineto{\pgfqpoint{2.226265in}{2.247964in}}%
\pgfpathlineto{\pgfqpoint{2.229437in}{2.259272in}}%
\pgfpathlineto{\pgfqpoint{2.232609in}{2.270595in}}%
\pgfpathlineto{\pgfqpoint{2.235781in}{2.281880in}}%
\pgfpathlineto{\pgfqpoint{2.238953in}{2.293746in}}%
\pgfpathlineto{\pgfqpoint{2.242125in}{2.305267in}}%
\pgfpathlineto{\pgfqpoint{2.245297in}{2.317725in}}%
\pgfpathlineto{\pgfqpoint{2.248469in}{2.329182in}}%
\pgfpathlineto{\pgfqpoint{2.251641in}{2.340809in}}%
\pgfpathlineto{\pgfqpoint{2.254813in}{2.352596in}}%
\pgfpathlineto{\pgfqpoint{2.257985in}{2.364457in}}%
\pgfpathlineto{\pgfqpoint{2.261157in}{2.376172in}}%
\pgfpathlineto{\pgfqpoint{2.264329in}{2.388217in}}%
\pgfpathlineto{\pgfqpoint{2.267501in}{2.399640in}}%
\pgfpathlineto{\pgfqpoint{2.270674in}{2.411399in}}%
\pgfpathlineto{\pgfqpoint{2.273846in}{2.422948in}}%
\pgfpathlineto{\pgfqpoint{2.277018in}{2.434361in}}%
\pgfpathlineto{\pgfqpoint{2.280190in}{2.445807in}}%
\pgfpathlineto{\pgfqpoint{2.283362in}{2.457884in}}%
\pgfpathlineto{\pgfqpoint{2.286534in}{2.469614in}}%
\pgfpathlineto{\pgfqpoint{2.289706in}{2.480799in}}%
\pgfpathlineto{\pgfqpoint{2.292878in}{2.492182in}}%
\pgfpathlineto{\pgfqpoint{2.296050in}{2.503839in}}%
\pgfpathlineto{\pgfqpoint{2.299222in}{2.515392in}}%
\pgfpathlineto{\pgfqpoint{2.302394in}{2.527119in}}%
\pgfpathlineto{\pgfqpoint{2.305566in}{2.538865in}}%
\pgfpathlineto{\pgfqpoint{2.308738in}{2.550814in}}%
\pgfpathlineto{\pgfqpoint{2.311910in}{2.562211in}}%
\pgfpathlineto{\pgfqpoint{2.315082in}{2.573752in}}%
\pgfpathlineto{\pgfqpoint{2.318254in}{2.585396in}}%
\pgfpathlineto{\pgfqpoint{2.321426in}{2.596974in}}%
\pgfpathlineto{\pgfqpoint{2.324598in}{2.608735in}}%
\pgfpathlineto{\pgfqpoint{2.327770in}{2.620199in}}%
\pgfpathlineto{\pgfqpoint{2.330942in}{2.631813in}}%
\pgfpathlineto{\pgfqpoint{2.334114in}{2.643306in}}%
\pgfpathlineto{\pgfqpoint{2.337286in}{2.654684in}}%
\pgfpathlineto{\pgfqpoint{2.340458in}{2.666646in}}%
\pgfpathlineto{\pgfqpoint{2.343630in}{2.678155in}}%
\pgfpathlineto{\pgfqpoint{2.346802in}{2.689519in}}%
\pgfpathlineto{\pgfqpoint{2.349975in}{2.700868in}}%
\pgfpathlineto{\pgfqpoint{2.353147in}{2.712688in}}%
\pgfpathlineto{\pgfqpoint{2.356319in}{2.724118in}}%
\pgfpathlineto{\pgfqpoint{2.359491in}{2.735521in}}%
\pgfpathlineto{\pgfqpoint{2.362663in}{2.747005in}}%
\pgfpathlineto{\pgfqpoint{2.365835in}{2.759166in}}%
\pgfpathlineto{\pgfqpoint{2.369007in}{2.771228in}}%
\pgfpathlineto{\pgfqpoint{2.372179in}{2.782397in}}%
\pgfpathlineto{\pgfqpoint{2.375351in}{2.794077in}}%
\pgfpathlineto{\pgfqpoint{2.378523in}{2.805735in}}%
\pgfpathlineto{\pgfqpoint{2.381695in}{2.817014in}}%
\pgfpathlineto{\pgfqpoint{2.384867in}{2.828892in}}%
\pgfpathlineto{\pgfqpoint{2.388039in}{2.840267in}}%
\pgfpathlineto{\pgfqpoint{2.391211in}{2.851895in}}%
\pgfpathlineto{\pgfqpoint{2.394383in}{2.863427in}}%
\pgfpathlineto{\pgfqpoint{2.397555in}{2.875111in}}%
\pgfpathlineto{\pgfqpoint{2.400727in}{2.883129in}}%
\pgfpathlineto{\pgfqpoint{2.403899in}{2.888385in}}%
\pgfpathlineto{\pgfqpoint{2.407071in}{2.891225in}}%
\pgfpathlineto{\pgfqpoint{2.410243in}{2.893377in}}%
\pgfpathlineto{\pgfqpoint{2.413415in}{2.895147in}}%
\pgfpathlineto{\pgfqpoint{2.416587in}{2.896582in}}%
\pgfpathlineto{\pgfqpoint{2.419759in}{2.897796in}}%
\pgfpathlineto{\pgfqpoint{2.422931in}{2.899810in}}%
\pgfpathlineto{\pgfqpoint{2.426103in}{2.901045in}}%
\pgfpathlineto{\pgfqpoint{2.429276in}{2.901088in}}%
\pgfpathlineto{\pgfqpoint{2.432448in}{2.901175in}}%
\pgfpathlineto{\pgfqpoint{2.435620in}{2.901211in}}%
\pgfpathlineto{\pgfqpoint{2.438792in}{2.901345in}}%
\pgfpathlineto{\pgfqpoint{2.441964in}{2.901129in}}%
\pgfpathlineto{\pgfqpoint{2.445136in}{2.900900in}}%
\pgfpathlineto{\pgfqpoint{2.448308in}{2.900962in}}%
\pgfpathlineto{\pgfqpoint{2.451480in}{2.900900in}}%
\pgfpathlineto{\pgfqpoint{2.454652in}{2.900910in}}%
\pgfpathlineto{\pgfqpoint{2.457824in}{2.900939in}}%
\pgfpathlineto{\pgfqpoint{2.460996in}{2.900938in}}%
\pgfpathlineto{\pgfqpoint{2.464168in}{2.900973in}}%
\pgfpathlineto{\pgfqpoint{2.467340in}{2.901116in}}%
\pgfpathlineto{\pgfqpoint{2.470512in}{2.901044in}}%
\pgfpathlineto{\pgfqpoint{2.473684in}{2.901095in}}%
\pgfpathlineto{\pgfqpoint{2.476856in}{2.901181in}}%
\pgfpathlineto{\pgfqpoint{2.480028in}{2.901100in}}%
\pgfpathlineto{\pgfqpoint{2.483200in}{2.901034in}}%
\pgfpathlineto{\pgfqpoint{2.486372in}{2.901100in}}%
\pgfpathlineto{\pgfqpoint{2.489544in}{2.901114in}}%
\pgfpathlineto{\pgfqpoint{2.492716in}{2.901110in}}%
\pgfpathlineto{\pgfqpoint{2.495888in}{2.901043in}}%
\pgfpathlineto{\pgfqpoint{2.499060in}{2.901033in}}%
\pgfpathlineto{\pgfqpoint{2.502232in}{2.901116in}}%
\pgfpathlineto{\pgfqpoint{2.505405in}{2.901291in}}%
\pgfpathlineto{\pgfqpoint{2.508577in}{2.901233in}}%
\pgfpathlineto{\pgfqpoint{2.511749in}{2.901219in}}%
\pgfpathlineto{\pgfqpoint{2.514921in}{2.901091in}}%
\pgfpathlineto{\pgfqpoint{2.518093in}{2.901065in}}%
\pgfpathlineto{\pgfqpoint{2.521265in}{2.901035in}}%
\pgfpathlineto{\pgfqpoint{2.524437in}{2.901042in}}%
\pgfpathlineto{\pgfqpoint{2.527609in}{2.900975in}}%
\pgfpathlineto{\pgfqpoint{2.530781in}{2.900943in}}%
\pgfpathlineto{\pgfqpoint{2.533953in}{2.901034in}}%
\pgfpathlineto{\pgfqpoint{2.537125in}{2.901044in}}%
\pgfpathlineto{\pgfqpoint{2.540297in}{2.901113in}}%
\pgfpathlineto{\pgfqpoint{2.543469in}{2.901065in}}%
\pgfpathlineto{\pgfqpoint{2.546641in}{2.900911in}}%
\pgfpathlineto{\pgfqpoint{2.549813in}{2.901046in}}%
\pgfpathlineto{\pgfqpoint{2.552985in}{2.900819in}}%
\pgfpathlineto{\pgfqpoint{2.556157in}{2.900623in}}%
\pgfpathlineto{\pgfqpoint{2.559329in}{2.900567in}}%
\pgfpathlineto{\pgfqpoint{2.562501in}{2.900661in}}%
\pgfpathlineto{\pgfqpoint{2.565673in}{2.900868in}}%
\pgfpathlineto{\pgfqpoint{2.568845in}{2.900891in}}%
\pgfpathlineto{\pgfqpoint{2.572017in}{2.901012in}}%
\pgfpathlineto{\pgfqpoint{2.575189in}{2.901029in}}%
\pgfpathlineto{\pgfqpoint{2.578361in}{2.901184in}}%
\pgfpathlineto{\pgfqpoint{2.581533in}{2.901185in}}%
\pgfpathlineto{\pgfqpoint{2.584706in}{2.901201in}}%
\pgfpathlineto{\pgfqpoint{2.587878in}{2.901361in}}%
\pgfpathlineto{\pgfqpoint{2.591050in}{2.901282in}}%
\pgfpathlineto{\pgfqpoint{2.594222in}{2.901241in}}%
\pgfpathlineto{\pgfqpoint{2.597394in}{2.901146in}}%
\pgfpathlineto{\pgfqpoint{2.600566in}{2.901177in}}%
\pgfpathlineto{\pgfqpoint{2.603738in}{2.901119in}}%
\pgfpathlineto{\pgfqpoint{2.606910in}{2.901005in}}%
\pgfpathlineto{\pgfqpoint{2.610082in}{2.901069in}}%
\pgfpathlineto{\pgfqpoint{2.613254in}{2.901149in}}%
\pgfpathlineto{\pgfqpoint{2.616426in}{2.901102in}}%
\pgfpathlineto{\pgfqpoint{2.619598in}{2.900859in}}%
\pgfpathlineto{\pgfqpoint{2.622770in}{2.901070in}}%
\pgfpathlineto{\pgfqpoint{2.625942in}{2.901113in}}%
\pgfpathlineto{\pgfqpoint{2.629114in}{2.901249in}}%
\pgfpathlineto{\pgfqpoint{2.632286in}{2.901221in}}%
\pgfpathlineto{\pgfqpoint{2.635458in}{2.901201in}}%
\pgfpathlineto{\pgfqpoint{2.638630in}{2.901176in}}%
\pgfpathlineto{\pgfqpoint{2.641802in}{2.900969in}}%
\pgfpathlineto{\pgfqpoint{2.644974in}{2.900953in}}%
\pgfpathlineto{\pgfqpoint{2.648146in}{2.901104in}}%
\pgfpathlineto{\pgfqpoint{2.651318in}{2.901146in}}%
\pgfpathlineto{\pgfqpoint{2.654490in}{2.901210in}}%
\pgfpathlineto{\pgfqpoint{2.657662in}{2.901487in}}%
\pgfpathlineto{\pgfqpoint{2.660834in}{2.901429in}}%
\pgfpathlineto{\pgfqpoint{2.664007in}{2.901408in}}%
\pgfpathlineto{\pgfqpoint{2.667179in}{2.901437in}}%
\pgfpathlineto{\pgfqpoint{2.670351in}{2.901296in}}%
\pgfpathlineto{\pgfqpoint{2.673523in}{2.901163in}}%
\pgfpathlineto{\pgfqpoint{2.676695in}{2.901156in}}%
\pgfpathlineto{\pgfqpoint{2.679867in}{2.901283in}}%
\pgfpathlineto{\pgfqpoint{2.683039in}{2.901195in}}%
\pgfpathlineto{\pgfqpoint{2.686211in}{2.900996in}}%
\pgfpathlineto{\pgfqpoint{2.689383in}{2.900878in}}%
\pgfpathlineto{\pgfqpoint{2.692555in}{2.900870in}}%
\pgfpathlineto{\pgfqpoint{2.695727in}{2.900879in}}%
\pgfpathlineto{\pgfqpoint{2.698899in}{2.900848in}}%
\pgfpathlineto{\pgfqpoint{2.702071in}{2.900978in}}%
\pgfpathlineto{\pgfqpoint{2.705243in}{2.900824in}}%
\pgfpathlineto{\pgfqpoint{2.708415in}{2.900784in}}%
\pgfpathlineto{\pgfqpoint{2.711587in}{2.900790in}}%
\pgfpathlineto{\pgfqpoint{2.714759in}{2.901065in}}%
\pgfpathlineto{\pgfqpoint{2.717931in}{2.900840in}}%
\pgfpathlineto{\pgfqpoint{2.721103in}{2.900690in}}%
\pgfpathlineto{\pgfqpoint{2.724275in}{2.900709in}}%
\pgfpathlineto{\pgfqpoint{2.727447in}{2.900712in}}%
\pgfpathlineto{\pgfqpoint{2.730619in}{2.900718in}}%
\pgfpathlineto{\pgfqpoint{2.733791in}{2.900816in}}%
\pgfpathlineto{\pgfqpoint{2.736963in}{2.900769in}}%
\pgfpathlineto{\pgfqpoint{2.740136in}{2.900684in}}%
\pgfpathlineto{\pgfqpoint{2.743308in}{2.900602in}}%
\pgfpathlineto{\pgfqpoint{2.746480in}{2.900697in}}%
\pgfpathlineto{\pgfqpoint{2.749652in}{2.900759in}}%
\pgfpathlineto{\pgfqpoint{2.752824in}{2.900769in}}%
\pgfpathlineto{\pgfqpoint{2.755996in}{2.900701in}}%
\pgfpathlineto{\pgfqpoint{2.759168in}{2.900633in}}%
\pgfpathlineto{\pgfqpoint{2.762340in}{2.900627in}}%
\pgfpathlineto{\pgfqpoint{2.765512in}{2.900690in}}%
\pgfpathlineto{\pgfqpoint{2.768684in}{2.900668in}}%
\pgfpathlineto{\pgfqpoint{2.771856in}{2.900654in}}%
\pgfpathlineto{\pgfqpoint{2.775028in}{2.900851in}}%
\pgfpathlineto{\pgfqpoint{2.778200in}{2.900839in}}%
\pgfpathlineto{\pgfqpoint{2.781372in}{2.900703in}}%
\pgfpathlineto{\pgfqpoint{2.784544in}{2.900767in}}%
\pgfpathlineto{\pgfqpoint{2.787716in}{2.900709in}}%
\pgfpathlineto{\pgfqpoint{2.790888in}{2.900572in}}%
\pgfpathlineto{\pgfqpoint{2.794060in}{2.900461in}}%
\pgfpathlineto{\pgfqpoint{2.797232in}{2.900403in}}%
\pgfpathlineto{\pgfqpoint{2.800404in}{2.900434in}}%
\pgfpathlineto{\pgfqpoint{2.803576in}{2.900493in}}%
\pgfpathlineto{\pgfqpoint{2.806748in}{2.900790in}}%
\pgfpathlineto{\pgfqpoint{2.809920in}{2.900753in}}%
\pgfpathlineto{\pgfqpoint{2.813092in}{2.900809in}}%
\pgfpathlineto{\pgfqpoint{2.816264in}{2.900860in}}%
\pgfpathlineto{\pgfqpoint{2.819437in}{2.900929in}}%
\pgfpathlineto{\pgfqpoint{2.822609in}{2.900847in}}%
\pgfpathlineto{\pgfqpoint{2.825781in}{2.900909in}}%
\pgfpathlineto{\pgfqpoint{2.828953in}{2.900835in}}%
\pgfpathlineto{\pgfqpoint{2.832125in}{2.900825in}}%
\pgfpathlineto{\pgfqpoint{2.835297in}{2.900879in}}%
\pgfpathlineto{\pgfqpoint{2.838469in}{2.900981in}}%
\pgfpathlineto{\pgfqpoint{2.841641in}{2.900943in}}%
\pgfpathlineto{\pgfqpoint{2.844813in}{2.900979in}}%
\pgfpathlineto{\pgfqpoint{2.847985in}{2.901322in}}%
\pgfpathlineto{\pgfqpoint{2.851157in}{2.901324in}}%
\pgfpathlineto{\pgfqpoint{2.854329in}{2.901472in}}%
\pgfpathlineto{\pgfqpoint{2.857501in}{2.901594in}}%
\pgfpathlineto{\pgfqpoint{2.860673in}{2.901487in}}%
\pgfpathlineto{\pgfqpoint{2.863845in}{2.901469in}}%
\pgfpathlineto{\pgfqpoint{2.867017in}{2.901626in}}%
\pgfpathlineto{\pgfqpoint{2.870189in}{2.901847in}}%
\pgfpathlineto{\pgfqpoint{2.873361in}{2.901872in}}%
\pgfpathlineto{\pgfqpoint{2.876533in}{2.901962in}}%
\pgfpathlineto{\pgfqpoint{2.879705in}{2.902071in}}%
\pgfpathlineto{\pgfqpoint{2.882877in}{2.902207in}}%
\pgfpathlineto{\pgfqpoint{2.886049in}{2.902075in}}%
\pgfpathlineto{\pgfqpoint{2.889221in}{2.902131in}}%
\pgfpathlineto{\pgfqpoint{2.892393in}{2.902285in}}%
\pgfpathlineto{\pgfqpoint{2.895565in}{2.902393in}}%
\pgfpathlineto{\pgfqpoint{2.898738in}{2.902553in}}%
\pgfpathlineto{\pgfqpoint{2.901910in}{2.902590in}}%
\pgfpathlineto{\pgfqpoint{2.905082in}{2.902470in}}%
\pgfpathlineto{\pgfqpoint{2.908254in}{2.902434in}}%
\pgfpathlineto{\pgfqpoint{2.911426in}{2.902477in}}%
\pgfpathlineto{\pgfqpoint{2.914598in}{2.902501in}}%
\pgfpathlineto{\pgfqpoint{2.917770in}{2.902553in}}%
\pgfpathlineto{\pgfqpoint{2.920942in}{2.902446in}}%
\pgfpathlineto{\pgfqpoint{2.924114in}{2.902380in}}%
\pgfpathlineto{\pgfqpoint{2.927286in}{2.902386in}}%
\pgfpathlineto{\pgfqpoint{2.930458in}{2.902452in}}%
\pgfpathlineto{\pgfqpoint{2.933630in}{2.902361in}}%
\pgfpathlineto{\pgfqpoint{2.936802in}{2.902497in}}%
\pgfpathlineto{\pgfqpoint{2.939974in}{2.902540in}}%
\pgfpathlineto{\pgfqpoint{2.943146in}{2.902609in}}%
\pgfpathlineto{\pgfqpoint{2.946318in}{2.902740in}}%
\pgfpathlineto{\pgfqpoint{2.949490in}{2.902719in}}%
\pgfpathlineto{\pgfqpoint{2.952662in}{2.902781in}}%
\pgfpathlineto{\pgfqpoint{2.955834in}{2.902806in}}%
\pgfpathlineto{\pgfqpoint{2.959006in}{2.904487in}}%
\pgfpathlineto{\pgfqpoint{2.962178in}{2.904428in}}%
\pgfpathlineto{\pgfqpoint{2.965350in}{2.904390in}}%
\pgfpathlineto{\pgfqpoint{2.968522in}{2.904439in}}%
\pgfpathlineto{\pgfqpoint{2.971694in}{2.904501in}}%
\pgfpathlineto{\pgfqpoint{2.974867in}{2.904396in}}%
\pgfpathlineto{\pgfqpoint{2.978039in}{2.904245in}}%
\pgfpathlineto{\pgfqpoint{2.981211in}{2.904144in}}%
\pgfpathlineto{\pgfqpoint{2.984383in}{2.904095in}}%
\pgfpathlineto{\pgfqpoint{2.987555in}{2.904209in}}%
\pgfpathlineto{\pgfqpoint{2.990727in}{2.904345in}}%
\pgfpathlineto{\pgfqpoint{2.993899in}{2.904438in}}%
\pgfpathlineto{\pgfqpoint{2.997071in}{2.904373in}}%
\pgfpathlineto{\pgfqpoint{3.000243in}{2.904510in}}%
\pgfpathlineto{\pgfqpoint{3.003415in}{2.904721in}}%
\pgfpathlineto{\pgfqpoint{3.006587in}{2.904599in}}%
\pgfpathlineto{\pgfqpoint{3.009759in}{2.904608in}}%
\pgfpathlineto{\pgfqpoint{3.012931in}{2.904793in}}%
\pgfpathlineto{\pgfqpoint{3.016103in}{2.904813in}}%
\pgfpathlineto{\pgfqpoint{3.019275in}{2.904927in}}%
\pgfpathlineto{\pgfqpoint{3.022447in}{2.904660in}}%
\pgfpathlineto{\pgfqpoint{3.025619in}{2.904579in}}%
\pgfpathlineto{\pgfqpoint{3.028791in}{2.904691in}}%
\pgfpathlineto{\pgfqpoint{3.031963in}{2.904709in}}%
\pgfpathlineto{\pgfqpoint{3.035135in}{2.904816in}}%
\pgfpathlineto{\pgfqpoint{3.038307in}{2.904764in}}%
\pgfpathlineto{\pgfqpoint{3.041479in}{2.904682in}}%
\pgfpathlineto{\pgfqpoint{3.044651in}{2.904581in}}%
\pgfpathlineto{\pgfqpoint{3.047823in}{2.904716in}}%
\pgfpathlineto{\pgfqpoint{3.050995in}{2.904541in}}%
\pgfpathlineto{\pgfqpoint{3.054168in}{2.904690in}}%
\pgfpathlineto{\pgfqpoint{3.057340in}{2.904676in}}%
\pgfpathlineto{\pgfqpoint{3.060512in}{2.904768in}}%
\pgfpathlineto{\pgfqpoint{3.063684in}{2.904715in}}%
\pgfpathlineto{\pgfqpoint{3.066856in}{2.904794in}}%
\pgfpathlineto{\pgfqpoint{3.070028in}{2.904940in}}%
\pgfpathlineto{\pgfqpoint{3.073200in}{2.904885in}}%
\pgfpathlineto{\pgfqpoint{3.076372in}{2.904797in}}%
\pgfpathlineto{\pgfqpoint{3.079544in}{2.904894in}}%
\pgfpathlineto{\pgfqpoint{3.082716in}{2.904795in}}%
\pgfpathlineto{\pgfqpoint{3.085888in}{2.904871in}}%
\pgfpathlineto{\pgfqpoint{3.089060in}{2.904909in}}%
\pgfpathlineto{\pgfqpoint{3.092232in}{2.904824in}}%
\pgfpathlineto{\pgfqpoint{3.095404in}{2.904643in}}%
\pgfpathlineto{\pgfqpoint{3.098576in}{2.904553in}}%
\pgfpathlineto{\pgfqpoint{3.101748in}{2.904560in}}%
\pgfpathlineto{\pgfqpoint{3.104920in}{2.904458in}}%
\pgfpathlineto{\pgfqpoint{3.108092in}{2.904303in}}%
\pgfpathlineto{\pgfqpoint{3.111264in}{2.904405in}}%
\pgfpathlineto{\pgfqpoint{3.114436in}{2.904243in}}%
\pgfpathlineto{\pgfqpoint{3.117608in}{2.904299in}}%
\pgfpathlineto{\pgfqpoint{3.120780in}{2.904433in}}%
\pgfpathlineto{\pgfqpoint{3.123952in}{2.904489in}}%
\pgfpathlineto{\pgfqpoint{3.127124in}{2.904411in}}%
\pgfpathlineto{\pgfqpoint{3.130297in}{2.904220in}}%
\pgfpathlineto{\pgfqpoint{3.133469in}{2.904186in}}%
\pgfpathlineto{\pgfqpoint{3.136641in}{2.904105in}}%
\pgfpathlineto{\pgfqpoint{3.139813in}{2.903985in}}%
\pgfpathlineto{\pgfqpoint{3.142985in}{2.903946in}}%
\pgfpathlineto{\pgfqpoint{3.146157in}{2.903715in}}%
\pgfpathlineto{\pgfqpoint{3.149329in}{2.903577in}}%
\pgfpathlineto{\pgfqpoint{3.152501in}{2.903798in}}%
\pgfpathlineto{\pgfqpoint{3.155673in}{2.904033in}}%
\pgfpathlineto{\pgfqpoint{3.158845in}{2.903964in}}%
\pgfpathlineto{\pgfqpoint{3.162017in}{2.903899in}}%
\pgfpathlineto{\pgfqpoint{3.165189in}{2.903951in}}%
\pgfpathlineto{\pgfqpoint{3.168361in}{2.903804in}}%
\pgfpathlineto{\pgfqpoint{3.171533in}{2.903763in}}%
\pgfpathlineto{\pgfqpoint{3.174705in}{2.903699in}}%
\pgfpathlineto{\pgfqpoint{3.177877in}{2.903753in}}%
\pgfpathlineto{\pgfqpoint{3.181049in}{2.904053in}}%
\pgfpathlineto{\pgfqpoint{3.184221in}{2.903765in}}%
\pgfpathlineto{\pgfqpoint{3.187393in}{2.904252in}}%
\pgfpathlineto{\pgfqpoint{3.190565in}{2.904516in}}%
\pgfpathlineto{\pgfqpoint{3.193737in}{2.904690in}}%
\pgfpathlineto{\pgfqpoint{3.196909in}{2.904796in}}%
\pgfpathlineto{\pgfqpoint{3.200081in}{2.904541in}}%
\pgfpathlineto{\pgfqpoint{3.203253in}{2.904706in}}%
\pgfpathlineto{\pgfqpoint{3.206425in}{2.904155in}}%
\pgfpathlineto{\pgfqpoint{3.209598in}{2.904249in}}%
\pgfpathlineto{\pgfqpoint{3.212770in}{2.904062in}}%
\pgfpathlineto{\pgfqpoint{3.215942in}{2.904013in}}%
\pgfpathlineto{\pgfqpoint{3.219114in}{2.903826in}}%
\pgfpathlineto{\pgfqpoint{3.222286in}{2.903335in}}%
\pgfpathlineto{\pgfqpoint{3.225458in}{2.903398in}}%
\pgfpathlineto{\pgfqpoint{3.228630in}{2.903508in}}%
\pgfpathlineto{\pgfqpoint{3.231802in}{2.903806in}}%
\pgfpathlineto{\pgfqpoint{3.234974in}{2.903780in}}%
\pgfpathlineto{\pgfqpoint{3.238146in}{2.903871in}}%
\pgfpathlineto{\pgfqpoint{3.241318in}{2.903977in}}%
\pgfpathlineto{\pgfqpoint{3.244490in}{2.903136in}}%
\pgfpathlineto{\pgfqpoint{3.247662in}{2.903061in}}%
\pgfpathlineto{\pgfqpoint{3.250834in}{2.902907in}}%
\pgfpathlineto{\pgfqpoint{3.254006in}{2.902348in}}%
\pgfpathlineto{\pgfqpoint{3.257178in}{2.902218in}}%
\pgfpathlineto{\pgfqpoint{3.260350in}{2.902038in}}%
\pgfpathlineto{\pgfqpoint{3.263522in}{2.901981in}}%
\pgfpathlineto{\pgfqpoint{3.266694in}{2.902086in}}%
\pgfpathlineto{\pgfqpoint{3.269866in}{2.902096in}}%
\pgfpathlineto{\pgfqpoint{3.273038in}{2.902076in}}%
\pgfpathlineto{\pgfqpoint{3.276210in}{2.902148in}}%
\pgfpathlineto{\pgfqpoint{3.279382in}{2.902443in}}%
\pgfpathlineto{\pgfqpoint{3.282554in}{2.902471in}}%
\pgfpathlineto{\pgfqpoint{3.285726in}{2.902277in}}%
\pgfpathlineto{\pgfqpoint{3.288899in}{2.901938in}}%
\pgfpathlineto{\pgfqpoint{3.292071in}{2.902092in}}%
\pgfpathlineto{\pgfqpoint{3.295243in}{2.902255in}}%
\pgfpathlineto{\pgfqpoint{3.298415in}{2.901987in}}%
\pgfpathlineto{\pgfqpoint{3.301587in}{2.902075in}}%
\pgfpathlineto{\pgfqpoint{3.304759in}{2.902063in}}%
\pgfpathlineto{\pgfqpoint{3.307931in}{2.902263in}}%
\pgfpathlineto{\pgfqpoint{3.311103in}{2.902543in}}%
\pgfpathlineto{\pgfqpoint{3.314275in}{2.902240in}}%
\pgfpathlineto{\pgfqpoint{3.317447in}{2.901944in}}%
\pgfpathlineto{\pgfqpoint{3.320619in}{2.901799in}}%
\pgfpathlineto{\pgfqpoint{3.323791in}{2.902025in}}%
\pgfpathlineto{\pgfqpoint{3.326963in}{2.902094in}}%
\pgfpathlineto{\pgfqpoint{3.330135in}{2.902073in}}%
\pgfpathlineto{\pgfqpoint{3.333307in}{2.902348in}}%
\pgfpathlineto{\pgfqpoint{3.336479in}{2.902847in}}%
\pgfpathlineto{\pgfqpoint{3.339651in}{2.903141in}}%
\pgfpathlineto{\pgfqpoint{3.342823in}{2.903308in}}%
\pgfpathlineto{\pgfqpoint{3.345995in}{2.903428in}}%
\pgfpathlineto{\pgfqpoint{3.349167in}{2.903281in}}%
\pgfpathlineto{\pgfqpoint{3.352339in}{2.903489in}}%
\pgfpathlineto{\pgfqpoint{3.355511in}{2.903327in}}%
\pgfpathlineto{\pgfqpoint{3.358683in}{2.903256in}}%
\pgfpathlineto{\pgfqpoint{3.361855in}{2.903269in}}%
\pgfpathlineto{\pgfqpoint{3.365028in}{2.903561in}}%
\pgfpathlineto{\pgfqpoint{3.368200in}{2.904065in}}%
\pgfpathlineto{\pgfqpoint{3.371372in}{2.904282in}}%
\pgfpathlineto{\pgfqpoint{3.374544in}{2.904473in}}%
\pgfpathlineto{\pgfqpoint{3.377716in}{2.904389in}}%
\pgfpathlineto{\pgfqpoint{3.380888in}{2.904833in}}%
\pgfpathlineto{\pgfqpoint{3.384060in}{2.904833in}}%
\pgfpathlineto{\pgfqpoint{3.387232in}{2.904722in}}%
\pgfpathlineto{\pgfqpoint{3.390404in}{2.904852in}}%
\pgfpathlineto{\pgfqpoint{3.393576in}{2.905040in}}%
\pgfpathlineto{\pgfqpoint{3.396748in}{2.904720in}}%
\pgfpathlineto{\pgfqpoint{3.399920in}{2.904984in}}%
\pgfpathlineto{\pgfqpoint{3.403092in}{2.905095in}}%
\pgfpathlineto{\pgfqpoint{3.406264in}{2.905188in}}%
\pgfpathlineto{\pgfqpoint{3.409436in}{2.905407in}}%
\pgfpathlineto{\pgfqpoint{3.412608in}{2.905333in}}%
\pgfpathlineto{\pgfqpoint{3.415780in}{2.905039in}}%
\pgfpathlineto{\pgfqpoint{3.418952in}{2.904932in}}%
\pgfpathlineto{\pgfqpoint{3.422124in}{2.904642in}}%
\pgfpathlineto{\pgfqpoint{3.425296in}{2.904800in}}%
\pgfpathlineto{\pgfqpoint{3.428468in}{2.904951in}}%
\pgfpathlineto{\pgfqpoint{3.431640in}{2.904986in}}%
\pgfpathlineto{\pgfqpoint{3.434812in}{2.904793in}}%
\pgfpathlineto{\pgfqpoint{3.437984in}{2.904901in}}%
\pgfpathlineto{\pgfqpoint{3.441156in}{2.904660in}}%
\pgfpathlineto{\pgfqpoint{3.444329in}{2.904543in}}%
\pgfpathlineto{\pgfqpoint{3.447501in}{2.904331in}}%
\pgfpathlineto{\pgfqpoint{3.450673in}{2.904226in}}%
\pgfpathlineto{\pgfqpoint{3.453845in}{2.903922in}}%
\pgfpathlineto{\pgfqpoint{3.457017in}{2.903714in}}%
\pgfpathlineto{\pgfqpoint{3.460189in}{2.903768in}}%
\pgfpathlineto{\pgfqpoint{3.463361in}{2.903377in}}%
\pgfpathlineto{\pgfqpoint{3.466533in}{2.903317in}}%
\pgfpathlineto{\pgfqpoint{3.469705in}{2.903181in}}%
\pgfpathlineto{\pgfqpoint{3.472877in}{2.903066in}}%
\pgfpathlineto{\pgfqpoint{3.476049in}{2.902986in}}%
\pgfpathlineto{\pgfqpoint{3.479221in}{2.902929in}}%
\pgfpathlineto{\pgfqpoint{3.482393in}{2.902925in}}%
\pgfpathlineto{\pgfqpoint{3.485565in}{2.902876in}}%
\pgfpathlineto{\pgfqpoint{3.488737in}{2.902919in}}%
\pgfpathlineto{\pgfqpoint{3.491909in}{2.902776in}}%
\pgfpathlineto{\pgfqpoint{3.495081in}{2.902706in}}%
\pgfpathlineto{\pgfqpoint{3.498253in}{2.902519in}}%
\pgfpathlineto{\pgfqpoint{3.501425in}{2.902739in}}%
\pgfpathlineto{\pgfqpoint{3.504597in}{2.903138in}}%
\pgfpathlineto{\pgfqpoint{3.507769in}{2.903153in}}%
\pgfpathlineto{\pgfqpoint{3.510941in}{2.903141in}}%
\pgfpathlineto{\pgfqpoint{3.514113in}{2.902958in}}%
\pgfpathlineto{\pgfqpoint{3.517285in}{2.903157in}}%
\pgfpathlineto{\pgfqpoint{3.520457in}{2.902685in}}%
\pgfpathlineto{\pgfqpoint{3.523630in}{2.902627in}}%
\pgfpathlineto{\pgfqpoint{3.526802in}{2.902181in}}%
\pgfpathlineto{\pgfqpoint{3.529974in}{2.902119in}}%
\pgfpathlineto{\pgfqpoint{3.533146in}{2.902095in}}%
\pgfpathlineto{\pgfqpoint{3.536318in}{2.902163in}}%
\pgfpathlineto{\pgfqpoint{3.539490in}{2.902116in}}%
\pgfpathlineto{\pgfqpoint{3.542662in}{2.902217in}}%
\pgfpathlineto{\pgfqpoint{3.545834in}{2.901982in}}%
\pgfpathlineto{\pgfqpoint{3.549006in}{2.901888in}}%
\pgfpathlineto{\pgfqpoint{3.552178in}{2.902085in}}%
\pgfpathlineto{\pgfqpoint{3.555350in}{2.902396in}}%
\pgfpathlineto{\pgfqpoint{3.558522in}{2.902557in}}%
\pgfpathlineto{\pgfqpoint{3.561694in}{2.903011in}}%
\pgfpathlineto{\pgfqpoint{3.564866in}{2.903096in}}%
\pgfpathlineto{\pgfqpoint{3.568038in}{2.903944in}}%
\pgfpathlineto{\pgfqpoint{3.571210in}{2.903787in}}%
\pgfpathlineto{\pgfqpoint{3.574382in}{2.903924in}}%
\pgfpathlineto{\pgfqpoint{3.577554in}{2.904015in}}%
\pgfpathlineto{\pgfqpoint{3.580726in}{2.904299in}}%
\pgfpathlineto{\pgfqpoint{3.583898in}{2.904396in}}%
\pgfpathlineto{\pgfqpoint{3.587070in}{2.904205in}}%
\pgfpathlineto{\pgfqpoint{3.590242in}{2.904192in}}%
\pgfpathlineto{\pgfqpoint{3.593414in}{2.904180in}}%
\pgfpathlineto{\pgfqpoint{3.596586in}{2.904157in}}%
\pgfpathlineto{\pgfqpoint{3.599759in}{2.904142in}}%
\pgfpathlineto{\pgfqpoint{3.602931in}{2.904285in}}%
\pgfpathlineto{\pgfqpoint{3.606103in}{2.904376in}}%
\pgfpathlineto{\pgfqpoint{3.609275in}{2.904689in}}%
\pgfpathlineto{\pgfqpoint{3.612447in}{2.905144in}}%
\pgfpathlineto{\pgfqpoint{3.615619in}{2.905458in}}%
\pgfpathlineto{\pgfqpoint{3.618791in}{2.905482in}}%
\pgfpathlineto{\pgfqpoint{3.621963in}{2.905539in}}%
\pgfpathlineto{\pgfqpoint{3.625135in}{2.905787in}}%
\pgfpathlineto{\pgfqpoint{3.628307in}{2.905662in}}%
\pgfpathlineto{\pgfqpoint{3.631479in}{2.905196in}}%
\pgfpathlineto{\pgfqpoint{3.634651in}{2.904936in}}%
\pgfpathlineto{\pgfqpoint{3.637823in}{2.904884in}}%
\pgfpathlineto{\pgfqpoint{3.640995in}{2.904791in}}%
\pgfpathlineto{\pgfqpoint{3.644167in}{2.904804in}}%
\pgfpathlineto{\pgfqpoint{3.647339in}{2.904749in}}%
\pgfpathlineto{\pgfqpoint{3.650511in}{2.904758in}}%
\pgfpathlineto{\pgfqpoint{3.653683in}{2.904979in}}%
\pgfpathlineto{\pgfqpoint{3.656855in}{2.905007in}}%
\pgfpathlineto{\pgfqpoint{3.660027in}{2.905286in}}%
\pgfpathlineto{\pgfqpoint{3.663199in}{2.905372in}}%
\pgfpathlineto{\pgfqpoint{3.666371in}{2.905305in}}%
\pgfpathlineto{\pgfqpoint{3.669543in}{2.905173in}}%
\pgfpathlineto{\pgfqpoint{3.672715in}{2.905552in}}%
\pgfpathlineto{\pgfqpoint{3.675887in}{2.905643in}}%
\pgfpathlineto{\pgfqpoint{3.679060in}{2.905684in}}%
\pgfpathlineto{\pgfqpoint{3.682232in}{2.905264in}}%
\pgfpathlineto{\pgfqpoint{3.685404in}{2.905357in}}%
\pgfpathlineto{\pgfqpoint{3.688576in}{2.905442in}}%
\pgfpathlineto{\pgfqpoint{3.691748in}{2.905111in}}%
\pgfpathlineto{\pgfqpoint{3.694920in}{2.904871in}}%
\pgfpathlineto{\pgfqpoint{3.698092in}{2.904884in}}%
\pgfpathlineto{\pgfqpoint{3.701264in}{2.904866in}}%
\pgfpathlineto{\pgfqpoint{3.704436in}{2.905043in}}%
\pgfpathlineto{\pgfqpoint{3.707608in}{2.905428in}}%
\pgfpathlineto{\pgfqpoint{3.710780in}{2.905449in}}%
\pgfpathlineto{\pgfqpoint{3.713952in}{2.905466in}}%
\pgfpathlineto{\pgfqpoint{3.717124in}{2.905360in}}%
\pgfpathlineto{\pgfqpoint{3.720296in}{2.905368in}}%
\pgfpathlineto{\pgfqpoint{3.723468in}{2.905157in}}%
\pgfpathlineto{\pgfqpoint{3.726640in}{2.905237in}}%
\pgfpathlineto{\pgfqpoint{3.729812in}{2.904951in}}%
\pgfpathlineto{\pgfqpoint{3.732984in}{2.905009in}}%
\pgfpathlineto{\pgfqpoint{3.736156in}{2.905068in}}%
\pgfpathlineto{\pgfqpoint{3.739328in}{2.905053in}}%
\pgfpathlineto{\pgfqpoint{3.742500in}{2.905082in}}%
\pgfpathlineto{\pgfqpoint{3.745672in}{2.905102in}}%
\pgfpathlineto{\pgfqpoint{3.748844in}{2.904739in}}%
\pgfpathlineto{\pgfqpoint{3.752016in}{2.904617in}}%
\pgfpathlineto{\pgfqpoint{3.755188in}{2.904540in}}%
\pgfpathlineto{\pgfqpoint{3.758361in}{2.904692in}}%
\pgfpathlineto{\pgfqpoint{3.761533in}{2.904641in}}%
\pgfpathlineto{\pgfqpoint{3.764705in}{2.904846in}}%
\pgfpathlineto{\pgfqpoint{3.767877in}{2.905120in}}%
\pgfpathlineto{\pgfqpoint{3.771049in}{2.905097in}}%
\pgfpathlineto{\pgfqpoint{3.774221in}{2.905204in}}%
\pgfpathlineto{\pgfqpoint{3.777393in}{2.905288in}}%
\pgfpathlineto{\pgfqpoint{3.780565in}{2.905758in}}%
\pgfpathlineto{\pgfqpoint{3.783737in}{2.905598in}}%
\pgfpathlineto{\pgfqpoint{3.786909in}{2.905679in}}%
\pgfpathlineto{\pgfqpoint{3.790081in}{2.905794in}}%
\pgfpathlineto{\pgfqpoint{3.793253in}{2.905621in}}%
\pgfpathlineto{\pgfqpoint{3.796425in}{2.905481in}}%
\pgfpathlineto{\pgfqpoint{3.799597in}{2.905713in}}%
\pgfpathlineto{\pgfqpoint{3.802769in}{2.906268in}}%
\pgfpathlineto{\pgfqpoint{3.805941in}{2.906399in}}%
\pgfpathlineto{\pgfqpoint{3.809113in}{2.906466in}}%
\pgfpathlineto{\pgfqpoint{3.812285in}{2.906397in}}%
\pgfpathlineto{\pgfqpoint{3.815457in}{2.906943in}}%
\pgfpathlineto{\pgfqpoint{3.818629in}{2.907438in}}%
\pgfpathlineto{\pgfqpoint{3.821801in}{2.907465in}}%
\pgfpathlineto{\pgfqpoint{3.824973in}{2.907282in}}%
\pgfpathlineto{\pgfqpoint{3.828145in}{2.906951in}}%
\pgfpathlineto{\pgfqpoint{3.831317in}{2.907011in}}%
\pgfpathlineto{\pgfqpoint{3.834490in}{2.906684in}}%
\pgfpathlineto{\pgfqpoint{3.837662in}{2.906736in}}%
\pgfpathlineto{\pgfqpoint{3.840834in}{2.906557in}}%
\pgfpathlineto{\pgfqpoint{3.844006in}{2.906405in}}%
\pgfpathlineto{\pgfqpoint{3.847178in}{2.906430in}}%
\pgfpathlineto{\pgfqpoint{3.850350in}{2.906593in}}%
\pgfpathlineto{\pgfqpoint{3.853522in}{2.906493in}}%
\pgfpathlineto{\pgfqpoint{3.856694in}{2.906726in}}%
\pgfpathlineto{\pgfqpoint{3.859866in}{2.906605in}}%
\pgfpathlineto{\pgfqpoint{3.863038in}{2.906511in}}%
\pgfpathlineto{\pgfqpoint{3.866210in}{2.906608in}}%
\pgfpathlineto{\pgfqpoint{3.869382in}{2.906310in}}%
\pgfpathlineto{\pgfqpoint{3.872554in}{2.906415in}}%
\pgfpathlineto{\pgfqpoint{3.875726in}{2.905971in}}%
\pgfpathlineto{\pgfqpoint{3.878898in}{2.905376in}}%
\pgfpathlineto{\pgfqpoint{3.882070in}{2.904632in}}%
\pgfpathlineto{\pgfqpoint{3.885242in}{2.904500in}}%
\pgfpathlineto{\pgfqpoint{3.888414in}{2.904219in}}%
\pgfpathlineto{\pgfqpoint{3.891586in}{2.904165in}}%
\pgfpathlineto{\pgfqpoint{3.894758in}{2.904202in}}%
\pgfpathlineto{\pgfqpoint{3.897930in}{2.904167in}}%
\pgfpathlineto{\pgfqpoint{3.901102in}{2.904121in}}%
\pgfpathlineto{\pgfqpoint{3.904274in}{2.904263in}}%
\pgfpathlineto{\pgfqpoint{3.907446in}{2.904313in}}%
\pgfpathlineto{\pgfqpoint{3.910618in}{2.904300in}}%
\pgfpathlineto{\pgfqpoint{3.913791in}{2.904164in}}%
\pgfpathlineto{\pgfqpoint{3.916963in}{2.904243in}}%
\pgfpathlineto{\pgfqpoint{3.920135in}{2.904434in}}%
\pgfpathlineto{\pgfqpoint{3.923307in}{2.904512in}}%
\pgfpathlineto{\pgfqpoint{3.926479in}{2.904489in}}%
\pgfpathlineto{\pgfqpoint{3.929651in}{2.904259in}}%
\pgfpathlineto{\pgfqpoint{3.932823in}{2.903763in}}%
\pgfpathlineto{\pgfqpoint{3.935995in}{2.903710in}}%
\pgfpathlineto{\pgfqpoint{3.939167in}{2.904058in}}%
\pgfpathlineto{\pgfqpoint{3.942339in}{2.903956in}}%
\pgfpathlineto{\pgfqpoint{3.945511in}{2.903816in}}%
\pgfpathlineto{\pgfqpoint{3.948683in}{2.903986in}}%
\pgfpathlineto{\pgfqpoint{3.951855in}{2.903954in}}%
\pgfpathlineto{\pgfqpoint{3.955027in}{2.903422in}}%
\pgfpathlineto{\pgfqpoint{3.958199in}{2.903668in}}%
\pgfpathlineto{\pgfqpoint{3.961371in}{2.903596in}}%
\pgfpathlineto{\pgfqpoint{3.964543in}{2.903311in}}%
\pgfpathlineto{\pgfqpoint{3.967715in}{2.903602in}}%
\pgfpathlineto{\pgfqpoint{3.970887in}{2.903564in}}%
\pgfpathlineto{\pgfqpoint{3.974059in}{2.903664in}}%
\pgfpathlineto{\pgfqpoint{3.977231in}{2.903663in}}%
\pgfpathlineto{\pgfqpoint{3.980403in}{2.903414in}}%
\pgfpathlineto{\pgfqpoint{3.983575in}{2.903430in}}%
\pgfpathlineto{\pgfqpoint{3.986747in}{2.903189in}}%
\pgfpathlineto{\pgfqpoint{3.989919in}{2.903272in}}%
\pgfpathlineto{\pgfqpoint{3.993092in}{2.903315in}}%
\pgfpathlineto{\pgfqpoint{3.996264in}{2.903216in}}%
\pgfpathlineto{\pgfqpoint{3.999436in}{2.903279in}}%
\pgfpathlineto{\pgfqpoint{4.002608in}{2.903361in}}%
\pgfpathlineto{\pgfqpoint{4.005780in}{2.903268in}}%
\pgfpathlineto{\pgfqpoint{4.008952in}{2.903393in}}%
\pgfpathlineto{\pgfqpoint{4.012124in}{2.903573in}}%
\pgfpathlineto{\pgfqpoint{4.015296in}{2.903537in}}%
\pgfpathlineto{\pgfqpoint{4.018468in}{2.903531in}}%
\pgfpathlineto{\pgfqpoint{4.021640in}{2.903552in}}%
\pgfpathlineto{\pgfqpoint{4.024812in}{2.903586in}}%
\pgfpathlineto{\pgfqpoint{4.027984in}{2.903687in}}%
\pgfpathlineto{\pgfqpoint{4.031156in}{2.903929in}}%
\pgfpathlineto{\pgfqpoint{4.034328in}{2.904099in}}%
\pgfpathlineto{\pgfqpoint{4.037500in}{2.904232in}}%
\pgfpathlineto{\pgfqpoint{4.040672in}{2.904307in}}%
\pgfpathlineto{\pgfqpoint{4.043844in}{2.904282in}}%
\pgfpathlineto{\pgfqpoint{4.047016in}{2.903945in}}%
\pgfpathlineto{\pgfqpoint{4.050188in}{2.904053in}}%
\pgfpathlineto{\pgfqpoint{4.053360in}{2.904090in}}%
\pgfpathlineto{\pgfqpoint{4.056532in}{2.903867in}}%
\pgfpathlineto{\pgfqpoint{4.059704in}{2.903707in}}%
\pgfpathlineto{\pgfqpoint{4.062876in}{2.903629in}}%
\pgfpathlineto{\pgfqpoint{4.066048in}{2.903567in}}%
\pgfpathlineto{\pgfqpoint{4.069221in}{2.903701in}}%
\pgfpathlineto{\pgfqpoint{4.072393in}{2.904012in}}%
\pgfpathlineto{\pgfqpoint{4.075565in}{2.904017in}}%
\pgfpathlineto{\pgfqpoint{4.078737in}{2.903547in}}%
\pgfpathlineto{\pgfqpoint{4.081909in}{2.903455in}}%
\pgfpathlineto{\pgfqpoint{4.085081in}{2.903376in}}%
\pgfpathlineto{\pgfqpoint{4.088253in}{2.903492in}}%
\pgfpathlineto{\pgfqpoint{4.091425in}{2.903519in}}%
\pgfpathlineto{\pgfqpoint{4.094597in}{2.903895in}}%
\pgfpathlineto{\pgfqpoint{4.097769in}{2.903982in}}%
\pgfpathlineto{\pgfqpoint{4.100941in}{2.903988in}}%
\pgfpathlineto{\pgfqpoint{4.104113in}{2.904096in}}%
\pgfpathlineto{\pgfqpoint{4.107285in}{2.903953in}}%
\pgfpathlineto{\pgfqpoint{4.110457in}{2.904358in}}%
\pgfpathlineto{\pgfqpoint{4.113629in}{2.904366in}}%
\pgfpathlineto{\pgfqpoint{4.116801in}{2.904539in}}%
\pgfpathlineto{\pgfqpoint{4.119973in}{2.904540in}}%
\pgfpathlineto{\pgfqpoint{4.123145in}{2.905190in}}%
\pgfpathlineto{\pgfqpoint{4.126317in}{2.905172in}}%
\pgfpathlineto{\pgfqpoint{4.129489in}{2.905044in}}%
\pgfpathlineto{\pgfqpoint{4.132661in}{2.904949in}}%
\pgfpathlineto{\pgfqpoint{4.135833in}{2.904868in}}%
\pgfpathlineto{\pgfqpoint{4.139005in}{2.904911in}}%
\pgfpathlineto{\pgfqpoint{4.142177in}{2.905018in}}%
\pgfpathlineto{\pgfqpoint{4.145349in}{2.905134in}}%
\pgfpathlineto{\pgfqpoint{4.148522in}{2.905727in}}%
\pgfpathlineto{\pgfqpoint{4.151694in}{2.905807in}}%
\pgfpathlineto{\pgfqpoint{4.154866in}{2.905959in}}%
\pgfpathlineto{\pgfqpoint{4.158038in}{2.905909in}}%
\pgfpathlineto{\pgfqpoint{4.161210in}{2.905893in}}%
\pgfpathlineto{\pgfqpoint{4.164382in}{2.906307in}}%
\pgfpathlineto{\pgfqpoint{4.167554in}{2.906340in}}%
\pgfpathlineto{\pgfqpoint{4.170726in}{2.906539in}}%
\pgfpathlineto{\pgfqpoint{4.173898in}{2.906330in}}%
\pgfpathlineto{\pgfqpoint{4.177070in}{2.906436in}}%
\pgfpathlineto{\pgfqpoint{4.180242in}{2.906958in}}%
\pgfpathlineto{\pgfqpoint{4.183414in}{2.906959in}}%
\pgfpathlineto{\pgfqpoint{4.186586in}{2.907068in}}%
\pgfpathlineto{\pgfqpoint{4.189758in}{2.906831in}}%
\pgfpathlineto{\pgfqpoint{4.192930in}{2.906645in}}%
\pgfpathlineto{\pgfqpoint{4.196102in}{2.906836in}}%
\pgfpathlineto{\pgfqpoint{4.199274in}{2.907391in}}%
\pgfpathlineto{\pgfqpoint{4.202446in}{2.907657in}}%
\pgfpathlineto{\pgfqpoint{4.205618in}{2.907623in}}%
\pgfpathlineto{\pgfqpoint{4.208790in}{2.907570in}}%
\pgfpathlineto{\pgfqpoint{4.211962in}{2.907450in}}%
\pgfpathlineto{\pgfqpoint{4.215134in}{2.907277in}}%
\pgfpathlineto{\pgfqpoint{4.218306in}{2.907396in}}%
\pgfpathlineto{\pgfqpoint{4.221478in}{2.907084in}}%
\pgfpathlineto{\pgfqpoint{4.224650in}{2.907130in}}%
\pgfpathlineto{\pgfqpoint{4.227823in}{2.906944in}}%
\pgfpathlineto{\pgfqpoint{4.230995in}{2.906962in}}%
\pgfpathlineto{\pgfqpoint{4.234167in}{2.906945in}}%
\pgfpathlineto{\pgfqpoint{4.237339in}{2.906887in}}%
\pgfpathlineto{\pgfqpoint{4.240511in}{2.906902in}}%
\pgfpathlineto{\pgfqpoint{4.243683in}{2.906906in}}%
\pgfpathlineto{\pgfqpoint{4.246855in}{2.906871in}}%
\pgfpathlineto{\pgfqpoint{4.250027in}{2.906873in}}%
\pgfpathlineto{\pgfqpoint{4.253199in}{2.906897in}}%
\pgfpathlineto{\pgfqpoint{4.256371in}{2.907025in}}%
\pgfpathlineto{\pgfqpoint{4.259543in}{2.907152in}}%
\pgfpathlineto{\pgfqpoint{4.262715in}{2.907251in}}%
\pgfpathlineto{\pgfqpoint{4.265887in}{2.907345in}}%
\pgfpathlineto{\pgfqpoint{4.269059in}{2.907496in}}%
\pgfpathlineto{\pgfqpoint{4.272231in}{2.907158in}}%
\pgfpathlineto{\pgfqpoint{4.275403in}{2.906999in}}%
\pgfpathlineto{\pgfqpoint{4.278575in}{2.906699in}}%
\pgfpathlineto{\pgfqpoint{4.281747in}{2.906707in}}%
\pgfpathlineto{\pgfqpoint{4.284919in}{2.906907in}}%
\pgfpathlineto{\pgfqpoint{4.288091in}{2.906830in}}%
\pgfpathlineto{\pgfqpoint{4.291263in}{2.906684in}}%
\pgfpathlineto{\pgfqpoint{4.294435in}{2.906907in}}%
\pgfpathlineto{\pgfqpoint{4.297607in}{2.907238in}}%
\pgfpathlineto{\pgfqpoint{4.300779in}{2.907401in}}%
\pgfpathlineto{\pgfqpoint{4.303952in}{2.906923in}}%
\pgfpathlineto{\pgfqpoint{4.307124in}{2.907239in}}%
\pgfpathlineto{\pgfqpoint{4.310296in}{2.907199in}}%
\pgfpathlineto{\pgfqpoint{4.313468in}{2.907170in}}%
\pgfpathlineto{\pgfqpoint{4.316640in}{2.907193in}}%
\pgfpathlineto{\pgfqpoint{4.319812in}{2.907105in}}%
\pgfpathlineto{\pgfqpoint{4.322984in}{2.907140in}}%
\pgfpathlineto{\pgfqpoint{4.326156in}{2.907020in}}%
\pgfpathlineto{\pgfqpoint{4.329328in}{2.907081in}}%
\pgfpathlineto{\pgfqpoint{4.332500in}{2.907181in}}%
\pgfpathlineto{\pgfqpoint{4.335672in}{2.907019in}}%
\pgfpathlineto{\pgfqpoint{4.338844in}{2.907172in}}%
\pgfpathlineto{\pgfqpoint{4.342016in}{2.907159in}}%
\pgfpathlineto{\pgfqpoint{4.345188in}{2.907350in}}%
\pgfpathlineto{\pgfqpoint{4.348360in}{2.907667in}}%
\pgfpathlineto{\pgfqpoint{4.351532in}{2.907975in}}%
\pgfpathlineto{\pgfqpoint{4.354704in}{2.908403in}}%
\pgfpathlineto{\pgfqpoint{4.357876in}{2.908413in}}%
\pgfpathlineto{\pgfqpoint{4.361048in}{2.908478in}}%
\pgfpathlineto{\pgfqpoint{4.364220in}{2.908457in}}%
\pgfpathlineto{\pgfqpoint{4.367392in}{2.908497in}}%
\pgfpathlineto{\pgfqpoint{4.370564in}{2.908382in}}%
\pgfpathlineto{\pgfqpoint{4.373736in}{2.908213in}}%
\pgfpathlineto{\pgfqpoint{4.376908in}{2.908322in}}%
\pgfpathlineto{\pgfqpoint{4.380080in}{2.908459in}}%
\pgfpathlineto{\pgfqpoint{4.383253in}{2.908336in}}%
\pgfpathlineto{\pgfqpoint{4.386425in}{2.908259in}}%
\pgfpathlineto{\pgfqpoint{4.389597in}{2.908342in}}%
\pgfpathlineto{\pgfqpoint{4.392769in}{2.908335in}}%
\pgfpathlineto{\pgfqpoint{4.395941in}{2.908550in}}%
\pgfpathlineto{\pgfqpoint{4.399113in}{2.908849in}}%
\pgfpathlineto{\pgfqpoint{4.402285in}{2.908920in}}%
\pgfpathlineto{\pgfqpoint{4.405457in}{2.908884in}}%
\pgfpathlineto{\pgfqpoint{4.408629in}{2.908816in}}%
\pgfpathlineto{\pgfqpoint{4.411801in}{2.908641in}}%
\pgfpathlineto{\pgfqpoint{4.414973in}{2.908728in}}%
\pgfpathlineto{\pgfqpoint{4.418145in}{2.908583in}}%
\pgfpathlineto{\pgfqpoint{4.421317in}{2.908589in}}%
\pgfpathlineto{\pgfqpoint{4.424489in}{2.908592in}}%
\pgfpathlineto{\pgfqpoint{4.427661in}{2.908563in}}%
\pgfpathlineto{\pgfqpoint{4.430833in}{2.908472in}}%
\pgfpathlineto{\pgfqpoint{4.434005in}{2.908144in}}%
\pgfpathlineto{\pgfqpoint{4.437177in}{2.908038in}}%
\pgfpathlineto{\pgfqpoint{4.440349in}{2.908266in}}%
\pgfpathlineto{\pgfqpoint{4.443521in}{2.908229in}}%
\pgfpathlineto{\pgfqpoint{4.446693in}{2.907887in}}%
\pgfpathlineto{\pgfqpoint{4.449865in}{2.907897in}}%
\pgfpathlineto{\pgfqpoint{4.453037in}{2.908035in}}%
\pgfpathlineto{\pgfqpoint{4.456209in}{2.907485in}}%
\pgfpathlineto{\pgfqpoint{4.459381in}{2.907585in}}%
\pgfpathlineto{\pgfqpoint{4.462554in}{2.907494in}}%
\pgfpathlineto{\pgfqpoint{4.465726in}{2.907572in}}%
\pgfpathlineto{\pgfqpoint{4.468898in}{2.907675in}}%
\pgfpathlineto{\pgfqpoint{4.472070in}{2.907926in}}%
\pgfpathlineto{\pgfqpoint{4.475242in}{2.907821in}}%
\pgfpathlineto{\pgfqpoint{4.478414in}{2.907915in}}%
\pgfpathlineto{\pgfqpoint{4.481586in}{2.908269in}}%
\pgfpathlineto{\pgfqpoint{4.484758in}{2.908234in}}%
\pgfpathlineto{\pgfqpoint{4.487930in}{2.908030in}}%
\pgfpathlineto{\pgfqpoint{4.491102in}{2.908001in}}%
\pgfpathlineto{\pgfqpoint{4.494274in}{2.907914in}}%
\pgfpathlineto{\pgfqpoint{4.497446in}{2.908013in}}%
\pgfpathlineto{\pgfqpoint{4.500618in}{2.907970in}}%
\pgfpathlineto{\pgfqpoint{4.503790in}{2.907814in}}%
\pgfpathlineto{\pgfqpoint{4.506962in}{2.907584in}}%
\pgfpathlineto{\pgfqpoint{4.510134in}{2.907443in}}%
\pgfpathlineto{\pgfqpoint{4.513306in}{2.907473in}}%
\pgfpathlineto{\pgfqpoint{4.516478in}{2.907318in}}%
\pgfpathlineto{\pgfqpoint{4.519650in}{2.907333in}}%
\pgfpathlineto{\pgfqpoint{4.522822in}{2.907379in}}%
\pgfpathlineto{\pgfqpoint{4.525994in}{2.907313in}}%
\pgfpathlineto{\pgfqpoint{4.529166in}{2.907436in}}%
\pgfpathlineto{\pgfqpoint{4.532338in}{2.907276in}}%
\pgfpathlineto{\pgfqpoint{4.535510in}{2.906958in}}%
\pgfpathlineto{\pgfqpoint{4.538683in}{2.906709in}}%
\pgfpathlineto{\pgfqpoint{4.541855in}{2.906439in}}%
\pgfpathlineto{\pgfqpoint{4.545027in}{2.906471in}}%
\pgfpathlineto{\pgfqpoint{4.548199in}{2.906430in}}%
\pgfpathlineto{\pgfqpoint{4.551371in}{2.906510in}}%
\pgfpathlineto{\pgfqpoint{4.554543in}{2.906492in}}%
\pgfpathlineto{\pgfqpoint{4.557715in}{2.906520in}}%
\pgfpathlineto{\pgfqpoint{4.560887in}{2.906602in}}%
\pgfpathlineto{\pgfqpoint{4.564059in}{2.906536in}}%
\pgfpathlineto{\pgfqpoint{4.567231in}{2.906081in}}%
\pgfpathlineto{\pgfqpoint{4.570403in}{2.906505in}}%
\pgfpathlineto{\pgfqpoint{4.573575in}{2.906821in}}%
\pgfpathlineto{\pgfqpoint{4.576747in}{2.906589in}}%
\pgfpathlineto{\pgfqpoint{4.579919in}{2.906518in}}%
\pgfpathlineto{\pgfqpoint{4.583091in}{2.906357in}}%
\pgfpathlineto{\pgfqpoint{4.586263in}{2.906185in}}%
\pgfpathlineto{\pgfqpoint{4.589435in}{2.906254in}}%
\pgfpathlineto{\pgfqpoint{4.592607in}{2.906208in}}%
\pgfpathlineto{\pgfqpoint{4.595779in}{2.906288in}}%
\pgfpathlineto{\pgfqpoint{4.598951in}{2.906399in}}%
\pgfpathlineto{\pgfqpoint{4.602123in}{2.906486in}}%
\pgfpathlineto{\pgfqpoint{4.605295in}{2.906455in}}%
\pgfpathlineto{\pgfqpoint{4.608467in}{2.906451in}}%
\pgfpathlineto{\pgfqpoint{4.611639in}{2.906422in}}%
\pgfpathlineto{\pgfqpoint{4.614811in}{2.906311in}}%
\pgfpathlineto{\pgfqpoint{4.617984in}{2.906524in}}%
\pgfpathlineto{\pgfqpoint{4.621156in}{2.906553in}}%
\pgfpathlineto{\pgfqpoint{4.624328in}{2.906248in}}%
\pgfpathlineto{\pgfqpoint{4.627500in}{2.906287in}}%
\pgfpathlineto{\pgfqpoint{4.630672in}{2.906236in}}%
\pgfpathlineto{\pgfqpoint{4.633844in}{2.906029in}}%
\pgfpathlineto{\pgfqpoint{4.637016in}{2.906048in}}%
\pgfpathlineto{\pgfqpoint{4.640188in}{2.906018in}}%
\pgfpathlineto{\pgfqpoint{4.643360in}{2.906120in}}%
\pgfpathlineto{\pgfqpoint{4.646532in}{2.906379in}}%
\pgfpathlineto{\pgfqpoint{4.649704in}{2.906617in}}%
\pgfpathlineto{\pgfqpoint{4.652876in}{2.906409in}}%
\pgfpathlineto{\pgfqpoint{4.656048in}{2.905890in}}%
\pgfpathlineto{\pgfqpoint{4.659220in}{2.906241in}}%
\pgfpathlineto{\pgfqpoint{4.662392in}{2.906409in}}%
\pgfpathlineto{\pgfqpoint{4.665564in}{2.906382in}}%
\pgfpathlineto{\pgfqpoint{4.668736in}{2.906270in}}%
\pgfpathlineto{\pgfqpoint{4.671908in}{2.906402in}}%
\pgfpathlineto{\pgfqpoint{4.675080in}{2.906987in}}%
\pgfpathlineto{\pgfqpoint{4.678252in}{2.907091in}}%
\pgfpathlineto{\pgfqpoint{4.681424in}{2.907036in}}%
\pgfpathlineto{\pgfqpoint{4.684596in}{2.907027in}}%
\pgfpathlineto{\pgfqpoint{4.687768in}{2.906765in}}%
\pgfpathlineto{\pgfqpoint{4.690940in}{2.906227in}}%
\pgfpathlineto{\pgfqpoint{4.694112in}{2.906378in}}%
\pgfpathlineto{\pgfqpoint{4.697285in}{2.906278in}}%
\pgfpathlineto{\pgfqpoint{4.700457in}{2.906265in}}%
\pgfpathlineto{\pgfqpoint{4.703629in}{2.906196in}}%
\pgfpathlineto{\pgfqpoint{4.706801in}{2.906048in}}%
\pgfpathlineto{\pgfqpoint{4.709973in}{2.905914in}}%
\pgfpathlineto{\pgfqpoint{4.713145in}{2.905699in}}%
\pgfpathlineto{\pgfqpoint{4.716317in}{2.905710in}}%
\pgfpathlineto{\pgfqpoint{4.719489in}{2.905625in}}%
\pgfpathlineto{\pgfqpoint{4.722661in}{2.905604in}}%
\pgfpathlineto{\pgfqpoint{4.725833in}{2.905859in}}%
\pgfpathlineto{\pgfqpoint{4.729005in}{2.906190in}}%
\pgfpathlineto{\pgfqpoint{4.732177in}{2.905961in}}%
\pgfpathlineto{\pgfqpoint{4.735349in}{2.906053in}}%
\pgfpathlineto{\pgfqpoint{4.738521in}{2.906127in}}%
\pgfpathlineto{\pgfqpoint{4.741693in}{2.906274in}}%
\pgfpathlineto{\pgfqpoint{4.744865in}{2.906030in}}%
\pgfpathlineto{\pgfqpoint{4.748037in}{2.906029in}}%
\pgfpathlineto{\pgfqpoint{4.751209in}{2.906333in}}%
\pgfpathlineto{\pgfqpoint{4.754381in}{2.906470in}}%
\pgfpathlineto{\pgfqpoint{4.757553in}{2.907045in}}%
\pgfpathlineto{\pgfqpoint{4.760725in}{2.907480in}}%
\pgfpathlineto{\pgfqpoint{4.763897in}{2.907647in}}%
\pgfpathlineto{\pgfqpoint{4.767069in}{2.907775in}}%
\pgfpathlineto{\pgfqpoint{4.770241in}{2.907934in}}%
\pgfpathlineto{\pgfqpoint{4.773414in}{2.908120in}}%
\pgfpathlineto{\pgfqpoint{4.776586in}{2.908110in}}%
\pgfpathlineto{\pgfqpoint{4.779758in}{2.907741in}}%
\pgfpathlineto{\pgfqpoint{4.782930in}{2.907313in}}%
\pgfpathlineto{\pgfqpoint{4.786102in}{2.907182in}}%
\pgfpathlineto{\pgfqpoint{4.789274in}{2.907239in}}%
\pgfpathlineto{\pgfqpoint{4.792446in}{2.907161in}}%
\pgfpathlineto{\pgfqpoint{4.795618in}{2.907268in}}%
\pgfpathlineto{\pgfqpoint{4.798790in}{2.907375in}}%
\pgfpathlineto{\pgfqpoint{4.801962in}{2.907345in}}%
\pgfpathlineto{\pgfqpoint{4.805134in}{2.907279in}}%
\pgfpathlineto{\pgfqpoint{4.808306in}{2.907220in}}%
\pgfpathlineto{\pgfqpoint{4.811478in}{2.907242in}}%
\pgfpathlineto{\pgfqpoint{4.814650in}{2.907198in}}%
\pgfpathlineto{\pgfqpoint{4.817822in}{2.907224in}}%
\pgfpathlineto{\pgfqpoint{4.820994in}{2.907209in}}%
\pgfpathlineto{\pgfqpoint{4.824166in}{2.906888in}}%
\pgfpathlineto{\pgfqpoint{4.827338in}{2.907008in}}%
\pgfpathlineto{\pgfqpoint{4.830510in}{2.907121in}}%
\pgfpathlineto{\pgfqpoint{4.833682in}{2.907183in}}%
\pgfpathlineto{\pgfqpoint{4.836854in}{2.906747in}}%
\pgfpathlineto{\pgfqpoint{4.840026in}{2.906855in}}%
\pgfpathlineto{\pgfqpoint{4.843198in}{2.906923in}}%
\pgfpathlineto{\pgfqpoint{4.846370in}{2.907096in}}%
\pgfpathlineto{\pgfqpoint{4.849542in}{2.907440in}}%
\pgfpathlineto{\pgfqpoint{4.852715in}{2.907340in}}%
\pgfpathlineto{\pgfqpoint{4.855887in}{2.907319in}}%
\pgfpathlineto{\pgfqpoint{4.859059in}{2.907352in}}%
\pgfpathlineto{\pgfqpoint{4.862231in}{2.907490in}}%
\pgfpathlineto{\pgfqpoint{4.865403in}{2.907569in}}%
\pgfpathlineto{\pgfqpoint{4.868575in}{2.907730in}}%
\pgfpathlineto{\pgfqpoint{4.871747in}{2.907668in}}%
\pgfpathlineto{\pgfqpoint{4.874919in}{2.907104in}}%
\pgfpathlineto{\pgfqpoint{4.878091in}{2.907037in}}%
\pgfpathlineto{\pgfqpoint{4.881263in}{2.907092in}}%
\pgfpathlineto{\pgfqpoint{4.884435in}{2.907735in}}%
\pgfpathlineto{\pgfqpoint{4.887607in}{2.907742in}}%
\pgfpathlineto{\pgfqpoint{4.890779in}{2.907719in}}%
\pgfpathlineto{\pgfqpoint{4.893951in}{2.907638in}}%
\pgfpathlineto{\pgfqpoint{4.897123in}{2.907578in}}%
\pgfpathlineto{\pgfqpoint{4.900295in}{2.907245in}}%
\pgfpathlineto{\pgfqpoint{4.903467in}{2.907309in}}%
\pgfpathlineto{\pgfqpoint{4.906639in}{2.907326in}}%
\pgfpathlineto{\pgfqpoint{4.909811in}{2.907021in}}%
\pgfpathlineto{\pgfqpoint{4.912983in}{2.906591in}}%
\pgfpathlineto{\pgfqpoint{4.916155in}{2.906313in}}%
\pgfpathlineto{\pgfqpoint{4.919327in}{2.906108in}}%
\pgfpathlineto{\pgfqpoint{4.922499in}{2.905909in}}%
\pgfpathlineto{\pgfqpoint{4.925671in}{2.905828in}}%
\pgfpathlineto{\pgfqpoint{4.928844in}{2.905440in}}%
\pgfpathlineto{\pgfqpoint{4.932016in}{2.905321in}}%
\pgfpathlineto{\pgfqpoint{4.935188in}{2.905286in}}%
\pgfpathlineto{\pgfqpoint{4.938360in}{2.905375in}}%
\pgfpathlineto{\pgfqpoint{4.941532in}{2.905710in}}%
\pgfpathlineto{\pgfqpoint{4.944704in}{2.905789in}}%
\pgfpathlineto{\pgfqpoint{4.947876in}{2.905619in}}%
\pgfpathlineto{\pgfqpoint{4.951048in}{2.905418in}}%
\pgfpathlineto{\pgfqpoint{4.954220in}{2.905196in}}%
\pgfpathlineto{\pgfqpoint{4.957392in}{2.905265in}}%
\pgfpathlineto{\pgfqpoint{4.960564in}{2.904924in}}%
\pgfpathlineto{\pgfqpoint{4.963736in}{2.904770in}}%
\pgfpathlineto{\pgfqpoint{4.966908in}{2.904891in}}%
\pgfpathlineto{\pgfqpoint{4.970080in}{2.904949in}}%
\pgfpathlineto{\pgfqpoint{4.973252in}{2.904897in}}%
\pgfpathlineto{\pgfqpoint{4.976424in}{2.904951in}}%
\pgfpathlineto{\pgfqpoint{4.979596in}{2.904990in}}%
\pgfpathlineto{\pgfqpoint{4.982768in}{2.905084in}}%
\pgfpathlineto{\pgfqpoint{4.985940in}{2.905067in}}%
\pgfpathlineto{\pgfqpoint{4.989112in}{2.905026in}}%
\pgfpathlineto{\pgfqpoint{4.992284in}{2.904925in}}%
\pgfpathlineto{\pgfqpoint{4.995456in}{2.905048in}}%
\pgfpathlineto{\pgfqpoint{4.998628in}{2.905098in}}%
\pgfpathlineto{\pgfqpoint{5.001800in}{2.905172in}}%
\pgfpathlineto{\pgfqpoint{5.004972in}{2.904897in}}%
\pgfpathlineto{\pgfqpoint{5.008145in}{2.904885in}}%
\pgfpathlineto{\pgfqpoint{5.011317in}{2.905072in}}%
\pgfpathlineto{\pgfqpoint{5.014489in}{2.904979in}}%
\pgfpathlineto{\pgfqpoint{5.017661in}{2.904749in}}%
\pgfpathlineto{\pgfqpoint{5.020833in}{2.904480in}}%
\pgfpathlineto{\pgfqpoint{5.024005in}{2.904059in}}%
\pgfpathlineto{\pgfqpoint{5.027177in}{2.903970in}}%
\pgfpathlineto{\pgfqpoint{5.030349in}{2.903862in}}%
\pgfpathlineto{\pgfqpoint{5.033521in}{2.903835in}}%
\pgfpathlineto{\pgfqpoint{5.036693in}{2.904003in}}%
\pgfpathlineto{\pgfqpoint{5.039865in}{2.904073in}}%
\pgfpathlineto{\pgfqpoint{5.043037in}{2.904002in}}%
\pgfpathlineto{\pgfqpoint{5.046209in}{2.903989in}}%
\pgfpathlineto{\pgfqpoint{5.049381in}{2.903389in}}%
\pgfpathlineto{\pgfqpoint{5.052553in}{2.903193in}}%
\pgfpathlineto{\pgfqpoint{5.055725in}{2.903488in}}%
\pgfpathlineto{\pgfqpoint{5.058897in}{2.903768in}}%
\pgfpathlineto{\pgfqpoint{5.062069in}{2.903739in}}%
\pgfpathlineto{\pgfqpoint{5.065241in}{2.903871in}}%
\pgfpathlineto{\pgfqpoint{5.068413in}{2.904134in}}%
\pgfpathlineto{\pgfqpoint{5.071585in}{2.904014in}}%
\pgfpathlineto{\pgfqpoint{5.074757in}{2.903866in}}%
\pgfpathlineto{\pgfqpoint{5.077929in}{2.903415in}}%
\pgfpathlineto{\pgfqpoint{5.081101in}{2.903550in}}%
\pgfpathlineto{\pgfqpoint{5.084273in}{2.903515in}}%
\pgfpathlineto{\pgfqpoint{5.087446in}{2.903307in}}%
\pgfpathlineto{\pgfqpoint{5.090618in}{2.903100in}}%
\pgfpathlineto{\pgfqpoint{5.093790in}{2.903088in}}%
\pgfpathlineto{\pgfqpoint{5.096962in}{2.903188in}}%
\pgfpathlineto{\pgfqpoint{5.100134in}{2.903179in}}%
\pgfpathlineto{\pgfqpoint{5.103306in}{2.903104in}}%
\pgfpathlineto{\pgfqpoint{5.106478in}{2.903097in}}%
\pgfpathlineto{\pgfqpoint{5.109650in}{2.903133in}}%
\pgfpathlineto{\pgfqpoint{5.112822in}{2.903348in}}%
\pgfpathlineto{\pgfqpoint{5.115994in}{2.903069in}}%
\pgfpathlineto{\pgfqpoint{5.119166in}{2.902951in}}%
\pgfpathlineto{\pgfqpoint{5.122338in}{2.902982in}}%
\pgfpathlineto{\pgfqpoint{5.125510in}{2.903060in}}%
\pgfpathlineto{\pgfqpoint{5.128682in}{2.903105in}}%
\pgfpathlineto{\pgfqpoint{5.131854in}{2.902977in}}%
\pgfpathlineto{\pgfqpoint{5.135026in}{2.902929in}}%
\pgfpathlineto{\pgfqpoint{5.138198in}{2.902637in}}%
\pgfpathlineto{\pgfqpoint{5.141370in}{2.902549in}}%
\pgfpathlineto{\pgfqpoint{5.144542in}{2.902640in}}%
\pgfpathlineto{\pgfqpoint{5.147714in}{2.902323in}}%
\pgfpathlineto{\pgfqpoint{5.150886in}{2.902308in}}%
\pgfpathlineto{\pgfqpoint{5.154058in}{2.902428in}}%
\pgfpathlineto{\pgfqpoint{5.157230in}{2.902520in}}%
\pgfpathlineto{\pgfqpoint{5.160402in}{2.902579in}}%
\pgfpathlineto{\pgfqpoint{5.163575in}{2.902453in}}%
\pgfpathlineto{\pgfqpoint{5.166747in}{2.902563in}}%
\pgfpathlineto{\pgfqpoint{5.169919in}{2.902677in}}%
\pgfpathlineto{\pgfqpoint{5.173091in}{2.902200in}}%
\pgfpathlineto{\pgfqpoint{5.176263in}{2.901909in}}%
\pgfpathlineto{\pgfqpoint{5.179435in}{2.901827in}}%
\pgfpathlineto{\pgfqpoint{5.182607in}{2.901717in}}%
\pgfpathlineto{\pgfqpoint{5.185779in}{2.901785in}}%
\pgfpathlineto{\pgfqpoint{5.188951in}{2.901568in}}%
\pgfpathlineto{\pgfqpoint{5.192123in}{2.901539in}}%
\pgfpathlineto{\pgfqpoint{5.195295in}{2.901422in}}%
\pgfpathlineto{\pgfqpoint{5.198467in}{2.901391in}}%
\pgfpathlineto{\pgfqpoint{5.201639in}{2.901628in}}%
\pgfpathlineto{\pgfqpoint{5.204811in}{2.901701in}}%
\pgfpathlineto{\pgfqpoint{5.207983in}{2.901923in}}%
\pgfpathlineto{\pgfqpoint{5.211155in}{2.901070in}}%
\pgfpathlineto{\pgfqpoint{5.214327in}{2.900828in}}%
\pgfpathlineto{\pgfqpoint{5.217499in}{2.901078in}}%
\pgfpathlineto{\pgfqpoint{5.220671in}{2.901074in}}%
\pgfpathlineto{\pgfqpoint{5.223843in}{2.900753in}}%
\pgfpathlineto{\pgfqpoint{5.227015in}{2.901012in}}%
\pgfpathlineto{\pgfqpoint{5.230187in}{2.901041in}}%
\pgfpathlineto{\pgfqpoint{5.233359in}{2.901184in}}%
\pgfpathlineto{\pgfqpoint{5.236531in}{2.901139in}}%
\pgfpathlineto{\pgfqpoint{5.239703in}{2.901199in}}%
\pgfpathlineto{\pgfqpoint{5.242876in}{2.901235in}}%
\pgfpathlineto{\pgfqpoint{5.246048in}{2.901442in}}%
\pgfpathlineto{\pgfqpoint{5.249220in}{2.901383in}}%
\pgfpathlineto{\pgfqpoint{5.252392in}{2.901642in}}%
\pgfpathlineto{\pgfqpoint{5.255564in}{2.901958in}}%
\pgfpathlineto{\pgfqpoint{5.258736in}{2.902312in}}%
\pgfpathlineto{\pgfqpoint{5.261908in}{2.902324in}}%
\pgfpathlineto{\pgfqpoint{5.265080in}{2.902393in}}%
\pgfpathlineto{\pgfqpoint{5.268252in}{2.902337in}}%
\pgfpathlineto{\pgfqpoint{5.271424in}{2.901906in}}%
\pgfpathlineto{\pgfqpoint{5.274596in}{2.901562in}}%
\pgfpathlineto{\pgfqpoint{5.277768in}{2.901340in}}%
\pgfpathlineto{\pgfqpoint{5.280940in}{2.901401in}}%
\pgfpathlineto{\pgfqpoint{5.284112in}{2.901039in}}%
\pgfpathlineto{\pgfqpoint{5.287284in}{2.900853in}}%
\pgfpathlineto{\pgfqpoint{5.290456in}{2.900870in}}%
\pgfpathlineto{\pgfqpoint{5.293628in}{2.900786in}}%
\pgfpathlineto{\pgfqpoint{5.296800in}{2.900324in}}%
\pgfpathlineto{\pgfqpoint{5.299972in}{2.900446in}}%
\pgfpathlineto{\pgfqpoint{5.303144in}{2.900725in}}%
\pgfpathlineto{\pgfqpoint{5.306316in}{2.900822in}}%
\pgfpathlineto{\pgfqpoint{5.309488in}{2.901048in}}%
\pgfpathlineto{\pgfqpoint{5.312660in}{2.901226in}}%
\pgfpathlineto{\pgfqpoint{5.315832in}{2.901188in}}%
\pgfpathlineto{\pgfqpoint{5.319004in}{2.901381in}}%
\pgfpathlineto{\pgfqpoint{5.322177in}{2.901132in}}%
\pgfpathlineto{\pgfqpoint{5.325349in}{2.901116in}}%
\pgfpathlineto{\pgfqpoint{5.328521in}{2.901220in}}%
\pgfpathlineto{\pgfqpoint{5.331693in}{2.901089in}}%
\pgfpathlineto{\pgfqpoint{5.334865in}{2.900421in}}%
\pgfpathlineto{\pgfqpoint{5.338037in}{2.900359in}}%
\pgfpathlineto{\pgfqpoint{5.341209in}{2.900334in}}%
\pgfpathlineto{\pgfqpoint{5.344381in}{2.900330in}}%
\pgfpathlineto{\pgfqpoint{5.347553in}{2.900151in}}%
\pgfpathlineto{\pgfqpoint{5.350725in}{2.900035in}}%
\pgfpathlineto{\pgfqpoint{5.353897in}{2.899799in}}%
\pgfpathlineto{\pgfqpoint{5.357069in}{2.899682in}}%
\pgfpathlineto{\pgfqpoint{5.360241in}{2.900243in}}%
\pgfpathlineto{\pgfqpoint{5.363413in}{2.900595in}}%
\pgfpathlineto{\pgfqpoint{5.366585in}{2.900591in}}%
\pgfpathlineto{\pgfqpoint{5.369757in}{2.900779in}}%
\pgfpathlineto{\pgfqpoint{5.372929in}{2.900778in}}%
\pgfpathlineto{\pgfqpoint{5.376101in}{2.900728in}}%
\pgfpathlineto{\pgfqpoint{5.379273in}{2.900671in}}%
\pgfpathlineto{\pgfqpoint{5.382445in}{2.900547in}}%
\pgfpathlineto{\pgfqpoint{5.385617in}{2.900529in}}%
\pgfpathlineto{\pgfqpoint{5.388789in}{2.900516in}}%
\pgfpathlineto{\pgfqpoint{5.391961in}{2.900548in}}%
\pgfpathlineto{\pgfqpoint{5.395133in}{2.900264in}}%
\pgfpathlineto{\pgfqpoint{5.398306in}{2.900242in}}%
\pgfpathlineto{\pgfqpoint{5.401478in}{2.899747in}}%
\pgfpathlineto{\pgfqpoint{5.404650in}{2.899677in}}%
\pgfpathlineto{\pgfqpoint{5.407822in}{2.899500in}}%
\pgfpathlineto{\pgfqpoint{5.410994in}{2.899157in}}%
\pgfpathlineto{\pgfqpoint{5.414166in}{2.898616in}}%
\pgfpathlineto{\pgfqpoint{5.417338in}{2.898139in}}%
\pgfpathlineto{\pgfqpoint{5.420510in}{2.898360in}}%
\pgfpathlineto{\pgfqpoint{5.423682in}{2.898202in}}%
\pgfpathlineto{\pgfqpoint{5.426854in}{2.898278in}}%
\pgfpathlineto{\pgfqpoint{5.430026in}{2.898346in}}%
\pgfpathlineto{\pgfqpoint{5.433198in}{2.898388in}}%
\pgfpathlineto{\pgfqpoint{5.436370in}{2.898342in}}%
\pgfpathlineto{\pgfqpoint{5.439542in}{2.898525in}}%
\pgfpathlineto{\pgfqpoint{5.442714in}{2.898578in}}%
\pgfpathlineto{\pgfqpoint{5.445886in}{2.898719in}}%
\pgfpathlineto{\pgfqpoint{5.449058in}{2.898665in}}%
\pgfpathlineto{\pgfqpoint{5.452230in}{2.898702in}}%
\pgfpathlineto{\pgfqpoint{5.455402in}{2.898773in}}%
\pgfpathlineto{\pgfqpoint{5.458574in}{2.899021in}}%
\pgfpathlineto{\pgfqpoint{5.461746in}{2.899017in}}%
\pgfpathlineto{\pgfqpoint{5.464918in}{2.898851in}}%
\pgfpathlineto{\pgfqpoint{5.468090in}{2.898793in}}%
\pgfpathlineto{\pgfqpoint{5.471262in}{2.898692in}}%
\pgfpathlineto{\pgfqpoint{5.474434in}{2.898544in}}%
\pgfpathlineto{\pgfqpoint{5.477607in}{2.898678in}}%
\pgfpathlineto{\pgfqpoint{5.480779in}{2.898752in}}%
\pgfpathlineto{\pgfqpoint{5.483951in}{2.898968in}}%
\pgfpathlineto{\pgfqpoint{5.487123in}{2.898901in}}%
\pgfpathlineto{\pgfqpoint{5.490295in}{2.898816in}}%
\pgfpathlineto{\pgfqpoint{5.493467in}{2.899060in}}%
\pgfpathlineto{\pgfqpoint{5.496639in}{2.899091in}}%
\pgfpathlineto{\pgfqpoint{5.499811in}{2.899531in}}%
\pgfpathlineto{\pgfqpoint{5.502983in}{2.899743in}}%
\pgfpathlineto{\pgfqpoint{5.506155in}{2.899916in}}%
\pgfpathlineto{\pgfqpoint{5.509327in}{2.900086in}}%
\pgfpathlineto{\pgfqpoint{5.512499in}{2.900415in}}%
\pgfpathlineto{\pgfqpoint{5.515671in}{2.900005in}}%
\pgfpathlineto{\pgfqpoint{5.518843in}{2.899872in}}%
\pgfpathlineto{\pgfqpoint{5.522015in}{2.899552in}}%
\pgfpathlineto{\pgfqpoint{5.525187in}{2.899453in}}%
\pgfpathlineto{\pgfqpoint{5.528359in}{2.899168in}}%
\pgfpathlineto{\pgfqpoint{5.531531in}{2.899530in}}%
\pgfpathlineto{\pgfqpoint{5.534703in}{2.899827in}}%
\pgfpathlineto{\pgfqpoint{5.537875in}{2.899802in}}%
\pgfpathlineto{\pgfqpoint{5.541047in}{2.900116in}}%
\pgfpathlineto{\pgfqpoint{5.544219in}{2.900024in}}%
\pgfpathlineto{\pgfqpoint{5.547391in}{2.900032in}}%
\pgfpathlineto{\pgfqpoint{5.550563in}{2.900013in}}%
\pgfpathlineto{\pgfqpoint{5.553735in}{2.899732in}}%
\pgfpathlineto{\pgfqpoint{5.556908in}{2.899733in}}%
\pgfpathlineto{\pgfqpoint{5.560080in}{2.899625in}}%
\pgfpathlineto{\pgfqpoint{5.563252in}{2.899335in}}%
\pgfpathlineto{\pgfqpoint{5.566424in}{2.899012in}}%
\pgfpathlineto{\pgfqpoint{5.569596in}{2.899191in}}%
\pgfpathlineto{\pgfqpoint{5.572768in}{2.899220in}}%
\pgfpathlineto{\pgfqpoint{5.575940in}{2.899264in}}%
\pgfpathlineto{\pgfqpoint{5.579112in}{2.899294in}}%
\pgfpathlineto{\pgfqpoint{5.582284in}{2.899027in}}%
\pgfpathlineto{\pgfqpoint{5.585456in}{2.899268in}}%
\pgfpathlineto{\pgfqpoint{5.588628in}{2.899308in}}%
\pgfpathlineto{\pgfqpoint{5.591800in}{2.899222in}}%
\pgfpathlineto{\pgfqpoint{5.594972in}{2.899160in}}%
\pgfpathlineto{\pgfqpoint{5.598144in}{2.898996in}}%
\pgfpathlineto{\pgfqpoint{5.601316in}{2.898513in}}%
\pgfpathlineto{\pgfqpoint{5.604488in}{2.898635in}}%
\pgfpathlineto{\pgfqpoint{5.607660in}{2.898859in}}%
\pgfpathlineto{\pgfqpoint{5.610832in}{2.898863in}}%
\pgfpathlineto{\pgfqpoint{5.614004in}{2.898702in}}%
\pgfpathlineto{\pgfqpoint{5.617176in}{2.898741in}}%
\pgfpathlineto{\pgfqpoint{5.620348in}{2.898489in}}%
\pgfpathlineto{\pgfqpoint{5.623520in}{2.898751in}}%
\pgfpathlineto{\pgfqpoint{5.626692in}{2.899189in}}%
\pgfpathlineto{\pgfqpoint{5.629864in}{2.899199in}}%
\pgfpathlineto{\pgfqpoint{5.633037in}{2.899382in}}%
\pgfpathlineto{\pgfqpoint{5.636209in}{2.899698in}}%
\pgfpathlineto{\pgfqpoint{5.639381in}{2.899638in}}%
\pgfpathlineto{\pgfqpoint{5.642553in}{2.899101in}}%
\pgfpathlineto{\pgfqpoint{5.645725in}{2.899186in}}%
\pgfpathlineto{\pgfqpoint{5.648897in}{2.899249in}}%
\pgfpathlineto{\pgfqpoint{5.652069in}{2.899039in}}%
\pgfpathlineto{\pgfqpoint{5.655241in}{2.898895in}}%
\pgfpathlineto{\pgfqpoint{5.658413in}{2.898774in}}%
\pgfpathlineto{\pgfqpoint{5.661585in}{2.898807in}}%
\pgfpathlineto{\pgfqpoint{5.664757in}{2.898638in}}%
\pgfpathlineto{\pgfqpoint{5.667929in}{2.898411in}}%
\pgfpathlineto{\pgfqpoint{5.671101in}{2.898277in}}%
\pgfpathlineto{\pgfqpoint{5.674273in}{2.898293in}}%
\pgfpathlineto{\pgfqpoint{5.677445in}{2.898205in}}%
\pgfpathlineto{\pgfqpoint{5.680617in}{2.898069in}}%
\pgfpathlineto{\pgfqpoint{5.683789in}{2.898175in}}%
\pgfpathlineto{\pgfqpoint{5.686961in}{2.898207in}}%
\pgfpathlineto{\pgfqpoint{5.690133in}{2.898090in}}%
\pgfpathlineto{\pgfqpoint{5.693305in}{2.898193in}}%
\pgfpathlineto{\pgfqpoint{5.696477in}{2.898181in}}%
\pgfpathlineto{\pgfqpoint{5.699649in}{2.898716in}}%
\pgfpathlineto{\pgfqpoint{5.702821in}{2.898684in}}%
\pgfpathlineto{\pgfqpoint{5.705993in}{2.898480in}}%
\pgfpathlineto{\pgfqpoint{5.709165in}{2.898524in}}%
\pgfpathlineto{\pgfqpoint{5.712338in}{2.898605in}}%
\pgfpathlineto{\pgfqpoint{5.715510in}{2.898270in}}%
\pgfpathlineto{\pgfqpoint{5.718682in}{2.897829in}}%
\pgfpathlineto{\pgfqpoint{5.721854in}{2.897740in}}%
\pgfpathlineto{\pgfqpoint{5.725026in}{2.896926in}}%
\pgfpathlineto{\pgfqpoint{5.728198in}{2.896926in}}%
\pgfpathlineto{\pgfqpoint{5.731370in}{2.896856in}}%
\pgfpathlineto{\pgfqpoint{5.734542in}{2.896649in}}%
\pgfpathlineto{\pgfqpoint{5.737714in}{2.896466in}}%
\pgfpathlineto{\pgfqpoint{5.740886in}{2.896424in}}%
\pgfpathlineto{\pgfqpoint{5.744058in}{2.896376in}}%
\pgfpathlineto{\pgfqpoint{5.747230in}{2.896111in}}%
\pgfpathlineto{\pgfqpoint{5.750402in}{2.896271in}}%
\pgfpathlineto{\pgfqpoint{5.753574in}{2.896270in}}%
\pgfpathlineto{\pgfqpoint{5.756746in}{2.896249in}}%
\pgfpathlineto{\pgfqpoint{5.759918in}{2.896373in}}%
\pgfpathlineto{\pgfqpoint{5.763090in}{2.896401in}}%
\pgfpathlineto{\pgfqpoint{5.766262in}{2.896192in}}%
\pgfpathlineto{\pgfqpoint{5.769434in}{2.896454in}}%
\pgfpathlineto{\pgfqpoint{5.772606in}{2.896484in}}%
\pgfpathlineto{\pgfqpoint{5.775778in}{2.896069in}}%
\pgfpathlineto{\pgfqpoint{5.778950in}{2.895796in}}%
\pgfpathlineto{\pgfqpoint{5.782122in}{2.895241in}}%
\pgfpathlineto{\pgfqpoint{5.785294in}{2.894340in}}%
\pgfpathlineto{\pgfqpoint{5.788466in}{2.894231in}}%
\pgfpathlineto{\pgfqpoint{5.791639in}{2.894438in}}%
\pgfpathlineto{\pgfqpoint{5.794811in}{2.894239in}}%
\pgfpathlineto{\pgfqpoint{5.797983in}{2.893875in}}%
\pgfpathlineto{\pgfqpoint{5.801155in}{2.893906in}}%
\pgfpathlineto{\pgfqpoint{5.804327in}{2.894258in}}%
\pgfpathlineto{\pgfqpoint{5.807499in}{2.894302in}}%
\pgfpathlineto{\pgfqpoint{5.810671in}{2.893829in}}%
\pgfpathlineto{\pgfqpoint{5.813843in}{2.893984in}}%
\pgfpathlineto{\pgfqpoint{5.817015in}{2.893823in}}%
\pgfpathlineto{\pgfqpoint{5.820187in}{2.893699in}}%
\pgfpathlineto{\pgfqpoint{5.823359in}{2.893593in}}%
\pgfpathlineto{\pgfqpoint{5.826531in}{2.893671in}}%
\pgfpathlineto{\pgfqpoint{5.829703in}{2.893189in}}%
\pgfpathlineto{\pgfqpoint{5.832875in}{2.892648in}}%
\pgfpathlineto{\pgfqpoint{5.836047in}{2.891812in}}%
\pgfpathlineto{\pgfqpoint{5.839219in}{2.891817in}}%
\pgfpathlineto{\pgfqpoint{5.842391in}{2.891615in}}%
\pgfpathlineto{\pgfqpoint{5.845563in}{2.891308in}}%
\pgfpathlineto{\pgfqpoint{5.848735in}{2.891040in}}%
\pgfpathlineto{\pgfqpoint{5.851907in}{2.890855in}}%
\pgfpathlineto{\pgfqpoint{5.855079in}{2.891128in}}%
\pgfpathlineto{\pgfqpoint{5.858251in}{2.891210in}}%
\pgfpathlineto{\pgfqpoint{5.861423in}{2.891344in}}%
\pgfpathlineto{\pgfqpoint{5.864595in}{2.891570in}}%
\pgfpathlineto{\pgfqpoint{5.867768in}{2.891459in}}%
\pgfpathlineto{\pgfqpoint{5.870940in}{2.891680in}}%
\pgfpathlineto{\pgfqpoint{5.874112in}{2.891907in}}%
\pgfpathlineto{\pgfqpoint{5.877284in}{2.891666in}}%
\pgfpathlineto{\pgfqpoint{5.880456in}{2.891353in}}%
\pgfpathlineto{\pgfqpoint{5.883628in}{2.891231in}}%
\pgfpathlineto{\pgfqpoint{5.886800in}{2.890946in}}%
\pgfpathlineto{\pgfqpoint{5.889972in}{2.891068in}}%
\pgfpathlineto{\pgfqpoint{5.893144in}{2.890963in}}%
\pgfpathlineto{\pgfqpoint{5.896316in}{2.890997in}}%
\pgfpathlineto{\pgfqpoint{5.899488in}{2.890978in}}%
\pgfpathlineto{\pgfqpoint{5.902660in}{2.891009in}}%
\pgfpathlineto{\pgfqpoint{5.905832in}{2.891164in}}%
\pgfpathlineto{\pgfqpoint{5.909004in}{2.890950in}}%
\pgfpathlineto{\pgfqpoint{5.912176in}{2.891160in}}%
\pgfpathlineto{\pgfqpoint{5.915348in}{2.891176in}}%
\pgfpathlineto{\pgfqpoint{5.918520in}{2.891173in}}%
\pgfpathlineto{\pgfqpoint{5.921692in}{2.891456in}}%
\pgfpathlineto{\pgfqpoint{5.924864in}{2.890953in}}%
\pgfpathlineto{\pgfqpoint{5.928036in}{2.890467in}}%
\pgfpathlineto{\pgfqpoint{5.931208in}{2.890420in}}%
\pgfpathlineto{\pgfqpoint{5.934380in}{2.890224in}}%
\pgfpathlineto{\pgfqpoint{5.937552in}{2.890046in}}%
\pgfpathlineto{\pgfqpoint{5.940724in}{2.889917in}}%
\pgfpathlineto{\pgfqpoint{5.943896in}{2.890149in}}%
\pgfpathlineto{\pgfqpoint{5.947069in}{2.889790in}}%
\pgfpathlineto{\pgfqpoint{5.950241in}{2.889319in}}%
\pgfpathlineto{\pgfqpoint{5.953413in}{2.889215in}}%
\pgfpathlineto{\pgfqpoint{5.956585in}{2.888803in}}%
\pgfpathlineto{\pgfqpoint{5.959757in}{2.888406in}}%
\pgfpathlineto{\pgfqpoint{5.962929in}{2.888244in}}%
\pgfpathlineto{\pgfqpoint{5.966101in}{2.887687in}}%
\pgfpathlineto{\pgfqpoint{5.969273in}{2.888075in}}%
\pgfpathlineto{\pgfqpoint{5.972445in}{2.888065in}}%
\pgfpathlineto{\pgfqpoint{5.975617in}{2.887547in}}%
\pgfpathlineto{\pgfqpoint{5.978789in}{2.887164in}}%
\pgfpathlineto{\pgfqpoint{5.981961in}{2.886754in}}%
\pgfpathlineto{\pgfqpoint{5.985133in}{2.886853in}}%
\pgfpathlineto{\pgfqpoint{5.988305in}{2.886386in}}%
\pgfpathlineto{\pgfqpoint{5.991477in}{2.885214in}}%
\pgfpathlineto{\pgfqpoint{5.994649in}{2.885129in}}%
\pgfpathlineto{\pgfqpoint{5.997821in}{2.884037in}}%
\pgfpathlineto{\pgfqpoint{6.000993in}{2.883984in}}%
\pgfpathlineto{\pgfqpoint{6.004165in}{2.884116in}}%
\pgfpathlineto{\pgfqpoint{6.007337in}{2.884029in}}%
\pgfpathlineto{\pgfqpoint{6.010509in}{2.883985in}}%
\pgfpathlineto{\pgfqpoint{6.013681in}{2.883958in}}%
\pgfpathlineto{\pgfqpoint{6.016853in}{2.883938in}}%
\pgfpathlineto{\pgfqpoint{6.020025in}{2.883571in}}%
\pgfpathlineto{\pgfqpoint{6.023197in}{2.883530in}}%
\pgfpathlineto{\pgfqpoint{6.026370in}{2.883132in}}%
\pgfpathlineto{\pgfqpoint{6.029542in}{2.882810in}}%
\pgfpathlineto{\pgfqpoint{6.032714in}{2.882626in}}%
\pgfpathlineto{\pgfqpoint{6.035886in}{2.882624in}}%
\pgfpathlineto{\pgfqpoint{6.039058in}{2.882731in}}%
\pgfpathlineto{\pgfqpoint{6.042230in}{2.882333in}}%
\pgfpathlineto{\pgfqpoint{6.045402in}{2.882640in}}%
\pgfpathlineto{\pgfqpoint{6.048574in}{2.882172in}}%
\pgfpathlineto{\pgfqpoint{6.051746in}{2.881906in}}%
\pgfpathlineto{\pgfqpoint{6.054918in}{2.881836in}}%
\pgfpathlineto{\pgfqpoint{6.058090in}{2.881634in}}%
\pgfpathlineto{\pgfqpoint{6.061262in}{2.881309in}}%
\pgfpathlineto{\pgfqpoint{6.064434in}{2.881061in}}%
\pgfpathlineto{\pgfqpoint{6.067606in}{2.881292in}}%
\pgfpathlineto{\pgfqpoint{6.070778in}{2.880644in}}%
\pgfpathlineto{\pgfqpoint{6.073950in}{2.880897in}}%
\pgfpathlineto{\pgfqpoint{6.077122in}{2.880812in}}%
\pgfpathlineto{\pgfqpoint{6.080294in}{2.880553in}}%
\pgfpathlineto{\pgfqpoint{6.083466in}{2.880332in}}%
\pgfpathlineto{\pgfqpoint{6.086638in}{2.879813in}}%
\pgfpathlineto{\pgfqpoint{6.089810in}{2.879393in}}%
\pgfpathlineto{\pgfqpoint{6.092982in}{2.879101in}}%
\pgfpathlineto{\pgfqpoint{6.096154in}{2.879036in}}%
\pgfpathlineto{\pgfqpoint{6.099326in}{2.878902in}}%
\pgfpathlineto{\pgfqpoint{6.102499in}{2.878632in}}%
\pgfpathlineto{\pgfqpoint{6.105671in}{2.878310in}}%
\pgfpathlineto{\pgfqpoint{6.108843in}{2.877844in}}%
\pgfpathlineto{\pgfqpoint{6.112015in}{2.878027in}}%
\pgfpathlineto{\pgfqpoint{6.115187in}{2.877744in}}%
\pgfpathlineto{\pgfqpoint{6.118359in}{2.877831in}}%
\pgfpathlineto{\pgfqpoint{6.121531in}{2.877415in}}%
\pgfpathlineto{\pgfqpoint{6.124703in}{2.877193in}}%
\pgfpathlineto{\pgfqpoint{6.127875in}{2.876954in}}%
\pgfpathlineto{\pgfqpoint{6.131047in}{2.876971in}}%
\pgfpathlineto{\pgfqpoint{6.134219in}{2.876727in}}%
\pgfpathlineto{\pgfqpoint{6.137391in}{2.876919in}}%
\pgfpathlineto{\pgfqpoint{6.140563in}{2.876533in}}%
\pgfpathlineto{\pgfqpoint{6.143735in}{2.876426in}}%
\pgfpathlineto{\pgfqpoint{6.146907in}{2.875885in}}%
\pgfpathlineto{\pgfqpoint{6.150079in}{2.876091in}}%
\pgfpathlineto{\pgfqpoint{6.153251in}{2.875565in}}%
\pgfpathlineto{\pgfqpoint{6.156423in}{2.875767in}}%
\pgfpathlineto{\pgfqpoint{6.159595in}{2.875540in}}%
\pgfpathlineto{\pgfqpoint{6.162767in}{2.875509in}}%
\pgfpathlineto{\pgfqpoint{6.165939in}{2.875294in}}%
\pgfpathlineto{\pgfqpoint{6.169111in}{2.874755in}}%
\pgfpathlineto{\pgfqpoint{6.172283in}{2.874691in}}%
\pgfpathlineto{\pgfqpoint{6.175455in}{2.874701in}}%
\pgfpathlineto{\pgfqpoint{6.178627in}{2.874099in}}%
\pgfpathlineto{\pgfqpoint{6.181800in}{2.873882in}}%
\pgfpathlineto{\pgfqpoint{6.184972in}{2.873573in}}%
\pgfpathlineto{\pgfqpoint{6.188144in}{2.873527in}}%
\pgfpathlineto{\pgfqpoint{6.191316in}{2.873516in}}%
\pgfpathlineto{\pgfqpoint{6.194488in}{2.873159in}}%
\pgfpathlineto{\pgfqpoint{6.197660in}{2.873332in}}%
\pgfpathlineto{\pgfqpoint{6.200832in}{2.873583in}}%
\pgfpathlineto{\pgfqpoint{6.204004in}{2.873682in}}%
\pgfpathlineto{\pgfqpoint{6.207176in}{2.873569in}}%
\pgfpathlineto{\pgfqpoint{6.210348in}{2.873628in}}%
\pgfpathlineto{\pgfqpoint{6.213520in}{2.873118in}}%
\pgfpathlineto{\pgfqpoint{6.216692in}{2.873106in}}%
\pgfpathlineto{\pgfqpoint{6.219864in}{2.873227in}}%
\pgfpathlineto{\pgfqpoint{6.223036in}{2.873161in}}%
\pgfpathlineto{\pgfqpoint{6.226208in}{2.873169in}}%
\pgfpathlineto{\pgfqpoint{6.229380in}{2.873053in}}%
\pgfpathlineto{\pgfqpoint{6.232552in}{2.872881in}}%
\pgfpathlineto{\pgfqpoint{6.235724in}{2.872751in}}%
\pgfpathlineto{\pgfqpoint{6.238896in}{2.872321in}}%
\pgfpathlineto{\pgfqpoint{6.242068in}{2.872106in}}%
\pgfpathlineto{\pgfqpoint{6.245240in}{2.871925in}}%
\pgfpathlineto{\pgfqpoint{6.248412in}{2.871969in}}%
\pgfpathlineto{\pgfqpoint{6.251584in}{2.871385in}}%
\pgfpathlineto{\pgfqpoint{6.254756in}{2.871421in}}%
\pgfpathlineto{\pgfqpoint{6.257928in}{2.871571in}}%
\pgfpathlineto{\pgfqpoint{6.261101in}{2.871702in}}%
\pgfpathlineto{\pgfqpoint{6.264273in}{2.871239in}}%
\pgfpathlineto{\pgfqpoint{6.267445in}{2.871165in}}%
\pgfpathlineto{\pgfqpoint{6.270617in}{2.870871in}}%
\pgfpathlineto{\pgfqpoint{6.273789in}{2.870314in}}%
\pgfpathlineto{\pgfqpoint{6.276961in}{2.870728in}}%
\pgfpathlineto{\pgfqpoint{6.280133in}{2.870743in}}%
\pgfpathlineto{\pgfqpoint{6.283305in}{2.870559in}}%
\pgfpathlineto{\pgfqpoint{6.286477in}{2.870681in}}%
\pgfpathlineto{\pgfqpoint{6.289649in}{2.870634in}}%
\pgfpathlineto{\pgfqpoint{6.292821in}{2.870300in}}%
\pgfpathlineto{\pgfqpoint{6.295993in}{2.870077in}}%
\pgfpathlineto{\pgfqpoint{6.299165in}{2.870282in}}%
\pgfpathlineto{\pgfqpoint{6.302337in}{2.870135in}}%
\pgfpathlineto{\pgfqpoint{6.305509in}{2.870051in}}%
\pgfpathlineto{\pgfqpoint{6.308681in}{2.869473in}}%
\pgfpathlineto{\pgfqpoint{6.311853in}{2.869090in}}%
\pgfpathlineto{\pgfqpoint{6.315025in}{2.868835in}}%
\pgfpathlineto{\pgfqpoint{6.318197in}{2.868407in}}%
\pgfpathlineto{\pgfqpoint{6.321369in}{2.868031in}}%
\pgfpathlineto{\pgfqpoint{6.324541in}{2.868071in}}%
\pgfpathlineto{\pgfqpoint{6.327713in}{2.868018in}}%
\pgfpathlineto{\pgfqpoint{6.330885in}{2.867889in}}%
\pgfpathlineto{\pgfqpoint{6.334057in}{2.867836in}}%
\pgfpathlineto{\pgfqpoint{6.337230in}{2.867860in}}%
\pgfpathlineto{\pgfqpoint{6.340402in}{2.867952in}}%
\pgfpathlineto{\pgfqpoint{6.343574in}{2.867787in}}%
\pgfpathlineto{\pgfqpoint{6.346746in}{2.868000in}}%
\pgfpathlineto{\pgfqpoint{6.349918in}{2.867700in}}%
\pgfpathlineto{\pgfqpoint{6.353090in}{2.867463in}}%
\pgfpathlineto{\pgfqpoint{6.356262in}{2.867447in}}%
\pgfpathlineto{\pgfqpoint{6.359434in}{2.867308in}}%
\pgfpathlineto{\pgfqpoint{6.362606in}{2.867122in}}%
\pgfpathlineto{\pgfqpoint{6.365778in}{2.867046in}}%
\pgfpathlineto{\pgfqpoint{6.368950in}{2.866995in}}%
\pgfpathlineto{\pgfqpoint{6.372122in}{2.866228in}}%
\pgfpathlineto{\pgfqpoint{6.375294in}{2.866095in}}%
\pgfpathlineto{\pgfqpoint{6.378466in}{2.866023in}}%
\pgfpathlineto{\pgfqpoint{6.381638in}{2.866122in}}%
\pgfpathlineto{\pgfqpoint{6.384810in}{2.865709in}}%
\pgfpathlineto{\pgfqpoint{6.387982in}{2.865375in}}%
\pgfpathlineto{\pgfqpoint{6.391154in}{2.865026in}}%
\pgfpathlineto{\pgfqpoint{6.394326in}{2.864373in}}%
\pgfpathlineto{\pgfqpoint{6.397498in}{2.864091in}}%
\pgfpathlineto{\pgfqpoint{6.400670in}{2.863570in}}%
\pgfpathlineto{\pgfqpoint{6.403842in}{2.863248in}}%
\pgfpathlineto{\pgfqpoint{6.407014in}{2.862973in}}%
\pgfpathlineto{\pgfqpoint{6.410186in}{2.862829in}}%
\pgfpathlineto{\pgfqpoint{6.413358in}{2.862309in}}%
\pgfpathlineto{\pgfqpoint{6.416531in}{2.862085in}}%
\pgfpathlineto{\pgfqpoint{6.419703in}{2.861755in}}%
\pgfpathlineto{\pgfqpoint{6.422875in}{2.861787in}}%
\pgfpathlineto{\pgfqpoint{6.426047in}{2.861656in}}%
\pgfpathlineto{\pgfqpoint{6.429219in}{2.861527in}}%
\pgfpathlineto{\pgfqpoint{6.432391in}{2.861598in}}%
\pgfpathlineto{\pgfqpoint{6.435563in}{2.861494in}}%
\pgfpathlineto{\pgfqpoint{6.438735in}{2.861183in}}%
\pgfpathlineto{\pgfqpoint{6.441907in}{2.860589in}}%
\pgfpathlineto{\pgfqpoint{6.445079in}{2.860476in}}%
\pgfpathlineto{\pgfqpoint{6.448251in}{2.860589in}}%
\pgfpathlineto{\pgfqpoint{6.451423in}{2.860242in}}%
\pgfpathlineto{\pgfqpoint{6.454595in}{2.860592in}}%
\pgfpathlineto{\pgfqpoint{6.457767in}{2.860625in}}%
\pgfpathlineto{\pgfqpoint{6.460939in}{2.860547in}}%
\pgfpathlineto{\pgfqpoint{6.464111in}{2.860889in}}%
\pgfpathlineto{\pgfqpoint{6.467283in}{2.860950in}}%
\pgfpathlineto{\pgfqpoint{6.470455in}{2.860662in}}%
\pgfpathlineto{\pgfqpoint{6.473627in}{2.860556in}}%
\pgfpathlineto{\pgfqpoint{6.476799in}{2.860274in}}%
\pgfpathlineto{\pgfqpoint{6.479971in}{2.860230in}}%
\pgfpathlineto{\pgfqpoint{6.483143in}{2.860225in}}%
\pgfpathlineto{\pgfqpoint{6.486315in}{2.860279in}}%
\pgfpathlineto{\pgfqpoint{6.489487in}{2.860234in}}%
\pgfpathlineto{\pgfqpoint{6.492659in}{2.860367in}}%
\pgfpathlineto{\pgfqpoint{6.495832in}{2.860269in}}%
\pgfpathlineto{\pgfqpoint{6.499004in}{2.860159in}}%
\pgfpathlineto{\pgfqpoint{6.502176in}{2.860139in}}%
\pgfpathlineto{\pgfqpoint{6.505348in}{2.860099in}}%
\pgfpathlineto{\pgfqpoint{6.508520in}{2.859752in}}%
\pgfpathlineto{\pgfqpoint{6.511692in}{2.859069in}}%
\pgfpathlineto{\pgfqpoint{6.514864in}{2.858905in}}%
\pgfpathlineto{\pgfqpoint{6.518036in}{2.858421in}}%
\pgfpathlineto{\pgfqpoint{6.521208in}{2.857986in}}%
\pgfpathlineto{\pgfqpoint{6.524380in}{2.857316in}}%
\pgfpathlineto{\pgfqpoint{6.527552in}{2.856857in}}%
\pgfpathlineto{\pgfqpoint{6.530724in}{2.856344in}}%
\pgfpathlineto{\pgfqpoint{6.533896in}{2.856621in}}%
\pgfpathlineto{\pgfqpoint{6.537068in}{2.856541in}}%
\pgfpathlineto{\pgfqpoint{6.540240in}{2.856210in}}%
\pgfpathlineto{\pgfqpoint{6.543412in}{2.855842in}}%
\pgfpathlineto{\pgfqpoint{6.546584in}{2.855179in}}%
\pgfpathlineto{\pgfqpoint{6.549756in}{2.855054in}}%
\pgfpathlineto{\pgfqpoint{6.552928in}{2.854982in}}%
\pgfpathlineto{\pgfqpoint{6.556100in}{2.854702in}}%
\pgfpathlineto{\pgfqpoint{6.559272in}{2.854553in}}%
\pgfpathlineto{\pgfqpoint{6.562444in}{2.854646in}}%
\pgfpathlineto{\pgfqpoint{6.565616in}{2.854920in}}%
\pgfpathlineto{\pgfqpoint{6.568788in}{2.854841in}}%
\pgfpathlineto{\pgfqpoint{6.571961in}{2.854582in}}%
\pgfpathlineto{\pgfqpoint{6.575133in}{2.854719in}}%
\pgfpathlineto{\pgfqpoint{6.578305in}{2.854583in}}%
\pgfpathlineto{\pgfqpoint{6.581477in}{2.854589in}}%
\pgfpathlineto{\pgfqpoint{6.584649in}{2.854675in}}%
\pgfpathlineto{\pgfqpoint{6.587821in}{2.855209in}}%
\pgfpathlineto{\pgfqpoint{6.590993in}{2.855100in}}%
\pgfpathlineto{\pgfqpoint{6.594165in}{2.854914in}}%
\pgfpathlineto{\pgfqpoint{6.597337in}{2.854779in}}%
\pgfpathlineto{\pgfqpoint{6.600509in}{2.854103in}}%
\pgfpathlineto{\pgfqpoint{6.603681in}{2.853963in}}%
\pgfpathlineto{\pgfqpoint{6.606853in}{2.853670in}}%
\pgfpathlineto{\pgfqpoint{6.610025in}{2.853303in}}%
\pgfpathlineto{\pgfqpoint{6.613197in}{2.852754in}}%
\pgfpathlineto{\pgfqpoint{6.616369in}{2.852313in}}%
\pgfpathlineto{\pgfqpoint{6.619541in}{2.852496in}}%
\pgfpathlineto{\pgfqpoint{6.622713in}{2.852420in}}%
\pgfpathlineto{\pgfqpoint{6.625885in}{2.852241in}}%
\pgfpathlineto{\pgfqpoint{6.629057in}{2.851991in}}%
\pgfpathlineto{\pgfqpoint{6.632229in}{2.851972in}}%
\pgfpathlineto{\pgfqpoint{6.635401in}{2.852214in}}%
\pgfpathlineto{\pgfqpoint{6.638573in}{2.851822in}}%
\pgfpathlineto{\pgfqpoint{6.641745in}{2.852256in}}%
\pgfpathlineto{\pgfqpoint{6.644917in}{2.852068in}}%
\pgfpathlineto{\pgfqpoint{6.648089in}{2.852012in}}%
\pgfpathlineto{\pgfqpoint{6.651262in}{2.851679in}}%
\pgfpathlineto{\pgfqpoint{6.654434in}{2.851801in}}%
\pgfpathlineto{\pgfqpoint{6.657606in}{2.851860in}}%
\pgfpathlineto{\pgfqpoint{6.660778in}{2.851569in}}%
\pgfpathlineto{\pgfqpoint{6.663950in}{2.851477in}}%
\pgfpathlineto{\pgfqpoint{6.667122in}{2.851286in}}%
\pgfpathlineto{\pgfqpoint{6.670294in}{2.851281in}}%
\pgfpathlineto{\pgfqpoint{6.673466in}{2.850965in}}%
\pgfpathlineto{\pgfqpoint{6.676638in}{2.851069in}}%
\pgfpathlineto{\pgfqpoint{6.679810in}{2.850967in}}%
\pgfpathlineto{\pgfqpoint{6.682982in}{2.851080in}}%
\pgfpathlineto{\pgfqpoint{6.686154in}{2.850797in}}%
\pgfpathlineto{\pgfqpoint{6.689326in}{2.850520in}}%
\pgfpathlineto{\pgfqpoint{6.692498in}{2.850597in}}%
\pgfpathlineto{\pgfqpoint{6.695670in}{2.850743in}}%
\pgfpathlineto{\pgfqpoint{6.698842in}{2.850664in}}%
\pgfpathlineto{\pgfqpoint{6.702014in}{2.850674in}}%
\pgfpathlineto{\pgfqpoint{6.705186in}{2.850171in}}%
\pgfpathlineto{\pgfqpoint{6.708358in}{2.850054in}}%
\pgfpathlineto{\pgfqpoint{6.711530in}{2.849511in}}%
\pgfpathlineto{\pgfqpoint{6.714702in}{2.849257in}}%
\pgfpathlineto{\pgfqpoint{6.717874in}{2.848897in}}%
\pgfpathlineto{\pgfqpoint{6.721046in}{2.848725in}}%
\pgfpathlineto{\pgfqpoint{6.724218in}{2.848384in}}%
\pgfpathlineto{\pgfqpoint{6.727391in}{2.847918in}}%
\pgfpathlineto{\pgfqpoint{6.730563in}{2.847617in}}%
\pgfpathlineto{\pgfqpoint{6.733735in}{2.847794in}}%
\pgfpathlineto{\pgfqpoint{6.736907in}{2.848265in}}%
\pgfpathlineto{\pgfqpoint{6.740079in}{2.848064in}}%
\pgfpathlineto{\pgfqpoint{6.743251in}{2.847594in}}%
\pgfpathlineto{\pgfqpoint{6.746423in}{2.847016in}}%
\pgfpathlineto{\pgfqpoint{6.749595in}{2.846683in}}%
\pgfpathlineto{\pgfqpoint{6.752767in}{2.846257in}}%
\pgfpathlineto{\pgfqpoint{6.755939in}{2.846075in}}%
\pgfpathlineto{\pgfqpoint{6.759111in}{2.846060in}}%
\pgfpathlineto{\pgfqpoint{6.762283in}{2.846698in}}%
\pgfpathlineto{\pgfqpoint{6.765455in}{2.846638in}}%
\pgfpathlineto{\pgfqpoint{6.768627in}{2.846273in}}%
\pgfpathlineto{\pgfqpoint{6.771799in}{2.846002in}}%
\pgfpathlineto{\pgfqpoint{6.774971in}{2.845852in}}%
\pgfpathlineto{\pgfqpoint{6.778143in}{2.845501in}}%
\pgfpathlineto{\pgfqpoint{6.781315in}{2.845376in}}%
\pgfpathlineto{\pgfqpoint{6.784487in}{2.845057in}}%
\pgfpathlineto{\pgfqpoint{6.787659in}{2.845219in}}%
\pgfpathlineto{\pgfqpoint{6.790831in}{2.845547in}}%
\pgfpathlineto{\pgfqpoint{6.794003in}{2.845643in}}%
\pgfpathlineto{\pgfqpoint{6.797175in}{2.845250in}}%
\pgfpathlineto{\pgfqpoint{6.800347in}{2.844712in}}%
\pgfpathlineto{\pgfqpoint{6.803519in}{2.844700in}}%
\pgfpathlineto{\pgfqpoint{6.806692in}{2.844566in}}%
\pgfpathlineto{\pgfqpoint{6.809864in}{2.844449in}}%
\pgfpathlineto{\pgfqpoint{6.813036in}{2.844166in}}%
\pgfpathlineto{\pgfqpoint{6.816208in}{2.843555in}}%
\pgfpathlineto{\pgfqpoint{6.819380in}{2.843523in}}%
\pgfpathlineto{\pgfqpoint{6.822552in}{2.843667in}}%
\pgfpathlineto{\pgfqpoint{6.825724in}{2.843777in}}%
\pgfpathlineto{\pgfqpoint{6.828896in}{2.843696in}}%
\pgfpathlineto{\pgfqpoint{6.832068in}{2.843621in}}%
\pgfpathlineto{\pgfqpoint{6.835240in}{2.843501in}}%
\pgfpathlineto{\pgfqpoint{6.838412in}{2.843524in}}%
\pgfpathlineto{\pgfqpoint{6.841584in}{2.843200in}}%
\pgfpathlineto{\pgfqpoint{6.844756in}{2.842977in}}%
\pgfpathlineto{\pgfqpoint{6.847928in}{2.842585in}}%
\pgfpathlineto{\pgfqpoint{6.851100in}{2.842333in}}%
\pgfpathlineto{\pgfqpoint{6.854272in}{2.842359in}}%
\pgfpathlineto{\pgfqpoint{6.857444in}{2.841982in}}%
\pgfpathlineto{\pgfqpoint{6.860616in}{2.841430in}}%
\pgfpathlineto{\pgfqpoint{6.863788in}{2.841301in}}%
\pgfpathlineto{\pgfqpoint{6.866960in}{2.840981in}}%
\pgfpathlineto{\pgfqpoint{6.870132in}{2.840733in}}%
\pgfpathlineto{\pgfqpoint{6.873304in}{2.840147in}}%
\pgfpathlineto{\pgfqpoint{6.876476in}{2.839648in}}%
\pgfpathlineto{\pgfqpoint{6.879648in}{2.839694in}}%
\pgfpathlineto{\pgfqpoint{6.882820in}{2.839418in}}%
\pgfpathlineto{\pgfqpoint{6.885993in}{2.839389in}}%
\pgfpathlineto{\pgfqpoint{6.889165in}{2.839256in}}%
\pgfpathlineto{\pgfqpoint{6.892337in}{2.839204in}}%
\pgfpathlineto{\pgfqpoint{6.895509in}{2.838841in}}%
\pgfpathlineto{\pgfqpoint{6.898681in}{2.839083in}}%
\pgfpathlineto{\pgfqpoint{6.901853in}{2.839140in}}%
\pgfpathlineto{\pgfqpoint{6.905025in}{2.838415in}}%
\pgfpathlineto{\pgfqpoint{6.908197in}{2.838347in}}%
\pgfpathlineto{\pgfqpoint{6.911369in}{2.837780in}}%
\pgfpathlineto{\pgfqpoint{6.914541in}{2.837734in}}%
\pgfpathlineto{\pgfqpoint{6.917713in}{2.837907in}}%
\pgfpathlineto{\pgfqpoint{6.920885in}{2.837803in}}%
\pgfpathlineto{\pgfqpoint{6.924057in}{2.837564in}}%
\pgfpathlineto{\pgfqpoint{6.927229in}{2.837411in}}%
\pgfpathlineto{\pgfqpoint{6.930401in}{2.837251in}}%
\pgfpathlineto{\pgfqpoint{6.933573in}{2.837195in}}%
\pgfpathlineto{\pgfqpoint{6.936745in}{2.836774in}}%
\pgfpathlineto{\pgfqpoint{6.939917in}{2.836518in}}%
\pgfpathlineto{\pgfqpoint{6.943089in}{2.836407in}}%
\pgfpathlineto{\pgfqpoint{6.946261in}{2.835770in}}%
\pgfpathlineto{\pgfqpoint{6.949433in}{2.835712in}}%
\pgfpathlineto{\pgfqpoint{6.952605in}{2.835609in}}%
\pgfpathlineto{\pgfqpoint{6.955777in}{2.835720in}}%
\pgfpathlineto{\pgfqpoint{6.958949in}{2.836008in}}%
\pgfpathlineto{\pgfqpoint{6.962122in}{2.835827in}}%
\pgfpathlineto{\pgfqpoint{6.965294in}{2.835923in}}%
\pgfpathlineto{\pgfqpoint{6.968466in}{2.835516in}}%
\pgfpathlineto{\pgfqpoint{6.971638in}{2.835620in}}%
\pgfpathlineto{\pgfqpoint{6.974810in}{2.835280in}}%
\pgfpathlineto{\pgfqpoint{6.977982in}{2.835467in}}%
\pgfpathlineto{\pgfqpoint{6.981154in}{2.835284in}}%
\pgfpathlineto{\pgfqpoint{6.984326in}{2.835410in}}%
\pgfpathlineto{\pgfqpoint{6.987498in}{2.835171in}}%
\pgfpathlineto{\pgfqpoint{6.990670in}{2.834819in}}%
\pgfpathlineto{\pgfqpoint{6.993842in}{2.834722in}}%
\pgfpathlineto{\pgfqpoint{6.997014in}{2.834733in}}%
\pgfpathlineto{\pgfqpoint{7.000186in}{2.834708in}}%
\pgfpathlineto{\pgfqpoint{7.003358in}{2.834616in}}%
\pgfpathlineto{\pgfqpoint{7.006530in}{2.834436in}}%
\pgfpathlineto{\pgfqpoint{7.009702in}{2.834263in}}%
\pgfpathlineto{\pgfqpoint{7.012874in}{2.833783in}}%
\pgfpathlineto{\pgfqpoint{7.016046in}{2.833500in}}%
\pgfpathlineto{\pgfqpoint{7.019218in}{2.832976in}}%
\pgfpathlineto{\pgfqpoint{7.022390in}{2.833067in}}%
\pgfpathlineto{\pgfqpoint{7.025562in}{2.832966in}}%
\pgfpathlineto{\pgfqpoint{7.028734in}{2.833506in}}%
\pgfpathlineto{\pgfqpoint{7.031906in}{2.833406in}}%
\pgfpathlineto{\pgfqpoint{7.035078in}{2.833707in}}%
\pgfpathlineto{\pgfqpoint{7.038250in}{2.833469in}}%
\pgfpathlineto{\pgfqpoint{7.041423in}{2.833925in}}%
\pgfpathlineto{\pgfqpoint{7.044595in}{2.833562in}}%
\pgfpathlineto{\pgfqpoint{7.047767in}{2.833645in}}%
\pgfpathlineto{\pgfqpoint{7.050939in}{2.833907in}}%
\pgfpathlineto{\pgfqpoint{7.054111in}{2.833627in}}%
\pgfpathlineto{\pgfqpoint{7.057283in}{2.833688in}}%
\pgfpathlineto{\pgfqpoint{7.060455in}{2.833542in}}%
\pgfpathlineto{\pgfqpoint{7.063627in}{2.833613in}}%
\pgfpathlineto{\pgfqpoint{7.066799in}{2.833341in}}%
\pgfpathlineto{\pgfqpoint{7.069971in}{2.832943in}}%
\pgfpathlineto{\pgfqpoint{7.073143in}{2.832824in}}%
\pgfpathlineto{\pgfqpoint{7.076315in}{2.832619in}}%
\pgfpathlineto{\pgfqpoint{7.079487in}{2.832449in}}%
\pgfpathlineto{\pgfqpoint{7.082659in}{2.832482in}}%
\pgfpathlineto{\pgfqpoint{7.085831in}{2.832488in}}%
\pgfpathlineto{\pgfqpoint{7.089003in}{2.832173in}}%
\pgfpathlineto{\pgfqpoint{7.092175in}{2.832081in}}%
\pgfpathlineto{\pgfqpoint{7.095347in}{2.831521in}}%
\pgfpathlineto{\pgfqpoint{7.098519in}{2.831096in}}%
\pgfpathlineto{\pgfqpoint{7.101691in}{2.831221in}}%
\pgfpathlineto{\pgfqpoint{7.104863in}{2.831252in}}%
\pgfpathlineto{\pgfqpoint{7.108035in}{2.831599in}}%
\pgfpathlineto{\pgfqpoint{7.111207in}{2.831091in}}%
\pgfpathlineto{\pgfqpoint{7.114379in}{2.831069in}}%
\pgfpathlineto{\pgfqpoint{7.117551in}{2.831324in}}%
\pgfpathlineto{\pgfqpoint{7.120724in}{2.830889in}}%
\pgfpathlineto{\pgfqpoint{7.123896in}{2.830779in}}%
\pgfpathlineto{\pgfqpoint{7.127068in}{2.830490in}}%
\pgfpathlineto{\pgfqpoint{7.130240in}{2.830321in}}%
\pgfpathlineto{\pgfqpoint{7.133412in}{2.830010in}}%
\pgfpathlineto{\pgfqpoint{7.136584in}{2.829209in}}%
\pgfpathlineto{\pgfqpoint{7.139756in}{2.829142in}}%
\pgfpathlineto{\pgfqpoint{7.142928in}{2.829586in}}%
\pgfpathlineto{\pgfqpoint{7.146100in}{2.829054in}}%
\pgfpathlineto{\pgfqpoint{7.149272in}{2.829058in}}%
\pgfpathlineto{\pgfqpoint{7.152444in}{2.829244in}}%
\pgfpathlineto{\pgfqpoint{7.155616in}{2.829028in}}%
\pgfpathlineto{\pgfqpoint{7.158788in}{2.828738in}}%
\pgfpathlineto{\pgfqpoint{7.161960in}{2.828503in}}%
\pgfpathlineto{\pgfqpoint{7.165132in}{2.828461in}}%
\pgfpathlineto{\pgfqpoint{7.168304in}{2.828217in}}%
\pgfpathlineto{\pgfqpoint{7.171476in}{2.828064in}}%
\pgfpathlineto{\pgfqpoint{7.174648in}{2.827750in}}%
\pgfpathlineto{\pgfqpoint{7.177820in}{2.827402in}}%
\pgfpathlineto{\pgfqpoint{7.180992in}{2.827126in}}%
\pgfpathlineto{\pgfqpoint{7.184164in}{2.827003in}}%
\pgfpathlineto{\pgfqpoint{7.187336in}{2.826606in}}%
\pgfpathlineto{\pgfqpoint{7.190508in}{2.826783in}}%
\pgfpathlineto{\pgfqpoint{7.193680in}{2.827060in}}%
\pgfpathlineto{\pgfqpoint{7.196853in}{2.826892in}}%
\pgfpathlineto{\pgfqpoint{7.200025in}{2.827110in}}%
\pgfpathlineto{\pgfqpoint{7.203197in}{2.827314in}}%
\pgfpathlineto{\pgfqpoint{7.206369in}{2.827325in}}%
\pgfpathlineto{\pgfqpoint{7.209541in}{2.827319in}}%
\pgfpathlineto{\pgfqpoint{7.212713in}{2.826676in}}%
\pgfpathlineto{\pgfqpoint{7.215885in}{2.826345in}}%
\pgfpathlineto{\pgfqpoint{7.219057in}{2.826249in}}%
\pgfpathlineto{\pgfqpoint{7.222229in}{2.825694in}}%
\pgfpathlineto{\pgfqpoint{7.225401in}{2.825652in}}%
\pgfpathlineto{\pgfqpoint{7.228573in}{2.825307in}}%
\pgfpathlineto{\pgfqpoint{7.231745in}{2.825149in}}%
\pgfpathlineto{\pgfqpoint{7.234917in}{2.824936in}}%
\pgfpathlineto{\pgfqpoint{7.238089in}{2.825318in}}%
\pgfpathlineto{\pgfqpoint{7.241261in}{2.825195in}}%
\pgfpathlineto{\pgfqpoint{7.244433in}{2.824897in}}%
\pgfpathlineto{\pgfqpoint{7.247605in}{2.824501in}}%
\pgfpathlineto{\pgfqpoint{7.250777in}{2.824246in}}%
\pgfpathlineto{\pgfqpoint{7.253949in}{2.824202in}}%
\pgfpathlineto{\pgfqpoint{7.257121in}{2.823870in}}%
\pgfpathlineto{\pgfqpoint{7.260293in}{2.823825in}}%
\pgfpathlineto{\pgfqpoint{7.263465in}{2.823395in}}%
\pgfpathlineto{\pgfqpoint{7.266637in}{2.823316in}}%
\pgfpathlineto{\pgfqpoint{7.269809in}{2.823525in}}%
\pgfpathlineto{\pgfqpoint{7.272981in}{2.823420in}}%
\pgfpathlineto{\pgfqpoint{7.276154in}{2.823255in}}%
\pgfpathlineto{\pgfqpoint{7.279326in}{2.822920in}}%
\pgfpathlineto{\pgfqpoint{7.282498in}{2.822979in}}%
\pgfpathlineto{\pgfqpoint{7.285670in}{2.822422in}}%
\pgfpathlineto{\pgfqpoint{7.288842in}{2.822101in}}%
\pgfpathlineto{\pgfqpoint{7.292014in}{2.821985in}}%
\pgfpathlineto{\pgfqpoint{7.295186in}{2.821753in}}%
\pgfpathlineto{\pgfqpoint{7.298358in}{2.821633in}}%
\pgfpathlineto{\pgfqpoint{7.301530in}{2.821561in}}%
\pgfpathlineto{\pgfqpoint{7.304702in}{2.821546in}}%
\pgfpathlineto{\pgfqpoint{7.307874in}{2.820435in}}%
\pgfpathlineto{\pgfqpoint{7.311046in}{2.819808in}}%
\pgfpathlineto{\pgfqpoint{7.314218in}{2.819437in}}%
\pgfpathlineto{\pgfqpoint{7.317390in}{2.819278in}}%
\pgfpathlineto{\pgfqpoint{7.320562in}{2.819029in}}%
\pgfpathlineto{\pgfqpoint{7.323734in}{2.818663in}}%
\pgfpathlineto{\pgfqpoint{7.326906in}{2.818416in}}%
\pgfpathlineto{\pgfqpoint{7.330078in}{2.817696in}}%
\pgfpathlineto{\pgfqpoint{7.333250in}{2.817347in}}%
\pgfpathlineto{\pgfqpoint{7.336422in}{2.817206in}}%
\pgfpathlineto{\pgfqpoint{7.339594in}{2.817298in}}%
\pgfpathlineto{\pgfqpoint{7.342766in}{2.816610in}}%
\pgfpathlineto{\pgfqpoint{7.345938in}{2.816423in}}%
\pgfpathlineto{\pgfqpoint{7.349110in}{2.816471in}}%
\pgfpathlineto{\pgfqpoint{7.352282in}{2.816327in}}%
\pgfpathlineto{\pgfqpoint{7.355455in}{2.816653in}}%
\pgfpathlineto{\pgfqpoint{7.358627in}{2.816725in}}%
\pgfpathlineto{\pgfqpoint{7.361799in}{2.816598in}}%
\pgfpathlineto{\pgfqpoint{7.364971in}{2.816522in}}%
\pgfpathlineto{\pgfqpoint{7.368143in}{2.816364in}}%
\pgfpathlineto{\pgfqpoint{7.371315in}{2.816319in}}%
\pgfpathlineto{\pgfqpoint{7.374487in}{2.816016in}}%
\pgfpathlineto{\pgfqpoint{7.377659in}{2.815715in}}%
\pgfpathlineto{\pgfqpoint{7.380831in}{2.815850in}}%
\pgfpathlineto{\pgfqpoint{7.384003in}{2.815120in}}%
\pgfpathlineto{\pgfqpoint{7.387175in}{2.814832in}}%
\pgfpathlineto{\pgfqpoint{7.390347in}{2.815040in}}%
\pgfpathlineto{\pgfqpoint{7.393519in}{2.814541in}}%
\pgfpathlineto{\pgfqpoint{7.396691in}{2.814348in}}%
\pgfpathlineto{\pgfqpoint{7.399863in}{2.813922in}}%
\pgfpathlineto{\pgfqpoint{7.403035in}{2.813386in}}%
\pgfpathlineto{\pgfqpoint{7.406207in}{2.813335in}}%
\pgfpathlineto{\pgfqpoint{7.409379in}{2.813390in}}%
\pgfpathlineto{\pgfqpoint{7.412551in}{2.812817in}}%
\pgfpathlineto{\pgfqpoint{7.415723in}{2.812812in}}%
\pgfpathlineto{\pgfqpoint{7.418895in}{2.812834in}}%
\pgfpathlineto{\pgfqpoint{7.422067in}{2.812854in}}%
\pgfpathlineto{\pgfqpoint{7.425239in}{2.812670in}}%
\pgfpathlineto{\pgfqpoint{7.428411in}{2.812628in}}%
\pgfpathlineto{\pgfqpoint{7.431584in}{2.812732in}}%
\pgfpathlineto{\pgfqpoint{7.434756in}{2.812750in}}%
\pgfpathlineto{\pgfqpoint{7.437928in}{2.812616in}}%
\pgfpathlineto{\pgfqpoint{7.441100in}{2.812635in}}%
\pgfpathlineto{\pgfqpoint{7.444272in}{2.812536in}}%
\pgfpathlineto{\pgfqpoint{7.447444in}{2.812270in}}%
\pgfpathlineto{\pgfqpoint{7.450616in}{2.812415in}}%
\pgfpathlineto{\pgfqpoint{7.453788in}{2.812336in}}%
\pgfpathlineto{\pgfqpoint{7.456960in}{2.812146in}}%
\pgfpathlineto{\pgfqpoint{7.460132in}{2.812164in}}%
\pgfpathlineto{\pgfqpoint{7.463304in}{2.812311in}}%
\pgfpathlineto{\pgfqpoint{7.466476in}{2.812223in}}%
\pgfpathlineto{\pgfqpoint{7.469648in}{2.811851in}}%
\pgfpathlineto{\pgfqpoint{7.472820in}{2.811919in}}%
\pgfpathlineto{\pgfqpoint{7.475992in}{2.811935in}}%
\pgfpathlineto{\pgfqpoint{7.479164in}{2.811962in}}%
\pgfpathlineto{\pgfqpoint{7.482336in}{2.811704in}}%
\pgfpathlineto{\pgfqpoint{7.485508in}{2.811586in}}%
\pgfpathlineto{\pgfqpoint{7.488680in}{2.811388in}}%
\pgfpathlineto{\pgfqpoint{7.491852in}{2.811416in}}%
\pgfpathlineto{\pgfqpoint{7.495024in}{2.811643in}}%
\pgfpathlineto{\pgfqpoint{7.498196in}{2.811524in}}%
\pgfpathlineto{\pgfqpoint{7.501368in}{2.811538in}}%
\pgfpathlineto{\pgfqpoint{7.504540in}{2.811512in}}%
\pgfpathlineto{\pgfqpoint{7.507712in}{2.811495in}}%
\pgfpathlineto{\pgfqpoint{7.510885in}{2.810937in}}%
\pgfpathlineto{\pgfqpoint{7.514057in}{2.810763in}}%
\pgfpathlineto{\pgfqpoint{7.517229in}{2.809895in}}%
\pgfpathlineto{\pgfqpoint{7.520401in}{2.809116in}}%
\pgfpathlineto{\pgfqpoint{7.523573in}{2.808937in}}%
\pgfpathlineto{\pgfqpoint{7.526745in}{2.808554in}}%
\pgfpathlineto{\pgfqpoint{7.529917in}{2.808028in}}%
\pgfpathlineto{\pgfqpoint{7.533089in}{2.807510in}}%
\pgfpathlineto{\pgfqpoint{7.536261in}{2.807228in}}%
\pgfpathlineto{\pgfqpoint{7.539433in}{2.807915in}}%
\pgfpathlineto{\pgfqpoint{7.542605in}{2.808120in}}%
\pgfpathlineto{\pgfqpoint{7.545777in}{2.808037in}}%
\pgfpathlineto{\pgfqpoint{7.548949in}{2.807841in}}%
\pgfpathlineto{\pgfqpoint{7.552121in}{2.807775in}}%
\pgfpathlineto{\pgfqpoint{7.555293in}{2.807823in}}%
\pgfpathlineto{\pgfqpoint{7.558465in}{2.807457in}}%
\pgfpathlineto{\pgfqpoint{7.561637in}{2.806906in}}%
\pgfpathlineto{\pgfqpoint{7.564809in}{2.806763in}}%
\pgfpathlineto{\pgfqpoint{7.567981in}{2.806196in}}%
\pgfpathlineto{\pgfqpoint{7.571153in}{2.805893in}}%
\pgfpathlineto{\pgfqpoint{7.574325in}{2.806081in}}%
\pgfpathlineto{\pgfqpoint{7.577497in}{2.806126in}}%
\pgfpathlineto{\pgfqpoint{7.580669in}{2.806513in}}%
\pgfpathlineto{\pgfqpoint{7.583841in}{2.806541in}}%
\pgfpathlineto{\pgfqpoint{7.587013in}{2.806279in}}%
\pgfpathlineto{\pgfqpoint{7.590186in}{2.806112in}}%
\pgfpathlineto{\pgfqpoint{7.593358in}{2.805695in}}%
\pgfpathlineto{\pgfqpoint{7.596530in}{2.805564in}}%
\pgfpathlineto{\pgfqpoint{7.599702in}{2.805490in}}%
\pgfpathlineto{\pgfqpoint{7.602874in}{2.805397in}}%
\pgfpathlineto{\pgfqpoint{7.606046in}{2.805530in}}%
\pgfpathlineto{\pgfqpoint{7.609218in}{2.805148in}}%
\pgfpathlineto{\pgfqpoint{7.612390in}{2.805091in}}%
\pgfpathlineto{\pgfqpoint{7.615562in}{2.804746in}}%
\pgfpathlineto{\pgfqpoint{7.618734in}{2.804195in}}%
\pgfpathlineto{\pgfqpoint{7.621906in}{2.804320in}}%
\pgfpathlineto{\pgfqpoint{7.625078in}{2.804082in}}%
\pgfpathlineto{\pgfqpoint{7.628250in}{2.803864in}}%
\pgfpathlineto{\pgfqpoint{7.631422in}{2.803719in}}%
\pgfpathlineto{\pgfqpoint{7.634594in}{2.803836in}}%
\pgfpathlineto{\pgfqpoint{7.637766in}{2.803435in}}%
\pgfpathlineto{\pgfqpoint{7.640938in}{2.803525in}}%
\pgfpathlineto{\pgfqpoint{7.644110in}{2.803302in}}%
\pgfpathlineto{\pgfqpoint{7.647282in}{2.803144in}}%
\pgfpathlineto{\pgfqpoint{7.650454in}{2.803091in}}%
\pgfpathlineto{\pgfqpoint{7.653626in}{2.802784in}}%
\pgfpathlineto{\pgfqpoint{7.656798in}{2.802726in}}%
\pgfpathlineto{\pgfqpoint{7.659970in}{2.802867in}}%
\pgfpathlineto{\pgfqpoint{7.663142in}{2.802478in}}%
\pgfpathlineto{\pgfqpoint{7.666315in}{2.802495in}}%
\pgfpathlineto{\pgfqpoint{7.669487in}{2.802727in}}%
\pgfpathlineto{\pgfqpoint{7.672659in}{2.802499in}}%
\pgfpathlineto{\pgfqpoint{7.675831in}{2.802512in}}%
\pgfpathlineto{\pgfqpoint{7.679003in}{2.802756in}}%
\pgfpathlineto{\pgfqpoint{7.682175in}{2.803220in}}%
\pgfpathlineto{\pgfqpoint{7.685347in}{2.803080in}}%
\pgfpathlineto{\pgfqpoint{7.688519in}{2.802889in}}%
\pgfpathlineto{\pgfqpoint{7.691691in}{2.802765in}}%
\pgfpathlineto{\pgfqpoint{7.694863in}{2.803032in}}%
\pgfpathlineto{\pgfqpoint{7.698035in}{2.803070in}}%
\pgfpathlineto{\pgfqpoint{7.701207in}{2.803220in}}%
\pgfpathlineto{\pgfqpoint{7.704379in}{2.803127in}}%
\pgfpathlineto{\pgfqpoint{7.707551in}{2.802795in}}%
\pgfpathlineto{\pgfqpoint{7.710723in}{2.802089in}}%
\pgfpathlineto{\pgfqpoint{7.713895in}{2.801908in}}%
\pgfpathlineto{\pgfqpoint{7.717067in}{2.801956in}}%
\pgfpathlineto{\pgfqpoint{7.720239in}{2.801636in}}%
\pgfpathlineto{\pgfqpoint{7.723411in}{2.802050in}}%
\pgfpathlineto{\pgfqpoint{7.726583in}{2.802124in}}%
\pgfpathlineto{\pgfqpoint{7.729755in}{2.802056in}}%
\pgfpathlineto{\pgfqpoint{7.732927in}{2.801704in}}%
\pgfpathlineto{\pgfqpoint{7.736099in}{2.801174in}}%
\pgfpathlineto{\pgfqpoint{7.739271in}{2.801075in}}%
\pgfpathlineto{\pgfqpoint{7.742443in}{2.800572in}}%
\pgfpathlineto{\pgfqpoint{7.745616in}{2.800565in}}%
\pgfpathlineto{\pgfqpoint{7.748788in}{2.800542in}}%
\pgfpathlineto{\pgfqpoint{7.751960in}{2.799659in}}%
\pgfpathlineto{\pgfqpoint{7.755132in}{2.799519in}}%
\pgfpathlineto{\pgfqpoint{7.758304in}{2.799626in}}%
\pgfpathlineto{\pgfqpoint{7.761476in}{2.799667in}}%
\pgfpathlineto{\pgfqpoint{7.764648in}{2.799204in}}%
\pgfpathlineto{\pgfqpoint{7.767820in}{2.799270in}}%
\pgfpathlineto{\pgfqpoint{7.770992in}{2.799266in}}%
\pgfpathlineto{\pgfqpoint{7.774164in}{2.799275in}}%
\pgfpathlineto{\pgfqpoint{7.777336in}{2.799262in}}%
\pgfpathlineto{\pgfqpoint{7.780508in}{2.799583in}}%
\pgfpathlineto{\pgfqpoint{7.783680in}{2.799761in}}%
\pgfpathlineto{\pgfqpoint{7.786852in}{2.799763in}}%
\pgfpathlineto{\pgfqpoint{7.790024in}{2.803280in}}%
\pgfpathlineto{\pgfqpoint{7.793196in}{2.806339in}}%
\pgfpathlineto{\pgfqpoint{7.796368in}{2.809341in}}%
\pgfpathlineto{\pgfqpoint{7.799540in}{2.812388in}}%
\pgfpathlineto{\pgfqpoint{7.802712in}{2.815405in}}%
\pgfpathlineto{\pgfqpoint{7.805884in}{2.818458in}}%
\pgfpathlineto{\pgfqpoint{7.809056in}{2.821430in}}%
\pgfpathlineto{\pgfqpoint{7.812228in}{2.824485in}}%
\pgfpathlineto{\pgfqpoint{7.815400in}{2.827507in}}%
\pgfpathlineto{\pgfqpoint{7.818572in}{2.830541in}}%
\pgfpathlineto{\pgfqpoint{7.821744in}{2.833584in}}%
\pgfpathlineto{\pgfqpoint{7.824917in}{2.836624in}}%
\pgfpathlineto{\pgfqpoint{7.828089in}{2.839607in}}%
\pgfpathlineto{\pgfqpoint{7.831261in}{2.842611in}}%
\pgfpathlineto{\pgfqpoint{7.834433in}{2.845631in}}%
\pgfpathlineto{\pgfqpoint{7.837605in}{2.848736in}}%
\pgfpathlineto{\pgfqpoint{7.840777in}{2.851739in}}%
\pgfpathlineto{\pgfqpoint{7.843949in}{2.854783in}}%
\pgfpathlineto{\pgfqpoint{7.847121in}{2.857810in}}%
\pgfpathlineto{\pgfqpoint{7.850293in}{2.860791in}}%
\pgfpathlineto{\pgfqpoint{7.853465in}{2.863818in}}%
\pgfpathlineto{\pgfqpoint{7.856637in}{2.866841in}}%
\pgfpathlineto{\pgfqpoint{7.859809in}{2.869829in}}%
\pgfpathlineto{\pgfqpoint{7.862981in}{2.872872in}}%
\pgfpathlineto{\pgfqpoint{7.866153in}{2.875938in}}%
\pgfpathlineto{\pgfqpoint{7.869325in}{2.878939in}}%
\pgfpathlineto{\pgfqpoint{7.872497in}{2.881962in}}%
\pgfpathlineto{\pgfqpoint{7.875669in}{2.884985in}}%
\pgfpathlineto{\pgfqpoint{7.878841in}{2.888026in}}%
\pgfpathlineto{\pgfqpoint{7.882013in}{2.891061in}}%
\pgfpathlineto{\pgfqpoint{7.885185in}{2.893988in}}%
\pgfpathlineto{\pgfqpoint{7.888357in}{2.896990in}}%
\pgfpathlineto{\pgfqpoint{7.891529in}{2.899958in}}%
\pgfpathlineto{\pgfqpoint{7.894701in}{2.902967in}}%
\pgfpathlineto{\pgfqpoint{7.897873in}{2.910584in}}%
\pgfpathlineto{\pgfqpoint{7.901046in}{2.913584in}}%
\pgfpathlineto{\pgfqpoint{7.904218in}{2.916589in}}%
\pgfpathlineto{\pgfqpoint{7.907390in}{2.919577in}}%
\pgfpathlineto{\pgfqpoint{7.910562in}{2.922636in}}%
\pgfpathlineto{\pgfqpoint{7.913734in}{2.925655in}}%
\pgfpathlineto{\pgfqpoint{7.916906in}{2.928680in}}%
\pgfpathlineto{\pgfqpoint{7.920078in}{2.931720in}}%
\pgfpathlineto{\pgfqpoint{7.923250in}{2.934786in}}%
\pgfpathlineto{\pgfqpoint{7.926422in}{2.937848in}}%
\pgfpathlineto{\pgfqpoint{7.929594in}{2.940860in}}%
\pgfpathlineto{\pgfqpoint{7.932766in}{2.943876in}}%
\pgfpathlineto{\pgfqpoint{7.935938in}{2.946903in}}%
\pgfpathlineto{\pgfqpoint{7.939110in}{2.949942in}}%
\pgfpathlineto{\pgfqpoint{7.942282in}{2.952974in}}%
\pgfpathlineto{\pgfqpoint{7.945454in}{2.955906in}}%
\pgfpathlineto{\pgfqpoint{7.948626in}{2.958845in}}%
\pgfpathlineto{\pgfqpoint{7.951798in}{2.961891in}}%
\pgfpathlineto{\pgfqpoint{7.954970in}{2.964892in}}%
\pgfpathlineto{\pgfqpoint{7.958142in}{2.967870in}}%
\pgfpathlineto{\pgfqpoint{7.961314in}{2.970893in}}%
\pgfpathlineto{\pgfqpoint{7.964486in}{2.973934in}}%
\pgfpathlineto{\pgfqpoint{7.967658in}{2.976967in}}%
\pgfpathlineto{\pgfqpoint{7.970830in}{2.980055in}}%
\pgfpathlineto{\pgfqpoint{7.974002in}{2.983080in}}%
\pgfpathlineto{\pgfqpoint{7.977174in}{2.986110in}}%
\pgfpathlineto{\pgfqpoint{7.980347in}{2.989157in}}%
\pgfpathlineto{\pgfqpoint{7.983519in}{2.992188in}}%
\pgfpathlineto{\pgfqpoint{7.986691in}{2.995218in}}%
\pgfpathlineto{\pgfqpoint{7.989863in}{2.998240in}}%
\pgfpathlineto{\pgfqpoint{7.993035in}{3.001281in}}%
\pgfpathlineto{\pgfqpoint{7.996207in}{3.004355in}}%
\pgfpathlineto{\pgfqpoint{7.999379in}{3.007242in}}%
\pgfpathlineto{\pgfqpoint{8.002551in}{3.010089in}}%
\pgfpathlineto{\pgfqpoint{8.005723in}{3.013124in}}%
\pgfpathlineto{\pgfqpoint{8.008895in}{3.016135in}}%
\pgfpathlineto{\pgfqpoint{8.012067in}{3.019155in}}%
\pgfpathlineto{\pgfqpoint{8.015239in}{3.022160in}}%
\pgfpathlineto{\pgfqpoint{8.018411in}{3.025116in}}%
\pgfpathlineto{\pgfqpoint{8.021583in}{3.028115in}}%
\pgfpathlineto{\pgfqpoint{8.024755in}{3.031200in}}%
\pgfpathlineto{\pgfqpoint{8.027927in}{3.034145in}}%
\pgfpathlineto{\pgfqpoint{8.031099in}{3.037122in}}%
\pgfpathlineto{\pgfqpoint{8.034271in}{3.040155in}}%
\pgfpathlineto{\pgfqpoint{8.037443in}{3.043201in}}%
\pgfpathlineto{\pgfqpoint{8.040615in}{3.046246in}}%
\pgfpathlineto{\pgfqpoint{8.043787in}{3.049260in}}%
\pgfpathlineto{\pgfqpoint{8.046959in}{3.052290in}}%
\pgfpathlineto{\pgfqpoint{8.050131in}{3.055313in}}%
\pgfpathlineto{\pgfqpoint{8.053303in}{3.058335in}}%
\pgfpathlineto{\pgfqpoint{8.056475in}{3.061365in}}%
\pgfpathlineto{\pgfqpoint{8.059648in}{3.064346in}}%
\pgfpathlineto{\pgfqpoint{8.062820in}{3.067375in}}%
\pgfpathlineto{\pgfqpoint{8.065992in}{3.070414in}}%
\pgfpathlineto{\pgfqpoint{8.069164in}{3.073472in}}%
\pgfpathlineto{\pgfqpoint{8.072336in}{3.076519in}}%
\pgfpathlineto{\pgfqpoint{8.075508in}{3.079520in}}%
\pgfpathlineto{\pgfqpoint{8.078680in}{3.082541in}}%
\pgfpathlineto{\pgfqpoint{8.081852in}{3.085600in}}%
\pgfpathlineto{\pgfqpoint{8.085024in}{3.088686in}}%
\pgfpathlineto{\pgfqpoint{8.088196in}{3.091741in}}%
\pgfpathlineto{\pgfqpoint{8.091368in}{3.094740in}}%
\pgfpathlineto{\pgfqpoint{8.094540in}{3.097785in}}%
\pgfpathlineto{\pgfqpoint{8.097712in}{3.100834in}}%
\pgfpathlineto{\pgfqpoint{8.100884in}{3.103855in}}%
\pgfpathlineto{\pgfqpoint{8.104056in}{3.106844in}}%
\pgfpathlineto{\pgfqpoint{8.107228in}{3.109882in}}%
\pgfpathlineto{\pgfqpoint{8.110400in}{3.112918in}}%
\pgfpathlineto{\pgfqpoint{8.113572in}{3.115992in}}%
\pgfpathlineto{\pgfqpoint{8.116744in}{3.119030in}}%
\pgfpathlineto{\pgfqpoint{8.119916in}{3.122059in}}%
\pgfpathlineto{\pgfqpoint{8.123088in}{3.125085in}}%
\pgfpathlineto{\pgfqpoint{8.126260in}{3.128129in}}%
\pgfpathlineto{\pgfqpoint{8.129432in}{3.131238in}}%
\pgfpathlineto{\pgfqpoint{8.132604in}{3.134253in}}%
\pgfpathlineto{\pgfqpoint{8.135777in}{3.137299in}}%
\pgfpathlineto{\pgfqpoint{8.138949in}{3.140329in}}%
\pgfpathlineto{\pgfqpoint{8.142121in}{3.143346in}}%
\pgfpathlineto{\pgfqpoint{8.145293in}{3.146411in}}%
\pgfpathlineto{\pgfqpoint{8.148465in}{3.149424in}}%
\pgfpathlineto{\pgfqpoint{8.151637in}{3.152502in}}%
\pgfpathlineto{\pgfqpoint{8.154809in}{3.155576in}}%
\pgfpathlineto{\pgfqpoint{8.157981in}{3.158618in}}%
\pgfpathlineto{\pgfqpoint{8.161153in}{3.161653in}}%
\pgfpathlineto{\pgfqpoint{8.164325in}{3.164690in}}%
\pgfpathlineto{\pgfqpoint{8.167497in}{3.167722in}}%
\pgfpathlineto{\pgfqpoint{8.170669in}{3.170700in}}%
\pgfpathlineto{\pgfqpoint{8.173841in}{3.173743in}}%
\pgfpathlineto{\pgfqpoint{8.177013in}{3.176818in}}%
\pgfpathlineto{\pgfqpoint{8.180185in}{3.179739in}}%
\pgfpathlineto{\pgfqpoint{8.183357in}{3.182804in}}%
\pgfpathlineto{\pgfqpoint{8.186529in}{3.185856in}}%
\pgfpathlineto{\pgfqpoint{8.189701in}{3.188887in}}%
\pgfpathlineto{\pgfqpoint{8.192873in}{3.191918in}}%
\pgfpathlineto{\pgfqpoint{8.196045in}{3.194884in}}%
\pgfpathlineto{\pgfqpoint{8.199217in}{3.197968in}}%
\pgfpathlineto{\pgfqpoint{8.202389in}{3.201027in}}%
\pgfpathlineto{\pgfqpoint{8.205561in}{3.204050in}}%
\pgfpathlineto{\pgfqpoint{8.208733in}{3.207099in}}%
\pgfpathlineto{\pgfqpoint{8.211905in}{3.210186in}}%
\pgfpathlineto{\pgfqpoint{8.215078in}{3.213258in}}%
\pgfpathlineto{\pgfqpoint{8.218250in}{3.216246in}}%
\pgfpathlineto{\pgfqpoint{8.221422in}{3.219299in}}%
\pgfpathlineto{\pgfqpoint{8.224594in}{3.222338in}}%
\pgfpathlineto{\pgfqpoint{8.227766in}{3.225385in}}%
\pgfpathlineto{\pgfqpoint{8.230938in}{3.228261in}}%
\pgfpathlineto{\pgfqpoint{8.234110in}{3.231327in}}%
\pgfpathlineto{\pgfqpoint{8.237282in}{3.234294in}}%
\pgfpathlineto{\pgfqpoint{8.240454in}{3.237281in}}%
\pgfpathlineto{\pgfqpoint{8.243626in}{3.240273in}}%
\pgfpathlineto{\pgfqpoint{8.246798in}{3.243309in}}%
\pgfpathlineto{\pgfqpoint{8.249970in}{3.246312in}}%
\pgfpathlineto{\pgfqpoint{8.253142in}{3.249302in}}%
\pgfpathlineto{\pgfqpoint{8.256314in}{3.252215in}}%
\pgfpathlineto{\pgfqpoint{8.259486in}{3.255238in}}%
\pgfpathlineto{\pgfqpoint{8.262658in}{3.258325in}}%
\pgfpathlineto{\pgfqpoint{8.265830in}{3.261411in}}%
\pgfpathlineto{\pgfqpoint{8.269002in}{3.264467in}}%
\pgfpathlineto{\pgfqpoint{8.272174in}{3.267484in}}%
\pgfpathlineto{\pgfqpoint{8.275346in}{3.270415in}}%
\pgfpathlineto{\pgfqpoint{8.278518in}{3.273482in}}%
\pgfpathlineto{\pgfqpoint{8.281690in}{3.276674in}}%
\pgfpathlineto{\pgfqpoint{8.281690in}{3.374425in}}%
\pgfpathlineto{\pgfqpoint{8.281690in}{3.374425in}}%
\pgfpathlineto{\pgfqpoint{8.278518in}{3.371324in}}%
\pgfpathlineto{\pgfqpoint{8.275346in}{3.368230in}}%
\pgfpathlineto{\pgfqpoint{8.272174in}{3.365137in}}%
\pgfpathlineto{\pgfqpoint{8.269002in}{3.362065in}}%
\pgfpathlineto{\pgfqpoint{8.265830in}{3.359049in}}%
\pgfpathlineto{\pgfqpoint{8.262658in}{3.355932in}}%
\pgfpathlineto{\pgfqpoint{8.259486in}{3.352822in}}%
\pgfpathlineto{\pgfqpoint{8.256314in}{3.349742in}}%
\pgfpathlineto{\pgfqpoint{8.253142in}{3.346644in}}%
\pgfpathlineto{\pgfqpoint{8.249970in}{3.343548in}}%
\pgfpathlineto{\pgfqpoint{8.246798in}{3.340459in}}%
\pgfpathlineto{\pgfqpoint{8.243626in}{3.337370in}}%
\pgfpathlineto{\pgfqpoint{8.240454in}{3.334289in}}%
\pgfpathlineto{\pgfqpoint{8.237282in}{3.331154in}}%
\pgfpathlineto{\pgfqpoint{8.234110in}{3.328041in}}%
\pgfpathlineto{\pgfqpoint{8.230938in}{3.324941in}}%
\pgfpathlineto{\pgfqpoint{8.227766in}{3.321741in}}%
\pgfpathlineto{\pgfqpoint{8.224594in}{3.318658in}}%
\pgfpathlineto{\pgfqpoint{8.221422in}{3.315562in}}%
\pgfpathlineto{\pgfqpoint{8.218250in}{3.312483in}}%
\pgfpathlineto{\pgfqpoint{8.215078in}{3.309349in}}%
\pgfpathlineto{\pgfqpoint{8.211905in}{3.306248in}}%
\pgfpathlineto{\pgfqpoint{8.208733in}{3.303181in}}%
\pgfpathlineto{\pgfqpoint{8.205561in}{3.300098in}}%
\pgfpathlineto{\pgfqpoint{8.202389in}{3.297033in}}%
\pgfpathlineto{\pgfqpoint{8.199217in}{3.293962in}}%
\pgfpathlineto{\pgfqpoint{8.196045in}{3.290853in}}%
\pgfpathlineto{\pgfqpoint{8.192873in}{3.287679in}}%
\pgfpathlineto{\pgfqpoint{8.189701in}{3.284617in}}%
\pgfpathlineto{\pgfqpoint{8.186529in}{3.281525in}}%
\pgfpathlineto{\pgfqpoint{8.183357in}{3.278428in}}%
\pgfpathlineto{\pgfqpoint{8.180185in}{3.275349in}}%
\pgfpathlineto{\pgfqpoint{8.177013in}{3.272250in}}%
\pgfpathlineto{\pgfqpoint{8.173841in}{3.269138in}}%
\pgfpathlineto{\pgfqpoint{8.170669in}{3.266008in}}%
\pgfpathlineto{\pgfqpoint{8.167497in}{3.262946in}}%
\pgfpathlineto{\pgfqpoint{8.164325in}{3.259847in}}%
\pgfpathlineto{\pgfqpoint{8.161153in}{3.256771in}}%
\pgfpathlineto{\pgfqpoint{8.157981in}{3.253691in}}%
\pgfpathlineto{\pgfqpoint{8.154809in}{3.250613in}}%
\pgfpathlineto{\pgfqpoint{8.151637in}{3.247525in}}%
\pgfpathlineto{\pgfqpoint{8.148465in}{3.244381in}}%
\pgfpathlineto{\pgfqpoint{8.145293in}{3.241285in}}%
\pgfpathlineto{\pgfqpoint{8.142121in}{3.238200in}}%
\pgfpathlineto{\pgfqpoint{8.138949in}{3.235146in}}%
\pgfpathlineto{\pgfqpoint{8.135777in}{3.232048in}}%
\pgfpathlineto{\pgfqpoint{8.132604in}{3.228984in}}%
\pgfpathlineto{\pgfqpoint{8.129432in}{3.225903in}}%
\pgfpathlineto{\pgfqpoint{8.126260in}{3.223145in}}%
\pgfpathlineto{\pgfqpoint{8.123088in}{3.220051in}}%
\pgfpathlineto{\pgfqpoint{8.119916in}{3.216971in}}%
\pgfpathlineto{\pgfqpoint{8.116744in}{3.213913in}}%
\pgfpathlineto{\pgfqpoint{8.113572in}{3.210838in}}%
\pgfpathlineto{\pgfqpoint{8.110400in}{3.207782in}}%
\pgfpathlineto{\pgfqpoint{8.107228in}{3.204702in}}%
\pgfpathlineto{\pgfqpoint{8.104056in}{3.201635in}}%
\pgfpathlineto{\pgfqpoint{8.100884in}{3.198564in}}%
\pgfpathlineto{\pgfqpoint{8.097712in}{3.195477in}}%
\pgfpathlineto{\pgfqpoint{8.094540in}{3.192426in}}%
\pgfpathlineto{\pgfqpoint{8.091368in}{3.189316in}}%
\pgfpathlineto{\pgfqpoint{8.088196in}{3.186246in}}%
\pgfpathlineto{\pgfqpoint{8.085024in}{3.183151in}}%
\pgfpathlineto{\pgfqpoint{8.081852in}{3.180043in}}%
\pgfpathlineto{\pgfqpoint{8.078680in}{3.176947in}}%
\pgfpathlineto{\pgfqpoint{8.075508in}{3.173834in}}%
\pgfpathlineto{\pgfqpoint{8.072336in}{3.170747in}}%
\pgfpathlineto{\pgfqpoint{8.069164in}{3.167659in}}%
\pgfpathlineto{\pgfqpoint{8.065992in}{3.164558in}}%
\pgfpathlineto{\pgfqpoint{8.062820in}{3.161439in}}%
\pgfpathlineto{\pgfqpoint{8.059648in}{3.158319in}}%
\pgfpathlineto{\pgfqpoint{8.056475in}{3.155244in}}%
\pgfpathlineto{\pgfqpoint{8.053303in}{3.152156in}}%
\pgfpathlineto{\pgfqpoint{8.050131in}{3.149074in}}%
\pgfpathlineto{\pgfqpoint{8.046959in}{3.146013in}}%
\pgfpathlineto{\pgfqpoint{8.043787in}{3.142949in}}%
\pgfpathlineto{\pgfqpoint{8.040615in}{3.139888in}}%
\pgfpathlineto{\pgfqpoint{8.037443in}{3.136798in}}%
\pgfpathlineto{\pgfqpoint{8.034271in}{3.133673in}}%
\pgfpathlineto{\pgfqpoint{8.031099in}{3.130575in}}%
\pgfpathlineto{\pgfqpoint{8.027927in}{3.127497in}}%
\pgfpathlineto{\pgfqpoint{8.024755in}{3.124409in}}%
\pgfpathlineto{\pgfqpoint{8.021583in}{3.121333in}}%
\pgfpathlineto{\pgfqpoint{8.018411in}{3.118282in}}%
\pgfpathlineto{\pgfqpoint{8.015239in}{3.115195in}}%
\pgfpathlineto{\pgfqpoint{8.012067in}{3.112115in}}%
\pgfpathlineto{\pgfqpoint{8.008895in}{3.109057in}}%
\pgfpathlineto{\pgfqpoint{8.005723in}{3.105999in}}%
\pgfpathlineto{\pgfqpoint{8.002551in}{3.102919in}}%
\pgfpathlineto{\pgfqpoint{7.999379in}{3.099870in}}%
\pgfpathlineto{\pgfqpoint{7.996207in}{3.097032in}}%
\pgfpathlineto{\pgfqpoint{7.993035in}{3.093947in}}%
\pgfpathlineto{\pgfqpoint{7.989863in}{3.090793in}}%
\pgfpathlineto{\pgfqpoint{7.986691in}{3.087737in}}%
\pgfpathlineto{\pgfqpoint{7.983519in}{3.084649in}}%
\pgfpathlineto{\pgfqpoint{7.980347in}{3.081600in}}%
\pgfpathlineto{\pgfqpoint{7.977174in}{3.078527in}}%
\pgfpathlineto{\pgfqpoint{7.974002in}{3.075440in}}%
\pgfpathlineto{\pgfqpoint{7.970830in}{3.072359in}}%
\pgfpathlineto{\pgfqpoint{7.967658in}{3.069254in}}%
\pgfpathlineto{\pgfqpoint{7.964486in}{3.066181in}}%
\pgfpathlineto{\pgfqpoint{7.961314in}{3.063075in}}%
\pgfpathlineto{\pgfqpoint{7.958142in}{3.060009in}}%
\pgfpathlineto{\pgfqpoint{7.954970in}{3.056939in}}%
\pgfpathlineto{\pgfqpoint{7.951798in}{3.053837in}}%
\pgfpathlineto{\pgfqpoint{7.948626in}{3.050714in}}%
\pgfpathlineto{\pgfqpoint{7.945454in}{3.047254in}}%
\pgfpathlineto{\pgfqpoint{7.942282in}{3.043864in}}%
\pgfpathlineto{\pgfqpoint{7.939110in}{3.040767in}}%
\pgfpathlineto{\pgfqpoint{7.935938in}{3.037702in}}%
\pgfpathlineto{\pgfqpoint{7.932766in}{3.034606in}}%
\pgfpathlineto{\pgfqpoint{7.929594in}{3.031534in}}%
\pgfpathlineto{\pgfqpoint{7.926422in}{3.028473in}}%
\pgfpathlineto{\pgfqpoint{7.923250in}{3.025453in}}%
\pgfpathlineto{\pgfqpoint{7.920078in}{3.022376in}}%
\pgfpathlineto{\pgfqpoint{7.916906in}{3.019320in}}%
\pgfpathlineto{\pgfqpoint{7.913734in}{3.016251in}}%
\pgfpathlineto{\pgfqpoint{7.910562in}{3.013193in}}%
\pgfpathlineto{\pgfqpoint{7.907390in}{3.010117in}}%
\pgfpathlineto{\pgfqpoint{7.904218in}{3.007055in}}%
\pgfpathlineto{\pgfqpoint{7.901046in}{3.003978in}}%
\pgfpathlineto{\pgfqpoint{7.897873in}{3.000890in}}%
\pgfpathlineto{\pgfqpoint{7.894701in}{2.995618in}}%
\pgfpathlineto{\pgfqpoint{7.891529in}{2.992557in}}%
\pgfpathlineto{\pgfqpoint{7.888357in}{2.989478in}}%
\pgfpathlineto{\pgfqpoint{7.885185in}{2.986466in}}%
\pgfpathlineto{\pgfqpoint{7.882013in}{2.983413in}}%
\pgfpathlineto{\pgfqpoint{7.878841in}{2.980373in}}%
\pgfpathlineto{\pgfqpoint{7.875669in}{2.977291in}}%
\pgfpathlineto{\pgfqpoint{7.872497in}{2.974232in}}%
\pgfpathlineto{\pgfqpoint{7.869325in}{2.971145in}}%
\pgfpathlineto{\pgfqpoint{7.866153in}{2.968084in}}%
\pgfpathlineto{\pgfqpoint{7.862981in}{2.964977in}}%
\pgfpathlineto{\pgfqpoint{7.859809in}{2.961900in}}%
\pgfpathlineto{\pgfqpoint{7.856637in}{2.958784in}}%
\pgfpathlineto{\pgfqpoint{7.853465in}{2.955718in}}%
\pgfpathlineto{\pgfqpoint{7.850293in}{2.952627in}}%
\pgfpathlineto{\pgfqpoint{7.847121in}{2.949549in}}%
\pgfpathlineto{\pgfqpoint{7.843949in}{2.946452in}}%
\pgfpathlineto{\pgfqpoint{7.840777in}{2.943373in}}%
\pgfpathlineto{\pgfqpoint{7.837605in}{2.940277in}}%
\pgfpathlineto{\pgfqpoint{7.834433in}{2.937176in}}%
\pgfpathlineto{\pgfqpoint{7.831261in}{2.934057in}}%
\pgfpathlineto{\pgfqpoint{7.828089in}{2.931050in}}%
\pgfpathlineto{\pgfqpoint{7.824917in}{2.927988in}}%
\pgfpathlineto{\pgfqpoint{7.821744in}{2.924844in}}%
\pgfpathlineto{\pgfqpoint{7.818572in}{2.921763in}}%
\pgfpathlineto{\pgfqpoint{7.815400in}{2.918672in}}%
\pgfpathlineto{\pgfqpoint{7.812228in}{2.915588in}}%
\pgfpathlineto{\pgfqpoint{7.809056in}{2.912533in}}%
\pgfpathlineto{\pgfqpoint{7.805884in}{2.909430in}}%
\pgfpathlineto{\pgfqpoint{7.802712in}{2.906315in}}%
\pgfpathlineto{\pgfqpoint{7.799540in}{2.903310in}}%
\pgfpathlineto{\pgfqpoint{7.796368in}{2.900224in}}%
\pgfpathlineto{\pgfqpoint{7.793196in}{2.897105in}}%
\pgfpathlineto{\pgfqpoint{7.790024in}{2.893937in}}%
\pgfpathlineto{\pgfqpoint{7.786852in}{2.890840in}}%
\pgfpathlineto{\pgfqpoint{7.783680in}{2.890772in}}%
\pgfpathlineto{\pgfqpoint{7.780508in}{2.891286in}}%
\pgfpathlineto{\pgfqpoint{7.777336in}{2.891391in}}%
\pgfpathlineto{\pgfqpoint{7.774164in}{2.891458in}}%
\pgfpathlineto{\pgfqpoint{7.770992in}{2.891490in}}%
\pgfpathlineto{\pgfqpoint{7.767820in}{2.891627in}}%
\pgfpathlineto{\pgfqpoint{7.764648in}{2.891861in}}%
\pgfpathlineto{\pgfqpoint{7.761476in}{2.891924in}}%
\pgfpathlineto{\pgfqpoint{7.758304in}{2.891978in}}%
\pgfpathlineto{\pgfqpoint{7.755132in}{2.891500in}}%
\pgfpathlineto{\pgfqpoint{7.751960in}{2.891640in}}%
\pgfpathlineto{\pgfqpoint{7.748788in}{2.891956in}}%
\pgfpathlineto{\pgfqpoint{7.745616in}{2.892080in}}%
\pgfpathlineto{\pgfqpoint{7.742443in}{2.892004in}}%
\pgfpathlineto{\pgfqpoint{7.739271in}{2.892235in}}%
\pgfpathlineto{\pgfqpoint{7.736099in}{2.892427in}}%
\pgfpathlineto{\pgfqpoint{7.732927in}{2.892262in}}%
\pgfpathlineto{\pgfqpoint{7.729755in}{2.892515in}}%
\pgfpathlineto{\pgfqpoint{7.726583in}{2.892744in}}%
\pgfpathlineto{\pgfqpoint{7.723411in}{2.892486in}}%
\pgfpathlineto{\pgfqpoint{7.720239in}{2.892621in}}%
\pgfpathlineto{\pgfqpoint{7.717067in}{2.892601in}}%
\pgfpathlineto{\pgfqpoint{7.713895in}{2.892462in}}%
\pgfpathlineto{\pgfqpoint{7.710723in}{2.892386in}}%
\pgfpathlineto{\pgfqpoint{7.707551in}{2.892380in}}%
\pgfpathlineto{\pgfqpoint{7.704379in}{2.892544in}}%
\pgfpathlineto{\pgfqpoint{7.701207in}{2.892808in}}%
\pgfpathlineto{\pgfqpoint{7.698035in}{2.892998in}}%
\pgfpathlineto{\pgfqpoint{7.694863in}{2.892913in}}%
\pgfpathlineto{\pgfqpoint{7.691691in}{2.892859in}}%
\pgfpathlineto{\pgfqpoint{7.688519in}{2.892939in}}%
\pgfpathlineto{\pgfqpoint{7.685347in}{2.892735in}}%
\pgfpathlineto{\pgfqpoint{7.682175in}{2.892648in}}%
\pgfpathlineto{\pgfqpoint{7.679003in}{2.892328in}}%
\pgfpathlineto{\pgfqpoint{7.675831in}{2.892421in}}%
\pgfpathlineto{\pgfqpoint{7.672659in}{2.892512in}}%
\pgfpathlineto{\pgfqpoint{7.669487in}{2.892535in}}%
\pgfpathlineto{\pgfqpoint{7.666315in}{2.892887in}}%
\pgfpathlineto{\pgfqpoint{7.663142in}{2.893032in}}%
\pgfpathlineto{\pgfqpoint{7.659970in}{2.892879in}}%
\pgfpathlineto{\pgfqpoint{7.656798in}{2.893109in}}%
\pgfpathlineto{\pgfqpoint{7.653626in}{2.893184in}}%
\pgfpathlineto{\pgfqpoint{7.650454in}{2.893251in}}%
\pgfpathlineto{\pgfqpoint{7.647282in}{2.893160in}}%
\pgfpathlineto{\pgfqpoint{7.644110in}{2.893036in}}%
\pgfpathlineto{\pgfqpoint{7.640938in}{2.893123in}}%
\pgfpathlineto{\pgfqpoint{7.637766in}{2.892944in}}%
\pgfpathlineto{\pgfqpoint{7.634594in}{2.892826in}}%
\pgfpathlineto{\pgfqpoint{7.631422in}{2.893101in}}%
\pgfpathlineto{\pgfqpoint{7.628250in}{2.893137in}}%
\pgfpathlineto{\pgfqpoint{7.625078in}{2.892878in}}%
\pgfpathlineto{\pgfqpoint{7.621906in}{2.892806in}}%
\pgfpathlineto{\pgfqpoint{7.618734in}{2.892771in}}%
\pgfpathlineto{\pgfqpoint{7.615562in}{2.892744in}}%
\pgfpathlineto{\pgfqpoint{7.612390in}{2.892907in}}%
\pgfpathlineto{\pgfqpoint{7.609218in}{2.892782in}}%
\pgfpathlineto{\pgfqpoint{7.606046in}{2.892652in}}%
\pgfpathlineto{\pgfqpoint{7.602874in}{2.892724in}}%
\pgfpathlineto{\pgfqpoint{7.599702in}{2.892904in}}%
\pgfpathlineto{\pgfqpoint{7.596530in}{2.892928in}}%
\pgfpathlineto{\pgfqpoint{7.593358in}{2.892927in}}%
\pgfpathlineto{\pgfqpoint{7.590186in}{2.892758in}}%
\pgfpathlineto{\pgfqpoint{7.587013in}{2.892710in}}%
\pgfpathlineto{\pgfqpoint{7.583841in}{2.892827in}}%
\pgfpathlineto{\pgfqpoint{7.580669in}{2.892997in}}%
\pgfpathlineto{\pgfqpoint{7.577497in}{2.893175in}}%
\pgfpathlineto{\pgfqpoint{7.574325in}{2.893064in}}%
\pgfpathlineto{\pgfqpoint{7.571153in}{2.893203in}}%
\pgfpathlineto{\pgfqpoint{7.567981in}{2.892721in}}%
\pgfpathlineto{\pgfqpoint{7.564809in}{2.892105in}}%
\pgfpathlineto{\pgfqpoint{7.561637in}{2.892230in}}%
\pgfpathlineto{\pgfqpoint{7.558465in}{2.892525in}}%
\pgfpathlineto{\pgfqpoint{7.555293in}{2.892456in}}%
\pgfpathlineto{\pgfqpoint{7.552121in}{2.892451in}}%
\pgfpathlineto{\pgfqpoint{7.548949in}{2.892381in}}%
\pgfpathlineto{\pgfqpoint{7.545777in}{2.892280in}}%
\pgfpathlineto{\pgfqpoint{7.542605in}{2.892362in}}%
\pgfpathlineto{\pgfqpoint{7.539433in}{2.892175in}}%
\pgfpathlineto{\pgfqpoint{7.536261in}{2.892561in}}%
\pgfpathlineto{\pgfqpoint{7.533089in}{2.892799in}}%
\pgfpathlineto{\pgfqpoint{7.529917in}{2.893005in}}%
\pgfpathlineto{\pgfqpoint{7.526745in}{2.892632in}}%
\pgfpathlineto{\pgfqpoint{7.523573in}{2.892458in}}%
\pgfpathlineto{\pgfqpoint{7.520401in}{2.892696in}}%
\pgfpathlineto{\pgfqpoint{7.517229in}{2.892743in}}%
\pgfpathlineto{\pgfqpoint{7.514057in}{2.892555in}}%
\pgfpathlineto{\pgfqpoint{7.510885in}{2.892718in}}%
\pgfpathlineto{\pgfqpoint{7.507712in}{2.892713in}}%
\pgfpathlineto{\pgfqpoint{7.504540in}{2.892683in}}%
\pgfpathlineto{\pgfqpoint{7.501368in}{2.892731in}}%
\pgfpathlineto{\pgfqpoint{7.498196in}{2.892942in}}%
\pgfpathlineto{\pgfqpoint{7.495024in}{2.892966in}}%
\pgfpathlineto{\pgfqpoint{7.491852in}{2.892821in}}%
\pgfpathlineto{\pgfqpoint{7.488680in}{2.892401in}}%
\pgfpathlineto{\pgfqpoint{7.485508in}{2.892280in}}%
\pgfpathlineto{\pgfqpoint{7.482336in}{2.892557in}}%
\pgfpathlineto{\pgfqpoint{7.479164in}{2.892105in}}%
\pgfpathlineto{\pgfqpoint{7.475992in}{2.892003in}}%
\pgfpathlineto{\pgfqpoint{7.472820in}{2.892038in}}%
\pgfpathlineto{\pgfqpoint{7.469648in}{2.892258in}}%
\pgfpathlineto{\pgfqpoint{7.466476in}{2.892476in}}%
\pgfpathlineto{\pgfqpoint{7.463304in}{2.892548in}}%
\pgfpathlineto{\pgfqpoint{7.460132in}{2.892604in}}%
\pgfpathlineto{\pgfqpoint{7.456960in}{2.892735in}}%
\pgfpathlineto{\pgfqpoint{7.453788in}{2.893038in}}%
\pgfpathlineto{\pgfqpoint{7.450616in}{2.892833in}}%
\pgfpathlineto{\pgfqpoint{7.447444in}{2.892929in}}%
\pgfpathlineto{\pgfqpoint{7.444272in}{2.893019in}}%
\pgfpathlineto{\pgfqpoint{7.441100in}{2.892981in}}%
\pgfpathlineto{\pgfqpoint{7.437928in}{2.892964in}}%
\pgfpathlineto{\pgfqpoint{7.434756in}{2.893015in}}%
\pgfpathlineto{\pgfqpoint{7.431584in}{2.892983in}}%
\pgfpathlineto{\pgfqpoint{7.428411in}{2.892680in}}%
\pgfpathlineto{\pgfqpoint{7.425239in}{2.892969in}}%
\pgfpathlineto{\pgfqpoint{7.422067in}{2.893271in}}%
\pgfpathlineto{\pgfqpoint{7.418895in}{2.893295in}}%
\pgfpathlineto{\pgfqpoint{7.415723in}{2.893334in}}%
\pgfpathlineto{\pgfqpoint{7.412551in}{2.893356in}}%
\pgfpathlineto{\pgfqpoint{7.409379in}{2.893307in}}%
\pgfpathlineto{\pgfqpoint{7.406207in}{2.893610in}}%
\pgfpathlineto{\pgfqpoint{7.403035in}{2.894115in}}%
\pgfpathlineto{\pgfqpoint{7.399863in}{2.894037in}}%
\pgfpathlineto{\pgfqpoint{7.396691in}{2.894201in}}%
\pgfpathlineto{\pgfqpoint{7.393519in}{2.894095in}}%
\pgfpathlineto{\pgfqpoint{7.390347in}{2.893828in}}%
\pgfpathlineto{\pgfqpoint{7.387175in}{2.893773in}}%
\pgfpathlineto{\pgfqpoint{7.384003in}{2.893562in}}%
\pgfpathlineto{\pgfqpoint{7.380831in}{2.893529in}}%
\pgfpathlineto{\pgfqpoint{7.377659in}{2.893233in}}%
\pgfpathlineto{\pgfqpoint{7.374487in}{2.893105in}}%
\pgfpathlineto{\pgfqpoint{7.371315in}{2.893076in}}%
\pgfpathlineto{\pgfqpoint{7.368143in}{2.892926in}}%
\pgfpathlineto{\pgfqpoint{7.364971in}{2.892990in}}%
\pgfpathlineto{\pgfqpoint{7.361799in}{2.892873in}}%
\pgfpathlineto{\pgfqpoint{7.358627in}{2.892811in}}%
\pgfpathlineto{\pgfqpoint{7.355455in}{2.892806in}}%
\pgfpathlineto{\pgfqpoint{7.352282in}{2.892718in}}%
\pgfpathlineto{\pgfqpoint{7.349110in}{2.892661in}}%
\pgfpathlineto{\pgfqpoint{7.345938in}{2.892769in}}%
\pgfpathlineto{\pgfqpoint{7.342766in}{2.893317in}}%
\pgfpathlineto{\pgfqpoint{7.339594in}{2.893320in}}%
\pgfpathlineto{\pgfqpoint{7.336422in}{2.893414in}}%
\pgfpathlineto{\pgfqpoint{7.333250in}{2.893496in}}%
\pgfpathlineto{\pgfqpoint{7.330078in}{2.893383in}}%
\pgfpathlineto{\pgfqpoint{7.326906in}{2.893264in}}%
\pgfpathlineto{\pgfqpoint{7.323734in}{2.893365in}}%
\pgfpathlineto{\pgfqpoint{7.320562in}{2.893102in}}%
\pgfpathlineto{\pgfqpoint{7.317390in}{2.892841in}}%
\pgfpathlineto{\pgfqpoint{7.314218in}{2.892882in}}%
\pgfpathlineto{\pgfqpoint{7.311046in}{2.893086in}}%
\pgfpathlineto{\pgfqpoint{7.307874in}{2.893112in}}%
\pgfpathlineto{\pgfqpoint{7.304702in}{2.893252in}}%
\pgfpathlineto{\pgfqpoint{7.301530in}{2.893190in}}%
\pgfpathlineto{\pgfqpoint{7.298358in}{2.893054in}}%
\pgfpathlineto{\pgfqpoint{7.295186in}{2.892666in}}%
\pgfpathlineto{\pgfqpoint{7.292014in}{2.892746in}}%
\pgfpathlineto{\pgfqpoint{7.288842in}{2.892774in}}%
\pgfpathlineto{\pgfqpoint{7.285670in}{2.892839in}}%
\pgfpathlineto{\pgfqpoint{7.282498in}{2.893011in}}%
\pgfpathlineto{\pgfqpoint{7.279326in}{2.893146in}}%
\pgfpathlineto{\pgfqpoint{7.276154in}{2.893202in}}%
\pgfpathlineto{\pgfqpoint{7.272981in}{2.892921in}}%
\pgfpathlineto{\pgfqpoint{7.269809in}{2.892455in}}%
\pgfpathlineto{\pgfqpoint{7.266637in}{2.892472in}}%
\pgfpathlineto{\pgfqpoint{7.263465in}{2.892801in}}%
\pgfpathlineto{\pgfqpoint{7.260293in}{2.892422in}}%
\pgfpathlineto{\pgfqpoint{7.257121in}{2.892496in}}%
\pgfpathlineto{\pgfqpoint{7.253949in}{2.892334in}}%
\pgfpathlineto{\pgfqpoint{7.250777in}{2.892138in}}%
\pgfpathlineto{\pgfqpoint{7.247605in}{2.892177in}}%
\pgfpathlineto{\pgfqpoint{7.244433in}{2.892495in}}%
\pgfpathlineto{\pgfqpoint{7.241261in}{2.892292in}}%
\pgfpathlineto{\pgfqpoint{7.238089in}{2.892157in}}%
\pgfpathlineto{\pgfqpoint{7.234917in}{2.892113in}}%
\pgfpathlineto{\pgfqpoint{7.231745in}{2.892340in}}%
\pgfpathlineto{\pgfqpoint{7.228573in}{2.892151in}}%
\pgfpathlineto{\pgfqpoint{7.225401in}{2.892215in}}%
\pgfpathlineto{\pgfqpoint{7.222229in}{2.892219in}}%
\pgfpathlineto{\pgfqpoint{7.219057in}{2.892063in}}%
\pgfpathlineto{\pgfqpoint{7.215885in}{2.891969in}}%
\pgfpathlineto{\pgfqpoint{7.212713in}{2.891807in}}%
\pgfpathlineto{\pgfqpoint{7.209541in}{2.891869in}}%
\pgfpathlineto{\pgfqpoint{7.206369in}{2.892003in}}%
\pgfpathlineto{\pgfqpoint{7.203197in}{2.892351in}}%
\pgfpathlineto{\pgfqpoint{7.200025in}{2.892532in}}%
\pgfpathlineto{\pgfqpoint{7.196853in}{2.892367in}}%
\pgfpathlineto{\pgfqpoint{7.193680in}{2.892473in}}%
\pgfpathlineto{\pgfqpoint{7.190508in}{2.892490in}}%
\pgfpathlineto{\pgfqpoint{7.187336in}{2.892395in}}%
\pgfpathlineto{\pgfqpoint{7.184164in}{2.891843in}}%
\pgfpathlineto{\pgfqpoint{7.180992in}{2.891694in}}%
\pgfpathlineto{\pgfqpoint{7.177820in}{2.891709in}}%
\pgfpathlineto{\pgfqpoint{7.174648in}{2.891736in}}%
\pgfpathlineto{\pgfqpoint{7.171476in}{2.891726in}}%
\pgfpathlineto{\pgfqpoint{7.168304in}{2.891493in}}%
\pgfpathlineto{\pgfqpoint{7.165132in}{2.891268in}}%
\pgfpathlineto{\pgfqpoint{7.161960in}{2.891156in}}%
\pgfpathlineto{\pgfqpoint{7.158788in}{2.891319in}}%
\pgfpathlineto{\pgfqpoint{7.155616in}{2.891097in}}%
\pgfpathlineto{\pgfqpoint{7.152444in}{2.891126in}}%
\pgfpathlineto{\pgfqpoint{7.149272in}{2.891215in}}%
\pgfpathlineto{\pgfqpoint{7.146100in}{2.891693in}}%
\pgfpathlineto{\pgfqpoint{7.142928in}{2.891807in}}%
\pgfpathlineto{\pgfqpoint{7.139756in}{2.891804in}}%
\pgfpathlineto{\pgfqpoint{7.136584in}{2.891913in}}%
\pgfpathlineto{\pgfqpoint{7.133412in}{2.891870in}}%
\pgfpathlineto{\pgfqpoint{7.130240in}{2.891773in}}%
\pgfpathlineto{\pgfqpoint{7.127068in}{2.891612in}}%
\pgfpathlineto{\pgfqpoint{7.123896in}{2.891813in}}%
\pgfpathlineto{\pgfqpoint{7.120724in}{2.891722in}}%
\pgfpathlineto{\pgfqpoint{7.117551in}{2.892047in}}%
\pgfpathlineto{\pgfqpoint{7.114379in}{2.892333in}}%
\pgfpathlineto{\pgfqpoint{7.111207in}{2.892255in}}%
\pgfpathlineto{\pgfqpoint{7.108035in}{2.892213in}}%
\pgfpathlineto{\pgfqpoint{7.104863in}{2.892218in}}%
\pgfpathlineto{\pgfqpoint{7.101691in}{2.892180in}}%
\pgfpathlineto{\pgfqpoint{7.098519in}{2.892158in}}%
\pgfpathlineto{\pgfqpoint{7.095347in}{2.892324in}}%
\pgfpathlineto{\pgfqpoint{7.092175in}{2.892315in}}%
\pgfpathlineto{\pgfqpoint{7.089003in}{2.892452in}}%
\pgfpathlineto{\pgfqpoint{7.085831in}{2.892551in}}%
\pgfpathlineto{\pgfqpoint{7.082659in}{2.892436in}}%
\pgfpathlineto{\pgfqpoint{7.079487in}{2.892786in}}%
\pgfpathlineto{\pgfqpoint{7.076315in}{2.892906in}}%
\pgfpathlineto{\pgfqpoint{7.073143in}{2.892683in}}%
\pgfpathlineto{\pgfqpoint{7.069971in}{2.892563in}}%
\pgfpathlineto{\pgfqpoint{7.066799in}{2.892486in}}%
\pgfpathlineto{\pgfqpoint{7.063627in}{2.892292in}}%
\pgfpathlineto{\pgfqpoint{7.060455in}{2.892841in}}%
\pgfpathlineto{\pgfqpoint{7.057283in}{2.892846in}}%
\pgfpathlineto{\pgfqpoint{7.054111in}{2.892637in}}%
\pgfpathlineto{\pgfqpoint{7.050939in}{2.892860in}}%
\pgfpathlineto{\pgfqpoint{7.047767in}{2.893302in}}%
\pgfpathlineto{\pgfqpoint{7.044595in}{2.893401in}}%
\pgfpathlineto{\pgfqpoint{7.041423in}{2.893435in}}%
\pgfpathlineto{\pgfqpoint{7.038250in}{2.893902in}}%
\pgfpathlineto{\pgfqpoint{7.035078in}{2.893782in}}%
\pgfpathlineto{\pgfqpoint{7.031906in}{2.893922in}}%
\pgfpathlineto{\pgfqpoint{7.028734in}{2.893993in}}%
\pgfpathlineto{\pgfqpoint{7.025562in}{2.894211in}}%
\pgfpathlineto{\pgfqpoint{7.022390in}{2.894293in}}%
\pgfpathlineto{\pgfqpoint{7.019218in}{2.894595in}}%
\pgfpathlineto{\pgfqpoint{7.016046in}{2.894336in}}%
\pgfpathlineto{\pgfqpoint{7.012874in}{2.894387in}}%
\pgfpathlineto{\pgfqpoint{7.009702in}{2.894227in}}%
\pgfpathlineto{\pgfqpoint{7.006530in}{2.894408in}}%
\pgfpathlineto{\pgfqpoint{7.003358in}{2.894120in}}%
\pgfpathlineto{\pgfqpoint{7.000186in}{2.893853in}}%
\pgfpathlineto{\pgfqpoint{6.997014in}{2.893821in}}%
\pgfpathlineto{\pgfqpoint{6.993842in}{2.893922in}}%
\pgfpathlineto{\pgfqpoint{6.990670in}{2.893821in}}%
\pgfpathlineto{\pgfqpoint{6.987498in}{2.893771in}}%
\pgfpathlineto{\pgfqpoint{6.984326in}{2.894007in}}%
\pgfpathlineto{\pgfqpoint{6.981154in}{2.894046in}}%
\pgfpathlineto{\pgfqpoint{6.977982in}{2.893917in}}%
\pgfpathlineto{\pgfqpoint{6.974810in}{2.893853in}}%
\pgfpathlineto{\pgfqpoint{6.971638in}{2.893762in}}%
\pgfpathlineto{\pgfqpoint{6.968466in}{2.894149in}}%
\pgfpathlineto{\pgfqpoint{6.965294in}{2.894431in}}%
\pgfpathlineto{\pgfqpoint{6.962122in}{2.894326in}}%
\pgfpathlineto{\pgfqpoint{6.958949in}{2.894439in}}%
\pgfpathlineto{\pgfqpoint{6.955777in}{2.894371in}}%
\pgfpathlineto{\pgfqpoint{6.952605in}{2.894396in}}%
\pgfpathlineto{\pgfqpoint{6.949433in}{2.894160in}}%
\pgfpathlineto{\pgfqpoint{6.946261in}{2.894310in}}%
\pgfpathlineto{\pgfqpoint{6.943089in}{2.894167in}}%
\pgfpathlineto{\pgfqpoint{6.939917in}{2.893957in}}%
\pgfpathlineto{\pgfqpoint{6.936745in}{2.893904in}}%
\pgfpathlineto{\pgfqpoint{6.933573in}{2.894001in}}%
\pgfpathlineto{\pgfqpoint{6.930401in}{2.893962in}}%
\pgfpathlineto{\pgfqpoint{6.927229in}{2.894229in}}%
\pgfpathlineto{\pgfqpoint{6.924057in}{2.894235in}}%
\pgfpathlineto{\pgfqpoint{6.920885in}{2.894142in}}%
\pgfpathlineto{\pgfqpoint{6.917713in}{2.894171in}}%
\pgfpathlineto{\pgfqpoint{6.914541in}{2.894297in}}%
\pgfpathlineto{\pgfqpoint{6.911369in}{2.894161in}}%
\pgfpathlineto{\pgfqpoint{6.908197in}{2.894190in}}%
\pgfpathlineto{\pgfqpoint{6.905025in}{2.894194in}}%
\pgfpathlineto{\pgfqpoint{6.901853in}{2.893958in}}%
\pgfpathlineto{\pgfqpoint{6.898681in}{2.894429in}}%
\pgfpathlineto{\pgfqpoint{6.895509in}{2.894470in}}%
\pgfpathlineto{\pgfqpoint{6.892337in}{2.894407in}}%
\pgfpathlineto{\pgfqpoint{6.889165in}{2.894242in}}%
\pgfpathlineto{\pgfqpoint{6.885993in}{2.894390in}}%
\pgfpathlineto{\pgfqpoint{6.882820in}{2.894484in}}%
\pgfpathlineto{\pgfqpoint{6.879648in}{2.894527in}}%
\pgfpathlineto{\pgfqpoint{6.876476in}{2.894853in}}%
\pgfpathlineto{\pgfqpoint{6.873304in}{2.894638in}}%
\pgfpathlineto{\pgfqpoint{6.870132in}{2.894423in}}%
\pgfpathlineto{\pgfqpoint{6.866960in}{2.894426in}}%
\pgfpathlineto{\pgfqpoint{6.863788in}{2.894885in}}%
\pgfpathlineto{\pgfqpoint{6.860616in}{2.894934in}}%
\pgfpathlineto{\pgfqpoint{6.857444in}{2.895114in}}%
\pgfpathlineto{\pgfqpoint{6.854272in}{2.894917in}}%
\pgfpathlineto{\pgfqpoint{6.851100in}{2.894995in}}%
\pgfpathlineto{\pgfqpoint{6.847928in}{2.895023in}}%
\pgfpathlineto{\pgfqpoint{6.844756in}{2.894875in}}%
\pgfpathlineto{\pgfqpoint{6.841584in}{2.894908in}}%
\pgfpathlineto{\pgfqpoint{6.838412in}{2.894913in}}%
\pgfpathlineto{\pgfqpoint{6.835240in}{2.894782in}}%
\pgfpathlineto{\pgfqpoint{6.832068in}{2.894999in}}%
\pgfpathlineto{\pgfqpoint{6.828896in}{2.894558in}}%
\pgfpathlineto{\pgfqpoint{6.825724in}{2.894448in}}%
\pgfpathlineto{\pgfqpoint{6.822552in}{2.894377in}}%
\pgfpathlineto{\pgfqpoint{6.819380in}{2.894019in}}%
\pgfpathlineto{\pgfqpoint{6.816208in}{2.894358in}}%
\pgfpathlineto{\pgfqpoint{6.813036in}{2.894304in}}%
\pgfpathlineto{\pgfqpoint{6.809864in}{2.894634in}}%
\pgfpathlineto{\pgfqpoint{6.806692in}{2.894526in}}%
\pgfpathlineto{\pgfqpoint{6.803519in}{2.894626in}}%
\pgfpathlineto{\pgfqpoint{6.800347in}{2.894632in}}%
\pgfpathlineto{\pgfqpoint{6.797175in}{2.894438in}}%
\pgfpathlineto{\pgfqpoint{6.794003in}{2.894320in}}%
\pgfpathlineto{\pgfqpoint{6.790831in}{2.894774in}}%
\pgfpathlineto{\pgfqpoint{6.787659in}{2.894925in}}%
\pgfpathlineto{\pgfqpoint{6.784487in}{2.894737in}}%
\pgfpathlineto{\pgfqpoint{6.781315in}{2.894678in}}%
\pgfpathlineto{\pgfqpoint{6.778143in}{2.894544in}}%
\pgfpathlineto{\pgfqpoint{6.774971in}{2.894393in}}%
\pgfpathlineto{\pgfqpoint{6.771799in}{2.894218in}}%
\pgfpathlineto{\pgfqpoint{6.768627in}{2.894070in}}%
\pgfpathlineto{\pgfqpoint{6.765455in}{2.894257in}}%
\pgfpathlineto{\pgfqpoint{6.762283in}{2.894291in}}%
\pgfpathlineto{\pgfqpoint{6.759111in}{2.894584in}}%
\pgfpathlineto{\pgfqpoint{6.755939in}{2.894723in}}%
\pgfpathlineto{\pgfqpoint{6.752767in}{2.894776in}}%
\pgfpathlineto{\pgfqpoint{6.749595in}{2.894817in}}%
\pgfpathlineto{\pgfqpoint{6.746423in}{2.894894in}}%
\pgfpathlineto{\pgfqpoint{6.743251in}{2.895030in}}%
\pgfpathlineto{\pgfqpoint{6.740079in}{2.895024in}}%
\pgfpathlineto{\pgfqpoint{6.736907in}{2.895186in}}%
\pgfpathlineto{\pgfqpoint{6.733735in}{2.895217in}}%
\pgfpathlineto{\pgfqpoint{6.730563in}{2.895139in}}%
\pgfpathlineto{\pgfqpoint{6.727391in}{2.895275in}}%
\pgfpathlineto{\pgfqpoint{6.724218in}{2.895365in}}%
\pgfpathlineto{\pgfqpoint{6.721046in}{2.895283in}}%
\pgfpathlineto{\pgfqpoint{6.717874in}{2.895272in}}%
\pgfpathlineto{\pgfqpoint{6.714702in}{2.895228in}}%
\pgfpathlineto{\pgfqpoint{6.711530in}{2.895336in}}%
\pgfpathlineto{\pgfqpoint{6.708358in}{2.895339in}}%
\pgfpathlineto{\pgfqpoint{6.705186in}{2.895392in}}%
\pgfpathlineto{\pgfqpoint{6.702014in}{2.895244in}}%
\pgfpathlineto{\pgfqpoint{6.698842in}{2.895069in}}%
\pgfpathlineto{\pgfqpoint{6.695670in}{2.894995in}}%
\pgfpathlineto{\pgfqpoint{6.692498in}{2.894802in}}%
\pgfpathlineto{\pgfqpoint{6.689326in}{2.894537in}}%
\pgfpathlineto{\pgfqpoint{6.686154in}{2.894571in}}%
\pgfpathlineto{\pgfqpoint{6.682982in}{2.894885in}}%
\pgfpathlineto{\pgfqpoint{6.679810in}{2.894730in}}%
\pgfpathlineto{\pgfqpoint{6.676638in}{2.894757in}}%
\pgfpathlineto{\pgfqpoint{6.673466in}{2.895194in}}%
\pgfpathlineto{\pgfqpoint{6.670294in}{2.895218in}}%
\pgfpathlineto{\pgfqpoint{6.667122in}{2.895155in}}%
\pgfpathlineto{\pgfqpoint{6.663950in}{2.895411in}}%
\pgfpathlineto{\pgfqpoint{6.660778in}{2.895831in}}%
\pgfpathlineto{\pgfqpoint{6.657606in}{2.895925in}}%
\pgfpathlineto{\pgfqpoint{6.654434in}{2.895787in}}%
\pgfpathlineto{\pgfqpoint{6.651262in}{2.895937in}}%
\pgfpathlineto{\pgfqpoint{6.648089in}{2.895946in}}%
\pgfpathlineto{\pgfqpoint{6.644917in}{2.896068in}}%
\pgfpathlineto{\pgfqpoint{6.641745in}{2.896126in}}%
\pgfpathlineto{\pgfqpoint{6.638573in}{2.896193in}}%
\pgfpathlineto{\pgfqpoint{6.635401in}{2.896433in}}%
\pgfpathlineto{\pgfqpoint{6.632229in}{2.896435in}}%
\pgfpathlineto{\pgfqpoint{6.629057in}{2.896323in}}%
\pgfpathlineto{\pgfqpoint{6.625885in}{2.896326in}}%
\pgfpathlineto{\pgfqpoint{6.622713in}{2.896323in}}%
\pgfpathlineto{\pgfqpoint{6.619541in}{2.896546in}}%
\pgfpathlineto{\pgfqpoint{6.616369in}{2.897127in}}%
\pgfpathlineto{\pgfqpoint{6.613197in}{2.896878in}}%
\pgfpathlineto{\pgfqpoint{6.610025in}{2.897055in}}%
\pgfpathlineto{\pgfqpoint{6.606853in}{2.897102in}}%
\pgfpathlineto{\pgfqpoint{6.603681in}{2.896871in}}%
\pgfpathlineto{\pgfqpoint{6.600509in}{2.896709in}}%
\pgfpathlineto{\pgfqpoint{6.597337in}{2.896816in}}%
\pgfpathlineto{\pgfqpoint{6.594165in}{2.896743in}}%
\pgfpathlineto{\pgfqpoint{6.590993in}{2.896824in}}%
\pgfpathlineto{\pgfqpoint{6.587821in}{2.896848in}}%
\pgfpathlineto{\pgfqpoint{6.584649in}{2.896860in}}%
\pgfpathlineto{\pgfqpoint{6.581477in}{2.896810in}}%
\pgfpathlineto{\pgfqpoint{6.578305in}{2.896679in}}%
\pgfpathlineto{\pgfqpoint{6.575133in}{2.896746in}}%
\pgfpathlineto{\pgfqpoint{6.571961in}{2.896706in}}%
\pgfpathlineto{\pgfqpoint{6.568788in}{2.896579in}}%
\pgfpathlineto{\pgfqpoint{6.565616in}{2.896560in}}%
\pgfpathlineto{\pgfqpoint{6.562444in}{2.896788in}}%
\pgfpathlineto{\pgfqpoint{6.559272in}{2.896940in}}%
\pgfpathlineto{\pgfqpoint{6.556100in}{2.897174in}}%
\pgfpathlineto{\pgfqpoint{6.552928in}{2.897224in}}%
\pgfpathlineto{\pgfqpoint{6.549756in}{2.897337in}}%
\pgfpathlineto{\pgfqpoint{6.546584in}{2.897238in}}%
\pgfpathlineto{\pgfqpoint{6.543412in}{2.897337in}}%
\pgfpathlineto{\pgfqpoint{6.540240in}{2.897230in}}%
\pgfpathlineto{\pgfqpoint{6.537068in}{2.897260in}}%
\pgfpathlineto{\pgfqpoint{6.533896in}{2.897349in}}%
\pgfpathlineto{\pgfqpoint{6.530724in}{2.897634in}}%
\pgfpathlineto{\pgfqpoint{6.527552in}{2.897419in}}%
\pgfpathlineto{\pgfqpoint{6.524380in}{2.897454in}}%
\pgfpathlineto{\pgfqpoint{6.521208in}{2.897418in}}%
\pgfpathlineto{\pgfqpoint{6.518036in}{2.897512in}}%
\pgfpathlineto{\pgfqpoint{6.514864in}{2.897761in}}%
\pgfpathlineto{\pgfqpoint{6.511692in}{2.898425in}}%
\pgfpathlineto{\pgfqpoint{6.508520in}{2.898344in}}%
\pgfpathlineto{\pgfqpoint{6.505348in}{2.898310in}}%
\pgfpathlineto{\pgfqpoint{6.502176in}{2.898541in}}%
\pgfpathlineto{\pgfqpoint{6.499004in}{2.898393in}}%
\pgfpathlineto{\pgfqpoint{6.495832in}{2.898734in}}%
\pgfpathlineto{\pgfqpoint{6.492659in}{2.898815in}}%
\pgfpathlineto{\pgfqpoint{6.489487in}{2.898867in}}%
\pgfpathlineto{\pgfqpoint{6.486315in}{2.898970in}}%
\pgfpathlineto{\pgfqpoint{6.483143in}{2.899081in}}%
\pgfpathlineto{\pgfqpoint{6.479971in}{2.898962in}}%
\pgfpathlineto{\pgfqpoint{6.476799in}{2.899233in}}%
\pgfpathlineto{\pgfqpoint{6.473627in}{2.899030in}}%
\pgfpathlineto{\pgfqpoint{6.470455in}{2.899162in}}%
\pgfpathlineto{\pgfqpoint{6.467283in}{2.898948in}}%
\pgfpathlineto{\pgfqpoint{6.464111in}{2.898846in}}%
\pgfpathlineto{\pgfqpoint{6.460939in}{2.899005in}}%
\pgfpathlineto{\pgfqpoint{6.457767in}{2.898860in}}%
\pgfpathlineto{\pgfqpoint{6.454595in}{2.899066in}}%
\pgfpathlineto{\pgfqpoint{6.451423in}{2.899380in}}%
\pgfpathlineto{\pgfqpoint{6.448251in}{2.899179in}}%
\pgfpathlineto{\pgfqpoint{6.445079in}{2.899244in}}%
\pgfpathlineto{\pgfqpoint{6.441907in}{2.898973in}}%
\pgfpathlineto{\pgfqpoint{6.438735in}{2.899027in}}%
\pgfpathlineto{\pgfqpoint{6.435563in}{2.899046in}}%
\pgfpathlineto{\pgfqpoint{6.432391in}{2.899195in}}%
\pgfpathlineto{\pgfqpoint{6.429219in}{2.899435in}}%
\pgfpathlineto{\pgfqpoint{6.426047in}{2.899536in}}%
\pgfpathlineto{\pgfqpoint{6.422875in}{2.899375in}}%
\pgfpathlineto{\pgfqpoint{6.419703in}{2.899356in}}%
\pgfpathlineto{\pgfqpoint{6.416531in}{2.899296in}}%
\pgfpathlineto{\pgfqpoint{6.413358in}{2.899511in}}%
\pgfpathlineto{\pgfqpoint{6.410186in}{2.899726in}}%
\pgfpathlineto{\pgfqpoint{6.407014in}{2.899837in}}%
\pgfpathlineto{\pgfqpoint{6.403842in}{2.899629in}}%
\pgfpathlineto{\pgfqpoint{6.400670in}{2.899985in}}%
\pgfpathlineto{\pgfqpoint{6.397498in}{2.899736in}}%
\pgfpathlineto{\pgfqpoint{6.394326in}{2.899723in}}%
\pgfpathlineto{\pgfqpoint{6.391154in}{2.899640in}}%
\pgfpathlineto{\pgfqpoint{6.387982in}{2.899781in}}%
\pgfpathlineto{\pgfqpoint{6.384810in}{2.899388in}}%
\pgfpathlineto{\pgfqpoint{6.381638in}{2.899098in}}%
\pgfpathlineto{\pgfqpoint{6.378466in}{2.898874in}}%
\pgfpathlineto{\pgfqpoint{6.375294in}{2.898865in}}%
\pgfpathlineto{\pgfqpoint{6.372122in}{2.898788in}}%
\pgfpathlineto{\pgfqpoint{6.368950in}{2.898654in}}%
\pgfpathlineto{\pgfqpoint{6.365778in}{2.898497in}}%
\pgfpathlineto{\pgfqpoint{6.362606in}{2.898250in}}%
\pgfpathlineto{\pgfqpoint{6.359434in}{2.898578in}}%
\pgfpathlineto{\pgfqpoint{6.356262in}{2.898923in}}%
\pgfpathlineto{\pgfqpoint{6.353090in}{2.898670in}}%
\pgfpathlineto{\pgfqpoint{6.349918in}{2.898276in}}%
\pgfpathlineto{\pgfqpoint{6.346746in}{2.898215in}}%
\pgfpathlineto{\pgfqpoint{6.343574in}{2.898196in}}%
\pgfpathlineto{\pgfqpoint{6.340402in}{2.898084in}}%
\pgfpathlineto{\pgfqpoint{6.337230in}{2.898217in}}%
\pgfpathlineto{\pgfqpoint{6.334057in}{2.898103in}}%
\pgfpathlineto{\pgfqpoint{6.330885in}{2.898032in}}%
\pgfpathlineto{\pgfqpoint{6.327713in}{2.898090in}}%
\pgfpathlineto{\pgfqpoint{6.324541in}{2.898121in}}%
\pgfpathlineto{\pgfqpoint{6.321369in}{2.898220in}}%
\pgfpathlineto{\pgfqpoint{6.318197in}{2.898278in}}%
\pgfpathlineto{\pgfqpoint{6.315025in}{2.898260in}}%
\pgfpathlineto{\pgfqpoint{6.311853in}{2.898121in}}%
\pgfpathlineto{\pgfqpoint{6.308681in}{2.898459in}}%
\pgfpathlineto{\pgfqpoint{6.305509in}{2.898401in}}%
\pgfpathlineto{\pgfqpoint{6.302337in}{2.898556in}}%
\pgfpathlineto{\pgfqpoint{6.299165in}{2.898523in}}%
\pgfpathlineto{\pgfqpoint{6.295993in}{2.898755in}}%
\pgfpathlineto{\pgfqpoint{6.292821in}{2.898683in}}%
\pgfpathlineto{\pgfqpoint{6.289649in}{2.898268in}}%
\pgfpathlineto{\pgfqpoint{6.286477in}{2.898154in}}%
\pgfpathlineto{\pgfqpoint{6.283305in}{2.898332in}}%
\pgfpathlineto{\pgfqpoint{6.280133in}{2.898358in}}%
\pgfpathlineto{\pgfqpoint{6.276961in}{2.898330in}}%
\pgfpathlineto{\pgfqpoint{6.273789in}{2.898411in}}%
\pgfpathlineto{\pgfqpoint{6.270617in}{2.898486in}}%
\pgfpathlineto{\pgfqpoint{6.267445in}{2.898435in}}%
\pgfpathlineto{\pgfqpoint{6.264273in}{2.898538in}}%
\pgfpathlineto{\pgfqpoint{6.261101in}{2.898265in}}%
\pgfpathlineto{\pgfqpoint{6.257928in}{2.898197in}}%
\pgfpathlineto{\pgfqpoint{6.254756in}{2.898228in}}%
\pgfpathlineto{\pgfqpoint{6.251584in}{2.898171in}}%
\pgfpathlineto{\pgfqpoint{6.248412in}{2.898367in}}%
\pgfpathlineto{\pgfqpoint{6.245240in}{2.898242in}}%
\pgfpathlineto{\pgfqpoint{6.242068in}{2.898269in}}%
\pgfpathlineto{\pgfqpoint{6.238896in}{2.898655in}}%
\pgfpathlineto{\pgfqpoint{6.235724in}{2.898798in}}%
\pgfpathlineto{\pgfqpoint{6.232552in}{2.899070in}}%
\pgfpathlineto{\pgfqpoint{6.229380in}{2.899075in}}%
\pgfpathlineto{\pgfqpoint{6.226208in}{2.899092in}}%
\pgfpathlineto{\pgfqpoint{6.223036in}{2.899225in}}%
\pgfpathlineto{\pgfqpoint{6.219864in}{2.899350in}}%
\pgfpathlineto{\pgfqpoint{6.216692in}{2.899147in}}%
\pgfpathlineto{\pgfqpoint{6.213520in}{2.899460in}}%
\pgfpathlineto{\pgfqpoint{6.210348in}{2.899550in}}%
\pgfpathlineto{\pgfqpoint{6.207176in}{2.899561in}}%
\pgfpathlineto{\pgfqpoint{6.204004in}{2.899497in}}%
\pgfpathlineto{\pgfqpoint{6.200832in}{2.899566in}}%
\pgfpathlineto{\pgfqpoint{6.197660in}{2.899655in}}%
\pgfpathlineto{\pgfqpoint{6.194488in}{2.899658in}}%
\pgfpathlineto{\pgfqpoint{6.191316in}{2.899682in}}%
\pgfpathlineto{\pgfqpoint{6.188144in}{2.899552in}}%
\pgfpathlineto{\pgfqpoint{6.184972in}{2.899668in}}%
\pgfpathlineto{\pgfqpoint{6.181800in}{2.899798in}}%
\pgfpathlineto{\pgfqpoint{6.178627in}{2.899980in}}%
\pgfpathlineto{\pgfqpoint{6.175455in}{2.899845in}}%
\pgfpathlineto{\pgfqpoint{6.172283in}{2.899974in}}%
\pgfpathlineto{\pgfqpoint{6.169111in}{2.899935in}}%
\pgfpathlineto{\pgfqpoint{6.165939in}{2.899816in}}%
\pgfpathlineto{\pgfqpoint{6.162767in}{2.900040in}}%
\pgfpathlineto{\pgfqpoint{6.159595in}{2.900051in}}%
\pgfpathlineto{\pgfqpoint{6.156423in}{2.900097in}}%
\pgfpathlineto{\pgfqpoint{6.153251in}{2.900596in}}%
\pgfpathlineto{\pgfqpoint{6.150079in}{2.900468in}}%
\pgfpathlineto{\pgfqpoint{6.146907in}{2.900253in}}%
\pgfpathlineto{\pgfqpoint{6.143735in}{2.900206in}}%
\pgfpathlineto{\pgfqpoint{6.140563in}{2.900406in}}%
\pgfpathlineto{\pgfqpoint{6.137391in}{2.900291in}}%
\pgfpathlineto{\pgfqpoint{6.134219in}{2.900411in}}%
\pgfpathlineto{\pgfqpoint{6.131047in}{2.900304in}}%
\pgfpathlineto{\pgfqpoint{6.127875in}{2.900268in}}%
\pgfpathlineto{\pgfqpoint{6.124703in}{2.900435in}}%
\pgfpathlineto{\pgfqpoint{6.121531in}{2.900561in}}%
\pgfpathlineto{\pgfqpoint{6.118359in}{2.900728in}}%
\pgfpathlineto{\pgfqpoint{6.115187in}{2.900884in}}%
\pgfpathlineto{\pgfqpoint{6.112015in}{2.900800in}}%
\pgfpathlineto{\pgfqpoint{6.108843in}{2.900860in}}%
\pgfpathlineto{\pgfqpoint{6.105671in}{2.900997in}}%
\pgfpathlineto{\pgfqpoint{6.102499in}{2.900918in}}%
\pgfpathlineto{\pgfqpoint{6.099326in}{2.900861in}}%
\pgfpathlineto{\pgfqpoint{6.096154in}{2.901033in}}%
\pgfpathlineto{\pgfqpoint{6.092982in}{2.900864in}}%
\pgfpathlineto{\pgfqpoint{6.089810in}{2.901341in}}%
\pgfpathlineto{\pgfqpoint{6.086638in}{2.901133in}}%
\pgfpathlineto{\pgfqpoint{6.083466in}{2.900919in}}%
\pgfpathlineto{\pgfqpoint{6.080294in}{2.901032in}}%
\pgfpathlineto{\pgfqpoint{6.077122in}{2.900920in}}%
\pgfpathlineto{\pgfqpoint{6.073950in}{2.900526in}}%
\pgfpathlineto{\pgfqpoint{6.070778in}{2.900453in}}%
\pgfpathlineto{\pgfqpoint{6.067606in}{2.900390in}}%
\pgfpathlineto{\pgfqpoint{6.064434in}{2.900408in}}%
\pgfpathlineto{\pgfqpoint{6.061262in}{2.900488in}}%
\pgfpathlineto{\pgfqpoint{6.058090in}{2.900356in}}%
\pgfpathlineto{\pgfqpoint{6.054918in}{2.900307in}}%
\pgfpathlineto{\pgfqpoint{6.051746in}{2.900411in}}%
\pgfpathlineto{\pgfqpoint{6.048574in}{2.900407in}}%
\pgfpathlineto{\pgfqpoint{6.045402in}{2.900432in}}%
\pgfpathlineto{\pgfqpoint{6.042230in}{2.900611in}}%
\pgfpathlineto{\pgfqpoint{6.039058in}{2.900718in}}%
\pgfpathlineto{\pgfqpoint{6.035886in}{2.900668in}}%
\pgfpathlineto{\pgfqpoint{6.032714in}{2.900876in}}%
\pgfpathlineto{\pgfqpoint{6.029542in}{2.900881in}}%
\pgfpathlineto{\pgfqpoint{6.026370in}{2.900567in}}%
\pgfpathlineto{\pgfqpoint{6.023197in}{2.899955in}}%
\pgfpathlineto{\pgfqpoint{6.020025in}{2.899876in}}%
\pgfpathlineto{\pgfqpoint{6.016853in}{2.900272in}}%
\pgfpathlineto{\pgfqpoint{6.013681in}{2.900348in}}%
\pgfpathlineto{\pgfqpoint{6.010509in}{2.900002in}}%
\pgfpathlineto{\pgfqpoint{6.007337in}{2.900128in}}%
\pgfpathlineto{\pgfqpoint{6.004165in}{2.899825in}}%
\pgfpathlineto{\pgfqpoint{6.000993in}{2.899477in}}%
\pgfpathlineto{\pgfqpoint{5.997821in}{2.899603in}}%
\pgfpathlineto{\pgfqpoint{5.994649in}{2.900084in}}%
\pgfpathlineto{\pgfqpoint{5.991477in}{2.900348in}}%
\pgfpathlineto{\pgfqpoint{5.988305in}{2.899248in}}%
\pgfpathlineto{\pgfqpoint{5.985133in}{2.898908in}}%
\pgfpathlineto{\pgfqpoint{5.981961in}{2.898853in}}%
\pgfpathlineto{\pgfqpoint{5.978789in}{2.898535in}}%
\pgfpathlineto{\pgfqpoint{5.975617in}{2.898471in}}%
\pgfpathlineto{\pgfqpoint{5.972445in}{2.898357in}}%
\pgfpathlineto{\pgfqpoint{5.969273in}{2.898324in}}%
\pgfpathlineto{\pgfqpoint{5.966101in}{2.898403in}}%
\pgfpathlineto{\pgfqpoint{5.962929in}{2.898271in}}%
\pgfpathlineto{\pgfqpoint{5.959757in}{2.898225in}}%
\pgfpathlineto{\pgfqpoint{5.956585in}{2.898127in}}%
\pgfpathlineto{\pgfqpoint{5.953413in}{2.898069in}}%
\pgfpathlineto{\pgfqpoint{5.950241in}{2.898171in}}%
\pgfpathlineto{\pgfqpoint{5.947069in}{2.898209in}}%
\pgfpathlineto{\pgfqpoint{5.943896in}{2.898051in}}%
\pgfpathlineto{\pgfqpoint{5.940724in}{2.898027in}}%
\pgfpathlineto{\pgfqpoint{5.937552in}{2.898321in}}%
\pgfpathlineto{\pgfqpoint{5.934380in}{2.898365in}}%
\pgfpathlineto{\pgfqpoint{5.931208in}{2.898295in}}%
\pgfpathlineto{\pgfqpoint{5.928036in}{2.898374in}}%
\pgfpathlineto{\pgfqpoint{5.924864in}{2.897965in}}%
\pgfpathlineto{\pgfqpoint{5.921692in}{2.897774in}}%
\pgfpathlineto{\pgfqpoint{5.918520in}{2.897526in}}%
\pgfpathlineto{\pgfqpoint{5.915348in}{2.897596in}}%
\pgfpathlineto{\pgfqpoint{5.912176in}{2.897517in}}%
\pgfpathlineto{\pgfqpoint{5.909004in}{2.897784in}}%
\pgfpathlineto{\pgfqpoint{5.905832in}{2.897671in}}%
\pgfpathlineto{\pgfqpoint{5.902660in}{2.897977in}}%
\pgfpathlineto{\pgfqpoint{5.899488in}{2.897995in}}%
\pgfpathlineto{\pgfqpoint{5.896316in}{2.897949in}}%
\pgfpathlineto{\pgfqpoint{5.893144in}{2.898163in}}%
\pgfpathlineto{\pgfqpoint{5.889972in}{2.898395in}}%
\pgfpathlineto{\pgfqpoint{5.886800in}{2.898436in}}%
\pgfpathlineto{\pgfqpoint{5.883628in}{2.898517in}}%
\pgfpathlineto{\pgfqpoint{5.880456in}{2.898536in}}%
\pgfpathlineto{\pgfqpoint{5.877284in}{2.899149in}}%
\pgfpathlineto{\pgfqpoint{5.874112in}{2.899108in}}%
\pgfpathlineto{\pgfqpoint{5.870940in}{2.898956in}}%
\pgfpathlineto{\pgfqpoint{5.867768in}{2.898669in}}%
\pgfpathlineto{\pgfqpoint{5.864595in}{2.898612in}}%
\pgfpathlineto{\pgfqpoint{5.861423in}{2.898105in}}%
\pgfpathlineto{\pgfqpoint{5.858251in}{2.898132in}}%
\pgfpathlineto{\pgfqpoint{5.855079in}{2.898146in}}%
\pgfpathlineto{\pgfqpoint{5.851907in}{2.898270in}}%
\pgfpathlineto{\pgfqpoint{5.848735in}{2.898011in}}%
\pgfpathlineto{\pgfqpoint{5.845563in}{2.898186in}}%
\pgfpathlineto{\pgfqpoint{5.842391in}{2.898434in}}%
\pgfpathlineto{\pgfqpoint{5.839219in}{2.898589in}}%
\pgfpathlineto{\pgfqpoint{5.836047in}{2.898548in}}%
\pgfpathlineto{\pgfqpoint{5.832875in}{2.898509in}}%
\pgfpathlineto{\pgfqpoint{5.829703in}{2.898281in}}%
\pgfpathlineto{\pgfqpoint{5.826531in}{2.898368in}}%
\pgfpathlineto{\pgfqpoint{5.823359in}{2.898508in}}%
\pgfpathlineto{\pgfqpoint{5.820187in}{2.898714in}}%
\pgfpathlineto{\pgfqpoint{5.817015in}{2.898826in}}%
\pgfpathlineto{\pgfqpoint{5.813843in}{2.898843in}}%
\pgfpathlineto{\pgfqpoint{5.810671in}{2.898911in}}%
\pgfpathlineto{\pgfqpoint{5.807499in}{2.898741in}}%
\pgfpathlineto{\pgfqpoint{5.804327in}{2.898503in}}%
\pgfpathlineto{\pgfqpoint{5.801155in}{2.898680in}}%
\pgfpathlineto{\pgfqpoint{5.797983in}{2.898288in}}%
\pgfpathlineto{\pgfqpoint{5.794811in}{2.898377in}}%
\pgfpathlineto{\pgfqpoint{5.791639in}{2.898221in}}%
\pgfpathlineto{\pgfqpoint{5.788466in}{2.898234in}}%
\pgfpathlineto{\pgfqpoint{5.785294in}{2.897936in}}%
\pgfpathlineto{\pgfqpoint{5.782122in}{2.897386in}}%
\pgfpathlineto{\pgfqpoint{5.778950in}{2.897572in}}%
\pgfpathlineto{\pgfqpoint{5.775778in}{2.897382in}}%
\pgfpathlineto{\pgfqpoint{5.772606in}{2.897543in}}%
\pgfpathlineto{\pgfqpoint{5.769434in}{2.897637in}}%
\pgfpathlineto{\pgfqpoint{5.766262in}{2.897640in}}%
\pgfpathlineto{\pgfqpoint{5.763090in}{2.897710in}}%
\pgfpathlineto{\pgfqpoint{5.759918in}{2.897714in}}%
\pgfpathlineto{\pgfqpoint{5.756746in}{2.898146in}}%
\pgfpathlineto{\pgfqpoint{5.753574in}{2.898048in}}%
\pgfpathlineto{\pgfqpoint{5.750402in}{2.898340in}}%
\pgfpathlineto{\pgfqpoint{5.747230in}{2.898719in}}%
\pgfpathlineto{\pgfqpoint{5.744058in}{2.898792in}}%
\pgfpathlineto{\pgfqpoint{5.740886in}{2.899023in}}%
\pgfpathlineto{\pgfqpoint{5.737714in}{2.898848in}}%
\pgfpathlineto{\pgfqpoint{5.734542in}{2.899148in}}%
\pgfpathlineto{\pgfqpoint{5.731370in}{2.899205in}}%
\pgfpathlineto{\pgfqpoint{5.728198in}{2.899181in}}%
\pgfpathlineto{\pgfqpoint{5.725026in}{2.899339in}}%
\pgfpathlineto{\pgfqpoint{5.721854in}{2.899201in}}%
\pgfpathlineto{\pgfqpoint{5.718682in}{2.899814in}}%
\pgfpathlineto{\pgfqpoint{5.715510in}{2.899925in}}%
\pgfpathlineto{\pgfqpoint{5.712338in}{2.900294in}}%
\pgfpathlineto{\pgfqpoint{5.709165in}{2.900705in}}%
\pgfpathlineto{\pgfqpoint{5.705993in}{2.900995in}}%
\pgfpathlineto{\pgfqpoint{5.702821in}{2.900977in}}%
\pgfpathlineto{\pgfqpoint{5.699649in}{2.901204in}}%
\pgfpathlineto{\pgfqpoint{5.696477in}{2.901794in}}%
\pgfpathlineto{\pgfqpoint{5.693305in}{2.901645in}}%
\pgfpathlineto{\pgfqpoint{5.690133in}{2.901665in}}%
\pgfpathlineto{\pgfqpoint{5.686961in}{2.901710in}}%
\pgfpathlineto{\pgfqpoint{5.683789in}{2.901809in}}%
\pgfpathlineto{\pgfqpoint{5.680617in}{2.901726in}}%
\pgfpathlineto{\pgfqpoint{5.677445in}{2.902000in}}%
\pgfpathlineto{\pgfqpoint{5.674273in}{2.902125in}}%
\pgfpathlineto{\pgfqpoint{5.671101in}{2.902112in}}%
\pgfpathlineto{\pgfqpoint{5.667929in}{2.902183in}}%
\pgfpathlineto{\pgfqpoint{5.664757in}{2.902320in}}%
\pgfpathlineto{\pgfqpoint{5.661585in}{2.902801in}}%
\pgfpathlineto{\pgfqpoint{5.658413in}{2.903230in}}%
\pgfpathlineto{\pgfqpoint{5.655241in}{2.903543in}}%
\pgfpathlineto{\pgfqpoint{5.652069in}{2.903832in}}%
\pgfpathlineto{\pgfqpoint{5.648897in}{2.904153in}}%
\pgfpathlineto{\pgfqpoint{5.645725in}{2.904169in}}%
\pgfpathlineto{\pgfqpoint{5.642553in}{2.904443in}}%
\pgfpathlineto{\pgfqpoint{5.639381in}{2.904300in}}%
\pgfpathlineto{\pgfqpoint{5.636209in}{2.904609in}}%
\pgfpathlineto{\pgfqpoint{5.633037in}{2.904478in}}%
\pgfpathlineto{\pgfqpoint{5.629864in}{2.905034in}}%
\pgfpathlineto{\pgfqpoint{5.626692in}{2.905704in}}%
\pgfpathlineto{\pgfqpoint{5.623520in}{2.906224in}}%
\pgfpathlineto{\pgfqpoint{5.620348in}{2.906455in}}%
\pgfpathlineto{\pgfqpoint{5.617176in}{2.906857in}}%
\pgfpathlineto{\pgfqpoint{5.614004in}{2.907399in}}%
\pgfpathlineto{\pgfqpoint{5.610832in}{2.907438in}}%
\pgfpathlineto{\pgfqpoint{5.607660in}{2.907134in}}%
\pgfpathlineto{\pgfqpoint{5.604488in}{2.907500in}}%
\pgfpathlineto{\pgfqpoint{5.601316in}{2.907846in}}%
\pgfpathlineto{\pgfqpoint{5.598144in}{2.907793in}}%
\pgfpathlineto{\pgfqpoint{5.594972in}{2.908158in}}%
\pgfpathlineto{\pgfqpoint{5.591800in}{2.908807in}}%
\pgfpathlineto{\pgfqpoint{5.588628in}{2.908873in}}%
\pgfpathlineto{\pgfqpoint{5.585456in}{2.908847in}}%
\pgfpathlineto{\pgfqpoint{5.582284in}{2.908753in}}%
\pgfpathlineto{\pgfqpoint{5.579112in}{2.909086in}}%
\pgfpathlineto{\pgfqpoint{5.575940in}{2.909612in}}%
\pgfpathlineto{\pgfqpoint{5.572768in}{2.909845in}}%
\pgfpathlineto{\pgfqpoint{5.569596in}{2.910116in}}%
\pgfpathlineto{\pgfqpoint{5.566424in}{2.909785in}}%
\pgfpathlineto{\pgfqpoint{5.563252in}{2.909898in}}%
\pgfpathlineto{\pgfqpoint{5.560080in}{2.909752in}}%
\pgfpathlineto{\pgfqpoint{5.556908in}{2.910109in}}%
\pgfpathlineto{\pgfqpoint{5.553735in}{2.910184in}}%
\pgfpathlineto{\pgfqpoint{5.550563in}{2.910158in}}%
\pgfpathlineto{\pgfqpoint{5.547391in}{2.909753in}}%
\pgfpathlineto{\pgfqpoint{5.544219in}{2.910181in}}%
\pgfpathlineto{\pgfqpoint{5.541047in}{2.909551in}}%
\pgfpathlineto{\pgfqpoint{5.537875in}{2.910013in}}%
\pgfpathlineto{\pgfqpoint{5.534703in}{2.910227in}}%
\pgfpathlineto{\pgfqpoint{5.531531in}{2.910477in}}%
\pgfpathlineto{\pgfqpoint{5.528359in}{2.910482in}}%
\pgfpathlineto{\pgfqpoint{5.525187in}{2.910269in}}%
\pgfpathlineto{\pgfqpoint{5.522015in}{2.910309in}}%
\pgfpathlineto{\pgfqpoint{5.518843in}{2.910545in}}%
\pgfpathlineto{\pgfqpoint{5.515671in}{2.910845in}}%
\pgfpathlineto{\pgfqpoint{5.512499in}{2.910916in}}%
\pgfpathlineto{\pgfqpoint{5.509327in}{2.910869in}}%
\pgfpathlineto{\pgfqpoint{5.506155in}{2.911468in}}%
\pgfpathlineto{\pgfqpoint{5.502983in}{2.911375in}}%
\pgfpathlineto{\pgfqpoint{5.499811in}{2.910961in}}%
\pgfpathlineto{\pgfqpoint{5.496639in}{2.911110in}}%
\pgfpathlineto{\pgfqpoint{5.493467in}{2.911102in}}%
\pgfpathlineto{\pgfqpoint{5.490295in}{2.911372in}}%
\pgfpathlineto{\pgfqpoint{5.487123in}{2.911726in}}%
\pgfpathlineto{\pgfqpoint{5.483951in}{2.911866in}}%
\pgfpathlineto{\pgfqpoint{5.480779in}{2.911800in}}%
\pgfpathlineto{\pgfqpoint{5.477607in}{2.912254in}}%
\pgfpathlineto{\pgfqpoint{5.474434in}{2.913018in}}%
\pgfpathlineto{\pgfqpoint{5.471262in}{2.913833in}}%
\pgfpathlineto{\pgfqpoint{5.468090in}{2.914083in}}%
\pgfpathlineto{\pgfqpoint{5.464918in}{2.914027in}}%
\pgfpathlineto{\pgfqpoint{5.461746in}{2.914025in}}%
\pgfpathlineto{\pgfqpoint{5.458574in}{2.914137in}}%
\pgfpathlineto{\pgfqpoint{5.455402in}{2.914354in}}%
\pgfpathlineto{\pgfqpoint{5.452230in}{2.914124in}}%
\pgfpathlineto{\pgfqpoint{5.449058in}{2.914323in}}%
\pgfpathlineto{\pgfqpoint{5.445886in}{2.914440in}}%
\pgfpathlineto{\pgfqpoint{5.442714in}{2.915126in}}%
\pgfpathlineto{\pgfqpoint{5.439542in}{2.915023in}}%
\pgfpathlineto{\pgfqpoint{5.436370in}{2.914948in}}%
\pgfpathlineto{\pgfqpoint{5.433198in}{2.914636in}}%
\pgfpathlineto{\pgfqpoint{5.430026in}{2.915007in}}%
\pgfpathlineto{\pgfqpoint{5.426854in}{2.914944in}}%
\pgfpathlineto{\pgfqpoint{5.423682in}{2.915457in}}%
\pgfpathlineto{\pgfqpoint{5.420510in}{2.915535in}}%
\pgfpathlineto{\pgfqpoint{5.417338in}{2.915217in}}%
\pgfpathlineto{\pgfqpoint{5.414166in}{2.915078in}}%
\pgfpathlineto{\pgfqpoint{5.410994in}{2.915169in}}%
\pgfpathlineto{\pgfqpoint{5.407822in}{2.915165in}}%
\pgfpathlineto{\pgfqpoint{5.404650in}{2.915383in}}%
\pgfpathlineto{\pgfqpoint{5.401478in}{2.915483in}}%
\pgfpathlineto{\pgfqpoint{5.398306in}{2.915659in}}%
\pgfpathlineto{\pgfqpoint{5.395133in}{2.915484in}}%
\pgfpathlineto{\pgfqpoint{5.391961in}{2.915522in}}%
\pgfpathlineto{\pgfqpoint{5.388789in}{2.915906in}}%
\pgfpathlineto{\pgfqpoint{5.385617in}{2.916104in}}%
\pgfpathlineto{\pgfqpoint{5.382445in}{2.916179in}}%
\pgfpathlineto{\pgfqpoint{5.379273in}{2.916657in}}%
\pgfpathlineto{\pgfqpoint{5.376101in}{2.916410in}}%
\pgfpathlineto{\pgfqpoint{5.372929in}{2.916353in}}%
\pgfpathlineto{\pgfqpoint{5.369757in}{2.916134in}}%
\pgfpathlineto{\pgfqpoint{5.366585in}{2.916033in}}%
\pgfpathlineto{\pgfqpoint{5.363413in}{2.916400in}}%
\pgfpathlineto{\pgfqpoint{5.360241in}{2.916551in}}%
\pgfpathlineto{\pgfqpoint{5.357069in}{2.917181in}}%
\pgfpathlineto{\pgfqpoint{5.353897in}{2.917230in}}%
\pgfpathlineto{\pgfqpoint{5.350725in}{2.917214in}}%
\pgfpathlineto{\pgfqpoint{5.347553in}{2.917261in}}%
\pgfpathlineto{\pgfqpoint{5.344381in}{2.916860in}}%
\pgfpathlineto{\pgfqpoint{5.341209in}{2.916859in}}%
\pgfpathlineto{\pgfqpoint{5.338037in}{2.917162in}}%
\pgfpathlineto{\pgfqpoint{5.334865in}{2.917079in}}%
\pgfpathlineto{\pgfqpoint{5.331693in}{2.916948in}}%
\pgfpathlineto{\pgfqpoint{5.328521in}{2.917050in}}%
\pgfpathlineto{\pgfqpoint{5.325349in}{2.917056in}}%
\pgfpathlineto{\pgfqpoint{5.322177in}{2.917429in}}%
\pgfpathlineto{\pgfqpoint{5.319004in}{2.917385in}}%
\pgfpathlineto{\pgfqpoint{5.315832in}{2.917485in}}%
\pgfpathlineto{\pgfqpoint{5.312660in}{2.917412in}}%
\pgfpathlineto{\pgfqpoint{5.309488in}{2.917386in}}%
\pgfpathlineto{\pgfqpoint{5.306316in}{2.917356in}}%
\pgfpathlineto{\pgfqpoint{5.303144in}{2.917372in}}%
\pgfpathlineto{\pgfqpoint{5.299972in}{2.917593in}}%
\pgfpathlineto{\pgfqpoint{5.296800in}{2.917659in}}%
\pgfpathlineto{\pgfqpoint{5.293628in}{2.917668in}}%
\pgfpathlineto{\pgfqpoint{5.290456in}{2.917678in}}%
\pgfpathlineto{\pgfqpoint{5.287284in}{2.918017in}}%
\pgfpathlineto{\pgfqpoint{5.284112in}{2.918291in}}%
\pgfpathlineto{\pgfqpoint{5.280940in}{2.917988in}}%
\pgfpathlineto{\pgfqpoint{5.277768in}{2.918304in}}%
\pgfpathlineto{\pgfqpoint{5.274596in}{2.918639in}}%
\pgfpathlineto{\pgfqpoint{5.271424in}{2.919053in}}%
\pgfpathlineto{\pgfqpoint{5.268252in}{2.918858in}}%
\pgfpathlineto{\pgfqpoint{5.265080in}{2.918695in}}%
\pgfpathlineto{\pgfqpoint{5.261908in}{2.918819in}}%
\pgfpathlineto{\pgfqpoint{5.258736in}{2.918817in}}%
\pgfpathlineto{\pgfqpoint{5.255564in}{2.918678in}}%
\pgfpathlineto{\pgfqpoint{5.252392in}{2.918343in}}%
\pgfpathlineto{\pgfqpoint{5.249220in}{2.918340in}}%
\pgfpathlineto{\pgfqpoint{5.246048in}{2.918496in}}%
\pgfpathlineto{\pgfqpoint{5.242876in}{2.918740in}}%
\pgfpathlineto{\pgfqpoint{5.239703in}{2.918889in}}%
\pgfpathlineto{\pgfqpoint{5.236531in}{2.918968in}}%
\pgfpathlineto{\pgfqpoint{5.233359in}{2.919123in}}%
\pgfpathlineto{\pgfqpoint{5.230187in}{2.919321in}}%
\pgfpathlineto{\pgfqpoint{5.227015in}{2.919660in}}%
\pgfpathlineto{\pgfqpoint{5.223843in}{2.919883in}}%
\pgfpathlineto{\pgfqpoint{5.220671in}{2.920105in}}%
\pgfpathlineto{\pgfqpoint{5.217499in}{2.920203in}}%
\pgfpathlineto{\pgfqpoint{5.214327in}{2.920611in}}%
\pgfpathlineto{\pgfqpoint{5.211155in}{2.920468in}}%
\pgfpathlineto{\pgfqpoint{5.207983in}{2.919935in}}%
\pgfpathlineto{\pgfqpoint{5.204811in}{2.920174in}}%
\pgfpathlineto{\pgfqpoint{5.201639in}{2.920543in}}%
\pgfpathlineto{\pgfqpoint{5.198467in}{2.920957in}}%
\pgfpathlineto{\pgfqpoint{5.195295in}{2.921415in}}%
\pgfpathlineto{\pgfqpoint{5.192123in}{2.921998in}}%
\pgfpathlineto{\pgfqpoint{5.188951in}{2.922109in}}%
\pgfpathlineto{\pgfqpoint{5.185779in}{2.922023in}}%
\pgfpathlineto{\pgfqpoint{5.182607in}{2.922047in}}%
\pgfpathlineto{\pgfqpoint{5.179435in}{2.922240in}}%
\pgfpathlineto{\pgfqpoint{5.176263in}{2.922201in}}%
\pgfpathlineto{\pgfqpoint{5.173091in}{2.922271in}}%
\pgfpathlineto{\pgfqpoint{5.169919in}{2.922661in}}%
\pgfpathlineto{\pgfqpoint{5.166747in}{2.922526in}}%
\pgfpathlineto{\pgfqpoint{5.163575in}{2.922575in}}%
\pgfpathlineto{\pgfqpoint{5.160402in}{2.922991in}}%
\pgfpathlineto{\pgfqpoint{5.157230in}{2.923305in}}%
\pgfpathlineto{\pgfqpoint{5.154058in}{2.923159in}}%
\pgfpathlineto{\pgfqpoint{5.150886in}{2.923128in}}%
\pgfpathlineto{\pgfqpoint{5.147714in}{2.923103in}}%
\pgfpathlineto{\pgfqpoint{5.144542in}{2.923148in}}%
\pgfpathlineto{\pgfqpoint{5.141370in}{2.923385in}}%
\pgfpathlineto{\pgfqpoint{5.138198in}{2.923468in}}%
\pgfpathlineto{\pgfqpoint{5.135026in}{2.923326in}}%
\pgfpathlineto{\pgfqpoint{5.131854in}{2.923494in}}%
\pgfpathlineto{\pgfqpoint{5.128682in}{2.923045in}}%
\pgfpathlineto{\pgfqpoint{5.125510in}{2.923115in}}%
\pgfpathlineto{\pgfqpoint{5.122338in}{2.923625in}}%
\pgfpathlineto{\pgfqpoint{5.119166in}{2.923656in}}%
\pgfpathlineto{\pgfqpoint{5.115994in}{2.923766in}}%
\pgfpathlineto{\pgfqpoint{5.112822in}{2.923608in}}%
\pgfpathlineto{\pgfqpoint{5.109650in}{2.923935in}}%
\pgfpathlineto{\pgfqpoint{5.106478in}{2.923725in}}%
\pgfpathlineto{\pgfqpoint{5.103306in}{2.923360in}}%
\pgfpathlineto{\pgfqpoint{5.100134in}{2.923421in}}%
\pgfpathlineto{\pgfqpoint{5.096962in}{2.923477in}}%
\pgfpathlineto{\pgfqpoint{5.093790in}{2.923271in}}%
\pgfpathlineto{\pgfqpoint{5.090618in}{2.923283in}}%
\pgfpathlineto{\pgfqpoint{5.087446in}{2.923474in}}%
\pgfpathlineto{\pgfqpoint{5.084273in}{2.923620in}}%
\pgfpathlineto{\pgfqpoint{5.081101in}{2.924180in}}%
\pgfpathlineto{\pgfqpoint{5.077929in}{2.924086in}}%
\pgfpathlineto{\pgfqpoint{5.074757in}{2.924126in}}%
\pgfpathlineto{\pgfqpoint{5.071585in}{2.924206in}}%
\pgfpathlineto{\pgfqpoint{5.068413in}{2.924621in}}%
\pgfpathlineto{\pgfqpoint{5.065241in}{2.924659in}}%
\pgfpathlineto{\pgfqpoint{5.062069in}{2.924823in}}%
\pgfpathlineto{\pgfqpoint{5.058897in}{2.924425in}}%
\pgfpathlineto{\pgfqpoint{5.055725in}{2.924622in}}%
\pgfpathlineto{\pgfqpoint{5.052553in}{2.924808in}}%
\pgfpathlineto{\pgfqpoint{5.049381in}{2.925224in}}%
\pgfpathlineto{\pgfqpoint{5.046209in}{2.925800in}}%
\pgfpathlineto{\pgfqpoint{5.043037in}{2.925999in}}%
\pgfpathlineto{\pgfqpoint{5.039865in}{2.926069in}}%
\pgfpathlineto{\pgfqpoint{5.036693in}{2.926149in}}%
\pgfpathlineto{\pgfqpoint{5.033521in}{2.926414in}}%
\pgfpathlineto{\pgfqpoint{5.030349in}{2.926754in}}%
\pgfpathlineto{\pgfqpoint{5.027177in}{2.926503in}}%
\pgfpathlineto{\pgfqpoint{5.024005in}{2.926660in}}%
\pgfpathlineto{\pgfqpoint{5.020833in}{2.926901in}}%
\pgfpathlineto{\pgfqpoint{5.017661in}{2.927049in}}%
\pgfpathlineto{\pgfqpoint{5.014489in}{2.927701in}}%
\pgfpathlineto{\pgfqpoint{5.011317in}{2.927917in}}%
\pgfpathlineto{\pgfqpoint{5.008145in}{2.928597in}}%
\pgfpathlineto{\pgfqpoint{5.004972in}{2.929116in}}%
\pgfpathlineto{\pgfqpoint{5.001800in}{2.929065in}}%
\pgfpathlineto{\pgfqpoint{4.998628in}{2.929111in}}%
\pgfpathlineto{\pgfqpoint{4.995456in}{2.929090in}}%
\pgfpathlineto{\pgfqpoint{4.992284in}{2.929064in}}%
\pgfpathlineto{\pgfqpoint{4.989112in}{2.929643in}}%
\pgfpathlineto{\pgfqpoint{4.985940in}{2.930097in}}%
\pgfpathlineto{\pgfqpoint{4.982768in}{2.930303in}}%
\pgfpathlineto{\pgfqpoint{4.979596in}{2.930240in}}%
\pgfpathlineto{\pgfqpoint{4.976424in}{2.929868in}}%
\pgfpathlineto{\pgfqpoint{4.973252in}{2.929893in}}%
\pgfpathlineto{\pgfqpoint{4.970080in}{2.930316in}}%
\pgfpathlineto{\pgfqpoint{4.966908in}{2.930465in}}%
\pgfpathlineto{\pgfqpoint{4.963736in}{2.930254in}}%
\pgfpathlineto{\pgfqpoint{4.960564in}{2.930412in}}%
\pgfpathlineto{\pgfqpoint{4.957392in}{2.930418in}}%
\pgfpathlineto{\pgfqpoint{4.954220in}{2.930496in}}%
\pgfpathlineto{\pgfqpoint{4.951048in}{2.930614in}}%
\pgfpathlineto{\pgfqpoint{4.947876in}{2.930685in}}%
\pgfpathlineto{\pgfqpoint{4.944704in}{2.930929in}}%
\pgfpathlineto{\pgfqpoint{4.941532in}{2.931007in}}%
\pgfpathlineto{\pgfqpoint{4.938360in}{2.931119in}}%
\pgfpathlineto{\pgfqpoint{4.935188in}{2.931087in}}%
\pgfpathlineto{\pgfqpoint{4.932016in}{2.930776in}}%
\pgfpathlineto{\pgfqpoint{4.928844in}{2.930907in}}%
\pgfpathlineto{\pgfqpoint{4.925671in}{2.930917in}}%
\pgfpathlineto{\pgfqpoint{4.922499in}{2.930993in}}%
\pgfpathlineto{\pgfqpoint{4.919327in}{2.930994in}}%
\pgfpathlineto{\pgfqpoint{4.916155in}{2.930822in}}%
\pgfpathlineto{\pgfqpoint{4.912983in}{2.930551in}}%
\pgfpathlineto{\pgfqpoint{4.909811in}{2.930369in}}%
\pgfpathlineto{\pgfqpoint{4.906639in}{2.930291in}}%
\pgfpathlineto{\pgfqpoint{4.903467in}{2.930503in}}%
\pgfpathlineto{\pgfqpoint{4.900295in}{2.930650in}}%
\pgfpathlineto{\pgfqpoint{4.897123in}{2.930615in}}%
\pgfpathlineto{\pgfqpoint{4.893951in}{2.930582in}}%
\pgfpathlineto{\pgfqpoint{4.890779in}{2.930836in}}%
\pgfpathlineto{\pgfqpoint{4.887607in}{2.930901in}}%
\pgfpathlineto{\pgfqpoint{4.884435in}{2.930664in}}%
\pgfpathlineto{\pgfqpoint{4.881263in}{2.931036in}}%
\pgfpathlineto{\pgfqpoint{4.878091in}{2.931072in}}%
\pgfpathlineto{\pgfqpoint{4.874919in}{2.930905in}}%
\pgfpathlineto{\pgfqpoint{4.871747in}{2.930872in}}%
\pgfpathlineto{\pgfqpoint{4.868575in}{2.931045in}}%
\pgfpathlineto{\pgfqpoint{4.865403in}{2.931334in}}%
\pgfpathlineto{\pgfqpoint{4.862231in}{2.931190in}}%
\pgfpathlineto{\pgfqpoint{4.859059in}{2.931181in}}%
\pgfpathlineto{\pgfqpoint{4.855887in}{2.931440in}}%
\pgfpathlineto{\pgfqpoint{4.852715in}{2.931649in}}%
\pgfpathlineto{\pgfqpoint{4.849542in}{2.931771in}}%
\pgfpathlineto{\pgfqpoint{4.846370in}{2.931763in}}%
\pgfpathlineto{\pgfqpoint{4.843198in}{2.931662in}}%
\pgfpathlineto{\pgfqpoint{4.840026in}{2.931752in}}%
\pgfpathlineto{\pgfqpoint{4.836854in}{2.931690in}}%
\pgfpathlineto{\pgfqpoint{4.833682in}{2.931453in}}%
\pgfpathlineto{\pgfqpoint{4.830510in}{2.931517in}}%
\pgfpathlineto{\pgfqpoint{4.827338in}{2.931351in}}%
\pgfpathlineto{\pgfqpoint{4.824166in}{2.931188in}}%
\pgfpathlineto{\pgfqpoint{4.820994in}{2.931058in}}%
\pgfpathlineto{\pgfqpoint{4.817822in}{2.931506in}}%
\pgfpathlineto{\pgfqpoint{4.814650in}{2.931667in}}%
\pgfpathlineto{\pgfqpoint{4.811478in}{2.931708in}}%
\pgfpathlineto{\pgfqpoint{4.808306in}{2.931833in}}%
\pgfpathlineto{\pgfqpoint{4.805134in}{2.931834in}}%
\pgfpathlineto{\pgfqpoint{4.801962in}{2.931917in}}%
\pgfpathlineto{\pgfqpoint{4.798790in}{2.932258in}}%
\pgfpathlineto{\pgfqpoint{4.795618in}{2.932391in}}%
\pgfpathlineto{\pgfqpoint{4.792446in}{2.933053in}}%
\pgfpathlineto{\pgfqpoint{4.789274in}{2.933324in}}%
\pgfpathlineto{\pgfqpoint{4.786102in}{2.933487in}}%
\pgfpathlineto{\pgfqpoint{4.782930in}{2.933495in}}%
\pgfpathlineto{\pgfqpoint{4.779758in}{2.934000in}}%
\pgfpathlineto{\pgfqpoint{4.776586in}{2.934129in}}%
\pgfpathlineto{\pgfqpoint{4.773414in}{2.934377in}}%
\pgfpathlineto{\pgfqpoint{4.770241in}{2.934224in}}%
\pgfpathlineto{\pgfqpoint{4.767069in}{2.934406in}}%
\pgfpathlineto{\pgfqpoint{4.763897in}{2.934536in}}%
\pgfpathlineto{\pgfqpoint{4.760725in}{2.934739in}}%
\pgfpathlineto{\pgfqpoint{4.757553in}{2.935042in}}%
\pgfpathlineto{\pgfqpoint{4.754381in}{2.935130in}}%
\pgfpathlineto{\pgfqpoint{4.751209in}{2.935328in}}%
\pgfpathlineto{\pgfqpoint{4.748037in}{2.935458in}}%
\pgfpathlineto{\pgfqpoint{4.744865in}{2.935699in}}%
\pgfpathlineto{\pgfqpoint{4.741693in}{2.935673in}}%
\pgfpathlineto{\pgfqpoint{4.738521in}{2.936030in}}%
\pgfpathlineto{\pgfqpoint{4.735349in}{2.936519in}}%
\pgfpathlineto{\pgfqpoint{4.732177in}{2.936398in}}%
\pgfpathlineto{\pgfqpoint{4.729005in}{2.936502in}}%
\pgfpathlineto{\pgfqpoint{4.725833in}{2.937378in}}%
\pgfpathlineto{\pgfqpoint{4.722661in}{2.937532in}}%
\pgfpathlineto{\pgfqpoint{4.719489in}{2.937441in}}%
\pgfpathlineto{\pgfqpoint{4.716317in}{2.938111in}}%
\pgfpathlineto{\pgfqpoint{4.713145in}{2.938356in}}%
\pgfpathlineto{\pgfqpoint{4.709973in}{2.938413in}}%
\pgfpathlineto{\pgfqpoint{4.706801in}{2.938686in}}%
\pgfpathlineto{\pgfqpoint{4.703629in}{2.938979in}}%
\pgfpathlineto{\pgfqpoint{4.700457in}{2.939151in}}%
\pgfpathlineto{\pgfqpoint{4.697285in}{2.939543in}}%
\pgfpathlineto{\pgfqpoint{4.694112in}{2.940180in}}%
\pgfpathlineto{\pgfqpoint{4.690940in}{2.940541in}}%
\pgfpathlineto{\pgfqpoint{4.687768in}{2.940399in}}%
\pgfpathlineto{\pgfqpoint{4.684596in}{2.940399in}}%
\pgfpathlineto{\pgfqpoint{4.681424in}{2.940820in}}%
\pgfpathlineto{\pgfqpoint{4.678252in}{2.940903in}}%
\pgfpathlineto{\pgfqpoint{4.675080in}{2.941291in}}%
\pgfpathlineto{\pgfqpoint{4.671908in}{2.941478in}}%
\pgfpathlineto{\pgfqpoint{4.668736in}{2.941513in}}%
\pgfpathlineto{\pgfqpoint{4.665564in}{2.941587in}}%
\pgfpathlineto{\pgfqpoint{4.662392in}{2.941101in}}%
\pgfpathlineto{\pgfqpoint{4.659220in}{2.941189in}}%
\pgfpathlineto{\pgfqpoint{4.656048in}{2.941374in}}%
\pgfpathlineto{\pgfqpoint{4.652876in}{2.941575in}}%
\pgfpathlineto{\pgfqpoint{4.649704in}{2.941942in}}%
\pgfpathlineto{\pgfqpoint{4.646532in}{2.942410in}}%
\pgfpathlineto{\pgfqpoint{4.643360in}{2.942560in}}%
\pgfpathlineto{\pgfqpoint{4.640188in}{2.943098in}}%
\pgfpathlineto{\pgfqpoint{4.637016in}{2.943465in}}%
\pgfpathlineto{\pgfqpoint{4.633844in}{2.943595in}}%
\pgfpathlineto{\pgfqpoint{4.630672in}{2.943897in}}%
\pgfpathlineto{\pgfqpoint{4.627500in}{2.944228in}}%
\pgfpathlineto{\pgfqpoint{4.624328in}{2.944344in}}%
\pgfpathlineto{\pgfqpoint{4.621156in}{2.944564in}}%
\pgfpathlineto{\pgfqpoint{4.617984in}{2.944698in}}%
\pgfpathlineto{\pgfqpoint{4.614811in}{2.945275in}}%
\pgfpathlineto{\pgfqpoint{4.611639in}{2.945698in}}%
\pgfpathlineto{\pgfqpoint{4.608467in}{2.945696in}}%
\pgfpathlineto{\pgfqpoint{4.605295in}{2.945774in}}%
\pgfpathlineto{\pgfqpoint{4.602123in}{2.946217in}}%
\pgfpathlineto{\pgfqpoint{4.598951in}{2.946488in}}%
\pgfpathlineto{\pgfqpoint{4.595779in}{2.946750in}}%
\pgfpathlineto{\pgfqpoint{4.592607in}{2.946520in}}%
\pgfpathlineto{\pgfqpoint{4.589435in}{2.946587in}}%
\pgfpathlineto{\pgfqpoint{4.586263in}{2.946538in}}%
\pgfpathlineto{\pgfqpoint{4.583091in}{2.946634in}}%
\pgfpathlineto{\pgfqpoint{4.579919in}{2.946654in}}%
\pgfpathlineto{\pgfqpoint{4.576747in}{2.947124in}}%
\pgfpathlineto{\pgfqpoint{4.573575in}{2.947247in}}%
\pgfpathlineto{\pgfqpoint{4.570403in}{2.947524in}}%
\pgfpathlineto{\pgfqpoint{4.567231in}{2.947585in}}%
\pgfpathlineto{\pgfqpoint{4.564059in}{2.947447in}}%
\pgfpathlineto{\pgfqpoint{4.560887in}{2.947433in}}%
\pgfpathlineto{\pgfqpoint{4.557715in}{2.947404in}}%
\pgfpathlineto{\pgfqpoint{4.554543in}{2.947765in}}%
\pgfpathlineto{\pgfqpoint{4.551371in}{2.948004in}}%
\pgfpathlineto{\pgfqpoint{4.548199in}{2.948149in}}%
\pgfpathlineto{\pgfqpoint{4.545027in}{2.948338in}}%
\pgfpathlineto{\pgfqpoint{4.541855in}{2.948366in}}%
\pgfpathlineto{\pgfqpoint{4.538683in}{2.948479in}}%
\pgfpathlineto{\pgfqpoint{4.535510in}{2.948322in}}%
\pgfpathlineto{\pgfqpoint{4.532338in}{2.948035in}}%
\pgfpathlineto{\pgfqpoint{4.529166in}{2.947943in}}%
\pgfpathlineto{\pgfqpoint{4.525994in}{2.948102in}}%
\pgfpathlineto{\pgfqpoint{4.522822in}{2.948142in}}%
\pgfpathlineto{\pgfqpoint{4.519650in}{2.948135in}}%
\pgfpathlineto{\pgfqpoint{4.516478in}{2.948017in}}%
\pgfpathlineto{\pgfqpoint{4.513306in}{2.948281in}}%
\pgfpathlineto{\pgfqpoint{4.510134in}{2.948462in}}%
\pgfpathlineto{\pgfqpoint{4.506962in}{2.948474in}}%
\pgfpathlineto{\pgfqpoint{4.503790in}{2.948583in}}%
\pgfpathlineto{\pgfqpoint{4.500618in}{2.948544in}}%
\pgfpathlineto{\pgfqpoint{4.497446in}{2.948534in}}%
\pgfpathlineto{\pgfqpoint{4.494274in}{2.949007in}}%
\pgfpathlineto{\pgfqpoint{4.491102in}{2.949245in}}%
\pgfpathlineto{\pgfqpoint{4.487930in}{2.949278in}}%
\pgfpathlineto{\pgfqpoint{4.484758in}{2.949042in}}%
\pgfpathlineto{\pgfqpoint{4.481586in}{2.948945in}}%
\pgfpathlineto{\pgfqpoint{4.478414in}{2.949458in}}%
\pgfpathlineto{\pgfqpoint{4.475242in}{2.949788in}}%
\pgfpathlineto{\pgfqpoint{4.472070in}{2.949984in}}%
\pgfpathlineto{\pgfqpoint{4.468898in}{2.950123in}}%
\pgfpathlineto{\pgfqpoint{4.465726in}{2.950145in}}%
\pgfpathlineto{\pgfqpoint{4.462554in}{2.950268in}}%
\pgfpathlineto{\pgfqpoint{4.459381in}{2.950298in}}%
\pgfpathlineto{\pgfqpoint{4.456209in}{2.950583in}}%
\pgfpathlineto{\pgfqpoint{4.453037in}{2.950442in}}%
\pgfpathlineto{\pgfqpoint{4.449865in}{2.950308in}}%
\pgfpathlineto{\pgfqpoint{4.446693in}{2.950519in}}%
\pgfpathlineto{\pgfqpoint{4.443521in}{2.950815in}}%
\pgfpathlineto{\pgfqpoint{4.440349in}{2.950830in}}%
\pgfpathlineto{\pgfqpoint{4.437177in}{2.950546in}}%
\pgfpathlineto{\pgfqpoint{4.434005in}{2.950650in}}%
\pgfpathlineto{\pgfqpoint{4.430833in}{2.950473in}}%
\pgfpathlineto{\pgfqpoint{4.427661in}{2.951362in}}%
\pgfpathlineto{\pgfqpoint{4.424489in}{2.951345in}}%
\pgfpathlineto{\pgfqpoint{4.421317in}{2.951506in}}%
\pgfpathlineto{\pgfqpoint{4.418145in}{2.951794in}}%
\pgfpathlineto{\pgfqpoint{4.414973in}{2.951895in}}%
\pgfpathlineto{\pgfqpoint{4.411801in}{2.951925in}}%
\pgfpathlineto{\pgfqpoint{4.408629in}{2.951766in}}%
\pgfpathlineto{\pgfqpoint{4.405457in}{2.951906in}}%
\pgfpathlineto{\pgfqpoint{4.402285in}{2.952150in}}%
\pgfpathlineto{\pgfqpoint{4.399113in}{2.952227in}}%
\pgfpathlineto{\pgfqpoint{4.395941in}{2.952265in}}%
\pgfpathlineto{\pgfqpoint{4.392769in}{2.952644in}}%
\pgfpathlineto{\pgfqpoint{4.389597in}{2.952862in}}%
\pgfpathlineto{\pgfqpoint{4.386425in}{2.953183in}}%
\pgfpathlineto{\pgfqpoint{4.383253in}{2.953210in}}%
\pgfpathlineto{\pgfqpoint{4.380080in}{2.953286in}}%
\pgfpathlineto{\pgfqpoint{4.376908in}{2.953345in}}%
\pgfpathlineto{\pgfqpoint{4.373736in}{2.953319in}}%
\pgfpathlineto{\pgfqpoint{4.370564in}{2.953439in}}%
\pgfpathlineto{\pgfqpoint{4.367392in}{2.954185in}}%
\pgfpathlineto{\pgfqpoint{4.364220in}{2.954239in}}%
\pgfpathlineto{\pgfqpoint{4.361048in}{2.954475in}}%
\pgfpathlineto{\pgfqpoint{4.357876in}{2.954639in}}%
\pgfpathlineto{\pgfqpoint{4.354704in}{2.955211in}}%
\pgfpathlineto{\pgfqpoint{4.351532in}{2.955617in}}%
\pgfpathlineto{\pgfqpoint{4.348360in}{2.955916in}}%
\pgfpathlineto{\pgfqpoint{4.345188in}{2.956290in}}%
\pgfpathlineto{\pgfqpoint{4.342016in}{2.956803in}}%
\pgfpathlineto{\pgfqpoint{4.338844in}{2.956864in}}%
\pgfpathlineto{\pgfqpoint{4.335672in}{2.956697in}}%
\pgfpathlineto{\pgfqpoint{4.332500in}{2.956379in}}%
\pgfpathlineto{\pgfqpoint{4.329328in}{2.955969in}}%
\pgfpathlineto{\pgfqpoint{4.326156in}{2.956307in}}%
\pgfpathlineto{\pgfqpoint{4.322984in}{2.956594in}}%
\pgfpathlineto{\pgfqpoint{4.319812in}{2.956785in}}%
\pgfpathlineto{\pgfqpoint{4.316640in}{2.956671in}}%
\pgfpathlineto{\pgfqpoint{4.313468in}{2.956743in}}%
\pgfpathlineto{\pgfqpoint{4.310296in}{2.956338in}}%
\pgfpathlineto{\pgfqpoint{4.307124in}{2.956282in}}%
\pgfpathlineto{\pgfqpoint{4.303952in}{2.956016in}}%
\pgfpathlineto{\pgfqpoint{4.300779in}{2.955899in}}%
\pgfpathlineto{\pgfqpoint{4.297607in}{2.955714in}}%
\pgfpathlineto{\pgfqpoint{4.294435in}{2.955530in}}%
\pgfpathlineto{\pgfqpoint{4.291263in}{2.955553in}}%
\pgfpathlineto{\pgfqpoint{4.288091in}{2.955831in}}%
\pgfpathlineto{\pgfqpoint{4.284919in}{2.955988in}}%
\pgfpathlineto{\pgfqpoint{4.281747in}{2.955915in}}%
\pgfpathlineto{\pgfqpoint{4.278575in}{2.956312in}}%
\pgfpathlineto{\pgfqpoint{4.275403in}{2.956214in}}%
\pgfpathlineto{\pgfqpoint{4.272231in}{2.956587in}}%
\pgfpathlineto{\pgfqpoint{4.269059in}{2.956624in}}%
\pgfpathlineto{\pgfqpoint{4.265887in}{2.956545in}}%
\pgfpathlineto{\pgfqpoint{4.262715in}{2.956261in}}%
\pgfpathlineto{\pgfqpoint{4.259543in}{2.956388in}}%
\pgfpathlineto{\pgfqpoint{4.256371in}{2.956485in}}%
\pgfpathlineto{\pgfqpoint{4.253199in}{2.956616in}}%
\pgfpathlineto{\pgfqpoint{4.250027in}{2.956853in}}%
\pgfpathlineto{\pgfqpoint{4.246855in}{2.957244in}}%
\pgfpathlineto{\pgfqpoint{4.243683in}{2.957317in}}%
\pgfpathlineto{\pgfqpoint{4.240511in}{2.957490in}}%
\pgfpathlineto{\pgfqpoint{4.237339in}{2.957780in}}%
\pgfpathlineto{\pgfqpoint{4.234167in}{2.958118in}}%
\pgfpathlineto{\pgfqpoint{4.230995in}{2.957961in}}%
\pgfpathlineto{\pgfqpoint{4.227823in}{2.957838in}}%
\pgfpathlineto{\pgfqpoint{4.224650in}{2.958207in}}%
\pgfpathlineto{\pgfqpoint{4.221478in}{2.958503in}}%
\pgfpathlineto{\pgfqpoint{4.218306in}{2.958660in}}%
\pgfpathlineto{\pgfqpoint{4.215134in}{2.958875in}}%
\pgfpathlineto{\pgfqpoint{4.211962in}{2.959268in}}%
\pgfpathlineto{\pgfqpoint{4.208790in}{2.960216in}}%
\pgfpathlineto{\pgfqpoint{4.205618in}{2.960628in}}%
\pgfpathlineto{\pgfqpoint{4.202446in}{2.960926in}}%
\pgfpathlineto{\pgfqpoint{4.199274in}{2.961061in}}%
\pgfpathlineto{\pgfqpoint{4.196102in}{2.961204in}}%
\pgfpathlineto{\pgfqpoint{4.192930in}{2.960879in}}%
\pgfpathlineto{\pgfqpoint{4.189758in}{2.960742in}}%
\pgfpathlineto{\pgfqpoint{4.186586in}{2.961243in}}%
\pgfpathlineto{\pgfqpoint{4.183414in}{2.961837in}}%
\pgfpathlineto{\pgfqpoint{4.180242in}{2.961850in}}%
\pgfpathlineto{\pgfqpoint{4.177070in}{2.962368in}}%
\pgfpathlineto{\pgfqpoint{4.173898in}{2.962194in}}%
\pgfpathlineto{\pgfqpoint{4.170726in}{2.962445in}}%
\pgfpathlineto{\pgfqpoint{4.167554in}{2.962565in}}%
\pgfpathlineto{\pgfqpoint{4.164382in}{2.962447in}}%
\pgfpathlineto{\pgfqpoint{4.161210in}{2.962317in}}%
\pgfpathlineto{\pgfqpoint{4.158038in}{2.962539in}}%
\pgfpathlineto{\pgfqpoint{4.154866in}{2.962391in}}%
\pgfpathlineto{\pgfqpoint{4.151694in}{2.962450in}}%
\pgfpathlineto{\pgfqpoint{4.148522in}{2.962421in}}%
\pgfpathlineto{\pgfqpoint{4.145349in}{2.962814in}}%
\pgfpathlineto{\pgfqpoint{4.142177in}{2.962866in}}%
\pgfpathlineto{\pgfqpoint{4.139005in}{2.963030in}}%
\pgfpathlineto{\pgfqpoint{4.135833in}{2.963055in}}%
\pgfpathlineto{\pgfqpoint{4.132661in}{2.963662in}}%
\pgfpathlineto{\pgfqpoint{4.129489in}{2.963764in}}%
\pgfpathlineto{\pgfqpoint{4.126317in}{2.964135in}}%
\pgfpathlineto{\pgfqpoint{4.123145in}{2.964086in}}%
\pgfpathlineto{\pgfqpoint{4.119973in}{2.964336in}}%
\pgfpathlineto{\pgfqpoint{4.116801in}{2.964265in}}%
\pgfpathlineto{\pgfqpoint{4.113629in}{2.964376in}}%
\pgfpathlineto{\pgfqpoint{4.110457in}{2.964361in}}%
\pgfpathlineto{\pgfqpoint{4.107285in}{2.964339in}}%
\pgfpathlineto{\pgfqpoint{4.104113in}{2.964064in}}%
\pgfpathlineto{\pgfqpoint{4.100941in}{2.964194in}}%
\pgfpathlineto{\pgfqpoint{4.097769in}{2.964091in}}%
\pgfpathlineto{\pgfqpoint{4.094597in}{2.964627in}}%
\pgfpathlineto{\pgfqpoint{4.091425in}{2.964655in}}%
\pgfpathlineto{\pgfqpoint{4.088253in}{2.965408in}}%
\pgfpathlineto{\pgfqpoint{4.085081in}{2.965548in}}%
\pgfpathlineto{\pgfqpoint{4.081909in}{2.965843in}}%
\pgfpathlineto{\pgfqpoint{4.078737in}{2.965752in}}%
\pgfpathlineto{\pgfqpoint{4.075565in}{2.965823in}}%
\pgfpathlineto{\pgfqpoint{4.072393in}{2.965714in}}%
\pgfpathlineto{\pgfqpoint{4.069221in}{2.965596in}}%
\pgfpathlineto{\pgfqpoint{4.066048in}{2.965788in}}%
\pgfpathlineto{\pgfqpoint{4.062876in}{2.965860in}}%
\pgfpathlineto{\pgfqpoint{4.059704in}{2.965880in}}%
\pgfpathlineto{\pgfqpoint{4.056532in}{2.965878in}}%
\pgfpathlineto{\pgfqpoint{4.053360in}{2.966095in}}%
\pgfpathlineto{\pgfqpoint{4.050188in}{2.966443in}}%
\pgfpathlineto{\pgfqpoint{4.047016in}{2.966244in}}%
\pgfpathlineto{\pgfqpoint{4.043844in}{2.966003in}}%
\pgfpathlineto{\pgfqpoint{4.040672in}{2.966039in}}%
\pgfpathlineto{\pgfqpoint{4.037500in}{2.966236in}}%
\pgfpathlineto{\pgfqpoint{4.034328in}{2.966623in}}%
\pgfpathlineto{\pgfqpoint{4.031156in}{2.966565in}}%
\pgfpathlineto{\pgfqpoint{4.027984in}{2.966383in}}%
\pgfpathlineto{\pgfqpoint{4.024812in}{2.966292in}}%
\pgfpathlineto{\pgfqpoint{4.021640in}{2.966347in}}%
\pgfpathlineto{\pgfqpoint{4.018468in}{2.966101in}}%
\pgfpathlineto{\pgfqpoint{4.015296in}{2.966082in}}%
\pgfpathlineto{\pgfqpoint{4.012124in}{2.966224in}}%
\pgfpathlineto{\pgfqpoint{4.008952in}{2.966177in}}%
\pgfpathlineto{\pgfqpoint{4.005780in}{2.966108in}}%
\pgfpathlineto{\pgfqpoint{4.002608in}{2.966105in}}%
\pgfpathlineto{\pgfqpoint{3.999436in}{2.966008in}}%
\pgfpathlineto{\pgfqpoint{3.996264in}{2.965941in}}%
\pgfpathlineto{\pgfqpoint{3.993092in}{2.966103in}}%
\pgfpathlineto{\pgfqpoint{3.989919in}{2.965828in}}%
\pgfpathlineto{\pgfqpoint{3.986747in}{2.965994in}}%
\pgfpathlineto{\pgfqpoint{3.983575in}{2.965950in}}%
\pgfpathlineto{\pgfqpoint{3.980403in}{2.966088in}}%
\pgfpathlineto{\pgfqpoint{3.977231in}{2.966164in}}%
\pgfpathlineto{\pgfqpoint{3.974059in}{2.966247in}}%
\pgfpathlineto{\pgfqpoint{3.970887in}{2.966429in}}%
\pgfpathlineto{\pgfqpoint{3.967715in}{2.966239in}}%
\pgfpathlineto{\pgfqpoint{3.964543in}{2.966427in}}%
\pgfpathlineto{\pgfqpoint{3.961371in}{2.966337in}}%
\pgfpathlineto{\pgfqpoint{3.958199in}{2.966374in}}%
\pgfpathlineto{\pgfqpoint{3.955027in}{2.966077in}}%
\pgfpathlineto{\pgfqpoint{3.951855in}{2.966374in}}%
\pgfpathlineto{\pgfqpoint{3.948683in}{2.966508in}}%
\pgfpathlineto{\pgfqpoint{3.945511in}{2.966568in}}%
\pgfpathlineto{\pgfqpoint{3.942339in}{2.966817in}}%
\pgfpathlineto{\pgfqpoint{3.939167in}{2.966880in}}%
\pgfpathlineto{\pgfqpoint{3.935995in}{2.967238in}}%
\pgfpathlineto{\pgfqpoint{3.932823in}{2.967565in}}%
\pgfpathlineto{\pgfqpoint{3.929651in}{2.967496in}}%
\pgfpathlineto{\pgfqpoint{3.926479in}{2.967414in}}%
\pgfpathlineto{\pgfqpoint{3.923307in}{2.967434in}}%
\pgfpathlineto{\pgfqpoint{3.920135in}{2.966983in}}%
\pgfpathlineto{\pgfqpoint{3.916963in}{2.967314in}}%
\pgfpathlineto{\pgfqpoint{3.913791in}{2.967379in}}%
\pgfpathlineto{\pgfqpoint{3.910618in}{2.967521in}}%
\pgfpathlineto{\pgfqpoint{3.907446in}{2.967467in}}%
\pgfpathlineto{\pgfqpoint{3.904274in}{2.967454in}}%
\pgfpathlineto{\pgfqpoint{3.901102in}{2.967101in}}%
\pgfpathlineto{\pgfqpoint{3.897930in}{2.967409in}}%
\pgfpathlineto{\pgfqpoint{3.894758in}{2.967467in}}%
\pgfpathlineto{\pgfqpoint{3.891586in}{2.967546in}}%
\pgfpathlineto{\pgfqpoint{3.888414in}{2.967521in}}%
\pgfpathlineto{\pgfqpoint{3.885242in}{2.967090in}}%
\pgfpathlineto{\pgfqpoint{3.882070in}{2.967472in}}%
\pgfpathlineto{\pgfqpoint{3.878898in}{2.967565in}}%
\pgfpathlineto{\pgfqpoint{3.875726in}{2.967384in}}%
\pgfpathlineto{\pgfqpoint{3.872554in}{2.967374in}}%
\pgfpathlineto{\pgfqpoint{3.869382in}{2.967164in}}%
\pgfpathlineto{\pgfqpoint{3.866210in}{2.967077in}}%
\pgfpathlineto{\pgfqpoint{3.863038in}{2.967471in}}%
\pgfpathlineto{\pgfqpoint{3.859866in}{2.967698in}}%
\pgfpathlineto{\pgfqpoint{3.856694in}{2.967935in}}%
\pgfpathlineto{\pgfqpoint{3.853522in}{2.967817in}}%
\pgfpathlineto{\pgfqpoint{3.850350in}{2.967995in}}%
\pgfpathlineto{\pgfqpoint{3.847178in}{2.967994in}}%
\pgfpathlineto{\pgfqpoint{3.844006in}{2.967882in}}%
\pgfpathlineto{\pgfqpoint{3.840834in}{2.967703in}}%
\pgfpathlineto{\pgfqpoint{3.837662in}{2.967671in}}%
\pgfpathlineto{\pgfqpoint{3.834490in}{2.967829in}}%
\pgfpathlineto{\pgfqpoint{3.831317in}{2.967686in}}%
\pgfpathlineto{\pgfqpoint{3.828145in}{2.967900in}}%
\pgfpathlineto{\pgfqpoint{3.824973in}{2.967774in}}%
\pgfpathlineto{\pgfqpoint{3.821801in}{2.967874in}}%
\pgfpathlineto{\pgfqpoint{3.818629in}{2.967923in}}%
\pgfpathlineto{\pgfqpoint{3.815457in}{2.968397in}}%
\pgfpathlineto{\pgfqpoint{3.812285in}{2.968680in}}%
\pgfpathlineto{\pgfqpoint{3.809113in}{2.968677in}}%
\pgfpathlineto{\pgfqpoint{3.805941in}{2.968898in}}%
\pgfpathlineto{\pgfqpoint{3.802769in}{2.969322in}}%
\pgfpathlineto{\pgfqpoint{3.799597in}{2.969171in}}%
\pgfpathlineto{\pgfqpoint{3.796425in}{2.969387in}}%
\pgfpathlineto{\pgfqpoint{3.793253in}{2.969847in}}%
\pgfpathlineto{\pgfqpoint{3.790081in}{2.969893in}}%
\pgfpathlineto{\pgfqpoint{3.786909in}{2.970039in}}%
\pgfpathlineto{\pgfqpoint{3.783737in}{2.970352in}}%
\pgfpathlineto{\pgfqpoint{3.780565in}{2.970605in}}%
\pgfpathlineto{\pgfqpoint{3.777393in}{2.970975in}}%
\pgfpathlineto{\pgfqpoint{3.774221in}{2.970716in}}%
\pgfpathlineto{\pgfqpoint{3.771049in}{2.970792in}}%
\pgfpathlineto{\pgfqpoint{3.767877in}{2.971197in}}%
\pgfpathlineto{\pgfqpoint{3.764705in}{2.971807in}}%
\pgfpathlineto{\pgfqpoint{3.761533in}{2.971922in}}%
\pgfpathlineto{\pgfqpoint{3.758361in}{2.971642in}}%
\pgfpathlineto{\pgfqpoint{3.755188in}{2.972292in}}%
\pgfpathlineto{\pgfqpoint{3.752016in}{2.972556in}}%
\pgfpathlineto{\pgfqpoint{3.748844in}{2.973138in}}%
\pgfpathlineto{\pgfqpoint{3.745672in}{2.973348in}}%
\pgfpathlineto{\pgfqpoint{3.742500in}{2.973682in}}%
\pgfpathlineto{\pgfqpoint{3.739328in}{2.974149in}}%
\pgfpathlineto{\pgfqpoint{3.736156in}{2.974277in}}%
\pgfpathlineto{\pgfqpoint{3.732984in}{2.974571in}}%
\pgfpathlineto{\pgfqpoint{3.729812in}{2.974713in}}%
\pgfpathlineto{\pgfqpoint{3.726640in}{2.974813in}}%
\pgfpathlineto{\pgfqpoint{3.723468in}{2.975340in}}%
\pgfpathlineto{\pgfqpoint{3.720296in}{2.975467in}}%
\pgfpathlineto{\pgfqpoint{3.717124in}{2.975963in}}%
\pgfpathlineto{\pgfqpoint{3.713952in}{2.976020in}}%
\pgfpathlineto{\pgfqpoint{3.710780in}{2.976422in}}%
\pgfpathlineto{\pgfqpoint{3.707608in}{2.976971in}}%
\pgfpathlineto{\pgfqpoint{3.704436in}{2.977453in}}%
\pgfpathlineto{\pgfqpoint{3.701264in}{2.977632in}}%
\pgfpathlineto{\pgfqpoint{3.698092in}{2.977620in}}%
\pgfpathlineto{\pgfqpoint{3.694920in}{2.977956in}}%
\pgfpathlineto{\pgfqpoint{3.691748in}{2.977982in}}%
\pgfpathlineto{\pgfqpoint{3.688576in}{2.978336in}}%
\pgfpathlineto{\pgfqpoint{3.685404in}{2.978535in}}%
\pgfpathlineto{\pgfqpoint{3.682232in}{2.978555in}}%
\pgfpathlineto{\pgfqpoint{3.679060in}{2.978247in}}%
\pgfpathlineto{\pgfqpoint{3.675887in}{2.978454in}}%
\pgfpathlineto{\pgfqpoint{3.672715in}{2.978579in}}%
\pgfpathlineto{\pgfqpoint{3.669543in}{2.978589in}}%
\pgfpathlineto{\pgfqpoint{3.666371in}{2.978958in}}%
\pgfpathlineto{\pgfqpoint{3.663199in}{2.979023in}}%
\pgfpathlineto{\pgfqpoint{3.660027in}{2.979570in}}%
\pgfpathlineto{\pgfqpoint{3.656855in}{2.980064in}}%
\pgfpathlineto{\pgfqpoint{3.653683in}{2.980655in}}%
\pgfpathlineto{\pgfqpoint{3.650511in}{2.980756in}}%
\pgfpathlineto{\pgfqpoint{3.647339in}{2.980728in}}%
\pgfpathlineto{\pgfqpoint{3.644167in}{2.980779in}}%
\pgfpathlineto{\pgfqpoint{3.640995in}{2.980825in}}%
\pgfpathlineto{\pgfqpoint{3.637823in}{2.981122in}}%
\pgfpathlineto{\pgfqpoint{3.634651in}{2.981333in}}%
\pgfpathlineto{\pgfqpoint{3.631479in}{2.981277in}}%
\pgfpathlineto{\pgfqpoint{3.628307in}{2.981017in}}%
\pgfpathlineto{\pgfqpoint{3.625135in}{2.980775in}}%
\pgfpathlineto{\pgfqpoint{3.621963in}{2.980686in}}%
\pgfpathlineto{\pgfqpoint{3.618791in}{2.980637in}}%
\pgfpathlineto{\pgfqpoint{3.615619in}{2.980485in}}%
\pgfpathlineto{\pgfqpoint{3.612447in}{2.980293in}}%
\pgfpathlineto{\pgfqpoint{3.609275in}{2.980334in}}%
\pgfpathlineto{\pgfqpoint{3.606103in}{2.980638in}}%
\pgfpathlineto{\pgfqpoint{3.602931in}{2.980575in}}%
\pgfpathlineto{\pgfqpoint{3.599759in}{2.980747in}}%
\pgfpathlineto{\pgfqpoint{3.596586in}{2.980747in}}%
\pgfpathlineto{\pgfqpoint{3.593414in}{2.980709in}}%
\pgfpathlineto{\pgfqpoint{3.590242in}{2.980424in}}%
\pgfpathlineto{\pgfqpoint{3.587070in}{2.980209in}}%
\pgfpathlineto{\pgfqpoint{3.583898in}{2.979837in}}%
\pgfpathlineto{\pgfqpoint{3.580726in}{2.980048in}}%
\pgfpathlineto{\pgfqpoint{3.577554in}{2.980388in}}%
\pgfpathlineto{\pgfqpoint{3.574382in}{2.980402in}}%
\pgfpathlineto{\pgfqpoint{3.571210in}{2.980767in}}%
\pgfpathlineto{\pgfqpoint{3.568038in}{2.980661in}}%
\pgfpathlineto{\pgfqpoint{3.564866in}{2.981094in}}%
\pgfpathlineto{\pgfqpoint{3.561694in}{2.981366in}}%
\pgfpathlineto{\pgfqpoint{3.558522in}{2.982224in}}%
\pgfpathlineto{\pgfqpoint{3.555350in}{2.982508in}}%
\pgfpathlineto{\pgfqpoint{3.552178in}{2.982798in}}%
\pgfpathlineto{\pgfqpoint{3.549006in}{2.982809in}}%
\pgfpathlineto{\pgfqpoint{3.545834in}{2.982668in}}%
\pgfpathlineto{\pgfqpoint{3.542662in}{2.982612in}}%
\pgfpathlineto{\pgfqpoint{3.539490in}{2.983116in}}%
\pgfpathlineto{\pgfqpoint{3.536318in}{2.983103in}}%
\pgfpathlineto{\pgfqpoint{3.533146in}{2.983245in}}%
\pgfpathlineto{\pgfqpoint{3.529974in}{2.983604in}}%
\pgfpathlineto{\pgfqpoint{3.526802in}{2.983708in}}%
\pgfpathlineto{\pgfqpoint{3.523630in}{2.983598in}}%
\pgfpathlineto{\pgfqpoint{3.520457in}{2.983372in}}%
\pgfpathlineto{\pgfqpoint{3.517285in}{2.983803in}}%
\pgfpathlineto{\pgfqpoint{3.514113in}{2.984035in}}%
\pgfpathlineto{\pgfqpoint{3.510941in}{2.984528in}}%
\pgfpathlineto{\pgfqpoint{3.507769in}{2.984630in}}%
\pgfpathlineto{\pgfqpoint{3.504597in}{2.984664in}}%
\pgfpathlineto{\pgfqpoint{3.501425in}{2.984773in}}%
\pgfpathlineto{\pgfqpoint{3.498253in}{2.985284in}}%
\pgfpathlineto{\pgfqpoint{3.495081in}{2.985518in}}%
\pgfpathlineto{\pgfqpoint{3.491909in}{2.985469in}}%
\pgfpathlineto{\pgfqpoint{3.488737in}{2.985432in}}%
\pgfpathlineto{\pgfqpoint{3.485565in}{2.985026in}}%
\pgfpathlineto{\pgfqpoint{3.482393in}{2.984765in}}%
\pgfpathlineto{\pgfqpoint{3.479221in}{2.984749in}}%
\pgfpathlineto{\pgfqpoint{3.476049in}{2.984886in}}%
\pgfpathlineto{\pgfqpoint{3.472877in}{2.984820in}}%
\pgfpathlineto{\pgfqpoint{3.469705in}{2.984630in}}%
\pgfpathlineto{\pgfqpoint{3.466533in}{2.984909in}}%
\pgfpathlineto{\pgfqpoint{3.463361in}{2.985720in}}%
\pgfpathlineto{\pgfqpoint{3.460189in}{2.985809in}}%
\pgfpathlineto{\pgfqpoint{3.457017in}{2.985659in}}%
\pgfpathlineto{\pgfqpoint{3.453845in}{2.985384in}}%
\pgfpathlineto{\pgfqpoint{3.450673in}{2.985337in}}%
\pgfpathlineto{\pgfqpoint{3.447501in}{2.985304in}}%
\pgfpathlineto{\pgfqpoint{3.444329in}{2.985667in}}%
\pgfpathlineto{\pgfqpoint{3.441156in}{2.985881in}}%
\pgfpathlineto{\pgfqpoint{3.437984in}{2.986040in}}%
\pgfpathlineto{\pgfqpoint{3.434812in}{2.986247in}}%
\pgfpathlineto{\pgfqpoint{3.431640in}{2.986399in}}%
\pgfpathlineto{\pgfqpoint{3.428468in}{2.986588in}}%
\pgfpathlineto{\pgfqpoint{3.425296in}{2.986718in}}%
\pgfpathlineto{\pgfqpoint{3.422124in}{2.986818in}}%
\pgfpathlineto{\pgfqpoint{3.418952in}{2.986577in}}%
\pgfpathlineto{\pgfqpoint{3.415780in}{2.986628in}}%
\pgfpathlineto{\pgfqpoint{3.412608in}{2.986879in}}%
\pgfpathlineto{\pgfqpoint{3.409436in}{2.987019in}}%
\pgfpathlineto{\pgfqpoint{3.406264in}{2.986961in}}%
\pgfpathlineto{\pgfqpoint{3.403092in}{2.987023in}}%
\pgfpathlineto{\pgfqpoint{3.399920in}{2.987609in}}%
\pgfpathlineto{\pgfqpoint{3.396748in}{2.987752in}}%
\pgfpathlineto{\pgfqpoint{3.393576in}{2.988147in}}%
\pgfpathlineto{\pgfqpoint{3.390404in}{2.987916in}}%
\pgfpathlineto{\pgfqpoint{3.387232in}{2.988051in}}%
\pgfpathlineto{\pgfqpoint{3.384060in}{2.988225in}}%
\pgfpathlineto{\pgfqpoint{3.380888in}{2.988006in}}%
\pgfpathlineto{\pgfqpoint{3.377716in}{2.987826in}}%
\pgfpathlineto{\pgfqpoint{3.374544in}{2.987754in}}%
\pgfpathlineto{\pgfqpoint{3.371372in}{2.988164in}}%
\pgfpathlineto{\pgfqpoint{3.368200in}{2.988832in}}%
\pgfpathlineto{\pgfqpoint{3.365028in}{2.988682in}}%
\pgfpathlineto{\pgfqpoint{3.361855in}{2.988437in}}%
\pgfpathlineto{\pgfqpoint{3.358683in}{2.988710in}}%
\pgfpathlineto{\pgfqpoint{3.355511in}{2.988598in}}%
\pgfpathlineto{\pgfqpoint{3.352339in}{2.988606in}}%
\pgfpathlineto{\pgfqpoint{3.349167in}{2.989218in}}%
\pgfpathlineto{\pgfqpoint{3.345995in}{2.989935in}}%
\pgfpathlineto{\pgfqpoint{3.342823in}{2.989534in}}%
\pgfpathlineto{\pgfqpoint{3.339651in}{2.989387in}}%
\pgfpathlineto{\pgfqpoint{3.336479in}{2.989448in}}%
\pgfpathlineto{\pgfqpoint{3.333307in}{2.990059in}}%
\pgfpathlineto{\pgfqpoint{3.330135in}{2.990025in}}%
\pgfpathlineto{\pgfqpoint{3.326963in}{2.990169in}}%
\pgfpathlineto{\pgfqpoint{3.323791in}{2.990283in}}%
\pgfpathlineto{\pgfqpoint{3.320619in}{2.990408in}}%
\pgfpathlineto{\pgfqpoint{3.317447in}{2.990338in}}%
\pgfpathlineto{\pgfqpoint{3.314275in}{2.990471in}}%
\pgfpathlineto{\pgfqpoint{3.311103in}{2.990041in}}%
\pgfpathlineto{\pgfqpoint{3.307931in}{2.990179in}}%
\pgfpathlineto{\pgfqpoint{3.304759in}{2.990322in}}%
\pgfpathlineto{\pgfqpoint{3.301587in}{2.990787in}}%
\pgfpathlineto{\pgfqpoint{3.298415in}{2.991168in}}%
\pgfpathlineto{\pgfqpoint{3.295243in}{2.991920in}}%
\pgfpathlineto{\pgfqpoint{3.292071in}{2.992554in}}%
\pgfpathlineto{\pgfqpoint{3.288899in}{2.992985in}}%
\pgfpathlineto{\pgfqpoint{3.285726in}{2.992577in}}%
\pgfpathlineto{\pgfqpoint{3.282554in}{2.993125in}}%
\pgfpathlineto{\pgfqpoint{3.279382in}{2.992913in}}%
\pgfpathlineto{\pgfqpoint{3.276210in}{2.992987in}}%
\pgfpathlineto{\pgfqpoint{3.273038in}{2.992856in}}%
\pgfpathlineto{\pgfqpoint{3.269866in}{2.993173in}}%
\pgfpathlineto{\pgfqpoint{3.266694in}{2.992979in}}%
\pgfpathlineto{\pgfqpoint{3.263522in}{2.993136in}}%
\pgfpathlineto{\pgfqpoint{3.260350in}{2.993490in}}%
\pgfpathlineto{\pgfqpoint{3.257178in}{2.993768in}}%
\pgfpathlineto{\pgfqpoint{3.254006in}{2.993449in}}%
\pgfpathlineto{\pgfqpoint{3.250834in}{2.993324in}}%
\pgfpathlineto{\pgfqpoint{3.247662in}{2.993325in}}%
\pgfpathlineto{\pgfqpoint{3.244490in}{2.993448in}}%
\pgfpathlineto{\pgfqpoint{3.241318in}{2.994963in}}%
\pgfpathlineto{\pgfqpoint{3.238146in}{2.995062in}}%
\pgfpathlineto{\pgfqpoint{3.234974in}{2.995710in}}%
\pgfpathlineto{\pgfqpoint{3.231802in}{2.996277in}}%
\pgfpathlineto{\pgfqpoint{3.228630in}{2.996440in}}%
\pgfpathlineto{\pgfqpoint{3.225458in}{2.997065in}}%
\pgfpathlineto{\pgfqpoint{3.222286in}{2.997026in}}%
\pgfpathlineto{\pgfqpoint{3.219114in}{2.996944in}}%
\pgfpathlineto{\pgfqpoint{3.215942in}{2.997160in}}%
\pgfpathlineto{\pgfqpoint{3.212770in}{2.997137in}}%
\pgfpathlineto{\pgfqpoint{3.209598in}{2.997511in}}%
\pgfpathlineto{\pgfqpoint{3.206425in}{2.998664in}}%
\pgfpathlineto{\pgfqpoint{3.203253in}{2.999161in}}%
\pgfpathlineto{\pgfqpoint{3.200081in}{2.999457in}}%
\pgfpathlineto{\pgfqpoint{3.196909in}{2.999476in}}%
\pgfpathlineto{\pgfqpoint{3.193737in}{2.999798in}}%
\pgfpathlineto{\pgfqpoint{3.190565in}{3.000097in}}%
\pgfpathlineto{\pgfqpoint{3.187393in}{3.000370in}}%
\pgfpathlineto{\pgfqpoint{3.184221in}{3.000590in}}%
\pgfpathlineto{\pgfqpoint{3.181049in}{3.000438in}}%
\pgfpathlineto{\pgfqpoint{3.177877in}{3.000448in}}%
\pgfpathlineto{\pgfqpoint{3.174705in}{3.000530in}}%
\pgfpathlineto{\pgfqpoint{3.171533in}{3.001117in}}%
\pgfpathlineto{\pgfqpoint{3.168361in}{3.001330in}}%
\pgfpathlineto{\pgfqpoint{3.165189in}{3.001877in}}%
\pgfpathlineto{\pgfqpoint{3.162017in}{3.002113in}}%
\pgfpathlineto{\pgfqpoint{3.158845in}{3.002192in}}%
\pgfpathlineto{\pgfqpoint{3.155673in}{3.002293in}}%
\pgfpathlineto{\pgfqpoint{3.152501in}{3.002359in}}%
\pgfpathlineto{\pgfqpoint{3.149329in}{3.000698in}}%
\pgfpathlineto{\pgfqpoint{3.146157in}{3.000538in}}%
\pgfpathlineto{\pgfqpoint{3.142985in}{3.000996in}}%
\pgfpathlineto{\pgfqpoint{3.139813in}{3.001304in}}%
\pgfpathlineto{\pgfqpoint{3.136641in}{3.001364in}}%
\pgfpathlineto{\pgfqpoint{3.133469in}{3.001326in}}%
\pgfpathlineto{\pgfqpoint{3.130297in}{3.001334in}}%
\pgfpathlineto{\pgfqpoint{3.127124in}{3.001411in}}%
\pgfpathlineto{\pgfqpoint{3.123952in}{3.001358in}}%
\pgfpathlineto{\pgfqpoint{3.120780in}{3.001368in}}%
\pgfpathlineto{\pgfqpoint{3.117608in}{3.001319in}}%
\pgfpathlineto{\pgfqpoint{3.114436in}{3.001377in}}%
\pgfpathlineto{\pgfqpoint{3.111264in}{3.001265in}}%
\pgfpathlineto{\pgfqpoint{3.108092in}{3.001267in}}%
\pgfpathlineto{\pgfqpoint{3.104920in}{3.001401in}}%
\pgfpathlineto{\pgfqpoint{3.101748in}{3.001471in}}%
\pgfpathlineto{\pgfqpoint{3.098576in}{3.001528in}}%
\pgfpathlineto{\pgfqpoint{3.095404in}{3.001433in}}%
\pgfpathlineto{\pgfqpoint{3.092232in}{3.001329in}}%
\pgfpathlineto{\pgfqpoint{3.089060in}{3.001393in}}%
\pgfpathlineto{\pgfqpoint{3.085888in}{3.001473in}}%
\pgfpathlineto{\pgfqpoint{3.082716in}{3.001372in}}%
\pgfpathlineto{\pgfqpoint{3.079544in}{3.001385in}}%
\pgfpathlineto{\pgfqpoint{3.076372in}{3.001290in}}%
\pgfpathlineto{\pgfqpoint{3.073200in}{3.001313in}}%
\pgfpathlineto{\pgfqpoint{3.070028in}{3.001302in}}%
\pgfpathlineto{\pgfqpoint{3.066856in}{3.001463in}}%
\pgfpathlineto{\pgfqpoint{3.063684in}{3.001479in}}%
\pgfpathlineto{\pgfqpoint{3.060512in}{3.001556in}}%
\pgfpathlineto{\pgfqpoint{3.057340in}{3.001507in}}%
\pgfpathlineto{\pgfqpoint{3.054168in}{3.001620in}}%
\pgfpathlineto{\pgfqpoint{3.050995in}{3.001651in}}%
\pgfpathlineto{\pgfqpoint{3.047823in}{3.001293in}}%
\pgfpathlineto{\pgfqpoint{3.044651in}{3.001428in}}%
\pgfpathlineto{\pgfqpoint{3.041479in}{3.001460in}}%
\pgfpathlineto{\pgfqpoint{3.038307in}{3.001418in}}%
\pgfpathlineto{\pgfqpoint{3.035135in}{3.001271in}}%
\pgfpathlineto{\pgfqpoint{3.031963in}{3.001222in}}%
\pgfpathlineto{\pgfqpoint{3.028791in}{3.001369in}}%
\pgfpathlineto{\pgfqpoint{3.025619in}{3.001296in}}%
\pgfpathlineto{\pgfqpoint{3.022447in}{3.001249in}}%
\pgfpathlineto{\pgfqpoint{3.019275in}{3.001228in}}%
\pgfpathlineto{\pgfqpoint{3.016103in}{3.001405in}}%
\pgfpathlineto{\pgfqpoint{3.012931in}{3.001489in}}%
\pgfpathlineto{\pgfqpoint{3.009759in}{3.001656in}}%
\pgfpathlineto{\pgfqpoint{3.006587in}{3.001619in}}%
\pgfpathlineto{\pgfqpoint{3.003415in}{3.001541in}}%
\pgfpathlineto{\pgfqpoint{3.000243in}{3.001594in}}%
\pgfpathlineto{\pgfqpoint{2.997071in}{3.001669in}}%
\pgfpathlineto{\pgfqpoint{2.993899in}{3.001519in}}%
\pgfpathlineto{\pgfqpoint{2.990727in}{3.001482in}}%
\pgfpathlineto{\pgfqpoint{2.987555in}{3.001445in}}%
\pgfpathlineto{\pgfqpoint{2.984383in}{3.001367in}}%
\pgfpathlineto{\pgfqpoint{2.981211in}{3.001435in}}%
\pgfpathlineto{\pgfqpoint{2.978039in}{3.001231in}}%
\pgfpathlineto{\pgfqpoint{2.974867in}{3.001160in}}%
\pgfpathlineto{\pgfqpoint{2.971694in}{3.001202in}}%
\pgfpathlineto{\pgfqpoint{2.968522in}{3.001282in}}%
\pgfpathlineto{\pgfqpoint{2.965350in}{3.001458in}}%
\pgfpathlineto{\pgfqpoint{2.962178in}{3.001497in}}%
\pgfpathlineto{\pgfqpoint{2.959006in}{3.001536in}}%
\pgfpathlineto{\pgfqpoint{2.955834in}{3.001918in}}%
\pgfpathlineto{\pgfqpoint{2.952662in}{3.001842in}}%
\pgfpathlineto{\pgfqpoint{2.949490in}{3.001819in}}%
\pgfpathlineto{\pgfqpoint{2.946318in}{3.002012in}}%
\pgfpathlineto{\pgfqpoint{2.943146in}{3.001985in}}%
\pgfpathlineto{\pgfqpoint{2.939974in}{3.001971in}}%
\pgfpathlineto{\pgfqpoint{2.936802in}{3.001854in}}%
\pgfpathlineto{\pgfqpoint{2.933630in}{3.002055in}}%
\pgfpathlineto{\pgfqpoint{2.930458in}{3.002117in}}%
\pgfpathlineto{\pgfqpoint{2.927286in}{3.002047in}}%
\pgfpathlineto{\pgfqpoint{2.924114in}{3.001972in}}%
\pgfpathlineto{\pgfqpoint{2.920942in}{3.001899in}}%
\pgfpathlineto{\pgfqpoint{2.917770in}{3.001831in}}%
\pgfpathlineto{\pgfqpoint{2.914598in}{3.001756in}}%
\pgfpathlineto{\pgfqpoint{2.911426in}{3.001710in}}%
\pgfpathlineto{\pgfqpoint{2.908254in}{3.001522in}}%
\pgfpathlineto{\pgfqpoint{2.905082in}{3.001589in}}%
\pgfpathlineto{\pgfqpoint{2.901910in}{3.001745in}}%
\pgfpathlineto{\pgfqpoint{2.898738in}{3.001819in}}%
\pgfpathlineto{\pgfqpoint{2.895565in}{3.001788in}}%
\pgfpathlineto{\pgfqpoint{2.892393in}{3.001660in}}%
\pgfpathlineto{\pgfqpoint{2.889221in}{3.001729in}}%
\pgfpathlineto{\pgfqpoint{2.886049in}{3.001534in}}%
\pgfpathlineto{\pgfqpoint{2.882877in}{3.001658in}}%
\pgfpathlineto{\pgfqpoint{2.879705in}{3.001730in}}%
\pgfpathlineto{\pgfqpoint{2.876533in}{3.001710in}}%
\pgfpathlineto{\pgfqpoint{2.873361in}{3.001566in}}%
\pgfpathlineto{\pgfqpoint{2.870189in}{3.001609in}}%
\pgfpathlineto{\pgfqpoint{2.867017in}{3.001664in}}%
\pgfpathlineto{\pgfqpoint{2.863845in}{3.001711in}}%
\pgfpathlineto{\pgfqpoint{2.860673in}{3.001822in}}%
\pgfpathlineto{\pgfqpoint{2.857501in}{3.001597in}}%
\pgfpathlineto{\pgfqpoint{2.854329in}{3.001661in}}%
\pgfpathlineto{\pgfqpoint{2.851157in}{3.001826in}}%
\pgfpathlineto{\pgfqpoint{2.847985in}{3.002024in}}%
\pgfpathlineto{\pgfqpoint{2.844813in}{3.002150in}}%
\pgfpathlineto{\pgfqpoint{2.841641in}{3.002252in}}%
\pgfpathlineto{\pgfqpoint{2.838469in}{3.002277in}}%
\pgfpathlineto{\pgfqpoint{2.835297in}{3.002212in}}%
\pgfpathlineto{\pgfqpoint{2.832125in}{3.002316in}}%
\pgfpathlineto{\pgfqpoint{2.828953in}{3.002314in}}%
\pgfpathlineto{\pgfqpoint{2.825781in}{3.001893in}}%
\pgfpathlineto{\pgfqpoint{2.822609in}{3.001739in}}%
\pgfpathlineto{\pgfqpoint{2.819437in}{3.001754in}}%
\pgfpathlineto{\pgfqpoint{2.816264in}{3.001537in}}%
\pgfpathlineto{\pgfqpoint{2.813092in}{3.001541in}}%
\pgfpathlineto{\pgfqpoint{2.809920in}{3.001478in}}%
\pgfpathlineto{\pgfqpoint{2.806748in}{3.001424in}}%
\pgfpathlineto{\pgfqpoint{2.803576in}{3.001479in}}%
\pgfpathlineto{\pgfqpoint{2.800404in}{3.001529in}}%
\pgfpathlineto{\pgfqpoint{2.797232in}{3.001431in}}%
\pgfpathlineto{\pgfqpoint{2.794060in}{3.001462in}}%
\pgfpathlineto{\pgfqpoint{2.790888in}{3.001404in}}%
\pgfpathlineto{\pgfqpoint{2.787716in}{3.001363in}}%
\pgfpathlineto{\pgfqpoint{2.784544in}{3.001416in}}%
\pgfpathlineto{\pgfqpoint{2.781372in}{3.001498in}}%
\pgfpathlineto{\pgfqpoint{2.778200in}{3.001528in}}%
\pgfpathlineto{\pgfqpoint{2.775028in}{3.001642in}}%
\pgfpathlineto{\pgfqpoint{2.771856in}{3.001766in}}%
\pgfpathlineto{\pgfqpoint{2.768684in}{3.001742in}}%
\pgfpathlineto{\pgfqpoint{2.765512in}{3.001713in}}%
\pgfpathlineto{\pgfqpoint{2.762340in}{3.001700in}}%
\pgfpathlineto{\pgfqpoint{2.759168in}{3.001854in}}%
\pgfpathlineto{\pgfqpoint{2.755996in}{3.001928in}}%
\pgfpathlineto{\pgfqpoint{2.752824in}{3.001955in}}%
\pgfpathlineto{\pgfqpoint{2.749652in}{3.001938in}}%
\pgfpathlineto{\pgfqpoint{2.746480in}{3.002117in}}%
\pgfpathlineto{\pgfqpoint{2.743308in}{3.002228in}}%
\pgfpathlineto{\pgfqpoint{2.740136in}{3.002273in}}%
\pgfpathlineto{\pgfqpoint{2.736963in}{3.002173in}}%
\pgfpathlineto{\pgfqpoint{2.733791in}{3.002190in}}%
\pgfpathlineto{\pgfqpoint{2.730619in}{3.002302in}}%
\pgfpathlineto{\pgfqpoint{2.727447in}{3.002217in}}%
\pgfpathlineto{\pgfqpoint{2.724275in}{3.002133in}}%
\pgfpathlineto{\pgfqpoint{2.721103in}{3.002025in}}%
\pgfpathlineto{\pgfqpoint{2.717931in}{3.002119in}}%
\pgfpathlineto{\pgfqpoint{2.714759in}{3.002054in}}%
\pgfpathlineto{\pgfqpoint{2.711587in}{3.002104in}}%
\pgfpathlineto{\pgfqpoint{2.708415in}{3.001772in}}%
\pgfpathlineto{\pgfqpoint{2.705243in}{3.001689in}}%
\pgfpathlineto{\pgfqpoint{2.702071in}{3.001713in}}%
\pgfpathlineto{\pgfqpoint{2.698899in}{3.002025in}}%
\pgfpathlineto{\pgfqpoint{2.695727in}{3.001953in}}%
\pgfpathlineto{\pgfqpoint{2.692555in}{3.002046in}}%
\pgfpathlineto{\pgfqpoint{2.689383in}{3.002028in}}%
\pgfpathlineto{\pgfqpoint{2.686211in}{3.002032in}}%
\pgfpathlineto{\pgfqpoint{2.683039in}{3.001979in}}%
\pgfpathlineto{\pgfqpoint{2.679867in}{3.001881in}}%
\pgfpathlineto{\pgfqpoint{2.676695in}{3.001879in}}%
\pgfpathlineto{\pgfqpoint{2.673523in}{3.001820in}}%
\pgfpathlineto{\pgfqpoint{2.670351in}{3.001883in}}%
\pgfpathlineto{\pgfqpoint{2.667179in}{3.001845in}}%
\pgfpathlineto{\pgfqpoint{2.664007in}{3.001668in}}%
\pgfpathlineto{\pgfqpoint{2.660834in}{3.001530in}}%
\pgfpathlineto{\pgfqpoint{2.657662in}{3.001409in}}%
\pgfpathlineto{\pgfqpoint{2.654490in}{3.001549in}}%
\pgfpathlineto{\pgfqpoint{2.651318in}{3.001659in}}%
\pgfpathlineto{\pgfqpoint{2.648146in}{3.001546in}}%
\pgfpathlineto{\pgfqpoint{2.644974in}{3.001515in}}%
\pgfpathlineto{\pgfqpoint{2.641802in}{3.001235in}}%
\pgfpathlineto{\pgfqpoint{2.638630in}{3.000811in}}%
\pgfpathlineto{\pgfqpoint{2.635458in}{3.000636in}}%
\pgfpathlineto{\pgfqpoint{2.632286in}{3.000530in}}%
\pgfpathlineto{\pgfqpoint{2.629114in}{3.000751in}}%
\pgfpathlineto{\pgfqpoint{2.625942in}{3.000949in}}%
\pgfpathlineto{\pgfqpoint{2.622770in}{3.000843in}}%
\pgfpathlineto{\pgfqpoint{2.619598in}{3.001050in}}%
\pgfpathlineto{\pgfqpoint{2.616426in}{3.001204in}}%
\pgfpathlineto{\pgfqpoint{2.613254in}{3.001242in}}%
\pgfpathlineto{\pgfqpoint{2.610082in}{3.001211in}}%
\pgfpathlineto{\pgfqpoint{2.606910in}{3.000977in}}%
\pgfpathlineto{\pgfqpoint{2.603738in}{3.001048in}}%
\pgfpathlineto{\pgfqpoint{2.600566in}{3.000990in}}%
\pgfpathlineto{\pgfqpoint{2.597394in}{3.001028in}}%
\pgfpathlineto{\pgfqpoint{2.594222in}{3.001113in}}%
\pgfpathlineto{\pgfqpoint{2.591050in}{3.001047in}}%
\pgfpathlineto{\pgfqpoint{2.587878in}{3.000898in}}%
\pgfpathlineto{\pgfqpoint{2.584706in}{3.000797in}}%
\pgfpathlineto{\pgfqpoint{2.581533in}{3.000786in}}%
\pgfpathlineto{\pgfqpoint{2.578361in}{3.000752in}}%
\pgfpathlineto{\pgfqpoint{2.575189in}{3.000640in}}%
\pgfpathlineto{\pgfqpoint{2.572017in}{3.000519in}}%
\pgfpathlineto{\pgfqpoint{2.568845in}{3.000580in}}%
\pgfpathlineto{\pgfqpoint{2.565673in}{3.000403in}}%
\pgfpathlineto{\pgfqpoint{2.562501in}{3.000304in}}%
\pgfpathlineto{\pgfqpoint{2.559329in}{3.000233in}}%
\pgfpathlineto{\pgfqpoint{2.556157in}{3.000321in}}%
\pgfpathlineto{\pgfqpoint{2.552985in}{3.000326in}}%
\pgfpathlineto{\pgfqpoint{2.549813in}{3.000375in}}%
\pgfpathlineto{\pgfqpoint{2.546641in}{3.000404in}}%
\pgfpathlineto{\pgfqpoint{2.543469in}{3.000311in}}%
\pgfpathlineto{\pgfqpoint{2.540297in}{3.000215in}}%
\pgfpathlineto{\pgfqpoint{2.537125in}{3.000303in}}%
\pgfpathlineto{\pgfqpoint{2.533953in}{3.000346in}}%
\pgfpathlineto{\pgfqpoint{2.530781in}{3.000345in}}%
\pgfpathlineto{\pgfqpoint{2.527609in}{3.000594in}}%
\pgfpathlineto{\pgfqpoint{2.524437in}{3.000616in}}%
\pgfpathlineto{\pgfqpoint{2.521265in}{3.000712in}}%
\pgfpathlineto{\pgfqpoint{2.518093in}{3.000795in}}%
\pgfpathlineto{\pgfqpoint{2.514921in}{3.000750in}}%
\pgfpathlineto{\pgfqpoint{2.511749in}{3.000589in}}%
\pgfpathlineto{\pgfqpoint{2.508577in}{3.000688in}}%
\pgfpathlineto{\pgfqpoint{2.505405in}{3.000707in}}%
\pgfpathlineto{\pgfqpoint{2.502232in}{3.000856in}}%
\pgfpathlineto{\pgfqpoint{2.499060in}{3.000948in}}%
\pgfpathlineto{\pgfqpoint{2.495888in}{3.001003in}}%
\pgfpathlineto{\pgfqpoint{2.492716in}{3.000938in}}%
\pgfpathlineto{\pgfqpoint{2.489544in}{3.001190in}}%
\pgfpathlineto{\pgfqpoint{2.486372in}{3.001170in}}%
\pgfpathlineto{\pgfqpoint{2.483200in}{3.001165in}}%
\pgfpathlineto{\pgfqpoint{2.480028in}{3.000976in}}%
\pgfpathlineto{\pgfqpoint{2.476856in}{3.000920in}}%
\pgfpathlineto{\pgfqpoint{2.473684in}{3.000909in}}%
\pgfpathlineto{\pgfqpoint{2.470512in}{3.000917in}}%
\pgfpathlineto{\pgfqpoint{2.467340in}{3.000866in}}%
\pgfpathlineto{\pgfqpoint{2.464168in}{3.000705in}}%
\pgfpathlineto{\pgfqpoint{2.460996in}{3.000785in}}%
\pgfpathlineto{\pgfqpoint{2.457824in}{3.000811in}}%
\pgfpathlineto{\pgfqpoint{2.454652in}{3.000833in}}%
\pgfpathlineto{\pgfqpoint{2.451480in}{3.000890in}}%
\pgfpathlineto{\pgfqpoint{2.448308in}{3.000827in}}%
\pgfpathlineto{\pgfqpoint{2.445136in}{3.000967in}}%
\pgfpathlineto{\pgfqpoint{2.441964in}{3.000797in}}%
\pgfpathlineto{\pgfqpoint{2.438792in}{3.000600in}}%
\pgfpathlineto{\pgfqpoint{2.435620in}{3.000794in}}%
\pgfpathlineto{\pgfqpoint{2.432448in}{3.000654in}}%
\pgfpathlineto{\pgfqpoint{2.429276in}{3.000507in}}%
\pgfpathlineto{\pgfqpoint{2.426103in}{3.000504in}}%
\pgfpathlineto{\pgfqpoint{2.422931in}{2.993813in}}%
\pgfpathlineto{\pgfqpoint{2.419759in}{2.988179in}}%
\pgfpathlineto{\pgfqpoint{2.416587in}{2.973799in}}%
\pgfpathlineto{\pgfqpoint{2.413415in}{2.959955in}}%
\pgfpathlineto{\pgfqpoint{2.410243in}{2.946621in}}%
\pgfpathlineto{\pgfqpoint{2.407071in}{2.933309in}}%
\pgfpathlineto{\pgfqpoint{2.403899in}{2.920386in}}%
\pgfpathlineto{\pgfqpoint{2.400727in}{2.909575in}}%
\pgfpathlineto{\pgfqpoint{2.397555in}{2.902051in}}%
\pgfpathlineto{\pgfqpoint{2.394383in}{2.889938in}}%
\pgfpathlineto{\pgfqpoint{2.391211in}{2.878361in}}%
\pgfpathlineto{\pgfqpoint{2.388039in}{2.866466in}}%
\pgfpathlineto{\pgfqpoint{2.384867in}{2.854589in}}%
\pgfpathlineto{\pgfqpoint{2.381695in}{2.842537in}}%
\pgfpathlineto{\pgfqpoint{2.378523in}{2.830923in}}%
\pgfpathlineto{\pgfqpoint{2.375351in}{2.819005in}}%
\pgfpathlineto{\pgfqpoint{2.372179in}{2.806757in}}%
\pgfpathlineto{\pgfqpoint{2.369007in}{2.794930in}}%
\pgfpathlineto{\pgfqpoint{2.365835in}{2.783237in}}%
\pgfpathlineto{\pgfqpoint{2.362663in}{2.771708in}}%
\pgfpathlineto{\pgfqpoint{2.359491in}{2.759478in}}%
\pgfpathlineto{\pgfqpoint{2.356319in}{2.747500in}}%
\pgfpathlineto{\pgfqpoint{2.353147in}{2.735860in}}%
\pgfpathlineto{\pgfqpoint{2.349975in}{2.724452in}}%
\pgfpathlineto{\pgfqpoint{2.346802in}{2.713187in}}%
\pgfpathlineto{\pgfqpoint{2.343630in}{2.701598in}}%
\pgfpathlineto{\pgfqpoint{2.340458in}{2.689941in}}%
\pgfpathlineto{\pgfqpoint{2.337286in}{2.678240in}}%
\pgfpathlineto{\pgfqpoint{2.334114in}{2.666792in}}%
\pgfpathlineto{\pgfqpoint{2.330942in}{2.655117in}}%
\pgfpathlineto{\pgfqpoint{2.327770in}{2.643298in}}%
\pgfpathlineto{\pgfqpoint{2.324598in}{2.631314in}}%
\pgfpathlineto{\pgfqpoint{2.321426in}{2.619870in}}%
\pgfpathlineto{\pgfqpoint{2.318254in}{2.608060in}}%
\pgfpathlineto{\pgfqpoint{2.315082in}{2.596308in}}%
\pgfpathlineto{\pgfqpoint{2.311910in}{2.584642in}}%
\pgfpathlineto{\pgfqpoint{2.308738in}{2.572861in}}%
\pgfpathlineto{\pgfqpoint{2.305566in}{2.561416in}}%
\pgfpathlineto{\pgfqpoint{2.302394in}{2.550040in}}%
\pgfpathlineto{\pgfqpoint{2.299222in}{2.538690in}}%
\pgfpathlineto{\pgfqpoint{2.296050in}{2.526921in}}%
\pgfpathlineto{\pgfqpoint{2.292878in}{2.515193in}}%
\pgfpathlineto{\pgfqpoint{2.289706in}{2.503572in}}%
\pgfpathlineto{\pgfqpoint{2.286534in}{2.491753in}}%
\pgfpathlineto{\pgfqpoint{2.283362in}{2.480350in}}%
\pgfpathlineto{\pgfqpoint{2.280190in}{2.468414in}}%
\pgfpathlineto{\pgfqpoint{2.277018in}{2.456659in}}%
\pgfpathlineto{\pgfqpoint{2.273846in}{2.444871in}}%
\pgfpathlineto{\pgfqpoint{2.270674in}{2.433078in}}%
\pgfpathlineto{\pgfqpoint{2.267501in}{2.421662in}}%
\pgfpathlineto{\pgfqpoint{2.264329in}{2.410053in}}%
\pgfpathlineto{\pgfqpoint{2.261157in}{2.398558in}}%
\pgfpathlineto{\pgfqpoint{2.257985in}{2.387027in}}%
\pgfpathlineto{\pgfqpoint{2.254813in}{2.375581in}}%
\pgfpathlineto{\pgfqpoint{2.251641in}{2.363628in}}%
\pgfpathlineto{\pgfqpoint{2.248469in}{2.351868in}}%
\pgfpathlineto{\pgfqpoint{2.245297in}{2.340325in}}%
\pgfpathlineto{\pgfqpoint{2.242125in}{2.328961in}}%
\pgfpathlineto{\pgfqpoint{2.238953in}{2.317201in}}%
\pgfpathlineto{\pgfqpoint{2.235781in}{2.305227in}}%
\pgfpathlineto{\pgfqpoint{2.232609in}{2.293298in}}%
\pgfpathlineto{\pgfqpoint{2.229437in}{2.281560in}}%
\pgfpathlineto{\pgfqpoint{2.226265in}{2.270013in}}%
\pgfpathlineto{\pgfqpoint{2.223093in}{2.258076in}}%
\pgfpathlineto{\pgfqpoint{2.219921in}{2.246504in}}%
\pgfpathlineto{\pgfqpoint{2.216749in}{2.234594in}}%
\pgfpathlineto{\pgfqpoint{2.213577in}{2.222979in}}%
\pgfpathlineto{\pgfqpoint{2.210405in}{2.210568in}}%
\pgfpathlineto{\pgfqpoint{2.207233in}{2.198780in}}%
\pgfpathlineto{\pgfqpoint{2.204061in}{2.187089in}}%
\pgfpathlineto{\pgfqpoint{2.200889in}{2.175025in}}%
\pgfpathlineto{\pgfqpoint{2.197717in}{2.162868in}}%
\pgfpathlineto{\pgfqpoint{2.194545in}{2.151264in}}%
\pgfpathlineto{\pgfqpoint{2.191372in}{2.139344in}}%
\pgfpathlineto{\pgfqpoint{2.188200in}{2.127815in}}%
\pgfpathlineto{\pgfqpoint{2.185028in}{2.116071in}}%
\pgfpathlineto{\pgfqpoint{2.181856in}{2.104052in}}%
\pgfpathlineto{\pgfqpoint{2.178684in}{2.092419in}}%
\pgfpathlineto{\pgfqpoint{2.175512in}{2.080507in}}%
\pgfpathlineto{\pgfqpoint{2.172340in}{2.068845in}}%
\pgfpathlineto{\pgfqpoint{2.169168in}{2.057559in}}%
\pgfpathlineto{\pgfqpoint{2.165996in}{2.046112in}}%
\pgfpathlineto{\pgfqpoint{2.162824in}{2.034417in}}%
\pgfpathlineto{\pgfqpoint{2.159652in}{2.023021in}}%
\pgfpathlineto{\pgfqpoint{2.156480in}{2.011475in}}%
\pgfpathlineto{\pgfqpoint{2.153308in}{1.999808in}}%
\pgfpathlineto{\pgfqpoint{2.150136in}{1.987951in}}%
\pgfpathlineto{\pgfqpoint{2.146964in}{1.976636in}}%
\pgfpathlineto{\pgfqpoint{2.143792in}{1.964888in}}%
\pgfpathlineto{\pgfqpoint{2.140620in}{1.953653in}}%
\pgfpathlineto{\pgfqpoint{2.137448in}{1.941901in}}%
\pgfpathlineto{\pgfqpoint{2.134276in}{1.930302in}}%
\pgfpathlineto{\pgfqpoint{2.131104in}{1.918588in}}%
\pgfpathlineto{\pgfqpoint{2.127932in}{1.906765in}}%
\pgfpathlineto{\pgfqpoint{2.124760in}{1.894920in}}%
\pgfpathlineto{\pgfqpoint{2.121588in}{1.883026in}}%
\pgfpathlineto{\pgfqpoint{2.118416in}{1.871278in}}%
\pgfpathlineto{\pgfqpoint{2.115244in}{1.859142in}}%
\pgfpathlineto{\pgfqpoint{2.112071in}{1.847686in}}%
\pgfpathlineto{\pgfqpoint{2.108899in}{1.836121in}}%
\pgfpathlineto{\pgfqpoint{2.105727in}{1.824286in}}%
\pgfpathlineto{\pgfqpoint{2.102555in}{1.812142in}}%
\pgfpathlineto{\pgfqpoint{2.099383in}{1.800741in}}%
\pgfpathlineto{\pgfqpoint{2.096211in}{1.789515in}}%
\pgfpathlineto{\pgfqpoint{2.093039in}{1.777636in}}%
\pgfpathlineto{\pgfqpoint{2.089867in}{1.766052in}}%
\pgfpathlineto{\pgfqpoint{2.086695in}{1.754673in}}%
\pgfpathlineto{\pgfqpoint{2.083523in}{1.742723in}}%
\pgfpathlineto{\pgfqpoint{2.080351in}{1.731102in}}%
\pgfpathlineto{\pgfqpoint{2.077179in}{1.719573in}}%
\pgfpathlineto{\pgfqpoint{2.074007in}{1.708299in}}%
\pgfpathlineto{\pgfqpoint{2.070835in}{1.696964in}}%
\pgfpathlineto{\pgfqpoint{2.067663in}{1.684974in}}%
\pgfpathlineto{\pgfqpoint{2.064491in}{1.673410in}}%
\pgfpathlineto{\pgfqpoint{2.061319in}{1.661878in}}%
\pgfpathlineto{\pgfqpoint{2.058147in}{1.650413in}}%
\pgfpathlineto{\pgfqpoint{2.054975in}{1.638708in}}%
\pgfpathlineto{\pgfqpoint{2.051803in}{1.626979in}}%
\pgfpathlineto{\pgfqpoint{2.048631in}{1.615425in}}%
\pgfpathlineto{\pgfqpoint{2.045459in}{1.606542in}}%
\pgfpathlineto{\pgfqpoint{2.042287in}{1.592711in}}%
\pgfpathlineto{\pgfqpoint{2.039115in}{1.581161in}}%
\pgfpathlineto{\pgfqpoint{2.035943in}{1.570351in}}%
\pgfpathlineto{\pgfqpoint{2.032770in}{1.560057in}}%
\pgfpathlineto{\pgfqpoint{2.029598in}{1.548509in}}%
\pgfpathlineto{\pgfqpoint{2.026426in}{1.537455in}}%
\pgfpathlineto{\pgfqpoint{2.023254in}{1.527146in}}%
\pgfpathlineto{\pgfqpoint{2.020082in}{1.517189in}}%
\pgfpathlineto{\pgfqpoint{2.016910in}{1.505985in}}%
\pgfpathlineto{\pgfqpoint{2.013738in}{1.495857in}}%
\pgfpathlineto{\pgfqpoint{2.010566in}{1.485539in}}%
\pgfpathlineto{\pgfqpoint{2.007394in}{1.473885in}}%
\pgfpathlineto{\pgfqpoint{2.004222in}{1.462545in}}%
\pgfpathlineto{\pgfqpoint{2.001050in}{1.451747in}}%
\pgfpathlineto{\pgfqpoint{1.997878in}{1.440282in}}%
\pgfpathlineto{\pgfqpoint{1.994706in}{1.428998in}}%
\pgfpathlineto{\pgfqpoint{1.991534in}{1.417604in}}%
\pgfpathlineto{\pgfqpoint{1.988362in}{1.406228in}}%
\pgfpathlineto{\pgfqpoint{1.985190in}{1.394411in}}%
\pgfpathlineto{\pgfqpoint{1.982018in}{1.372758in}}%
\pgfpathlineto{\pgfqpoint{1.978846in}{1.351116in}}%
\pgfpathlineto{\pgfqpoint{1.975674in}{1.326702in}}%
\pgfpathlineto{\pgfqpoint{1.972502in}{1.301943in}}%
\pgfpathlineto{\pgfqpoint{1.969330in}{1.279786in}}%
\pgfpathlineto{\pgfqpoint{1.966158in}{1.259054in}}%
\pgfpathlineto{\pgfqpoint{1.962986in}{1.237530in}}%
\pgfpathlineto{\pgfqpoint{1.959814in}{1.216806in}}%
\pgfpathlineto{\pgfqpoint{1.956641in}{1.196951in}}%
\pgfpathlineto{\pgfqpoint{1.953469in}{1.173733in}}%
\pgfpathlineto{\pgfqpoint{1.950297in}{1.145413in}}%
\pgfpathlineto{\pgfqpoint{1.947125in}{1.126818in}}%
\pgfpathlineto{\pgfqpoint{1.943953in}{1.126818in}}%
\pgfpathlineto{\pgfqpoint{1.940781in}{1.127084in}}%
\pgfpathclose%
\pgfusepath{stroke,fill}%
\end{pgfscope}%
\begin{pgfscope}%
\pgfpathrectangle{\pgfqpoint{1.623736in}{1.000625in}}{\pgfqpoint{6.975000in}{3.020000in}} %
\pgfusepath{clip}%
\pgfsetbuttcap%
\pgfsetroundjoin%
\definecolor{currentfill}{rgb}{0.768627,0.305882,0.321569}%
\pgfsetfillcolor{currentfill}%
\pgfsetfillopacity{0.200000}%
\pgfsetlinewidth{0.803000pt}%
\definecolor{currentstroke}{rgb}{0.768627,0.305882,0.321569}%
\pgfsetstrokecolor{currentstroke}%
\pgfsetstrokeopacity{0.200000}%
\pgfsetdash{}{0pt}%
\pgfpathmoveto{\pgfqpoint{1.940781in}{1.128807in}}%
\pgfpathlineto{\pgfqpoint{1.940781in}{1.128313in}}%
\pgfpathlineto{\pgfqpoint{1.943953in}{1.126997in}}%
\pgfpathlineto{\pgfqpoint{1.947125in}{1.126940in}}%
\pgfpathlineto{\pgfqpoint{1.950297in}{1.146683in}}%
\pgfpathlineto{\pgfqpoint{1.953469in}{1.167392in}}%
\pgfpathlineto{\pgfqpoint{1.956641in}{1.183773in}}%
\pgfpathlineto{\pgfqpoint{1.959814in}{1.205706in}}%
\pgfpathlineto{\pgfqpoint{1.962986in}{1.225243in}}%
\pgfpathlineto{\pgfqpoint{1.966158in}{1.246987in}}%
\pgfpathlineto{\pgfqpoint{1.969330in}{1.266485in}}%
\pgfpathlineto{\pgfqpoint{1.972502in}{1.281771in}}%
\pgfpathlineto{\pgfqpoint{1.975674in}{1.300517in}}%
\pgfpathlineto{\pgfqpoint{1.978846in}{1.320568in}}%
\pgfpathlineto{\pgfqpoint{1.982018in}{1.340114in}}%
\pgfpathlineto{\pgfqpoint{1.985190in}{1.360755in}}%
\pgfpathlineto{\pgfqpoint{1.988362in}{1.372012in}}%
\pgfpathlineto{\pgfqpoint{1.991534in}{1.384739in}}%
\pgfpathlineto{\pgfqpoint{1.994706in}{1.396088in}}%
\pgfpathlineto{\pgfqpoint{1.997878in}{1.407364in}}%
\pgfpathlineto{\pgfqpoint{2.001050in}{1.419718in}}%
\pgfpathlineto{\pgfqpoint{2.004222in}{1.432625in}}%
\pgfpathlineto{\pgfqpoint{2.007394in}{1.444163in}}%
\pgfpathlineto{\pgfqpoint{2.010566in}{1.455815in}}%
\pgfpathlineto{\pgfqpoint{2.013738in}{1.467197in}}%
\pgfpathlineto{\pgfqpoint{2.016910in}{1.479625in}}%
\pgfpathlineto{\pgfqpoint{2.020082in}{1.490999in}}%
\pgfpathlineto{\pgfqpoint{2.023254in}{1.501563in}}%
\pgfpathlineto{\pgfqpoint{2.026426in}{1.512829in}}%
\pgfpathlineto{\pgfqpoint{2.029598in}{1.523488in}}%
\pgfpathlineto{\pgfqpoint{2.032770in}{1.536104in}}%
\pgfpathlineto{\pgfqpoint{2.035943in}{1.546873in}}%
\pgfpathlineto{\pgfqpoint{2.039115in}{1.558157in}}%
\pgfpathlineto{\pgfqpoint{2.042287in}{1.568798in}}%
\pgfpathlineto{\pgfqpoint{2.045459in}{1.580863in}}%
\pgfpathlineto{\pgfqpoint{2.048631in}{1.591339in}}%
\pgfpathlineto{\pgfqpoint{2.051803in}{1.602884in}}%
\pgfpathlineto{\pgfqpoint{2.054975in}{1.614558in}}%
\pgfpathlineto{\pgfqpoint{2.058147in}{1.626408in}}%
\pgfpathlineto{\pgfqpoint{2.061319in}{1.637826in}}%
\pgfpathlineto{\pgfqpoint{2.064491in}{1.649787in}}%
\pgfpathlineto{\pgfqpoint{2.067663in}{1.661550in}}%
\pgfpathlineto{\pgfqpoint{2.070835in}{1.673626in}}%
\pgfpathlineto{\pgfqpoint{2.074007in}{1.685000in}}%
\pgfpathlineto{\pgfqpoint{2.077179in}{1.696660in}}%
\pgfpathlineto{\pgfqpoint{2.080351in}{1.708696in}}%
\pgfpathlineto{\pgfqpoint{2.083523in}{1.720283in}}%
\pgfpathlineto{\pgfqpoint{2.086695in}{1.732280in}}%
\pgfpathlineto{\pgfqpoint{2.089867in}{1.743765in}}%
\pgfpathlineto{\pgfqpoint{2.093039in}{1.755221in}}%
\pgfpathlineto{\pgfqpoint{2.096211in}{1.766818in}}%
\pgfpathlineto{\pgfqpoint{2.099383in}{1.778169in}}%
\pgfpathlineto{\pgfqpoint{2.102555in}{1.789809in}}%
\pgfpathlineto{\pgfqpoint{2.105727in}{1.801713in}}%
\pgfpathlineto{\pgfqpoint{2.108899in}{1.813707in}}%
\pgfpathlineto{\pgfqpoint{2.112071in}{1.825407in}}%
\pgfpathlineto{\pgfqpoint{2.115244in}{1.837254in}}%
\pgfpathlineto{\pgfqpoint{2.118416in}{1.849027in}}%
\pgfpathlineto{\pgfqpoint{2.121588in}{1.860301in}}%
\pgfpathlineto{\pgfqpoint{2.124760in}{1.872171in}}%
\pgfpathlineto{\pgfqpoint{2.127932in}{1.884340in}}%
\pgfpathlineto{\pgfqpoint{2.131104in}{1.896282in}}%
\pgfpathlineto{\pgfqpoint{2.134276in}{1.908109in}}%
\pgfpathlineto{\pgfqpoint{2.137448in}{1.919683in}}%
\pgfpathlineto{\pgfqpoint{2.140620in}{1.931306in}}%
\pgfpathlineto{\pgfqpoint{2.143792in}{1.943443in}}%
\pgfpathlineto{\pgfqpoint{2.146964in}{1.954913in}}%
\pgfpathlineto{\pgfqpoint{2.150136in}{1.966556in}}%
\pgfpathlineto{\pgfqpoint{2.153308in}{1.978332in}}%
\pgfpathlineto{\pgfqpoint{2.156480in}{1.990036in}}%
\pgfpathlineto{\pgfqpoint{2.159652in}{2.001576in}}%
\pgfpathlineto{\pgfqpoint{2.162824in}{2.013203in}}%
\pgfpathlineto{\pgfqpoint{2.165996in}{2.025176in}}%
\pgfpathlineto{\pgfqpoint{2.169168in}{2.036809in}}%
\pgfpathlineto{\pgfqpoint{2.172340in}{2.048361in}}%
\pgfpathlineto{\pgfqpoint{2.175512in}{2.059907in}}%
\pgfpathlineto{\pgfqpoint{2.178684in}{2.071390in}}%
\pgfpathlineto{\pgfqpoint{2.181856in}{2.083199in}}%
\pgfpathlineto{\pgfqpoint{2.185028in}{2.094684in}}%
\pgfpathlineto{\pgfqpoint{2.188200in}{2.106163in}}%
\pgfpathlineto{\pgfqpoint{2.191372in}{2.118172in}}%
\pgfpathlineto{\pgfqpoint{2.194545in}{2.129564in}}%
\pgfpathlineto{\pgfqpoint{2.197717in}{2.141023in}}%
\pgfpathlineto{\pgfqpoint{2.200889in}{2.152677in}}%
\pgfpathlineto{\pgfqpoint{2.204061in}{2.164429in}}%
\pgfpathlineto{\pgfqpoint{2.207233in}{2.175956in}}%
\pgfpathlineto{\pgfqpoint{2.210405in}{2.187794in}}%
\pgfpathlineto{\pgfqpoint{2.213577in}{2.199715in}}%
\pgfpathlineto{\pgfqpoint{2.216749in}{2.211680in}}%
\pgfpathlineto{\pgfqpoint{2.219921in}{2.223962in}}%
\pgfpathlineto{\pgfqpoint{2.223093in}{2.235523in}}%
\pgfpathlineto{\pgfqpoint{2.226265in}{2.247078in}}%
\pgfpathlineto{\pgfqpoint{2.229437in}{2.258963in}}%
\pgfpathlineto{\pgfqpoint{2.232609in}{2.270444in}}%
\pgfpathlineto{\pgfqpoint{2.235781in}{2.282733in}}%
\pgfpathlineto{\pgfqpoint{2.238953in}{2.294734in}}%
\pgfpathlineto{\pgfqpoint{2.242125in}{2.306040in}}%
\pgfpathlineto{\pgfqpoint{2.245297in}{2.318248in}}%
\pgfpathlineto{\pgfqpoint{2.248469in}{2.330222in}}%
\pgfpathlineto{\pgfqpoint{2.251641in}{2.342164in}}%
\pgfpathlineto{\pgfqpoint{2.254813in}{2.354019in}}%
\pgfpathlineto{\pgfqpoint{2.257985in}{2.365168in}}%
\pgfpathlineto{\pgfqpoint{2.261157in}{2.376425in}}%
\pgfpathlineto{\pgfqpoint{2.264329in}{2.387971in}}%
\pgfpathlineto{\pgfqpoint{2.267501in}{2.399526in}}%
\pgfpathlineto{\pgfqpoint{2.270674in}{2.411108in}}%
\pgfpathlineto{\pgfqpoint{2.273846in}{2.422823in}}%
\pgfpathlineto{\pgfqpoint{2.277018in}{2.435363in}}%
\pgfpathlineto{\pgfqpoint{2.280190in}{2.446782in}}%
\pgfpathlineto{\pgfqpoint{2.283362in}{2.457915in}}%
\pgfpathlineto{\pgfqpoint{2.286534in}{2.469549in}}%
\pgfpathlineto{\pgfqpoint{2.289706in}{2.481489in}}%
\pgfpathlineto{\pgfqpoint{2.292878in}{2.493282in}}%
\pgfpathlineto{\pgfqpoint{2.296050in}{2.505558in}}%
\pgfpathlineto{\pgfqpoint{2.299222in}{2.516977in}}%
\pgfpathlineto{\pgfqpoint{2.302394in}{2.528135in}}%
\pgfpathlineto{\pgfqpoint{2.305566in}{2.540255in}}%
\pgfpathlineto{\pgfqpoint{2.308738in}{2.552663in}}%
\pgfpathlineto{\pgfqpoint{2.311910in}{2.564339in}}%
\pgfpathlineto{\pgfqpoint{2.315082in}{2.576053in}}%
\pgfpathlineto{\pgfqpoint{2.318254in}{2.587607in}}%
\pgfpathlineto{\pgfqpoint{2.321426in}{2.599647in}}%
\pgfpathlineto{\pgfqpoint{2.324598in}{2.611366in}}%
\pgfpathlineto{\pgfqpoint{2.327770in}{2.623421in}}%
\pgfpathlineto{\pgfqpoint{2.330942in}{2.634924in}}%
\pgfpathlineto{\pgfqpoint{2.334114in}{2.647089in}}%
\pgfpathlineto{\pgfqpoint{2.337286in}{2.644156in}}%
\pgfpathlineto{\pgfqpoint{2.340458in}{2.641244in}}%
\pgfpathlineto{\pgfqpoint{2.343630in}{2.638648in}}%
\pgfpathlineto{\pgfqpoint{2.346802in}{2.635888in}}%
\pgfpathlineto{\pgfqpoint{2.349975in}{2.633127in}}%
\pgfpathlineto{\pgfqpoint{2.353147in}{2.630196in}}%
\pgfpathlineto{\pgfqpoint{2.356319in}{2.627454in}}%
\pgfpathlineto{\pgfqpoint{2.359491in}{2.624806in}}%
\pgfpathlineto{\pgfqpoint{2.362663in}{2.622039in}}%
\pgfpathlineto{\pgfqpoint{2.365835in}{2.619362in}}%
\pgfpathlineto{\pgfqpoint{2.369007in}{2.616564in}}%
\pgfpathlineto{\pgfqpoint{2.372179in}{2.613862in}}%
\pgfpathlineto{\pgfqpoint{2.375351in}{2.613779in}}%
\pgfpathlineto{\pgfqpoint{2.378523in}{2.613814in}}%
\pgfpathlineto{\pgfqpoint{2.381695in}{2.614006in}}%
\pgfpathlineto{\pgfqpoint{2.384867in}{2.614271in}}%
\pgfpathlineto{\pgfqpoint{2.388039in}{2.614231in}}%
\pgfpathlineto{\pgfqpoint{2.391211in}{2.614133in}}%
\pgfpathlineto{\pgfqpoint{2.394383in}{2.614065in}}%
\pgfpathlineto{\pgfqpoint{2.397555in}{2.614180in}}%
\pgfpathlineto{\pgfqpoint{2.400727in}{2.614105in}}%
\pgfpathlineto{\pgfqpoint{2.403899in}{2.614126in}}%
\pgfpathlineto{\pgfqpoint{2.407071in}{2.613987in}}%
\pgfpathlineto{\pgfqpoint{2.410243in}{2.613975in}}%
\pgfpathlineto{\pgfqpoint{2.413415in}{2.614084in}}%
\pgfpathlineto{\pgfqpoint{2.416587in}{2.614208in}}%
\pgfpathlineto{\pgfqpoint{2.419759in}{2.614084in}}%
\pgfpathlineto{\pgfqpoint{2.422931in}{2.614357in}}%
\pgfpathlineto{\pgfqpoint{2.426103in}{2.614566in}}%
\pgfpathlineto{\pgfqpoint{2.429276in}{2.614639in}}%
\pgfpathlineto{\pgfqpoint{2.432448in}{2.614682in}}%
\pgfpathlineto{\pgfqpoint{2.435620in}{2.614649in}}%
\pgfpathlineto{\pgfqpoint{2.438792in}{2.614774in}}%
\pgfpathlineto{\pgfqpoint{2.441964in}{2.614805in}}%
\pgfpathlineto{\pgfqpoint{2.445136in}{2.614713in}}%
\pgfpathlineto{\pgfqpoint{2.448308in}{2.614683in}}%
\pgfpathlineto{\pgfqpoint{2.451480in}{2.614750in}}%
\pgfpathlineto{\pgfqpoint{2.454652in}{2.614846in}}%
\pgfpathlineto{\pgfqpoint{2.457824in}{2.615067in}}%
\pgfpathlineto{\pgfqpoint{2.460996in}{2.615246in}}%
\pgfpathlineto{\pgfqpoint{2.464168in}{2.615172in}}%
\pgfpathlineto{\pgfqpoint{2.467340in}{2.615435in}}%
\pgfpathlineto{\pgfqpoint{2.470512in}{2.615369in}}%
\pgfpathlineto{\pgfqpoint{2.473684in}{2.615444in}}%
\pgfpathlineto{\pgfqpoint{2.476856in}{2.615607in}}%
\pgfpathlineto{\pgfqpoint{2.480028in}{2.615463in}}%
\pgfpathlineto{\pgfqpoint{2.483200in}{2.615385in}}%
\pgfpathlineto{\pgfqpoint{2.486372in}{2.615466in}}%
\pgfpathlineto{\pgfqpoint{2.489544in}{2.615469in}}%
\pgfpathlineto{\pgfqpoint{2.492716in}{2.615408in}}%
\pgfpathlineto{\pgfqpoint{2.495888in}{2.615556in}}%
\pgfpathlineto{\pgfqpoint{2.499060in}{2.615502in}}%
\pgfpathlineto{\pgfqpoint{2.502232in}{2.615558in}}%
\pgfpathlineto{\pgfqpoint{2.505405in}{2.615751in}}%
\pgfpathlineto{\pgfqpoint{2.508577in}{2.615906in}}%
\pgfpathlineto{\pgfqpoint{2.511749in}{2.616232in}}%
\pgfpathlineto{\pgfqpoint{2.514921in}{2.616273in}}%
\pgfpathlineto{\pgfqpoint{2.518093in}{2.616259in}}%
\pgfpathlineto{\pgfqpoint{2.521265in}{2.616352in}}%
\pgfpathlineto{\pgfqpoint{2.524437in}{2.616346in}}%
\pgfpathlineto{\pgfqpoint{2.527609in}{2.616253in}}%
\pgfpathlineto{\pgfqpoint{2.530781in}{2.616271in}}%
\pgfpathlineto{\pgfqpoint{2.533953in}{2.616329in}}%
\pgfpathlineto{\pgfqpoint{2.537125in}{2.616515in}}%
\pgfpathlineto{\pgfqpoint{2.540297in}{2.616583in}}%
\pgfpathlineto{\pgfqpoint{2.543469in}{2.616867in}}%
\pgfpathlineto{\pgfqpoint{2.546641in}{2.616895in}}%
\pgfpathlineto{\pgfqpoint{2.549813in}{2.617020in}}%
\pgfpathlineto{\pgfqpoint{2.552985in}{2.616922in}}%
\pgfpathlineto{\pgfqpoint{2.556157in}{2.617108in}}%
\pgfpathlineto{\pgfqpoint{2.559329in}{2.617178in}}%
\pgfpathlineto{\pgfqpoint{2.562501in}{2.617232in}}%
\pgfpathlineto{\pgfqpoint{2.565673in}{2.617149in}}%
\pgfpathlineto{\pgfqpoint{2.568845in}{2.617232in}}%
\pgfpathlineto{\pgfqpoint{2.572017in}{2.617279in}}%
\pgfpathlineto{\pgfqpoint{2.575189in}{2.617453in}}%
\pgfpathlineto{\pgfqpoint{2.578361in}{2.617403in}}%
\pgfpathlineto{\pgfqpoint{2.581533in}{2.617499in}}%
\pgfpathlineto{\pgfqpoint{2.584706in}{2.617666in}}%
\pgfpathlineto{\pgfqpoint{2.587878in}{2.617606in}}%
\pgfpathlineto{\pgfqpoint{2.591050in}{2.617608in}}%
\pgfpathlineto{\pgfqpoint{2.594222in}{2.617623in}}%
\pgfpathlineto{\pgfqpoint{2.597394in}{2.617558in}}%
\pgfpathlineto{\pgfqpoint{2.600566in}{2.617433in}}%
\pgfpathlineto{\pgfqpoint{2.603738in}{2.617503in}}%
\pgfpathlineto{\pgfqpoint{2.606910in}{2.617370in}}%
\pgfpathlineto{\pgfqpoint{2.610082in}{2.617212in}}%
\pgfpathlineto{\pgfqpoint{2.613254in}{2.617241in}}%
\pgfpathlineto{\pgfqpoint{2.616426in}{2.617379in}}%
\pgfpathlineto{\pgfqpoint{2.619598in}{2.617383in}}%
\pgfpathlineto{\pgfqpoint{2.622770in}{2.617462in}}%
\pgfpathlineto{\pgfqpoint{2.625942in}{2.617578in}}%
\pgfpathlineto{\pgfqpoint{2.629114in}{2.617600in}}%
\pgfpathlineto{\pgfqpoint{2.632286in}{2.617337in}}%
\pgfpathlineto{\pgfqpoint{2.635458in}{2.617361in}}%
\pgfpathlineto{\pgfqpoint{2.638630in}{2.617394in}}%
\pgfpathlineto{\pgfqpoint{2.641802in}{2.617383in}}%
\pgfpathlineto{\pgfqpoint{2.644974in}{2.617366in}}%
\pgfpathlineto{\pgfqpoint{2.648146in}{2.617153in}}%
\pgfpathlineto{\pgfqpoint{2.651318in}{2.617206in}}%
\pgfpathlineto{\pgfqpoint{2.654490in}{2.617467in}}%
\pgfpathlineto{\pgfqpoint{2.657662in}{2.617762in}}%
\pgfpathlineto{\pgfqpoint{2.660834in}{2.617839in}}%
\pgfpathlineto{\pgfqpoint{2.664007in}{2.617850in}}%
\pgfpathlineto{\pgfqpoint{2.667179in}{2.618087in}}%
\pgfpathlineto{\pgfqpoint{2.670351in}{2.618118in}}%
\pgfpathlineto{\pgfqpoint{2.673523in}{2.618018in}}%
\pgfpathlineto{\pgfqpoint{2.676695in}{2.617944in}}%
\pgfpathlineto{\pgfqpoint{2.679867in}{2.618173in}}%
\pgfpathlineto{\pgfqpoint{2.683039in}{2.618249in}}%
\pgfpathlineto{\pgfqpoint{2.686211in}{2.618112in}}%
\pgfpathlineto{\pgfqpoint{2.689383in}{2.618182in}}%
\pgfpathlineto{\pgfqpoint{2.692555in}{2.618123in}}%
\pgfpathlineto{\pgfqpoint{2.695727in}{2.618182in}}%
\pgfpathlineto{\pgfqpoint{2.698899in}{2.618312in}}%
\pgfpathlineto{\pgfqpoint{2.702071in}{2.618262in}}%
\pgfpathlineto{\pgfqpoint{2.705243in}{2.617985in}}%
\pgfpathlineto{\pgfqpoint{2.708415in}{2.617991in}}%
\pgfpathlineto{\pgfqpoint{2.711587in}{2.618207in}}%
\pgfpathlineto{\pgfqpoint{2.714759in}{2.618475in}}%
\pgfpathlineto{\pgfqpoint{2.717931in}{2.618364in}}%
\pgfpathlineto{\pgfqpoint{2.721103in}{2.618268in}}%
\pgfpathlineto{\pgfqpoint{2.724275in}{2.618390in}}%
\pgfpathlineto{\pgfqpoint{2.727447in}{2.618354in}}%
\pgfpathlineto{\pgfqpoint{2.730619in}{2.618487in}}%
\pgfpathlineto{\pgfqpoint{2.733791in}{2.618564in}}%
\pgfpathlineto{\pgfqpoint{2.736963in}{2.618448in}}%
\pgfpathlineto{\pgfqpoint{2.740136in}{2.618331in}}%
\pgfpathlineto{\pgfqpoint{2.743308in}{2.618374in}}%
\pgfpathlineto{\pgfqpoint{2.746480in}{2.618561in}}%
\pgfpathlineto{\pgfqpoint{2.749652in}{2.618544in}}%
\pgfpathlineto{\pgfqpoint{2.752824in}{2.618552in}}%
\pgfpathlineto{\pgfqpoint{2.755996in}{2.618883in}}%
\pgfpathlineto{\pgfqpoint{2.759168in}{2.618821in}}%
\pgfpathlineto{\pgfqpoint{2.762340in}{2.618947in}}%
\pgfpathlineto{\pgfqpoint{2.765512in}{2.618979in}}%
\pgfpathlineto{\pgfqpoint{2.768684in}{2.618899in}}%
\pgfpathlineto{\pgfqpoint{2.771856in}{2.618829in}}%
\pgfpathlineto{\pgfqpoint{2.775028in}{2.618832in}}%
\pgfpathlineto{\pgfqpoint{2.778200in}{2.618693in}}%
\pgfpathlineto{\pgfqpoint{2.781372in}{2.618653in}}%
\pgfpathlineto{\pgfqpoint{2.784544in}{2.618790in}}%
\pgfpathlineto{\pgfqpoint{2.787716in}{2.618668in}}%
\pgfpathlineto{\pgfqpoint{2.790888in}{2.618566in}}%
\pgfpathlineto{\pgfqpoint{2.794060in}{2.618565in}}%
\pgfpathlineto{\pgfqpoint{2.797232in}{2.618387in}}%
\pgfpathlineto{\pgfqpoint{2.800404in}{2.618381in}}%
\pgfpathlineto{\pgfqpoint{2.803576in}{2.618429in}}%
\pgfpathlineto{\pgfqpoint{2.806748in}{2.618385in}}%
\pgfpathlineto{\pgfqpoint{2.809920in}{2.618325in}}%
\pgfpathlineto{\pgfqpoint{2.813092in}{2.618373in}}%
\pgfpathlineto{\pgfqpoint{2.816264in}{2.618437in}}%
\pgfpathlineto{\pgfqpoint{2.819437in}{2.618564in}}%
\pgfpathlineto{\pgfqpoint{2.822609in}{2.618446in}}%
\pgfpathlineto{\pgfqpoint{2.825781in}{2.618520in}}%
\pgfpathlineto{\pgfqpoint{2.828953in}{2.618552in}}%
\pgfpathlineto{\pgfqpoint{2.832125in}{2.618607in}}%
\pgfpathlineto{\pgfqpoint{2.835297in}{2.618618in}}%
\pgfpathlineto{\pgfqpoint{2.838469in}{2.618724in}}%
\pgfpathlineto{\pgfqpoint{2.841641in}{2.618735in}}%
\pgfpathlineto{\pgfqpoint{2.844813in}{2.618766in}}%
\pgfpathlineto{\pgfqpoint{2.847985in}{2.618948in}}%
\pgfpathlineto{\pgfqpoint{2.851157in}{2.618753in}}%
\pgfpathlineto{\pgfqpoint{2.854329in}{2.619015in}}%
\pgfpathlineto{\pgfqpoint{2.857501in}{2.618962in}}%
\pgfpathlineto{\pgfqpoint{2.860673in}{2.619226in}}%
\pgfpathlineto{\pgfqpoint{2.863845in}{2.619279in}}%
\pgfpathlineto{\pgfqpoint{2.867017in}{2.619492in}}%
\pgfpathlineto{\pgfqpoint{2.870189in}{2.619683in}}%
\pgfpathlineto{\pgfqpoint{2.873361in}{2.619704in}}%
\pgfpathlineto{\pgfqpoint{2.876533in}{2.619921in}}%
\pgfpathlineto{\pgfqpoint{2.879705in}{2.620459in}}%
\pgfpathlineto{\pgfqpoint{2.882877in}{2.620462in}}%
\pgfpathlineto{\pgfqpoint{2.886049in}{2.620444in}}%
\pgfpathlineto{\pgfqpoint{2.889221in}{2.620537in}}%
\pgfpathlineto{\pgfqpoint{2.892393in}{2.620694in}}%
\pgfpathlineto{\pgfqpoint{2.895565in}{2.621043in}}%
\pgfpathlineto{\pgfqpoint{2.898738in}{2.621352in}}%
\pgfpathlineto{\pgfqpoint{2.901910in}{2.621433in}}%
\pgfpathlineto{\pgfqpoint{2.905082in}{2.621404in}}%
\pgfpathlineto{\pgfqpoint{2.908254in}{2.621536in}}%
\pgfpathlineto{\pgfqpoint{2.911426in}{2.621604in}}%
\pgfpathlineto{\pgfqpoint{2.914598in}{2.621860in}}%
\pgfpathlineto{\pgfqpoint{2.917770in}{2.622044in}}%
\pgfpathlineto{\pgfqpoint{2.920942in}{2.622160in}}%
\pgfpathlineto{\pgfqpoint{2.924114in}{2.622389in}}%
\pgfpathlineto{\pgfqpoint{2.927286in}{2.622431in}}%
\pgfpathlineto{\pgfqpoint{2.930458in}{2.622364in}}%
\pgfpathlineto{\pgfqpoint{2.933630in}{2.622438in}}%
\pgfpathlineto{\pgfqpoint{2.936802in}{2.622303in}}%
\pgfpathlineto{\pgfqpoint{2.939974in}{2.622208in}}%
\pgfpathlineto{\pgfqpoint{2.943146in}{2.622139in}}%
\pgfpathlineto{\pgfqpoint{2.946318in}{2.622161in}}%
\pgfpathlineto{\pgfqpoint{2.949490in}{2.622192in}}%
\pgfpathlineto{\pgfqpoint{2.952662in}{2.622027in}}%
\pgfpathlineto{\pgfqpoint{2.955834in}{2.622115in}}%
\pgfpathlineto{\pgfqpoint{2.959006in}{2.623378in}}%
\pgfpathlineto{\pgfqpoint{2.962178in}{2.623289in}}%
\pgfpathlineto{\pgfqpoint{2.965350in}{2.623151in}}%
\pgfpathlineto{\pgfqpoint{2.968522in}{2.623181in}}%
\pgfpathlineto{\pgfqpoint{2.971694in}{2.623493in}}%
\pgfpathlineto{\pgfqpoint{2.974867in}{2.623492in}}%
\pgfpathlineto{\pgfqpoint{2.978039in}{2.623528in}}%
\pgfpathlineto{\pgfqpoint{2.981211in}{2.623563in}}%
\pgfpathlineto{\pgfqpoint{2.984383in}{2.623656in}}%
\pgfpathlineto{\pgfqpoint{2.987555in}{2.623979in}}%
\pgfpathlineto{\pgfqpoint{2.990727in}{2.624202in}}%
\pgfpathlineto{\pgfqpoint{2.993899in}{2.624365in}}%
\pgfpathlineto{\pgfqpoint{2.997071in}{2.624548in}}%
\pgfpathlineto{\pgfqpoint{3.000243in}{2.624666in}}%
\pgfpathlineto{\pgfqpoint{3.003415in}{2.624642in}}%
\pgfpathlineto{\pgfqpoint{3.006587in}{2.624510in}}%
\pgfpathlineto{\pgfqpoint{3.009759in}{2.624603in}}%
\pgfpathlineto{\pgfqpoint{3.012931in}{2.624455in}}%
\pgfpathlineto{\pgfqpoint{3.016103in}{2.624426in}}%
\pgfpathlineto{\pgfqpoint{3.019275in}{2.624382in}}%
\pgfpathlineto{\pgfqpoint{3.022447in}{2.624244in}}%
\pgfpathlineto{\pgfqpoint{3.025619in}{2.624162in}}%
\pgfpathlineto{\pgfqpoint{3.028791in}{2.624353in}}%
\pgfpathlineto{\pgfqpoint{3.031963in}{2.624438in}}%
\pgfpathlineto{\pgfqpoint{3.035135in}{2.624428in}}%
\pgfpathlineto{\pgfqpoint{3.038307in}{2.624326in}}%
\pgfpathlineto{\pgfqpoint{3.041479in}{2.624333in}}%
\pgfpathlineto{\pgfqpoint{3.044651in}{2.624385in}}%
\pgfpathlineto{\pgfqpoint{3.047823in}{2.624442in}}%
\pgfpathlineto{\pgfqpoint{3.050995in}{2.624372in}}%
\pgfpathlineto{\pgfqpoint{3.054168in}{2.624375in}}%
\pgfpathlineto{\pgfqpoint{3.057340in}{2.624494in}}%
\pgfpathlineto{\pgfqpoint{3.060512in}{2.624656in}}%
\pgfpathlineto{\pgfqpoint{3.063684in}{2.624626in}}%
\pgfpathlineto{\pgfqpoint{3.066856in}{2.624779in}}%
\pgfpathlineto{\pgfqpoint{3.070028in}{2.624902in}}%
\pgfpathlineto{\pgfqpoint{3.073200in}{2.624984in}}%
\pgfpathlineto{\pgfqpoint{3.076372in}{2.624916in}}%
\pgfpathlineto{\pgfqpoint{3.079544in}{2.624805in}}%
\pgfpathlineto{\pgfqpoint{3.082716in}{2.624771in}}%
\pgfpathlineto{\pgfqpoint{3.085888in}{2.624805in}}%
\pgfpathlineto{\pgfqpoint{3.089060in}{2.625017in}}%
\pgfpathlineto{\pgfqpoint{3.092232in}{2.625083in}}%
\pgfpathlineto{\pgfqpoint{3.095404in}{2.625164in}}%
\pgfpathlineto{\pgfqpoint{3.098576in}{2.625034in}}%
\pgfpathlineto{\pgfqpoint{3.101748in}{2.625214in}}%
\pgfpathlineto{\pgfqpoint{3.104920in}{2.625145in}}%
\pgfpathlineto{\pgfqpoint{3.108092in}{2.625246in}}%
\pgfpathlineto{\pgfqpoint{3.111264in}{2.625469in}}%
\pgfpathlineto{\pgfqpoint{3.114436in}{2.625369in}}%
\pgfpathlineto{\pgfqpoint{3.117608in}{2.625737in}}%
\pgfpathlineto{\pgfqpoint{3.120780in}{2.625967in}}%
\pgfpathlineto{\pgfqpoint{3.123952in}{2.626105in}}%
\pgfpathlineto{\pgfqpoint{3.127124in}{2.625897in}}%
\pgfpathlineto{\pgfqpoint{3.130297in}{2.626216in}}%
\pgfpathlineto{\pgfqpoint{3.133469in}{2.626138in}}%
\pgfpathlineto{\pgfqpoint{3.136641in}{2.626312in}}%
\pgfpathlineto{\pgfqpoint{3.139813in}{2.626999in}}%
\pgfpathlineto{\pgfqpoint{3.142985in}{2.627288in}}%
\pgfpathlineto{\pgfqpoint{3.146157in}{2.626621in}}%
\pgfpathlineto{\pgfqpoint{3.149329in}{2.626655in}}%
\pgfpathlineto{\pgfqpoint{3.152501in}{2.627378in}}%
\pgfpathlineto{\pgfqpoint{3.155673in}{2.627736in}}%
\pgfpathlineto{\pgfqpoint{3.158845in}{2.627952in}}%
\pgfpathlineto{\pgfqpoint{3.162017in}{2.627965in}}%
\pgfpathlineto{\pgfqpoint{3.165189in}{2.627906in}}%
\pgfpathlineto{\pgfqpoint{3.168361in}{2.628615in}}%
\pgfpathlineto{\pgfqpoint{3.171533in}{2.628867in}}%
\pgfpathlineto{\pgfqpoint{3.174705in}{2.629117in}}%
\pgfpathlineto{\pgfqpoint{3.177877in}{2.629202in}}%
\pgfpathlineto{\pgfqpoint{3.181049in}{2.628994in}}%
\pgfpathlineto{\pgfqpoint{3.184221in}{2.629247in}}%
\pgfpathlineto{\pgfqpoint{3.187393in}{2.629113in}}%
\pgfpathlineto{\pgfqpoint{3.190565in}{2.629147in}}%
\pgfpathlineto{\pgfqpoint{3.193737in}{2.629006in}}%
\pgfpathlineto{\pgfqpoint{3.196909in}{2.628728in}}%
\pgfpathlineto{\pgfqpoint{3.200081in}{2.628550in}}%
\pgfpathlineto{\pgfqpoint{3.203253in}{2.628588in}}%
\pgfpathlineto{\pgfqpoint{3.206425in}{2.627766in}}%
\pgfpathlineto{\pgfqpoint{3.209598in}{2.626369in}}%
\pgfpathlineto{\pgfqpoint{3.212770in}{2.626292in}}%
\pgfpathlineto{\pgfqpoint{3.215942in}{2.626445in}}%
\pgfpathlineto{\pgfqpoint{3.219114in}{2.626324in}}%
\pgfpathlineto{\pgfqpoint{3.222286in}{2.626575in}}%
\pgfpathlineto{\pgfqpoint{3.225458in}{2.626987in}}%
\pgfpathlineto{\pgfqpoint{3.228630in}{2.626630in}}%
\pgfpathlineto{\pgfqpoint{3.231802in}{2.626499in}}%
\pgfpathlineto{\pgfqpoint{3.234974in}{2.626629in}}%
\pgfpathlineto{\pgfqpoint{3.238146in}{2.626192in}}%
\pgfpathlineto{\pgfqpoint{3.241318in}{2.626328in}}%
\pgfpathlineto{\pgfqpoint{3.244490in}{2.625774in}}%
\pgfpathlineto{\pgfqpoint{3.247662in}{2.625843in}}%
\pgfpathlineto{\pgfqpoint{3.250834in}{2.625731in}}%
\pgfpathlineto{\pgfqpoint{3.254006in}{2.625290in}}%
\pgfpathlineto{\pgfqpoint{3.257178in}{2.625442in}}%
\pgfpathlineto{\pgfqpoint{3.260350in}{2.625568in}}%
\pgfpathlineto{\pgfqpoint{3.263522in}{2.625152in}}%
\pgfpathlineto{\pgfqpoint{3.266694in}{2.625280in}}%
\pgfpathlineto{\pgfqpoint{3.269866in}{2.625528in}}%
\pgfpathlineto{\pgfqpoint{3.273038in}{2.625172in}}%
\pgfpathlineto{\pgfqpoint{3.276210in}{2.625998in}}%
\pgfpathlineto{\pgfqpoint{3.279382in}{2.625988in}}%
\pgfpathlineto{\pgfqpoint{3.282554in}{2.626243in}}%
\pgfpathlineto{\pgfqpoint{3.285726in}{2.626373in}}%
\pgfpathlineto{\pgfqpoint{3.288899in}{2.626606in}}%
\pgfpathlineto{\pgfqpoint{3.292071in}{2.627660in}}%
\pgfpathlineto{\pgfqpoint{3.295243in}{2.627904in}}%
\pgfpathlineto{\pgfqpoint{3.298415in}{2.628884in}}%
\pgfpathlineto{\pgfqpoint{3.301587in}{2.629144in}}%
\pgfpathlineto{\pgfqpoint{3.304759in}{2.629390in}}%
\pgfpathlineto{\pgfqpoint{3.307931in}{2.629992in}}%
\pgfpathlineto{\pgfqpoint{3.311103in}{2.630368in}}%
\pgfpathlineto{\pgfqpoint{3.314275in}{2.630385in}}%
\pgfpathlineto{\pgfqpoint{3.317447in}{2.630690in}}%
\pgfpathlineto{\pgfqpoint{3.320619in}{2.630543in}}%
\pgfpathlineto{\pgfqpoint{3.323791in}{2.631003in}}%
\pgfpathlineto{\pgfqpoint{3.326963in}{2.631385in}}%
\pgfpathlineto{\pgfqpoint{3.330135in}{2.631897in}}%
\pgfpathlineto{\pgfqpoint{3.333307in}{2.632044in}}%
\pgfpathlineto{\pgfqpoint{3.336479in}{2.633024in}}%
\pgfpathlineto{\pgfqpoint{3.339651in}{2.632924in}}%
\pgfpathlineto{\pgfqpoint{3.342823in}{2.632941in}}%
\pgfpathlineto{\pgfqpoint{3.345995in}{2.633174in}}%
\pgfpathlineto{\pgfqpoint{3.349167in}{2.633196in}}%
\pgfpathlineto{\pgfqpoint{3.352339in}{2.633044in}}%
\pgfpathlineto{\pgfqpoint{3.355511in}{2.633068in}}%
\pgfpathlineto{\pgfqpoint{3.358683in}{2.633171in}}%
\pgfpathlineto{\pgfqpoint{3.361855in}{2.633135in}}%
\pgfpathlineto{\pgfqpoint{3.365028in}{2.633157in}}%
\pgfpathlineto{\pgfqpoint{3.368200in}{2.633756in}}%
\pgfpathlineto{\pgfqpoint{3.371372in}{2.633636in}}%
\pgfpathlineto{\pgfqpoint{3.374544in}{2.634335in}}%
\pgfpathlineto{\pgfqpoint{3.377716in}{2.634245in}}%
\pgfpathlineto{\pgfqpoint{3.380888in}{2.634435in}}%
\pgfpathlineto{\pgfqpoint{3.384060in}{2.634574in}}%
\pgfpathlineto{\pgfqpoint{3.387232in}{2.635034in}}%
\pgfpathlineto{\pgfqpoint{3.390404in}{2.635150in}}%
\pgfpathlineto{\pgfqpoint{3.393576in}{2.635111in}}%
\pgfpathlineto{\pgfqpoint{3.396748in}{2.634860in}}%
\pgfpathlineto{\pgfqpoint{3.399920in}{2.634564in}}%
\pgfpathlineto{\pgfqpoint{3.403092in}{2.634488in}}%
\pgfpathlineto{\pgfqpoint{3.406264in}{2.634401in}}%
\pgfpathlineto{\pgfqpoint{3.409436in}{2.634837in}}%
\pgfpathlineto{\pgfqpoint{3.412608in}{2.635061in}}%
\pgfpathlineto{\pgfqpoint{3.415780in}{2.634630in}}%
\pgfpathlineto{\pgfqpoint{3.418952in}{2.634384in}}%
\pgfpathlineto{\pgfqpoint{3.422124in}{2.634255in}}%
\pgfpathlineto{\pgfqpoint{3.425296in}{2.634070in}}%
\pgfpathlineto{\pgfqpoint{3.428468in}{2.634042in}}%
\pgfpathlineto{\pgfqpoint{3.431640in}{2.634375in}}%
\pgfpathlineto{\pgfqpoint{3.434812in}{2.634191in}}%
\pgfpathlineto{\pgfqpoint{3.437984in}{2.634073in}}%
\pgfpathlineto{\pgfqpoint{3.441156in}{2.634688in}}%
\pgfpathlineto{\pgfqpoint{3.444329in}{2.635009in}}%
\pgfpathlineto{\pgfqpoint{3.447501in}{2.634793in}}%
\pgfpathlineto{\pgfqpoint{3.450673in}{2.634853in}}%
\pgfpathlineto{\pgfqpoint{3.453845in}{2.634844in}}%
\pgfpathlineto{\pgfqpoint{3.457017in}{2.635046in}}%
\pgfpathlineto{\pgfqpoint{3.460189in}{2.635043in}}%
\pgfpathlineto{\pgfqpoint{3.463361in}{2.634773in}}%
\pgfpathlineto{\pgfqpoint{3.466533in}{2.634354in}}%
\pgfpathlineto{\pgfqpoint{3.469705in}{2.634625in}}%
\pgfpathlineto{\pgfqpoint{3.472877in}{2.634250in}}%
\pgfpathlineto{\pgfqpoint{3.476049in}{2.634526in}}%
\pgfpathlineto{\pgfqpoint{3.479221in}{2.634633in}}%
\pgfpathlineto{\pgfqpoint{3.482393in}{2.635000in}}%
\pgfpathlineto{\pgfqpoint{3.485565in}{2.635572in}}%
\pgfpathlineto{\pgfqpoint{3.488737in}{2.636028in}}%
\pgfpathlineto{\pgfqpoint{3.491909in}{2.636290in}}%
\pgfpathlineto{\pgfqpoint{3.495081in}{2.636468in}}%
\pgfpathlineto{\pgfqpoint{3.498253in}{2.636259in}}%
\pgfpathlineto{\pgfqpoint{3.501425in}{2.636052in}}%
\pgfpathlineto{\pgfqpoint{3.504597in}{2.636666in}}%
\pgfpathlineto{\pgfqpoint{3.507769in}{2.636631in}}%
\pgfpathlineto{\pgfqpoint{3.510941in}{2.636524in}}%
\pgfpathlineto{\pgfqpoint{3.514113in}{2.636251in}}%
\pgfpathlineto{\pgfqpoint{3.517285in}{2.635960in}}%
\pgfpathlineto{\pgfqpoint{3.520457in}{2.635640in}}%
\pgfpathlineto{\pgfqpoint{3.523630in}{2.635369in}}%
\pgfpathlineto{\pgfqpoint{3.526802in}{2.635233in}}%
\pgfpathlineto{\pgfqpoint{3.529974in}{2.635635in}}%
\pgfpathlineto{\pgfqpoint{3.533146in}{2.635758in}}%
\pgfpathlineto{\pgfqpoint{3.536318in}{2.635972in}}%
\pgfpathlineto{\pgfqpoint{3.539490in}{2.635996in}}%
\pgfpathlineto{\pgfqpoint{3.542662in}{2.635827in}}%
\pgfpathlineto{\pgfqpoint{3.545834in}{2.636279in}}%
\pgfpathlineto{\pgfqpoint{3.549006in}{2.636450in}}%
\pgfpathlineto{\pgfqpoint{3.552178in}{2.636805in}}%
\pgfpathlineto{\pgfqpoint{3.555350in}{2.636884in}}%
\pgfpathlineto{\pgfqpoint{3.558522in}{2.637324in}}%
\pgfpathlineto{\pgfqpoint{3.561694in}{2.637594in}}%
\pgfpathlineto{\pgfqpoint{3.564866in}{2.637760in}}%
\pgfpathlineto{\pgfqpoint{3.568038in}{2.637497in}}%
\pgfpathlineto{\pgfqpoint{3.571210in}{2.637633in}}%
\pgfpathlineto{\pgfqpoint{3.574382in}{2.637384in}}%
\pgfpathlineto{\pgfqpoint{3.577554in}{2.637388in}}%
\pgfpathlineto{\pgfqpoint{3.580726in}{2.637654in}}%
\pgfpathlineto{\pgfqpoint{3.583898in}{2.637007in}}%
\pgfpathlineto{\pgfqpoint{3.587070in}{2.637342in}}%
\pgfpathlineto{\pgfqpoint{3.590242in}{2.637526in}}%
\pgfpathlineto{\pgfqpoint{3.593414in}{2.637567in}}%
\pgfpathlineto{\pgfqpoint{3.596586in}{2.637582in}}%
\pgfpathlineto{\pgfqpoint{3.599759in}{2.636834in}}%
\pgfpathlineto{\pgfqpoint{3.602931in}{2.637108in}}%
\pgfpathlineto{\pgfqpoint{3.606103in}{2.636973in}}%
\pgfpathlineto{\pgfqpoint{3.609275in}{2.637094in}}%
\pgfpathlineto{\pgfqpoint{3.612447in}{2.636969in}}%
\pgfpathlineto{\pgfqpoint{3.615619in}{2.637137in}}%
\pgfpathlineto{\pgfqpoint{3.618791in}{2.637424in}}%
\pgfpathlineto{\pgfqpoint{3.621963in}{2.637313in}}%
\pgfpathlineto{\pgfqpoint{3.625135in}{2.637312in}}%
\pgfpathlineto{\pgfqpoint{3.628307in}{2.637496in}}%
\pgfpathlineto{\pgfqpoint{3.631479in}{2.637709in}}%
\pgfpathlineto{\pgfqpoint{3.634651in}{2.637627in}}%
\pgfpathlineto{\pgfqpoint{3.637823in}{2.637339in}}%
\pgfpathlineto{\pgfqpoint{3.640995in}{2.637236in}}%
\pgfpathlineto{\pgfqpoint{3.644167in}{2.637283in}}%
\pgfpathlineto{\pgfqpoint{3.647339in}{2.636719in}}%
\pgfpathlineto{\pgfqpoint{3.650511in}{2.636799in}}%
\pgfpathlineto{\pgfqpoint{3.653683in}{2.636407in}}%
\pgfpathlineto{\pgfqpoint{3.656855in}{2.636037in}}%
\pgfpathlineto{\pgfqpoint{3.660027in}{2.635965in}}%
\pgfpathlineto{\pgfqpoint{3.663199in}{2.636253in}}%
\pgfpathlineto{\pgfqpoint{3.666371in}{2.636441in}}%
\pgfpathlineto{\pgfqpoint{3.669543in}{2.635176in}}%
\pgfpathlineto{\pgfqpoint{3.672715in}{2.635016in}}%
\pgfpathlineto{\pgfqpoint{3.675887in}{2.635529in}}%
\pgfpathlineto{\pgfqpoint{3.679060in}{2.635549in}}%
\pgfpathlineto{\pgfqpoint{3.682232in}{2.635570in}}%
\pgfpathlineto{\pgfqpoint{3.685404in}{2.635426in}}%
\pgfpathlineto{\pgfqpoint{3.688576in}{2.634932in}}%
\pgfpathlineto{\pgfqpoint{3.691748in}{2.634653in}}%
\pgfpathlineto{\pgfqpoint{3.694920in}{2.634604in}}%
\pgfpathlineto{\pgfqpoint{3.698092in}{2.634828in}}%
\pgfpathlineto{\pgfqpoint{3.701264in}{2.635283in}}%
\pgfpathlineto{\pgfqpoint{3.704436in}{2.636122in}}%
\pgfpathlineto{\pgfqpoint{3.707608in}{2.636261in}}%
\pgfpathlineto{\pgfqpoint{3.710780in}{2.636214in}}%
\pgfpathlineto{\pgfqpoint{3.713952in}{2.636538in}}%
\pgfpathlineto{\pgfqpoint{3.717124in}{2.636997in}}%
\pgfpathlineto{\pgfqpoint{3.720296in}{2.636881in}}%
\pgfpathlineto{\pgfqpoint{3.723468in}{2.636874in}}%
\pgfpathlineto{\pgfqpoint{3.726640in}{2.637021in}}%
\pgfpathlineto{\pgfqpoint{3.729812in}{2.637154in}}%
\pgfpathlineto{\pgfqpoint{3.732984in}{2.637256in}}%
\pgfpathlineto{\pgfqpoint{3.736156in}{2.637425in}}%
\pgfpathlineto{\pgfqpoint{3.739328in}{2.637436in}}%
\pgfpathlineto{\pgfqpoint{3.742500in}{2.637415in}}%
\pgfpathlineto{\pgfqpoint{3.745672in}{2.637229in}}%
\pgfpathlineto{\pgfqpoint{3.748844in}{2.637191in}}%
\pgfpathlineto{\pgfqpoint{3.752016in}{2.637005in}}%
\pgfpathlineto{\pgfqpoint{3.755188in}{2.637149in}}%
\pgfpathlineto{\pgfqpoint{3.758361in}{2.636895in}}%
\pgfpathlineto{\pgfqpoint{3.761533in}{2.637027in}}%
\pgfpathlineto{\pgfqpoint{3.764705in}{2.637685in}}%
\pgfpathlineto{\pgfqpoint{3.767877in}{2.637831in}}%
\pgfpathlineto{\pgfqpoint{3.771049in}{2.637653in}}%
\pgfpathlineto{\pgfqpoint{3.774221in}{2.637600in}}%
\pgfpathlineto{\pgfqpoint{3.777393in}{2.637929in}}%
\pgfpathlineto{\pgfqpoint{3.780565in}{2.638000in}}%
\pgfpathlineto{\pgfqpoint{3.783737in}{2.638242in}}%
\pgfpathlineto{\pgfqpoint{3.786909in}{2.638010in}}%
\pgfpathlineto{\pgfqpoint{3.790081in}{2.638069in}}%
\pgfpathlineto{\pgfqpoint{3.793253in}{2.638519in}}%
\pgfpathlineto{\pgfqpoint{3.796425in}{2.638485in}}%
\pgfpathlineto{\pgfqpoint{3.799597in}{2.639004in}}%
\pgfpathlineto{\pgfqpoint{3.802769in}{2.639313in}}%
\pgfpathlineto{\pgfqpoint{3.805941in}{2.639059in}}%
\pgfpathlineto{\pgfqpoint{3.809113in}{2.638714in}}%
\pgfpathlineto{\pgfqpoint{3.812285in}{2.639373in}}%
\pgfpathlineto{\pgfqpoint{3.815457in}{2.639631in}}%
\pgfpathlineto{\pgfqpoint{3.818629in}{2.640013in}}%
\pgfpathlineto{\pgfqpoint{3.821801in}{2.639916in}}%
\pgfpathlineto{\pgfqpoint{3.824973in}{2.640056in}}%
\pgfpathlineto{\pgfqpoint{3.828145in}{2.640428in}}%
\pgfpathlineto{\pgfqpoint{3.831317in}{2.640363in}}%
\pgfpathlineto{\pgfqpoint{3.834490in}{2.640268in}}%
\pgfpathlineto{\pgfqpoint{3.837662in}{2.639771in}}%
\pgfpathlineto{\pgfqpoint{3.840834in}{2.639522in}}%
\pgfpathlineto{\pgfqpoint{3.844006in}{2.639239in}}%
\pgfpathlineto{\pgfqpoint{3.847178in}{2.639073in}}%
\pgfpathlineto{\pgfqpoint{3.850350in}{2.638797in}}%
\pgfpathlineto{\pgfqpoint{3.853522in}{2.638629in}}%
\pgfpathlineto{\pgfqpoint{3.856694in}{2.638723in}}%
\pgfpathlineto{\pgfqpoint{3.859866in}{2.638976in}}%
\pgfpathlineto{\pgfqpoint{3.863038in}{2.638841in}}%
\pgfpathlineto{\pgfqpoint{3.866210in}{2.638939in}}%
\pgfpathlineto{\pgfqpoint{3.869382in}{2.638853in}}%
\pgfpathlineto{\pgfqpoint{3.872554in}{2.638769in}}%
\pgfpathlineto{\pgfqpoint{3.875726in}{2.639141in}}%
\pgfpathlineto{\pgfqpoint{3.878898in}{2.639547in}}%
\pgfpathlineto{\pgfqpoint{3.882070in}{2.639371in}}%
\pgfpathlineto{\pgfqpoint{3.885242in}{2.638816in}}%
\pgfpathlineto{\pgfqpoint{3.888414in}{2.638770in}}%
\pgfpathlineto{\pgfqpoint{3.891586in}{2.638725in}}%
\pgfpathlineto{\pgfqpoint{3.894758in}{2.638700in}}%
\pgfpathlineto{\pgfqpoint{3.897930in}{2.638705in}}%
\pgfpathlineto{\pgfqpoint{3.901102in}{2.638591in}}%
\pgfpathlineto{\pgfqpoint{3.904274in}{2.638775in}}%
\pgfpathlineto{\pgfqpoint{3.907446in}{2.638788in}}%
\pgfpathlineto{\pgfqpoint{3.910618in}{2.638830in}}%
\pgfpathlineto{\pgfqpoint{3.913791in}{2.638923in}}%
\pgfpathlineto{\pgfqpoint{3.916963in}{2.639117in}}%
\pgfpathlineto{\pgfqpoint{3.920135in}{2.639377in}}%
\pgfpathlineto{\pgfqpoint{3.923307in}{2.639293in}}%
\pgfpathlineto{\pgfqpoint{3.926479in}{2.639259in}}%
\pgfpathlineto{\pgfqpoint{3.929651in}{2.639026in}}%
\pgfpathlineto{\pgfqpoint{3.932823in}{2.639223in}}%
\pgfpathlineto{\pgfqpoint{3.935995in}{2.639169in}}%
\pgfpathlineto{\pgfqpoint{3.939167in}{2.639432in}}%
\pgfpathlineto{\pgfqpoint{3.942339in}{2.639764in}}%
\pgfpathlineto{\pgfqpoint{3.945511in}{2.639611in}}%
\pgfpathlineto{\pgfqpoint{3.948683in}{2.639627in}}%
\pgfpathlineto{\pgfqpoint{3.951855in}{2.639721in}}%
\pgfpathlineto{\pgfqpoint{3.955027in}{2.639063in}}%
\pgfpathlineto{\pgfqpoint{3.958199in}{2.639308in}}%
\pgfpathlineto{\pgfqpoint{3.961371in}{2.638677in}}%
\pgfpathlineto{\pgfqpoint{3.964543in}{2.638646in}}%
\pgfpathlineto{\pgfqpoint{3.967715in}{2.638599in}}%
\pgfpathlineto{\pgfqpoint{3.970887in}{2.638900in}}%
\pgfpathlineto{\pgfqpoint{3.974059in}{2.639451in}}%
\pgfpathlineto{\pgfqpoint{3.977231in}{2.639335in}}%
\pgfpathlineto{\pgfqpoint{3.980403in}{2.640255in}}%
\pgfpathlineto{\pgfqpoint{3.983575in}{2.640255in}}%
\pgfpathlineto{\pgfqpoint{3.986747in}{2.640430in}}%
\pgfpathlineto{\pgfqpoint{3.989919in}{2.640477in}}%
\pgfpathlineto{\pgfqpoint{3.993092in}{2.640553in}}%
\pgfpathlineto{\pgfqpoint{3.996264in}{2.640898in}}%
\pgfpathlineto{\pgfqpoint{3.999436in}{2.640892in}}%
\pgfpathlineto{\pgfqpoint{4.002608in}{2.640740in}}%
\pgfpathlineto{\pgfqpoint{4.005780in}{2.640869in}}%
\pgfpathlineto{\pgfqpoint{4.008952in}{2.641288in}}%
\pgfpathlineto{\pgfqpoint{4.012124in}{2.641209in}}%
\pgfpathlineto{\pgfqpoint{4.015296in}{2.641010in}}%
\pgfpathlineto{\pgfqpoint{4.018468in}{2.640846in}}%
\pgfpathlineto{\pgfqpoint{4.021640in}{2.640657in}}%
\pgfpathlineto{\pgfqpoint{4.024812in}{2.640824in}}%
\pgfpathlineto{\pgfqpoint{4.027984in}{2.641020in}}%
\pgfpathlineto{\pgfqpoint{4.031156in}{2.641142in}}%
\pgfpathlineto{\pgfqpoint{4.034328in}{2.641223in}}%
\pgfpathlineto{\pgfqpoint{4.037500in}{2.641525in}}%
\pgfpathlineto{\pgfqpoint{4.040672in}{2.641319in}}%
\pgfpathlineto{\pgfqpoint{4.043844in}{2.641133in}}%
\pgfpathlineto{\pgfqpoint{4.047016in}{2.641054in}}%
\pgfpathlineto{\pgfqpoint{4.050188in}{2.641121in}}%
\pgfpathlineto{\pgfqpoint{4.053360in}{2.640725in}}%
\pgfpathlineto{\pgfqpoint{4.056532in}{2.640786in}}%
\pgfpathlineto{\pgfqpoint{4.059704in}{2.641093in}}%
\pgfpathlineto{\pgfqpoint{4.062876in}{2.641263in}}%
\pgfpathlineto{\pgfqpoint{4.066048in}{2.641523in}}%
\pgfpathlineto{\pgfqpoint{4.069221in}{2.641427in}}%
\pgfpathlineto{\pgfqpoint{4.072393in}{2.641271in}}%
\pgfpathlineto{\pgfqpoint{4.075565in}{2.641139in}}%
\pgfpathlineto{\pgfqpoint{4.078737in}{2.640927in}}%
\pgfpathlineto{\pgfqpoint{4.081909in}{2.641007in}}%
\pgfpathlineto{\pgfqpoint{4.085081in}{2.641361in}}%
\pgfpathlineto{\pgfqpoint{4.088253in}{2.641396in}}%
\pgfpathlineto{\pgfqpoint{4.091425in}{2.641487in}}%
\pgfpathlineto{\pgfqpoint{4.094597in}{2.642055in}}%
\pgfpathlineto{\pgfqpoint{4.097769in}{2.642235in}}%
\pgfpathlineto{\pgfqpoint{4.100941in}{2.642343in}}%
\pgfpathlineto{\pgfqpoint{4.104113in}{2.642261in}}%
\pgfpathlineto{\pgfqpoint{4.107285in}{2.642743in}}%
\pgfpathlineto{\pgfqpoint{4.110457in}{2.642851in}}%
\pgfpathlineto{\pgfqpoint{4.113629in}{2.643346in}}%
\pgfpathlineto{\pgfqpoint{4.116801in}{2.643350in}}%
\pgfpathlineto{\pgfqpoint{4.119973in}{2.643490in}}%
\pgfpathlineto{\pgfqpoint{4.123145in}{2.643673in}}%
\pgfpathlineto{\pgfqpoint{4.126317in}{2.643638in}}%
\pgfpathlineto{\pgfqpoint{4.129489in}{2.643730in}}%
\pgfpathlineto{\pgfqpoint{4.132661in}{2.643800in}}%
\pgfpathlineto{\pgfqpoint{4.135833in}{2.644051in}}%
\pgfpathlineto{\pgfqpoint{4.139005in}{2.643775in}}%
\pgfpathlineto{\pgfqpoint{4.142177in}{2.643691in}}%
\pgfpathlineto{\pgfqpoint{4.145349in}{2.643720in}}%
\pgfpathlineto{\pgfqpoint{4.148522in}{2.643787in}}%
\pgfpathlineto{\pgfqpoint{4.151694in}{2.643748in}}%
\pgfpathlineto{\pgfqpoint{4.154866in}{2.643997in}}%
\pgfpathlineto{\pgfqpoint{4.158038in}{2.644034in}}%
\pgfpathlineto{\pgfqpoint{4.161210in}{2.644224in}}%
\pgfpathlineto{\pgfqpoint{4.164382in}{2.644184in}}%
\pgfpathlineto{\pgfqpoint{4.167554in}{2.644367in}}%
\pgfpathlineto{\pgfqpoint{4.170726in}{2.644804in}}%
\pgfpathlineto{\pgfqpoint{4.173898in}{2.645229in}}%
\pgfpathlineto{\pgfqpoint{4.177070in}{2.645252in}}%
\pgfpathlineto{\pgfqpoint{4.180242in}{2.645174in}}%
\pgfpathlineto{\pgfqpoint{4.183414in}{2.645355in}}%
\pgfpathlineto{\pgfqpoint{4.186586in}{2.645590in}}%
\pgfpathlineto{\pgfqpoint{4.189758in}{2.645512in}}%
\pgfpathlineto{\pgfqpoint{4.192930in}{2.645447in}}%
\pgfpathlineto{\pgfqpoint{4.196102in}{2.645530in}}%
\pgfpathlineto{\pgfqpoint{4.199274in}{2.645587in}}%
\pgfpathlineto{\pgfqpoint{4.202446in}{2.645894in}}%
\pgfpathlineto{\pgfqpoint{4.205618in}{2.646075in}}%
\pgfpathlineto{\pgfqpoint{4.208790in}{2.646229in}}%
\pgfpathlineto{\pgfqpoint{4.211962in}{2.645852in}}%
\pgfpathlineto{\pgfqpoint{4.215134in}{2.645818in}}%
\pgfpathlineto{\pgfqpoint{4.218306in}{2.645325in}}%
\pgfpathlineto{\pgfqpoint{4.221478in}{2.645413in}}%
\pgfpathlineto{\pgfqpoint{4.224650in}{2.645852in}}%
\pgfpathlineto{\pgfqpoint{4.227823in}{2.645397in}}%
\pgfpathlineto{\pgfqpoint{4.230995in}{2.645478in}}%
\pgfpathlineto{\pgfqpoint{4.234167in}{2.645610in}}%
\pgfpathlineto{\pgfqpoint{4.237339in}{2.645522in}}%
\pgfpathlineto{\pgfqpoint{4.240511in}{2.645481in}}%
\pgfpathlineto{\pgfqpoint{4.243683in}{2.645728in}}%
\pgfpathlineto{\pgfqpoint{4.246855in}{2.645547in}}%
\pgfpathlineto{\pgfqpoint{4.250027in}{2.646139in}}%
\pgfpathlineto{\pgfqpoint{4.253199in}{2.646564in}}%
\pgfpathlineto{\pgfqpoint{4.256371in}{2.647087in}}%
\pgfpathlineto{\pgfqpoint{4.259543in}{2.647229in}}%
\pgfpathlineto{\pgfqpoint{4.262715in}{2.647095in}}%
\pgfpathlineto{\pgfqpoint{4.265887in}{2.647498in}}%
\pgfpathlineto{\pgfqpoint{4.269059in}{2.647562in}}%
\pgfpathlineto{\pgfqpoint{4.272231in}{2.647457in}}%
\pgfpathlineto{\pgfqpoint{4.275403in}{2.646925in}}%
\pgfpathlineto{\pgfqpoint{4.278575in}{2.646694in}}%
\pgfpathlineto{\pgfqpoint{4.281747in}{2.647149in}}%
\pgfpathlineto{\pgfqpoint{4.284919in}{2.647164in}}%
\pgfpathlineto{\pgfqpoint{4.288091in}{2.647105in}}%
\pgfpathlineto{\pgfqpoint{4.291263in}{2.646840in}}%
\pgfpathlineto{\pgfqpoint{4.294435in}{2.647253in}}%
\pgfpathlineto{\pgfqpoint{4.297607in}{2.647619in}}%
\pgfpathlineto{\pgfqpoint{4.300779in}{2.647692in}}%
\pgfpathlineto{\pgfqpoint{4.303952in}{2.647345in}}%
\pgfpathlineto{\pgfqpoint{4.307124in}{2.647551in}}%
\pgfpathlineto{\pgfqpoint{4.310296in}{2.647509in}}%
\pgfpathlineto{\pgfqpoint{4.313468in}{2.647484in}}%
\pgfpathlineto{\pgfqpoint{4.316640in}{2.647204in}}%
\pgfpathlineto{\pgfqpoint{4.319812in}{2.647231in}}%
\pgfpathlineto{\pgfqpoint{4.322984in}{2.647536in}}%
\pgfpathlineto{\pgfqpoint{4.326156in}{2.647080in}}%
\pgfpathlineto{\pgfqpoint{4.329328in}{2.646942in}}%
\pgfpathlineto{\pgfqpoint{4.332500in}{2.647114in}}%
\pgfpathlineto{\pgfqpoint{4.335672in}{2.647219in}}%
\pgfpathlineto{\pgfqpoint{4.338844in}{2.647515in}}%
\pgfpathlineto{\pgfqpoint{4.342016in}{2.647417in}}%
\pgfpathlineto{\pgfqpoint{4.345188in}{2.647262in}}%
\pgfpathlineto{\pgfqpoint{4.348360in}{2.647693in}}%
\pgfpathlineto{\pgfqpoint{4.351532in}{2.647670in}}%
\pgfpathlineto{\pgfqpoint{4.354704in}{2.647826in}}%
\pgfpathlineto{\pgfqpoint{4.357876in}{2.648386in}}%
\pgfpathlineto{\pgfqpoint{4.361048in}{2.648873in}}%
\pgfpathlineto{\pgfqpoint{4.364220in}{2.648819in}}%
\pgfpathlineto{\pgfqpoint{4.367392in}{2.649190in}}%
\pgfpathlineto{\pgfqpoint{4.370564in}{2.649061in}}%
\pgfpathlineto{\pgfqpoint{4.373736in}{2.649579in}}%
\pgfpathlineto{\pgfqpoint{4.376908in}{2.649533in}}%
\pgfpathlineto{\pgfqpoint{4.380080in}{2.650182in}}%
\pgfpathlineto{\pgfqpoint{4.383253in}{2.650374in}}%
\pgfpathlineto{\pgfqpoint{4.386425in}{2.650539in}}%
\pgfpathlineto{\pgfqpoint{4.389597in}{2.650487in}}%
\pgfpathlineto{\pgfqpoint{4.392769in}{2.650688in}}%
\pgfpathlineto{\pgfqpoint{4.395941in}{2.651321in}}%
\pgfpathlineto{\pgfqpoint{4.399113in}{2.652263in}}%
\pgfpathlineto{\pgfqpoint{4.402285in}{2.652487in}}%
\pgfpathlineto{\pgfqpoint{4.405457in}{2.652532in}}%
\pgfpathlineto{\pgfqpoint{4.408629in}{2.652408in}}%
\pgfpathlineto{\pgfqpoint{4.411801in}{2.652639in}}%
\pgfpathlineto{\pgfqpoint{4.414973in}{2.653235in}}%
\pgfpathlineto{\pgfqpoint{4.418145in}{2.653467in}}%
\pgfpathlineto{\pgfqpoint{4.421317in}{2.653725in}}%
\pgfpathlineto{\pgfqpoint{4.424489in}{2.653834in}}%
\pgfpathlineto{\pgfqpoint{4.427661in}{2.654240in}}%
\pgfpathlineto{\pgfqpoint{4.430833in}{2.653401in}}%
\pgfpathlineto{\pgfqpoint{4.434005in}{2.653348in}}%
\pgfpathlineto{\pgfqpoint{4.437177in}{2.653430in}}%
\pgfpathlineto{\pgfqpoint{4.440349in}{2.653376in}}%
\pgfpathlineto{\pgfqpoint{4.443521in}{2.653039in}}%
\pgfpathlineto{\pgfqpoint{4.446693in}{2.652627in}}%
\pgfpathlineto{\pgfqpoint{4.449865in}{2.652731in}}%
\pgfpathlineto{\pgfqpoint{4.453037in}{2.652971in}}%
\pgfpathlineto{\pgfqpoint{4.456209in}{2.652918in}}%
\pgfpathlineto{\pgfqpoint{4.459381in}{2.652846in}}%
\pgfpathlineto{\pgfqpoint{4.462554in}{2.653171in}}%
\pgfpathlineto{\pgfqpoint{4.465726in}{2.653357in}}%
\pgfpathlineto{\pgfqpoint{4.468898in}{2.653416in}}%
\pgfpathlineto{\pgfqpoint{4.472070in}{2.653325in}}%
\pgfpathlineto{\pgfqpoint{4.475242in}{2.653409in}}%
\pgfpathlineto{\pgfqpoint{4.478414in}{2.653157in}}%
\pgfpathlineto{\pgfqpoint{4.481586in}{2.652945in}}%
\pgfpathlineto{\pgfqpoint{4.484758in}{2.653266in}}%
\pgfpathlineto{\pgfqpoint{4.487930in}{2.653410in}}%
\pgfpathlineto{\pgfqpoint{4.491102in}{2.653535in}}%
\pgfpathlineto{\pgfqpoint{4.494274in}{2.653217in}}%
\pgfpathlineto{\pgfqpoint{4.497446in}{2.653026in}}%
\pgfpathlineto{\pgfqpoint{4.500618in}{2.652913in}}%
\pgfpathlineto{\pgfqpoint{4.503790in}{2.653053in}}%
\pgfpathlineto{\pgfqpoint{4.506962in}{2.652995in}}%
\pgfpathlineto{\pgfqpoint{4.510134in}{2.653341in}}%
\pgfpathlineto{\pgfqpoint{4.513306in}{2.653320in}}%
\pgfpathlineto{\pgfqpoint{4.516478in}{2.653742in}}%
\pgfpathlineto{\pgfqpoint{4.519650in}{2.654003in}}%
\pgfpathlineto{\pgfqpoint{4.522822in}{2.654017in}}%
\pgfpathlineto{\pgfqpoint{4.525994in}{2.654102in}}%
\pgfpathlineto{\pgfqpoint{4.529166in}{2.653724in}}%
\pgfpathlineto{\pgfqpoint{4.532338in}{2.653716in}}%
\pgfpathlineto{\pgfqpoint{4.535510in}{2.653831in}}%
\pgfpathlineto{\pgfqpoint{4.538683in}{2.653564in}}%
\pgfpathlineto{\pgfqpoint{4.541855in}{2.653491in}}%
\pgfpathlineto{\pgfqpoint{4.545027in}{2.653599in}}%
\pgfpathlineto{\pgfqpoint{4.548199in}{2.653620in}}%
\pgfpathlineto{\pgfqpoint{4.551371in}{2.653554in}}%
\pgfpathlineto{\pgfqpoint{4.554543in}{2.653538in}}%
\pgfpathlineto{\pgfqpoint{4.557715in}{2.653554in}}%
\pgfpathlineto{\pgfqpoint{4.560887in}{2.653821in}}%
\pgfpathlineto{\pgfqpoint{4.564059in}{2.653628in}}%
\pgfpathlineto{\pgfqpoint{4.567231in}{2.653402in}}%
\pgfpathlineto{\pgfqpoint{4.570403in}{2.653301in}}%
\pgfpathlineto{\pgfqpoint{4.573575in}{2.653553in}}%
\pgfpathlineto{\pgfqpoint{4.576747in}{2.653690in}}%
\pgfpathlineto{\pgfqpoint{4.579919in}{2.653550in}}%
\pgfpathlineto{\pgfqpoint{4.583091in}{2.653387in}}%
\pgfpathlineto{\pgfqpoint{4.586263in}{2.653253in}}%
\pgfpathlineto{\pgfqpoint{4.589435in}{2.653294in}}%
\pgfpathlineto{\pgfqpoint{4.592607in}{2.653180in}}%
\pgfpathlineto{\pgfqpoint{4.595779in}{2.653314in}}%
\pgfpathlineto{\pgfqpoint{4.598951in}{2.653696in}}%
\pgfpathlineto{\pgfqpoint{4.602123in}{2.653934in}}%
\pgfpathlineto{\pgfqpoint{4.605295in}{2.654375in}}%
\pgfpathlineto{\pgfqpoint{4.608467in}{2.654576in}}%
\pgfpathlineto{\pgfqpoint{4.611639in}{2.654712in}}%
\pgfpathlineto{\pgfqpoint{4.614811in}{2.654504in}}%
\pgfpathlineto{\pgfqpoint{4.617984in}{2.654518in}}%
\pgfpathlineto{\pgfqpoint{4.621156in}{2.654443in}}%
\pgfpathlineto{\pgfqpoint{4.624328in}{2.654595in}}%
\pgfpathlineto{\pgfqpoint{4.627500in}{2.654675in}}%
\pgfpathlineto{\pgfqpoint{4.630672in}{2.654719in}}%
\pgfpathlineto{\pgfqpoint{4.633844in}{2.654399in}}%
\pgfpathlineto{\pgfqpoint{4.637016in}{2.654360in}}%
\pgfpathlineto{\pgfqpoint{4.640188in}{2.654422in}}%
\pgfpathlineto{\pgfqpoint{4.643360in}{2.654198in}}%
\pgfpathlineto{\pgfqpoint{4.646532in}{2.654182in}}%
\pgfpathlineto{\pgfqpoint{4.649704in}{2.653912in}}%
\pgfpathlineto{\pgfqpoint{4.652876in}{2.653720in}}%
\pgfpathlineto{\pgfqpoint{4.656048in}{2.653429in}}%
\pgfpathlineto{\pgfqpoint{4.659220in}{2.653568in}}%
\pgfpathlineto{\pgfqpoint{4.662392in}{2.653629in}}%
\pgfpathlineto{\pgfqpoint{4.665564in}{2.653370in}}%
\pgfpathlineto{\pgfqpoint{4.668736in}{2.653025in}}%
\pgfpathlineto{\pgfqpoint{4.671908in}{2.653025in}}%
\pgfpathlineto{\pgfqpoint{4.675080in}{2.653073in}}%
\pgfpathlineto{\pgfqpoint{4.678252in}{2.652935in}}%
\pgfpathlineto{\pgfqpoint{4.681424in}{2.653460in}}%
\pgfpathlineto{\pgfqpoint{4.684596in}{2.653217in}}%
\pgfpathlineto{\pgfqpoint{4.687768in}{2.653150in}}%
\pgfpathlineto{\pgfqpoint{4.690940in}{2.653146in}}%
\pgfpathlineto{\pgfqpoint{4.694112in}{2.652720in}}%
\pgfpathlineto{\pgfqpoint{4.697285in}{2.652201in}}%
\pgfpathlineto{\pgfqpoint{4.700457in}{2.652231in}}%
\pgfpathlineto{\pgfqpoint{4.703629in}{2.651855in}}%
\pgfpathlineto{\pgfqpoint{4.706801in}{2.651502in}}%
\pgfpathlineto{\pgfqpoint{4.709973in}{2.651731in}}%
\pgfpathlineto{\pgfqpoint{4.713145in}{2.651655in}}%
\pgfpathlineto{\pgfqpoint{4.716317in}{2.651536in}}%
\pgfpathlineto{\pgfqpoint{4.719489in}{2.651644in}}%
\pgfpathlineto{\pgfqpoint{4.722661in}{2.652401in}}%
\pgfpathlineto{\pgfqpoint{4.725833in}{2.652535in}}%
\pgfpathlineto{\pgfqpoint{4.729005in}{2.652082in}}%
\pgfpathlineto{\pgfqpoint{4.732177in}{2.652409in}}%
\pgfpathlineto{\pgfqpoint{4.735349in}{2.652472in}}%
\pgfpathlineto{\pgfqpoint{4.738521in}{2.652018in}}%
\pgfpathlineto{\pgfqpoint{4.741693in}{2.652331in}}%
\pgfpathlineto{\pgfqpoint{4.744865in}{2.652716in}}%
\pgfpathlineto{\pgfqpoint{4.748037in}{2.653281in}}%
\pgfpathlineto{\pgfqpoint{4.751209in}{2.653192in}}%
\pgfpathlineto{\pgfqpoint{4.754381in}{2.653285in}}%
\pgfpathlineto{\pgfqpoint{4.757553in}{2.653361in}}%
\pgfpathlineto{\pgfqpoint{4.760725in}{2.653507in}}%
\pgfpathlineto{\pgfqpoint{4.763897in}{2.653749in}}%
\pgfpathlineto{\pgfqpoint{4.767069in}{2.653603in}}%
\pgfpathlineto{\pgfqpoint{4.770241in}{2.653534in}}%
\pgfpathlineto{\pgfqpoint{4.773414in}{2.653561in}}%
\pgfpathlineto{\pgfqpoint{4.776586in}{2.653294in}}%
\pgfpathlineto{\pgfqpoint{4.779758in}{2.653064in}}%
\pgfpathlineto{\pgfqpoint{4.782930in}{2.652982in}}%
\pgfpathlineto{\pgfqpoint{4.786102in}{2.652968in}}%
\pgfpathlineto{\pgfqpoint{4.789274in}{2.653119in}}%
\pgfpathlineto{\pgfqpoint{4.792446in}{2.652680in}}%
\pgfpathlineto{\pgfqpoint{4.795618in}{2.652293in}}%
\pgfpathlineto{\pgfqpoint{4.798790in}{2.652225in}}%
\pgfpathlineto{\pgfqpoint{4.801962in}{2.651921in}}%
\pgfpathlineto{\pgfqpoint{4.805134in}{2.651851in}}%
\pgfpathlineto{\pgfqpoint{4.808306in}{2.652249in}}%
\pgfpathlineto{\pgfqpoint{4.811478in}{2.652361in}}%
\pgfpathlineto{\pgfqpoint{4.814650in}{2.652787in}}%
\pgfpathlineto{\pgfqpoint{4.817822in}{2.652543in}}%
\pgfpathlineto{\pgfqpoint{4.820994in}{2.652248in}}%
\pgfpathlineto{\pgfqpoint{4.824166in}{2.652039in}}%
\pgfpathlineto{\pgfqpoint{4.827338in}{2.651991in}}%
\pgfpathlineto{\pgfqpoint{4.830510in}{2.651985in}}%
\pgfpathlineto{\pgfqpoint{4.833682in}{2.652244in}}%
\pgfpathlineto{\pgfqpoint{4.836854in}{2.652514in}}%
\pgfpathlineto{\pgfqpoint{4.840026in}{2.652386in}}%
\pgfpathlineto{\pgfqpoint{4.843198in}{2.652194in}}%
\pgfpathlineto{\pgfqpoint{4.846370in}{2.652166in}}%
\pgfpathlineto{\pgfqpoint{4.849542in}{2.652558in}}%
\pgfpathlineto{\pgfqpoint{4.852715in}{2.652495in}}%
\pgfpathlineto{\pgfqpoint{4.855887in}{2.652897in}}%
\pgfpathlineto{\pgfqpoint{4.859059in}{2.653248in}}%
\pgfpathlineto{\pgfqpoint{4.862231in}{2.653782in}}%
\pgfpathlineto{\pgfqpoint{4.865403in}{2.653852in}}%
\pgfpathlineto{\pgfqpoint{4.868575in}{2.653737in}}%
\pgfpathlineto{\pgfqpoint{4.871747in}{2.653563in}}%
\pgfpathlineto{\pgfqpoint{4.874919in}{2.653670in}}%
\pgfpathlineto{\pgfqpoint{4.878091in}{2.653621in}}%
\pgfpathlineto{\pgfqpoint{4.881263in}{2.653893in}}%
\pgfpathlineto{\pgfqpoint{4.884435in}{2.653486in}}%
\pgfpathlineto{\pgfqpoint{4.887607in}{2.653231in}}%
\pgfpathlineto{\pgfqpoint{4.890779in}{2.653363in}}%
\pgfpathlineto{\pgfqpoint{4.893951in}{2.653000in}}%
\pgfpathlineto{\pgfqpoint{4.897123in}{2.652660in}}%
\pgfpathlineto{\pgfqpoint{4.900295in}{2.652366in}}%
\pgfpathlineto{\pgfqpoint{4.903467in}{2.652446in}}%
\pgfpathlineto{\pgfqpoint{4.906639in}{2.652474in}}%
\pgfpathlineto{\pgfqpoint{4.909811in}{2.652243in}}%
\pgfpathlineto{\pgfqpoint{4.912983in}{2.652044in}}%
\pgfpathlineto{\pgfqpoint{4.916155in}{2.652188in}}%
\pgfpathlineto{\pgfqpoint{4.919327in}{2.652216in}}%
\pgfpathlineto{\pgfqpoint{4.922499in}{2.652374in}}%
\pgfpathlineto{\pgfqpoint{4.925671in}{2.652294in}}%
\pgfpathlineto{\pgfqpoint{4.928844in}{2.652067in}}%
\pgfpathlineto{\pgfqpoint{4.932016in}{2.651865in}}%
\pgfpathlineto{\pgfqpoint{4.935188in}{2.652149in}}%
\pgfpathlineto{\pgfqpoint{4.938360in}{2.652241in}}%
\pgfpathlineto{\pgfqpoint{4.941532in}{2.652255in}}%
\pgfpathlineto{\pgfqpoint{4.944704in}{2.652452in}}%
\pgfpathlineto{\pgfqpoint{4.947876in}{2.652239in}}%
\pgfpathlineto{\pgfqpoint{4.951048in}{2.652159in}}%
\pgfpathlineto{\pgfqpoint{4.954220in}{2.652062in}}%
\pgfpathlineto{\pgfqpoint{4.957392in}{2.651827in}}%
\pgfpathlineto{\pgfqpoint{4.960564in}{2.651692in}}%
\pgfpathlineto{\pgfqpoint{4.963736in}{2.651600in}}%
\pgfpathlineto{\pgfqpoint{4.966908in}{2.651713in}}%
\pgfpathlineto{\pgfqpoint{4.970080in}{2.651878in}}%
\pgfpathlineto{\pgfqpoint{4.973252in}{2.651483in}}%
\pgfpathlineto{\pgfqpoint{4.976424in}{2.651985in}}%
\pgfpathlineto{\pgfqpoint{4.979596in}{2.652245in}}%
\pgfpathlineto{\pgfqpoint{4.982768in}{2.652235in}}%
\pgfpathlineto{\pgfqpoint{4.985940in}{2.652375in}}%
\pgfpathlineto{\pgfqpoint{4.989112in}{2.652566in}}%
\pgfpathlineto{\pgfqpoint{4.992284in}{2.652125in}}%
\pgfpathlineto{\pgfqpoint{4.995456in}{2.652299in}}%
\pgfpathlineto{\pgfqpoint{4.998628in}{2.652804in}}%
\pgfpathlineto{\pgfqpoint{5.001800in}{2.653304in}}%
\pgfpathlineto{\pgfqpoint{5.004972in}{2.653358in}}%
\pgfpathlineto{\pgfqpoint{5.008145in}{2.653018in}}%
\pgfpathlineto{\pgfqpoint{5.011317in}{2.652814in}}%
\pgfpathlineto{\pgfqpoint{5.014489in}{2.652588in}}%
\pgfpathlineto{\pgfqpoint{5.017661in}{2.653144in}}%
\pgfpathlineto{\pgfqpoint{5.020833in}{2.653290in}}%
\pgfpathlineto{\pgfqpoint{5.024005in}{2.653205in}}%
\pgfpathlineto{\pgfqpoint{5.027177in}{2.652876in}}%
\pgfpathlineto{\pgfqpoint{5.030349in}{2.652936in}}%
\pgfpathlineto{\pgfqpoint{5.033521in}{2.652750in}}%
\pgfpathlineto{\pgfqpoint{5.036693in}{2.652838in}}%
\pgfpathlineto{\pgfqpoint{5.039865in}{2.653361in}}%
\pgfpathlineto{\pgfqpoint{5.043037in}{2.653253in}}%
\pgfpathlineto{\pgfqpoint{5.046209in}{2.653666in}}%
\pgfpathlineto{\pgfqpoint{5.049381in}{2.653170in}}%
\pgfpathlineto{\pgfqpoint{5.052553in}{2.653128in}}%
\pgfpathlineto{\pgfqpoint{5.055725in}{2.653507in}}%
\pgfpathlineto{\pgfqpoint{5.058897in}{2.653607in}}%
\pgfpathlineto{\pgfqpoint{5.062069in}{2.653690in}}%
\pgfpathlineto{\pgfqpoint{5.065241in}{2.654076in}}%
\pgfpathlineto{\pgfqpoint{5.068413in}{2.654330in}}%
\pgfpathlineto{\pgfqpoint{5.071585in}{2.654390in}}%
\pgfpathlineto{\pgfqpoint{5.074757in}{2.654104in}}%
\pgfpathlineto{\pgfqpoint{5.077929in}{2.654004in}}%
\pgfpathlineto{\pgfqpoint{5.081101in}{2.654292in}}%
\pgfpathlineto{\pgfqpoint{5.084273in}{2.653948in}}%
\pgfpathlineto{\pgfqpoint{5.087446in}{2.654035in}}%
\pgfpathlineto{\pgfqpoint{5.090618in}{2.653792in}}%
\pgfpathlineto{\pgfqpoint{5.093790in}{2.654082in}}%
\pgfpathlineto{\pgfqpoint{5.096962in}{2.654120in}}%
\pgfpathlineto{\pgfqpoint{5.100134in}{2.654092in}}%
\pgfpathlineto{\pgfqpoint{5.103306in}{2.654302in}}%
\pgfpathlineto{\pgfqpoint{5.106478in}{2.654590in}}%
\pgfpathlineto{\pgfqpoint{5.109650in}{2.654928in}}%
\pgfpathlineto{\pgfqpoint{5.112822in}{2.654961in}}%
\pgfpathlineto{\pgfqpoint{5.115994in}{2.654828in}}%
\pgfpathlineto{\pgfqpoint{5.119166in}{2.654633in}}%
\pgfpathlineto{\pgfqpoint{5.122338in}{2.654990in}}%
\pgfpathlineto{\pgfqpoint{5.125510in}{2.655245in}}%
\pgfpathlineto{\pgfqpoint{5.128682in}{2.655483in}}%
\pgfpathlineto{\pgfqpoint{5.131854in}{2.655855in}}%
\pgfpathlineto{\pgfqpoint{5.135026in}{2.655889in}}%
\pgfpathlineto{\pgfqpoint{5.138198in}{2.656040in}}%
\pgfpathlineto{\pgfqpoint{5.141370in}{2.656100in}}%
\pgfpathlineto{\pgfqpoint{5.144542in}{2.656353in}}%
\pgfpathlineto{\pgfqpoint{5.147714in}{2.656398in}}%
\pgfpathlineto{\pgfqpoint{5.150886in}{2.656665in}}%
\pgfpathlineto{\pgfqpoint{5.154058in}{2.657087in}}%
\pgfpathlineto{\pgfqpoint{5.157230in}{2.657306in}}%
\pgfpathlineto{\pgfqpoint{5.160402in}{2.657269in}}%
\pgfpathlineto{\pgfqpoint{5.163575in}{2.656954in}}%
\pgfpathlineto{\pgfqpoint{5.166747in}{2.656697in}}%
\pgfpathlineto{\pgfqpoint{5.169919in}{2.656367in}}%
\pgfpathlineto{\pgfqpoint{5.173091in}{2.656641in}}%
\pgfpathlineto{\pgfqpoint{5.176263in}{2.656484in}}%
\pgfpathlineto{\pgfqpoint{5.179435in}{2.656655in}}%
\pgfpathlineto{\pgfqpoint{5.182607in}{2.657080in}}%
\pgfpathlineto{\pgfqpoint{5.185779in}{2.657557in}}%
\pgfpathlineto{\pgfqpoint{5.188951in}{2.657276in}}%
\pgfpathlineto{\pgfqpoint{5.192123in}{2.657318in}}%
\pgfpathlineto{\pgfqpoint{5.195295in}{2.657177in}}%
\pgfpathlineto{\pgfqpoint{5.198467in}{2.656613in}}%
\pgfpathlineto{\pgfqpoint{5.201639in}{2.656532in}}%
\pgfpathlineto{\pgfqpoint{5.204811in}{2.656343in}}%
\pgfpathlineto{\pgfqpoint{5.207983in}{2.656151in}}%
\pgfpathlineto{\pgfqpoint{5.211155in}{2.655503in}}%
\pgfpathlineto{\pgfqpoint{5.214327in}{2.655471in}}%
\pgfpathlineto{\pgfqpoint{5.217499in}{2.655701in}}%
\pgfpathlineto{\pgfqpoint{5.220671in}{2.656000in}}%
\pgfpathlineto{\pgfqpoint{5.223843in}{2.655926in}}%
\pgfpathlineto{\pgfqpoint{5.227015in}{2.656363in}}%
\pgfpathlineto{\pgfqpoint{5.230187in}{2.656960in}}%
\pgfpathlineto{\pgfqpoint{5.233359in}{2.657332in}}%
\pgfpathlineto{\pgfqpoint{5.236531in}{2.657372in}}%
\pgfpathlineto{\pgfqpoint{5.239703in}{2.657192in}}%
\pgfpathlineto{\pgfqpoint{5.242876in}{2.657210in}}%
\pgfpathlineto{\pgfqpoint{5.246048in}{2.656845in}}%
\pgfpathlineto{\pgfqpoint{5.249220in}{2.656605in}}%
\pgfpathlineto{\pgfqpoint{5.252392in}{2.657182in}}%
\pgfpathlineto{\pgfqpoint{5.255564in}{2.657458in}}%
\pgfpathlineto{\pgfqpoint{5.258736in}{2.656948in}}%
\pgfpathlineto{\pgfqpoint{5.261908in}{2.657238in}}%
\pgfpathlineto{\pgfqpoint{5.265080in}{2.657425in}}%
\pgfpathlineto{\pgfqpoint{5.268252in}{2.657058in}}%
\pgfpathlineto{\pgfqpoint{5.271424in}{2.656289in}}%
\pgfpathlineto{\pgfqpoint{5.274596in}{2.655856in}}%
\pgfpathlineto{\pgfqpoint{5.277768in}{2.655005in}}%
\pgfpathlineto{\pgfqpoint{5.280940in}{2.654680in}}%
\pgfpathlineto{\pgfqpoint{5.284112in}{2.654742in}}%
\pgfpathlineto{\pgfqpoint{5.287284in}{2.655076in}}%
\pgfpathlineto{\pgfqpoint{5.290456in}{2.655147in}}%
\pgfpathlineto{\pgfqpoint{5.293628in}{2.654548in}}%
\pgfpathlineto{\pgfqpoint{5.296800in}{2.654031in}}%
\pgfpathlineto{\pgfqpoint{5.299972in}{2.654260in}}%
\pgfpathlineto{\pgfqpoint{5.303144in}{2.654365in}}%
\pgfpathlineto{\pgfqpoint{5.306316in}{2.654530in}}%
\pgfpathlineto{\pgfqpoint{5.309488in}{2.654917in}}%
\pgfpathlineto{\pgfqpoint{5.312660in}{2.655309in}}%
\pgfpathlineto{\pgfqpoint{5.315832in}{2.655641in}}%
\pgfpathlineto{\pgfqpoint{5.319004in}{2.655539in}}%
\pgfpathlineto{\pgfqpoint{5.322177in}{2.655519in}}%
\pgfpathlineto{\pgfqpoint{5.325349in}{2.655544in}}%
\pgfpathlineto{\pgfqpoint{5.328521in}{2.655607in}}%
\pgfpathlineto{\pgfqpoint{5.331693in}{2.654823in}}%
\pgfpathlineto{\pgfqpoint{5.334865in}{2.654496in}}%
\pgfpathlineto{\pgfqpoint{5.338037in}{2.654362in}}%
\pgfpathlineto{\pgfqpoint{5.341209in}{2.654598in}}%
\pgfpathlineto{\pgfqpoint{5.344381in}{2.654408in}}%
\pgfpathlineto{\pgfqpoint{5.347553in}{2.654258in}}%
\pgfpathlineto{\pgfqpoint{5.350725in}{2.654229in}}%
\pgfpathlineto{\pgfqpoint{5.353897in}{2.653911in}}%
\pgfpathlineto{\pgfqpoint{5.357069in}{2.653692in}}%
\pgfpathlineto{\pgfqpoint{5.360241in}{2.654035in}}%
\pgfpathlineto{\pgfqpoint{5.363413in}{2.654179in}}%
\pgfpathlineto{\pgfqpoint{5.366585in}{2.654636in}}%
\pgfpathlineto{\pgfqpoint{5.369757in}{2.654803in}}%
\pgfpathlineto{\pgfqpoint{5.372929in}{2.654841in}}%
\pgfpathlineto{\pgfqpoint{5.376101in}{2.655275in}}%
\pgfpathlineto{\pgfqpoint{5.379273in}{2.655505in}}%
\pgfpathlineto{\pgfqpoint{5.382445in}{2.655266in}}%
\pgfpathlineto{\pgfqpoint{5.385617in}{2.655324in}}%
\pgfpathlineto{\pgfqpoint{5.388789in}{2.655086in}}%
\pgfpathlineto{\pgfqpoint{5.391961in}{2.655723in}}%
\pgfpathlineto{\pgfqpoint{5.395133in}{2.655708in}}%
\pgfpathlineto{\pgfqpoint{5.398306in}{2.655539in}}%
\pgfpathlineto{\pgfqpoint{5.401478in}{2.655531in}}%
\pgfpathlineto{\pgfqpoint{5.404650in}{2.655428in}}%
\pgfpathlineto{\pgfqpoint{5.407822in}{2.655287in}}%
\pgfpathlineto{\pgfqpoint{5.410994in}{2.655184in}}%
\pgfpathlineto{\pgfqpoint{5.414166in}{2.654887in}}%
\pgfpathlineto{\pgfqpoint{5.417338in}{2.654669in}}%
\pgfpathlineto{\pgfqpoint{5.420510in}{2.655349in}}%
\pgfpathlineto{\pgfqpoint{5.423682in}{2.655364in}}%
\pgfpathlineto{\pgfqpoint{5.426854in}{2.655028in}}%
\pgfpathlineto{\pgfqpoint{5.430026in}{2.655134in}}%
\pgfpathlineto{\pgfqpoint{5.433198in}{2.655415in}}%
\pgfpathlineto{\pgfqpoint{5.436370in}{2.655415in}}%
\pgfpathlineto{\pgfqpoint{5.439542in}{2.655679in}}%
\pgfpathlineto{\pgfqpoint{5.442714in}{2.655558in}}%
\pgfpathlineto{\pgfqpoint{5.445886in}{2.655766in}}%
\pgfpathlineto{\pgfqpoint{5.449058in}{2.655792in}}%
\pgfpathlineto{\pgfqpoint{5.452230in}{2.656221in}}%
\pgfpathlineto{\pgfqpoint{5.455402in}{2.656544in}}%
\pgfpathlineto{\pgfqpoint{5.458574in}{2.656329in}}%
\pgfpathlineto{\pgfqpoint{5.461746in}{2.656045in}}%
\pgfpathlineto{\pgfqpoint{5.464918in}{2.655756in}}%
\pgfpathlineto{\pgfqpoint{5.468090in}{2.655842in}}%
\pgfpathlineto{\pgfqpoint{5.471262in}{2.656100in}}%
\pgfpathlineto{\pgfqpoint{5.474434in}{2.656142in}}%
\pgfpathlineto{\pgfqpoint{5.477607in}{2.656043in}}%
\pgfpathlineto{\pgfqpoint{5.480779in}{2.656252in}}%
\pgfpathlineto{\pgfqpoint{5.483951in}{2.656765in}}%
\pgfpathlineto{\pgfqpoint{5.487123in}{2.657289in}}%
\pgfpathlineto{\pgfqpoint{5.490295in}{2.657176in}}%
\pgfpathlineto{\pgfqpoint{5.493467in}{2.657246in}}%
\pgfpathlineto{\pgfqpoint{5.496639in}{2.656938in}}%
\pgfpathlineto{\pgfqpoint{5.499811in}{2.656872in}}%
\pgfpathlineto{\pgfqpoint{5.502983in}{2.656855in}}%
\pgfpathlineto{\pgfqpoint{5.506155in}{2.657574in}}%
\pgfpathlineto{\pgfqpoint{5.509327in}{2.657973in}}%
\pgfpathlineto{\pgfqpoint{5.512499in}{2.658154in}}%
\pgfpathlineto{\pgfqpoint{5.515671in}{2.658010in}}%
\pgfpathlineto{\pgfqpoint{5.518843in}{2.657178in}}%
\pgfpathlineto{\pgfqpoint{5.522015in}{2.657029in}}%
\pgfpathlineto{\pgfqpoint{5.525187in}{2.657379in}}%
\pgfpathlineto{\pgfqpoint{5.528359in}{2.657428in}}%
\pgfpathlineto{\pgfqpoint{5.531531in}{2.657533in}}%
\pgfpathlineto{\pgfqpoint{5.534703in}{2.657362in}}%
\pgfpathlineto{\pgfqpoint{5.537875in}{2.657424in}}%
\pgfpathlineto{\pgfqpoint{5.541047in}{2.658270in}}%
\pgfpathlineto{\pgfqpoint{5.544219in}{2.658686in}}%
\pgfpathlineto{\pgfqpoint{5.547391in}{2.658550in}}%
\pgfpathlineto{\pgfqpoint{5.550563in}{2.658487in}}%
\pgfpathlineto{\pgfqpoint{5.553735in}{2.658446in}}%
\pgfpathlineto{\pgfqpoint{5.556908in}{2.658578in}}%
\pgfpathlineto{\pgfqpoint{5.560080in}{2.658610in}}%
\pgfpathlineto{\pgfqpoint{5.563252in}{2.658679in}}%
\pgfpathlineto{\pgfqpoint{5.566424in}{2.658854in}}%
\pgfpathlineto{\pgfqpoint{5.569596in}{2.659559in}}%
\pgfpathlineto{\pgfqpoint{5.572768in}{2.659996in}}%
\pgfpathlineto{\pgfqpoint{5.575940in}{2.660238in}}%
\pgfpathlineto{\pgfqpoint{5.579112in}{2.659930in}}%
\pgfpathlineto{\pgfqpoint{5.582284in}{2.659712in}}%
\pgfpathlineto{\pgfqpoint{5.585456in}{2.659783in}}%
\pgfpathlineto{\pgfqpoint{5.588628in}{2.659822in}}%
\pgfpathlineto{\pgfqpoint{5.591800in}{2.660408in}}%
\pgfpathlineto{\pgfqpoint{5.594972in}{2.659694in}}%
\pgfpathlineto{\pgfqpoint{5.598144in}{2.659721in}}%
\pgfpathlineto{\pgfqpoint{5.601316in}{2.659748in}}%
\pgfpathlineto{\pgfqpoint{5.604488in}{2.659607in}}%
\pgfpathlineto{\pgfqpoint{5.607660in}{2.659888in}}%
\pgfpathlineto{\pgfqpoint{5.610832in}{2.660140in}}%
\pgfpathlineto{\pgfqpoint{5.614004in}{2.660610in}}%
\pgfpathlineto{\pgfqpoint{5.617176in}{2.660705in}}%
\pgfpathlineto{\pgfqpoint{5.620348in}{2.660594in}}%
\pgfpathlineto{\pgfqpoint{5.623520in}{2.660275in}}%
\pgfpathlineto{\pgfqpoint{5.626692in}{2.660004in}}%
\pgfpathlineto{\pgfqpoint{5.629864in}{2.659720in}}%
\pgfpathlineto{\pgfqpoint{5.633037in}{2.659616in}}%
\pgfpathlineto{\pgfqpoint{5.636209in}{2.659955in}}%
\pgfpathlineto{\pgfqpoint{5.639381in}{2.659417in}}%
\pgfpathlineto{\pgfqpoint{5.642553in}{2.659822in}}%
\pgfpathlineto{\pgfqpoint{5.645725in}{2.660026in}}%
\pgfpathlineto{\pgfqpoint{5.648897in}{2.660103in}}%
\pgfpathlineto{\pgfqpoint{5.652069in}{2.659799in}}%
\pgfpathlineto{\pgfqpoint{5.655241in}{2.659609in}}%
\pgfpathlineto{\pgfqpoint{5.658413in}{2.660218in}}%
\pgfpathlineto{\pgfqpoint{5.661585in}{2.660348in}}%
\pgfpathlineto{\pgfqpoint{5.664757in}{2.660860in}}%
\pgfpathlineto{\pgfqpoint{5.667929in}{2.660659in}}%
\pgfpathlineto{\pgfqpoint{5.671101in}{2.661144in}}%
\pgfpathlineto{\pgfqpoint{5.674273in}{2.660842in}}%
\pgfpathlineto{\pgfqpoint{5.677445in}{2.660702in}}%
\pgfpathlineto{\pgfqpoint{5.680617in}{2.660510in}}%
\pgfpathlineto{\pgfqpoint{5.683789in}{2.660362in}}%
\pgfpathlineto{\pgfqpoint{5.686961in}{2.660443in}}%
\pgfpathlineto{\pgfqpoint{5.690133in}{2.660354in}}%
\pgfpathlineto{\pgfqpoint{5.693305in}{2.660338in}}%
\pgfpathlineto{\pgfqpoint{5.696477in}{2.660493in}}%
\pgfpathlineto{\pgfqpoint{5.699649in}{2.660180in}}%
\pgfpathlineto{\pgfqpoint{5.702821in}{2.659602in}}%
\pgfpathlineto{\pgfqpoint{5.705993in}{2.660064in}}%
\pgfpathlineto{\pgfqpoint{5.709165in}{2.660288in}}%
\pgfpathlineto{\pgfqpoint{5.712338in}{2.660346in}}%
\pgfpathlineto{\pgfqpoint{5.715510in}{2.660737in}}%
\pgfpathlineto{\pgfqpoint{5.718682in}{2.661440in}}%
\pgfpathlineto{\pgfqpoint{5.721854in}{2.661737in}}%
\pgfpathlineto{\pgfqpoint{5.725026in}{2.661634in}}%
\pgfpathlineto{\pgfqpoint{5.728198in}{2.661403in}}%
\pgfpathlineto{\pgfqpoint{5.731370in}{2.661630in}}%
\pgfpathlineto{\pgfqpoint{5.734542in}{2.661814in}}%
\pgfpathlineto{\pgfqpoint{5.737714in}{2.661656in}}%
\pgfpathlineto{\pgfqpoint{5.740886in}{2.661966in}}%
\pgfpathlineto{\pgfqpoint{5.744058in}{2.661870in}}%
\pgfpathlineto{\pgfqpoint{5.747230in}{2.661448in}}%
\pgfpathlineto{\pgfqpoint{5.750402in}{2.661470in}}%
\pgfpathlineto{\pgfqpoint{5.753574in}{2.661335in}}%
\pgfpathlineto{\pgfqpoint{5.756746in}{2.661415in}}%
\pgfpathlineto{\pgfqpoint{5.759918in}{2.661788in}}%
\pgfpathlineto{\pgfqpoint{5.763090in}{2.661924in}}%
\pgfpathlineto{\pgfqpoint{5.766262in}{2.661727in}}%
\pgfpathlineto{\pgfqpoint{5.769434in}{2.662025in}}%
\pgfpathlineto{\pgfqpoint{5.772606in}{2.661844in}}%
\pgfpathlineto{\pgfqpoint{5.775778in}{2.661565in}}%
\pgfpathlineto{\pgfqpoint{5.778950in}{2.661419in}}%
\pgfpathlineto{\pgfqpoint{5.782122in}{2.660750in}}%
\pgfpathlineto{\pgfqpoint{5.785294in}{2.660235in}}%
\pgfpathlineto{\pgfqpoint{5.788466in}{2.660170in}}%
\pgfpathlineto{\pgfqpoint{5.791639in}{2.660310in}}%
\pgfpathlineto{\pgfqpoint{5.794811in}{2.660609in}}%
\pgfpathlineto{\pgfqpoint{5.797983in}{2.660722in}}%
\pgfpathlineto{\pgfqpoint{5.801155in}{2.660831in}}%
\pgfpathlineto{\pgfqpoint{5.804327in}{2.660708in}}%
\pgfpathlineto{\pgfqpoint{5.807499in}{2.660646in}}%
\pgfpathlineto{\pgfqpoint{5.810671in}{2.660640in}}%
\pgfpathlineto{\pgfqpoint{5.813843in}{2.661426in}}%
\pgfpathlineto{\pgfqpoint{5.817015in}{2.661723in}}%
\pgfpathlineto{\pgfqpoint{5.820187in}{2.661845in}}%
\pgfpathlineto{\pgfqpoint{5.823359in}{2.661897in}}%
\pgfpathlineto{\pgfqpoint{5.826531in}{2.661978in}}%
\pgfpathlineto{\pgfqpoint{5.829703in}{2.661588in}}%
\pgfpathlineto{\pgfqpoint{5.832875in}{2.661302in}}%
\pgfpathlineto{\pgfqpoint{5.836047in}{2.660744in}}%
\pgfpathlineto{\pgfqpoint{5.839219in}{2.660971in}}%
\pgfpathlineto{\pgfqpoint{5.842391in}{2.660751in}}%
\pgfpathlineto{\pgfqpoint{5.845563in}{2.660607in}}%
\pgfpathlineto{\pgfqpoint{5.848735in}{2.660531in}}%
\pgfpathlineto{\pgfqpoint{5.851907in}{2.660715in}}%
\pgfpathlineto{\pgfqpoint{5.855079in}{2.660673in}}%
\pgfpathlineto{\pgfqpoint{5.858251in}{2.660748in}}%
\pgfpathlineto{\pgfqpoint{5.861423in}{2.660777in}}%
\pgfpathlineto{\pgfqpoint{5.864595in}{2.661570in}}%
\pgfpathlineto{\pgfqpoint{5.867768in}{2.661965in}}%
\pgfpathlineto{\pgfqpoint{5.870940in}{2.662084in}}%
\pgfpathlineto{\pgfqpoint{5.874112in}{2.662547in}}%
\pgfpathlineto{\pgfqpoint{5.877284in}{2.662990in}}%
\pgfpathlineto{\pgfqpoint{5.880456in}{2.663678in}}%
\pgfpathlineto{\pgfqpoint{5.883628in}{2.663827in}}%
\pgfpathlineto{\pgfqpoint{5.886800in}{2.663496in}}%
\pgfpathlineto{\pgfqpoint{5.889972in}{2.663187in}}%
\pgfpathlineto{\pgfqpoint{5.893144in}{2.663124in}}%
\pgfpathlineto{\pgfqpoint{5.896316in}{2.662898in}}%
\pgfpathlineto{\pgfqpoint{5.899488in}{2.662824in}}%
\pgfpathlineto{\pgfqpoint{5.902660in}{2.662811in}}%
\pgfpathlineto{\pgfqpoint{5.905832in}{2.662576in}}%
\pgfpathlineto{\pgfqpoint{5.909004in}{2.662224in}}%
\pgfpathlineto{\pgfqpoint{5.912176in}{2.662302in}}%
\pgfpathlineto{\pgfqpoint{5.915348in}{2.662377in}}%
\pgfpathlineto{\pgfqpoint{5.918520in}{2.662079in}}%
\pgfpathlineto{\pgfqpoint{5.921692in}{2.662426in}}%
\pgfpathlineto{\pgfqpoint{5.924864in}{2.662454in}}%
\pgfpathlineto{\pgfqpoint{5.928036in}{2.662515in}}%
\pgfpathlineto{\pgfqpoint{5.931208in}{2.662333in}}%
\pgfpathlineto{\pgfqpoint{5.934380in}{2.662431in}}%
\pgfpathlineto{\pgfqpoint{5.937552in}{2.663015in}}%
\pgfpathlineto{\pgfqpoint{5.940724in}{2.662867in}}%
\pgfpathlineto{\pgfqpoint{5.943896in}{2.663228in}}%
\pgfpathlineto{\pgfqpoint{5.947069in}{2.662952in}}%
\pgfpathlineto{\pgfqpoint{5.950241in}{2.662733in}}%
\pgfpathlineto{\pgfqpoint{5.953413in}{2.662497in}}%
\pgfpathlineto{\pgfqpoint{5.956585in}{2.662233in}}%
\pgfpathlineto{\pgfqpoint{5.959757in}{2.662115in}}%
\pgfpathlineto{\pgfqpoint{5.962929in}{2.662182in}}%
\pgfpathlineto{\pgfqpoint{5.966101in}{2.661853in}}%
\pgfpathlineto{\pgfqpoint{5.969273in}{2.662242in}}%
\pgfpathlineto{\pgfqpoint{5.972445in}{2.662247in}}%
\pgfpathlineto{\pgfqpoint{5.975617in}{2.662558in}}%
\pgfpathlineto{\pgfqpoint{5.978789in}{2.661870in}}%
\pgfpathlineto{\pgfqpoint{5.981961in}{2.661916in}}%
\pgfpathlineto{\pgfqpoint{5.985133in}{2.661900in}}%
\pgfpathlineto{\pgfqpoint{5.988305in}{2.662179in}}%
\pgfpathlineto{\pgfqpoint{5.991477in}{2.662673in}}%
\pgfpathlineto{\pgfqpoint{5.994649in}{2.662500in}}%
\pgfpathlineto{\pgfqpoint{5.997821in}{2.661874in}}%
\pgfpathlineto{\pgfqpoint{6.000993in}{2.662049in}}%
\pgfpathlineto{\pgfqpoint{6.004165in}{2.662216in}}%
\pgfpathlineto{\pgfqpoint{6.007337in}{2.662183in}}%
\pgfpathlineto{\pgfqpoint{6.010509in}{2.662255in}}%
\pgfpathlineto{\pgfqpoint{6.013681in}{2.662461in}}%
\pgfpathlineto{\pgfqpoint{6.016853in}{2.663141in}}%
\pgfpathlineto{\pgfqpoint{6.020025in}{2.662600in}}%
\pgfpathlineto{\pgfqpoint{6.023197in}{2.663094in}}%
\pgfpathlineto{\pgfqpoint{6.026370in}{2.663399in}}%
\pgfpathlineto{\pgfqpoint{6.029542in}{2.663318in}}%
\pgfpathlineto{\pgfqpoint{6.032714in}{2.663020in}}%
\pgfpathlineto{\pgfqpoint{6.035886in}{2.662562in}}%
\pgfpathlineto{\pgfqpoint{6.039058in}{2.662752in}}%
\pgfpathlineto{\pgfqpoint{6.042230in}{2.662514in}}%
\pgfpathlineto{\pgfqpoint{6.045402in}{2.662708in}}%
\pgfpathlineto{\pgfqpoint{6.048574in}{2.662100in}}%
\pgfpathlineto{\pgfqpoint{6.051746in}{2.662325in}}%
\pgfpathlineto{\pgfqpoint{6.054918in}{2.662484in}}%
\pgfpathlineto{\pgfqpoint{6.058090in}{2.662127in}}%
\pgfpathlineto{\pgfqpoint{6.061262in}{2.661786in}}%
\pgfpathlineto{\pgfqpoint{6.064434in}{2.661806in}}%
\pgfpathlineto{\pgfqpoint{6.067606in}{2.661601in}}%
\pgfpathlineto{\pgfqpoint{6.070778in}{2.661538in}}%
\pgfpathlineto{\pgfqpoint{6.073950in}{2.661580in}}%
\pgfpathlineto{\pgfqpoint{6.077122in}{2.661728in}}%
\pgfpathlineto{\pgfqpoint{6.080294in}{2.661820in}}%
\pgfpathlineto{\pgfqpoint{6.083466in}{2.661876in}}%
\pgfpathlineto{\pgfqpoint{6.086638in}{2.662064in}}%
\pgfpathlineto{\pgfqpoint{6.089810in}{2.662167in}}%
\pgfpathlineto{\pgfqpoint{6.092982in}{2.662008in}}%
\pgfpathlineto{\pgfqpoint{6.096154in}{2.661901in}}%
\pgfpathlineto{\pgfqpoint{6.099326in}{2.661921in}}%
\pgfpathlineto{\pgfqpoint{6.102499in}{2.662045in}}%
\pgfpathlineto{\pgfqpoint{6.105671in}{2.662031in}}%
\pgfpathlineto{\pgfqpoint{6.108843in}{2.662352in}}%
\pgfpathlineto{\pgfqpoint{6.112015in}{2.662567in}}%
\pgfpathlineto{\pgfqpoint{6.115187in}{2.662605in}}%
\pgfpathlineto{\pgfqpoint{6.118359in}{2.662581in}}%
\pgfpathlineto{\pgfqpoint{6.121531in}{2.661871in}}%
\pgfpathlineto{\pgfqpoint{6.124703in}{2.661320in}}%
\pgfpathlineto{\pgfqpoint{6.127875in}{2.661412in}}%
\pgfpathlineto{\pgfqpoint{6.131047in}{2.661672in}}%
\pgfpathlineto{\pgfqpoint{6.134219in}{2.661645in}}%
\pgfpathlineto{\pgfqpoint{6.137391in}{2.661517in}}%
\pgfpathlineto{\pgfqpoint{6.140563in}{2.661007in}}%
\pgfpathlineto{\pgfqpoint{6.143735in}{2.660892in}}%
\pgfpathlineto{\pgfqpoint{6.146907in}{2.660301in}}%
\pgfpathlineto{\pgfqpoint{6.150079in}{2.660761in}}%
\pgfpathlineto{\pgfqpoint{6.153251in}{2.660733in}}%
\pgfpathlineto{\pgfqpoint{6.156423in}{2.660992in}}%
\pgfpathlineto{\pgfqpoint{6.159595in}{2.661060in}}%
\pgfpathlineto{\pgfqpoint{6.162767in}{2.661209in}}%
\pgfpathlineto{\pgfqpoint{6.165939in}{2.661662in}}%
\pgfpathlineto{\pgfqpoint{6.169111in}{2.661365in}}%
\pgfpathlineto{\pgfqpoint{6.172283in}{2.661774in}}%
\pgfpathlineto{\pgfqpoint{6.175455in}{2.662320in}}%
\pgfpathlineto{\pgfqpoint{6.178627in}{2.662321in}}%
\pgfpathlineto{\pgfqpoint{6.181800in}{2.662315in}}%
\pgfpathlineto{\pgfqpoint{6.184972in}{2.662196in}}%
\pgfpathlineto{\pgfqpoint{6.188144in}{2.662374in}}%
\pgfpathlineto{\pgfqpoint{6.191316in}{2.662724in}}%
\pgfpathlineto{\pgfqpoint{6.194488in}{2.662543in}}%
\pgfpathlineto{\pgfqpoint{6.197660in}{2.662575in}}%
\pgfpathlineto{\pgfqpoint{6.200832in}{2.662642in}}%
\pgfpathlineto{\pgfqpoint{6.204004in}{2.662883in}}%
\pgfpathlineto{\pgfqpoint{6.207176in}{2.663007in}}%
\pgfpathlineto{\pgfqpoint{6.210348in}{2.663105in}}%
\pgfpathlineto{\pgfqpoint{6.213520in}{2.662514in}}%
\pgfpathlineto{\pgfqpoint{6.216692in}{2.662223in}}%
\pgfpathlineto{\pgfqpoint{6.219864in}{2.662500in}}%
\pgfpathlineto{\pgfqpoint{6.223036in}{2.662638in}}%
\pgfpathlineto{\pgfqpoint{6.226208in}{2.662700in}}%
\pgfpathlineto{\pgfqpoint{6.229380in}{2.662608in}}%
\pgfpathlineto{\pgfqpoint{6.232552in}{2.662640in}}%
\pgfpathlineto{\pgfqpoint{6.235724in}{2.662256in}}%
\pgfpathlineto{\pgfqpoint{6.238896in}{2.662065in}}%
\pgfpathlineto{\pgfqpoint{6.242068in}{2.661600in}}%
\pgfpathlineto{\pgfqpoint{6.245240in}{2.661816in}}%
\pgfpathlineto{\pgfqpoint{6.248412in}{2.661918in}}%
\pgfpathlineto{\pgfqpoint{6.251584in}{2.661439in}}%
\pgfpathlineto{\pgfqpoint{6.254756in}{2.661310in}}%
\pgfpathlineto{\pgfqpoint{6.257928in}{2.661374in}}%
\pgfpathlineto{\pgfqpoint{6.261101in}{2.661365in}}%
\pgfpathlineto{\pgfqpoint{6.264273in}{2.660944in}}%
\pgfpathlineto{\pgfqpoint{6.267445in}{2.661086in}}%
\pgfpathlineto{\pgfqpoint{6.270617in}{2.660879in}}%
\pgfpathlineto{\pgfqpoint{6.273789in}{2.660687in}}%
\pgfpathlineto{\pgfqpoint{6.276961in}{2.660715in}}%
\pgfpathlineto{\pgfqpoint{6.280133in}{2.660961in}}%
\pgfpathlineto{\pgfqpoint{6.283305in}{2.660801in}}%
\pgfpathlineto{\pgfqpoint{6.286477in}{2.660915in}}%
\pgfpathlineto{\pgfqpoint{6.289649in}{2.661110in}}%
\pgfpathlineto{\pgfqpoint{6.292821in}{2.661141in}}%
\pgfpathlineto{\pgfqpoint{6.295993in}{2.660702in}}%
\pgfpathlineto{\pgfqpoint{6.299165in}{2.660491in}}%
\pgfpathlineto{\pgfqpoint{6.302337in}{2.660876in}}%
\pgfpathlineto{\pgfqpoint{6.305509in}{2.660836in}}%
\pgfpathlineto{\pgfqpoint{6.308681in}{2.660568in}}%
\pgfpathlineto{\pgfqpoint{6.311853in}{2.660296in}}%
\pgfpathlineto{\pgfqpoint{6.315025in}{2.660573in}}%
\pgfpathlineto{\pgfqpoint{6.318197in}{2.660395in}}%
\pgfpathlineto{\pgfqpoint{6.321369in}{2.660486in}}%
\pgfpathlineto{\pgfqpoint{6.324541in}{2.660619in}}%
\pgfpathlineto{\pgfqpoint{6.327713in}{2.660915in}}%
\pgfpathlineto{\pgfqpoint{6.330885in}{2.660809in}}%
\pgfpathlineto{\pgfqpoint{6.334057in}{2.660928in}}%
\pgfpathlineto{\pgfqpoint{6.337230in}{2.660839in}}%
\pgfpathlineto{\pgfqpoint{6.340402in}{2.660650in}}%
\pgfpathlineto{\pgfqpoint{6.343574in}{2.660468in}}%
\pgfpathlineto{\pgfqpoint{6.346746in}{2.660495in}}%
\pgfpathlineto{\pgfqpoint{6.349918in}{2.660508in}}%
\pgfpathlineto{\pgfqpoint{6.353090in}{2.660842in}}%
\pgfpathlineto{\pgfqpoint{6.356262in}{2.661130in}}%
\pgfpathlineto{\pgfqpoint{6.359434in}{2.660684in}}%
\pgfpathlineto{\pgfqpoint{6.362606in}{2.660766in}}%
\pgfpathlineto{\pgfqpoint{6.365778in}{2.660710in}}%
\pgfpathlineto{\pgfqpoint{6.368950in}{2.661049in}}%
\pgfpathlineto{\pgfqpoint{6.372122in}{2.660735in}}%
\pgfpathlineto{\pgfqpoint{6.375294in}{2.660593in}}%
\pgfpathlineto{\pgfqpoint{6.378466in}{2.660541in}}%
\pgfpathlineto{\pgfqpoint{6.381638in}{2.660684in}}%
\pgfpathlineto{\pgfqpoint{6.384810in}{2.660944in}}%
\pgfpathlineto{\pgfqpoint{6.387982in}{2.660474in}}%
\pgfpathlineto{\pgfqpoint{6.391154in}{2.660407in}}%
\pgfpathlineto{\pgfqpoint{6.394326in}{2.660399in}}%
\pgfpathlineto{\pgfqpoint{6.397498in}{2.660086in}}%
\pgfpathlineto{\pgfqpoint{6.400670in}{2.659854in}}%
\pgfpathlineto{\pgfqpoint{6.403842in}{2.659839in}}%
\pgfpathlineto{\pgfqpoint{6.407014in}{2.660120in}}%
\pgfpathlineto{\pgfqpoint{6.410186in}{2.659893in}}%
\pgfpathlineto{\pgfqpoint{6.413358in}{2.659558in}}%
\pgfpathlineto{\pgfqpoint{6.416531in}{2.659665in}}%
\pgfpathlineto{\pgfqpoint{6.419703in}{2.659730in}}%
\pgfpathlineto{\pgfqpoint{6.422875in}{2.659747in}}%
\pgfpathlineto{\pgfqpoint{6.426047in}{2.659928in}}%
\pgfpathlineto{\pgfqpoint{6.429219in}{2.660266in}}%
\pgfpathlineto{\pgfqpoint{6.432391in}{2.660167in}}%
\pgfpathlineto{\pgfqpoint{6.435563in}{2.659833in}}%
\pgfpathlineto{\pgfqpoint{6.438735in}{2.659674in}}%
\pgfpathlineto{\pgfqpoint{6.441907in}{2.659337in}}%
\pgfpathlineto{\pgfqpoint{6.445079in}{2.659138in}}%
\pgfpathlineto{\pgfqpoint{6.448251in}{2.659100in}}%
\pgfpathlineto{\pgfqpoint{6.451423in}{2.659546in}}%
\pgfpathlineto{\pgfqpoint{6.454595in}{2.659798in}}%
\pgfpathlineto{\pgfqpoint{6.457767in}{2.659748in}}%
\pgfpathlineto{\pgfqpoint{6.460939in}{2.660094in}}%
\pgfpathlineto{\pgfqpoint{6.464111in}{2.660263in}}%
\pgfpathlineto{\pgfqpoint{6.467283in}{2.660488in}}%
\pgfpathlineto{\pgfqpoint{6.470455in}{2.660426in}}%
\pgfpathlineto{\pgfqpoint{6.473627in}{2.660621in}}%
\pgfpathlineto{\pgfqpoint{6.476799in}{2.660433in}}%
\pgfpathlineto{\pgfqpoint{6.479971in}{2.660166in}}%
\pgfpathlineto{\pgfqpoint{6.483143in}{2.660327in}}%
\pgfpathlineto{\pgfqpoint{6.486315in}{2.660352in}}%
\pgfpathlineto{\pgfqpoint{6.489487in}{2.660339in}}%
\pgfpathlineto{\pgfqpoint{6.492659in}{2.660553in}}%
\pgfpathlineto{\pgfqpoint{6.495832in}{2.660323in}}%
\pgfpathlineto{\pgfqpoint{6.499004in}{2.659810in}}%
\pgfpathlineto{\pgfqpoint{6.502176in}{2.659726in}}%
\pgfpathlineto{\pgfqpoint{6.505348in}{2.659631in}}%
\pgfpathlineto{\pgfqpoint{6.508520in}{2.659663in}}%
\pgfpathlineto{\pgfqpoint{6.511692in}{2.659213in}}%
\pgfpathlineto{\pgfqpoint{6.514864in}{2.659364in}}%
\pgfpathlineto{\pgfqpoint{6.518036in}{2.659184in}}%
\pgfpathlineto{\pgfqpoint{6.521208in}{2.658805in}}%
\pgfpathlineto{\pgfqpoint{6.524380in}{2.658479in}}%
\pgfpathlineto{\pgfqpoint{6.527552in}{2.658068in}}%
\pgfpathlineto{\pgfqpoint{6.530724in}{2.657752in}}%
\pgfpathlineto{\pgfqpoint{6.533896in}{2.657882in}}%
\pgfpathlineto{\pgfqpoint{6.537068in}{2.657767in}}%
\pgfpathlineto{\pgfqpoint{6.540240in}{2.657471in}}%
\pgfpathlineto{\pgfqpoint{6.543412in}{2.657577in}}%
\pgfpathlineto{\pgfqpoint{6.546584in}{2.657238in}}%
\pgfpathlineto{\pgfqpoint{6.549756in}{2.657065in}}%
\pgfpathlineto{\pgfqpoint{6.552928in}{2.656817in}}%
\pgfpathlineto{\pgfqpoint{6.556100in}{2.656306in}}%
\pgfpathlineto{\pgfqpoint{6.559272in}{2.656381in}}%
\pgfpathlineto{\pgfqpoint{6.562444in}{2.655964in}}%
\pgfpathlineto{\pgfqpoint{6.565616in}{2.656186in}}%
\pgfpathlineto{\pgfqpoint{6.568788in}{2.655864in}}%
\pgfpathlineto{\pgfqpoint{6.571961in}{2.656250in}}%
\pgfpathlineto{\pgfqpoint{6.575133in}{2.656583in}}%
\pgfpathlineto{\pgfqpoint{6.578305in}{2.656592in}}%
\pgfpathlineto{\pgfqpoint{6.581477in}{2.656821in}}%
\pgfpathlineto{\pgfqpoint{6.584649in}{2.656644in}}%
\pgfpathlineto{\pgfqpoint{6.587821in}{2.656768in}}%
\pgfpathlineto{\pgfqpoint{6.590993in}{2.657036in}}%
\pgfpathlineto{\pgfqpoint{6.594165in}{2.657155in}}%
\pgfpathlineto{\pgfqpoint{6.597337in}{2.657653in}}%
\pgfpathlineto{\pgfqpoint{6.600509in}{2.658115in}}%
\pgfpathlineto{\pgfqpoint{6.603681in}{2.658121in}}%
\pgfpathlineto{\pgfqpoint{6.606853in}{2.658106in}}%
\pgfpathlineto{\pgfqpoint{6.610025in}{2.658033in}}%
\pgfpathlineto{\pgfqpoint{6.613197in}{2.657425in}}%
\pgfpathlineto{\pgfqpoint{6.616369in}{2.657712in}}%
\pgfpathlineto{\pgfqpoint{6.619541in}{2.657550in}}%
\pgfpathlineto{\pgfqpoint{6.622713in}{2.657414in}}%
\pgfpathlineto{\pgfqpoint{6.625885in}{2.657533in}}%
\pgfpathlineto{\pgfqpoint{6.629057in}{2.658008in}}%
\pgfpathlineto{\pgfqpoint{6.632229in}{2.657932in}}%
\pgfpathlineto{\pgfqpoint{6.635401in}{2.657934in}}%
\pgfpathlineto{\pgfqpoint{6.638573in}{2.657462in}}%
\pgfpathlineto{\pgfqpoint{6.641745in}{2.657938in}}%
\pgfpathlineto{\pgfqpoint{6.644917in}{2.657750in}}%
\pgfpathlineto{\pgfqpoint{6.648089in}{2.657664in}}%
\pgfpathlineto{\pgfqpoint{6.651262in}{2.657869in}}%
\pgfpathlineto{\pgfqpoint{6.654434in}{2.658218in}}%
\pgfpathlineto{\pgfqpoint{6.657606in}{2.658102in}}%
\pgfpathlineto{\pgfqpoint{6.660778in}{2.658004in}}%
\pgfpathlineto{\pgfqpoint{6.663950in}{2.657882in}}%
\pgfpathlineto{\pgfqpoint{6.667122in}{2.657553in}}%
\pgfpathlineto{\pgfqpoint{6.670294in}{2.657802in}}%
\pgfpathlineto{\pgfqpoint{6.673466in}{2.657649in}}%
\pgfpathlineto{\pgfqpoint{6.676638in}{2.657983in}}%
\pgfpathlineto{\pgfqpoint{6.679810in}{2.657812in}}%
\pgfpathlineto{\pgfqpoint{6.682982in}{2.658051in}}%
\pgfpathlineto{\pgfqpoint{6.686154in}{2.657717in}}%
\pgfpathlineto{\pgfqpoint{6.689326in}{2.657520in}}%
\pgfpathlineto{\pgfqpoint{6.692498in}{2.657427in}}%
\pgfpathlineto{\pgfqpoint{6.695670in}{2.657676in}}%
\pgfpathlineto{\pgfqpoint{6.698842in}{2.657398in}}%
\pgfpathlineto{\pgfqpoint{6.702014in}{2.657383in}}%
\pgfpathlineto{\pgfqpoint{6.705186in}{2.657331in}}%
\pgfpathlineto{\pgfqpoint{6.708358in}{2.657703in}}%
\pgfpathlineto{\pgfqpoint{6.711530in}{2.657690in}}%
\pgfpathlineto{\pgfqpoint{6.714702in}{2.657266in}}%
\pgfpathlineto{\pgfqpoint{6.717874in}{2.657226in}}%
\pgfpathlineto{\pgfqpoint{6.721046in}{2.657234in}}%
\pgfpathlineto{\pgfqpoint{6.724218in}{2.657522in}}%
\pgfpathlineto{\pgfqpoint{6.727391in}{2.657072in}}%
\pgfpathlineto{\pgfqpoint{6.730563in}{2.657383in}}%
\pgfpathlineto{\pgfqpoint{6.733735in}{2.657590in}}%
\pgfpathlineto{\pgfqpoint{6.736907in}{2.657697in}}%
\pgfpathlineto{\pgfqpoint{6.740079in}{2.657434in}}%
\pgfpathlineto{\pgfqpoint{6.743251in}{2.657747in}}%
\pgfpathlineto{\pgfqpoint{6.746423in}{2.657233in}}%
\pgfpathlineto{\pgfqpoint{6.749595in}{2.657465in}}%
\pgfpathlineto{\pgfqpoint{6.752767in}{2.657084in}}%
\pgfpathlineto{\pgfqpoint{6.755939in}{2.657037in}}%
\pgfpathlineto{\pgfqpoint{6.759111in}{2.656932in}}%
\pgfpathlineto{\pgfqpoint{6.762283in}{2.657551in}}%
\pgfpathlineto{\pgfqpoint{6.765455in}{2.657356in}}%
\pgfpathlineto{\pgfqpoint{6.768627in}{2.656948in}}%
\pgfpathlineto{\pgfqpoint{6.771799in}{2.657264in}}%
\pgfpathlineto{\pgfqpoint{6.774971in}{2.657265in}}%
\pgfpathlineto{\pgfqpoint{6.778143in}{2.657447in}}%
\pgfpathlineto{\pgfqpoint{6.781315in}{2.657651in}}%
\pgfpathlineto{\pgfqpoint{6.784487in}{2.658039in}}%
\pgfpathlineto{\pgfqpoint{6.787659in}{2.658670in}}%
\pgfpathlineto{\pgfqpoint{6.790831in}{2.659207in}}%
\pgfpathlineto{\pgfqpoint{6.794003in}{2.658881in}}%
\pgfpathlineto{\pgfqpoint{6.797175in}{2.658585in}}%
\pgfpathlineto{\pgfqpoint{6.800347in}{2.658347in}}%
\pgfpathlineto{\pgfqpoint{6.803519in}{2.658530in}}%
\pgfpathlineto{\pgfqpoint{6.806692in}{2.659154in}}%
\pgfpathlineto{\pgfqpoint{6.809864in}{2.659068in}}%
\pgfpathlineto{\pgfqpoint{6.813036in}{2.658563in}}%
\pgfpathlineto{\pgfqpoint{6.816208in}{2.658052in}}%
\pgfpathlineto{\pgfqpoint{6.819380in}{2.658505in}}%
\pgfpathlineto{\pgfqpoint{6.822552in}{2.658575in}}%
\pgfpathlineto{\pgfqpoint{6.825724in}{2.658680in}}%
\pgfpathlineto{\pgfqpoint{6.828896in}{2.658667in}}%
\pgfpathlineto{\pgfqpoint{6.832068in}{2.658619in}}%
\pgfpathlineto{\pgfqpoint{6.835240in}{2.658632in}}%
\pgfpathlineto{\pgfqpoint{6.838412in}{2.658646in}}%
\pgfpathlineto{\pgfqpoint{6.841584in}{2.658429in}}%
\pgfpathlineto{\pgfqpoint{6.844756in}{2.657834in}}%
\pgfpathlineto{\pgfqpoint{6.847928in}{2.657476in}}%
\pgfpathlineto{\pgfqpoint{6.851100in}{2.657432in}}%
\pgfpathlineto{\pgfqpoint{6.854272in}{2.657428in}}%
\pgfpathlineto{\pgfqpoint{6.857444in}{2.657333in}}%
\pgfpathlineto{\pgfqpoint{6.860616in}{2.657140in}}%
\pgfpathlineto{\pgfqpoint{6.863788in}{2.656702in}}%
\pgfpathlineto{\pgfqpoint{6.866960in}{2.656433in}}%
\pgfpathlineto{\pgfqpoint{6.870132in}{2.656038in}}%
\pgfpathlineto{\pgfqpoint{6.873304in}{2.655583in}}%
\pgfpathlineto{\pgfqpoint{6.876476in}{2.655199in}}%
\pgfpathlineto{\pgfqpoint{6.879648in}{2.655038in}}%
\pgfpathlineto{\pgfqpoint{6.882820in}{2.654667in}}%
\pgfpathlineto{\pgfqpoint{6.885993in}{2.654861in}}%
\pgfpathlineto{\pgfqpoint{6.889165in}{2.654627in}}%
\pgfpathlineto{\pgfqpoint{6.892337in}{2.654839in}}%
\pgfpathlineto{\pgfqpoint{6.895509in}{2.655152in}}%
\pgfpathlineto{\pgfqpoint{6.898681in}{2.655980in}}%
\pgfpathlineto{\pgfqpoint{6.901853in}{2.656185in}}%
\pgfpathlineto{\pgfqpoint{6.905025in}{2.656159in}}%
\pgfpathlineto{\pgfqpoint{6.908197in}{2.656386in}}%
\pgfpathlineto{\pgfqpoint{6.911369in}{2.656165in}}%
\pgfpathlineto{\pgfqpoint{6.914541in}{2.656495in}}%
\pgfpathlineto{\pgfqpoint{6.917713in}{2.656652in}}%
\pgfpathlineto{\pgfqpoint{6.920885in}{2.656479in}}%
\pgfpathlineto{\pgfqpoint{6.924057in}{2.656293in}}%
\pgfpathlineto{\pgfqpoint{6.927229in}{2.656129in}}%
\pgfpathlineto{\pgfqpoint{6.930401in}{2.656195in}}%
\pgfpathlineto{\pgfqpoint{6.933573in}{2.656931in}}%
\pgfpathlineto{\pgfqpoint{6.936745in}{2.656849in}}%
\pgfpathlineto{\pgfqpoint{6.939917in}{2.656770in}}%
\pgfpathlineto{\pgfqpoint{6.943089in}{2.656644in}}%
\pgfpathlineto{\pgfqpoint{6.946261in}{2.656547in}}%
\pgfpathlineto{\pgfqpoint{6.949433in}{2.656449in}}%
\pgfpathlineto{\pgfqpoint{6.952605in}{2.656850in}}%
\pgfpathlineto{\pgfqpoint{6.955777in}{2.656444in}}%
\pgfpathlineto{\pgfqpoint{6.958949in}{2.656822in}}%
\pgfpathlineto{\pgfqpoint{6.962122in}{2.656935in}}%
\pgfpathlineto{\pgfqpoint{6.965294in}{2.656874in}}%
\pgfpathlineto{\pgfqpoint{6.968466in}{2.656705in}}%
\pgfpathlineto{\pgfqpoint{6.971638in}{2.656478in}}%
\pgfpathlineto{\pgfqpoint{6.974810in}{2.656628in}}%
\pgfpathlineto{\pgfqpoint{6.977982in}{2.656722in}}%
\pgfpathlineto{\pgfqpoint{6.981154in}{2.656844in}}%
\pgfpathlineto{\pgfqpoint{6.984326in}{2.657015in}}%
\pgfpathlineto{\pgfqpoint{6.987498in}{2.657046in}}%
\pgfpathlineto{\pgfqpoint{6.990670in}{2.656949in}}%
\pgfpathlineto{\pgfqpoint{6.993842in}{2.656826in}}%
\pgfpathlineto{\pgfqpoint{6.997014in}{2.657072in}}%
\pgfpathlineto{\pgfqpoint{7.000186in}{2.657116in}}%
\pgfpathlineto{\pgfqpoint{7.003358in}{2.657173in}}%
\pgfpathlineto{\pgfqpoint{7.006530in}{2.657190in}}%
\pgfpathlineto{\pgfqpoint{7.009702in}{2.657261in}}%
\pgfpathlineto{\pgfqpoint{7.012874in}{2.657270in}}%
\pgfpathlineto{\pgfqpoint{7.016046in}{2.657139in}}%
\pgfpathlineto{\pgfqpoint{7.019218in}{2.657285in}}%
\pgfpathlineto{\pgfqpoint{7.022390in}{2.657123in}}%
\pgfpathlineto{\pgfqpoint{7.025562in}{2.657185in}}%
\pgfpathlineto{\pgfqpoint{7.028734in}{2.657226in}}%
\pgfpathlineto{\pgfqpoint{7.031906in}{2.656993in}}%
\pgfpathlineto{\pgfqpoint{7.035078in}{2.657157in}}%
\pgfpathlineto{\pgfqpoint{7.038250in}{2.657285in}}%
\pgfpathlineto{\pgfqpoint{7.041423in}{2.657605in}}%
\pgfpathlineto{\pgfqpoint{7.044595in}{2.658218in}}%
\pgfpathlineto{\pgfqpoint{7.047767in}{2.658915in}}%
\pgfpathlineto{\pgfqpoint{7.050939in}{2.658827in}}%
\pgfpathlineto{\pgfqpoint{7.054111in}{2.658535in}}%
\pgfpathlineto{\pgfqpoint{7.057283in}{2.659041in}}%
\pgfpathlineto{\pgfqpoint{7.060455in}{2.659124in}}%
\pgfpathlineto{\pgfqpoint{7.063627in}{2.659212in}}%
\pgfpathlineto{\pgfqpoint{7.066799in}{2.659263in}}%
\pgfpathlineto{\pgfqpoint{7.069971in}{2.659208in}}%
\pgfpathlineto{\pgfqpoint{7.073143in}{2.658959in}}%
\pgfpathlineto{\pgfqpoint{7.076315in}{2.659015in}}%
\pgfpathlineto{\pgfqpoint{7.079487in}{2.658730in}}%
\pgfpathlineto{\pgfqpoint{7.082659in}{2.658226in}}%
\pgfpathlineto{\pgfqpoint{7.085831in}{2.658219in}}%
\pgfpathlineto{\pgfqpoint{7.089003in}{2.657888in}}%
\pgfpathlineto{\pgfqpoint{7.092175in}{2.658127in}}%
\pgfpathlineto{\pgfqpoint{7.095347in}{2.657595in}}%
\pgfpathlineto{\pgfqpoint{7.098519in}{2.657628in}}%
\pgfpathlineto{\pgfqpoint{7.101691in}{2.657644in}}%
\pgfpathlineto{\pgfqpoint{7.104863in}{2.657597in}}%
\pgfpathlineto{\pgfqpoint{7.108035in}{2.657797in}}%
\pgfpathlineto{\pgfqpoint{7.111207in}{2.657927in}}%
\pgfpathlineto{\pgfqpoint{7.114379in}{2.657915in}}%
\pgfpathlineto{\pgfqpoint{7.117551in}{2.658084in}}%
\pgfpathlineto{\pgfqpoint{7.120724in}{2.658061in}}%
\pgfpathlineto{\pgfqpoint{7.123896in}{2.658414in}}%
\pgfpathlineto{\pgfqpoint{7.127068in}{2.658228in}}%
\pgfpathlineto{\pgfqpoint{7.130240in}{2.658417in}}%
\pgfpathlineto{\pgfqpoint{7.133412in}{2.658938in}}%
\pgfpathlineto{\pgfqpoint{7.136584in}{2.658433in}}%
\pgfpathlineto{\pgfqpoint{7.139756in}{2.658263in}}%
\pgfpathlineto{\pgfqpoint{7.142928in}{2.658489in}}%
\pgfpathlineto{\pgfqpoint{7.146100in}{2.658165in}}%
\pgfpathlineto{\pgfqpoint{7.149272in}{2.658235in}}%
\pgfpathlineto{\pgfqpoint{7.152444in}{2.658137in}}%
\pgfpathlineto{\pgfqpoint{7.155616in}{2.658505in}}%
\pgfpathlineto{\pgfqpoint{7.158788in}{2.658824in}}%
\pgfpathlineto{\pgfqpoint{7.161960in}{2.659057in}}%
\pgfpathlineto{\pgfqpoint{7.165132in}{2.659330in}}%
\pgfpathlineto{\pgfqpoint{7.168304in}{2.659834in}}%
\pgfpathlineto{\pgfqpoint{7.171476in}{2.659981in}}%
\pgfpathlineto{\pgfqpoint{7.174648in}{2.660039in}}%
\pgfpathlineto{\pgfqpoint{7.177820in}{2.659743in}}%
\pgfpathlineto{\pgfqpoint{7.180992in}{2.659826in}}%
\pgfpathlineto{\pgfqpoint{7.184164in}{2.659463in}}%
\pgfpathlineto{\pgfqpoint{7.187336in}{2.659540in}}%
\pgfpathlineto{\pgfqpoint{7.190508in}{2.659574in}}%
\pgfpathlineto{\pgfqpoint{7.193680in}{2.659554in}}%
\pgfpathlineto{\pgfqpoint{7.196853in}{2.659450in}}%
\pgfpathlineto{\pgfqpoint{7.200025in}{2.659533in}}%
\pgfpathlineto{\pgfqpoint{7.203197in}{2.659418in}}%
\pgfpathlineto{\pgfqpoint{7.206369in}{2.659601in}}%
\pgfpathlineto{\pgfqpoint{7.209541in}{2.659979in}}%
\pgfpathlineto{\pgfqpoint{7.212713in}{2.659537in}}%
\pgfpathlineto{\pgfqpoint{7.215885in}{2.659482in}}%
\pgfpathlineto{\pgfqpoint{7.219057in}{2.659620in}}%
\pgfpathlineto{\pgfqpoint{7.222229in}{2.659296in}}%
\pgfpathlineto{\pgfqpoint{7.225401in}{2.659276in}}%
\pgfpathlineto{\pgfqpoint{7.228573in}{2.659182in}}%
\pgfpathlineto{\pgfqpoint{7.231745in}{2.659029in}}%
\pgfpathlineto{\pgfqpoint{7.234917in}{2.658829in}}%
\pgfpathlineto{\pgfqpoint{7.238089in}{2.659287in}}%
\pgfpathlineto{\pgfqpoint{7.241261in}{2.659243in}}%
\pgfpathlineto{\pgfqpoint{7.244433in}{2.659215in}}%
\pgfpathlineto{\pgfqpoint{7.247605in}{2.659346in}}%
\pgfpathlineto{\pgfqpoint{7.250777in}{2.659444in}}%
\pgfpathlineto{\pgfqpoint{7.253949in}{2.659648in}}%
\pgfpathlineto{\pgfqpoint{7.257121in}{2.659989in}}%
\pgfpathlineto{\pgfqpoint{7.260293in}{2.660089in}}%
\pgfpathlineto{\pgfqpoint{7.263465in}{2.659743in}}%
\pgfpathlineto{\pgfqpoint{7.266637in}{2.659214in}}%
\pgfpathlineto{\pgfqpoint{7.269809in}{2.659328in}}%
\pgfpathlineto{\pgfqpoint{7.272981in}{2.659394in}}%
\pgfpathlineto{\pgfqpoint{7.276154in}{2.659341in}}%
\pgfpathlineto{\pgfqpoint{7.279326in}{2.659284in}}%
\pgfpathlineto{\pgfqpoint{7.282498in}{2.659284in}}%
\pgfpathlineto{\pgfqpoint{7.285670in}{2.659307in}}%
\pgfpathlineto{\pgfqpoint{7.288842in}{2.658880in}}%
\pgfpathlineto{\pgfqpoint{7.292014in}{2.659074in}}%
\pgfpathlineto{\pgfqpoint{7.295186in}{2.658921in}}%
\pgfpathlineto{\pgfqpoint{7.298358in}{2.658992in}}%
\pgfpathlineto{\pgfqpoint{7.301530in}{2.659100in}}%
\pgfpathlineto{\pgfqpoint{7.304702in}{2.659017in}}%
\pgfpathlineto{\pgfqpoint{7.307874in}{2.657918in}}%
\pgfpathlineto{\pgfqpoint{7.311046in}{2.657192in}}%
\pgfpathlineto{\pgfqpoint{7.314218in}{2.656635in}}%
\pgfpathlineto{\pgfqpoint{7.317390in}{2.656277in}}%
\pgfpathlineto{\pgfqpoint{7.320562in}{2.656331in}}%
\pgfpathlineto{\pgfqpoint{7.323734in}{2.656729in}}%
\pgfpathlineto{\pgfqpoint{7.326906in}{2.656730in}}%
\pgfpathlineto{\pgfqpoint{7.330078in}{2.656421in}}%
\pgfpathlineto{\pgfqpoint{7.333250in}{2.656849in}}%
\pgfpathlineto{\pgfqpoint{7.336422in}{2.656611in}}%
\pgfpathlineto{\pgfqpoint{7.339594in}{2.656469in}}%
\pgfpathlineto{\pgfqpoint{7.342766in}{2.656100in}}%
\pgfpathlineto{\pgfqpoint{7.345938in}{2.655904in}}%
\pgfpathlineto{\pgfqpoint{7.349110in}{2.655888in}}%
\pgfpathlineto{\pgfqpoint{7.352282in}{2.655905in}}%
\pgfpathlineto{\pgfqpoint{7.355455in}{2.655757in}}%
\pgfpathlineto{\pgfqpoint{7.358627in}{2.655937in}}%
\pgfpathlineto{\pgfqpoint{7.361799in}{2.655938in}}%
\pgfpathlineto{\pgfqpoint{7.364971in}{2.655598in}}%
\pgfpathlineto{\pgfqpoint{7.368143in}{2.655477in}}%
\pgfpathlineto{\pgfqpoint{7.371315in}{2.655643in}}%
\pgfpathlineto{\pgfqpoint{7.374487in}{2.655639in}}%
\pgfpathlineto{\pgfqpoint{7.377659in}{2.655663in}}%
\pgfpathlineto{\pgfqpoint{7.380831in}{2.655997in}}%
\pgfpathlineto{\pgfqpoint{7.384003in}{2.656461in}}%
\pgfpathlineto{\pgfqpoint{7.387175in}{2.656747in}}%
\pgfpathlineto{\pgfqpoint{7.390347in}{2.656784in}}%
\pgfpathlineto{\pgfqpoint{7.393519in}{2.656777in}}%
\pgfpathlineto{\pgfqpoint{7.396691in}{2.656839in}}%
\pgfpathlineto{\pgfqpoint{7.399863in}{2.656720in}}%
\pgfpathlineto{\pgfqpoint{7.403035in}{2.656665in}}%
\pgfpathlineto{\pgfqpoint{7.406207in}{2.656899in}}%
\pgfpathlineto{\pgfqpoint{7.409379in}{2.656966in}}%
\pgfpathlineto{\pgfqpoint{7.412551in}{2.656954in}}%
\pgfpathlineto{\pgfqpoint{7.415723in}{2.657488in}}%
\pgfpathlineto{\pgfqpoint{7.418895in}{2.657332in}}%
\pgfpathlineto{\pgfqpoint{7.422067in}{2.657300in}}%
\pgfpathlineto{\pgfqpoint{7.425239in}{2.657396in}}%
\pgfpathlineto{\pgfqpoint{7.428411in}{2.657140in}}%
\pgfpathlineto{\pgfqpoint{7.431584in}{2.657444in}}%
\pgfpathlineto{\pgfqpoint{7.434756in}{2.657409in}}%
\pgfpathlineto{\pgfqpoint{7.437928in}{2.657400in}}%
\pgfpathlineto{\pgfqpoint{7.441100in}{2.658151in}}%
\pgfpathlineto{\pgfqpoint{7.444272in}{2.658537in}}%
\pgfpathlineto{\pgfqpoint{7.447444in}{2.658608in}}%
\pgfpathlineto{\pgfqpoint{7.450616in}{2.658714in}}%
\pgfpathlineto{\pgfqpoint{7.453788in}{2.659026in}}%
\pgfpathlineto{\pgfqpoint{7.456960in}{2.658586in}}%
\pgfpathlineto{\pgfqpoint{7.460132in}{2.659073in}}%
\pgfpathlineto{\pgfqpoint{7.463304in}{2.659548in}}%
\pgfpathlineto{\pgfqpoint{7.466476in}{2.659232in}}%
\pgfpathlineto{\pgfqpoint{7.469648in}{2.659104in}}%
\pgfpathlineto{\pgfqpoint{7.472820in}{2.659770in}}%
\pgfpathlineto{\pgfqpoint{7.475992in}{2.659567in}}%
\pgfpathlineto{\pgfqpoint{7.479164in}{2.659540in}}%
\pgfpathlineto{\pgfqpoint{7.482336in}{2.659187in}}%
\pgfpathlineto{\pgfqpoint{7.485508in}{2.659086in}}%
\pgfpathlineto{\pgfqpoint{7.488680in}{2.658938in}}%
\pgfpathlineto{\pgfqpoint{7.491852in}{2.658438in}}%
\pgfpathlineto{\pgfqpoint{7.495024in}{2.658812in}}%
\pgfpathlineto{\pgfqpoint{7.498196in}{2.658769in}}%
\pgfpathlineto{\pgfqpoint{7.501368in}{2.658497in}}%
\pgfpathlineto{\pgfqpoint{7.504540in}{2.658602in}}%
\pgfpathlineto{\pgfqpoint{7.507712in}{2.658516in}}%
\pgfpathlineto{\pgfqpoint{7.510885in}{2.658240in}}%
\pgfpathlineto{\pgfqpoint{7.514057in}{2.658224in}}%
\pgfpathlineto{\pgfqpoint{7.517229in}{2.657889in}}%
\pgfpathlineto{\pgfqpoint{7.520401in}{2.657372in}}%
\pgfpathlineto{\pgfqpoint{7.523573in}{2.657414in}}%
\pgfpathlineto{\pgfqpoint{7.526745in}{2.657479in}}%
\pgfpathlineto{\pgfqpoint{7.529917in}{2.656933in}}%
\pgfpathlineto{\pgfqpoint{7.533089in}{2.656742in}}%
\pgfpathlineto{\pgfqpoint{7.536261in}{2.656617in}}%
\pgfpathlineto{\pgfqpoint{7.539433in}{2.656626in}}%
\pgfpathlineto{\pgfqpoint{7.542605in}{2.656633in}}%
\pgfpathlineto{\pgfqpoint{7.545777in}{2.656440in}}%
\pgfpathlineto{\pgfqpoint{7.548949in}{2.656743in}}%
\pgfpathlineto{\pgfqpoint{7.552121in}{2.656384in}}%
\pgfpathlineto{\pgfqpoint{7.555293in}{2.656650in}}%
\pgfpathlineto{\pgfqpoint{7.558465in}{2.656303in}}%
\pgfpathlineto{\pgfqpoint{7.561637in}{2.656202in}}%
\pgfpathlineto{\pgfqpoint{7.564809in}{2.655982in}}%
\pgfpathlineto{\pgfqpoint{7.567981in}{2.655860in}}%
\pgfpathlineto{\pgfqpoint{7.571153in}{2.655936in}}%
\pgfpathlineto{\pgfqpoint{7.574325in}{2.656672in}}%
\pgfpathlineto{\pgfqpoint{7.577497in}{2.657221in}}%
\pgfpathlineto{\pgfqpoint{7.580669in}{2.657260in}}%
\pgfpathlineto{\pgfqpoint{7.583841in}{2.657243in}}%
\pgfpathlineto{\pgfqpoint{7.587013in}{2.657485in}}%
\pgfpathlineto{\pgfqpoint{7.590186in}{2.657673in}}%
\pgfpathlineto{\pgfqpoint{7.593358in}{2.657497in}}%
\pgfpathlineto{\pgfqpoint{7.596530in}{2.657606in}}%
\pgfpathlineto{\pgfqpoint{7.599702in}{2.657765in}}%
\pgfpathlineto{\pgfqpoint{7.602874in}{2.657803in}}%
\pgfpathlineto{\pgfqpoint{7.606046in}{2.657414in}}%
\pgfpathlineto{\pgfqpoint{7.609218in}{2.657111in}}%
\pgfpathlineto{\pgfqpoint{7.612390in}{2.657084in}}%
\pgfpathlineto{\pgfqpoint{7.615562in}{2.657098in}}%
\pgfpathlineto{\pgfqpoint{7.618734in}{2.657160in}}%
\pgfpathlineto{\pgfqpoint{7.621906in}{2.657649in}}%
\pgfpathlineto{\pgfqpoint{7.625078in}{2.657678in}}%
\pgfpathlineto{\pgfqpoint{7.628250in}{2.657373in}}%
\pgfpathlineto{\pgfqpoint{7.631422in}{2.657406in}}%
\pgfpathlineto{\pgfqpoint{7.634594in}{2.657789in}}%
\pgfpathlineto{\pgfqpoint{7.637766in}{2.657839in}}%
\pgfpathlineto{\pgfqpoint{7.640938in}{2.657596in}}%
\pgfpathlineto{\pgfqpoint{7.644110in}{2.657789in}}%
\pgfpathlineto{\pgfqpoint{7.647282in}{2.658553in}}%
\pgfpathlineto{\pgfqpoint{7.650454in}{2.659042in}}%
\pgfpathlineto{\pgfqpoint{7.653626in}{2.659083in}}%
\pgfpathlineto{\pgfqpoint{7.656798in}{2.659104in}}%
\pgfpathlineto{\pgfqpoint{7.659970in}{2.659120in}}%
\pgfpathlineto{\pgfqpoint{7.663142in}{2.658869in}}%
\pgfpathlineto{\pgfqpoint{7.666315in}{2.658987in}}%
\pgfpathlineto{\pgfqpoint{7.669487in}{2.658999in}}%
\pgfpathlineto{\pgfqpoint{7.672659in}{2.659011in}}%
\pgfpathlineto{\pgfqpoint{7.675831in}{2.659374in}}%
\pgfpathlineto{\pgfqpoint{7.679003in}{2.659367in}}%
\pgfpathlineto{\pgfqpoint{7.682175in}{2.659257in}}%
\pgfpathlineto{\pgfqpoint{7.685347in}{2.659319in}}%
\pgfpathlineto{\pgfqpoint{7.688519in}{2.659564in}}%
\pgfpathlineto{\pgfqpoint{7.691691in}{2.659581in}}%
\pgfpathlineto{\pgfqpoint{7.694863in}{2.659704in}}%
\pgfpathlineto{\pgfqpoint{7.698035in}{2.659760in}}%
\pgfpathlineto{\pgfqpoint{7.701207in}{2.659752in}}%
\pgfpathlineto{\pgfqpoint{7.704379in}{2.659623in}}%
\pgfpathlineto{\pgfqpoint{7.707551in}{2.659192in}}%
\pgfpathlineto{\pgfqpoint{7.710723in}{2.658719in}}%
\pgfpathlineto{\pgfqpoint{7.713895in}{2.659025in}}%
\pgfpathlineto{\pgfqpoint{7.717067in}{2.658838in}}%
\pgfpathlineto{\pgfqpoint{7.720239in}{2.658953in}}%
\pgfpathlineto{\pgfqpoint{7.723411in}{2.659139in}}%
\pgfpathlineto{\pgfqpoint{7.726583in}{2.658936in}}%
\pgfpathlineto{\pgfqpoint{7.729755in}{2.658938in}}%
\pgfpathlineto{\pgfqpoint{7.732927in}{2.658035in}}%
\pgfpathlineto{\pgfqpoint{7.736099in}{2.658262in}}%
\pgfpathlineto{\pgfqpoint{7.739271in}{2.658183in}}%
\pgfpathlineto{\pgfqpoint{7.742443in}{2.657589in}}%
\pgfpathlineto{\pgfqpoint{7.745616in}{2.657376in}}%
\pgfpathlineto{\pgfqpoint{7.748788in}{2.657205in}}%
\pgfpathlineto{\pgfqpoint{7.751960in}{2.656947in}}%
\pgfpathlineto{\pgfqpoint{7.755132in}{2.656848in}}%
\pgfpathlineto{\pgfqpoint{7.758304in}{2.657177in}}%
\pgfpathlineto{\pgfqpoint{7.761476in}{2.657668in}}%
\pgfpathlineto{\pgfqpoint{7.764648in}{2.657405in}}%
\pgfpathlineto{\pgfqpoint{7.767820in}{2.657947in}}%
\pgfpathlineto{\pgfqpoint{7.770992in}{2.658168in}}%
\pgfpathlineto{\pgfqpoint{7.774164in}{2.658549in}}%
\pgfpathlineto{\pgfqpoint{7.777336in}{2.658363in}}%
\pgfpathlineto{\pgfqpoint{7.780508in}{2.658560in}}%
\pgfpathlineto{\pgfqpoint{7.783680in}{2.658452in}}%
\pgfpathlineto{\pgfqpoint{7.786852in}{2.658187in}}%
\pgfpathlineto{\pgfqpoint{7.790024in}{2.661081in}}%
\pgfpathlineto{\pgfqpoint{7.793196in}{2.664194in}}%
\pgfpathlineto{\pgfqpoint{7.796368in}{2.667326in}}%
\pgfpathlineto{\pgfqpoint{7.799540in}{2.670412in}}%
\pgfpathlineto{\pgfqpoint{7.802712in}{2.673332in}}%
\pgfpathlineto{\pgfqpoint{7.805884in}{2.676483in}}%
\pgfpathlineto{\pgfqpoint{7.809056in}{2.679554in}}%
\pgfpathlineto{\pgfqpoint{7.812228in}{2.682620in}}%
\pgfpathlineto{\pgfqpoint{7.815400in}{2.685729in}}%
\pgfpathlineto{\pgfqpoint{7.818572in}{2.688809in}}%
\pgfpathlineto{\pgfqpoint{7.821744in}{2.691983in}}%
\pgfpathlineto{\pgfqpoint{7.824917in}{2.695153in}}%
\pgfpathlineto{\pgfqpoint{7.828089in}{2.698137in}}%
\pgfpathlineto{\pgfqpoint{7.831261in}{2.701189in}}%
\pgfpathlineto{\pgfqpoint{7.834433in}{2.704397in}}%
\pgfpathlineto{\pgfqpoint{7.837605in}{2.707506in}}%
\pgfpathlineto{\pgfqpoint{7.840777in}{2.710549in}}%
\pgfpathlineto{\pgfqpoint{7.843949in}{2.713598in}}%
\pgfpathlineto{\pgfqpoint{7.847121in}{2.716720in}}%
\pgfpathlineto{\pgfqpoint{7.850293in}{2.719748in}}%
\pgfpathlineto{\pgfqpoint{7.853465in}{2.722840in}}%
\pgfpathlineto{\pgfqpoint{7.856637in}{2.725917in}}%
\pgfpathlineto{\pgfqpoint{7.859809in}{2.728966in}}%
\pgfpathlineto{\pgfqpoint{7.862981in}{2.732075in}}%
\pgfpathlineto{\pgfqpoint{7.866153in}{2.735222in}}%
\pgfpathlineto{\pgfqpoint{7.869325in}{2.738286in}}%
\pgfpathlineto{\pgfqpoint{7.872497in}{2.741381in}}%
\pgfpathlineto{\pgfqpoint{7.875669in}{2.744468in}}%
\pgfpathlineto{\pgfqpoint{7.878841in}{2.747570in}}%
\pgfpathlineto{\pgfqpoint{7.882013in}{2.750647in}}%
\pgfpathlineto{\pgfqpoint{7.885185in}{2.753653in}}%
\pgfpathlineto{\pgfqpoint{7.888357in}{2.756754in}}%
\pgfpathlineto{\pgfqpoint{7.891529in}{2.759688in}}%
\pgfpathlineto{\pgfqpoint{7.894701in}{2.762746in}}%
\pgfpathlineto{\pgfqpoint{7.897873in}{2.769550in}}%
\pgfpathlineto{\pgfqpoint{7.901046in}{2.773312in}}%
\pgfpathlineto{\pgfqpoint{7.904218in}{2.776393in}}%
\pgfpathlineto{\pgfqpoint{7.907390in}{2.779387in}}%
\pgfpathlineto{\pgfqpoint{7.910562in}{2.782487in}}%
\pgfpathlineto{\pgfqpoint{7.913734in}{2.785534in}}%
\pgfpathlineto{\pgfqpoint{7.916906in}{2.788624in}}%
\pgfpathlineto{\pgfqpoint{7.920078in}{2.791704in}}%
\pgfpathlineto{\pgfqpoint{7.923250in}{2.794808in}}%
\pgfpathlineto{\pgfqpoint{7.926422in}{2.797905in}}%
\pgfpathlineto{\pgfqpoint{7.929594in}{2.800942in}}%
\pgfpathlineto{\pgfqpoint{7.932766in}{2.804026in}}%
\pgfpathlineto{\pgfqpoint{7.935938in}{2.807114in}}%
\pgfpathlineto{\pgfqpoint{7.939110in}{2.810218in}}%
\pgfpathlineto{\pgfqpoint{7.942282in}{2.813237in}}%
\pgfpathlineto{\pgfqpoint{7.945454in}{2.816339in}}%
\pgfpathlineto{\pgfqpoint{7.948626in}{2.819414in}}%
\pgfpathlineto{\pgfqpoint{7.951798in}{2.822533in}}%
\pgfpathlineto{\pgfqpoint{7.954970in}{2.825571in}}%
\pgfpathlineto{\pgfqpoint{7.958142in}{2.828597in}}%
\pgfpathlineto{\pgfqpoint{7.961314in}{2.831693in}}%
\pgfpathlineto{\pgfqpoint{7.964486in}{2.834809in}}%
\pgfpathlineto{\pgfqpoint{7.967658in}{2.837925in}}%
\pgfpathlineto{\pgfqpoint{7.970830in}{2.840982in}}%
\pgfpathlineto{\pgfqpoint{7.974002in}{2.844055in}}%
\pgfpathlineto{\pgfqpoint{7.977174in}{2.847160in}}%
\pgfpathlineto{\pgfqpoint{7.980347in}{2.850239in}}%
\pgfpathlineto{\pgfqpoint{7.983519in}{2.853300in}}%
\pgfpathlineto{\pgfqpoint{7.986691in}{2.856387in}}%
\pgfpathlineto{\pgfqpoint{7.989863in}{2.859604in}}%
\pgfpathlineto{\pgfqpoint{7.993035in}{2.862673in}}%
\pgfpathlineto{\pgfqpoint{7.996207in}{2.865767in}}%
\pgfpathlineto{\pgfqpoint{7.999379in}{2.868493in}}%
\pgfpathlineto{\pgfqpoint{8.002551in}{2.871273in}}%
\pgfpathlineto{\pgfqpoint{8.005723in}{2.874364in}}%
\pgfpathlineto{\pgfqpoint{8.008895in}{2.877413in}}%
\pgfpathlineto{\pgfqpoint{8.012067in}{2.880454in}}%
\pgfpathlineto{\pgfqpoint{8.015239in}{2.883464in}}%
\pgfpathlineto{\pgfqpoint{8.018411in}{2.886520in}}%
\pgfpathlineto{\pgfqpoint{8.021583in}{2.889590in}}%
\pgfpathlineto{\pgfqpoint{8.024755in}{2.892697in}}%
\pgfpathlineto{\pgfqpoint{8.027927in}{2.895701in}}%
\pgfpathlineto{\pgfqpoint{8.031099in}{2.898762in}}%
\pgfpathlineto{\pgfqpoint{8.034271in}{2.901863in}}%
\pgfpathlineto{\pgfqpoint{8.037443in}{2.904911in}}%
\pgfpathlineto{\pgfqpoint{8.040615in}{2.908058in}}%
\pgfpathlineto{\pgfqpoint{8.043787in}{2.911167in}}%
\pgfpathlineto{\pgfqpoint{8.046959in}{2.914381in}}%
\pgfpathlineto{\pgfqpoint{8.050131in}{2.917478in}}%
\pgfpathlineto{\pgfqpoint{8.053303in}{2.920529in}}%
\pgfpathlineto{\pgfqpoint{8.056475in}{2.923596in}}%
\pgfpathlineto{\pgfqpoint{8.059648in}{2.926626in}}%
\pgfpathlineto{\pgfqpoint{8.062820in}{2.929703in}}%
\pgfpathlineto{\pgfqpoint{8.065992in}{2.932819in}}%
\pgfpathlineto{\pgfqpoint{8.069164in}{2.935940in}}%
\pgfpathlineto{\pgfqpoint{8.072336in}{2.939058in}}%
\pgfpathlineto{\pgfqpoint{8.075508in}{2.942098in}}%
\pgfpathlineto{\pgfqpoint{8.078680in}{2.945170in}}%
\pgfpathlineto{\pgfqpoint{8.081852in}{2.948298in}}%
\pgfpathlineto{\pgfqpoint{8.085024in}{2.951526in}}%
\pgfpathlineto{\pgfqpoint{8.088196in}{2.954637in}}%
\pgfpathlineto{\pgfqpoint{8.091368in}{2.957677in}}%
\pgfpathlineto{\pgfqpoint{8.094540in}{2.960778in}}%
\pgfpathlineto{\pgfqpoint{8.097712in}{2.963959in}}%
\pgfpathlineto{\pgfqpoint{8.100884in}{2.967046in}}%
\pgfpathlineto{\pgfqpoint{8.104056in}{2.970104in}}%
\pgfpathlineto{\pgfqpoint{8.107228in}{2.973141in}}%
\pgfpathlineto{\pgfqpoint{8.110400in}{2.976242in}}%
\pgfpathlineto{\pgfqpoint{8.113572in}{2.979345in}}%
\pgfpathlineto{\pgfqpoint{8.116744in}{2.982452in}}%
\pgfpathlineto{\pgfqpoint{8.119916in}{2.985497in}}%
\pgfpathlineto{\pgfqpoint{8.123088in}{2.988591in}}%
\pgfpathlineto{\pgfqpoint{8.126260in}{2.991685in}}%
\pgfpathlineto{\pgfqpoint{8.129432in}{2.994726in}}%
\pgfpathlineto{\pgfqpoint{8.132604in}{2.997781in}}%
\pgfpathlineto{\pgfqpoint{8.135777in}{3.000850in}}%
\pgfpathlineto{\pgfqpoint{8.138949in}{3.003918in}}%
\pgfpathlineto{\pgfqpoint{8.142121in}{3.006971in}}%
\pgfpathlineto{\pgfqpoint{8.145293in}{3.010088in}}%
\pgfpathlineto{\pgfqpoint{8.148465in}{3.013256in}}%
\pgfpathlineto{\pgfqpoint{8.151637in}{3.016348in}}%
\pgfpathlineto{\pgfqpoint{8.154809in}{3.019436in}}%
\pgfpathlineto{\pgfqpoint{8.157981in}{3.022657in}}%
\pgfpathlineto{\pgfqpoint{8.161153in}{3.025737in}}%
\pgfpathlineto{\pgfqpoint{8.164325in}{3.028804in}}%
\pgfpathlineto{\pgfqpoint{8.167497in}{3.031897in}}%
\pgfpathlineto{\pgfqpoint{8.170669in}{3.034957in}}%
\pgfpathlineto{\pgfqpoint{8.173841in}{3.038176in}}%
\pgfpathlineto{\pgfqpoint{8.177013in}{3.041271in}}%
\pgfpathlineto{\pgfqpoint{8.180185in}{3.044225in}}%
\pgfpathlineto{\pgfqpoint{8.183357in}{3.047323in}}%
\pgfpathlineto{\pgfqpoint{8.186529in}{3.050427in}}%
\pgfpathlineto{\pgfqpoint{8.189701in}{3.053533in}}%
\pgfpathlineto{\pgfqpoint{8.192873in}{3.056623in}}%
\pgfpathlineto{\pgfqpoint{8.196045in}{3.059640in}}%
\pgfpathlineto{\pgfqpoint{8.199217in}{3.062758in}}%
\pgfpathlineto{\pgfqpoint{8.202389in}{3.065872in}}%
\pgfpathlineto{\pgfqpoint{8.205561in}{3.068931in}}%
\pgfpathlineto{\pgfqpoint{8.208733in}{3.072044in}}%
\pgfpathlineto{\pgfqpoint{8.211905in}{3.075156in}}%
\pgfpathlineto{\pgfqpoint{8.215078in}{3.078309in}}%
\pgfpathlineto{\pgfqpoint{8.218250in}{3.081352in}}%
\pgfpathlineto{\pgfqpoint{8.221422in}{3.084400in}}%
\pgfpathlineto{\pgfqpoint{8.224594in}{3.087495in}}%
\pgfpathlineto{\pgfqpoint{8.227766in}{3.090586in}}%
\pgfpathlineto{\pgfqpoint{8.230938in}{3.093497in}}%
\pgfpathlineto{\pgfqpoint{8.234110in}{3.096599in}}%
\pgfpathlineto{\pgfqpoint{8.237282in}{3.099601in}}%
\pgfpathlineto{\pgfqpoint{8.240454in}{3.102668in}}%
\pgfpathlineto{\pgfqpoint{8.243626in}{3.105677in}}%
\pgfpathlineto{\pgfqpoint{8.246798in}{3.108755in}}%
\pgfpathlineto{\pgfqpoint{8.249970in}{3.111813in}}%
\pgfpathlineto{\pgfqpoint{8.253142in}{3.114910in}}%
\pgfpathlineto{\pgfqpoint{8.256314in}{3.117887in}}%
\pgfpathlineto{\pgfqpoint{8.259486in}{3.121000in}}%
\pgfpathlineto{\pgfqpoint{8.262658in}{3.124103in}}%
\pgfpathlineto{\pgfqpoint{8.265830in}{3.127237in}}%
\pgfpathlineto{\pgfqpoint{8.269002in}{3.130328in}}%
\pgfpathlineto{\pgfqpoint{8.272174in}{3.133405in}}%
\pgfpathlineto{\pgfqpoint{8.275346in}{3.136419in}}%
\pgfpathlineto{\pgfqpoint{8.278518in}{3.139532in}}%
\pgfpathlineto{\pgfqpoint{8.281690in}{3.142662in}}%
\pgfpathlineto{\pgfqpoint{8.281690in}{3.391584in}}%
\pgfpathlineto{\pgfqpoint{8.281690in}{3.391584in}}%
\pgfpathlineto{\pgfqpoint{8.278518in}{3.388497in}}%
\pgfpathlineto{\pgfqpoint{8.275346in}{3.385357in}}%
\pgfpathlineto{\pgfqpoint{8.272174in}{3.382257in}}%
\pgfpathlineto{\pgfqpoint{8.269002in}{3.378784in}}%
\pgfpathlineto{\pgfqpoint{8.265830in}{3.375676in}}%
\pgfpathlineto{\pgfqpoint{8.262658in}{3.372507in}}%
\pgfpathlineto{\pgfqpoint{8.259486in}{3.369378in}}%
\pgfpathlineto{\pgfqpoint{8.256314in}{3.366174in}}%
\pgfpathlineto{\pgfqpoint{8.253142in}{3.363005in}}%
\pgfpathlineto{\pgfqpoint{8.249970in}{3.359888in}}%
\pgfpathlineto{\pgfqpoint{8.246798in}{3.356739in}}%
\pgfpathlineto{\pgfqpoint{8.243626in}{3.353608in}}%
\pgfpathlineto{\pgfqpoint{8.240454in}{3.350491in}}%
\pgfpathlineto{\pgfqpoint{8.237282in}{3.347344in}}%
\pgfpathlineto{\pgfqpoint{8.234110in}{3.344178in}}%
\pgfpathlineto{\pgfqpoint{8.230938in}{3.341026in}}%
\pgfpathlineto{\pgfqpoint{8.227766in}{3.337860in}}%
\pgfpathlineto{\pgfqpoint{8.224594in}{3.334753in}}%
\pgfpathlineto{\pgfqpoint{8.221422in}{3.331563in}}%
\pgfpathlineto{\pgfqpoint{8.218250in}{3.328417in}}%
\pgfpathlineto{\pgfqpoint{8.215078in}{3.325282in}}%
\pgfpathlineto{\pgfqpoint{8.211905in}{3.322169in}}%
\pgfpathlineto{\pgfqpoint{8.208733in}{3.318995in}}%
\pgfpathlineto{\pgfqpoint{8.205561in}{3.315782in}}%
\pgfpathlineto{\pgfqpoint{8.202389in}{3.312553in}}%
\pgfpathlineto{\pgfqpoint{8.199217in}{3.309393in}}%
\pgfpathlineto{\pgfqpoint{8.196045in}{3.306278in}}%
\pgfpathlineto{\pgfqpoint{8.192873in}{3.303134in}}%
\pgfpathlineto{\pgfqpoint{8.189701in}{3.300005in}}%
\pgfpathlineto{\pgfqpoint{8.186529in}{3.296840in}}%
\pgfpathlineto{\pgfqpoint{8.183357in}{3.293650in}}%
\pgfpathlineto{\pgfqpoint{8.180185in}{3.290518in}}%
\pgfpathlineto{\pgfqpoint{8.177013in}{3.287235in}}%
\pgfpathlineto{\pgfqpoint{8.173841in}{3.284103in}}%
\pgfpathlineto{\pgfqpoint{8.170669in}{3.281033in}}%
\pgfpathlineto{\pgfqpoint{8.167497in}{3.277911in}}%
\pgfpathlineto{\pgfqpoint{8.164325in}{3.274767in}}%
\pgfpathlineto{\pgfqpoint{8.161153in}{3.271643in}}%
\pgfpathlineto{\pgfqpoint{8.157981in}{3.268495in}}%
\pgfpathlineto{\pgfqpoint{8.154809in}{3.265434in}}%
\pgfpathlineto{\pgfqpoint{8.151637in}{3.262290in}}%
\pgfpathlineto{\pgfqpoint{8.148465in}{3.259007in}}%
\pgfpathlineto{\pgfqpoint{8.145293in}{3.255865in}}%
\pgfpathlineto{\pgfqpoint{8.142121in}{3.252608in}}%
\pgfpathlineto{\pgfqpoint{8.138949in}{3.249498in}}%
\pgfpathlineto{\pgfqpoint{8.135777in}{3.246391in}}%
\pgfpathlineto{\pgfqpoint{8.132604in}{3.243249in}}%
\pgfpathlineto{\pgfqpoint{8.129432in}{3.240091in}}%
\pgfpathlineto{\pgfqpoint{8.126260in}{3.237173in}}%
\pgfpathlineto{\pgfqpoint{8.123088in}{3.234044in}}%
\pgfpathlineto{\pgfqpoint{8.119916in}{3.230880in}}%
\pgfpathlineto{\pgfqpoint{8.116744in}{3.227746in}}%
\pgfpathlineto{\pgfqpoint{8.113572in}{3.224585in}}%
\pgfpathlineto{\pgfqpoint{8.110400in}{3.221455in}}%
\pgfpathlineto{\pgfqpoint{8.107228in}{3.218300in}}%
\pgfpathlineto{\pgfqpoint{8.104056in}{3.215152in}}%
\pgfpathlineto{\pgfqpoint{8.100884in}{3.212031in}}%
\pgfpathlineto{\pgfqpoint{8.097712in}{3.208918in}}%
\pgfpathlineto{\pgfqpoint{8.094540in}{3.205835in}}%
\pgfpathlineto{\pgfqpoint{8.091368in}{3.202689in}}%
\pgfpathlineto{\pgfqpoint{8.088196in}{3.199544in}}%
\pgfpathlineto{\pgfqpoint{8.085024in}{3.196427in}}%
\pgfpathlineto{\pgfqpoint{8.081852in}{3.193338in}}%
\pgfpathlineto{\pgfqpoint{8.078680in}{3.190213in}}%
\pgfpathlineto{\pgfqpoint{8.075508in}{3.187066in}}%
\pgfpathlineto{\pgfqpoint{8.072336in}{3.183936in}}%
\pgfpathlineto{\pgfqpoint{8.069164in}{3.180795in}}%
\pgfpathlineto{\pgfqpoint{8.065992in}{3.177669in}}%
\pgfpathlineto{\pgfqpoint{8.062820in}{3.174529in}}%
\pgfpathlineto{\pgfqpoint{8.059648in}{3.171375in}}%
\pgfpathlineto{\pgfqpoint{8.056475in}{3.168220in}}%
\pgfpathlineto{\pgfqpoint{8.053303in}{3.165071in}}%
\pgfpathlineto{\pgfqpoint{8.050131in}{3.161897in}}%
\pgfpathlineto{\pgfqpoint{8.046959in}{3.158654in}}%
\pgfpathlineto{\pgfqpoint{8.043787in}{3.155595in}}%
\pgfpathlineto{\pgfqpoint{8.040615in}{3.152411in}}%
\pgfpathlineto{\pgfqpoint{8.037443in}{3.148961in}}%
\pgfpathlineto{\pgfqpoint{8.034271in}{3.145824in}}%
\pgfpathlineto{\pgfqpoint{8.031099in}{3.142708in}}%
\pgfpathlineto{\pgfqpoint{8.027927in}{3.139596in}}%
\pgfpathlineto{\pgfqpoint{8.024755in}{3.136459in}}%
\pgfpathlineto{\pgfqpoint{8.021583in}{3.133339in}}%
\pgfpathlineto{\pgfqpoint{8.018411in}{3.130231in}}%
\pgfpathlineto{\pgfqpoint{8.015239in}{3.127140in}}%
\pgfpathlineto{\pgfqpoint{8.012067in}{3.124008in}}%
\pgfpathlineto{\pgfqpoint{8.008895in}{3.120818in}}%
\pgfpathlineto{\pgfqpoint{8.005723in}{3.117670in}}%
\pgfpathlineto{\pgfqpoint{8.002551in}{3.114526in}}%
\pgfpathlineto{\pgfqpoint{7.999379in}{3.111327in}}%
\pgfpathlineto{\pgfqpoint{7.996207in}{3.108220in}}%
\pgfpathlineto{\pgfqpoint{7.993035in}{3.105073in}}%
\pgfpathlineto{\pgfqpoint{7.989863in}{3.101865in}}%
\pgfpathlineto{\pgfqpoint{7.986691in}{3.098816in}}%
\pgfpathlineto{\pgfqpoint{7.983519in}{3.095713in}}%
\pgfpathlineto{\pgfqpoint{7.980347in}{3.092618in}}%
\pgfpathlineto{\pgfqpoint{7.977174in}{3.089496in}}%
\pgfpathlineto{\pgfqpoint{7.974002in}{3.086374in}}%
\pgfpathlineto{\pgfqpoint{7.970830in}{3.083219in}}%
\pgfpathlineto{\pgfqpoint{7.967658in}{3.080065in}}%
\pgfpathlineto{\pgfqpoint{7.964486in}{3.076936in}}%
\pgfpathlineto{\pgfqpoint{7.961314in}{3.073783in}}%
\pgfpathlineto{\pgfqpoint{7.958142in}{3.070629in}}%
\pgfpathlineto{\pgfqpoint{7.954970in}{3.067492in}}%
\pgfpathlineto{\pgfqpoint{7.951798in}{3.064397in}}%
\pgfpathlineto{\pgfqpoint{7.948626in}{3.061297in}}%
\pgfpathlineto{\pgfqpoint{7.945454in}{3.058161in}}%
\pgfpathlineto{\pgfqpoint{7.942282in}{3.055004in}}%
\pgfpathlineto{\pgfqpoint{7.939110in}{3.051882in}}%
\pgfpathlineto{\pgfqpoint{7.935938in}{3.048746in}}%
\pgfpathlineto{\pgfqpoint{7.932766in}{3.045615in}}%
\pgfpathlineto{\pgfqpoint{7.929594in}{3.042530in}}%
\pgfpathlineto{\pgfqpoint{7.926422in}{3.039390in}}%
\pgfpathlineto{\pgfqpoint{7.923250in}{3.036289in}}%
\pgfpathlineto{\pgfqpoint{7.920078in}{3.033180in}}%
\pgfpathlineto{\pgfqpoint{7.916906in}{3.030051in}}%
\pgfpathlineto{\pgfqpoint{7.913734in}{3.026930in}}%
\pgfpathlineto{\pgfqpoint{7.910562in}{3.023787in}}%
\pgfpathlineto{\pgfqpoint{7.907390in}{3.020638in}}%
\pgfpathlineto{\pgfqpoint{7.904218in}{3.017447in}}%
\pgfpathlineto{\pgfqpoint{7.901046in}{3.014326in}}%
\pgfpathlineto{\pgfqpoint{7.897873in}{3.011353in}}%
\pgfpathlineto{\pgfqpoint{7.894701in}{3.009098in}}%
\pgfpathlineto{\pgfqpoint{7.891529in}{3.005978in}}%
\pgfpathlineto{\pgfqpoint{7.888357in}{3.002816in}}%
\pgfpathlineto{\pgfqpoint{7.885185in}{2.999748in}}%
\pgfpathlineto{\pgfqpoint{7.882013in}{2.996637in}}%
\pgfpathlineto{\pgfqpoint{7.878841in}{2.993548in}}%
\pgfpathlineto{\pgfqpoint{7.875669in}{2.990448in}}%
\pgfpathlineto{\pgfqpoint{7.872497in}{2.987292in}}%
\pgfpathlineto{\pgfqpoint{7.869325in}{2.984161in}}%
\pgfpathlineto{\pgfqpoint{7.866153in}{2.980990in}}%
\pgfpathlineto{\pgfqpoint{7.862981in}{2.977815in}}%
\pgfpathlineto{\pgfqpoint{7.859809in}{2.974653in}}%
\pgfpathlineto{\pgfqpoint{7.856637in}{2.971460in}}%
\pgfpathlineto{\pgfqpoint{7.853465in}{2.968354in}}%
\pgfpathlineto{\pgfqpoint{7.850293in}{2.965241in}}%
\pgfpathlineto{\pgfqpoint{7.847121in}{2.962099in}}%
\pgfpathlineto{\pgfqpoint{7.843949in}{2.958984in}}%
\pgfpathlineto{\pgfqpoint{7.840777in}{2.955821in}}%
\pgfpathlineto{\pgfqpoint{7.837605in}{2.952693in}}%
\pgfpathlineto{\pgfqpoint{7.834433in}{2.949578in}}%
\pgfpathlineto{\pgfqpoint{7.831261in}{2.946492in}}%
\pgfpathlineto{\pgfqpoint{7.828089in}{2.943425in}}%
\pgfpathlineto{\pgfqpoint{7.824917in}{2.940032in}}%
\pgfpathlineto{\pgfqpoint{7.821744in}{2.936572in}}%
\pgfpathlineto{\pgfqpoint{7.818572in}{2.933136in}}%
\pgfpathlineto{\pgfqpoint{7.815400in}{2.930003in}}%
\pgfpathlineto{\pgfqpoint{7.812228in}{2.926879in}}%
\pgfpathlineto{\pgfqpoint{7.809056in}{2.923793in}}%
\pgfpathlineto{\pgfqpoint{7.805884in}{2.920659in}}%
\pgfpathlineto{\pgfqpoint{7.802712in}{2.917565in}}%
\pgfpathlineto{\pgfqpoint{7.799540in}{2.914349in}}%
\pgfpathlineto{\pgfqpoint{7.796368in}{2.911160in}}%
\pgfpathlineto{\pgfqpoint{7.793196in}{2.907914in}}%
\pgfpathlineto{\pgfqpoint{7.790024in}{2.904838in}}%
\pgfpathlineto{\pgfqpoint{7.786852in}{2.901640in}}%
\pgfpathlineto{\pgfqpoint{7.783680in}{2.901596in}}%
\pgfpathlineto{\pgfqpoint{7.780508in}{2.902184in}}%
\pgfpathlineto{\pgfqpoint{7.777336in}{2.901981in}}%
\pgfpathlineto{\pgfqpoint{7.774164in}{2.901628in}}%
\pgfpathlineto{\pgfqpoint{7.770992in}{2.901456in}}%
\pgfpathlineto{\pgfqpoint{7.767820in}{2.901196in}}%
\pgfpathlineto{\pgfqpoint{7.764648in}{2.901447in}}%
\pgfpathlineto{\pgfqpoint{7.761476in}{2.901648in}}%
\pgfpathlineto{\pgfqpoint{7.758304in}{2.901518in}}%
\pgfpathlineto{\pgfqpoint{7.755132in}{2.901337in}}%
\pgfpathlineto{\pgfqpoint{7.751960in}{2.901424in}}%
\pgfpathlineto{\pgfqpoint{7.748788in}{2.901678in}}%
\pgfpathlineto{\pgfqpoint{7.745616in}{2.901816in}}%
\pgfpathlineto{\pgfqpoint{7.742443in}{2.901567in}}%
\pgfpathlineto{\pgfqpoint{7.739271in}{2.901697in}}%
\pgfpathlineto{\pgfqpoint{7.736099in}{2.902071in}}%
\pgfpathlineto{\pgfqpoint{7.732927in}{2.902285in}}%
\pgfpathlineto{\pgfqpoint{7.729755in}{2.902147in}}%
\pgfpathlineto{\pgfqpoint{7.726583in}{2.902586in}}%
\pgfpathlineto{\pgfqpoint{7.723411in}{2.902275in}}%
\pgfpathlineto{\pgfqpoint{7.720239in}{2.902439in}}%
\pgfpathlineto{\pgfqpoint{7.717067in}{2.902566in}}%
\pgfpathlineto{\pgfqpoint{7.713895in}{2.902482in}}%
\pgfpathlineto{\pgfqpoint{7.710723in}{2.901783in}}%
\pgfpathlineto{\pgfqpoint{7.707551in}{2.902023in}}%
\pgfpathlineto{\pgfqpoint{7.704379in}{2.902104in}}%
\pgfpathlineto{\pgfqpoint{7.701207in}{2.902001in}}%
\pgfpathlineto{\pgfqpoint{7.698035in}{2.901746in}}%
\pgfpathlineto{\pgfqpoint{7.694863in}{2.901517in}}%
\pgfpathlineto{\pgfqpoint{7.691691in}{2.901299in}}%
\pgfpathlineto{\pgfqpoint{7.688519in}{2.901220in}}%
\pgfpathlineto{\pgfqpoint{7.685347in}{2.900622in}}%
\pgfpathlineto{\pgfqpoint{7.682175in}{2.900696in}}%
\pgfpathlineto{\pgfqpoint{7.679003in}{2.900459in}}%
\pgfpathlineto{\pgfqpoint{7.675831in}{2.900387in}}%
\pgfpathlineto{\pgfqpoint{7.672659in}{2.900525in}}%
\pgfpathlineto{\pgfqpoint{7.669487in}{2.900211in}}%
\pgfpathlineto{\pgfqpoint{7.666315in}{2.899990in}}%
\pgfpathlineto{\pgfqpoint{7.663142in}{2.899649in}}%
\pgfpathlineto{\pgfqpoint{7.659970in}{2.899349in}}%
\pgfpathlineto{\pgfqpoint{7.656798in}{2.899186in}}%
\pgfpathlineto{\pgfqpoint{7.653626in}{2.898801in}}%
\pgfpathlineto{\pgfqpoint{7.650454in}{2.899114in}}%
\pgfpathlineto{\pgfqpoint{7.647282in}{2.899114in}}%
\pgfpathlineto{\pgfqpoint{7.644110in}{2.899117in}}%
\pgfpathlineto{\pgfqpoint{7.640938in}{2.899018in}}%
\pgfpathlineto{\pgfqpoint{7.637766in}{2.898572in}}%
\pgfpathlineto{\pgfqpoint{7.634594in}{2.898825in}}%
\pgfpathlineto{\pgfqpoint{7.631422in}{2.899038in}}%
\pgfpathlineto{\pgfqpoint{7.628250in}{2.899018in}}%
\pgfpathlineto{\pgfqpoint{7.625078in}{2.898547in}}%
\pgfpathlineto{\pgfqpoint{7.621906in}{2.898467in}}%
\pgfpathlineto{\pgfqpoint{7.618734in}{2.898144in}}%
\pgfpathlineto{\pgfqpoint{7.615562in}{2.898308in}}%
\pgfpathlineto{\pgfqpoint{7.612390in}{2.898634in}}%
\pgfpathlineto{\pgfqpoint{7.609218in}{2.898595in}}%
\pgfpathlineto{\pgfqpoint{7.606046in}{2.898413in}}%
\pgfpathlineto{\pgfqpoint{7.602874in}{2.897978in}}%
\pgfpathlineto{\pgfqpoint{7.599702in}{2.897985in}}%
\pgfpathlineto{\pgfqpoint{7.596530in}{2.898224in}}%
\pgfpathlineto{\pgfqpoint{7.593358in}{2.897886in}}%
\pgfpathlineto{\pgfqpoint{7.590186in}{2.897811in}}%
\pgfpathlineto{\pgfqpoint{7.587013in}{2.897736in}}%
\pgfpathlineto{\pgfqpoint{7.583841in}{2.897954in}}%
\pgfpathlineto{\pgfqpoint{7.580669in}{2.897669in}}%
\pgfpathlineto{\pgfqpoint{7.577497in}{2.897171in}}%
\pgfpathlineto{\pgfqpoint{7.574325in}{2.896873in}}%
\pgfpathlineto{\pgfqpoint{7.571153in}{2.896734in}}%
\pgfpathlineto{\pgfqpoint{7.567981in}{2.896364in}}%
\pgfpathlineto{\pgfqpoint{7.564809in}{2.895812in}}%
\pgfpathlineto{\pgfqpoint{7.561637in}{2.895933in}}%
\pgfpathlineto{\pgfqpoint{7.558465in}{2.896391in}}%
\pgfpathlineto{\pgfqpoint{7.555293in}{2.896616in}}%
\pgfpathlineto{\pgfqpoint{7.552121in}{2.896301in}}%
\pgfpathlineto{\pgfqpoint{7.548949in}{2.896158in}}%
\pgfpathlineto{\pgfqpoint{7.545777in}{2.896215in}}%
\pgfpathlineto{\pgfqpoint{7.542605in}{2.896249in}}%
\pgfpathlineto{\pgfqpoint{7.539433in}{2.895943in}}%
\pgfpathlineto{\pgfqpoint{7.536261in}{2.896154in}}%
\pgfpathlineto{\pgfqpoint{7.533089in}{2.895884in}}%
\pgfpathlineto{\pgfqpoint{7.529917in}{2.896033in}}%
\pgfpathlineto{\pgfqpoint{7.526745in}{2.895512in}}%
\pgfpathlineto{\pgfqpoint{7.523573in}{2.895801in}}%
\pgfpathlineto{\pgfqpoint{7.520401in}{2.895541in}}%
\pgfpathlineto{\pgfqpoint{7.517229in}{2.895475in}}%
\pgfpathlineto{\pgfqpoint{7.514057in}{2.895518in}}%
\pgfpathlineto{\pgfqpoint{7.510885in}{2.895446in}}%
\pgfpathlineto{\pgfqpoint{7.507712in}{2.895867in}}%
\pgfpathlineto{\pgfqpoint{7.504540in}{2.896174in}}%
\pgfpathlineto{\pgfqpoint{7.501368in}{2.895818in}}%
\pgfpathlineto{\pgfqpoint{7.498196in}{2.895979in}}%
\pgfpathlineto{\pgfqpoint{7.495024in}{2.895660in}}%
\pgfpathlineto{\pgfqpoint{7.491852in}{2.895659in}}%
\pgfpathlineto{\pgfqpoint{7.488680in}{2.894892in}}%
\pgfpathlineto{\pgfqpoint{7.485508in}{2.894924in}}%
\pgfpathlineto{\pgfqpoint{7.482336in}{2.894951in}}%
\pgfpathlineto{\pgfqpoint{7.479164in}{2.895081in}}%
\pgfpathlineto{\pgfqpoint{7.475992in}{2.895044in}}%
\pgfpathlineto{\pgfqpoint{7.472820in}{2.894731in}}%
\pgfpathlineto{\pgfqpoint{7.469648in}{2.895169in}}%
\pgfpathlineto{\pgfqpoint{7.466476in}{2.894899in}}%
\pgfpathlineto{\pgfqpoint{7.463304in}{2.894708in}}%
\pgfpathlineto{\pgfqpoint{7.460132in}{2.894864in}}%
\pgfpathlineto{\pgfqpoint{7.456960in}{2.895414in}}%
\pgfpathlineto{\pgfqpoint{7.453788in}{2.895392in}}%
\pgfpathlineto{\pgfqpoint{7.450616in}{2.895298in}}%
\pgfpathlineto{\pgfqpoint{7.447444in}{2.895414in}}%
\pgfpathlineto{\pgfqpoint{7.444272in}{2.895352in}}%
\pgfpathlineto{\pgfqpoint{7.441100in}{2.895538in}}%
\pgfpathlineto{\pgfqpoint{7.437928in}{2.895802in}}%
\pgfpathlineto{\pgfqpoint{7.434756in}{2.895762in}}%
\pgfpathlineto{\pgfqpoint{7.431584in}{2.895480in}}%
\pgfpathlineto{\pgfqpoint{7.428411in}{2.895305in}}%
\pgfpathlineto{\pgfqpoint{7.425239in}{2.895343in}}%
\pgfpathlineto{\pgfqpoint{7.422067in}{2.895978in}}%
\pgfpathlineto{\pgfqpoint{7.418895in}{2.895730in}}%
\pgfpathlineto{\pgfqpoint{7.415723in}{2.895341in}}%
\pgfpathlineto{\pgfqpoint{7.412551in}{2.895405in}}%
\pgfpathlineto{\pgfqpoint{7.409379in}{2.895152in}}%
\pgfpathlineto{\pgfqpoint{7.406207in}{2.895188in}}%
\pgfpathlineto{\pgfqpoint{7.403035in}{2.895381in}}%
\pgfpathlineto{\pgfqpoint{7.399863in}{2.895524in}}%
\pgfpathlineto{\pgfqpoint{7.396691in}{2.895702in}}%
\pgfpathlineto{\pgfqpoint{7.393519in}{2.895763in}}%
\pgfpathlineto{\pgfqpoint{7.390347in}{2.895887in}}%
\pgfpathlineto{\pgfqpoint{7.387175in}{2.895954in}}%
\pgfpathlineto{\pgfqpoint{7.384003in}{2.896147in}}%
\pgfpathlineto{\pgfqpoint{7.380831in}{2.896674in}}%
\pgfpathlineto{\pgfqpoint{7.377659in}{2.896231in}}%
\pgfpathlineto{\pgfqpoint{7.374487in}{2.895944in}}%
\pgfpathlineto{\pgfqpoint{7.371315in}{2.894924in}}%
\pgfpathlineto{\pgfqpoint{7.368143in}{2.894275in}}%
\pgfpathlineto{\pgfqpoint{7.364971in}{2.894035in}}%
\pgfpathlineto{\pgfqpoint{7.361799in}{2.893733in}}%
\pgfpathlineto{\pgfqpoint{7.358627in}{2.893590in}}%
\pgfpathlineto{\pgfqpoint{7.355455in}{2.893361in}}%
\pgfpathlineto{\pgfqpoint{7.352282in}{2.893081in}}%
\pgfpathlineto{\pgfqpoint{7.349110in}{2.893069in}}%
\pgfpathlineto{\pgfqpoint{7.345938in}{2.893598in}}%
\pgfpathlineto{\pgfqpoint{7.342766in}{2.893434in}}%
\pgfpathlineto{\pgfqpoint{7.339594in}{2.893380in}}%
\pgfpathlineto{\pgfqpoint{7.336422in}{2.892869in}}%
\pgfpathlineto{\pgfqpoint{7.333250in}{2.892786in}}%
\pgfpathlineto{\pgfqpoint{7.330078in}{2.893073in}}%
\pgfpathlineto{\pgfqpoint{7.326906in}{2.893149in}}%
\pgfpathlineto{\pgfqpoint{7.323734in}{2.893125in}}%
\pgfpathlineto{\pgfqpoint{7.320562in}{2.893381in}}%
\pgfpathlineto{\pgfqpoint{7.317390in}{2.893118in}}%
\pgfpathlineto{\pgfqpoint{7.314218in}{2.892626in}}%
\pgfpathlineto{\pgfqpoint{7.311046in}{2.892052in}}%
\pgfpathlineto{\pgfqpoint{7.307874in}{2.891946in}}%
\pgfpathlineto{\pgfqpoint{7.304702in}{2.892240in}}%
\pgfpathlineto{\pgfqpoint{7.301530in}{2.891977in}}%
\pgfpathlineto{\pgfqpoint{7.298358in}{2.891886in}}%
\pgfpathlineto{\pgfqpoint{7.295186in}{2.891908in}}%
\pgfpathlineto{\pgfqpoint{7.292014in}{2.892013in}}%
\pgfpathlineto{\pgfqpoint{7.288842in}{2.891797in}}%
\pgfpathlineto{\pgfqpoint{7.285670in}{2.891736in}}%
\pgfpathlineto{\pgfqpoint{7.282498in}{2.891454in}}%
\pgfpathlineto{\pgfqpoint{7.279326in}{2.891324in}}%
\pgfpathlineto{\pgfqpoint{7.276154in}{2.891391in}}%
\pgfpathlineto{\pgfqpoint{7.272981in}{2.891202in}}%
\pgfpathlineto{\pgfqpoint{7.269809in}{2.891091in}}%
\pgfpathlineto{\pgfqpoint{7.266637in}{2.891314in}}%
\pgfpathlineto{\pgfqpoint{7.263465in}{2.891397in}}%
\pgfpathlineto{\pgfqpoint{7.260293in}{2.891148in}}%
\pgfpathlineto{\pgfqpoint{7.257121in}{2.890899in}}%
\pgfpathlineto{\pgfqpoint{7.253949in}{2.890830in}}%
\pgfpathlineto{\pgfqpoint{7.250777in}{2.890797in}}%
\pgfpathlineto{\pgfqpoint{7.247605in}{2.890652in}}%
\pgfpathlineto{\pgfqpoint{7.244433in}{2.890651in}}%
\pgfpathlineto{\pgfqpoint{7.241261in}{2.890244in}}%
\pgfpathlineto{\pgfqpoint{7.238089in}{2.890209in}}%
\pgfpathlineto{\pgfqpoint{7.234917in}{2.890201in}}%
\pgfpathlineto{\pgfqpoint{7.231745in}{2.889957in}}%
\pgfpathlineto{\pgfqpoint{7.228573in}{2.889866in}}%
\pgfpathlineto{\pgfqpoint{7.225401in}{2.890103in}}%
\pgfpathlineto{\pgfqpoint{7.222229in}{2.890075in}}%
\pgfpathlineto{\pgfqpoint{7.219057in}{2.890375in}}%
\pgfpathlineto{\pgfqpoint{7.215885in}{2.890346in}}%
\pgfpathlineto{\pgfqpoint{7.212713in}{2.890295in}}%
\pgfpathlineto{\pgfqpoint{7.209541in}{2.890503in}}%
\pgfpathlineto{\pgfqpoint{7.206369in}{2.890108in}}%
\pgfpathlineto{\pgfqpoint{7.203197in}{2.889930in}}%
\pgfpathlineto{\pgfqpoint{7.200025in}{2.889546in}}%
\pgfpathlineto{\pgfqpoint{7.196853in}{2.889365in}}%
\pgfpathlineto{\pgfqpoint{7.193680in}{2.889482in}}%
\pgfpathlineto{\pgfqpoint{7.190508in}{2.889122in}}%
\pgfpathlineto{\pgfqpoint{7.187336in}{2.889188in}}%
\pgfpathlineto{\pgfqpoint{7.184164in}{2.889027in}}%
\pgfpathlineto{\pgfqpoint{7.180992in}{2.888373in}}%
\pgfpathlineto{\pgfqpoint{7.177820in}{2.887966in}}%
\pgfpathlineto{\pgfqpoint{7.174648in}{2.888008in}}%
\pgfpathlineto{\pgfqpoint{7.171476in}{2.887922in}}%
\pgfpathlineto{\pgfqpoint{7.168304in}{2.888446in}}%
\pgfpathlineto{\pgfqpoint{7.165132in}{2.888890in}}%
\pgfpathlineto{\pgfqpoint{7.161960in}{2.888838in}}%
\pgfpathlineto{\pgfqpoint{7.158788in}{2.889076in}}%
\pgfpathlineto{\pgfqpoint{7.155616in}{2.888781in}}%
\pgfpathlineto{\pgfqpoint{7.152444in}{2.889213in}}%
\pgfpathlineto{\pgfqpoint{7.149272in}{2.889181in}}%
\pgfpathlineto{\pgfqpoint{7.146100in}{2.889497in}}%
\pgfpathlineto{\pgfqpoint{7.142928in}{2.889638in}}%
\pgfpathlineto{\pgfqpoint{7.139756in}{2.889922in}}%
\pgfpathlineto{\pgfqpoint{7.136584in}{2.889668in}}%
\pgfpathlineto{\pgfqpoint{7.133412in}{2.889374in}}%
\pgfpathlineto{\pgfqpoint{7.130240in}{2.889387in}}%
\pgfpathlineto{\pgfqpoint{7.127068in}{2.889417in}}%
\pgfpathlineto{\pgfqpoint{7.123896in}{2.889178in}}%
\pgfpathlineto{\pgfqpoint{7.120724in}{2.889217in}}%
\pgfpathlineto{\pgfqpoint{7.117551in}{2.888999in}}%
\pgfpathlineto{\pgfqpoint{7.114379in}{2.889039in}}%
\pgfpathlineto{\pgfqpoint{7.111207in}{2.888610in}}%
\pgfpathlineto{\pgfqpoint{7.108035in}{2.888699in}}%
\pgfpathlineto{\pgfqpoint{7.104863in}{2.888566in}}%
\pgfpathlineto{\pgfqpoint{7.101691in}{2.888411in}}%
\pgfpathlineto{\pgfqpoint{7.098519in}{2.888231in}}%
\pgfpathlineto{\pgfqpoint{7.095347in}{2.888474in}}%
\pgfpathlineto{\pgfqpoint{7.092175in}{2.888193in}}%
\pgfpathlineto{\pgfqpoint{7.089003in}{2.888222in}}%
\pgfpathlineto{\pgfqpoint{7.085831in}{2.888222in}}%
\pgfpathlineto{\pgfqpoint{7.082659in}{2.888392in}}%
\pgfpathlineto{\pgfqpoint{7.079487in}{2.888332in}}%
\pgfpathlineto{\pgfqpoint{7.076315in}{2.887976in}}%
\pgfpathlineto{\pgfqpoint{7.073143in}{2.887441in}}%
\pgfpathlineto{\pgfqpoint{7.069971in}{2.887135in}}%
\pgfpathlineto{\pgfqpoint{7.066799in}{2.887155in}}%
\pgfpathlineto{\pgfqpoint{7.063627in}{2.886505in}}%
\pgfpathlineto{\pgfqpoint{7.060455in}{2.886657in}}%
\pgfpathlineto{\pgfqpoint{7.057283in}{2.886665in}}%
\pgfpathlineto{\pgfqpoint{7.054111in}{2.886535in}}%
\pgfpathlineto{\pgfqpoint{7.050939in}{2.886911in}}%
\pgfpathlineto{\pgfqpoint{7.047767in}{2.886802in}}%
\pgfpathlineto{\pgfqpoint{7.044595in}{2.886893in}}%
\pgfpathlineto{\pgfqpoint{7.041423in}{2.887174in}}%
\pgfpathlineto{\pgfqpoint{7.038250in}{2.887567in}}%
\pgfpathlineto{\pgfqpoint{7.035078in}{2.887385in}}%
\pgfpathlineto{\pgfqpoint{7.031906in}{2.887148in}}%
\pgfpathlineto{\pgfqpoint{7.028734in}{2.886913in}}%
\pgfpathlineto{\pgfqpoint{7.025562in}{2.886666in}}%
\pgfpathlineto{\pgfqpoint{7.022390in}{2.886568in}}%
\pgfpathlineto{\pgfqpoint{7.019218in}{2.886341in}}%
\pgfpathlineto{\pgfqpoint{7.016046in}{2.886754in}}%
\pgfpathlineto{\pgfqpoint{7.012874in}{2.886989in}}%
\pgfpathlineto{\pgfqpoint{7.009702in}{2.887078in}}%
\pgfpathlineto{\pgfqpoint{7.006530in}{2.887358in}}%
\pgfpathlineto{\pgfqpoint{7.003358in}{2.886899in}}%
\pgfpathlineto{\pgfqpoint{7.000186in}{2.887291in}}%
\pgfpathlineto{\pgfqpoint{6.997014in}{2.887160in}}%
\pgfpathlineto{\pgfqpoint{6.993842in}{2.887354in}}%
\pgfpathlineto{\pgfqpoint{6.990670in}{2.887626in}}%
\pgfpathlineto{\pgfqpoint{6.987498in}{2.887282in}}%
\pgfpathlineto{\pgfqpoint{6.984326in}{2.887623in}}%
\pgfpathlineto{\pgfqpoint{6.981154in}{2.887305in}}%
\pgfpathlineto{\pgfqpoint{6.977982in}{2.887239in}}%
\pgfpathlineto{\pgfqpoint{6.974810in}{2.887288in}}%
\pgfpathlineto{\pgfqpoint{6.971638in}{2.887685in}}%
\pgfpathlineto{\pgfqpoint{6.968466in}{2.887691in}}%
\pgfpathlineto{\pgfqpoint{6.965294in}{2.887263in}}%
\pgfpathlineto{\pgfqpoint{6.962122in}{2.886981in}}%
\pgfpathlineto{\pgfqpoint{6.958949in}{2.886790in}}%
\pgfpathlineto{\pgfqpoint{6.955777in}{2.886440in}}%
\pgfpathlineto{\pgfqpoint{6.952605in}{2.885975in}}%
\pgfpathlineto{\pgfqpoint{6.949433in}{2.885709in}}%
\pgfpathlineto{\pgfqpoint{6.946261in}{2.885612in}}%
\pgfpathlineto{\pgfqpoint{6.943089in}{2.885546in}}%
\pgfpathlineto{\pgfqpoint{6.939917in}{2.885308in}}%
\pgfpathlineto{\pgfqpoint{6.936745in}{2.885723in}}%
\pgfpathlineto{\pgfqpoint{6.933573in}{2.885818in}}%
\pgfpathlineto{\pgfqpoint{6.930401in}{2.885860in}}%
\pgfpathlineto{\pgfqpoint{6.927229in}{2.886153in}}%
\pgfpathlineto{\pgfqpoint{6.924057in}{2.885881in}}%
\pgfpathlineto{\pgfqpoint{6.920885in}{2.885610in}}%
\pgfpathlineto{\pgfqpoint{6.917713in}{2.885665in}}%
\pgfpathlineto{\pgfqpoint{6.914541in}{2.885405in}}%
\pgfpathlineto{\pgfqpoint{6.911369in}{2.885166in}}%
\pgfpathlineto{\pgfqpoint{6.908197in}{2.885102in}}%
\pgfpathlineto{\pgfqpoint{6.905025in}{2.885199in}}%
\pgfpathlineto{\pgfqpoint{6.901853in}{2.885189in}}%
\pgfpathlineto{\pgfqpoint{6.898681in}{2.885025in}}%
\pgfpathlineto{\pgfqpoint{6.895509in}{2.884591in}}%
\pgfpathlineto{\pgfqpoint{6.892337in}{2.884602in}}%
\pgfpathlineto{\pgfqpoint{6.889165in}{2.884388in}}%
\pgfpathlineto{\pgfqpoint{6.885993in}{2.884047in}}%
\pgfpathlineto{\pgfqpoint{6.882820in}{2.883818in}}%
\pgfpathlineto{\pgfqpoint{6.879648in}{2.883787in}}%
\pgfpathlineto{\pgfqpoint{6.876476in}{2.883544in}}%
\pgfpathlineto{\pgfqpoint{6.873304in}{2.883322in}}%
\pgfpathlineto{\pgfqpoint{6.870132in}{2.882787in}}%
\pgfpathlineto{\pgfqpoint{6.866960in}{2.882309in}}%
\pgfpathlineto{\pgfqpoint{6.863788in}{2.882436in}}%
\pgfpathlineto{\pgfqpoint{6.860616in}{2.882537in}}%
\pgfpathlineto{\pgfqpoint{6.857444in}{2.882839in}}%
\pgfpathlineto{\pgfqpoint{6.854272in}{2.882825in}}%
\pgfpathlineto{\pgfqpoint{6.851100in}{2.882757in}}%
\pgfpathlineto{\pgfqpoint{6.847928in}{2.882395in}}%
\pgfpathlineto{\pgfqpoint{6.844756in}{2.882159in}}%
\pgfpathlineto{\pgfqpoint{6.841584in}{2.881604in}}%
\pgfpathlineto{\pgfqpoint{6.838412in}{2.881788in}}%
\pgfpathlineto{\pgfqpoint{6.835240in}{2.881662in}}%
\pgfpathlineto{\pgfqpoint{6.832068in}{2.881516in}}%
\pgfpathlineto{\pgfqpoint{6.828896in}{2.881196in}}%
\pgfpathlineto{\pgfqpoint{6.825724in}{2.881143in}}%
\pgfpathlineto{\pgfqpoint{6.822552in}{2.881067in}}%
\pgfpathlineto{\pgfqpoint{6.819380in}{2.880847in}}%
\pgfpathlineto{\pgfqpoint{6.816208in}{2.881562in}}%
\pgfpathlineto{\pgfqpoint{6.813036in}{2.880795in}}%
\pgfpathlineto{\pgfqpoint{6.809864in}{2.880590in}}%
\pgfpathlineto{\pgfqpoint{6.806692in}{2.880615in}}%
\pgfpathlineto{\pgfqpoint{6.803519in}{2.880680in}}%
\pgfpathlineto{\pgfqpoint{6.800347in}{2.880513in}}%
\pgfpathlineto{\pgfqpoint{6.797175in}{2.880084in}}%
\pgfpathlineto{\pgfqpoint{6.794003in}{2.879685in}}%
\pgfpathlineto{\pgfqpoint{6.790831in}{2.879798in}}%
\pgfpathlineto{\pgfqpoint{6.787659in}{2.879272in}}%
\pgfpathlineto{\pgfqpoint{6.784487in}{2.879165in}}%
\pgfpathlineto{\pgfqpoint{6.781315in}{2.879401in}}%
\pgfpathlineto{\pgfqpoint{6.778143in}{2.879196in}}%
\pgfpathlineto{\pgfqpoint{6.774971in}{2.879585in}}%
\pgfpathlineto{\pgfqpoint{6.771799in}{2.879487in}}%
\pgfpathlineto{\pgfqpoint{6.768627in}{2.879087in}}%
\pgfpathlineto{\pgfqpoint{6.765455in}{2.878998in}}%
\pgfpathlineto{\pgfqpoint{6.762283in}{2.878604in}}%
\pgfpathlineto{\pgfqpoint{6.759111in}{2.878698in}}%
\pgfpathlineto{\pgfqpoint{6.755939in}{2.878140in}}%
\pgfpathlineto{\pgfqpoint{6.752767in}{2.878059in}}%
\pgfpathlineto{\pgfqpoint{6.749595in}{2.878107in}}%
\pgfpathlineto{\pgfqpoint{6.746423in}{2.878326in}}%
\pgfpathlineto{\pgfqpoint{6.743251in}{2.878557in}}%
\pgfpathlineto{\pgfqpoint{6.740079in}{2.878391in}}%
\pgfpathlineto{\pgfqpoint{6.736907in}{2.878690in}}%
\pgfpathlineto{\pgfqpoint{6.733735in}{2.878287in}}%
\pgfpathlineto{\pgfqpoint{6.730563in}{2.878149in}}%
\pgfpathlineto{\pgfqpoint{6.727391in}{2.878372in}}%
\pgfpathlineto{\pgfqpoint{6.724218in}{2.878408in}}%
\pgfpathlineto{\pgfqpoint{6.721046in}{2.878543in}}%
\pgfpathlineto{\pgfqpoint{6.717874in}{2.878379in}}%
\pgfpathlineto{\pgfqpoint{6.714702in}{2.878221in}}%
\pgfpathlineto{\pgfqpoint{6.711530in}{2.877883in}}%
\pgfpathlineto{\pgfqpoint{6.708358in}{2.877636in}}%
\pgfpathlineto{\pgfqpoint{6.705186in}{2.877579in}}%
\pgfpathlineto{\pgfqpoint{6.702014in}{2.878023in}}%
\pgfpathlineto{\pgfqpoint{6.698842in}{2.877647in}}%
\pgfpathlineto{\pgfqpoint{6.695670in}{2.877839in}}%
\pgfpathlineto{\pgfqpoint{6.692498in}{2.877801in}}%
\pgfpathlineto{\pgfqpoint{6.689326in}{2.877315in}}%
\pgfpathlineto{\pgfqpoint{6.686154in}{2.877558in}}%
\pgfpathlineto{\pgfqpoint{6.682982in}{2.877477in}}%
\pgfpathlineto{\pgfqpoint{6.679810in}{2.877780in}}%
\pgfpathlineto{\pgfqpoint{6.676638in}{2.877501in}}%
\pgfpathlineto{\pgfqpoint{6.673466in}{2.877175in}}%
\pgfpathlineto{\pgfqpoint{6.670294in}{2.877045in}}%
\pgfpathlineto{\pgfqpoint{6.667122in}{2.877196in}}%
\pgfpathlineto{\pgfqpoint{6.663950in}{2.876897in}}%
\pgfpathlineto{\pgfqpoint{6.660778in}{2.876919in}}%
\pgfpathlineto{\pgfqpoint{6.657606in}{2.877374in}}%
\pgfpathlineto{\pgfqpoint{6.654434in}{2.877097in}}%
\pgfpathlineto{\pgfqpoint{6.651262in}{2.876744in}}%
\pgfpathlineto{\pgfqpoint{6.648089in}{2.876850in}}%
\pgfpathlineto{\pgfqpoint{6.644917in}{2.876830in}}%
\pgfpathlineto{\pgfqpoint{6.641745in}{2.876963in}}%
\pgfpathlineto{\pgfqpoint{6.638573in}{2.877164in}}%
\pgfpathlineto{\pgfqpoint{6.635401in}{2.877206in}}%
\pgfpathlineto{\pgfqpoint{6.632229in}{2.876895in}}%
\pgfpathlineto{\pgfqpoint{6.629057in}{2.876350in}}%
\pgfpathlineto{\pgfqpoint{6.625885in}{2.877080in}}%
\pgfpathlineto{\pgfqpoint{6.622713in}{2.876915in}}%
\pgfpathlineto{\pgfqpoint{6.619541in}{2.877236in}}%
\pgfpathlineto{\pgfqpoint{6.616369in}{2.877062in}}%
\pgfpathlineto{\pgfqpoint{6.613197in}{2.877521in}}%
\pgfpathlineto{\pgfqpoint{6.610025in}{2.877906in}}%
\pgfpathlineto{\pgfqpoint{6.606853in}{2.878044in}}%
\pgfpathlineto{\pgfqpoint{6.603681in}{2.878016in}}%
\pgfpathlineto{\pgfqpoint{6.600509in}{2.878089in}}%
\pgfpathlineto{\pgfqpoint{6.597337in}{2.878340in}}%
\pgfpathlineto{\pgfqpoint{6.594165in}{2.878385in}}%
\pgfpathlineto{\pgfqpoint{6.590993in}{2.878087in}}%
\pgfpathlineto{\pgfqpoint{6.587821in}{2.877979in}}%
\pgfpathlineto{\pgfqpoint{6.584649in}{2.877574in}}%
\pgfpathlineto{\pgfqpoint{6.581477in}{2.877577in}}%
\pgfpathlineto{\pgfqpoint{6.578305in}{2.877709in}}%
\pgfpathlineto{\pgfqpoint{6.575133in}{2.877879in}}%
\pgfpathlineto{\pgfqpoint{6.571961in}{2.877506in}}%
\pgfpathlineto{\pgfqpoint{6.568788in}{2.877131in}}%
\pgfpathlineto{\pgfqpoint{6.565616in}{2.876877in}}%
\pgfpathlineto{\pgfqpoint{6.562444in}{2.876476in}}%
\pgfpathlineto{\pgfqpoint{6.559272in}{2.875834in}}%
\pgfpathlineto{\pgfqpoint{6.556100in}{2.875797in}}%
\pgfpathlineto{\pgfqpoint{6.552928in}{2.875363in}}%
\pgfpathlineto{\pgfqpoint{6.549756in}{2.874943in}}%
\pgfpathlineto{\pgfqpoint{6.546584in}{2.874389in}}%
\pgfpathlineto{\pgfqpoint{6.543412in}{2.874684in}}%
\pgfpathlineto{\pgfqpoint{6.540240in}{2.874962in}}%
\pgfpathlineto{\pgfqpoint{6.537068in}{2.874924in}}%
\pgfpathlineto{\pgfqpoint{6.533896in}{2.874713in}}%
\pgfpathlineto{\pgfqpoint{6.530724in}{2.874795in}}%
\pgfpathlineto{\pgfqpoint{6.527552in}{2.874615in}}%
\pgfpathlineto{\pgfqpoint{6.524380in}{2.874659in}}%
\pgfpathlineto{\pgfqpoint{6.521208in}{2.874762in}}%
\pgfpathlineto{\pgfqpoint{6.518036in}{2.874759in}}%
\pgfpathlineto{\pgfqpoint{6.514864in}{2.874382in}}%
\pgfpathlineto{\pgfqpoint{6.511692in}{2.874574in}}%
\pgfpathlineto{\pgfqpoint{6.508520in}{2.874427in}}%
\pgfpathlineto{\pgfqpoint{6.505348in}{2.874233in}}%
\pgfpathlineto{\pgfqpoint{6.502176in}{2.873923in}}%
\pgfpathlineto{\pgfqpoint{6.499004in}{2.873774in}}%
\pgfpathlineto{\pgfqpoint{6.495832in}{2.873424in}}%
\pgfpathlineto{\pgfqpoint{6.492659in}{2.873078in}}%
\pgfpathlineto{\pgfqpoint{6.489487in}{2.872868in}}%
\pgfpathlineto{\pgfqpoint{6.486315in}{2.872983in}}%
\pgfpathlineto{\pgfqpoint{6.483143in}{2.873136in}}%
\pgfpathlineto{\pgfqpoint{6.479971in}{2.872885in}}%
\pgfpathlineto{\pgfqpoint{6.476799in}{2.873228in}}%
\pgfpathlineto{\pgfqpoint{6.473627in}{2.873137in}}%
\pgfpathlineto{\pgfqpoint{6.470455in}{2.872829in}}%
\pgfpathlineto{\pgfqpoint{6.467283in}{2.872262in}}%
\pgfpathlineto{\pgfqpoint{6.464111in}{2.871826in}}%
\pgfpathlineto{\pgfqpoint{6.460939in}{2.871593in}}%
\pgfpathlineto{\pgfqpoint{6.457767in}{2.871734in}}%
\pgfpathlineto{\pgfqpoint{6.454595in}{2.871603in}}%
\pgfpathlineto{\pgfqpoint{6.451423in}{2.871956in}}%
\pgfpathlineto{\pgfqpoint{6.448251in}{2.871927in}}%
\pgfpathlineto{\pgfqpoint{6.445079in}{2.871007in}}%
\pgfpathlineto{\pgfqpoint{6.441907in}{2.870341in}}%
\pgfpathlineto{\pgfqpoint{6.438735in}{2.870288in}}%
\pgfpathlineto{\pgfqpoint{6.435563in}{2.869669in}}%
\pgfpathlineto{\pgfqpoint{6.432391in}{2.869662in}}%
\pgfpathlineto{\pgfqpoint{6.429219in}{2.869841in}}%
\pgfpathlineto{\pgfqpoint{6.426047in}{2.870479in}}%
\pgfpathlineto{\pgfqpoint{6.422875in}{2.870671in}}%
\pgfpathlineto{\pgfqpoint{6.419703in}{2.870335in}}%
\pgfpathlineto{\pgfqpoint{6.416531in}{2.870531in}}%
\pgfpathlineto{\pgfqpoint{6.413358in}{2.870548in}}%
\pgfpathlineto{\pgfqpoint{6.410186in}{2.870523in}}%
\pgfpathlineto{\pgfqpoint{6.407014in}{2.870867in}}%
\pgfpathlineto{\pgfqpoint{6.403842in}{2.870849in}}%
\pgfpathlineto{\pgfqpoint{6.400670in}{2.870718in}}%
\pgfpathlineto{\pgfqpoint{6.397498in}{2.870418in}}%
\pgfpathlineto{\pgfqpoint{6.394326in}{2.870028in}}%
\pgfpathlineto{\pgfqpoint{6.391154in}{2.870096in}}%
\pgfpathlineto{\pgfqpoint{6.387982in}{2.869867in}}%
\pgfpathlineto{\pgfqpoint{6.384810in}{2.869730in}}%
\pgfpathlineto{\pgfqpoint{6.381638in}{2.869527in}}%
\pgfpathlineto{\pgfqpoint{6.378466in}{2.869143in}}%
\pgfpathlineto{\pgfqpoint{6.375294in}{2.869181in}}%
\pgfpathlineto{\pgfqpoint{6.372122in}{2.868717in}}%
\pgfpathlineto{\pgfqpoint{6.368950in}{2.868565in}}%
\pgfpathlineto{\pgfqpoint{6.365778in}{2.868604in}}%
\pgfpathlineto{\pgfqpoint{6.362606in}{2.868226in}}%
\pgfpathlineto{\pgfqpoint{6.359434in}{2.868756in}}%
\pgfpathlineto{\pgfqpoint{6.356262in}{2.868692in}}%
\pgfpathlineto{\pgfqpoint{6.353090in}{2.868834in}}%
\pgfpathlineto{\pgfqpoint{6.349918in}{2.868597in}}%
\pgfpathlineto{\pgfqpoint{6.346746in}{2.868609in}}%
\pgfpathlineto{\pgfqpoint{6.343574in}{2.868529in}}%
\pgfpathlineto{\pgfqpoint{6.340402in}{2.868383in}}%
\pgfpathlineto{\pgfqpoint{6.337230in}{2.868385in}}%
\pgfpathlineto{\pgfqpoint{6.334057in}{2.868265in}}%
\pgfpathlineto{\pgfqpoint{6.330885in}{2.867969in}}%
\pgfpathlineto{\pgfqpoint{6.327713in}{2.867691in}}%
\pgfpathlineto{\pgfqpoint{6.324541in}{2.867564in}}%
\pgfpathlineto{\pgfqpoint{6.321369in}{2.867609in}}%
\pgfpathlineto{\pgfqpoint{6.318197in}{2.867697in}}%
\pgfpathlineto{\pgfqpoint{6.315025in}{2.867502in}}%
\pgfpathlineto{\pgfqpoint{6.311853in}{2.867559in}}%
\pgfpathlineto{\pgfqpoint{6.308681in}{2.867284in}}%
\pgfpathlineto{\pgfqpoint{6.305509in}{2.867420in}}%
\pgfpathlineto{\pgfqpoint{6.302337in}{2.867554in}}%
\pgfpathlineto{\pgfqpoint{6.299165in}{2.867519in}}%
\pgfpathlineto{\pgfqpoint{6.295993in}{2.867195in}}%
\pgfpathlineto{\pgfqpoint{6.292821in}{2.867196in}}%
\pgfpathlineto{\pgfqpoint{6.289649in}{2.866666in}}%
\pgfpathlineto{\pgfqpoint{6.286477in}{2.866923in}}%
\pgfpathlineto{\pgfqpoint{6.283305in}{2.867326in}}%
\pgfpathlineto{\pgfqpoint{6.280133in}{2.866832in}}%
\pgfpathlineto{\pgfqpoint{6.276961in}{2.866660in}}%
\pgfpathlineto{\pgfqpoint{6.273789in}{2.866398in}}%
\pgfpathlineto{\pgfqpoint{6.270617in}{2.865930in}}%
\pgfpathlineto{\pgfqpoint{6.267445in}{2.866324in}}%
\pgfpathlineto{\pgfqpoint{6.264273in}{2.866087in}}%
\pgfpathlineto{\pgfqpoint{6.261101in}{2.865592in}}%
\pgfpathlineto{\pgfqpoint{6.257928in}{2.865067in}}%
\pgfpathlineto{\pgfqpoint{6.254756in}{2.864905in}}%
\pgfpathlineto{\pgfqpoint{6.251584in}{2.864624in}}%
\pgfpathlineto{\pgfqpoint{6.248412in}{2.864025in}}%
\pgfpathlineto{\pgfqpoint{6.245240in}{2.863872in}}%
\pgfpathlineto{\pgfqpoint{6.242068in}{2.863969in}}%
\pgfpathlineto{\pgfqpoint{6.238896in}{2.863921in}}%
\pgfpathlineto{\pgfqpoint{6.235724in}{2.863926in}}%
\pgfpathlineto{\pgfqpoint{6.232552in}{2.863402in}}%
\pgfpathlineto{\pgfqpoint{6.229380in}{2.863396in}}%
\pgfpathlineto{\pgfqpoint{6.226208in}{2.862752in}}%
\pgfpathlineto{\pgfqpoint{6.223036in}{2.862733in}}%
\pgfpathlineto{\pgfqpoint{6.219864in}{2.862779in}}%
\pgfpathlineto{\pgfqpoint{6.216692in}{2.862531in}}%
\pgfpathlineto{\pgfqpoint{6.213520in}{2.862499in}}%
\pgfpathlineto{\pgfqpoint{6.210348in}{2.862052in}}%
\pgfpathlineto{\pgfqpoint{6.207176in}{2.861609in}}%
\pgfpathlineto{\pgfqpoint{6.204004in}{2.861462in}}%
\pgfpathlineto{\pgfqpoint{6.200832in}{2.861456in}}%
\pgfpathlineto{\pgfqpoint{6.197660in}{2.861221in}}%
\pgfpathlineto{\pgfqpoint{6.194488in}{2.861176in}}%
\pgfpathlineto{\pgfqpoint{6.191316in}{2.861257in}}%
\pgfpathlineto{\pgfqpoint{6.188144in}{2.861138in}}%
\pgfpathlineto{\pgfqpoint{6.184972in}{2.861175in}}%
\pgfpathlineto{\pgfqpoint{6.181800in}{2.861475in}}%
\pgfpathlineto{\pgfqpoint{6.178627in}{2.861321in}}%
\pgfpathlineto{\pgfqpoint{6.175455in}{2.861633in}}%
\pgfpathlineto{\pgfqpoint{6.172283in}{2.861563in}}%
\pgfpathlineto{\pgfqpoint{6.169111in}{2.861378in}}%
\pgfpathlineto{\pgfqpoint{6.165939in}{2.861023in}}%
\pgfpathlineto{\pgfqpoint{6.162767in}{2.860767in}}%
\pgfpathlineto{\pgfqpoint{6.159595in}{2.860685in}}%
\pgfpathlineto{\pgfqpoint{6.156423in}{2.860714in}}%
\pgfpathlineto{\pgfqpoint{6.153251in}{2.860503in}}%
\pgfpathlineto{\pgfqpoint{6.150079in}{2.860678in}}%
\pgfpathlineto{\pgfqpoint{6.146907in}{2.860960in}}%
\pgfpathlineto{\pgfqpoint{6.143735in}{2.861084in}}%
\pgfpathlineto{\pgfqpoint{6.140563in}{2.861004in}}%
\pgfpathlineto{\pgfqpoint{6.137391in}{2.860857in}}%
\pgfpathlineto{\pgfqpoint{6.134219in}{2.860614in}}%
\pgfpathlineto{\pgfqpoint{6.131047in}{2.860647in}}%
\pgfpathlineto{\pgfqpoint{6.127875in}{2.860967in}}%
\pgfpathlineto{\pgfqpoint{6.124703in}{2.861193in}}%
\pgfpathlineto{\pgfqpoint{6.121531in}{2.860847in}}%
\pgfpathlineto{\pgfqpoint{6.118359in}{2.860799in}}%
\pgfpathlineto{\pgfqpoint{6.115187in}{2.860418in}}%
\pgfpathlineto{\pgfqpoint{6.112015in}{2.860297in}}%
\pgfpathlineto{\pgfqpoint{6.108843in}{2.860296in}}%
\pgfpathlineto{\pgfqpoint{6.105671in}{2.860575in}}%
\pgfpathlineto{\pgfqpoint{6.102499in}{2.860760in}}%
\pgfpathlineto{\pgfqpoint{6.099326in}{2.861064in}}%
\pgfpathlineto{\pgfqpoint{6.096154in}{2.860900in}}%
\pgfpathlineto{\pgfqpoint{6.092982in}{2.860302in}}%
\pgfpathlineto{\pgfqpoint{6.089810in}{2.860504in}}%
\pgfpathlineto{\pgfqpoint{6.086638in}{2.860795in}}%
\pgfpathlineto{\pgfqpoint{6.083466in}{2.861206in}}%
\pgfpathlineto{\pgfqpoint{6.080294in}{2.861284in}}%
\pgfpathlineto{\pgfqpoint{6.077122in}{2.860996in}}%
\pgfpathlineto{\pgfqpoint{6.073950in}{2.860895in}}%
\pgfpathlineto{\pgfqpoint{6.070778in}{2.860592in}}%
\pgfpathlineto{\pgfqpoint{6.067606in}{2.860702in}}%
\pgfpathlineto{\pgfqpoint{6.064434in}{2.860422in}}%
\pgfpathlineto{\pgfqpoint{6.061262in}{2.860995in}}%
\pgfpathlineto{\pgfqpoint{6.058090in}{2.860317in}}%
\pgfpathlineto{\pgfqpoint{6.054918in}{2.860041in}}%
\pgfpathlineto{\pgfqpoint{6.051746in}{2.860020in}}%
\pgfpathlineto{\pgfqpoint{6.048574in}{2.860277in}}%
\pgfpathlineto{\pgfqpoint{6.045402in}{2.859809in}}%
\pgfpathlineto{\pgfqpoint{6.042230in}{2.859905in}}%
\pgfpathlineto{\pgfqpoint{6.039058in}{2.859784in}}%
\pgfpathlineto{\pgfqpoint{6.035886in}{2.859883in}}%
\pgfpathlineto{\pgfqpoint{6.032714in}{2.859521in}}%
\pgfpathlineto{\pgfqpoint{6.029542in}{2.859238in}}%
\pgfpathlineto{\pgfqpoint{6.026370in}{2.859144in}}%
\pgfpathlineto{\pgfqpoint{6.023197in}{2.859462in}}%
\pgfpathlineto{\pgfqpoint{6.020025in}{2.859533in}}%
\pgfpathlineto{\pgfqpoint{6.016853in}{2.859215in}}%
\pgfpathlineto{\pgfqpoint{6.013681in}{2.859141in}}%
\pgfpathlineto{\pgfqpoint{6.010509in}{2.858896in}}%
\pgfpathlineto{\pgfqpoint{6.007337in}{2.859098in}}%
\pgfpathlineto{\pgfqpoint{6.004165in}{2.858978in}}%
\pgfpathlineto{\pgfqpoint{6.000993in}{2.858762in}}%
\pgfpathlineto{\pgfqpoint{5.997821in}{2.858660in}}%
\pgfpathlineto{\pgfqpoint{5.994649in}{2.859448in}}%
\pgfpathlineto{\pgfqpoint{5.991477in}{2.859613in}}%
\pgfpathlineto{\pgfqpoint{5.988305in}{2.860231in}}%
\pgfpathlineto{\pgfqpoint{5.985133in}{2.860448in}}%
\pgfpathlineto{\pgfqpoint{5.981961in}{2.860346in}}%
\pgfpathlineto{\pgfqpoint{5.978789in}{2.860231in}}%
\pgfpathlineto{\pgfqpoint{5.975617in}{2.859988in}}%
\pgfpathlineto{\pgfqpoint{5.972445in}{2.860588in}}%
\pgfpathlineto{\pgfqpoint{5.969273in}{2.860813in}}%
\pgfpathlineto{\pgfqpoint{5.966101in}{2.860620in}}%
\pgfpathlineto{\pgfqpoint{5.962929in}{2.860676in}}%
\pgfpathlineto{\pgfqpoint{5.959757in}{2.860933in}}%
\pgfpathlineto{\pgfqpoint{5.956585in}{2.860779in}}%
\pgfpathlineto{\pgfqpoint{5.953413in}{2.860769in}}%
\pgfpathlineto{\pgfqpoint{5.950241in}{2.860589in}}%
\pgfpathlineto{\pgfqpoint{5.947069in}{2.860693in}}%
\pgfpathlineto{\pgfqpoint{5.943896in}{2.860416in}}%
\pgfpathlineto{\pgfqpoint{5.940724in}{2.859720in}}%
\pgfpathlineto{\pgfqpoint{5.937552in}{2.860204in}}%
\pgfpathlineto{\pgfqpoint{5.934380in}{2.860234in}}%
\pgfpathlineto{\pgfqpoint{5.931208in}{2.860300in}}%
\pgfpathlineto{\pgfqpoint{5.928036in}{2.859446in}}%
\pgfpathlineto{\pgfqpoint{5.924864in}{2.859308in}}%
\pgfpathlineto{\pgfqpoint{5.921692in}{2.859498in}}%
\pgfpathlineto{\pgfqpoint{5.918520in}{2.859443in}}%
\pgfpathlineto{\pgfqpoint{5.915348in}{2.859096in}}%
\pgfpathlineto{\pgfqpoint{5.912176in}{2.859267in}}%
\pgfpathlineto{\pgfqpoint{5.909004in}{2.859534in}}%
\pgfpathlineto{\pgfqpoint{5.905832in}{2.858854in}}%
\pgfpathlineto{\pgfqpoint{5.902660in}{2.858750in}}%
\pgfpathlineto{\pgfqpoint{5.899488in}{2.858326in}}%
\pgfpathlineto{\pgfqpoint{5.896316in}{2.858045in}}%
\pgfpathlineto{\pgfqpoint{5.893144in}{2.857746in}}%
\pgfpathlineto{\pgfqpoint{5.889972in}{2.857969in}}%
\pgfpathlineto{\pgfqpoint{5.886800in}{2.857934in}}%
\pgfpathlineto{\pgfqpoint{5.883628in}{2.858517in}}%
\pgfpathlineto{\pgfqpoint{5.880456in}{2.858425in}}%
\pgfpathlineto{\pgfqpoint{5.877284in}{2.859156in}}%
\pgfpathlineto{\pgfqpoint{5.874112in}{2.859150in}}%
\pgfpathlineto{\pgfqpoint{5.870940in}{2.859061in}}%
\pgfpathlineto{\pgfqpoint{5.867768in}{2.858807in}}%
\pgfpathlineto{\pgfqpoint{5.864595in}{2.858487in}}%
\pgfpathlineto{\pgfqpoint{5.861423in}{2.858300in}}%
\pgfpathlineto{\pgfqpoint{5.858251in}{2.858341in}}%
\pgfpathlineto{\pgfqpoint{5.855079in}{2.858202in}}%
\pgfpathlineto{\pgfqpoint{5.851907in}{2.858053in}}%
\pgfpathlineto{\pgfqpoint{5.848735in}{2.858271in}}%
\pgfpathlineto{\pgfqpoint{5.845563in}{2.858864in}}%
\pgfpathlineto{\pgfqpoint{5.842391in}{2.858582in}}%
\pgfpathlineto{\pgfqpoint{5.839219in}{2.858421in}}%
\pgfpathlineto{\pgfqpoint{5.836047in}{2.858891in}}%
\pgfpathlineto{\pgfqpoint{5.832875in}{2.859291in}}%
\pgfpathlineto{\pgfqpoint{5.829703in}{2.859420in}}%
\pgfpathlineto{\pgfqpoint{5.826531in}{2.859412in}}%
\pgfpathlineto{\pgfqpoint{5.823359in}{2.859140in}}%
\pgfpathlineto{\pgfqpoint{5.820187in}{2.859053in}}%
\pgfpathlineto{\pgfqpoint{5.817015in}{2.859279in}}%
\pgfpathlineto{\pgfqpoint{5.813843in}{2.859204in}}%
\pgfpathlineto{\pgfqpoint{5.810671in}{2.858775in}}%
\pgfpathlineto{\pgfqpoint{5.807499in}{2.858583in}}%
\pgfpathlineto{\pgfqpoint{5.804327in}{2.858098in}}%
\pgfpathlineto{\pgfqpoint{5.801155in}{2.857819in}}%
\pgfpathlineto{\pgfqpoint{5.797983in}{2.857355in}}%
\pgfpathlineto{\pgfqpoint{5.794811in}{2.857266in}}%
\pgfpathlineto{\pgfqpoint{5.791639in}{2.857213in}}%
\pgfpathlineto{\pgfqpoint{5.788466in}{2.857048in}}%
\pgfpathlineto{\pgfqpoint{5.785294in}{2.857265in}}%
\pgfpathlineto{\pgfqpoint{5.782122in}{2.857529in}}%
\pgfpathlineto{\pgfqpoint{5.778950in}{2.857451in}}%
\pgfpathlineto{\pgfqpoint{5.775778in}{2.857939in}}%
\pgfpathlineto{\pgfqpoint{5.772606in}{2.857827in}}%
\pgfpathlineto{\pgfqpoint{5.769434in}{2.857900in}}%
\pgfpathlineto{\pgfqpoint{5.766262in}{2.857864in}}%
\pgfpathlineto{\pgfqpoint{5.763090in}{2.857817in}}%
\pgfpathlineto{\pgfqpoint{5.759918in}{2.857361in}}%
\pgfpathlineto{\pgfqpoint{5.756746in}{2.856704in}}%
\pgfpathlineto{\pgfqpoint{5.753574in}{2.856463in}}%
\pgfpathlineto{\pgfqpoint{5.750402in}{2.856674in}}%
\pgfpathlineto{\pgfqpoint{5.747230in}{2.856477in}}%
\pgfpathlineto{\pgfqpoint{5.744058in}{2.856061in}}%
\pgfpathlineto{\pgfqpoint{5.740886in}{2.855893in}}%
\pgfpathlineto{\pgfqpoint{5.737714in}{2.855699in}}%
\pgfpathlineto{\pgfqpoint{5.734542in}{2.855269in}}%
\pgfpathlineto{\pgfqpoint{5.731370in}{2.855133in}}%
\pgfpathlineto{\pgfqpoint{5.728198in}{2.855156in}}%
\pgfpathlineto{\pgfqpoint{5.725026in}{2.854510in}}%
\pgfpathlineto{\pgfqpoint{5.721854in}{2.854816in}}%
\pgfpathlineto{\pgfqpoint{5.718682in}{2.854699in}}%
\pgfpathlineto{\pgfqpoint{5.715510in}{2.855071in}}%
\pgfpathlineto{\pgfqpoint{5.712338in}{2.855194in}}%
\pgfpathlineto{\pgfqpoint{5.709165in}{2.855051in}}%
\pgfpathlineto{\pgfqpoint{5.705993in}{2.855422in}}%
\pgfpathlineto{\pgfqpoint{5.702821in}{2.855458in}}%
\pgfpathlineto{\pgfqpoint{5.699649in}{2.855183in}}%
\pgfpathlineto{\pgfqpoint{5.696477in}{2.855229in}}%
\pgfpathlineto{\pgfqpoint{5.693305in}{2.855324in}}%
\pgfpathlineto{\pgfqpoint{5.690133in}{2.855273in}}%
\pgfpathlineto{\pgfqpoint{5.686961in}{2.855080in}}%
\pgfpathlineto{\pgfqpoint{5.683789in}{2.855252in}}%
\pgfpathlineto{\pgfqpoint{5.680617in}{2.855126in}}%
\pgfpathlineto{\pgfqpoint{5.677445in}{2.855379in}}%
\pgfpathlineto{\pgfqpoint{5.674273in}{2.855233in}}%
\pgfpathlineto{\pgfqpoint{5.671101in}{2.854916in}}%
\pgfpathlineto{\pgfqpoint{5.667929in}{2.855395in}}%
\pgfpathlineto{\pgfqpoint{5.664757in}{2.855567in}}%
\pgfpathlineto{\pgfqpoint{5.661585in}{2.856269in}}%
\pgfpathlineto{\pgfqpoint{5.658413in}{2.856439in}}%
\pgfpathlineto{\pgfqpoint{5.655241in}{2.856823in}}%
\pgfpathlineto{\pgfqpoint{5.652069in}{2.856912in}}%
\pgfpathlineto{\pgfqpoint{5.648897in}{2.856690in}}%
\pgfpathlineto{\pgfqpoint{5.645725in}{2.856618in}}%
\pgfpathlineto{\pgfqpoint{5.642553in}{2.856440in}}%
\pgfpathlineto{\pgfqpoint{5.639381in}{2.856210in}}%
\pgfpathlineto{\pgfqpoint{5.636209in}{2.856061in}}%
\pgfpathlineto{\pgfqpoint{5.633037in}{2.856447in}}%
\pgfpathlineto{\pgfqpoint{5.629864in}{2.856302in}}%
\pgfpathlineto{\pgfqpoint{5.626692in}{2.856295in}}%
\pgfpathlineto{\pgfqpoint{5.623520in}{2.855890in}}%
\pgfpathlineto{\pgfqpoint{5.620348in}{2.855432in}}%
\pgfpathlineto{\pgfqpoint{5.617176in}{2.855280in}}%
\pgfpathlineto{\pgfqpoint{5.614004in}{2.854937in}}%
\pgfpathlineto{\pgfqpoint{5.610832in}{2.855298in}}%
\pgfpathlineto{\pgfqpoint{5.607660in}{2.855301in}}%
\pgfpathlineto{\pgfqpoint{5.604488in}{2.854651in}}%
\pgfpathlineto{\pgfqpoint{5.601316in}{2.854509in}}%
\pgfpathlineto{\pgfqpoint{5.598144in}{2.855456in}}%
\pgfpathlineto{\pgfqpoint{5.594972in}{2.855462in}}%
\pgfpathlineto{\pgfqpoint{5.591800in}{2.855986in}}%
\pgfpathlineto{\pgfqpoint{5.588628in}{2.856126in}}%
\pgfpathlineto{\pgfqpoint{5.585456in}{2.855830in}}%
\pgfpathlineto{\pgfqpoint{5.582284in}{2.855753in}}%
\pgfpathlineto{\pgfqpoint{5.579112in}{2.856045in}}%
\pgfpathlineto{\pgfqpoint{5.575940in}{2.855989in}}%
\pgfpathlineto{\pgfqpoint{5.572768in}{2.856241in}}%
\pgfpathlineto{\pgfqpoint{5.569596in}{2.856422in}}%
\pgfpathlineto{\pgfqpoint{5.566424in}{2.856002in}}%
\pgfpathlineto{\pgfqpoint{5.563252in}{2.856367in}}%
\pgfpathlineto{\pgfqpoint{5.560080in}{2.856285in}}%
\pgfpathlineto{\pgfqpoint{5.556908in}{2.856145in}}%
\pgfpathlineto{\pgfqpoint{5.553735in}{2.855784in}}%
\pgfpathlineto{\pgfqpoint{5.550563in}{2.855841in}}%
\pgfpathlineto{\pgfqpoint{5.547391in}{2.855773in}}%
\pgfpathlineto{\pgfqpoint{5.544219in}{2.855910in}}%
\pgfpathlineto{\pgfqpoint{5.541047in}{2.855864in}}%
\pgfpathlineto{\pgfqpoint{5.537875in}{2.856445in}}%
\pgfpathlineto{\pgfqpoint{5.534703in}{2.856656in}}%
\pgfpathlineto{\pgfqpoint{5.531531in}{2.856784in}}%
\pgfpathlineto{\pgfqpoint{5.528359in}{2.856151in}}%
\pgfpathlineto{\pgfqpoint{5.525187in}{2.855753in}}%
\pgfpathlineto{\pgfqpoint{5.522015in}{2.855820in}}%
\pgfpathlineto{\pgfqpoint{5.518843in}{2.855997in}}%
\pgfpathlineto{\pgfqpoint{5.515671in}{2.855837in}}%
\pgfpathlineto{\pgfqpoint{5.512499in}{2.855808in}}%
\pgfpathlineto{\pgfqpoint{5.509327in}{2.855625in}}%
\pgfpathlineto{\pgfqpoint{5.506155in}{2.855966in}}%
\pgfpathlineto{\pgfqpoint{5.502983in}{2.856043in}}%
\pgfpathlineto{\pgfqpoint{5.499811in}{2.855297in}}%
\pgfpathlineto{\pgfqpoint{5.496639in}{2.855057in}}%
\pgfpathlineto{\pgfqpoint{5.493467in}{2.854936in}}%
\pgfpathlineto{\pgfqpoint{5.490295in}{2.855224in}}%
\pgfpathlineto{\pgfqpoint{5.487123in}{2.855309in}}%
\pgfpathlineto{\pgfqpoint{5.483951in}{2.855817in}}%
\pgfpathlineto{\pgfqpoint{5.480779in}{2.855531in}}%
\pgfpathlineto{\pgfqpoint{5.477607in}{2.856223in}}%
\pgfpathlineto{\pgfqpoint{5.474434in}{2.856739in}}%
\pgfpathlineto{\pgfqpoint{5.471262in}{2.856946in}}%
\pgfpathlineto{\pgfqpoint{5.468090in}{2.857216in}}%
\pgfpathlineto{\pgfqpoint{5.464918in}{2.857090in}}%
\pgfpathlineto{\pgfqpoint{5.461746in}{2.856863in}}%
\pgfpathlineto{\pgfqpoint{5.458574in}{2.856662in}}%
\pgfpathlineto{\pgfqpoint{5.455402in}{2.856483in}}%
\pgfpathlineto{\pgfqpoint{5.452230in}{2.856316in}}%
\pgfpathlineto{\pgfqpoint{5.449058in}{2.856548in}}%
\pgfpathlineto{\pgfqpoint{5.445886in}{2.856530in}}%
\pgfpathlineto{\pgfqpoint{5.442714in}{2.856695in}}%
\pgfpathlineto{\pgfqpoint{5.439542in}{2.856523in}}%
\pgfpathlineto{\pgfqpoint{5.436370in}{2.856026in}}%
\pgfpathlineto{\pgfqpoint{5.433198in}{2.855918in}}%
\pgfpathlineto{\pgfqpoint{5.430026in}{2.856237in}}%
\pgfpathlineto{\pgfqpoint{5.426854in}{2.855734in}}%
\pgfpathlineto{\pgfqpoint{5.423682in}{2.855232in}}%
\pgfpathlineto{\pgfqpoint{5.420510in}{2.855080in}}%
\pgfpathlineto{\pgfqpoint{5.417338in}{2.855118in}}%
\pgfpathlineto{\pgfqpoint{5.414166in}{2.855028in}}%
\pgfpathlineto{\pgfqpoint{5.410994in}{2.855258in}}%
\pgfpathlineto{\pgfqpoint{5.407822in}{2.855293in}}%
\pgfpathlineto{\pgfqpoint{5.404650in}{2.855114in}}%
\pgfpathlineto{\pgfqpoint{5.401478in}{2.854581in}}%
\pgfpathlineto{\pgfqpoint{5.398306in}{2.854413in}}%
\pgfpathlineto{\pgfqpoint{5.395133in}{2.853722in}}%
\pgfpathlineto{\pgfqpoint{5.391961in}{2.853515in}}%
\pgfpathlineto{\pgfqpoint{5.388789in}{2.854066in}}%
\pgfpathlineto{\pgfqpoint{5.385617in}{2.853781in}}%
\pgfpathlineto{\pgfqpoint{5.382445in}{2.853322in}}%
\pgfpathlineto{\pgfqpoint{5.379273in}{2.853774in}}%
\pgfpathlineto{\pgfqpoint{5.376101in}{2.853312in}}%
\pgfpathlineto{\pgfqpoint{5.372929in}{2.853036in}}%
\pgfpathlineto{\pgfqpoint{5.369757in}{2.852831in}}%
\pgfpathlineto{\pgfqpoint{5.366585in}{2.852430in}}%
\pgfpathlineto{\pgfqpoint{5.363413in}{2.852615in}}%
\pgfpathlineto{\pgfqpoint{5.360241in}{2.852648in}}%
\pgfpathlineto{\pgfqpoint{5.357069in}{2.852368in}}%
\pgfpathlineto{\pgfqpoint{5.353897in}{2.852009in}}%
\pgfpathlineto{\pgfqpoint{5.350725in}{2.852053in}}%
\pgfpathlineto{\pgfqpoint{5.347553in}{2.852049in}}%
\pgfpathlineto{\pgfqpoint{5.344381in}{2.852024in}}%
\pgfpathlineto{\pgfqpoint{5.341209in}{2.851895in}}%
\pgfpathlineto{\pgfqpoint{5.338037in}{2.851816in}}%
\pgfpathlineto{\pgfqpoint{5.334865in}{2.851577in}}%
\pgfpathlineto{\pgfqpoint{5.331693in}{2.851597in}}%
\pgfpathlineto{\pgfqpoint{5.328521in}{2.850985in}}%
\pgfpathlineto{\pgfqpoint{5.325349in}{2.850945in}}%
\pgfpathlineto{\pgfqpoint{5.322177in}{2.851152in}}%
\pgfpathlineto{\pgfqpoint{5.319004in}{2.850931in}}%
\pgfpathlineto{\pgfqpoint{5.315832in}{2.850313in}}%
\pgfpathlineto{\pgfqpoint{5.312660in}{2.850361in}}%
\pgfpathlineto{\pgfqpoint{5.309488in}{2.850044in}}%
\pgfpathlineto{\pgfqpoint{5.306316in}{2.850113in}}%
\pgfpathlineto{\pgfqpoint{5.303144in}{2.849967in}}%
\pgfpathlineto{\pgfqpoint{5.299972in}{2.849506in}}%
\pgfpathlineto{\pgfqpoint{5.296800in}{2.849397in}}%
\pgfpathlineto{\pgfqpoint{5.293628in}{2.849010in}}%
\pgfpathlineto{\pgfqpoint{5.290456in}{2.848448in}}%
\pgfpathlineto{\pgfqpoint{5.287284in}{2.848956in}}%
\pgfpathlineto{\pgfqpoint{5.284112in}{2.849359in}}%
\pgfpathlineto{\pgfqpoint{5.280940in}{2.849526in}}%
\pgfpathlineto{\pgfqpoint{5.277768in}{2.849169in}}%
\pgfpathlineto{\pgfqpoint{5.274596in}{2.848626in}}%
\pgfpathlineto{\pgfqpoint{5.271424in}{2.848377in}}%
\pgfpathlineto{\pgfqpoint{5.268252in}{2.848187in}}%
\pgfpathlineto{\pgfqpoint{5.265080in}{2.847514in}}%
\pgfpathlineto{\pgfqpoint{5.261908in}{2.847432in}}%
\pgfpathlineto{\pgfqpoint{5.258736in}{2.847213in}}%
\pgfpathlineto{\pgfqpoint{5.255564in}{2.847004in}}%
\pgfpathlineto{\pgfqpoint{5.252392in}{2.846335in}}%
\pgfpathlineto{\pgfqpoint{5.249220in}{2.846293in}}%
\pgfpathlineto{\pgfqpoint{5.246048in}{2.846297in}}%
\pgfpathlineto{\pgfqpoint{5.242876in}{2.845699in}}%
\pgfpathlineto{\pgfqpoint{5.239703in}{2.845722in}}%
\pgfpathlineto{\pgfqpoint{5.236531in}{2.845461in}}%
\pgfpathlineto{\pgfqpoint{5.233359in}{2.845766in}}%
\pgfpathlineto{\pgfqpoint{5.230187in}{2.845910in}}%
\pgfpathlineto{\pgfqpoint{5.227015in}{2.846300in}}%
\pgfpathlineto{\pgfqpoint{5.223843in}{2.846194in}}%
\pgfpathlineto{\pgfqpoint{5.220671in}{2.846142in}}%
\pgfpathlineto{\pgfqpoint{5.217499in}{2.846388in}}%
\pgfpathlineto{\pgfqpoint{5.214327in}{2.846552in}}%
\pgfpathlineto{\pgfqpoint{5.211155in}{2.846494in}}%
\pgfpathlineto{\pgfqpoint{5.207983in}{2.845772in}}%
\pgfpathlineto{\pgfqpoint{5.204811in}{2.845397in}}%
\pgfpathlineto{\pgfqpoint{5.201639in}{2.845557in}}%
\pgfpathlineto{\pgfqpoint{5.198467in}{2.845487in}}%
\pgfpathlineto{\pgfqpoint{5.195295in}{2.845500in}}%
\pgfpathlineto{\pgfqpoint{5.192123in}{2.845285in}}%
\pgfpathlineto{\pgfqpoint{5.188951in}{2.845570in}}%
\pgfpathlineto{\pgfqpoint{5.185779in}{2.845577in}}%
\pgfpathlineto{\pgfqpoint{5.182607in}{2.845428in}}%
\pgfpathlineto{\pgfqpoint{5.179435in}{2.845591in}}%
\pgfpathlineto{\pgfqpoint{5.176263in}{2.845211in}}%
\pgfpathlineto{\pgfqpoint{5.173091in}{2.845427in}}%
\pgfpathlineto{\pgfqpoint{5.169919in}{2.846119in}}%
\pgfpathlineto{\pgfqpoint{5.166747in}{2.845950in}}%
\pgfpathlineto{\pgfqpoint{5.163575in}{2.845841in}}%
\pgfpathlineto{\pgfqpoint{5.160402in}{2.845745in}}%
\pgfpathlineto{\pgfqpoint{5.157230in}{2.845616in}}%
\pgfpathlineto{\pgfqpoint{5.154058in}{2.845223in}}%
\pgfpathlineto{\pgfqpoint{5.150886in}{2.845216in}}%
\pgfpathlineto{\pgfqpoint{5.147714in}{2.845249in}}%
\pgfpathlineto{\pgfqpoint{5.144542in}{2.845545in}}%
\pgfpathlineto{\pgfqpoint{5.141370in}{2.845393in}}%
\pgfpathlineto{\pgfqpoint{5.138198in}{2.845462in}}%
\pgfpathlineto{\pgfqpoint{5.135026in}{2.845720in}}%
\pgfpathlineto{\pgfqpoint{5.131854in}{2.845607in}}%
\pgfpathlineto{\pgfqpoint{5.128682in}{2.845434in}}%
\pgfpathlineto{\pgfqpoint{5.125510in}{2.845267in}}%
\pgfpathlineto{\pgfqpoint{5.122338in}{2.845531in}}%
\pgfpathlineto{\pgfqpoint{5.119166in}{2.845350in}}%
\pgfpathlineto{\pgfqpoint{5.115994in}{2.845046in}}%
\pgfpathlineto{\pgfqpoint{5.112822in}{2.844524in}}%
\pgfpathlineto{\pgfqpoint{5.109650in}{2.844985in}}%
\pgfpathlineto{\pgfqpoint{5.106478in}{2.844996in}}%
\pgfpathlineto{\pgfqpoint{5.103306in}{2.844874in}}%
\pgfpathlineto{\pgfqpoint{5.100134in}{2.845174in}}%
\pgfpathlineto{\pgfqpoint{5.096962in}{2.845091in}}%
\pgfpathlineto{\pgfqpoint{5.093790in}{2.845303in}}%
\pgfpathlineto{\pgfqpoint{5.090618in}{2.845071in}}%
\pgfpathlineto{\pgfqpoint{5.087446in}{2.845162in}}%
\pgfpathlineto{\pgfqpoint{5.084273in}{2.845281in}}%
\pgfpathlineto{\pgfqpoint{5.081101in}{2.845055in}}%
\pgfpathlineto{\pgfqpoint{5.077929in}{2.844805in}}%
\pgfpathlineto{\pgfqpoint{5.074757in}{2.845239in}}%
\pgfpathlineto{\pgfqpoint{5.071585in}{2.845183in}}%
\pgfpathlineto{\pgfqpoint{5.068413in}{2.844962in}}%
\pgfpathlineto{\pgfqpoint{5.065241in}{2.845015in}}%
\pgfpathlineto{\pgfqpoint{5.062069in}{2.845029in}}%
\pgfpathlineto{\pgfqpoint{5.058897in}{2.844665in}}%
\pgfpathlineto{\pgfqpoint{5.055725in}{2.844320in}}%
\pgfpathlineto{\pgfqpoint{5.052553in}{2.844549in}}%
\pgfpathlineto{\pgfqpoint{5.049381in}{2.844761in}}%
\pgfpathlineto{\pgfqpoint{5.046209in}{2.845679in}}%
\pgfpathlineto{\pgfqpoint{5.043037in}{2.845565in}}%
\pgfpathlineto{\pgfqpoint{5.039865in}{2.845477in}}%
\pgfpathlineto{\pgfqpoint{5.036693in}{2.845391in}}%
\pgfpathlineto{\pgfqpoint{5.033521in}{2.845542in}}%
\pgfpathlineto{\pgfqpoint{5.030349in}{2.845534in}}%
\pgfpathlineto{\pgfqpoint{5.027177in}{2.845318in}}%
\pgfpathlineto{\pgfqpoint{5.024005in}{2.844799in}}%
\pgfpathlineto{\pgfqpoint{5.020833in}{2.844409in}}%
\pgfpathlineto{\pgfqpoint{5.017661in}{2.844087in}}%
\pgfpathlineto{\pgfqpoint{5.014489in}{2.844012in}}%
\pgfpathlineto{\pgfqpoint{5.011317in}{2.843465in}}%
\pgfpathlineto{\pgfqpoint{5.008145in}{2.843354in}}%
\pgfpathlineto{\pgfqpoint{5.004972in}{2.843825in}}%
\pgfpathlineto{\pgfqpoint{5.001800in}{2.843550in}}%
\pgfpathlineto{\pgfqpoint{4.998628in}{2.843634in}}%
\pgfpathlineto{\pgfqpoint{4.995456in}{2.843857in}}%
\pgfpathlineto{\pgfqpoint{4.992284in}{2.843653in}}%
\pgfpathlineto{\pgfqpoint{4.989112in}{2.843723in}}%
\pgfpathlineto{\pgfqpoint{4.985940in}{2.844051in}}%
\pgfpathlineto{\pgfqpoint{4.982768in}{2.843648in}}%
\pgfpathlineto{\pgfqpoint{4.979596in}{2.843655in}}%
\pgfpathlineto{\pgfqpoint{4.976424in}{2.843245in}}%
\pgfpathlineto{\pgfqpoint{4.973252in}{2.843025in}}%
\pgfpathlineto{\pgfqpoint{4.970080in}{2.842727in}}%
\pgfpathlineto{\pgfqpoint{4.966908in}{2.842935in}}%
\pgfpathlineto{\pgfqpoint{4.963736in}{2.842574in}}%
\pgfpathlineto{\pgfqpoint{4.960564in}{2.842751in}}%
\pgfpathlineto{\pgfqpoint{4.957392in}{2.842620in}}%
\pgfpathlineto{\pgfqpoint{4.954220in}{2.842872in}}%
\pgfpathlineto{\pgfqpoint{4.951048in}{2.842439in}}%
\pgfpathlineto{\pgfqpoint{4.947876in}{2.842574in}}%
\pgfpathlineto{\pgfqpoint{4.944704in}{2.842717in}}%
\pgfpathlineto{\pgfqpoint{4.941532in}{2.842618in}}%
\pgfpathlineto{\pgfqpoint{4.938360in}{2.842402in}}%
\pgfpathlineto{\pgfqpoint{4.935188in}{2.842622in}}%
\pgfpathlineto{\pgfqpoint{4.932016in}{2.842453in}}%
\pgfpathlineto{\pgfqpoint{4.928844in}{2.842078in}}%
\pgfpathlineto{\pgfqpoint{4.925671in}{2.841724in}}%
\pgfpathlineto{\pgfqpoint{4.922499in}{2.841222in}}%
\pgfpathlineto{\pgfqpoint{4.919327in}{2.841324in}}%
\pgfpathlineto{\pgfqpoint{4.916155in}{2.841166in}}%
\pgfpathlineto{\pgfqpoint{4.912983in}{2.840818in}}%
\pgfpathlineto{\pgfqpoint{4.909811in}{2.840922in}}%
\pgfpathlineto{\pgfqpoint{4.906639in}{2.840617in}}%
\pgfpathlineto{\pgfqpoint{4.903467in}{2.840597in}}%
\pgfpathlineto{\pgfqpoint{4.900295in}{2.840088in}}%
\pgfpathlineto{\pgfqpoint{4.897123in}{2.839859in}}%
\pgfpathlineto{\pgfqpoint{4.893951in}{2.839818in}}%
\pgfpathlineto{\pgfqpoint{4.890779in}{2.839563in}}%
\pgfpathlineto{\pgfqpoint{4.887607in}{2.839887in}}%
\pgfpathlineto{\pgfqpoint{4.884435in}{2.839867in}}%
\pgfpathlineto{\pgfqpoint{4.881263in}{2.839602in}}%
\pgfpathlineto{\pgfqpoint{4.878091in}{2.839340in}}%
\pgfpathlineto{\pgfqpoint{4.874919in}{2.839316in}}%
\pgfpathlineto{\pgfqpoint{4.871747in}{2.839062in}}%
\pgfpathlineto{\pgfqpoint{4.868575in}{2.838452in}}%
\pgfpathlineto{\pgfqpoint{4.865403in}{2.838035in}}%
\pgfpathlineto{\pgfqpoint{4.862231in}{2.837621in}}%
\pgfpathlineto{\pgfqpoint{4.859059in}{2.837473in}}%
\pgfpathlineto{\pgfqpoint{4.855887in}{2.836911in}}%
\pgfpathlineto{\pgfqpoint{4.852715in}{2.836702in}}%
\pgfpathlineto{\pgfqpoint{4.849542in}{2.836644in}}%
\pgfpathlineto{\pgfqpoint{4.846370in}{2.836668in}}%
\pgfpathlineto{\pgfqpoint{4.843198in}{2.836342in}}%
\pgfpathlineto{\pgfqpoint{4.840026in}{2.836232in}}%
\pgfpathlineto{\pgfqpoint{4.836854in}{2.836145in}}%
\pgfpathlineto{\pgfqpoint{4.833682in}{2.836253in}}%
\pgfpathlineto{\pgfqpoint{4.830510in}{2.836011in}}%
\pgfpathlineto{\pgfqpoint{4.827338in}{2.835996in}}%
\pgfpathlineto{\pgfqpoint{4.824166in}{2.835842in}}%
\pgfpathlineto{\pgfqpoint{4.820994in}{2.835957in}}%
\pgfpathlineto{\pgfqpoint{4.817822in}{2.835451in}}%
\pgfpathlineto{\pgfqpoint{4.814650in}{2.834989in}}%
\pgfpathlineto{\pgfqpoint{4.811478in}{2.834769in}}%
\pgfpathlineto{\pgfqpoint{4.808306in}{2.834877in}}%
\pgfpathlineto{\pgfqpoint{4.805134in}{2.834618in}}%
\pgfpathlineto{\pgfqpoint{4.801962in}{2.834734in}}%
\pgfpathlineto{\pgfqpoint{4.798790in}{2.834723in}}%
\pgfpathlineto{\pgfqpoint{4.795618in}{2.834612in}}%
\pgfpathlineto{\pgfqpoint{4.792446in}{2.834553in}}%
\pgfpathlineto{\pgfqpoint{4.789274in}{2.834118in}}%
\pgfpathlineto{\pgfqpoint{4.786102in}{2.834157in}}%
\pgfpathlineto{\pgfqpoint{4.782930in}{2.834037in}}%
\pgfpathlineto{\pgfqpoint{4.779758in}{2.834543in}}%
\pgfpathlineto{\pgfqpoint{4.776586in}{2.834940in}}%
\pgfpathlineto{\pgfqpoint{4.773414in}{2.834751in}}%
\pgfpathlineto{\pgfqpoint{4.770241in}{2.834078in}}%
\pgfpathlineto{\pgfqpoint{4.767069in}{2.833629in}}%
\pgfpathlineto{\pgfqpoint{4.763897in}{2.833596in}}%
\pgfpathlineto{\pgfqpoint{4.760725in}{2.833367in}}%
\pgfpathlineto{\pgfqpoint{4.757553in}{2.833453in}}%
\pgfpathlineto{\pgfqpoint{4.754381in}{2.833238in}}%
\pgfpathlineto{\pgfqpoint{4.751209in}{2.833007in}}%
\pgfpathlineto{\pgfqpoint{4.748037in}{2.832892in}}%
\pgfpathlineto{\pgfqpoint{4.744865in}{2.833156in}}%
\pgfpathlineto{\pgfqpoint{4.741693in}{2.832667in}}%
\pgfpathlineto{\pgfqpoint{4.738521in}{2.832527in}}%
\pgfpathlineto{\pgfqpoint{4.735349in}{2.832489in}}%
\pgfpathlineto{\pgfqpoint{4.732177in}{2.831946in}}%
\pgfpathlineto{\pgfqpoint{4.729005in}{2.831994in}}%
\pgfpathlineto{\pgfqpoint{4.725833in}{2.832311in}}%
\pgfpathlineto{\pgfqpoint{4.722661in}{2.832049in}}%
\pgfpathlineto{\pgfqpoint{4.719489in}{2.832331in}}%
\pgfpathlineto{\pgfqpoint{4.716317in}{2.832449in}}%
\pgfpathlineto{\pgfqpoint{4.713145in}{2.831864in}}%
\pgfpathlineto{\pgfqpoint{4.709973in}{2.831655in}}%
\pgfpathlineto{\pgfqpoint{4.706801in}{2.831284in}}%
\pgfpathlineto{\pgfqpoint{4.703629in}{2.831025in}}%
\pgfpathlineto{\pgfqpoint{4.700457in}{2.830826in}}%
\pgfpathlineto{\pgfqpoint{4.697285in}{2.831116in}}%
\pgfpathlineto{\pgfqpoint{4.694112in}{2.830590in}}%
\pgfpathlineto{\pgfqpoint{4.690940in}{2.830128in}}%
\pgfpathlineto{\pgfqpoint{4.687768in}{2.830091in}}%
\pgfpathlineto{\pgfqpoint{4.684596in}{2.830142in}}%
\pgfpathlineto{\pgfqpoint{4.681424in}{2.829353in}}%
\pgfpathlineto{\pgfqpoint{4.678252in}{2.829555in}}%
\pgfpathlineto{\pgfqpoint{4.675080in}{2.829122in}}%
\pgfpathlineto{\pgfqpoint{4.671908in}{2.828941in}}%
\pgfpathlineto{\pgfqpoint{4.668736in}{2.828972in}}%
\pgfpathlineto{\pgfqpoint{4.665564in}{2.828811in}}%
\pgfpathlineto{\pgfqpoint{4.662392in}{2.828624in}}%
\pgfpathlineto{\pgfqpoint{4.659220in}{2.828864in}}%
\pgfpathlineto{\pgfqpoint{4.656048in}{2.828475in}}%
\pgfpathlineto{\pgfqpoint{4.652876in}{2.828766in}}%
\pgfpathlineto{\pgfqpoint{4.649704in}{2.829151in}}%
\pgfpathlineto{\pgfqpoint{4.646532in}{2.828798in}}%
\pgfpathlineto{\pgfqpoint{4.643360in}{2.828972in}}%
\pgfpathlineto{\pgfqpoint{4.640188in}{2.829065in}}%
\pgfpathlineto{\pgfqpoint{4.637016in}{2.829290in}}%
\pgfpathlineto{\pgfqpoint{4.633844in}{2.829314in}}%
\pgfpathlineto{\pgfqpoint{4.630672in}{2.829423in}}%
\pgfpathlineto{\pgfqpoint{4.627500in}{2.829277in}}%
\pgfpathlineto{\pgfqpoint{4.624328in}{2.828951in}}%
\pgfpathlineto{\pgfqpoint{4.621156in}{2.829476in}}%
\pgfpathlineto{\pgfqpoint{4.617984in}{2.829554in}}%
\pgfpathlineto{\pgfqpoint{4.614811in}{2.829779in}}%
\pgfpathlineto{\pgfqpoint{4.611639in}{2.829912in}}%
\pgfpathlineto{\pgfqpoint{4.608467in}{2.829908in}}%
\pgfpathlineto{\pgfqpoint{4.605295in}{2.830128in}}%
\pgfpathlineto{\pgfqpoint{4.602123in}{2.830701in}}%
\pgfpathlineto{\pgfqpoint{4.598951in}{2.830611in}}%
\pgfpathlineto{\pgfqpoint{4.595779in}{2.830848in}}%
\pgfpathlineto{\pgfqpoint{4.592607in}{2.830493in}}%
\pgfpathlineto{\pgfqpoint{4.589435in}{2.830738in}}%
\pgfpathlineto{\pgfqpoint{4.586263in}{2.830372in}}%
\pgfpathlineto{\pgfqpoint{4.583091in}{2.830319in}}%
\pgfpathlineto{\pgfqpoint{4.579919in}{2.830280in}}%
\pgfpathlineto{\pgfqpoint{4.576747in}{2.830445in}}%
\pgfpathlineto{\pgfqpoint{4.573575in}{2.830656in}}%
\pgfpathlineto{\pgfqpoint{4.570403in}{2.830402in}}%
\pgfpathlineto{\pgfqpoint{4.567231in}{2.830274in}}%
\pgfpathlineto{\pgfqpoint{4.564059in}{2.830269in}}%
\pgfpathlineto{\pgfqpoint{4.560887in}{2.830042in}}%
\pgfpathlineto{\pgfqpoint{4.557715in}{2.829693in}}%
\pgfpathlineto{\pgfqpoint{4.554543in}{2.829537in}}%
\pgfpathlineto{\pgfqpoint{4.551371in}{2.829763in}}%
\pgfpathlineto{\pgfqpoint{4.548199in}{2.829855in}}%
\pgfpathlineto{\pgfqpoint{4.545027in}{2.829849in}}%
\pgfpathlineto{\pgfqpoint{4.541855in}{2.830196in}}%
\pgfpathlineto{\pgfqpoint{4.538683in}{2.829910in}}%
\pgfpathlineto{\pgfqpoint{4.535510in}{2.829681in}}%
\pgfpathlineto{\pgfqpoint{4.532338in}{2.829810in}}%
\pgfpathlineto{\pgfqpoint{4.529166in}{2.829616in}}%
\pgfpathlineto{\pgfqpoint{4.525994in}{2.829005in}}%
\pgfpathlineto{\pgfqpoint{4.522822in}{2.829057in}}%
\pgfpathlineto{\pgfqpoint{4.519650in}{2.828953in}}%
\pgfpathlineto{\pgfqpoint{4.516478in}{2.828665in}}%
\pgfpathlineto{\pgfqpoint{4.513306in}{2.829099in}}%
\pgfpathlineto{\pgfqpoint{4.510134in}{2.829047in}}%
\pgfpathlineto{\pgfqpoint{4.506962in}{2.828985in}}%
\pgfpathlineto{\pgfqpoint{4.503790in}{2.828781in}}%
\pgfpathlineto{\pgfqpoint{4.500618in}{2.828331in}}%
\pgfpathlineto{\pgfqpoint{4.497446in}{2.828036in}}%
\pgfpathlineto{\pgfqpoint{4.494274in}{2.828032in}}%
\pgfpathlineto{\pgfqpoint{4.491102in}{2.827650in}}%
\pgfpathlineto{\pgfqpoint{4.487930in}{2.827525in}}%
\pgfpathlineto{\pgfqpoint{4.484758in}{2.827475in}}%
\pgfpathlineto{\pgfqpoint{4.481586in}{2.827109in}}%
\pgfpathlineto{\pgfqpoint{4.478414in}{2.827253in}}%
\pgfpathlineto{\pgfqpoint{4.475242in}{2.827222in}}%
\pgfpathlineto{\pgfqpoint{4.472070in}{2.827279in}}%
\pgfpathlineto{\pgfqpoint{4.468898in}{2.827368in}}%
\pgfpathlineto{\pgfqpoint{4.465726in}{2.827278in}}%
\pgfpathlineto{\pgfqpoint{4.462554in}{2.827442in}}%
\pgfpathlineto{\pgfqpoint{4.459381in}{2.827753in}}%
\pgfpathlineto{\pgfqpoint{4.456209in}{2.828039in}}%
\pgfpathlineto{\pgfqpoint{4.453037in}{2.828297in}}%
\pgfpathlineto{\pgfqpoint{4.449865in}{2.828266in}}%
\pgfpathlineto{\pgfqpoint{4.446693in}{2.828218in}}%
\pgfpathlineto{\pgfqpoint{4.443521in}{2.828217in}}%
\pgfpathlineto{\pgfqpoint{4.440349in}{2.827923in}}%
\pgfpathlineto{\pgfqpoint{4.437177in}{2.827499in}}%
\pgfpathlineto{\pgfqpoint{4.434005in}{2.827065in}}%
\pgfpathlineto{\pgfqpoint{4.430833in}{2.827033in}}%
\pgfpathlineto{\pgfqpoint{4.427661in}{2.827038in}}%
\pgfpathlineto{\pgfqpoint{4.424489in}{2.826765in}}%
\pgfpathlineto{\pgfqpoint{4.421317in}{2.826785in}}%
\pgfpathlineto{\pgfqpoint{4.418145in}{2.826583in}}%
\pgfpathlineto{\pgfqpoint{4.414973in}{2.826202in}}%
\pgfpathlineto{\pgfqpoint{4.411801in}{2.826780in}}%
\pgfpathlineto{\pgfqpoint{4.408629in}{2.826686in}}%
\pgfpathlineto{\pgfqpoint{4.405457in}{2.826465in}}%
\pgfpathlineto{\pgfqpoint{4.402285in}{2.826974in}}%
\pgfpathlineto{\pgfqpoint{4.399113in}{2.826727in}}%
\pgfpathlineto{\pgfqpoint{4.395941in}{2.826747in}}%
\pgfpathlineto{\pgfqpoint{4.392769in}{2.826849in}}%
\pgfpathlineto{\pgfqpoint{4.389597in}{2.826742in}}%
\pgfpathlineto{\pgfqpoint{4.386425in}{2.826467in}}%
\pgfpathlineto{\pgfqpoint{4.383253in}{2.825861in}}%
\pgfpathlineto{\pgfqpoint{4.380080in}{2.825782in}}%
\pgfpathlineto{\pgfqpoint{4.376908in}{2.825932in}}%
\pgfpathlineto{\pgfqpoint{4.373736in}{2.825872in}}%
\pgfpathlineto{\pgfqpoint{4.370564in}{2.826077in}}%
\pgfpathlineto{\pgfqpoint{4.367392in}{2.826276in}}%
\pgfpathlineto{\pgfqpoint{4.364220in}{2.826226in}}%
\pgfpathlineto{\pgfqpoint{4.361048in}{2.826251in}}%
\pgfpathlineto{\pgfqpoint{4.357876in}{2.826425in}}%
\pgfpathlineto{\pgfqpoint{4.354704in}{2.826746in}}%
\pgfpathlineto{\pgfqpoint{4.351532in}{2.826607in}}%
\pgfpathlineto{\pgfqpoint{4.348360in}{2.826645in}}%
\pgfpathlineto{\pgfqpoint{4.345188in}{2.826741in}}%
\pgfpathlineto{\pgfqpoint{4.342016in}{2.825752in}}%
\pgfpathlineto{\pgfqpoint{4.338844in}{2.825654in}}%
\pgfpathlineto{\pgfqpoint{4.335672in}{2.825620in}}%
\pgfpathlineto{\pgfqpoint{4.332500in}{2.825539in}}%
\pgfpathlineto{\pgfqpoint{4.329328in}{2.825785in}}%
\pgfpathlineto{\pgfqpoint{4.326156in}{2.825726in}}%
\pgfpathlineto{\pgfqpoint{4.322984in}{2.825099in}}%
\pgfpathlineto{\pgfqpoint{4.319812in}{2.825342in}}%
\pgfpathlineto{\pgfqpoint{4.316640in}{2.824984in}}%
\pgfpathlineto{\pgfqpoint{4.313468in}{2.825059in}}%
\pgfpathlineto{\pgfqpoint{4.310296in}{2.824688in}}%
\pgfpathlineto{\pgfqpoint{4.307124in}{2.824270in}}%
\pgfpathlineto{\pgfqpoint{4.303952in}{2.824092in}}%
\pgfpathlineto{\pgfqpoint{4.300779in}{2.824156in}}%
\pgfpathlineto{\pgfqpoint{4.297607in}{2.823744in}}%
\pgfpathlineto{\pgfqpoint{4.294435in}{2.823651in}}%
\pgfpathlineto{\pgfqpoint{4.291263in}{2.824204in}}%
\pgfpathlineto{\pgfqpoint{4.288091in}{2.824445in}}%
\pgfpathlineto{\pgfqpoint{4.284919in}{2.824371in}}%
\pgfpathlineto{\pgfqpoint{4.281747in}{2.824214in}}%
\pgfpathlineto{\pgfqpoint{4.278575in}{2.824514in}}%
\pgfpathlineto{\pgfqpoint{4.275403in}{2.824278in}}%
\pgfpathlineto{\pgfqpoint{4.272231in}{2.823837in}}%
\pgfpathlineto{\pgfqpoint{4.269059in}{2.823893in}}%
\pgfpathlineto{\pgfqpoint{4.265887in}{2.823709in}}%
\pgfpathlineto{\pgfqpoint{4.262715in}{2.823501in}}%
\pgfpathlineto{\pgfqpoint{4.259543in}{2.823219in}}%
\pgfpathlineto{\pgfqpoint{4.256371in}{2.822718in}}%
\pgfpathlineto{\pgfqpoint{4.253199in}{2.822757in}}%
\pgfpathlineto{\pgfqpoint{4.250027in}{2.823082in}}%
\pgfpathlineto{\pgfqpoint{4.246855in}{2.823384in}}%
\pgfpathlineto{\pgfqpoint{4.243683in}{2.823564in}}%
\pgfpathlineto{\pgfqpoint{4.240511in}{2.823744in}}%
\pgfpathlineto{\pgfqpoint{4.237339in}{2.823767in}}%
\pgfpathlineto{\pgfqpoint{4.234167in}{2.823704in}}%
\pgfpathlineto{\pgfqpoint{4.230995in}{2.823141in}}%
\pgfpathlineto{\pgfqpoint{4.227823in}{2.822116in}}%
\pgfpathlineto{\pgfqpoint{4.224650in}{2.822179in}}%
\pgfpathlineto{\pgfqpoint{4.221478in}{2.822383in}}%
\pgfpathlineto{\pgfqpoint{4.218306in}{2.822790in}}%
\pgfpathlineto{\pgfqpoint{4.215134in}{2.822612in}}%
\pgfpathlineto{\pgfqpoint{4.211962in}{2.823102in}}%
\pgfpathlineto{\pgfqpoint{4.208790in}{2.823801in}}%
\pgfpathlineto{\pgfqpoint{4.205618in}{2.824159in}}%
\pgfpathlineto{\pgfqpoint{4.202446in}{2.824521in}}%
\pgfpathlineto{\pgfqpoint{4.199274in}{2.824349in}}%
\pgfpathlineto{\pgfqpoint{4.196102in}{2.824090in}}%
\pgfpathlineto{\pgfqpoint{4.192930in}{2.823994in}}%
\pgfpathlineto{\pgfqpoint{4.189758in}{2.824237in}}%
\pgfpathlineto{\pgfqpoint{4.186586in}{2.824484in}}%
\pgfpathlineto{\pgfqpoint{4.183414in}{2.824831in}}%
\pgfpathlineto{\pgfqpoint{4.180242in}{2.824442in}}%
\pgfpathlineto{\pgfqpoint{4.177070in}{2.824296in}}%
\pgfpathlineto{\pgfqpoint{4.173898in}{2.823978in}}%
\pgfpathlineto{\pgfqpoint{4.170726in}{2.824202in}}%
\pgfpathlineto{\pgfqpoint{4.167554in}{2.824046in}}%
\pgfpathlineto{\pgfqpoint{4.164382in}{2.823935in}}%
\pgfpathlineto{\pgfqpoint{4.161210in}{2.823930in}}%
\pgfpathlineto{\pgfqpoint{4.158038in}{2.824039in}}%
\pgfpathlineto{\pgfqpoint{4.154866in}{2.823638in}}%
\pgfpathlineto{\pgfqpoint{4.151694in}{2.823608in}}%
\pgfpathlineto{\pgfqpoint{4.148522in}{2.823549in}}%
\pgfpathlineto{\pgfqpoint{4.145349in}{2.823698in}}%
\pgfpathlineto{\pgfqpoint{4.142177in}{2.823449in}}%
\pgfpathlineto{\pgfqpoint{4.139005in}{2.823401in}}%
\pgfpathlineto{\pgfqpoint{4.135833in}{2.823155in}}%
\pgfpathlineto{\pgfqpoint{4.132661in}{2.823514in}}%
\pgfpathlineto{\pgfqpoint{4.129489in}{2.823635in}}%
\pgfpathlineto{\pgfqpoint{4.126317in}{2.823379in}}%
\pgfpathlineto{\pgfqpoint{4.123145in}{2.822785in}}%
\pgfpathlineto{\pgfqpoint{4.119973in}{2.822585in}}%
\pgfpathlineto{\pgfqpoint{4.116801in}{2.822455in}}%
\pgfpathlineto{\pgfqpoint{4.113629in}{2.822457in}}%
\pgfpathlineto{\pgfqpoint{4.110457in}{2.822124in}}%
\pgfpathlineto{\pgfqpoint{4.107285in}{2.821928in}}%
\pgfpathlineto{\pgfqpoint{4.104113in}{2.821579in}}%
\pgfpathlineto{\pgfqpoint{4.100941in}{2.821670in}}%
\pgfpathlineto{\pgfqpoint{4.097769in}{2.821666in}}%
\pgfpathlineto{\pgfqpoint{4.094597in}{2.821859in}}%
\pgfpathlineto{\pgfqpoint{4.091425in}{2.822130in}}%
\pgfpathlineto{\pgfqpoint{4.088253in}{2.822117in}}%
\pgfpathlineto{\pgfqpoint{4.085081in}{2.821546in}}%
\pgfpathlineto{\pgfqpoint{4.081909in}{2.821899in}}%
\pgfpathlineto{\pgfqpoint{4.078737in}{2.821998in}}%
\pgfpathlineto{\pgfqpoint{4.075565in}{2.821750in}}%
\pgfpathlineto{\pgfqpoint{4.072393in}{2.821153in}}%
\pgfpathlineto{\pgfqpoint{4.069221in}{2.820887in}}%
\pgfpathlineto{\pgfqpoint{4.066048in}{2.820638in}}%
\pgfpathlineto{\pgfqpoint{4.062876in}{2.820732in}}%
\pgfpathlineto{\pgfqpoint{4.059704in}{2.820778in}}%
\pgfpathlineto{\pgfqpoint{4.056532in}{2.820583in}}%
\pgfpathlineto{\pgfqpoint{4.053360in}{2.820363in}}%
\pgfpathlineto{\pgfqpoint{4.050188in}{2.820096in}}%
\pgfpathlineto{\pgfqpoint{4.047016in}{2.820242in}}%
\pgfpathlineto{\pgfqpoint{4.043844in}{2.820079in}}%
\pgfpathlineto{\pgfqpoint{4.040672in}{2.819983in}}%
\pgfpathlineto{\pgfqpoint{4.037500in}{2.819895in}}%
\pgfpathlineto{\pgfqpoint{4.034328in}{2.820345in}}%
\pgfpathlineto{\pgfqpoint{4.031156in}{2.819558in}}%
\pgfpathlineto{\pgfqpoint{4.027984in}{2.819393in}}%
\pgfpathlineto{\pgfqpoint{4.024812in}{2.819182in}}%
\pgfpathlineto{\pgfqpoint{4.021640in}{2.819140in}}%
\pgfpathlineto{\pgfqpoint{4.018468in}{2.818676in}}%
\pgfpathlineto{\pgfqpoint{4.015296in}{2.818529in}}%
\pgfpathlineto{\pgfqpoint{4.012124in}{2.818048in}}%
\pgfpathlineto{\pgfqpoint{4.008952in}{2.818053in}}%
\pgfpathlineto{\pgfqpoint{4.005780in}{2.817367in}}%
\pgfpathlineto{\pgfqpoint{4.002608in}{2.817342in}}%
\pgfpathlineto{\pgfqpoint{3.999436in}{2.816654in}}%
\pgfpathlineto{\pgfqpoint{3.996264in}{2.816489in}}%
\pgfpathlineto{\pgfqpoint{3.993092in}{2.816643in}}%
\pgfpathlineto{\pgfqpoint{3.989919in}{2.816561in}}%
\pgfpathlineto{\pgfqpoint{3.986747in}{2.816778in}}%
\pgfpathlineto{\pgfqpoint{3.983575in}{2.816691in}}%
\pgfpathlineto{\pgfqpoint{3.980403in}{2.816607in}}%
\pgfpathlineto{\pgfqpoint{3.977231in}{2.817145in}}%
\pgfpathlineto{\pgfqpoint{3.974059in}{2.817324in}}%
\pgfpathlineto{\pgfqpoint{3.970887in}{2.817614in}}%
\pgfpathlineto{\pgfqpoint{3.967715in}{2.817700in}}%
\pgfpathlineto{\pgfqpoint{3.964543in}{2.817775in}}%
\pgfpathlineto{\pgfqpoint{3.961371in}{2.817915in}}%
\pgfpathlineto{\pgfqpoint{3.958199in}{2.817301in}}%
\pgfpathlineto{\pgfqpoint{3.955027in}{2.817470in}}%
\pgfpathlineto{\pgfqpoint{3.951855in}{2.817522in}}%
\pgfpathlineto{\pgfqpoint{3.948683in}{2.817508in}}%
\pgfpathlineto{\pgfqpoint{3.945511in}{2.816763in}}%
\pgfpathlineto{\pgfqpoint{3.942339in}{2.816640in}}%
\pgfpathlineto{\pgfqpoint{3.939167in}{2.816784in}}%
\pgfpathlineto{\pgfqpoint{3.935995in}{2.816744in}}%
\pgfpathlineto{\pgfqpoint{3.932823in}{2.816781in}}%
\pgfpathlineto{\pgfqpoint{3.929651in}{2.816890in}}%
\pgfpathlineto{\pgfqpoint{3.926479in}{2.816570in}}%
\pgfpathlineto{\pgfqpoint{3.923307in}{2.816187in}}%
\pgfpathlineto{\pgfqpoint{3.920135in}{2.815317in}}%
\pgfpathlineto{\pgfqpoint{3.916963in}{2.815409in}}%
\pgfpathlineto{\pgfqpoint{3.913791in}{2.815113in}}%
\pgfpathlineto{\pgfqpoint{3.910618in}{2.815537in}}%
\pgfpathlineto{\pgfqpoint{3.907446in}{2.815511in}}%
\pgfpathlineto{\pgfqpoint{3.904274in}{2.815114in}}%
\pgfpathlineto{\pgfqpoint{3.901102in}{2.814377in}}%
\pgfpathlineto{\pgfqpoint{3.897930in}{2.814629in}}%
\pgfpathlineto{\pgfqpoint{3.894758in}{2.814428in}}%
\pgfpathlineto{\pgfqpoint{3.891586in}{2.813791in}}%
\pgfpathlineto{\pgfqpoint{3.888414in}{2.813876in}}%
\pgfpathlineto{\pgfqpoint{3.885242in}{2.813850in}}%
\pgfpathlineto{\pgfqpoint{3.882070in}{2.813522in}}%
\pgfpathlineto{\pgfqpoint{3.878898in}{2.813492in}}%
\pgfpathlineto{\pgfqpoint{3.875726in}{2.813082in}}%
\pgfpathlineto{\pgfqpoint{3.872554in}{2.812739in}}%
\pgfpathlineto{\pgfqpoint{3.869382in}{2.812576in}}%
\pgfpathlineto{\pgfqpoint{3.866210in}{2.812397in}}%
\pgfpathlineto{\pgfqpoint{3.863038in}{2.811705in}}%
\pgfpathlineto{\pgfqpoint{3.859866in}{2.811678in}}%
\pgfpathlineto{\pgfqpoint{3.856694in}{2.812006in}}%
\pgfpathlineto{\pgfqpoint{3.853522in}{2.811865in}}%
\pgfpathlineto{\pgfqpoint{3.850350in}{2.811748in}}%
\pgfpathlineto{\pgfqpoint{3.847178in}{2.812097in}}%
\pgfpathlineto{\pgfqpoint{3.844006in}{2.811997in}}%
\pgfpathlineto{\pgfqpoint{3.840834in}{2.811903in}}%
\pgfpathlineto{\pgfqpoint{3.837662in}{2.812061in}}%
\pgfpathlineto{\pgfqpoint{3.834490in}{2.811887in}}%
\pgfpathlineto{\pgfqpoint{3.831317in}{2.811669in}}%
\pgfpathlineto{\pgfqpoint{3.828145in}{2.811581in}}%
\pgfpathlineto{\pgfqpoint{3.824973in}{2.811629in}}%
\pgfpathlineto{\pgfqpoint{3.821801in}{2.811773in}}%
\pgfpathlineto{\pgfqpoint{3.818629in}{2.811859in}}%
\pgfpathlineto{\pgfqpoint{3.815457in}{2.812130in}}%
\pgfpathlineto{\pgfqpoint{3.812285in}{2.812359in}}%
\pgfpathlineto{\pgfqpoint{3.809113in}{2.812546in}}%
\pgfpathlineto{\pgfqpoint{3.805941in}{2.811992in}}%
\pgfpathlineto{\pgfqpoint{3.802769in}{2.811984in}}%
\pgfpathlineto{\pgfqpoint{3.799597in}{2.811685in}}%
\pgfpathlineto{\pgfqpoint{3.796425in}{2.811758in}}%
\pgfpathlineto{\pgfqpoint{3.793253in}{2.811670in}}%
\pgfpathlineto{\pgfqpoint{3.790081in}{2.811947in}}%
\pgfpathlineto{\pgfqpoint{3.786909in}{2.811957in}}%
\pgfpathlineto{\pgfqpoint{3.783737in}{2.811683in}}%
\pgfpathlineto{\pgfqpoint{3.780565in}{2.811789in}}%
\pgfpathlineto{\pgfqpoint{3.777393in}{2.811648in}}%
\pgfpathlineto{\pgfqpoint{3.774221in}{2.811588in}}%
\pgfpathlineto{\pgfqpoint{3.771049in}{2.811396in}}%
\pgfpathlineto{\pgfqpoint{3.767877in}{2.811916in}}%
\pgfpathlineto{\pgfqpoint{3.764705in}{2.811985in}}%
\pgfpathlineto{\pgfqpoint{3.761533in}{2.812302in}}%
\pgfpathlineto{\pgfqpoint{3.758361in}{2.812510in}}%
\pgfpathlineto{\pgfqpoint{3.755188in}{2.812783in}}%
\pgfpathlineto{\pgfqpoint{3.752016in}{2.813040in}}%
\pgfpathlineto{\pgfqpoint{3.748844in}{2.813028in}}%
\pgfpathlineto{\pgfqpoint{3.745672in}{2.813413in}}%
\pgfpathlineto{\pgfqpoint{3.742500in}{2.813322in}}%
\pgfpathlineto{\pgfqpoint{3.739328in}{2.813252in}}%
\pgfpathlineto{\pgfqpoint{3.736156in}{2.812536in}}%
\pgfpathlineto{\pgfqpoint{3.732984in}{2.812004in}}%
\pgfpathlineto{\pgfqpoint{3.729812in}{2.811813in}}%
\pgfpathlineto{\pgfqpoint{3.726640in}{2.811622in}}%
\pgfpathlineto{\pgfqpoint{3.723468in}{2.811999in}}%
\pgfpathlineto{\pgfqpoint{3.720296in}{2.812099in}}%
\pgfpathlineto{\pgfqpoint{3.717124in}{2.812180in}}%
\pgfpathlineto{\pgfqpoint{3.713952in}{2.812369in}}%
\pgfpathlineto{\pgfqpoint{3.710780in}{2.812919in}}%
\pgfpathlineto{\pgfqpoint{3.707608in}{2.812844in}}%
\pgfpathlineto{\pgfqpoint{3.704436in}{2.812895in}}%
\pgfpathlineto{\pgfqpoint{3.701264in}{2.812975in}}%
\pgfpathlineto{\pgfqpoint{3.698092in}{2.812927in}}%
\pgfpathlineto{\pgfqpoint{3.694920in}{2.812849in}}%
\pgfpathlineto{\pgfqpoint{3.691748in}{2.812613in}}%
\pgfpathlineto{\pgfqpoint{3.688576in}{2.812870in}}%
\pgfpathlineto{\pgfqpoint{3.685404in}{2.812373in}}%
\pgfpathlineto{\pgfqpoint{3.682232in}{2.812356in}}%
\pgfpathlineto{\pgfqpoint{3.679060in}{2.811935in}}%
\pgfpathlineto{\pgfqpoint{3.675887in}{2.811423in}}%
\pgfpathlineto{\pgfqpoint{3.672715in}{2.811579in}}%
\pgfpathlineto{\pgfqpoint{3.669543in}{2.810830in}}%
\pgfpathlineto{\pgfqpoint{3.666371in}{2.810270in}}%
\pgfpathlineto{\pgfqpoint{3.663199in}{2.810224in}}%
\pgfpathlineto{\pgfqpoint{3.660027in}{2.810579in}}%
\pgfpathlineto{\pgfqpoint{3.656855in}{2.810827in}}%
\pgfpathlineto{\pgfqpoint{3.653683in}{2.810790in}}%
\pgfpathlineto{\pgfqpoint{3.650511in}{2.810548in}}%
\pgfpathlineto{\pgfqpoint{3.647339in}{2.810047in}}%
\pgfpathlineto{\pgfqpoint{3.644167in}{2.809777in}}%
\pgfpathlineto{\pgfqpoint{3.640995in}{2.809707in}}%
\pgfpathlineto{\pgfqpoint{3.637823in}{2.809349in}}%
\pgfpathlineto{\pgfqpoint{3.634651in}{2.808666in}}%
\pgfpathlineto{\pgfqpoint{3.631479in}{2.808859in}}%
\pgfpathlineto{\pgfqpoint{3.628307in}{2.808752in}}%
\pgfpathlineto{\pgfqpoint{3.625135in}{2.808322in}}%
\pgfpathlineto{\pgfqpoint{3.621963in}{2.807891in}}%
\pgfpathlineto{\pgfqpoint{3.618791in}{2.807859in}}%
\pgfpathlineto{\pgfqpoint{3.615619in}{2.807682in}}%
\pgfpathlineto{\pgfqpoint{3.612447in}{2.807751in}}%
\pgfpathlineto{\pgfqpoint{3.609275in}{2.807526in}}%
\pgfpathlineto{\pgfqpoint{3.606103in}{2.807376in}}%
\pgfpathlineto{\pgfqpoint{3.602931in}{2.807219in}}%
\pgfpathlineto{\pgfqpoint{3.599759in}{2.807434in}}%
\pgfpathlineto{\pgfqpoint{3.596586in}{2.807006in}}%
\pgfpathlineto{\pgfqpoint{3.593414in}{2.806935in}}%
\pgfpathlineto{\pgfqpoint{3.590242in}{2.806622in}}%
\pgfpathlineto{\pgfqpoint{3.587070in}{2.806419in}}%
\pgfpathlineto{\pgfqpoint{3.583898in}{2.806101in}}%
\pgfpathlineto{\pgfqpoint{3.580726in}{2.805466in}}%
\pgfpathlineto{\pgfqpoint{3.577554in}{2.805677in}}%
\pgfpathlineto{\pgfqpoint{3.574382in}{2.805553in}}%
\pgfpathlineto{\pgfqpoint{3.571210in}{2.805659in}}%
\pgfpathlineto{\pgfqpoint{3.568038in}{2.805345in}}%
\pgfpathlineto{\pgfqpoint{3.564866in}{2.804636in}}%
\pgfpathlineto{\pgfqpoint{3.561694in}{2.804916in}}%
\pgfpathlineto{\pgfqpoint{3.558522in}{2.805362in}}%
\pgfpathlineto{\pgfqpoint{3.555350in}{2.805897in}}%
\pgfpathlineto{\pgfqpoint{3.552178in}{2.806179in}}%
\pgfpathlineto{\pgfqpoint{3.549006in}{2.806653in}}%
\pgfpathlineto{\pgfqpoint{3.545834in}{2.806609in}}%
\pgfpathlineto{\pgfqpoint{3.542662in}{2.807188in}}%
\pgfpathlineto{\pgfqpoint{3.539490in}{2.807418in}}%
\pgfpathlineto{\pgfqpoint{3.536318in}{2.806965in}}%
\pgfpathlineto{\pgfqpoint{3.533146in}{2.806810in}}%
\pgfpathlineto{\pgfqpoint{3.529974in}{2.806489in}}%
\pgfpathlineto{\pgfqpoint{3.526802in}{2.806598in}}%
\pgfpathlineto{\pgfqpoint{3.523630in}{2.806720in}}%
\pgfpathlineto{\pgfqpoint{3.520457in}{2.806746in}}%
\pgfpathlineto{\pgfqpoint{3.517285in}{2.806317in}}%
\pgfpathlineto{\pgfqpoint{3.514113in}{2.805835in}}%
\pgfpathlineto{\pgfqpoint{3.510941in}{2.805713in}}%
\pgfpathlineto{\pgfqpoint{3.507769in}{2.805107in}}%
\pgfpathlineto{\pgfqpoint{3.504597in}{2.805080in}}%
\pgfpathlineto{\pgfqpoint{3.501425in}{2.805286in}}%
\pgfpathlineto{\pgfqpoint{3.498253in}{2.805322in}}%
\pgfpathlineto{\pgfqpoint{3.495081in}{2.805051in}}%
\pgfpathlineto{\pgfqpoint{3.491909in}{2.804862in}}%
\pgfpathlineto{\pgfqpoint{3.488737in}{2.805011in}}%
\pgfpathlineto{\pgfqpoint{3.485565in}{2.804949in}}%
\pgfpathlineto{\pgfqpoint{3.482393in}{2.805258in}}%
\pgfpathlineto{\pgfqpoint{3.479221in}{2.804831in}}%
\pgfpathlineto{\pgfqpoint{3.476049in}{2.804861in}}%
\pgfpathlineto{\pgfqpoint{3.472877in}{2.804484in}}%
\pgfpathlineto{\pgfqpoint{3.469705in}{2.804112in}}%
\pgfpathlineto{\pgfqpoint{3.466533in}{2.804625in}}%
\pgfpathlineto{\pgfqpoint{3.463361in}{2.804954in}}%
\pgfpathlineto{\pgfqpoint{3.460189in}{2.805069in}}%
\pgfpathlineto{\pgfqpoint{3.457017in}{2.804858in}}%
\pgfpathlineto{\pgfqpoint{3.453845in}{2.804968in}}%
\pgfpathlineto{\pgfqpoint{3.450673in}{2.804755in}}%
\pgfpathlineto{\pgfqpoint{3.447501in}{2.804702in}}%
\pgfpathlineto{\pgfqpoint{3.444329in}{2.804626in}}%
\pgfpathlineto{\pgfqpoint{3.441156in}{2.804705in}}%
\pgfpathlineto{\pgfqpoint{3.437984in}{2.804505in}}%
\pgfpathlineto{\pgfqpoint{3.434812in}{2.804008in}}%
\pgfpathlineto{\pgfqpoint{3.431640in}{2.803554in}}%
\pgfpathlineto{\pgfqpoint{3.428468in}{2.803307in}}%
\pgfpathlineto{\pgfqpoint{3.425296in}{2.802939in}}%
\pgfpathlineto{\pgfqpoint{3.422124in}{2.802399in}}%
\pgfpathlineto{\pgfqpoint{3.418952in}{2.802099in}}%
\pgfpathlineto{\pgfqpoint{3.415780in}{2.801940in}}%
\pgfpathlineto{\pgfqpoint{3.412608in}{2.801312in}}%
\pgfpathlineto{\pgfqpoint{3.409436in}{2.801213in}}%
\pgfpathlineto{\pgfqpoint{3.406264in}{2.800928in}}%
\pgfpathlineto{\pgfqpoint{3.403092in}{2.800779in}}%
\pgfpathlineto{\pgfqpoint{3.399920in}{2.800663in}}%
\pgfpathlineto{\pgfqpoint{3.396748in}{2.799997in}}%
\pgfpathlineto{\pgfqpoint{3.393576in}{2.800219in}}%
\pgfpathlineto{\pgfqpoint{3.390404in}{2.799795in}}%
\pgfpathlineto{\pgfqpoint{3.387232in}{2.799760in}}%
\pgfpathlineto{\pgfqpoint{3.384060in}{2.800297in}}%
\pgfpathlineto{\pgfqpoint{3.380888in}{2.800415in}}%
\pgfpathlineto{\pgfqpoint{3.377716in}{2.799594in}}%
\pgfpathlineto{\pgfqpoint{3.374544in}{2.799177in}}%
\pgfpathlineto{\pgfqpoint{3.371372in}{2.798870in}}%
\pgfpathlineto{\pgfqpoint{3.368200in}{2.798702in}}%
\pgfpathlineto{\pgfqpoint{3.365028in}{2.798144in}}%
\pgfpathlineto{\pgfqpoint{3.361855in}{2.798018in}}%
\pgfpathlineto{\pgfqpoint{3.358683in}{2.797934in}}%
\pgfpathlineto{\pgfqpoint{3.355511in}{2.797834in}}%
\pgfpathlineto{\pgfqpoint{3.352339in}{2.798407in}}%
\pgfpathlineto{\pgfqpoint{3.349167in}{2.798398in}}%
\pgfpathlineto{\pgfqpoint{3.345995in}{2.798769in}}%
\pgfpathlineto{\pgfqpoint{3.342823in}{2.798567in}}%
\pgfpathlineto{\pgfqpoint{3.339651in}{2.798613in}}%
\pgfpathlineto{\pgfqpoint{3.336479in}{2.798545in}}%
\pgfpathlineto{\pgfqpoint{3.333307in}{2.798895in}}%
\pgfpathlineto{\pgfqpoint{3.330135in}{2.798775in}}%
\pgfpathlineto{\pgfqpoint{3.326963in}{2.798422in}}%
\pgfpathlineto{\pgfqpoint{3.323791in}{2.798175in}}%
\pgfpathlineto{\pgfqpoint{3.320619in}{2.798564in}}%
\pgfpathlineto{\pgfqpoint{3.317447in}{2.798500in}}%
\pgfpathlineto{\pgfqpoint{3.314275in}{2.798602in}}%
\pgfpathlineto{\pgfqpoint{3.311103in}{2.798115in}}%
\pgfpathlineto{\pgfqpoint{3.307931in}{2.798215in}}%
\pgfpathlineto{\pgfqpoint{3.304759in}{2.798275in}}%
\pgfpathlineto{\pgfqpoint{3.301587in}{2.798498in}}%
\pgfpathlineto{\pgfqpoint{3.298415in}{2.798025in}}%
\pgfpathlineto{\pgfqpoint{3.295243in}{2.797281in}}%
\pgfpathlineto{\pgfqpoint{3.292071in}{2.797319in}}%
\pgfpathlineto{\pgfqpoint{3.288899in}{2.797187in}}%
\pgfpathlineto{\pgfqpoint{3.285726in}{2.797665in}}%
\pgfpathlineto{\pgfqpoint{3.282554in}{2.797809in}}%
\pgfpathlineto{\pgfqpoint{3.279382in}{2.797773in}}%
\pgfpathlineto{\pgfqpoint{3.276210in}{2.797668in}}%
\pgfpathlineto{\pgfqpoint{3.273038in}{2.797918in}}%
\pgfpathlineto{\pgfqpoint{3.269866in}{2.797756in}}%
\pgfpathlineto{\pgfqpoint{3.266694in}{2.797808in}}%
\pgfpathlineto{\pgfqpoint{3.263522in}{2.796686in}}%
\pgfpathlineto{\pgfqpoint{3.260350in}{2.796644in}}%
\pgfpathlineto{\pgfqpoint{3.257178in}{2.797031in}}%
\pgfpathlineto{\pgfqpoint{3.254006in}{2.796889in}}%
\pgfpathlineto{\pgfqpoint{3.250834in}{2.796688in}}%
\pgfpathlineto{\pgfqpoint{3.247662in}{2.796759in}}%
\pgfpathlineto{\pgfqpoint{3.244490in}{2.796681in}}%
\pgfpathlineto{\pgfqpoint{3.241318in}{2.798251in}}%
\pgfpathlineto{\pgfqpoint{3.238146in}{2.798238in}}%
\pgfpathlineto{\pgfqpoint{3.234974in}{2.798505in}}%
\pgfpathlineto{\pgfqpoint{3.231802in}{2.799241in}}%
\pgfpathlineto{\pgfqpoint{3.228630in}{2.798117in}}%
\pgfpathlineto{\pgfqpoint{3.225458in}{2.797439in}}%
\pgfpathlineto{\pgfqpoint{3.222286in}{2.797080in}}%
\pgfpathlineto{\pgfqpoint{3.219114in}{2.796797in}}%
\pgfpathlineto{\pgfqpoint{3.215942in}{2.796385in}}%
\pgfpathlineto{\pgfqpoint{3.212770in}{2.796589in}}%
\pgfpathlineto{\pgfqpoint{3.209598in}{2.796573in}}%
\pgfpathlineto{\pgfqpoint{3.206425in}{2.796243in}}%
\pgfpathlineto{\pgfqpoint{3.203253in}{2.795770in}}%
\pgfpathlineto{\pgfqpoint{3.200081in}{2.795780in}}%
\pgfpathlineto{\pgfqpoint{3.196909in}{2.795935in}}%
\pgfpathlineto{\pgfqpoint{3.193737in}{2.795684in}}%
\pgfpathlineto{\pgfqpoint{3.190565in}{2.795278in}}%
\pgfpathlineto{\pgfqpoint{3.187393in}{2.795465in}}%
\pgfpathlineto{\pgfqpoint{3.184221in}{2.794584in}}%
\pgfpathlineto{\pgfqpoint{3.181049in}{2.794559in}}%
\pgfpathlineto{\pgfqpoint{3.177877in}{2.793764in}}%
\pgfpathlineto{\pgfqpoint{3.174705in}{2.793378in}}%
\pgfpathlineto{\pgfqpoint{3.171533in}{2.793688in}}%
\pgfpathlineto{\pgfqpoint{3.168361in}{2.793634in}}%
\pgfpathlineto{\pgfqpoint{3.165189in}{2.794114in}}%
\pgfpathlineto{\pgfqpoint{3.162017in}{2.794033in}}%
\pgfpathlineto{\pgfqpoint{3.158845in}{2.793998in}}%
\pgfpathlineto{\pgfqpoint{3.155673in}{2.793701in}}%
\pgfpathlineto{\pgfqpoint{3.152501in}{2.793587in}}%
\pgfpathlineto{\pgfqpoint{3.149329in}{2.792826in}}%
\pgfpathlineto{\pgfqpoint{3.146157in}{2.792718in}}%
\pgfpathlineto{\pgfqpoint{3.142985in}{2.792651in}}%
\pgfpathlineto{\pgfqpoint{3.139813in}{2.792854in}}%
\pgfpathlineto{\pgfqpoint{3.136641in}{2.793095in}}%
\pgfpathlineto{\pgfqpoint{3.133469in}{2.793118in}}%
\pgfpathlineto{\pgfqpoint{3.130297in}{2.793087in}}%
\pgfpathlineto{\pgfqpoint{3.127124in}{2.793068in}}%
\pgfpathlineto{\pgfqpoint{3.123952in}{2.792926in}}%
\pgfpathlineto{\pgfqpoint{3.120780in}{2.792855in}}%
\pgfpathlineto{\pgfqpoint{3.117608in}{2.792885in}}%
\pgfpathlineto{\pgfqpoint{3.114436in}{2.793053in}}%
\pgfpathlineto{\pgfqpoint{3.111264in}{2.792973in}}%
\pgfpathlineto{\pgfqpoint{3.108092in}{2.792874in}}%
\pgfpathlineto{\pgfqpoint{3.104920in}{2.792795in}}%
\pgfpathlineto{\pgfqpoint{3.101748in}{2.792982in}}%
\pgfpathlineto{\pgfqpoint{3.098576in}{2.793000in}}%
\pgfpathlineto{\pgfqpoint{3.095404in}{2.792799in}}%
\pgfpathlineto{\pgfqpoint{3.092232in}{2.792754in}}%
\pgfpathlineto{\pgfqpoint{3.089060in}{2.792598in}}%
\pgfpathlineto{\pgfqpoint{3.085888in}{2.792503in}}%
\pgfpathlineto{\pgfqpoint{3.082716in}{2.792337in}}%
\pgfpathlineto{\pgfqpoint{3.079544in}{2.792361in}}%
\pgfpathlineto{\pgfqpoint{3.076372in}{2.792064in}}%
\pgfpathlineto{\pgfqpoint{3.073200in}{2.791970in}}%
\pgfpathlineto{\pgfqpoint{3.070028in}{2.791993in}}%
\pgfpathlineto{\pgfqpoint{3.066856in}{2.792150in}}%
\pgfpathlineto{\pgfqpoint{3.063684in}{2.792164in}}%
\pgfpathlineto{\pgfqpoint{3.060512in}{2.792268in}}%
\pgfpathlineto{\pgfqpoint{3.057340in}{2.792290in}}%
\pgfpathlineto{\pgfqpoint{3.054168in}{2.792380in}}%
\pgfpathlineto{\pgfqpoint{3.050995in}{2.792362in}}%
\pgfpathlineto{\pgfqpoint{3.047823in}{2.792153in}}%
\pgfpathlineto{\pgfqpoint{3.044651in}{2.792201in}}%
\pgfpathlineto{\pgfqpoint{3.041479in}{2.792156in}}%
\pgfpathlineto{\pgfqpoint{3.038307in}{2.792170in}}%
\pgfpathlineto{\pgfqpoint{3.035135in}{2.791789in}}%
\pgfpathlineto{\pgfqpoint{3.031963in}{2.791859in}}%
\pgfpathlineto{\pgfqpoint{3.028791in}{2.791871in}}%
\pgfpathlineto{\pgfqpoint{3.025619in}{2.791967in}}%
\pgfpathlineto{\pgfqpoint{3.022447in}{2.791980in}}%
\pgfpathlineto{\pgfqpoint{3.019275in}{2.791955in}}%
\pgfpathlineto{\pgfqpoint{3.016103in}{2.791944in}}%
\pgfpathlineto{\pgfqpoint{3.012931in}{2.791877in}}%
\pgfpathlineto{\pgfqpoint{3.009759in}{2.791893in}}%
\pgfpathlineto{\pgfqpoint{3.006587in}{2.791754in}}%
\pgfpathlineto{\pgfqpoint{3.003415in}{2.791583in}}%
\pgfpathlineto{\pgfqpoint{3.000243in}{2.791572in}}%
\pgfpathlineto{\pgfqpoint{2.997071in}{2.791651in}}%
\pgfpathlineto{\pgfqpoint{2.993899in}{2.791585in}}%
\pgfpathlineto{\pgfqpoint{2.990727in}{2.791656in}}%
\pgfpathlineto{\pgfqpoint{2.987555in}{2.791573in}}%
\pgfpathlineto{\pgfqpoint{2.984383in}{2.791377in}}%
\pgfpathlineto{\pgfqpoint{2.981211in}{2.791397in}}%
\pgfpathlineto{\pgfqpoint{2.978039in}{2.791255in}}%
\pgfpathlineto{\pgfqpoint{2.974867in}{2.791137in}}%
\pgfpathlineto{\pgfqpoint{2.971694in}{2.791140in}}%
\pgfpathlineto{\pgfqpoint{2.968522in}{2.791123in}}%
\pgfpathlineto{\pgfqpoint{2.965350in}{2.791107in}}%
\pgfpathlineto{\pgfqpoint{2.962178in}{2.791003in}}%
\pgfpathlineto{\pgfqpoint{2.959006in}{2.790793in}}%
\pgfpathlineto{\pgfqpoint{2.955834in}{2.790640in}}%
\pgfpathlineto{\pgfqpoint{2.952662in}{2.790558in}}%
\pgfpathlineto{\pgfqpoint{2.949490in}{2.790366in}}%
\pgfpathlineto{\pgfqpoint{2.946318in}{2.790186in}}%
\pgfpathlineto{\pgfqpoint{2.943146in}{2.790262in}}%
\pgfpathlineto{\pgfqpoint{2.939974in}{2.790039in}}%
\pgfpathlineto{\pgfqpoint{2.936802in}{2.789869in}}%
\pgfpathlineto{\pgfqpoint{2.933630in}{2.789889in}}%
\pgfpathlineto{\pgfqpoint{2.930458in}{2.789971in}}%
\pgfpathlineto{\pgfqpoint{2.927286in}{2.789816in}}%
\pgfpathlineto{\pgfqpoint{2.924114in}{2.789822in}}%
\pgfpathlineto{\pgfqpoint{2.920942in}{2.789912in}}%
\pgfpathlineto{\pgfqpoint{2.917770in}{2.789854in}}%
\pgfpathlineto{\pgfqpoint{2.914598in}{2.789723in}}%
\pgfpathlineto{\pgfqpoint{2.911426in}{2.789781in}}%
\pgfpathlineto{\pgfqpoint{2.908254in}{2.789871in}}%
\pgfpathlineto{\pgfqpoint{2.905082in}{2.789842in}}%
\pgfpathlineto{\pgfqpoint{2.901910in}{2.789737in}}%
\pgfpathlineto{\pgfqpoint{2.898738in}{2.789582in}}%
\pgfpathlineto{\pgfqpoint{2.895565in}{2.789701in}}%
\pgfpathlineto{\pgfqpoint{2.892393in}{2.789647in}}%
\pgfpathlineto{\pgfqpoint{2.889221in}{2.789645in}}%
\pgfpathlineto{\pgfqpoint{2.886049in}{2.789432in}}%
\pgfpathlineto{\pgfqpoint{2.882877in}{2.789513in}}%
\pgfpathlineto{\pgfqpoint{2.879705in}{2.789588in}}%
\pgfpathlineto{\pgfqpoint{2.876533in}{2.789633in}}%
\pgfpathlineto{\pgfqpoint{2.873361in}{2.789535in}}%
\pgfpathlineto{\pgfqpoint{2.870189in}{2.789718in}}%
\pgfpathlineto{\pgfqpoint{2.867017in}{2.789727in}}%
\pgfpathlineto{\pgfqpoint{2.863845in}{2.789862in}}%
\pgfpathlineto{\pgfqpoint{2.860673in}{2.789626in}}%
\pgfpathlineto{\pgfqpoint{2.857501in}{2.789668in}}%
\pgfpathlineto{\pgfqpoint{2.854329in}{2.789832in}}%
\pgfpathlineto{\pgfqpoint{2.851157in}{2.789877in}}%
\pgfpathlineto{\pgfqpoint{2.847985in}{2.789788in}}%
\pgfpathlineto{\pgfqpoint{2.844813in}{2.789754in}}%
\pgfpathlineto{\pgfqpoint{2.841641in}{2.789908in}}%
\pgfpathlineto{\pgfqpoint{2.838469in}{2.789975in}}%
\pgfpathlineto{\pgfqpoint{2.835297in}{2.789845in}}%
\pgfpathlineto{\pgfqpoint{2.832125in}{2.789746in}}%
\pgfpathlineto{\pgfqpoint{2.828953in}{2.789806in}}%
\pgfpathlineto{\pgfqpoint{2.825781in}{2.789757in}}%
\pgfpathlineto{\pgfqpoint{2.822609in}{2.789466in}}%
\pgfpathlineto{\pgfqpoint{2.819437in}{2.789435in}}%
\pgfpathlineto{\pgfqpoint{2.816264in}{2.789434in}}%
\pgfpathlineto{\pgfqpoint{2.813092in}{2.789394in}}%
\pgfpathlineto{\pgfqpoint{2.809920in}{2.789232in}}%
\pgfpathlineto{\pgfqpoint{2.806748in}{2.789011in}}%
\pgfpathlineto{\pgfqpoint{2.803576in}{2.788943in}}%
\pgfpathlineto{\pgfqpoint{2.800404in}{2.789102in}}%
\pgfpathlineto{\pgfqpoint{2.797232in}{2.788955in}}%
\pgfpathlineto{\pgfqpoint{2.794060in}{2.788909in}}%
\pgfpathlineto{\pgfqpoint{2.790888in}{2.788838in}}%
\pgfpathlineto{\pgfqpoint{2.787716in}{2.788858in}}%
\pgfpathlineto{\pgfqpoint{2.784544in}{2.788992in}}%
\pgfpathlineto{\pgfqpoint{2.781372in}{2.788969in}}%
\pgfpathlineto{\pgfqpoint{2.778200in}{2.788860in}}%
\pgfpathlineto{\pgfqpoint{2.775028in}{2.788838in}}%
\pgfpathlineto{\pgfqpoint{2.771856in}{2.788834in}}%
\pgfpathlineto{\pgfqpoint{2.768684in}{2.788804in}}%
\pgfpathlineto{\pgfqpoint{2.765512in}{2.788701in}}%
\pgfpathlineto{\pgfqpoint{2.762340in}{2.788957in}}%
\pgfpathlineto{\pgfqpoint{2.759168in}{2.788740in}}%
\pgfpathlineto{\pgfqpoint{2.755996in}{2.789023in}}%
\pgfpathlineto{\pgfqpoint{2.752824in}{2.789059in}}%
\pgfpathlineto{\pgfqpoint{2.749652in}{2.789012in}}%
\pgfpathlineto{\pgfqpoint{2.746480in}{2.789046in}}%
\pgfpathlineto{\pgfqpoint{2.743308in}{2.789153in}}%
\pgfpathlineto{\pgfqpoint{2.740136in}{2.789156in}}%
\pgfpathlineto{\pgfqpoint{2.736963in}{2.789044in}}%
\pgfpathlineto{\pgfqpoint{2.733791in}{2.788990in}}%
\pgfpathlineto{\pgfqpoint{2.730619in}{2.788649in}}%
\pgfpathlineto{\pgfqpoint{2.727447in}{2.788552in}}%
\pgfpathlineto{\pgfqpoint{2.724275in}{2.787885in}}%
\pgfpathlineto{\pgfqpoint{2.721103in}{2.787761in}}%
\pgfpathlineto{\pgfqpoint{2.717931in}{2.787741in}}%
\pgfpathlineto{\pgfqpoint{2.714759in}{2.787764in}}%
\pgfpathlineto{\pgfqpoint{2.711587in}{2.787602in}}%
\pgfpathlineto{\pgfqpoint{2.708415in}{2.787521in}}%
\pgfpathlineto{\pgfqpoint{2.705243in}{2.787353in}}%
\pgfpathlineto{\pgfqpoint{2.702071in}{2.787387in}}%
\pgfpathlineto{\pgfqpoint{2.698899in}{2.787611in}}%
\pgfpathlineto{\pgfqpoint{2.695727in}{2.787687in}}%
\pgfpathlineto{\pgfqpoint{2.692555in}{2.787795in}}%
\pgfpathlineto{\pgfqpoint{2.689383in}{2.787815in}}%
\pgfpathlineto{\pgfqpoint{2.686211in}{2.787762in}}%
\pgfpathlineto{\pgfqpoint{2.683039in}{2.787720in}}%
\pgfpathlineto{\pgfqpoint{2.679867in}{2.787736in}}%
\pgfpathlineto{\pgfqpoint{2.676695in}{2.787740in}}%
\pgfpathlineto{\pgfqpoint{2.673523in}{2.787757in}}%
\pgfpathlineto{\pgfqpoint{2.670351in}{2.787782in}}%
\pgfpathlineto{\pgfqpoint{2.667179in}{2.787681in}}%
\pgfpathlineto{\pgfqpoint{2.664007in}{2.787585in}}%
\pgfpathlineto{\pgfqpoint{2.660834in}{2.787459in}}%
\pgfpathlineto{\pgfqpoint{2.657662in}{2.787514in}}%
\pgfpathlineto{\pgfqpoint{2.654490in}{2.787532in}}%
\pgfpathlineto{\pgfqpoint{2.651318in}{2.787740in}}%
\pgfpathlineto{\pgfqpoint{2.648146in}{2.787846in}}%
\pgfpathlineto{\pgfqpoint{2.644974in}{2.787728in}}%
\pgfpathlineto{\pgfqpoint{2.641802in}{2.787509in}}%
\pgfpathlineto{\pgfqpoint{2.638630in}{2.787351in}}%
\pgfpathlineto{\pgfqpoint{2.635458in}{2.787365in}}%
\pgfpathlineto{\pgfqpoint{2.632286in}{2.787413in}}%
\pgfpathlineto{\pgfqpoint{2.629114in}{2.787385in}}%
\pgfpathlineto{\pgfqpoint{2.625942in}{2.787382in}}%
\pgfpathlineto{\pgfqpoint{2.622770in}{2.787214in}}%
\pgfpathlineto{\pgfqpoint{2.619598in}{2.787282in}}%
\pgfpathlineto{\pgfqpoint{2.616426in}{2.787270in}}%
\pgfpathlineto{\pgfqpoint{2.613254in}{2.787256in}}%
\pgfpathlineto{\pgfqpoint{2.610082in}{2.787292in}}%
\pgfpathlineto{\pgfqpoint{2.606910in}{2.787205in}}%
\pgfpathlineto{\pgfqpoint{2.603738in}{2.787196in}}%
\pgfpathlineto{\pgfqpoint{2.600566in}{2.787204in}}%
\pgfpathlineto{\pgfqpoint{2.597394in}{2.787138in}}%
\pgfpathlineto{\pgfqpoint{2.594222in}{2.787255in}}%
\pgfpathlineto{\pgfqpoint{2.591050in}{2.787064in}}%
\pgfpathlineto{\pgfqpoint{2.587878in}{2.786934in}}%
\pgfpathlineto{\pgfqpoint{2.584706in}{2.786920in}}%
\pgfpathlineto{\pgfqpoint{2.581533in}{2.786821in}}%
\pgfpathlineto{\pgfqpoint{2.578361in}{2.786756in}}%
\pgfpathlineto{\pgfqpoint{2.575189in}{2.786443in}}%
\pgfpathlineto{\pgfqpoint{2.572017in}{2.786404in}}%
\pgfpathlineto{\pgfqpoint{2.568845in}{2.786235in}}%
\pgfpathlineto{\pgfqpoint{2.565673in}{2.785979in}}%
\pgfpathlineto{\pgfqpoint{2.562501in}{2.786076in}}%
\pgfpathlineto{\pgfqpoint{2.559329in}{2.785808in}}%
\pgfpathlineto{\pgfqpoint{2.556157in}{2.785908in}}%
\pgfpathlineto{\pgfqpoint{2.552985in}{2.785898in}}%
\pgfpathlineto{\pgfqpoint{2.549813in}{2.785841in}}%
\pgfpathlineto{\pgfqpoint{2.546641in}{2.785783in}}%
\pgfpathlineto{\pgfqpoint{2.543469in}{2.785870in}}%
\pgfpathlineto{\pgfqpoint{2.540297in}{2.785917in}}%
\pgfpathlineto{\pgfqpoint{2.537125in}{2.785968in}}%
\pgfpathlineto{\pgfqpoint{2.533953in}{2.786178in}}%
\pgfpathlineto{\pgfqpoint{2.530781in}{2.786074in}}%
\pgfpathlineto{\pgfqpoint{2.527609in}{2.786027in}}%
\pgfpathlineto{\pgfqpoint{2.524437in}{2.785980in}}%
\pgfpathlineto{\pgfqpoint{2.521265in}{2.786089in}}%
\pgfpathlineto{\pgfqpoint{2.518093in}{2.786057in}}%
\pgfpathlineto{\pgfqpoint{2.514921in}{2.786043in}}%
\pgfpathlineto{\pgfqpoint{2.511749in}{2.785891in}}%
\pgfpathlineto{\pgfqpoint{2.508577in}{2.785968in}}%
\pgfpathlineto{\pgfqpoint{2.505405in}{2.786083in}}%
\pgfpathlineto{\pgfqpoint{2.502232in}{2.786174in}}%
\pgfpathlineto{\pgfqpoint{2.499060in}{2.786136in}}%
\pgfpathlineto{\pgfqpoint{2.495888in}{2.786063in}}%
\pgfpathlineto{\pgfqpoint{2.492716in}{2.786084in}}%
\pgfpathlineto{\pgfqpoint{2.489544in}{2.786001in}}%
\pgfpathlineto{\pgfqpoint{2.486372in}{2.785873in}}%
\pgfpathlineto{\pgfqpoint{2.483200in}{2.785735in}}%
\pgfpathlineto{\pgfqpoint{2.480028in}{2.785608in}}%
\pgfpathlineto{\pgfqpoint{2.476856in}{2.785183in}}%
\pgfpathlineto{\pgfqpoint{2.473684in}{2.785174in}}%
\pgfpathlineto{\pgfqpoint{2.470512in}{2.785171in}}%
\pgfpathlineto{\pgfqpoint{2.467340in}{2.784917in}}%
\pgfpathlineto{\pgfqpoint{2.464168in}{2.784839in}}%
\pgfpathlineto{\pgfqpoint{2.460996in}{2.784811in}}%
\pgfpathlineto{\pgfqpoint{2.457824in}{2.784809in}}%
\pgfpathlineto{\pgfqpoint{2.454652in}{2.784815in}}%
\pgfpathlineto{\pgfqpoint{2.451480in}{2.784847in}}%
\pgfpathlineto{\pgfqpoint{2.448308in}{2.784742in}}%
\pgfpathlineto{\pgfqpoint{2.445136in}{2.784537in}}%
\pgfpathlineto{\pgfqpoint{2.441964in}{2.784621in}}%
\pgfpathlineto{\pgfqpoint{2.438792in}{2.784545in}}%
\pgfpathlineto{\pgfqpoint{2.435620in}{2.784621in}}%
\pgfpathlineto{\pgfqpoint{2.432448in}{2.784517in}}%
\pgfpathlineto{\pgfqpoint{2.429276in}{2.784557in}}%
\pgfpathlineto{\pgfqpoint{2.426103in}{2.784744in}}%
\pgfpathlineto{\pgfqpoint{2.422931in}{2.784741in}}%
\pgfpathlineto{\pgfqpoint{2.419759in}{2.784664in}}%
\pgfpathlineto{\pgfqpoint{2.416587in}{2.784354in}}%
\pgfpathlineto{\pgfqpoint{2.413415in}{2.784180in}}%
\pgfpathlineto{\pgfqpoint{2.410243in}{2.783969in}}%
\pgfpathlineto{\pgfqpoint{2.407071in}{2.784022in}}%
\pgfpathlineto{\pgfqpoint{2.403899in}{2.784062in}}%
\pgfpathlineto{\pgfqpoint{2.400727in}{2.783978in}}%
\pgfpathlineto{\pgfqpoint{2.397555in}{2.783812in}}%
\pgfpathlineto{\pgfqpoint{2.394383in}{2.783867in}}%
\pgfpathlineto{\pgfqpoint{2.391211in}{2.783850in}}%
\pgfpathlineto{\pgfqpoint{2.388039in}{2.783495in}}%
\pgfpathlineto{\pgfqpoint{2.384867in}{2.783474in}}%
\pgfpathlineto{\pgfqpoint{2.381695in}{2.783539in}}%
\pgfpathlineto{\pgfqpoint{2.378523in}{2.783825in}}%
\pgfpathlineto{\pgfqpoint{2.375351in}{2.783543in}}%
\pgfpathlineto{\pgfqpoint{2.372179in}{2.783068in}}%
\pgfpathlineto{\pgfqpoint{2.369007in}{2.772332in}}%
\pgfpathlineto{\pgfqpoint{2.365835in}{2.761803in}}%
\pgfpathlineto{\pgfqpoint{2.362663in}{2.751218in}}%
\pgfpathlineto{\pgfqpoint{2.359491in}{2.740825in}}%
\pgfpathlineto{\pgfqpoint{2.356319in}{2.730231in}}%
\pgfpathlineto{\pgfqpoint{2.353147in}{2.719855in}}%
\pgfpathlineto{\pgfqpoint{2.349975in}{2.709328in}}%
\pgfpathlineto{\pgfqpoint{2.346802in}{2.698946in}}%
\pgfpathlineto{\pgfqpoint{2.343630in}{2.688488in}}%
\pgfpathlineto{\pgfqpoint{2.340458in}{2.678200in}}%
\pgfpathlineto{\pgfqpoint{2.337286in}{2.667179in}}%
\pgfpathlineto{\pgfqpoint{2.334114in}{2.656370in}}%
\pgfpathlineto{\pgfqpoint{2.330942in}{2.644752in}}%
\pgfpathlineto{\pgfqpoint{2.327770in}{2.632988in}}%
\pgfpathlineto{\pgfqpoint{2.324598in}{2.621665in}}%
\pgfpathlineto{\pgfqpoint{2.321426in}{2.609548in}}%
\pgfpathlineto{\pgfqpoint{2.318254in}{2.597703in}}%
\pgfpathlineto{\pgfqpoint{2.315082in}{2.586115in}}%
\pgfpathlineto{\pgfqpoint{2.311910in}{2.573811in}}%
\pgfpathlineto{\pgfqpoint{2.308738in}{2.561915in}}%
\pgfpathlineto{\pgfqpoint{2.305566in}{2.550525in}}%
\pgfpathlineto{\pgfqpoint{2.302394in}{2.539234in}}%
\pgfpathlineto{\pgfqpoint{2.299222in}{2.527608in}}%
\pgfpathlineto{\pgfqpoint{2.296050in}{2.516148in}}%
\pgfpathlineto{\pgfqpoint{2.292878in}{2.504337in}}%
\pgfpathlineto{\pgfqpoint{2.289706in}{2.492771in}}%
\pgfpathlineto{\pgfqpoint{2.286534in}{2.481445in}}%
\pgfpathlineto{\pgfqpoint{2.283362in}{2.469945in}}%
\pgfpathlineto{\pgfqpoint{2.280190in}{2.457981in}}%
\pgfpathlineto{\pgfqpoint{2.277018in}{2.446998in}}%
\pgfpathlineto{\pgfqpoint{2.273846in}{2.435773in}}%
\pgfpathlineto{\pgfqpoint{2.270674in}{2.424090in}}%
\pgfpathlineto{\pgfqpoint{2.267501in}{2.412215in}}%
\pgfpathlineto{\pgfqpoint{2.264329in}{2.400000in}}%
\pgfpathlineto{\pgfqpoint{2.261157in}{2.388395in}}%
\pgfpathlineto{\pgfqpoint{2.257985in}{2.376980in}}%
\pgfpathlineto{\pgfqpoint{2.254813in}{2.364833in}}%
\pgfpathlineto{\pgfqpoint{2.251641in}{2.352951in}}%
\pgfpathlineto{\pgfqpoint{2.248469in}{2.341639in}}%
\pgfpathlineto{\pgfqpoint{2.245297in}{2.330452in}}%
\pgfpathlineto{\pgfqpoint{2.242125in}{2.318695in}}%
\pgfpathlineto{\pgfqpoint{2.238953in}{2.306942in}}%
\pgfpathlineto{\pgfqpoint{2.235781in}{2.294921in}}%
\pgfpathlineto{\pgfqpoint{2.232609in}{2.283451in}}%
\pgfpathlineto{\pgfqpoint{2.229437in}{2.271698in}}%
\pgfpathlineto{\pgfqpoint{2.226265in}{2.260087in}}%
\pgfpathlineto{\pgfqpoint{2.223093in}{2.248381in}}%
\pgfpathlineto{\pgfqpoint{2.219921in}{2.236395in}}%
\pgfpathlineto{\pgfqpoint{2.216749in}{2.224747in}}%
\pgfpathlineto{\pgfqpoint{2.213577in}{2.213047in}}%
\pgfpathlineto{\pgfqpoint{2.210405in}{2.201668in}}%
\pgfpathlineto{\pgfqpoint{2.207233in}{2.189654in}}%
\pgfpathlineto{\pgfqpoint{2.204061in}{2.177620in}}%
\pgfpathlineto{\pgfqpoint{2.200889in}{2.165914in}}%
\pgfpathlineto{\pgfqpoint{2.197717in}{2.153755in}}%
\pgfpathlineto{\pgfqpoint{2.194545in}{2.142533in}}%
\pgfpathlineto{\pgfqpoint{2.191372in}{2.130550in}}%
\pgfpathlineto{\pgfqpoint{2.188200in}{2.119037in}}%
\pgfpathlineto{\pgfqpoint{2.185028in}{2.107448in}}%
\pgfpathlineto{\pgfqpoint{2.181856in}{2.095898in}}%
\pgfpathlineto{\pgfqpoint{2.178684in}{2.083666in}}%
\pgfpathlineto{\pgfqpoint{2.175512in}{2.071768in}}%
\pgfpathlineto{\pgfqpoint{2.172340in}{2.059936in}}%
\pgfpathlineto{\pgfqpoint{2.169168in}{2.048599in}}%
\pgfpathlineto{\pgfqpoint{2.165996in}{2.037157in}}%
\pgfpathlineto{\pgfqpoint{2.162824in}{2.025486in}}%
\pgfpathlineto{\pgfqpoint{2.159652in}{2.013765in}}%
\pgfpathlineto{\pgfqpoint{2.156480in}{2.002088in}}%
\pgfpathlineto{\pgfqpoint{2.153308in}{1.990593in}}%
\pgfpathlineto{\pgfqpoint{2.150136in}{1.978878in}}%
\pgfpathlineto{\pgfqpoint{2.146964in}{1.967063in}}%
\pgfpathlineto{\pgfqpoint{2.143792in}{1.955586in}}%
\pgfpathlineto{\pgfqpoint{2.140620in}{1.944115in}}%
\pgfpathlineto{\pgfqpoint{2.137448in}{1.932125in}}%
\pgfpathlineto{\pgfqpoint{2.134276in}{1.920317in}}%
\pgfpathlineto{\pgfqpoint{2.131104in}{1.908822in}}%
\pgfpathlineto{\pgfqpoint{2.127932in}{1.897147in}}%
\pgfpathlineto{\pgfqpoint{2.124760in}{1.885404in}}%
\pgfpathlineto{\pgfqpoint{2.121588in}{1.873974in}}%
\pgfpathlineto{\pgfqpoint{2.118416in}{1.862248in}}%
\pgfpathlineto{\pgfqpoint{2.115244in}{1.850516in}}%
\pgfpathlineto{\pgfqpoint{2.112071in}{1.838048in}}%
\pgfpathlineto{\pgfqpoint{2.108899in}{1.826086in}}%
\pgfpathlineto{\pgfqpoint{2.105727in}{1.814431in}}%
\pgfpathlineto{\pgfqpoint{2.102555in}{1.803005in}}%
\pgfpathlineto{\pgfqpoint{2.099383in}{1.791211in}}%
\pgfpathlineto{\pgfqpoint{2.096211in}{1.779838in}}%
\pgfpathlineto{\pgfqpoint{2.093039in}{1.768227in}}%
\pgfpathlineto{\pgfqpoint{2.089867in}{1.756376in}}%
\pgfpathlineto{\pgfqpoint{2.086695in}{1.744280in}}%
\pgfpathlineto{\pgfqpoint{2.083523in}{1.732700in}}%
\pgfpathlineto{\pgfqpoint{2.080351in}{1.720791in}}%
\pgfpathlineto{\pgfqpoint{2.077179in}{1.708983in}}%
\pgfpathlineto{\pgfqpoint{2.074007in}{1.697576in}}%
\pgfpathlineto{\pgfqpoint{2.070835in}{1.685560in}}%
\pgfpathlineto{\pgfqpoint{2.067663in}{1.673972in}}%
\pgfpathlineto{\pgfqpoint{2.064491in}{1.662609in}}%
\pgfpathlineto{\pgfqpoint{2.061319in}{1.650847in}}%
\pgfpathlineto{\pgfqpoint{2.058147in}{1.639349in}}%
\pgfpathlineto{\pgfqpoint{2.054975in}{1.628016in}}%
\pgfpathlineto{\pgfqpoint{2.051803in}{1.616341in}}%
\pgfpathlineto{\pgfqpoint{2.048631in}{1.604965in}}%
\pgfpathlineto{\pgfqpoint{2.045459in}{1.592325in}}%
\pgfpathlineto{\pgfqpoint{2.042287in}{1.579745in}}%
\pgfpathlineto{\pgfqpoint{2.039115in}{1.567534in}}%
\pgfpathlineto{\pgfqpoint{2.035943in}{1.556075in}}%
\pgfpathlineto{\pgfqpoint{2.032770in}{1.543567in}}%
\pgfpathlineto{\pgfqpoint{2.029598in}{1.532827in}}%
\pgfpathlineto{\pgfqpoint{2.026426in}{1.520843in}}%
\pgfpathlineto{\pgfqpoint{2.023254in}{1.510628in}}%
\pgfpathlineto{\pgfqpoint{2.020082in}{1.499495in}}%
\pgfpathlineto{\pgfqpoint{2.016910in}{1.487479in}}%
\pgfpathlineto{\pgfqpoint{2.013738in}{1.476473in}}%
\pgfpathlineto{\pgfqpoint{2.010566in}{1.465895in}}%
\pgfpathlineto{\pgfqpoint{2.007394in}{1.453822in}}%
\pgfpathlineto{\pgfqpoint{2.004222in}{1.443505in}}%
\pgfpathlineto{\pgfqpoint{2.001050in}{1.432402in}}%
\pgfpathlineto{\pgfqpoint{1.997878in}{1.421792in}}%
\pgfpathlineto{\pgfqpoint{1.994706in}{1.409778in}}%
\pgfpathlineto{\pgfqpoint{1.991534in}{1.397796in}}%
\pgfpathlineto{\pgfqpoint{1.988362in}{1.385919in}}%
\pgfpathlineto{\pgfqpoint{1.985190in}{1.373685in}}%
\pgfpathlineto{\pgfqpoint{1.982018in}{1.352266in}}%
\pgfpathlineto{\pgfqpoint{1.978846in}{1.332087in}}%
\pgfpathlineto{\pgfqpoint{1.975674in}{1.314433in}}%
\pgfpathlineto{\pgfqpoint{1.972502in}{1.294322in}}%
\pgfpathlineto{\pgfqpoint{1.969330in}{1.275415in}}%
\pgfpathlineto{\pgfqpoint{1.966158in}{1.257268in}}%
\pgfpathlineto{\pgfqpoint{1.962986in}{1.236168in}}%
\pgfpathlineto{\pgfqpoint{1.959814in}{1.218970in}}%
\pgfpathlineto{\pgfqpoint{1.956641in}{1.199884in}}%
\pgfpathlineto{\pgfqpoint{1.953469in}{1.178524in}}%
\pgfpathlineto{\pgfqpoint{1.950297in}{1.147810in}}%
\pgfpathlineto{\pgfqpoint{1.947125in}{1.129588in}}%
\pgfpathlineto{\pgfqpoint{1.943953in}{1.129315in}}%
\pgfpathlineto{\pgfqpoint{1.940781in}{1.128807in}}%
\pgfpathclose%
\pgfusepath{stroke,fill}%
\end{pgfscope}%
\begin{pgfscope}%
\pgfpathrectangle{\pgfqpoint{1.623736in}{1.000625in}}{\pgfqpoint{6.975000in}{3.020000in}} %
\pgfusepath{clip}%
\pgfsetbuttcap%
\pgfsetroundjoin%
\definecolor{currentfill}{rgb}{0.298039,0.447059,0.690196}%
\pgfsetfillcolor{currentfill}%
\pgfsetfillopacity{0.200000}%
\pgfsetlinewidth{0.803000pt}%
\definecolor{currentstroke}{rgb}{0.298039,0.447059,0.690196}%
\pgfsetstrokecolor{currentstroke}%
\pgfsetstrokeopacity{0.200000}%
\pgfsetdash{}{0pt}%
\pgfpathmoveto{\pgfqpoint{1.940781in}{1.127591in}}%
\pgfpathlineto{\pgfqpoint{1.940781in}{1.126759in}}%
\pgfpathlineto{\pgfqpoint{1.943953in}{1.125967in}}%
\pgfpathlineto{\pgfqpoint{1.947125in}{1.125823in}}%
\pgfpathlineto{\pgfqpoint{1.950297in}{1.141663in}}%
\pgfpathlineto{\pgfqpoint{1.953469in}{1.165446in}}%
\pgfpathlineto{\pgfqpoint{1.956641in}{1.185793in}}%
\pgfpathlineto{\pgfqpoint{1.959814in}{1.204481in}}%
\pgfpathlineto{\pgfqpoint{1.962986in}{1.224114in}}%
\pgfpathlineto{\pgfqpoint{1.966158in}{1.243651in}}%
\pgfpathlineto{\pgfqpoint{1.969330in}{1.263066in}}%
\pgfpathlineto{\pgfqpoint{1.972502in}{1.282590in}}%
\pgfpathlineto{\pgfqpoint{1.975674in}{1.301638in}}%
\pgfpathlineto{\pgfqpoint{1.978846in}{1.320884in}}%
\pgfpathlineto{\pgfqpoint{1.982018in}{1.342003in}}%
\pgfpathlineto{\pgfqpoint{1.985190in}{1.364247in}}%
\pgfpathlineto{\pgfqpoint{1.988362in}{1.376169in}}%
\pgfpathlineto{\pgfqpoint{1.991534in}{1.389215in}}%
\pgfpathlineto{\pgfqpoint{1.994706in}{1.400087in}}%
\pgfpathlineto{\pgfqpoint{1.997878in}{1.412277in}}%
\pgfpathlineto{\pgfqpoint{2.001050in}{1.424352in}}%
\pgfpathlineto{\pgfqpoint{2.004222in}{1.436257in}}%
\pgfpathlineto{\pgfqpoint{2.007394in}{1.447924in}}%
\pgfpathlineto{\pgfqpoint{2.010566in}{1.459661in}}%
\pgfpathlineto{\pgfqpoint{2.013738in}{1.472165in}}%
\pgfpathlineto{\pgfqpoint{2.016910in}{1.484688in}}%
\pgfpathlineto{\pgfqpoint{2.020082in}{1.496823in}}%
\pgfpathlineto{\pgfqpoint{2.023254in}{1.509252in}}%
\pgfpathlineto{\pgfqpoint{2.026426in}{1.520290in}}%
\pgfpathlineto{\pgfqpoint{2.029598in}{1.531492in}}%
\pgfpathlineto{\pgfqpoint{2.032770in}{1.542724in}}%
\pgfpathlineto{\pgfqpoint{2.035943in}{1.554910in}}%
\pgfpathlineto{\pgfqpoint{2.039115in}{1.566651in}}%
\pgfpathlineto{\pgfqpoint{2.042287in}{1.578175in}}%
\pgfpathlineto{\pgfqpoint{2.045459in}{1.590725in}}%
\pgfpathlineto{\pgfqpoint{2.048631in}{1.600776in}}%
\pgfpathlineto{\pgfqpoint{2.051803in}{1.612610in}}%
\pgfpathlineto{\pgfqpoint{2.054975in}{1.623705in}}%
\pgfpathlineto{\pgfqpoint{2.058147in}{1.635379in}}%
\pgfpathlineto{\pgfqpoint{2.061319in}{1.647103in}}%
\pgfpathlineto{\pgfqpoint{2.064491in}{1.658598in}}%
\pgfpathlineto{\pgfqpoint{2.067663in}{1.670276in}}%
\pgfpathlineto{\pgfqpoint{2.070835in}{1.681755in}}%
\pgfpathlineto{\pgfqpoint{2.074007in}{1.693184in}}%
\pgfpathlineto{\pgfqpoint{2.077179in}{1.704511in}}%
\pgfpathlineto{\pgfqpoint{2.080351in}{1.716120in}}%
\pgfpathlineto{\pgfqpoint{2.083523in}{1.727568in}}%
\pgfpathlineto{\pgfqpoint{2.086695in}{1.739240in}}%
\pgfpathlineto{\pgfqpoint{2.089867in}{1.750557in}}%
\pgfpathlineto{\pgfqpoint{2.093039in}{1.762164in}}%
\pgfpathlineto{\pgfqpoint{2.096211in}{1.773761in}}%
\pgfpathlineto{\pgfqpoint{2.099383in}{1.785300in}}%
\pgfpathlineto{\pgfqpoint{2.102555in}{1.796713in}}%
\pgfpathlineto{\pgfqpoint{2.105727in}{1.808110in}}%
\pgfpathlineto{\pgfqpoint{2.108899in}{1.819925in}}%
\pgfpathlineto{\pgfqpoint{2.112071in}{1.831085in}}%
\pgfpathlineto{\pgfqpoint{2.115244in}{1.842777in}}%
\pgfpathlineto{\pgfqpoint{2.118416in}{1.854393in}}%
\pgfpathlineto{\pgfqpoint{2.121588in}{1.865948in}}%
\pgfpathlineto{\pgfqpoint{2.124760in}{1.878115in}}%
\pgfpathlineto{\pgfqpoint{2.127932in}{1.889686in}}%
\pgfpathlineto{\pgfqpoint{2.131104in}{1.901161in}}%
\pgfpathlineto{\pgfqpoint{2.134276in}{1.913242in}}%
\pgfpathlineto{\pgfqpoint{2.137448in}{1.916521in}}%
\pgfpathlineto{\pgfqpoint{2.140620in}{1.918162in}}%
\pgfpathlineto{\pgfqpoint{2.143792in}{1.919365in}}%
\pgfpathlineto{\pgfqpoint{2.146964in}{1.918182in}}%
\pgfpathlineto{\pgfqpoint{2.150136in}{1.916325in}}%
\pgfpathlineto{\pgfqpoint{2.153308in}{1.914083in}}%
\pgfpathlineto{\pgfqpoint{2.156480in}{1.914015in}}%
\pgfpathlineto{\pgfqpoint{2.159652in}{1.914060in}}%
\pgfpathlineto{\pgfqpoint{2.162824in}{1.914040in}}%
\pgfpathlineto{\pgfqpoint{2.165996in}{1.913949in}}%
\pgfpathlineto{\pgfqpoint{2.169168in}{1.914149in}}%
\pgfpathlineto{\pgfqpoint{2.172340in}{1.914076in}}%
\pgfpathlineto{\pgfqpoint{2.175512in}{1.914014in}}%
\pgfpathlineto{\pgfqpoint{2.178684in}{1.914074in}}%
\pgfpathlineto{\pgfqpoint{2.181856in}{1.914247in}}%
\pgfpathlineto{\pgfqpoint{2.185028in}{1.914218in}}%
\pgfpathlineto{\pgfqpoint{2.188200in}{1.914078in}}%
\pgfpathlineto{\pgfqpoint{2.191372in}{1.914004in}}%
\pgfpathlineto{\pgfqpoint{2.194545in}{1.914193in}}%
\pgfpathlineto{\pgfqpoint{2.197717in}{1.914169in}}%
\pgfpathlineto{\pgfqpoint{2.200889in}{1.914059in}}%
\pgfpathlineto{\pgfqpoint{2.204061in}{1.914094in}}%
\pgfpathlineto{\pgfqpoint{2.207233in}{1.913897in}}%
\pgfpathlineto{\pgfqpoint{2.210405in}{1.914190in}}%
\pgfpathlineto{\pgfqpoint{2.213577in}{1.914246in}}%
\pgfpathlineto{\pgfqpoint{2.216749in}{1.914315in}}%
\pgfpathlineto{\pgfqpoint{2.219921in}{1.914137in}}%
\pgfpathlineto{\pgfqpoint{2.223093in}{1.914400in}}%
\pgfpathlineto{\pgfqpoint{2.226265in}{1.914402in}}%
\pgfpathlineto{\pgfqpoint{2.229437in}{1.914185in}}%
\pgfpathlineto{\pgfqpoint{2.232609in}{1.914060in}}%
\pgfpathlineto{\pgfqpoint{2.235781in}{1.913923in}}%
\pgfpathlineto{\pgfqpoint{2.238953in}{1.914038in}}%
\pgfpathlineto{\pgfqpoint{2.242125in}{1.914064in}}%
\pgfpathlineto{\pgfqpoint{2.245297in}{1.914098in}}%
\pgfpathlineto{\pgfqpoint{2.248469in}{1.913963in}}%
\pgfpathlineto{\pgfqpoint{2.251641in}{1.913869in}}%
\pgfpathlineto{\pgfqpoint{2.254813in}{1.914015in}}%
\pgfpathlineto{\pgfqpoint{2.257985in}{1.914126in}}%
\pgfpathlineto{\pgfqpoint{2.261157in}{1.914035in}}%
\pgfpathlineto{\pgfqpoint{2.264329in}{1.914211in}}%
\pgfpathlineto{\pgfqpoint{2.267501in}{1.914067in}}%
\pgfpathlineto{\pgfqpoint{2.270674in}{1.913969in}}%
\pgfpathlineto{\pgfqpoint{2.273846in}{1.913932in}}%
\pgfpathlineto{\pgfqpoint{2.277018in}{1.913974in}}%
\pgfpathlineto{\pgfqpoint{2.280190in}{1.913826in}}%
\pgfpathlineto{\pgfqpoint{2.283362in}{1.914181in}}%
\pgfpathlineto{\pgfqpoint{2.286534in}{1.914153in}}%
\pgfpathlineto{\pgfqpoint{2.289706in}{1.914128in}}%
\pgfpathlineto{\pgfqpoint{2.292878in}{1.914098in}}%
\pgfpathlineto{\pgfqpoint{2.296050in}{1.914091in}}%
\pgfpathlineto{\pgfqpoint{2.299222in}{1.913966in}}%
\pgfpathlineto{\pgfqpoint{2.302394in}{1.913873in}}%
\pgfpathlineto{\pgfqpoint{2.305566in}{1.913758in}}%
\pgfpathlineto{\pgfqpoint{2.308738in}{1.913696in}}%
\pgfpathlineto{\pgfqpoint{2.311910in}{1.913570in}}%
\pgfpathlineto{\pgfqpoint{2.315082in}{1.913459in}}%
\pgfpathlineto{\pgfqpoint{2.318254in}{1.913252in}}%
\pgfpathlineto{\pgfqpoint{2.321426in}{1.913459in}}%
\pgfpathlineto{\pgfqpoint{2.324598in}{1.913685in}}%
\pgfpathlineto{\pgfqpoint{2.327770in}{1.913428in}}%
\pgfpathlineto{\pgfqpoint{2.330942in}{1.913469in}}%
\pgfpathlineto{\pgfqpoint{2.334114in}{1.913328in}}%
\pgfpathlineto{\pgfqpoint{2.337286in}{1.913293in}}%
\pgfpathlineto{\pgfqpoint{2.340458in}{1.913239in}}%
\pgfpathlineto{\pgfqpoint{2.343630in}{1.912989in}}%
\pgfpathlineto{\pgfqpoint{2.346802in}{1.912985in}}%
\pgfpathlineto{\pgfqpoint{2.349975in}{1.912790in}}%
\pgfpathlineto{\pgfqpoint{2.353147in}{1.912798in}}%
\pgfpathlineto{\pgfqpoint{2.356319in}{1.912790in}}%
\pgfpathlineto{\pgfqpoint{2.359491in}{1.913020in}}%
\pgfpathlineto{\pgfqpoint{2.362663in}{1.913005in}}%
\pgfpathlineto{\pgfqpoint{2.365835in}{1.913099in}}%
\pgfpathlineto{\pgfqpoint{2.369007in}{1.913090in}}%
\pgfpathlineto{\pgfqpoint{2.372179in}{1.912911in}}%
\pgfpathlineto{\pgfqpoint{2.375351in}{1.913051in}}%
\pgfpathlineto{\pgfqpoint{2.378523in}{1.913120in}}%
\pgfpathlineto{\pgfqpoint{2.381695in}{1.912888in}}%
\pgfpathlineto{\pgfqpoint{2.384867in}{1.912749in}}%
\pgfpathlineto{\pgfqpoint{2.388039in}{1.912719in}}%
\pgfpathlineto{\pgfqpoint{2.391211in}{1.913018in}}%
\pgfpathlineto{\pgfqpoint{2.394383in}{1.912964in}}%
\pgfpathlineto{\pgfqpoint{2.397555in}{1.912890in}}%
\pgfpathlineto{\pgfqpoint{2.400727in}{1.913172in}}%
\pgfpathlineto{\pgfqpoint{2.403899in}{1.913053in}}%
\pgfpathlineto{\pgfqpoint{2.407071in}{1.913020in}}%
\pgfpathlineto{\pgfqpoint{2.410243in}{1.913154in}}%
\pgfpathlineto{\pgfqpoint{2.413415in}{1.913017in}}%
\pgfpathlineto{\pgfqpoint{2.416587in}{1.912840in}}%
\pgfpathlineto{\pgfqpoint{2.419759in}{1.913002in}}%
\pgfpathlineto{\pgfqpoint{2.422931in}{1.913096in}}%
\pgfpathlineto{\pgfqpoint{2.426103in}{1.913217in}}%
\pgfpathlineto{\pgfqpoint{2.429276in}{1.913008in}}%
\pgfpathlineto{\pgfqpoint{2.432448in}{1.912894in}}%
\pgfpathlineto{\pgfqpoint{2.435620in}{1.912972in}}%
\pgfpathlineto{\pgfqpoint{2.438792in}{1.912984in}}%
\pgfpathlineto{\pgfqpoint{2.441964in}{1.912994in}}%
\pgfpathlineto{\pgfqpoint{2.445136in}{1.912903in}}%
\pgfpathlineto{\pgfqpoint{2.448308in}{1.912940in}}%
\pgfpathlineto{\pgfqpoint{2.451480in}{1.913070in}}%
\pgfpathlineto{\pgfqpoint{2.454652in}{1.913040in}}%
\pgfpathlineto{\pgfqpoint{2.457824in}{1.912958in}}%
\pgfpathlineto{\pgfqpoint{2.460996in}{1.913087in}}%
\pgfpathlineto{\pgfqpoint{2.464168in}{1.912991in}}%
\pgfpathlineto{\pgfqpoint{2.467340in}{1.912751in}}%
\pgfpathlineto{\pgfqpoint{2.470512in}{1.912844in}}%
\pgfpathlineto{\pgfqpoint{2.473684in}{1.912809in}}%
\pgfpathlineto{\pgfqpoint{2.476856in}{1.912691in}}%
\pgfpathlineto{\pgfqpoint{2.480028in}{1.912733in}}%
\pgfpathlineto{\pgfqpoint{2.483200in}{1.912762in}}%
\pgfpathlineto{\pgfqpoint{2.486372in}{1.913046in}}%
\pgfpathlineto{\pgfqpoint{2.489544in}{1.912987in}}%
\pgfpathlineto{\pgfqpoint{2.492716in}{1.912797in}}%
\pgfpathlineto{\pgfqpoint{2.495888in}{1.912590in}}%
\pgfpathlineto{\pgfqpoint{2.499060in}{1.912593in}}%
\pgfpathlineto{\pgfqpoint{2.502232in}{1.912871in}}%
\pgfpathlineto{\pgfqpoint{2.505405in}{1.912849in}}%
\pgfpathlineto{\pgfqpoint{2.508577in}{1.912881in}}%
\pgfpathlineto{\pgfqpoint{2.511749in}{1.912802in}}%
\pgfpathlineto{\pgfqpoint{2.514921in}{1.912786in}}%
\pgfpathlineto{\pgfqpoint{2.518093in}{1.912800in}}%
\pgfpathlineto{\pgfqpoint{2.521265in}{1.912808in}}%
\pgfpathlineto{\pgfqpoint{2.524437in}{1.913062in}}%
\pgfpathlineto{\pgfqpoint{2.527609in}{1.913163in}}%
\pgfpathlineto{\pgfqpoint{2.530781in}{1.912988in}}%
\pgfpathlineto{\pgfqpoint{2.533953in}{1.913135in}}%
\pgfpathlineto{\pgfqpoint{2.537125in}{1.912965in}}%
\pgfpathlineto{\pgfqpoint{2.540297in}{1.912884in}}%
\pgfpathlineto{\pgfqpoint{2.543469in}{1.912696in}}%
\pgfpathlineto{\pgfqpoint{2.546641in}{1.912567in}}%
\pgfpathlineto{\pgfqpoint{2.549813in}{1.912520in}}%
\pgfpathlineto{\pgfqpoint{2.552985in}{1.912440in}}%
\pgfpathlineto{\pgfqpoint{2.556157in}{1.912341in}}%
\pgfpathlineto{\pgfqpoint{2.559329in}{1.912130in}}%
\pgfpathlineto{\pgfqpoint{2.562501in}{1.912264in}}%
\pgfpathlineto{\pgfqpoint{2.565673in}{1.912248in}}%
\pgfpathlineto{\pgfqpoint{2.568845in}{1.912573in}}%
\pgfpathlineto{\pgfqpoint{2.572017in}{1.912674in}}%
\pgfpathlineto{\pgfqpoint{2.575189in}{1.912710in}}%
\pgfpathlineto{\pgfqpoint{2.578361in}{1.912917in}}%
\pgfpathlineto{\pgfqpoint{2.581533in}{1.913124in}}%
\pgfpathlineto{\pgfqpoint{2.584706in}{1.912991in}}%
\pgfpathlineto{\pgfqpoint{2.587878in}{1.912979in}}%
\pgfpathlineto{\pgfqpoint{2.591050in}{1.913081in}}%
\pgfpathlineto{\pgfqpoint{2.594222in}{1.913133in}}%
\pgfpathlineto{\pgfqpoint{2.597394in}{1.913142in}}%
\pgfpathlineto{\pgfqpoint{2.600566in}{1.913099in}}%
\pgfpathlineto{\pgfqpoint{2.603738in}{1.912957in}}%
\pgfpathlineto{\pgfqpoint{2.606910in}{1.912869in}}%
\pgfpathlineto{\pgfqpoint{2.610082in}{1.912848in}}%
\pgfpathlineto{\pgfqpoint{2.613254in}{1.912821in}}%
\pgfpathlineto{\pgfqpoint{2.616426in}{1.912668in}}%
\pgfpathlineto{\pgfqpoint{2.619598in}{1.912462in}}%
\pgfpathlineto{\pgfqpoint{2.622770in}{1.912577in}}%
\pgfpathlineto{\pgfqpoint{2.625942in}{1.912615in}}%
\pgfpathlineto{\pgfqpoint{2.629114in}{1.912623in}}%
\pgfpathlineto{\pgfqpoint{2.632286in}{1.912724in}}%
\pgfpathlineto{\pgfqpoint{2.635458in}{1.912685in}}%
\pgfpathlineto{\pgfqpoint{2.638630in}{1.912724in}}%
\pgfpathlineto{\pgfqpoint{2.641802in}{1.912698in}}%
\pgfpathlineto{\pgfqpoint{2.644974in}{1.912625in}}%
\pgfpathlineto{\pgfqpoint{2.648146in}{1.912726in}}%
\pgfpathlineto{\pgfqpoint{2.651318in}{1.912689in}}%
\pgfpathlineto{\pgfqpoint{2.654490in}{1.912494in}}%
\pgfpathlineto{\pgfqpoint{2.657662in}{1.912396in}}%
\pgfpathlineto{\pgfqpoint{2.660834in}{1.912428in}}%
\pgfpathlineto{\pgfqpoint{2.664007in}{1.912474in}}%
\pgfpathlineto{\pgfqpoint{2.667179in}{1.912464in}}%
\pgfpathlineto{\pgfqpoint{2.670351in}{1.912400in}}%
\pgfpathlineto{\pgfqpoint{2.673523in}{1.912248in}}%
\pgfpathlineto{\pgfqpoint{2.676695in}{1.912242in}}%
\pgfpathlineto{\pgfqpoint{2.679867in}{1.912162in}}%
\pgfpathlineto{\pgfqpoint{2.683039in}{1.912038in}}%
\pgfpathlineto{\pgfqpoint{2.686211in}{1.911863in}}%
\pgfpathlineto{\pgfqpoint{2.689383in}{1.911886in}}%
\pgfpathlineto{\pgfqpoint{2.692555in}{1.912008in}}%
\pgfpathlineto{\pgfqpoint{2.695727in}{1.911980in}}%
\pgfpathlineto{\pgfqpoint{2.698899in}{1.911873in}}%
\pgfpathlineto{\pgfqpoint{2.702071in}{1.911844in}}%
\pgfpathlineto{\pgfqpoint{2.705243in}{1.911854in}}%
\pgfpathlineto{\pgfqpoint{2.708415in}{1.911965in}}%
\pgfpathlineto{\pgfqpoint{2.711587in}{1.911937in}}%
\pgfpathlineto{\pgfqpoint{2.714759in}{1.912253in}}%
\pgfpathlineto{\pgfqpoint{2.717931in}{1.912243in}}%
\pgfpathlineto{\pgfqpoint{2.721103in}{1.912169in}}%
\pgfpathlineto{\pgfqpoint{2.724275in}{1.912101in}}%
\pgfpathlineto{\pgfqpoint{2.727447in}{1.912120in}}%
\pgfpathlineto{\pgfqpoint{2.730619in}{1.912084in}}%
\pgfpathlineto{\pgfqpoint{2.733791in}{1.911945in}}%
\pgfpathlineto{\pgfqpoint{2.736963in}{1.911898in}}%
\pgfpathlineto{\pgfqpoint{2.740136in}{1.911682in}}%
\pgfpathlineto{\pgfqpoint{2.743308in}{1.911600in}}%
\pgfpathlineto{\pgfqpoint{2.746480in}{1.911548in}}%
\pgfpathlineto{\pgfqpoint{2.749652in}{1.911341in}}%
\pgfpathlineto{\pgfqpoint{2.752824in}{1.911289in}}%
\pgfpathlineto{\pgfqpoint{2.755996in}{1.911337in}}%
\pgfpathlineto{\pgfqpoint{2.759168in}{1.911227in}}%
\pgfpathlineto{\pgfqpoint{2.762340in}{1.911032in}}%
\pgfpathlineto{\pgfqpoint{2.765512in}{1.910746in}}%
\pgfpathlineto{\pgfqpoint{2.768684in}{1.910818in}}%
\pgfpathlineto{\pgfqpoint{2.771856in}{1.910827in}}%
\pgfpathlineto{\pgfqpoint{2.775028in}{1.910961in}}%
\pgfpathlineto{\pgfqpoint{2.778200in}{1.910919in}}%
\pgfpathlineto{\pgfqpoint{2.781372in}{1.910852in}}%
\pgfpathlineto{\pgfqpoint{2.784544in}{1.910827in}}%
\pgfpathlineto{\pgfqpoint{2.787716in}{1.910757in}}%
\pgfpathlineto{\pgfqpoint{2.790888in}{1.910700in}}%
\pgfpathlineto{\pgfqpoint{2.794060in}{1.910806in}}%
\pgfpathlineto{\pgfqpoint{2.797232in}{1.910861in}}%
\pgfpathlineto{\pgfqpoint{2.800404in}{1.910828in}}%
\pgfpathlineto{\pgfqpoint{2.803576in}{1.910674in}}%
\pgfpathlineto{\pgfqpoint{2.806748in}{1.910556in}}%
\pgfpathlineto{\pgfqpoint{2.809920in}{1.910607in}}%
\pgfpathlineto{\pgfqpoint{2.813092in}{1.910669in}}%
\pgfpathlineto{\pgfqpoint{2.816264in}{1.910580in}}%
\pgfpathlineto{\pgfqpoint{2.819437in}{1.910399in}}%
\pgfpathlineto{\pgfqpoint{2.822609in}{1.910394in}}%
\pgfpathlineto{\pgfqpoint{2.825781in}{1.910554in}}%
\pgfpathlineto{\pgfqpoint{2.828953in}{1.910638in}}%
\pgfpathlineto{\pgfqpoint{2.832125in}{1.910702in}}%
\pgfpathlineto{\pgfqpoint{2.835297in}{1.910860in}}%
\pgfpathlineto{\pgfqpoint{2.838469in}{1.911035in}}%
\pgfpathlineto{\pgfqpoint{2.841641in}{1.910776in}}%
\pgfpathlineto{\pgfqpoint{2.844813in}{1.910911in}}%
\pgfpathlineto{\pgfqpoint{2.847985in}{1.910939in}}%
\pgfpathlineto{\pgfqpoint{2.851157in}{1.910820in}}%
\pgfpathlineto{\pgfqpoint{2.854329in}{1.910810in}}%
\pgfpathlineto{\pgfqpoint{2.857501in}{1.910701in}}%
\pgfpathlineto{\pgfqpoint{2.860673in}{1.910526in}}%
\pgfpathlineto{\pgfqpoint{2.863845in}{1.910672in}}%
\pgfpathlineto{\pgfqpoint{2.867017in}{1.910609in}}%
\pgfpathlineto{\pgfqpoint{2.870189in}{1.910338in}}%
\pgfpathlineto{\pgfqpoint{2.873361in}{1.910292in}}%
\pgfpathlineto{\pgfqpoint{2.876533in}{1.910482in}}%
\pgfpathlineto{\pgfqpoint{2.879705in}{1.910358in}}%
\pgfpathlineto{\pgfqpoint{2.882877in}{1.910323in}}%
\pgfpathlineto{\pgfqpoint{2.886049in}{1.910305in}}%
\pgfpathlineto{\pgfqpoint{2.889221in}{1.910152in}}%
\pgfpathlineto{\pgfqpoint{2.892393in}{1.910184in}}%
\pgfpathlineto{\pgfqpoint{2.895565in}{1.910091in}}%
\pgfpathlineto{\pgfqpoint{2.898738in}{1.910081in}}%
\pgfpathlineto{\pgfqpoint{2.901910in}{1.910124in}}%
\pgfpathlineto{\pgfqpoint{2.905082in}{1.910166in}}%
\pgfpathlineto{\pgfqpoint{2.908254in}{1.910095in}}%
\pgfpathlineto{\pgfqpoint{2.911426in}{1.910063in}}%
\pgfpathlineto{\pgfqpoint{2.914598in}{1.910067in}}%
\pgfpathlineto{\pgfqpoint{2.917770in}{1.910017in}}%
\pgfpathlineto{\pgfqpoint{2.920942in}{1.909928in}}%
\pgfpathlineto{\pgfqpoint{2.924114in}{1.909723in}}%
\pgfpathlineto{\pgfqpoint{2.927286in}{1.909762in}}%
\pgfpathlineto{\pgfqpoint{2.930458in}{1.909893in}}%
\pgfpathlineto{\pgfqpoint{2.933630in}{1.909695in}}%
\pgfpathlineto{\pgfqpoint{2.936802in}{1.909710in}}%
\pgfpathlineto{\pgfqpoint{2.939974in}{1.909705in}}%
\pgfpathlineto{\pgfqpoint{2.943146in}{1.909752in}}%
\pgfpathlineto{\pgfqpoint{2.946318in}{1.909565in}}%
\pgfpathlineto{\pgfqpoint{2.949490in}{1.909740in}}%
\pgfpathlineto{\pgfqpoint{2.952662in}{1.909859in}}%
\pgfpathlineto{\pgfqpoint{2.955834in}{1.909919in}}%
\pgfpathlineto{\pgfqpoint{2.959006in}{1.910525in}}%
\pgfpathlineto{\pgfqpoint{2.962178in}{1.910547in}}%
\pgfpathlineto{\pgfqpoint{2.965350in}{1.910501in}}%
\pgfpathlineto{\pgfqpoint{2.968522in}{1.910483in}}%
\pgfpathlineto{\pgfqpoint{2.971694in}{1.910326in}}%
\pgfpathlineto{\pgfqpoint{2.974867in}{1.910092in}}%
\pgfpathlineto{\pgfqpoint{2.978039in}{1.909997in}}%
\pgfpathlineto{\pgfqpoint{2.981211in}{1.910121in}}%
\pgfpathlineto{\pgfqpoint{2.984383in}{1.910005in}}%
\pgfpathlineto{\pgfqpoint{2.987555in}{1.909915in}}%
\pgfpathlineto{\pgfqpoint{2.990727in}{1.909986in}}%
\pgfpathlineto{\pgfqpoint{2.993899in}{1.910065in}}%
\pgfpathlineto{\pgfqpoint{2.997071in}{1.909935in}}%
\pgfpathlineto{\pgfqpoint{3.000243in}{1.909874in}}%
\pgfpathlineto{\pgfqpoint{3.003415in}{1.909802in}}%
\pgfpathlineto{\pgfqpoint{3.006587in}{1.909757in}}%
\pgfpathlineto{\pgfqpoint{3.009759in}{1.909750in}}%
\pgfpathlineto{\pgfqpoint{3.012931in}{1.909814in}}%
\pgfpathlineto{\pgfqpoint{3.016103in}{1.909702in}}%
\pgfpathlineto{\pgfqpoint{3.019275in}{1.909625in}}%
\pgfpathlineto{\pgfqpoint{3.022447in}{1.909717in}}%
\pgfpathlineto{\pgfqpoint{3.025619in}{1.909627in}}%
\pgfpathlineto{\pgfqpoint{3.028791in}{1.909645in}}%
\pgfpathlineto{\pgfqpoint{3.031963in}{1.909666in}}%
\pgfpathlineto{\pgfqpoint{3.035135in}{1.909677in}}%
\pgfpathlineto{\pgfqpoint{3.038307in}{1.909597in}}%
\pgfpathlineto{\pgfqpoint{3.041479in}{1.909613in}}%
\pgfpathlineto{\pgfqpoint{3.044651in}{1.909545in}}%
\pgfpathlineto{\pgfqpoint{3.047823in}{1.909609in}}%
\pgfpathlineto{\pgfqpoint{3.050995in}{1.909636in}}%
\pgfpathlineto{\pgfqpoint{3.054168in}{1.909408in}}%
\pgfpathlineto{\pgfqpoint{3.057340in}{1.909413in}}%
\pgfpathlineto{\pgfqpoint{3.060512in}{1.909247in}}%
\pgfpathlineto{\pgfqpoint{3.063684in}{1.909064in}}%
\pgfpathlineto{\pgfqpoint{3.066856in}{1.909069in}}%
\pgfpathlineto{\pgfqpoint{3.070028in}{1.908834in}}%
\pgfpathlineto{\pgfqpoint{3.073200in}{1.908702in}}%
\pgfpathlineto{\pgfqpoint{3.076372in}{1.908647in}}%
\pgfpathlineto{\pgfqpoint{3.079544in}{1.908609in}}%
\pgfpathlineto{\pgfqpoint{3.082716in}{1.908488in}}%
\pgfpathlineto{\pgfqpoint{3.085888in}{1.908500in}}%
\pgfpathlineto{\pgfqpoint{3.089060in}{1.908457in}}%
\pgfpathlineto{\pgfqpoint{3.092232in}{1.908596in}}%
\pgfpathlineto{\pgfqpoint{3.095404in}{1.908258in}}%
\pgfpathlineto{\pgfqpoint{3.098576in}{1.908295in}}%
\pgfpathlineto{\pgfqpoint{3.101748in}{1.908381in}}%
\pgfpathlineto{\pgfqpoint{3.104920in}{1.908178in}}%
\pgfpathlineto{\pgfqpoint{3.108092in}{1.907950in}}%
\pgfpathlineto{\pgfqpoint{3.111264in}{1.907936in}}%
\pgfpathlineto{\pgfqpoint{3.114436in}{1.907840in}}%
\pgfpathlineto{\pgfqpoint{3.117608in}{1.907715in}}%
\pgfpathlineto{\pgfqpoint{3.120780in}{1.907784in}}%
\pgfpathlineto{\pgfqpoint{3.123952in}{1.907804in}}%
\pgfpathlineto{\pgfqpoint{3.127124in}{1.907833in}}%
\pgfpathlineto{\pgfqpoint{3.130297in}{1.907858in}}%
\pgfpathlineto{\pgfqpoint{3.133469in}{1.908094in}}%
\pgfpathlineto{\pgfqpoint{3.136641in}{1.907989in}}%
\pgfpathlineto{\pgfqpoint{3.139813in}{1.908161in}}%
\pgfpathlineto{\pgfqpoint{3.142985in}{1.907793in}}%
\pgfpathlineto{\pgfqpoint{3.146157in}{1.907934in}}%
\pgfpathlineto{\pgfqpoint{3.149329in}{1.907825in}}%
\pgfpathlineto{\pgfqpoint{3.152501in}{1.908086in}}%
\pgfpathlineto{\pgfqpoint{3.155673in}{1.908101in}}%
\pgfpathlineto{\pgfqpoint{3.158845in}{1.908133in}}%
\pgfpathlineto{\pgfqpoint{3.162017in}{1.907881in}}%
\pgfpathlineto{\pgfqpoint{3.165189in}{1.907747in}}%
\pgfpathlineto{\pgfqpoint{3.168361in}{1.906820in}}%
\pgfpathlineto{\pgfqpoint{3.171533in}{1.906786in}}%
\pgfpathlineto{\pgfqpoint{3.174705in}{1.906124in}}%
\pgfpathlineto{\pgfqpoint{3.177877in}{1.906331in}}%
\pgfpathlineto{\pgfqpoint{3.181049in}{1.907116in}}%
\pgfpathlineto{\pgfqpoint{3.184221in}{1.906632in}}%
\pgfpathlineto{\pgfqpoint{3.187393in}{1.906867in}}%
\pgfpathlineto{\pgfqpoint{3.190565in}{1.906262in}}%
\pgfpathlineto{\pgfqpoint{3.193737in}{1.906590in}}%
\pgfpathlineto{\pgfqpoint{3.196909in}{1.906819in}}%
\pgfpathlineto{\pgfqpoint{3.200081in}{1.906358in}}%
\pgfpathlineto{\pgfqpoint{3.203253in}{1.906062in}}%
\pgfpathlineto{\pgfqpoint{3.206425in}{1.905661in}}%
\pgfpathlineto{\pgfqpoint{3.209598in}{1.906659in}}%
\pgfpathlineto{\pgfqpoint{3.212770in}{1.906724in}}%
\pgfpathlineto{\pgfqpoint{3.215942in}{1.906762in}}%
\pgfpathlineto{\pgfqpoint{3.219114in}{1.906659in}}%
\pgfpathlineto{\pgfqpoint{3.222286in}{1.906260in}}%
\pgfpathlineto{\pgfqpoint{3.225458in}{1.905473in}}%
\pgfpathlineto{\pgfqpoint{3.228630in}{1.905343in}}%
\pgfpathlineto{\pgfqpoint{3.231802in}{1.905129in}}%
\pgfpathlineto{\pgfqpoint{3.234974in}{1.904573in}}%
\pgfpathlineto{\pgfqpoint{3.238146in}{1.904686in}}%
\pgfpathlineto{\pgfqpoint{3.241318in}{1.904325in}}%
\pgfpathlineto{\pgfqpoint{3.244490in}{1.902523in}}%
\pgfpathlineto{\pgfqpoint{3.247662in}{1.902483in}}%
\pgfpathlineto{\pgfqpoint{3.250834in}{1.902301in}}%
\pgfpathlineto{\pgfqpoint{3.254006in}{1.902418in}}%
\pgfpathlineto{\pgfqpoint{3.257178in}{1.902003in}}%
\pgfpathlineto{\pgfqpoint{3.260350in}{1.901780in}}%
\pgfpathlineto{\pgfqpoint{3.263522in}{1.901791in}}%
\pgfpathlineto{\pgfqpoint{3.266694in}{1.902076in}}%
\pgfpathlineto{\pgfqpoint{3.269866in}{1.902285in}}%
\pgfpathlineto{\pgfqpoint{3.273038in}{1.902338in}}%
\pgfpathlineto{\pgfqpoint{3.276210in}{1.901672in}}%
\pgfpathlineto{\pgfqpoint{3.279382in}{1.901266in}}%
\pgfpathlineto{\pgfqpoint{3.282554in}{1.900760in}}%
\pgfpathlineto{\pgfqpoint{3.285726in}{1.900731in}}%
\pgfpathlineto{\pgfqpoint{3.288899in}{1.900110in}}%
\pgfpathlineto{\pgfqpoint{3.292071in}{1.899864in}}%
\pgfpathlineto{\pgfqpoint{3.295243in}{1.899906in}}%
\pgfpathlineto{\pgfqpoint{3.298415in}{1.900096in}}%
\pgfpathlineto{\pgfqpoint{3.301587in}{1.898953in}}%
\pgfpathlineto{\pgfqpoint{3.304759in}{1.898225in}}%
\pgfpathlineto{\pgfqpoint{3.307931in}{1.898151in}}%
\pgfpathlineto{\pgfqpoint{3.311103in}{1.898139in}}%
\pgfpathlineto{\pgfqpoint{3.314275in}{1.898127in}}%
\pgfpathlineto{\pgfqpoint{3.317447in}{1.897919in}}%
\pgfpathlineto{\pgfqpoint{3.320619in}{1.897887in}}%
\pgfpathlineto{\pgfqpoint{3.323791in}{1.897496in}}%
\pgfpathlineto{\pgfqpoint{3.326963in}{1.897590in}}%
\pgfpathlineto{\pgfqpoint{3.330135in}{1.897255in}}%
\pgfpathlineto{\pgfqpoint{3.333307in}{1.897250in}}%
\pgfpathlineto{\pgfqpoint{3.336479in}{1.896666in}}%
\pgfpathlineto{\pgfqpoint{3.339651in}{1.896669in}}%
\pgfpathlineto{\pgfqpoint{3.342823in}{1.896856in}}%
\pgfpathlineto{\pgfqpoint{3.345995in}{1.896963in}}%
\pgfpathlineto{\pgfqpoint{3.349167in}{1.896418in}}%
\pgfpathlineto{\pgfqpoint{3.352339in}{1.896328in}}%
\pgfpathlineto{\pgfqpoint{3.355511in}{1.896131in}}%
\pgfpathlineto{\pgfqpoint{3.358683in}{1.896164in}}%
\pgfpathlineto{\pgfqpoint{3.361855in}{1.895874in}}%
\pgfpathlineto{\pgfqpoint{3.365028in}{1.896057in}}%
\pgfpathlineto{\pgfqpoint{3.368200in}{1.896211in}}%
\pgfpathlineto{\pgfqpoint{3.371372in}{1.895824in}}%
\pgfpathlineto{\pgfqpoint{3.374544in}{1.896013in}}%
\pgfpathlineto{\pgfqpoint{3.377716in}{1.895993in}}%
\pgfpathlineto{\pgfqpoint{3.380888in}{1.895843in}}%
\pgfpathlineto{\pgfqpoint{3.384060in}{1.895542in}}%
\pgfpathlineto{\pgfqpoint{3.387232in}{1.894832in}}%
\pgfpathlineto{\pgfqpoint{3.390404in}{1.894825in}}%
\pgfpathlineto{\pgfqpoint{3.393576in}{1.894861in}}%
\pgfpathlineto{\pgfqpoint{3.396748in}{1.894368in}}%
\pgfpathlineto{\pgfqpoint{3.399920in}{1.894275in}}%
\pgfpathlineto{\pgfqpoint{3.403092in}{1.894356in}}%
\pgfpathlineto{\pgfqpoint{3.406264in}{1.894319in}}%
\pgfpathlineto{\pgfqpoint{3.409436in}{1.894504in}}%
\pgfpathlineto{\pgfqpoint{3.412608in}{1.894590in}}%
\pgfpathlineto{\pgfqpoint{3.415780in}{1.894929in}}%
\pgfpathlineto{\pgfqpoint{3.418952in}{1.895275in}}%
\pgfpathlineto{\pgfqpoint{3.422124in}{1.895289in}}%
\pgfpathlineto{\pgfqpoint{3.425296in}{1.895401in}}%
\pgfpathlineto{\pgfqpoint{3.428468in}{1.895382in}}%
\pgfpathlineto{\pgfqpoint{3.431640in}{1.895248in}}%
\pgfpathlineto{\pgfqpoint{3.434812in}{1.894968in}}%
\pgfpathlineto{\pgfqpoint{3.437984in}{1.895033in}}%
\pgfpathlineto{\pgfqpoint{3.441156in}{1.894833in}}%
\pgfpathlineto{\pgfqpoint{3.444329in}{1.894907in}}%
\pgfpathlineto{\pgfqpoint{3.447501in}{1.894817in}}%
\pgfpathlineto{\pgfqpoint{3.450673in}{1.894482in}}%
\pgfpathlineto{\pgfqpoint{3.453845in}{1.894644in}}%
\pgfpathlineto{\pgfqpoint{3.457017in}{1.894362in}}%
\pgfpathlineto{\pgfqpoint{3.460189in}{1.894325in}}%
\pgfpathlineto{\pgfqpoint{3.463361in}{1.893996in}}%
\pgfpathlineto{\pgfqpoint{3.466533in}{1.893440in}}%
\pgfpathlineto{\pgfqpoint{3.469705in}{1.892956in}}%
\pgfpathlineto{\pgfqpoint{3.472877in}{1.893134in}}%
\pgfpathlineto{\pgfqpoint{3.476049in}{1.892882in}}%
\pgfpathlineto{\pgfqpoint{3.479221in}{1.892570in}}%
\pgfpathlineto{\pgfqpoint{3.482393in}{1.892194in}}%
\pgfpathlineto{\pgfqpoint{3.485565in}{1.891990in}}%
\pgfpathlineto{\pgfqpoint{3.488737in}{1.891778in}}%
\pgfpathlineto{\pgfqpoint{3.491909in}{1.891694in}}%
\pgfpathlineto{\pgfqpoint{3.495081in}{1.891592in}}%
\pgfpathlineto{\pgfqpoint{3.498253in}{1.891380in}}%
\pgfpathlineto{\pgfqpoint{3.501425in}{1.891412in}}%
\pgfpathlineto{\pgfqpoint{3.504597in}{1.891137in}}%
\pgfpathlineto{\pgfqpoint{3.507769in}{1.891010in}}%
\pgfpathlineto{\pgfqpoint{3.510941in}{1.890670in}}%
\pgfpathlineto{\pgfqpoint{3.514113in}{1.890275in}}%
\pgfpathlineto{\pgfqpoint{3.517285in}{1.890417in}}%
\pgfpathlineto{\pgfqpoint{3.520457in}{1.890333in}}%
\pgfpathlineto{\pgfqpoint{3.523630in}{1.890169in}}%
\pgfpathlineto{\pgfqpoint{3.526802in}{1.889962in}}%
\pgfpathlineto{\pgfqpoint{3.529974in}{1.889741in}}%
\pgfpathlineto{\pgfqpoint{3.533146in}{1.889894in}}%
\pgfpathlineto{\pgfqpoint{3.536318in}{1.889775in}}%
\pgfpathlineto{\pgfqpoint{3.539490in}{1.889836in}}%
\pgfpathlineto{\pgfqpoint{3.542662in}{1.889944in}}%
\pgfpathlineto{\pgfqpoint{3.545834in}{1.889616in}}%
\pgfpathlineto{\pgfqpoint{3.549006in}{1.889829in}}%
\pgfpathlineto{\pgfqpoint{3.552178in}{1.889539in}}%
\pgfpathlineto{\pgfqpoint{3.555350in}{1.889453in}}%
\pgfpathlineto{\pgfqpoint{3.558522in}{1.888939in}}%
\pgfpathlineto{\pgfqpoint{3.561694in}{1.888586in}}%
\pgfpathlineto{\pgfqpoint{3.564866in}{1.888226in}}%
\pgfpathlineto{\pgfqpoint{3.568038in}{1.888568in}}%
\pgfpathlineto{\pgfqpoint{3.571210in}{1.888629in}}%
\pgfpathlineto{\pgfqpoint{3.574382in}{1.888605in}}%
\pgfpathlineto{\pgfqpoint{3.577554in}{1.888658in}}%
\pgfpathlineto{\pgfqpoint{3.580726in}{1.888376in}}%
\pgfpathlineto{\pgfqpoint{3.583898in}{1.888684in}}%
\pgfpathlineto{\pgfqpoint{3.587070in}{1.888827in}}%
\pgfpathlineto{\pgfqpoint{3.590242in}{1.888821in}}%
\pgfpathlineto{\pgfqpoint{3.593414in}{1.888948in}}%
\pgfpathlineto{\pgfqpoint{3.596586in}{1.889107in}}%
\pgfpathlineto{\pgfqpoint{3.599759in}{1.889481in}}%
\pgfpathlineto{\pgfqpoint{3.602931in}{1.889339in}}%
\pgfpathlineto{\pgfqpoint{3.606103in}{1.889485in}}%
\pgfpathlineto{\pgfqpoint{3.609275in}{1.889239in}}%
\pgfpathlineto{\pgfqpoint{3.612447in}{1.889169in}}%
\pgfpathlineto{\pgfqpoint{3.615619in}{1.889147in}}%
\pgfpathlineto{\pgfqpoint{3.618791in}{1.888940in}}%
\pgfpathlineto{\pgfqpoint{3.621963in}{1.889005in}}%
\pgfpathlineto{\pgfqpoint{3.625135in}{1.888959in}}%
\pgfpathlineto{\pgfqpoint{3.628307in}{1.888961in}}%
\pgfpathlineto{\pgfqpoint{3.631479in}{1.888908in}}%
\pgfpathlineto{\pgfqpoint{3.634651in}{1.888729in}}%
\pgfpathlineto{\pgfqpoint{3.637823in}{1.888716in}}%
\pgfpathlineto{\pgfqpoint{3.640995in}{1.888839in}}%
\pgfpathlineto{\pgfqpoint{3.644167in}{1.888700in}}%
\pgfpathlineto{\pgfqpoint{3.647339in}{1.888702in}}%
\pgfpathlineto{\pgfqpoint{3.650511in}{1.888621in}}%
\pgfpathlineto{\pgfqpoint{3.653683in}{1.888707in}}%
\pgfpathlineto{\pgfqpoint{3.656855in}{1.888519in}}%
\pgfpathlineto{\pgfqpoint{3.660027in}{1.888311in}}%
\pgfpathlineto{\pgfqpoint{3.663199in}{1.887983in}}%
\pgfpathlineto{\pgfqpoint{3.666371in}{1.887574in}}%
\pgfpathlineto{\pgfqpoint{3.669543in}{1.887500in}}%
\pgfpathlineto{\pgfqpoint{3.672715in}{1.887805in}}%
\pgfpathlineto{\pgfqpoint{3.675887in}{1.887780in}}%
\pgfpathlineto{\pgfqpoint{3.679060in}{1.887654in}}%
\pgfpathlineto{\pgfqpoint{3.682232in}{1.887741in}}%
\pgfpathlineto{\pgfqpoint{3.685404in}{1.887632in}}%
\pgfpathlineto{\pgfqpoint{3.688576in}{1.887683in}}%
\pgfpathlineto{\pgfqpoint{3.691748in}{1.887118in}}%
\pgfpathlineto{\pgfqpoint{3.694920in}{1.887013in}}%
\pgfpathlineto{\pgfqpoint{3.698092in}{1.886794in}}%
\pgfpathlineto{\pgfqpoint{3.701264in}{1.887123in}}%
\pgfpathlineto{\pgfqpoint{3.704436in}{1.886906in}}%
\pgfpathlineto{\pgfqpoint{3.707608in}{1.886562in}}%
\pgfpathlineto{\pgfqpoint{3.710780in}{1.886291in}}%
\pgfpathlineto{\pgfqpoint{3.713952in}{1.885788in}}%
\pgfpathlineto{\pgfqpoint{3.717124in}{1.885375in}}%
\pgfpathlineto{\pgfqpoint{3.720296in}{1.885008in}}%
\pgfpathlineto{\pgfqpoint{3.723468in}{1.884669in}}%
\pgfpathlineto{\pgfqpoint{3.726640in}{1.884098in}}%
\pgfpathlineto{\pgfqpoint{3.729812in}{1.883790in}}%
\pgfpathlineto{\pgfqpoint{3.732984in}{1.883813in}}%
\pgfpathlineto{\pgfqpoint{3.736156in}{1.883770in}}%
\pgfpathlineto{\pgfqpoint{3.739328in}{1.883901in}}%
\pgfpathlineto{\pgfqpoint{3.742500in}{1.883790in}}%
\pgfpathlineto{\pgfqpoint{3.745672in}{1.883630in}}%
\pgfpathlineto{\pgfqpoint{3.748844in}{1.883143in}}%
\pgfpathlineto{\pgfqpoint{3.752016in}{1.882967in}}%
\pgfpathlineto{\pgfqpoint{3.755188in}{1.882666in}}%
\pgfpathlineto{\pgfqpoint{3.758361in}{1.882274in}}%
\pgfpathlineto{\pgfqpoint{3.761533in}{1.882161in}}%
\pgfpathlineto{\pgfqpoint{3.764705in}{1.881790in}}%
\pgfpathlineto{\pgfqpoint{3.767877in}{1.881323in}}%
\pgfpathlineto{\pgfqpoint{3.771049in}{1.880817in}}%
\pgfpathlineto{\pgfqpoint{3.774221in}{1.880816in}}%
\pgfpathlineto{\pgfqpoint{3.777393in}{1.880826in}}%
\pgfpathlineto{\pgfqpoint{3.780565in}{1.881109in}}%
\pgfpathlineto{\pgfqpoint{3.783737in}{1.880737in}}%
\pgfpathlineto{\pgfqpoint{3.786909in}{1.880827in}}%
\pgfpathlineto{\pgfqpoint{3.790081in}{1.880590in}}%
\pgfpathlineto{\pgfqpoint{3.793253in}{1.880383in}}%
\pgfpathlineto{\pgfqpoint{3.796425in}{1.879920in}}%
\pgfpathlineto{\pgfqpoint{3.799597in}{1.879719in}}%
\pgfpathlineto{\pgfqpoint{3.802769in}{1.879860in}}%
\pgfpathlineto{\pgfqpoint{3.805941in}{1.879894in}}%
\pgfpathlineto{\pgfqpoint{3.809113in}{1.879803in}}%
\pgfpathlineto{\pgfqpoint{3.812285in}{1.879546in}}%
\pgfpathlineto{\pgfqpoint{3.815457in}{1.879330in}}%
\pgfpathlineto{\pgfqpoint{3.818629in}{1.878918in}}%
\pgfpathlineto{\pgfqpoint{3.821801in}{1.878894in}}%
\pgfpathlineto{\pgfqpoint{3.824973in}{1.878546in}}%
\pgfpathlineto{\pgfqpoint{3.828145in}{1.878227in}}%
\pgfpathlineto{\pgfqpoint{3.831317in}{1.878450in}}%
\pgfpathlineto{\pgfqpoint{3.834490in}{1.878235in}}%
\pgfpathlineto{\pgfqpoint{3.837662in}{1.877820in}}%
\pgfpathlineto{\pgfqpoint{3.840834in}{1.877651in}}%
\pgfpathlineto{\pgfqpoint{3.844006in}{1.877474in}}%
\pgfpathlineto{\pgfqpoint{3.847178in}{1.877660in}}%
\pgfpathlineto{\pgfqpoint{3.850350in}{1.877394in}}%
\pgfpathlineto{\pgfqpoint{3.853522in}{1.877555in}}%
\pgfpathlineto{\pgfqpoint{3.856694in}{1.877562in}}%
\pgfpathlineto{\pgfqpoint{3.859866in}{1.876975in}}%
\pgfpathlineto{\pgfqpoint{3.863038in}{1.877190in}}%
\pgfpathlineto{\pgfqpoint{3.866210in}{1.877412in}}%
\pgfpathlineto{\pgfqpoint{3.869382in}{1.877271in}}%
\pgfpathlineto{\pgfqpoint{3.872554in}{1.877066in}}%
\pgfpathlineto{\pgfqpoint{3.875726in}{1.876816in}}%
\pgfpathlineto{\pgfqpoint{3.878898in}{1.876317in}}%
\pgfpathlineto{\pgfqpoint{3.882070in}{1.875851in}}%
\pgfpathlineto{\pgfqpoint{3.885242in}{1.875739in}}%
\pgfpathlineto{\pgfqpoint{3.888414in}{1.875768in}}%
\pgfpathlineto{\pgfqpoint{3.891586in}{1.875347in}}%
\pgfpathlineto{\pgfqpoint{3.894758in}{1.875556in}}%
\pgfpathlineto{\pgfqpoint{3.897930in}{1.875564in}}%
\pgfpathlineto{\pgfqpoint{3.901102in}{1.875448in}}%
\pgfpathlineto{\pgfqpoint{3.904274in}{1.875353in}}%
\pgfpathlineto{\pgfqpoint{3.907446in}{1.875623in}}%
\pgfpathlineto{\pgfqpoint{3.910618in}{1.875610in}}%
\pgfpathlineto{\pgfqpoint{3.913791in}{1.875222in}}%
\pgfpathlineto{\pgfqpoint{3.916963in}{1.875471in}}%
\pgfpathlineto{\pgfqpoint{3.920135in}{1.875190in}}%
\pgfpathlineto{\pgfqpoint{3.923307in}{1.875399in}}%
\pgfpathlineto{\pgfqpoint{3.926479in}{1.875312in}}%
\pgfpathlineto{\pgfqpoint{3.929651in}{1.875574in}}%
\pgfpathlineto{\pgfqpoint{3.932823in}{1.875282in}}%
\pgfpathlineto{\pgfqpoint{3.935995in}{1.875477in}}%
\pgfpathlineto{\pgfqpoint{3.939167in}{1.875537in}}%
\pgfpathlineto{\pgfqpoint{3.942339in}{1.875452in}}%
\pgfpathlineto{\pgfqpoint{3.945511in}{1.875285in}}%
\pgfpathlineto{\pgfqpoint{3.948683in}{1.875463in}}%
\pgfpathlineto{\pgfqpoint{3.951855in}{1.875331in}}%
\pgfpathlineto{\pgfqpoint{3.955027in}{1.875287in}}%
\pgfpathlineto{\pgfqpoint{3.958199in}{1.874992in}}%
\pgfpathlineto{\pgfqpoint{3.961371in}{1.875093in}}%
\pgfpathlineto{\pgfqpoint{3.964543in}{1.874667in}}%
\pgfpathlineto{\pgfqpoint{3.967715in}{1.874374in}}%
\pgfpathlineto{\pgfqpoint{3.970887in}{1.874177in}}%
\pgfpathlineto{\pgfqpoint{3.974059in}{1.873659in}}%
\pgfpathlineto{\pgfqpoint{3.977231in}{1.873345in}}%
\pgfpathlineto{\pgfqpoint{3.980403in}{1.872804in}}%
\pgfpathlineto{\pgfqpoint{3.983575in}{1.872740in}}%
\pgfpathlineto{\pgfqpoint{3.986747in}{1.872762in}}%
\pgfpathlineto{\pgfqpoint{3.989919in}{1.872685in}}%
\pgfpathlineto{\pgfqpoint{3.993092in}{1.872685in}}%
\pgfpathlineto{\pgfqpoint{3.996264in}{1.872636in}}%
\pgfpathlineto{\pgfqpoint{3.999436in}{1.872865in}}%
\pgfpathlineto{\pgfqpoint{4.002608in}{1.873342in}}%
\pgfpathlineto{\pgfqpoint{4.005780in}{1.873165in}}%
\pgfpathlineto{\pgfqpoint{4.008952in}{1.873154in}}%
\pgfpathlineto{\pgfqpoint{4.012124in}{1.873119in}}%
\pgfpathlineto{\pgfqpoint{4.015296in}{1.873035in}}%
\pgfpathlineto{\pgfqpoint{4.018468in}{1.872964in}}%
\pgfpathlineto{\pgfqpoint{4.021640in}{1.873114in}}%
\pgfpathlineto{\pgfqpoint{4.024812in}{1.872992in}}%
\pgfpathlineto{\pgfqpoint{4.027984in}{1.873034in}}%
\pgfpathlineto{\pgfqpoint{4.031156in}{1.873407in}}%
\pgfpathlineto{\pgfqpoint{4.034328in}{1.873155in}}%
\pgfpathlineto{\pgfqpoint{4.037500in}{1.872671in}}%
\pgfpathlineto{\pgfqpoint{4.040672in}{1.872693in}}%
\pgfpathlineto{\pgfqpoint{4.043844in}{1.872549in}}%
\pgfpathlineto{\pgfqpoint{4.047016in}{1.872404in}}%
\pgfpathlineto{\pgfqpoint{4.050188in}{1.872333in}}%
\pgfpathlineto{\pgfqpoint{4.053360in}{1.872335in}}%
\pgfpathlineto{\pgfqpoint{4.056532in}{1.872218in}}%
\pgfpathlineto{\pgfqpoint{4.059704in}{1.872010in}}%
\pgfpathlineto{\pgfqpoint{4.062876in}{1.871956in}}%
\pgfpathlineto{\pgfqpoint{4.066048in}{1.871658in}}%
\pgfpathlineto{\pgfqpoint{4.069221in}{1.871739in}}%
\pgfpathlineto{\pgfqpoint{4.072393in}{1.871607in}}%
\pgfpathlineto{\pgfqpoint{4.075565in}{1.871638in}}%
\pgfpathlineto{\pgfqpoint{4.078737in}{1.871594in}}%
\pgfpathlineto{\pgfqpoint{4.081909in}{1.871840in}}%
\pgfpathlineto{\pgfqpoint{4.085081in}{1.871659in}}%
\pgfpathlineto{\pgfqpoint{4.088253in}{1.871567in}}%
\pgfpathlineto{\pgfqpoint{4.091425in}{1.870828in}}%
\pgfpathlineto{\pgfqpoint{4.094597in}{1.870261in}}%
\pgfpathlineto{\pgfqpoint{4.097769in}{1.869827in}}%
\pgfpathlineto{\pgfqpoint{4.100941in}{1.870090in}}%
\pgfpathlineto{\pgfqpoint{4.104113in}{1.869860in}}%
\pgfpathlineto{\pgfqpoint{4.107285in}{1.870031in}}%
\pgfpathlineto{\pgfqpoint{4.110457in}{1.869860in}}%
\pgfpathlineto{\pgfqpoint{4.113629in}{1.869908in}}%
\pgfpathlineto{\pgfqpoint{4.116801in}{1.869761in}}%
\pgfpathlineto{\pgfqpoint{4.119973in}{1.869951in}}%
\pgfpathlineto{\pgfqpoint{4.123145in}{1.870577in}}%
\pgfpathlineto{\pgfqpoint{4.126317in}{1.870817in}}%
\pgfpathlineto{\pgfqpoint{4.129489in}{1.870423in}}%
\pgfpathlineto{\pgfqpoint{4.132661in}{1.870268in}}%
\pgfpathlineto{\pgfqpoint{4.135833in}{1.869635in}}%
\pgfpathlineto{\pgfqpoint{4.139005in}{1.869574in}}%
\pgfpathlineto{\pgfqpoint{4.142177in}{1.869399in}}%
\pgfpathlineto{\pgfqpoint{4.145349in}{1.869429in}}%
\pgfpathlineto{\pgfqpoint{4.148522in}{1.869082in}}%
\pgfpathlineto{\pgfqpoint{4.151694in}{1.868700in}}%
\pgfpathlineto{\pgfqpoint{4.154866in}{1.868829in}}%
\pgfpathlineto{\pgfqpoint{4.158038in}{1.868537in}}%
\pgfpathlineto{\pgfqpoint{4.161210in}{1.868471in}}%
\pgfpathlineto{\pgfqpoint{4.164382in}{1.868870in}}%
\pgfpathlineto{\pgfqpoint{4.167554in}{1.868897in}}%
\pgfpathlineto{\pgfqpoint{4.170726in}{1.868678in}}%
\pgfpathlineto{\pgfqpoint{4.173898in}{1.868269in}}%
\pgfpathlineto{\pgfqpoint{4.177070in}{1.868301in}}%
\pgfpathlineto{\pgfqpoint{4.180242in}{1.868243in}}%
\pgfpathlineto{\pgfqpoint{4.183414in}{1.868257in}}%
\pgfpathlineto{\pgfqpoint{4.186586in}{1.867888in}}%
\pgfpathlineto{\pgfqpoint{4.189758in}{1.867286in}}%
\pgfpathlineto{\pgfqpoint{4.192930in}{1.867072in}}%
\pgfpathlineto{\pgfqpoint{4.196102in}{1.867195in}}%
\pgfpathlineto{\pgfqpoint{4.199274in}{1.867222in}}%
\pgfpathlineto{\pgfqpoint{4.202446in}{1.867145in}}%
\pgfpathlineto{\pgfqpoint{4.205618in}{1.867236in}}%
\pgfpathlineto{\pgfqpoint{4.208790in}{1.866572in}}%
\pgfpathlineto{\pgfqpoint{4.211962in}{1.865793in}}%
\pgfpathlineto{\pgfqpoint{4.215134in}{1.865483in}}%
\pgfpathlineto{\pgfqpoint{4.218306in}{1.865651in}}%
\pgfpathlineto{\pgfqpoint{4.221478in}{1.865237in}}%
\pgfpathlineto{\pgfqpoint{4.224650in}{1.864880in}}%
\pgfpathlineto{\pgfqpoint{4.227823in}{1.864560in}}%
\pgfpathlineto{\pgfqpoint{4.230995in}{1.864739in}}%
\pgfpathlineto{\pgfqpoint{4.234167in}{1.864826in}}%
\pgfpathlineto{\pgfqpoint{4.237339in}{1.864593in}}%
\pgfpathlineto{\pgfqpoint{4.240511in}{1.864242in}}%
\pgfpathlineto{\pgfqpoint{4.243683in}{1.864211in}}%
\pgfpathlineto{\pgfqpoint{4.246855in}{1.863745in}}%
\pgfpathlineto{\pgfqpoint{4.250027in}{1.863349in}}%
\pgfpathlineto{\pgfqpoint{4.253199in}{1.862881in}}%
\pgfpathlineto{\pgfqpoint{4.256371in}{1.862904in}}%
\pgfpathlineto{\pgfqpoint{4.259543in}{1.863200in}}%
\pgfpathlineto{\pgfqpoint{4.262715in}{1.863053in}}%
\pgfpathlineto{\pgfqpoint{4.265887in}{1.863163in}}%
\pgfpathlineto{\pgfqpoint{4.269059in}{1.863271in}}%
\pgfpathlineto{\pgfqpoint{4.272231in}{1.863075in}}%
\pgfpathlineto{\pgfqpoint{4.275403in}{1.862980in}}%
\pgfpathlineto{\pgfqpoint{4.278575in}{1.862939in}}%
\pgfpathlineto{\pgfqpoint{4.281747in}{1.862864in}}%
\pgfpathlineto{\pgfqpoint{4.284919in}{1.863133in}}%
\pgfpathlineto{\pgfqpoint{4.288091in}{1.863129in}}%
\pgfpathlineto{\pgfqpoint{4.291263in}{1.863094in}}%
\pgfpathlineto{\pgfqpoint{4.294435in}{1.863054in}}%
\pgfpathlineto{\pgfqpoint{4.297607in}{1.863343in}}%
\pgfpathlineto{\pgfqpoint{4.300779in}{1.863372in}}%
\pgfpathlineto{\pgfqpoint{4.303952in}{1.863420in}}%
\pgfpathlineto{\pgfqpoint{4.307124in}{1.863845in}}%
\pgfpathlineto{\pgfqpoint{4.310296in}{1.864085in}}%
\pgfpathlineto{\pgfqpoint{4.313468in}{1.863936in}}%
\pgfpathlineto{\pgfqpoint{4.316640in}{1.863847in}}%
\pgfpathlineto{\pgfqpoint{4.319812in}{1.863910in}}%
\pgfpathlineto{\pgfqpoint{4.322984in}{1.863640in}}%
\pgfpathlineto{\pgfqpoint{4.326156in}{1.863589in}}%
\pgfpathlineto{\pgfqpoint{4.329328in}{1.863631in}}%
\pgfpathlineto{\pgfqpoint{4.332500in}{1.863459in}}%
\pgfpathlineto{\pgfqpoint{4.335672in}{1.863349in}}%
\pgfpathlineto{\pgfqpoint{4.338844in}{1.863400in}}%
\pgfpathlineto{\pgfqpoint{4.342016in}{1.863510in}}%
\pgfpathlineto{\pgfqpoint{4.345188in}{1.863614in}}%
\pgfpathlineto{\pgfqpoint{4.348360in}{1.863285in}}%
\pgfpathlineto{\pgfqpoint{4.351532in}{1.863128in}}%
\pgfpathlineto{\pgfqpoint{4.354704in}{1.863174in}}%
\pgfpathlineto{\pgfqpoint{4.357876in}{1.862426in}}%
\pgfpathlineto{\pgfqpoint{4.361048in}{1.862151in}}%
\pgfpathlineto{\pgfqpoint{4.364220in}{1.862114in}}%
\pgfpathlineto{\pgfqpoint{4.367392in}{1.862003in}}%
\pgfpathlineto{\pgfqpoint{4.370564in}{1.861720in}}%
\pgfpathlineto{\pgfqpoint{4.373736in}{1.861695in}}%
\pgfpathlineto{\pgfqpoint{4.376908in}{1.861580in}}%
\pgfpathlineto{\pgfqpoint{4.380080in}{1.861606in}}%
\pgfpathlineto{\pgfqpoint{4.383253in}{1.861443in}}%
\pgfpathlineto{\pgfqpoint{4.386425in}{1.861291in}}%
\pgfpathlineto{\pgfqpoint{4.389597in}{1.861069in}}%
\pgfpathlineto{\pgfqpoint{4.392769in}{1.860568in}}%
\pgfpathlineto{\pgfqpoint{4.395941in}{1.860324in}}%
\pgfpathlineto{\pgfqpoint{4.399113in}{1.860032in}}%
\pgfpathlineto{\pgfqpoint{4.402285in}{1.860100in}}%
\pgfpathlineto{\pgfqpoint{4.405457in}{1.859419in}}%
\pgfpathlineto{\pgfqpoint{4.408629in}{1.859401in}}%
\pgfpathlineto{\pgfqpoint{4.411801in}{1.859241in}}%
\pgfpathlineto{\pgfqpoint{4.414973in}{1.858845in}}%
\pgfpathlineto{\pgfqpoint{4.418145in}{1.858659in}}%
\pgfpathlineto{\pgfqpoint{4.421317in}{1.858795in}}%
\pgfpathlineto{\pgfqpoint{4.424489in}{1.858711in}}%
\pgfpathlineto{\pgfqpoint{4.427661in}{1.858910in}}%
\pgfpathlineto{\pgfqpoint{4.430833in}{1.859155in}}%
\pgfpathlineto{\pgfqpoint{4.434005in}{1.859307in}}%
\pgfpathlineto{\pgfqpoint{4.437177in}{1.859060in}}%
\pgfpathlineto{\pgfqpoint{4.440349in}{1.859023in}}%
\pgfpathlineto{\pgfqpoint{4.443521in}{1.858926in}}%
\pgfpathlineto{\pgfqpoint{4.446693in}{1.858385in}}%
\pgfpathlineto{\pgfqpoint{4.449865in}{1.858094in}}%
\pgfpathlineto{\pgfqpoint{4.453037in}{1.857781in}}%
\pgfpathlineto{\pgfqpoint{4.456209in}{1.857670in}}%
\pgfpathlineto{\pgfqpoint{4.459381in}{1.857555in}}%
\pgfpathlineto{\pgfqpoint{4.462554in}{1.857045in}}%
\pgfpathlineto{\pgfqpoint{4.465726in}{1.856667in}}%
\pgfpathlineto{\pgfqpoint{4.468898in}{1.856539in}}%
\pgfpathlineto{\pgfqpoint{4.472070in}{1.856649in}}%
\pgfpathlineto{\pgfqpoint{4.475242in}{1.856217in}}%
\pgfpathlineto{\pgfqpoint{4.478414in}{1.856183in}}%
\pgfpathlineto{\pgfqpoint{4.481586in}{1.856107in}}%
\pgfpathlineto{\pgfqpoint{4.484758in}{1.856295in}}%
\pgfpathlineto{\pgfqpoint{4.487930in}{1.856584in}}%
\pgfpathlineto{\pgfqpoint{4.491102in}{1.856653in}}%
\pgfpathlineto{\pgfqpoint{4.494274in}{1.856714in}}%
\pgfpathlineto{\pgfqpoint{4.497446in}{1.856677in}}%
\pgfpathlineto{\pgfqpoint{4.500618in}{1.856913in}}%
\pgfpathlineto{\pgfqpoint{4.503790in}{1.857207in}}%
\pgfpathlineto{\pgfqpoint{4.506962in}{1.856955in}}%
\pgfpathlineto{\pgfqpoint{4.510134in}{1.856770in}}%
\pgfpathlineto{\pgfqpoint{4.513306in}{1.856632in}}%
\pgfpathlineto{\pgfqpoint{4.516478in}{1.856261in}}%
\pgfpathlineto{\pgfqpoint{4.519650in}{1.856573in}}%
\pgfpathlineto{\pgfqpoint{4.522822in}{1.856370in}}%
\pgfpathlineto{\pgfqpoint{4.525994in}{1.856221in}}%
\pgfpathlineto{\pgfqpoint{4.529166in}{1.856084in}}%
\pgfpathlineto{\pgfqpoint{4.532338in}{1.855919in}}%
\pgfpathlineto{\pgfqpoint{4.535510in}{1.855703in}}%
\pgfpathlineto{\pgfqpoint{4.538683in}{1.855707in}}%
\pgfpathlineto{\pgfqpoint{4.541855in}{1.855732in}}%
\pgfpathlineto{\pgfqpoint{4.545027in}{1.855366in}}%
\pgfpathlineto{\pgfqpoint{4.548199in}{1.855221in}}%
\pgfpathlineto{\pgfqpoint{4.551371in}{1.854833in}}%
\pgfpathlineto{\pgfqpoint{4.554543in}{1.854494in}}%
\pgfpathlineto{\pgfqpoint{4.557715in}{1.854011in}}%
\pgfpathlineto{\pgfqpoint{4.560887in}{1.854012in}}%
\pgfpathlineto{\pgfqpoint{4.564059in}{1.853897in}}%
\pgfpathlineto{\pgfqpoint{4.567231in}{1.854004in}}%
\pgfpathlineto{\pgfqpoint{4.570403in}{1.853967in}}%
\pgfpathlineto{\pgfqpoint{4.573575in}{1.853779in}}%
\pgfpathlineto{\pgfqpoint{4.576747in}{1.853268in}}%
\pgfpathlineto{\pgfqpoint{4.579919in}{1.852866in}}%
\pgfpathlineto{\pgfqpoint{4.583091in}{1.852931in}}%
\pgfpathlineto{\pgfqpoint{4.586263in}{1.852936in}}%
\pgfpathlineto{\pgfqpoint{4.589435in}{1.853054in}}%
\pgfpathlineto{\pgfqpoint{4.592607in}{1.853053in}}%
\pgfpathlineto{\pgfqpoint{4.595779in}{1.853268in}}%
\pgfpathlineto{\pgfqpoint{4.598951in}{1.853502in}}%
\pgfpathlineto{\pgfqpoint{4.602123in}{1.853471in}}%
\pgfpathlineto{\pgfqpoint{4.605295in}{1.852799in}}%
\pgfpathlineto{\pgfqpoint{4.608467in}{1.852912in}}%
\pgfpathlineto{\pgfqpoint{4.611639in}{1.852759in}}%
\pgfpathlineto{\pgfqpoint{4.614811in}{1.852336in}}%
\pgfpathlineto{\pgfqpoint{4.617984in}{1.851934in}}%
\pgfpathlineto{\pgfqpoint{4.621156in}{1.851745in}}%
\pgfpathlineto{\pgfqpoint{4.624328in}{1.851594in}}%
\pgfpathlineto{\pgfqpoint{4.627500in}{1.851677in}}%
\pgfpathlineto{\pgfqpoint{4.630672in}{1.851585in}}%
\pgfpathlineto{\pgfqpoint{4.633844in}{1.851284in}}%
\pgfpathlineto{\pgfqpoint{4.637016in}{1.851116in}}%
\pgfpathlineto{\pgfqpoint{4.640188in}{1.850681in}}%
\pgfpathlineto{\pgfqpoint{4.643360in}{1.850223in}}%
\pgfpathlineto{\pgfqpoint{4.646532in}{1.850214in}}%
\pgfpathlineto{\pgfqpoint{4.649704in}{1.850705in}}%
\pgfpathlineto{\pgfqpoint{4.652876in}{1.850697in}}%
\pgfpathlineto{\pgfqpoint{4.656048in}{1.850180in}}%
\pgfpathlineto{\pgfqpoint{4.659220in}{1.849883in}}%
\pgfpathlineto{\pgfqpoint{4.662392in}{1.849729in}}%
\pgfpathlineto{\pgfqpoint{4.665564in}{1.849883in}}%
\pgfpathlineto{\pgfqpoint{4.668736in}{1.849745in}}%
\pgfpathlineto{\pgfqpoint{4.671908in}{1.849510in}}%
\pgfpathlineto{\pgfqpoint{4.675080in}{1.849578in}}%
\pgfpathlineto{\pgfqpoint{4.678252in}{1.849684in}}%
\pgfpathlineto{\pgfqpoint{4.681424in}{1.849319in}}%
\pgfpathlineto{\pgfqpoint{4.684596in}{1.849664in}}%
\pgfpathlineto{\pgfqpoint{4.687768in}{1.849648in}}%
\pgfpathlineto{\pgfqpoint{4.690940in}{1.849883in}}%
\pgfpathlineto{\pgfqpoint{4.694112in}{1.849991in}}%
\pgfpathlineto{\pgfqpoint{4.697285in}{1.850177in}}%
\pgfpathlineto{\pgfqpoint{4.700457in}{1.849780in}}%
\pgfpathlineto{\pgfqpoint{4.703629in}{1.849985in}}%
\pgfpathlineto{\pgfqpoint{4.706801in}{1.849800in}}%
\pgfpathlineto{\pgfqpoint{4.709973in}{1.849787in}}%
\pgfpathlineto{\pgfqpoint{4.713145in}{1.849553in}}%
\pgfpathlineto{\pgfqpoint{4.716317in}{1.849568in}}%
\pgfpathlineto{\pgfqpoint{4.719489in}{1.848997in}}%
\pgfpathlineto{\pgfqpoint{4.722661in}{1.848756in}}%
\pgfpathlineto{\pgfqpoint{4.725833in}{1.848565in}}%
\pgfpathlineto{\pgfqpoint{4.729005in}{1.848073in}}%
\pgfpathlineto{\pgfqpoint{4.732177in}{1.847833in}}%
\pgfpathlineto{\pgfqpoint{4.735349in}{1.848011in}}%
\pgfpathlineto{\pgfqpoint{4.738521in}{1.848082in}}%
\pgfpathlineto{\pgfqpoint{4.741693in}{1.848505in}}%
\pgfpathlineto{\pgfqpoint{4.744865in}{1.848491in}}%
\pgfpathlineto{\pgfqpoint{4.748037in}{1.848153in}}%
\pgfpathlineto{\pgfqpoint{4.751209in}{1.848109in}}%
\pgfpathlineto{\pgfqpoint{4.754381in}{1.848141in}}%
\pgfpathlineto{\pgfqpoint{4.757553in}{1.848260in}}%
\pgfpathlineto{\pgfqpoint{4.760725in}{1.848249in}}%
\pgfpathlineto{\pgfqpoint{4.763897in}{1.848014in}}%
\pgfpathlineto{\pgfqpoint{4.767069in}{1.847989in}}%
\pgfpathlineto{\pgfqpoint{4.770241in}{1.848143in}}%
\pgfpathlineto{\pgfqpoint{4.773414in}{1.848118in}}%
\pgfpathlineto{\pgfqpoint{4.776586in}{1.848025in}}%
\pgfpathlineto{\pgfqpoint{4.779758in}{1.847681in}}%
\pgfpathlineto{\pgfqpoint{4.782930in}{1.846868in}}%
\pgfpathlineto{\pgfqpoint{4.786102in}{1.846739in}}%
\pgfpathlineto{\pgfqpoint{4.789274in}{1.846846in}}%
\pgfpathlineto{\pgfqpoint{4.792446in}{1.846983in}}%
\pgfpathlineto{\pgfqpoint{4.795618in}{1.847047in}}%
\pgfpathlineto{\pgfqpoint{4.798790in}{1.846925in}}%
\pgfpathlineto{\pgfqpoint{4.801962in}{1.846846in}}%
\pgfpathlineto{\pgfqpoint{4.805134in}{1.846848in}}%
\pgfpathlineto{\pgfqpoint{4.808306in}{1.846591in}}%
\pgfpathlineto{\pgfqpoint{4.811478in}{1.846696in}}%
\pgfpathlineto{\pgfqpoint{4.814650in}{1.846837in}}%
\pgfpathlineto{\pgfqpoint{4.817822in}{1.846868in}}%
\pgfpathlineto{\pgfqpoint{4.820994in}{1.846765in}}%
\pgfpathlineto{\pgfqpoint{4.824166in}{1.846738in}}%
\pgfpathlineto{\pgfqpoint{4.827338in}{1.846643in}}%
\pgfpathlineto{\pgfqpoint{4.830510in}{1.846560in}}%
\pgfpathlineto{\pgfqpoint{4.833682in}{1.846563in}}%
\pgfpathlineto{\pgfqpoint{4.836854in}{1.846307in}}%
\pgfpathlineto{\pgfqpoint{4.840026in}{1.846284in}}%
\pgfpathlineto{\pgfqpoint{4.843198in}{1.846442in}}%
\pgfpathlineto{\pgfqpoint{4.846370in}{1.846797in}}%
\pgfpathlineto{\pgfqpoint{4.849542in}{1.846857in}}%
\pgfpathlineto{\pgfqpoint{4.852715in}{1.846959in}}%
\pgfpathlineto{\pgfqpoint{4.855887in}{1.847087in}}%
\pgfpathlineto{\pgfqpoint{4.859059in}{1.847147in}}%
\pgfpathlineto{\pgfqpoint{4.862231in}{1.846829in}}%
\pgfpathlineto{\pgfqpoint{4.865403in}{1.846980in}}%
\pgfpathlineto{\pgfqpoint{4.868575in}{1.847021in}}%
\pgfpathlineto{\pgfqpoint{4.871747in}{1.846952in}}%
\pgfpathlineto{\pgfqpoint{4.874919in}{1.846805in}}%
\pgfpathlineto{\pgfqpoint{4.878091in}{1.846629in}}%
\pgfpathlineto{\pgfqpoint{4.881263in}{1.846687in}}%
\pgfpathlineto{\pgfqpoint{4.884435in}{1.846567in}}%
\pgfpathlineto{\pgfqpoint{4.887607in}{1.846449in}}%
\pgfpathlineto{\pgfqpoint{4.890779in}{1.846141in}}%
\pgfpathlineto{\pgfqpoint{4.893951in}{1.846833in}}%
\pgfpathlineto{\pgfqpoint{4.897123in}{1.846914in}}%
\pgfpathlineto{\pgfqpoint{4.900295in}{1.846852in}}%
\pgfpathlineto{\pgfqpoint{4.903467in}{1.846889in}}%
\pgfpathlineto{\pgfqpoint{4.906639in}{1.846697in}}%
\pgfpathlineto{\pgfqpoint{4.909811in}{1.846472in}}%
\pgfpathlineto{\pgfqpoint{4.912983in}{1.846236in}}%
\pgfpathlineto{\pgfqpoint{4.916155in}{1.846092in}}%
\pgfpathlineto{\pgfqpoint{4.919327in}{1.846004in}}%
\pgfpathlineto{\pgfqpoint{4.922499in}{1.845732in}}%
\pgfpathlineto{\pgfqpoint{4.925671in}{1.845792in}}%
\pgfpathlineto{\pgfqpoint{4.928844in}{1.845690in}}%
\pgfpathlineto{\pgfqpoint{4.932016in}{1.845451in}}%
\pgfpathlineto{\pgfqpoint{4.935188in}{1.845591in}}%
\pgfpathlineto{\pgfqpoint{4.938360in}{1.845529in}}%
\pgfpathlineto{\pgfqpoint{4.941532in}{1.845361in}}%
\pgfpathlineto{\pgfqpoint{4.944704in}{1.845243in}}%
\pgfpathlineto{\pgfqpoint{4.947876in}{1.845676in}}%
\pgfpathlineto{\pgfqpoint{4.951048in}{1.845570in}}%
\pgfpathlineto{\pgfqpoint{4.954220in}{1.845925in}}%
\pgfpathlineto{\pgfqpoint{4.957392in}{1.846193in}}%
\pgfpathlineto{\pgfqpoint{4.960564in}{1.846475in}}%
\pgfpathlineto{\pgfqpoint{4.963736in}{1.846476in}}%
\pgfpathlineto{\pgfqpoint{4.966908in}{1.846292in}}%
\pgfpathlineto{\pgfqpoint{4.970080in}{1.846209in}}%
\pgfpathlineto{\pgfqpoint{4.973252in}{1.846359in}}%
\pgfpathlineto{\pgfqpoint{4.976424in}{1.846204in}}%
\pgfpathlineto{\pgfqpoint{4.979596in}{1.846312in}}%
\pgfpathlineto{\pgfqpoint{4.982768in}{1.846192in}}%
\pgfpathlineto{\pgfqpoint{4.985940in}{1.846282in}}%
\pgfpathlineto{\pgfqpoint{4.989112in}{1.845727in}}%
\pgfpathlineto{\pgfqpoint{4.992284in}{1.845774in}}%
\pgfpathlineto{\pgfqpoint{4.995456in}{1.845469in}}%
\pgfpathlineto{\pgfqpoint{4.998628in}{1.845392in}}%
\pgfpathlineto{\pgfqpoint{5.001800in}{1.845377in}}%
\pgfpathlineto{\pgfqpoint{5.004972in}{1.845327in}}%
\pgfpathlineto{\pgfqpoint{5.008145in}{1.845291in}}%
\pgfpathlineto{\pgfqpoint{5.011317in}{1.844851in}}%
\pgfpathlineto{\pgfqpoint{5.014489in}{1.844541in}}%
\pgfpathlineto{\pgfqpoint{5.017661in}{1.844543in}}%
\pgfpathlineto{\pgfqpoint{5.020833in}{1.844813in}}%
\pgfpathlineto{\pgfqpoint{5.024005in}{1.844814in}}%
\pgfpathlineto{\pgfqpoint{5.027177in}{1.844629in}}%
\pgfpathlineto{\pgfqpoint{5.030349in}{1.844816in}}%
\pgfpathlineto{\pgfqpoint{5.033521in}{1.844633in}}%
\pgfpathlineto{\pgfqpoint{5.036693in}{1.844320in}}%
\pgfpathlineto{\pgfqpoint{5.039865in}{1.844053in}}%
\pgfpathlineto{\pgfqpoint{5.043037in}{1.843997in}}%
\pgfpathlineto{\pgfqpoint{5.046209in}{1.844105in}}%
\pgfpathlineto{\pgfqpoint{5.049381in}{1.843333in}}%
\pgfpathlineto{\pgfqpoint{5.052553in}{1.843204in}}%
\pgfpathlineto{\pgfqpoint{5.055725in}{1.842752in}}%
\pgfpathlineto{\pgfqpoint{5.058897in}{1.843320in}}%
\pgfpathlineto{\pgfqpoint{5.062069in}{1.843364in}}%
\pgfpathlineto{\pgfqpoint{5.065241in}{1.843071in}}%
\pgfpathlineto{\pgfqpoint{5.068413in}{1.843026in}}%
\pgfpathlineto{\pgfqpoint{5.071585in}{1.842877in}}%
\pgfpathlineto{\pgfqpoint{5.074757in}{1.842621in}}%
\pgfpathlineto{\pgfqpoint{5.077929in}{1.842535in}}%
\pgfpathlineto{\pgfqpoint{5.081101in}{1.842979in}}%
\pgfpathlineto{\pgfqpoint{5.084273in}{1.843205in}}%
\pgfpathlineto{\pgfqpoint{5.087446in}{1.843164in}}%
\pgfpathlineto{\pgfqpoint{5.090618in}{1.843203in}}%
\pgfpathlineto{\pgfqpoint{5.093790in}{1.843161in}}%
\pgfpathlineto{\pgfqpoint{5.096962in}{1.843121in}}%
\pgfpathlineto{\pgfqpoint{5.100134in}{1.843305in}}%
\pgfpathlineto{\pgfqpoint{5.103306in}{1.843377in}}%
\pgfpathlineto{\pgfqpoint{5.106478in}{1.843852in}}%
\pgfpathlineto{\pgfqpoint{5.109650in}{1.843680in}}%
\pgfpathlineto{\pgfqpoint{5.112822in}{1.843020in}}%
\pgfpathlineto{\pgfqpoint{5.115994in}{1.842955in}}%
\pgfpathlineto{\pgfqpoint{5.119166in}{1.842844in}}%
\pgfpathlineto{\pgfqpoint{5.122338in}{1.842763in}}%
\pgfpathlineto{\pgfqpoint{5.125510in}{1.842362in}}%
\pgfpathlineto{\pgfqpoint{5.128682in}{1.842313in}}%
\pgfpathlineto{\pgfqpoint{5.131854in}{1.842869in}}%
\pgfpathlineto{\pgfqpoint{5.135026in}{1.843155in}}%
\pgfpathlineto{\pgfqpoint{5.138198in}{1.842875in}}%
\pgfpathlineto{\pgfqpoint{5.141370in}{1.843324in}}%
\pgfpathlineto{\pgfqpoint{5.144542in}{1.843083in}}%
\pgfpathlineto{\pgfqpoint{5.147714in}{1.842868in}}%
\pgfpathlineto{\pgfqpoint{5.150886in}{1.842769in}}%
\pgfpathlineto{\pgfqpoint{5.154058in}{1.842788in}}%
\pgfpathlineto{\pgfqpoint{5.157230in}{1.842801in}}%
\pgfpathlineto{\pgfqpoint{5.160402in}{1.842434in}}%
\pgfpathlineto{\pgfqpoint{5.163575in}{1.841926in}}%
\pgfpathlineto{\pgfqpoint{5.166747in}{1.841865in}}%
\pgfpathlineto{\pgfqpoint{5.169919in}{1.841961in}}%
\pgfpathlineto{\pgfqpoint{5.173091in}{1.841229in}}%
\pgfpathlineto{\pgfqpoint{5.176263in}{1.840870in}}%
\pgfpathlineto{\pgfqpoint{5.179435in}{1.840898in}}%
\pgfpathlineto{\pgfqpoint{5.182607in}{1.840739in}}%
\pgfpathlineto{\pgfqpoint{5.185779in}{1.840521in}}%
\pgfpathlineto{\pgfqpoint{5.188951in}{1.840268in}}%
\pgfpathlineto{\pgfqpoint{5.192123in}{1.840057in}}%
\pgfpathlineto{\pgfqpoint{5.195295in}{1.839531in}}%
\pgfpathlineto{\pgfqpoint{5.198467in}{1.839317in}}%
\pgfpathlineto{\pgfqpoint{5.201639in}{1.838814in}}%
\pgfpathlineto{\pgfqpoint{5.204811in}{1.838676in}}%
\pgfpathlineto{\pgfqpoint{5.207983in}{1.839185in}}%
\pgfpathlineto{\pgfqpoint{5.211155in}{1.839280in}}%
\pgfpathlineto{\pgfqpoint{5.214327in}{1.839378in}}%
\pgfpathlineto{\pgfqpoint{5.217499in}{1.839248in}}%
\pgfpathlineto{\pgfqpoint{5.220671in}{1.838772in}}%
\pgfpathlineto{\pgfqpoint{5.223843in}{1.838406in}}%
\pgfpathlineto{\pgfqpoint{5.227015in}{1.838247in}}%
\pgfpathlineto{\pgfqpoint{5.230187in}{1.837784in}}%
\pgfpathlineto{\pgfqpoint{5.233359in}{1.837825in}}%
\pgfpathlineto{\pgfqpoint{5.236531in}{1.837624in}}%
\pgfpathlineto{\pgfqpoint{5.239703in}{1.837759in}}%
\pgfpathlineto{\pgfqpoint{5.242876in}{1.838010in}}%
\pgfpathlineto{\pgfqpoint{5.246048in}{1.838340in}}%
\pgfpathlineto{\pgfqpoint{5.249220in}{1.838124in}}%
\pgfpathlineto{\pgfqpoint{5.252392in}{1.838058in}}%
\pgfpathlineto{\pgfqpoint{5.255564in}{1.838247in}}%
\pgfpathlineto{\pgfqpoint{5.258736in}{1.838540in}}%
\pgfpathlineto{\pgfqpoint{5.261908in}{1.838475in}}%
\pgfpathlineto{\pgfqpoint{5.265080in}{1.838380in}}%
\pgfpathlineto{\pgfqpoint{5.268252in}{1.839112in}}%
\pgfpathlineto{\pgfqpoint{5.271424in}{1.839611in}}%
\pgfpathlineto{\pgfqpoint{5.274596in}{1.839406in}}%
\pgfpathlineto{\pgfqpoint{5.277768in}{1.839237in}}%
\pgfpathlineto{\pgfqpoint{5.280940in}{1.839071in}}%
\pgfpathlineto{\pgfqpoint{5.284112in}{1.839077in}}%
\pgfpathlineto{\pgfqpoint{5.287284in}{1.838769in}}%
\pgfpathlineto{\pgfqpoint{5.290456in}{1.838159in}}%
\pgfpathlineto{\pgfqpoint{5.293628in}{1.838186in}}%
\pgfpathlineto{\pgfqpoint{5.296800in}{1.838004in}}%
\pgfpathlineto{\pgfqpoint{5.299972in}{1.838102in}}%
\pgfpathlineto{\pgfqpoint{5.303144in}{1.838108in}}%
\pgfpathlineto{\pgfqpoint{5.306316in}{1.838617in}}%
\pgfpathlineto{\pgfqpoint{5.309488in}{1.838733in}}%
\pgfpathlineto{\pgfqpoint{5.312660in}{1.838522in}}%
\pgfpathlineto{\pgfqpoint{5.315832in}{1.838556in}}%
\pgfpathlineto{\pgfqpoint{5.319004in}{1.838554in}}%
\pgfpathlineto{\pgfqpoint{5.322177in}{1.838253in}}%
\pgfpathlineto{\pgfqpoint{5.325349in}{1.837844in}}%
\pgfpathlineto{\pgfqpoint{5.328521in}{1.838034in}}%
\pgfpathlineto{\pgfqpoint{5.331693in}{1.838750in}}%
\pgfpathlineto{\pgfqpoint{5.334865in}{1.839269in}}%
\pgfpathlineto{\pgfqpoint{5.338037in}{1.839000in}}%
\pgfpathlineto{\pgfqpoint{5.341209in}{1.838917in}}%
\pgfpathlineto{\pgfqpoint{5.344381in}{1.839111in}}%
\pgfpathlineto{\pgfqpoint{5.347553in}{1.839016in}}%
\pgfpathlineto{\pgfqpoint{5.350725in}{1.839180in}}%
\pgfpathlineto{\pgfqpoint{5.353897in}{1.839219in}}%
\pgfpathlineto{\pgfqpoint{5.357069in}{1.839453in}}%
\pgfpathlineto{\pgfqpoint{5.360241in}{1.839296in}}%
\pgfpathlineto{\pgfqpoint{5.363413in}{1.838924in}}%
\pgfpathlineto{\pgfqpoint{5.366585in}{1.838309in}}%
\pgfpathlineto{\pgfqpoint{5.369757in}{1.838380in}}%
\pgfpathlineto{\pgfqpoint{5.372929in}{1.838191in}}%
\pgfpathlineto{\pgfqpoint{5.376101in}{1.838057in}}%
\pgfpathlineto{\pgfqpoint{5.379273in}{1.837978in}}%
\pgfpathlineto{\pgfqpoint{5.382445in}{1.837497in}}%
\pgfpathlineto{\pgfqpoint{5.385617in}{1.837557in}}%
\pgfpathlineto{\pgfqpoint{5.388789in}{1.837495in}}%
\pgfpathlineto{\pgfqpoint{5.391961in}{1.836762in}}%
\pgfpathlineto{\pgfqpoint{5.395133in}{1.836724in}}%
\pgfpathlineto{\pgfqpoint{5.398306in}{1.836924in}}%
\pgfpathlineto{\pgfqpoint{5.401478in}{1.836804in}}%
\pgfpathlineto{\pgfqpoint{5.404650in}{1.836940in}}%
\pgfpathlineto{\pgfqpoint{5.407822in}{1.837148in}}%
\pgfpathlineto{\pgfqpoint{5.410994in}{1.837083in}}%
\pgfpathlineto{\pgfqpoint{5.414166in}{1.836928in}}%
\pgfpathlineto{\pgfqpoint{5.417338in}{1.836622in}}%
\pgfpathlineto{\pgfqpoint{5.420510in}{1.836428in}}%
\pgfpathlineto{\pgfqpoint{5.423682in}{1.836581in}}%
\pgfpathlineto{\pgfqpoint{5.426854in}{1.836629in}}%
\pgfpathlineto{\pgfqpoint{5.430026in}{1.836842in}}%
\pgfpathlineto{\pgfqpoint{5.433198in}{1.836509in}}%
\pgfpathlineto{\pgfqpoint{5.436370in}{1.836649in}}%
\pgfpathlineto{\pgfqpoint{5.439542in}{1.836689in}}%
\pgfpathlineto{\pgfqpoint{5.442714in}{1.836857in}}%
\pgfpathlineto{\pgfqpoint{5.445886in}{1.836662in}}%
\pgfpathlineto{\pgfqpoint{5.449058in}{1.836582in}}%
\pgfpathlineto{\pgfqpoint{5.452230in}{1.836426in}}%
\pgfpathlineto{\pgfqpoint{5.455402in}{1.836279in}}%
\pgfpathlineto{\pgfqpoint{5.458574in}{1.836120in}}%
\pgfpathlineto{\pgfqpoint{5.461746in}{1.836157in}}%
\pgfpathlineto{\pgfqpoint{5.464918in}{1.836228in}}%
\pgfpathlineto{\pgfqpoint{5.468090in}{1.836165in}}%
\pgfpathlineto{\pgfqpoint{5.471262in}{1.836342in}}%
\pgfpathlineto{\pgfqpoint{5.474434in}{1.835657in}}%
\pgfpathlineto{\pgfqpoint{5.477607in}{1.835102in}}%
\pgfpathlineto{\pgfqpoint{5.480779in}{1.834298in}}%
\pgfpathlineto{\pgfqpoint{5.483951in}{1.834402in}}%
\pgfpathlineto{\pgfqpoint{5.487123in}{1.834037in}}%
\pgfpathlineto{\pgfqpoint{5.490295in}{1.834006in}}%
\pgfpathlineto{\pgfqpoint{5.493467in}{1.834152in}}%
\pgfpathlineto{\pgfqpoint{5.496639in}{1.834421in}}%
\pgfpathlineto{\pgfqpoint{5.499811in}{1.834713in}}%
\pgfpathlineto{\pgfqpoint{5.502983in}{1.834962in}}%
\pgfpathlineto{\pgfqpoint{5.506155in}{1.834631in}}%
\pgfpathlineto{\pgfqpoint{5.509327in}{1.833946in}}%
\pgfpathlineto{\pgfqpoint{5.512499in}{1.833918in}}%
\pgfpathlineto{\pgfqpoint{5.515671in}{1.833567in}}%
\pgfpathlineto{\pgfqpoint{5.518843in}{1.833508in}}%
\pgfpathlineto{\pgfqpoint{5.522015in}{1.833305in}}%
\pgfpathlineto{\pgfqpoint{5.525187in}{1.833059in}}%
\pgfpathlineto{\pgfqpoint{5.528359in}{1.833523in}}%
\pgfpathlineto{\pgfqpoint{5.531531in}{1.833290in}}%
\pgfpathlineto{\pgfqpoint{5.534703in}{1.833100in}}%
\pgfpathlineto{\pgfqpoint{5.537875in}{1.832863in}}%
\pgfpathlineto{\pgfqpoint{5.541047in}{1.832249in}}%
\pgfpathlineto{\pgfqpoint{5.544219in}{1.832389in}}%
\pgfpathlineto{\pgfqpoint{5.547391in}{1.832072in}}%
\pgfpathlineto{\pgfqpoint{5.550563in}{1.832139in}}%
\pgfpathlineto{\pgfqpoint{5.553735in}{1.831979in}}%
\pgfpathlineto{\pgfqpoint{5.556908in}{1.831881in}}%
\pgfpathlineto{\pgfqpoint{5.560080in}{1.831678in}}%
\pgfpathlineto{\pgfqpoint{5.563252in}{1.831514in}}%
\pgfpathlineto{\pgfqpoint{5.566424in}{1.831112in}}%
\pgfpathlineto{\pgfqpoint{5.569596in}{1.831151in}}%
\pgfpathlineto{\pgfqpoint{5.572768in}{1.831110in}}%
\pgfpathlineto{\pgfqpoint{5.575940in}{1.830968in}}%
\pgfpathlineto{\pgfqpoint{5.579112in}{1.830897in}}%
\pgfpathlineto{\pgfqpoint{5.582284in}{1.830797in}}%
\pgfpathlineto{\pgfqpoint{5.585456in}{1.830904in}}%
\pgfpathlineto{\pgfqpoint{5.588628in}{1.831277in}}%
\pgfpathlineto{\pgfqpoint{5.591800in}{1.830931in}}%
\pgfpathlineto{\pgfqpoint{5.594972in}{1.830436in}}%
\pgfpathlineto{\pgfqpoint{5.598144in}{1.829825in}}%
\pgfpathlineto{\pgfqpoint{5.601316in}{1.828970in}}%
\pgfpathlineto{\pgfqpoint{5.604488in}{1.828666in}}%
\pgfpathlineto{\pgfqpoint{5.607660in}{1.828830in}}%
\pgfpathlineto{\pgfqpoint{5.610832in}{1.828670in}}%
\pgfpathlineto{\pgfqpoint{5.614004in}{1.828370in}}%
\pgfpathlineto{\pgfqpoint{5.617176in}{1.828195in}}%
\pgfpathlineto{\pgfqpoint{5.620348in}{1.827841in}}%
\pgfpathlineto{\pgfqpoint{5.623520in}{1.828069in}}%
\pgfpathlineto{\pgfqpoint{5.626692in}{1.828205in}}%
\pgfpathlineto{\pgfqpoint{5.629864in}{1.828186in}}%
\pgfpathlineto{\pgfqpoint{5.633037in}{1.827972in}}%
\pgfpathlineto{\pgfqpoint{5.636209in}{1.827462in}}%
\pgfpathlineto{\pgfqpoint{5.639381in}{1.827371in}}%
\pgfpathlineto{\pgfqpoint{5.642553in}{1.827340in}}%
\pgfpathlineto{\pgfqpoint{5.645725in}{1.827266in}}%
\pgfpathlineto{\pgfqpoint{5.648897in}{1.827324in}}%
\pgfpathlineto{\pgfqpoint{5.652069in}{1.827148in}}%
\pgfpathlineto{\pgfqpoint{5.655241in}{1.826846in}}%
\pgfpathlineto{\pgfqpoint{5.658413in}{1.826309in}}%
\pgfpathlineto{\pgfqpoint{5.661585in}{1.826119in}}%
\pgfpathlineto{\pgfqpoint{5.664757in}{1.825257in}}%
\pgfpathlineto{\pgfqpoint{5.667929in}{1.824844in}}%
\pgfpathlineto{\pgfqpoint{5.671101in}{1.824469in}}%
\pgfpathlineto{\pgfqpoint{5.674273in}{1.824479in}}%
\pgfpathlineto{\pgfqpoint{5.677445in}{1.824385in}}%
\pgfpathlineto{\pgfqpoint{5.680617in}{1.824320in}}%
\pgfpathlineto{\pgfqpoint{5.683789in}{1.824361in}}%
\pgfpathlineto{\pgfqpoint{5.686961in}{1.824234in}}%
\pgfpathlineto{\pgfqpoint{5.690133in}{1.824311in}}%
\pgfpathlineto{\pgfqpoint{5.693305in}{1.824701in}}%
\pgfpathlineto{\pgfqpoint{5.696477in}{1.824672in}}%
\pgfpathlineto{\pgfqpoint{5.699649in}{1.824475in}}%
\pgfpathlineto{\pgfqpoint{5.702821in}{1.824639in}}%
\pgfpathlineto{\pgfqpoint{5.705993in}{1.824605in}}%
\pgfpathlineto{\pgfqpoint{5.709165in}{1.824260in}}%
\pgfpathlineto{\pgfqpoint{5.712338in}{1.824175in}}%
\pgfpathlineto{\pgfqpoint{5.715510in}{1.823469in}}%
\pgfpathlineto{\pgfqpoint{5.718682in}{1.823110in}}%
\pgfpathlineto{\pgfqpoint{5.721854in}{1.823222in}}%
\pgfpathlineto{\pgfqpoint{5.725026in}{1.822760in}}%
\pgfpathlineto{\pgfqpoint{5.728198in}{1.823092in}}%
\pgfpathlineto{\pgfqpoint{5.731370in}{1.822918in}}%
\pgfpathlineto{\pgfqpoint{5.734542in}{1.822814in}}%
\pgfpathlineto{\pgfqpoint{5.737714in}{1.822808in}}%
\pgfpathlineto{\pgfqpoint{5.740886in}{1.822756in}}%
\pgfpathlineto{\pgfqpoint{5.744058in}{1.822788in}}%
\pgfpathlineto{\pgfqpoint{5.747230in}{1.822493in}}%
\pgfpathlineto{\pgfqpoint{5.750402in}{1.822768in}}%
\pgfpathlineto{\pgfqpoint{5.753574in}{1.822586in}}%
\pgfpathlineto{\pgfqpoint{5.756746in}{1.822301in}}%
\pgfpathlineto{\pgfqpoint{5.759918in}{1.822281in}}%
\pgfpathlineto{\pgfqpoint{5.763090in}{1.822292in}}%
\pgfpathlineto{\pgfqpoint{5.766262in}{1.822047in}}%
\pgfpathlineto{\pgfqpoint{5.769434in}{1.822163in}}%
\pgfpathlineto{\pgfqpoint{5.772606in}{1.822101in}}%
\pgfpathlineto{\pgfqpoint{5.775778in}{1.821561in}}%
\pgfpathlineto{\pgfqpoint{5.778950in}{1.821461in}}%
\pgfpathlineto{\pgfqpoint{5.782122in}{1.821097in}}%
\pgfpathlineto{\pgfqpoint{5.785294in}{1.820738in}}%
\pgfpathlineto{\pgfqpoint{5.788466in}{1.820400in}}%
\pgfpathlineto{\pgfqpoint{5.791639in}{1.820503in}}%
\pgfpathlineto{\pgfqpoint{5.794811in}{1.820713in}}%
\pgfpathlineto{\pgfqpoint{5.797983in}{1.820500in}}%
\pgfpathlineto{\pgfqpoint{5.801155in}{1.820608in}}%
\pgfpathlineto{\pgfqpoint{5.804327in}{1.820390in}}%
\pgfpathlineto{\pgfqpoint{5.807499in}{1.820471in}}%
\pgfpathlineto{\pgfqpoint{5.810671in}{1.820168in}}%
\pgfpathlineto{\pgfqpoint{5.813843in}{1.820194in}}%
\pgfpathlineto{\pgfqpoint{5.817015in}{1.820138in}}%
\pgfpathlineto{\pgfqpoint{5.820187in}{1.819679in}}%
\pgfpathlineto{\pgfqpoint{5.823359in}{1.819572in}}%
\pgfpathlineto{\pgfqpoint{5.826531in}{1.819819in}}%
\pgfpathlineto{\pgfqpoint{5.829703in}{1.819620in}}%
\pgfpathlineto{\pgfqpoint{5.832875in}{1.819477in}}%
\pgfpathlineto{\pgfqpoint{5.836047in}{1.818679in}}%
\pgfpathlineto{\pgfqpoint{5.839219in}{1.818224in}}%
\pgfpathlineto{\pgfqpoint{5.842391in}{1.818202in}}%
\pgfpathlineto{\pgfqpoint{5.845563in}{1.817923in}}%
\pgfpathlineto{\pgfqpoint{5.848735in}{1.817258in}}%
\pgfpathlineto{\pgfqpoint{5.851907in}{1.816637in}}%
\pgfpathlineto{\pgfqpoint{5.855079in}{1.816553in}}%
\pgfpathlineto{\pgfqpoint{5.858251in}{1.816626in}}%
\pgfpathlineto{\pgfqpoint{5.861423in}{1.816876in}}%
\pgfpathlineto{\pgfqpoint{5.864595in}{1.817020in}}%
\pgfpathlineto{\pgfqpoint{5.867768in}{1.816845in}}%
\pgfpathlineto{\pgfqpoint{5.870940in}{1.816663in}}%
\pgfpathlineto{\pgfqpoint{5.874112in}{1.816553in}}%
\pgfpathlineto{\pgfqpoint{5.877284in}{1.816703in}}%
\pgfpathlineto{\pgfqpoint{5.880456in}{1.816090in}}%
\pgfpathlineto{\pgfqpoint{5.883628in}{1.815839in}}%
\pgfpathlineto{\pgfqpoint{5.886800in}{1.815250in}}%
\pgfpathlineto{\pgfqpoint{5.889972in}{1.815716in}}%
\pgfpathlineto{\pgfqpoint{5.893144in}{1.815546in}}%
\pgfpathlineto{\pgfqpoint{5.896316in}{1.815552in}}%
\pgfpathlineto{\pgfqpoint{5.899488in}{1.815530in}}%
\pgfpathlineto{\pgfqpoint{5.902660in}{1.815391in}}%
\pgfpathlineto{\pgfqpoint{5.905832in}{1.815288in}}%
\pgfpathlineto{\pgfqpoint{5.909004in}{1.815813in}}%
\pgfpathlineto{\pgfqpoint{5.912176in}{1.815625in}}%
\pgfpathlineto{\pgfqpoint{5.915348in}{1.815780in}}%
\pgfpathlineto{\pgfqpoint{5.918520in}{1.815908in}}%
\pgfpathlineto{\pgfqpoint{5.921692in}{1.815901in}}%
\pgfpathlineto{\pgfqpoint{5.924864in}{1.815853in}}%
\pgfpathlineto{\pgfqpoint{5.928036in}{1.815875in}}%
\pgfpathlineto{\pgfqpoint{5.931208in}{1.815816in}}%
\pgfpathlineto{\pgfqpoint{5.934380in}{1.815589in}}%
\pgfpathlineto{\pgfqpoint{5.937552in}{1.815305in}}%
\pgfpathlineto{\pgfqpoint{5.940724in}{1.814752in}}%
\pgfpathlineto{\pgfqpoint{5.943896in}{1.814749in}}%
\pgfpathlineto{\pgfqpoint{5.947069in}{1.814863in}}%
\pgfpathlineto{\pgfqpoint{5.950241in}{1.814780in}}%
\pgfpathlineto{\pgfqpoint{5.953413in}{1.814744in}}%
\pgfpathlineto{\pgfqpoint{5.956585in}{1.814861in}}%
\pgfpathlineto{\pgfqpoint{5.959757in}{1.814641in}}%
\pgfpathlineto{\pgfqpoint{5.962929in}{1.814432in}}%
\pgfpathlineto{\pgfqpoint{5.966101in}{1.814074in}}%
\pgfpathlineto{\pgfqpoint{5.969273in}{1.814041in}}%
\pgfpathlineto{\pgfqpoint{5.972445in}{1.814113in}}%
\pgfpathlineto{\pgfqpoint{5.975617in}{1.813429in}}%
\pgfpathlineto{\pgfqpoint{5.978789in}{1.813578in}}%
\pgfpathlineto{\pgfqpoint{5.981961in}{1.813802in}}%
\pgfpathlineto{\pgfqpoint{5.985133in}{1.813847in}}%
\pgfpathlineto{\pgfqpoint{5.988305in}{1.813284in}}%
\pgfpathlineto{\pgfqpoint{5.991477in}{1.812451in}}%
\pgfpathlineto{\pgfqpoint{5.994649in}{1.812156in}}%
\pgfpathlineto{\pgfqpoint{5.997821in}{1.811274in}}%
\pgfpathlineto{\pgfqpoint{6.000993in}{1.811131in}}%
\pgfpathlineto{\pgfqpoint{6.004165in}{1.811162in}}%
\pgfpathlineto{\pgfqpoint{6.007337in}{1.811250in}}%
\pgfpathlineto{\pgfqpoint{6.010509in}{1.811113in}}%
\pgfpathlineto{\pgfqpoint{6.013681in}{1.810938in}}%
\pgfpathlineto{\pgfqpoint{6.016853in}{1.810785in}}%
\pgfpathlineto{\pgfqpoint{6.020025in}{1.810787in}}%
\pgfpathlineto{\pgfqpoint{6.023197in}{1.810428in}}%
\pgfpathlineto{\pgfqpoint{6.026370in}{1.810068in}}%
\pgfpathlineto{\pgfqpoint{6.029542in}{1.810095in}}%
\pgfpathlineto{\pgfqpoint{6.032714in}{1.810399in}}%
\pgfpathlineto{\pgfqpoint{6.035886in}{1.810457in}}%
\pgfpathlineto{\pgfqpoint{6.039058in}{1.810868in}}%
\pgfpathlineto{\pgfqpoint{6.042230in}{1.810949in}}%
\pgfpathlineto{\pgfqpoint{6.045402in}{1.810910in}}%
\pgfpathlineto{\pgfqpoint{6.048574in}{1.810871in}}%
\pgfpathlineto{\pgfqpoint{6.051746in}{1.810304in}}%
\pgfpathlineto{\pgfqpoint{6.054918in}{1.810376in}}%
\pgfpathlineto{\pgfqpoint{6.058090in}{1.810329in}}%
\pgfpathlineto{\pgfqpoint{6.061262in}{1.810225in}}%
\pgfpathlineto{\pgfqpoint{6.064434in}{1.809489in}}%
\pgfpathlineto{\pgfqpoint{6.067606in}{1.809596in}}%
\pgfpathlineto{\pgfqpoint{6.070778in}{1.809340in}}%
\pgfpathlineto{\pgfqpoint{6.073950in}{1.809558in}}%
\pgfpathlineto{\pgfqpoint{6.077122in}{1.809829in}}%
\pgfpathlineto{\pgfqpoint{6.080294in}{1.809779in}}%
\pgfpathlineto{\pgfqpoint{6.083466in}{1.808976in}}%
\pgfpathlineto{\pgfqpoint{6.086638in}{1.808578in}}%
\pgfpathlineto{\pgfqpoint{6.089810in}{1.808495in}}%
\pgfpathlineto{\pgfqpoint{6.092982in}{1.807932in}}%
\pgfpathlineto{\pgfqpoint{6.096154in}{1.808103in}}%
\pgfpathlineto{\pgfqpoint{6.099326in}{1.808209in}}%
\pgfpathlineto{\pgfqpoint{6.102499in}{1.807755in}}%
\pgfpathlineto{\pgfqpoint{6.105671in}{1.807473in}}%
\pgfpathlineto{\pgfqpoint{6.108843in}{1.807037in}}%
\pgfpathlineto{\pgfqpoint{6.112015in}{1.806955in}}%
\pgfpathlineto{\pgfqpoint{6.115187in}{1.806764in}}%
\pgfpathlineto{\pgfqpoint{6.118359in}{1.806874in}}%
\pgfpathlineto{\pgfqpoint{6.121531in}{1.806837in}}%
\pgfpathlineto{\pgfqpoint{6.124703in}{1.807116in}}%
\pgfpathlineto{\pgfqpoint{6.127875in}{1.807023in}}%
\pgfpathlineto{\pgfqpoint{6.131047in}{1.806837in}}%
\pgfpathlineto{\pgfqpoint{6.134219in}{1.806500in}}%
\pgfpathlineto{\pgfqpoint{6.137391in}{1.806695in}}%
\pgfpathlineto{\pgfqpoint{6.140563in}{1.806894in}}%
\pgfpathlineto{\pgfqpoint{6.143735in}{1.806785in}}%
\pgfpathlineto{\pgfqpoint{6.146907in}{1.806484in}}%
\pgfpathlineto{\pgfqpoint{6.150079in}{1.806010in}}%
\pgfpathlineto{\pgfqpoint{6.153251in}{1.805833in}}%
\pgfpathlineto{\pgfqpoint{6.156423in}{1.805770in}}%
\pgfpathlineto{\pgfqpoint{6.159595in}{1.805650in}}%
\pgfpathlineto{\pgfqpoint{6.162767in}{1.805508in}}%
\pgfpathlineto{\pgfqpoint{6.165939in}{1.805889in}}%
\pgfpathlineto{\pgfqpoint{6.169111in}{1.806011in}}%
\pgfpathlineto{\pgfqpoint{6.172283in}{1.805981in}}%
\pgfpathlineto{\pgfqpoint{6.175455in}{1.805854in}}%
\pgfpathlineto{\pgfqpoint{6.178627in}{1.805366in}}%
\pgfpathlineto{\pgfqpoint{6.181800in}{1.805035in}}%
\pgfpathlineto{\pgfqpoint{6.184972in}{1.804589in}}%
\pgfpathlineto{\pgfqpoint{6.188144in}{1.804619in}}%
\pgfpathlineto{\pgfqpoint{6.191316in}{1.804521in}}%
\pgfpathlineto{\pgfqpoint{6.194488in}{1.804081in}}%
\pgfpathlineto{\pgfqpoint{6.197660in}{1.804003in}}%
\pgfpathlineto{\pgfqpoint{6.200832in}{1.804212in}}%
\pgfpathlineto{\pgfqpoint{6.204004in}{1.803650in}}%
\pgfpathlineto{\pgfqpoint{6.207176in}{1.803387in}}%
\pgfpathlineto{\pgfqpoint{6.210348in}{1.803406in}}%
\pgfpathlineto{\pgfqpoint{6.213520in}{1.803493in}}%
\pgfpathlineto{\pgfqpoint{6.216692in}{1.803615in}}%
\pgfpathlineto{\pgfqpoint{6.219864in}{1.803539in}}%
\pgfpathlineto{\pgfqpoint{6.223036in}{1.803461in}}%
\pgfpathlineto{\pgfqpoint{6.226208in}{1.803332in}}%
\pgfpathlineto{\pgfqpoint{6.229380in}{1.803451in}}%
\pgfpathlineto{\pgfqpoint{6.232552in}{1.803292in}}%
\pgfpathlineto{\pgfqpoint{6.235724in}{1.803284in}}%
\pgfpathlineto{\pgfqpoint{6.238896in}{1.803206in}}%
\pgfpathlineto{\pgfqpoint{6.242068in}{1.802883in}}%
\pgfpathlineto{\pgfqpoint{6.245240in}{1.802910in}}%
\pgfpathlineto{\pgfqpoint{6.248412in}{1.802825in}}%
\pgfpathlineto{\pgfqpoint{6.251584in}{1.802720in}}%
\pgfpathlineto{\pgfqpoint{6.254756in}{1.802901in}}%
\pgfpathlineto{\pgfqpoint{6.257928in}{1.802937in}}%
\pgfpathlineto{\pgfqpoint{6.261101in}{1.803036in}}%
\pgfpathlineto{\pgfqpoint{6.264273in}{1.802964in}}%
\pgfpathlineto{\pgfqpoint{6.267445in}{1.802777in}}%
\pgfpathlineto{\pgfqpoint{6.270617in}{1.802111in}}%
\pgfpathlineto{\pgfqpoint{6.273789in}{1.801943in}}%
\pgfpathlineto{\pgfqpoint{6.276961in}{1.801862in}}%
\pgfpathlineto{\pgfqpoint{6.280133in}{1.801468in}}%
\pgfpathlineto{\pgfqpoint{6.283305in}{1.801691in}}%
\pgfpathlineto{\pgfqpoint{6.286477in}{1.801060in}}%
\pgfpathlineto{\pgfqpoint{6.289649in}{1.800784in}}%
\pgfpathlineto{\pgfqpoint{6.292821in}{1.800818in}}%
\pgfpathlineto{\pgfqpoint{6.295993in}{1.801230in}}%
\pgfpathlineto{\pgfqpoint{6.299165in}{1.801295in}}%
\pgfpathlineto{\pgfqpoint{6.302337in}{1.801144in}}%
\pgfpathlineto{\pgfqpoint{6.305509in}{1.801009in}}%
\pgfpathlineto{\pgfqpoint{6.308681in}{1.800719in}}%
\pgfpathlineto{\pgfqpoint{6.311853in}{1.800607in}}%
\pgfpathlineto{\pgfqpoint{6.315025in}{1.800197in}}%
\pgfpathlineto{\pgfqpoint{6.318197in}{1.800183in}}%
\pgfpathlineto{\pgfqpoint{6.321369in}{1.800091in}}%
\pgfpathlineto{\pgfqpoint{6.324541in}{1.800127in}}%
\pgfpathlineto{\pgfqpoint{6.327713in}{1.800053in}}%
\pgfpathlineto{\pgfqpoint{6.330885in}{1.800001in}}%
\pgfpathlineto{\pgfqpoint{6.334057in}{1.799989in}}%
\pgfpathlineto{\pgfqpoint{6.337230in}{1.800144in}}%
\pgfpathlineto{\pgfqpoint{6.340402in}{1.799949in}}%
\pgfpathlineto{\pgfqpoint{6.343574in}{1.799975in}}%
\pgfpathlineto{\pgfqpoint{6.346746in}{1.800402in}}%
\pgfpathlineto{\pgfqpoint{6.349918in}{1.800096in}}%
\pgfpathlineto{\pgfqpoint{6.353090in}{1.800176in}}%
\pgfpathlineto{\pgfqpoint{6.356262in}{1.800164in}}%
\pgfpathlineto{\pgfqpoint{6.359434in}{1.799929in}}%
\pgfpathlineto{\pgfqpoint{6.362606in}{1.799635in}}%
\pgfpathlineto{\pgfqpoint{6.365778in}{1.800163in}}%
\pgfpathlineto{\pgfqpoint{6.368950in}{1.799950in}}%
\pgfpathlineto{\pgfqpoint{6.372122in}{1.799677in}}%
\pgfpathlineto{\pgfqpoint{6.375294in}{1.799613in}}%
\pgfpathlineto{\pgfqpoint{6.378466in}{1.799417in}}%
\pgfpathlineto{\pgfqpoint{6.381638in}{1.799601in}}%
\pgfpathlineto{\pgfqpoint{6.384810in}{1.799370in}}%
\pgfpathlineto{\pgfqpoint{6.387982in}{1.799225in}}%
\pgfpathlineto{\pgfqpoint{6.391154in}{1.798853in}}%
\pgfpathlineto{\pgfqpoint{6.394326in}{1.798455in}}%
\pgfpathlineto{\pgfqpoint{6.397498in}{1.798417in}}%
\pgfpathlineto{\pgfqpoint{6.400670in}{1.798184in}}%
\pgfpathlineto{\pgfqpoint{6.403842in}{1.798067in}}%
\pgfpathlineto{\pgfqpoint{6.407014in}{1.797749in}}%
\pgfpathlineto{\pgfqpoint{6.410186in}{1.797572in}}%
\pgfpathlineto{\pgfqpoint{6.413358in}{1.797161in}}%
\pgfpathlineto{\pgfqpoint{6.416531in}{1.796793in}}%
\pgfpathlineto{\pgfqpoint{6.419703in}{1.796438in}}%
\pgfpathlineto{\pgfqpoint{6.422875in}{1.796435in}}%
\pgfpathlineto{\pgfqpoint{6.426047in}{1.796390in}}%
\pgfpathlineto{\pgfqpoint{6.429219in}{1.796047in}}%
\pgfpathlineto{\pgfqpoint{6.432391in}{1.795775in}}%
\pgfpathlineto{\pgfqpoint{6.435563in}{1.795591in}}%
\pgfpathlineto{\pgfqpoint{6.438735in}{1.795523in}}%
\pgfpathlineto{\pgfqpoint{6.441907in}{1.795244in}}%
\pgfpathlineto{\pgfqpoint{6.445079in}{1.795337in}}%
\pgfpathlineto{\pgfqpoint{6.448251in}{1.795499in}}%
\pgfpathlineto{\pgfqpoint{6.451423in}{1.794897in}}%
\pgfpathlineto{\pgfqpoint{6.454595in}{1.794728in}}%
\pgfpathlineto{\pgfqpoint{6.457767in}{1.794996in}}%
\pgfpathlineto{\pgfqpoint{6.460939in}{1.795329in}}%
\pgfpathlineto{\pgfqpoint{6.464111in}{1.795805in}}%
\pgfpathlineto{\pgfqpoint{6.467283in}{1.795863in}}%
\pgfpathlineto{\pgfqpoint{6.470455in}{1.795604in}}%
\pgfpathlineto{\pgfqpoint{6.473627in}{1.795573in}}%
\pgfpathlineto{\pgfqpoint{6.476799in}{1.795589in}}%
\pgfpathlineto{\pgfqpoint{6.479971in}{1.795273in}}%
\pgfpathlineto{\pgfqpoint{6.483143in}{1.795359in}}%
\pgfpathlineto{\pgfqpoint{6.486315in}{1.795127in}}%
\pgfpathlineto{\pgfqpoint{6.489487in}{1.795265in}}%
\pgfpathlineto{\pgfqpoint{6.492659in}{1.795346in}}%
\pgfpathlineto{\pgfqpoint{6.495832in}{1.795399in}}%
\pgfpathlineto{\pgfqpoint{6.499004in}{1.795416in}}%
\pgfpathlineto{\pgfqpoint{6.502176in}{1.795414in}}%
\pgfpathlineto{\pgfqpoint{6.505348in}{1.795433in}}%
\pgfpathlineto{\pgfqpoint{6.508520in}{1.795362in}}%
\pgfpathlineto{\pgfqpoint{6.511692in}{1.795777in}}%
\pgfpathlineto{\pgfqpoint{6.514864in}{1.795255in}}%
\pgfpathlineto{\pgfqpoint{6.518036in}{1.795145in}}%
\pgfpathlineto{\pgfqpoint{6.521208in}{1.795077in}}%
\pgfpathlineto{\pgfqpoint{6.524380in}{1.794420in}}%
\pgfpathlineto{\pgfqpoint{6.527552in}{1.794239in}}%
\pgfpathlineto{\pgfqpoint{6.530724in}{1.793935in}}%
\pgfpathlineto{\pgfqpoint{6.533896in}{1.793972in}}%
\pgfpathlineto{\pgfqpoint{6.537068in}{1.793613in}}%
\pgfpathlineto{\pgfqpoint{6.540240in}{1.793435in}}%
\pgfpathlineto{\pgfqpoint{6.543412in}{1.793114in}}%
\pgfpathlineto{\pgfqpoint{6.546584in}{1.792276in}}%
\pgfpathlineto{\pgfqpoint{6.549756in}{1.792420in}}%
\pgfpathlineto{\pgfqpoint{6.552928in}{1.792111in}}%
\pgfpathlineto{\pgfqpoint{6.556100in}{1.792232in}}%
\pgfpathlineto{\pgfqpoint{6.559272in}{1.792056in}}%
\pgfpathlineto{\pgfqpoint{6.562444in}{1.791883in}}%
\pgfpathlineto{\pgfqpoint{6.565616in}{1.792115in}}%
\pgfpathlineto{\pgfqpoint{6.568788in}{1.792126in}}%
\pgfpathlineto{\pgfqpoint{6.571961in}{1.792130in}}%
\pgfpathlineto{\pgfqpoint{6.575133in}{1.792764in}}%
\pgfpathlineto{\pgfqpoint{6.578305in}{1.792680in}}%
\pgfpathlineto{\pgfqpoint{6.581477in}{1.792583in}}%
\pgfpathlineto{\pgfqpoint{6.584649in}{1.792726in}}%
\pgfpathlineto{\pgfqpoint{6.587821in}{1.793134in}}%
\pgfpathlineto{\pgfqpoint{6.590993in}{1.792917in}}%
\pgfpathlineto{\pgfqpoint{6.594165in}{1.792947in}}%
\pgfpathlineto{\pgfqpoint{6.597337in}{1.792855in}}%
\pgfpathlineto{\pgfqpoint{6.600509in}{1.792360in}}%
\pgfpathlineto{\pgfqpoint{6.603681in}{1.791802in}}%
\pgfpathlineto{\pgfqpoint{6.606853in}{1.791778in}}%
\pgfpathlineto{\pgfqpoint{6.610025in}{1.791799in}}%
\pgfpathlineto{\pgfqpoint{6.613197in}{1.791214in}}%
\pgfpathlineto{\pgfqpoint{6.616369in}{1.790769in}}%
\pgfpathlineto{\pgfqpoint{6.619541in}{1.790575in}}%
\pgfpathlineto{\pgfqpoint{6.622713in}{1.790443in}}%
\pgfpathlineto{\pgfqpoint{6.625885in}{1.790864in}}%
\pgfpathlineto{\pgfqpoint{6.629057in}{1.790368in}}%
\pgfpathlineto{\pgfqpoint{6.632229in}{1.790489in}}%
\pgfpathlineto{\pgfqpoint{6.635401in}{1.790709in}}%
\pgfpathlineto{\pgfqpoint{6.638573in}{1.790910in}}%
\pgfpathlineto{\pgfqpoint{6.641745in}{1.790861in}}%
\pgfpathlineto{\pgfqpoint{6.644917in}{1.790748in}}%
\pgfpathlineto{\pgfqpoint{6.648089in}{1.790326in}}%
\pgfpathlineto{\pgfqpoint{6.651262in}{1.790202in}}%
\pgfpathlineto{\pgfqpoint{6.654434in}{1.790023in}}%
\pgfpathlineto{\pgfqpoint{6.657606in}{1.789981in}}%
\pgfpathlineto{\pgfqpoint{6.660778in}{1.789549in}}%
\pgfpathlineto{\pgfqpoint{6.663950in}{1.789284in}}%
\pgfpathlineto{\pgfqpoint{6.667122in}{1.789231in}}%
\pgfpathlineto{\pgfqpoint{6.670294in}{1.789076in}}%
\pgfpathlineto{\pgfqpoint{6.673466in}{1.788873in}}%
\pgfpathlineto{\pgfqpoint{6.676638in}{1.788675in}}%
\pgfpathlineto{\pgfqpoint{6.679810in}{1.788930in}}%
\pgfpathlineto{\pgfqpoint{6.682982in}{1.788527in}}%
\pgfpathlineto{\pgfqpoint{6.686154in}{1.788484in}}%
\pgfpathlineto{\pgfqpoint{6.689326in}{1.788012in}}%
\pgfpathlineto{\pgfqpoint{6.692498in}{1.788045in}}%
\pgfpathlineto{\pgfqpoint{6.695670in}{1.788137in}}%
\pgfpathlineto{\pgfqpoint{6.698842in}{1.788094in}}%
\pgfpathlineto{\pgfqpoint{6.702014in}{1.787858in}}%
\pgfpathlineto{\pgfqpoint{6.705186in}{1.787648in}}%
\pgfpathlineto{\pgfqpoint{6.708358in}{1.787710in}}%
\pgfpathlineto{\pgfqpoint{6.711530in}{1.787648in}}%
\pgfpathlineto{\pgfqpoint{6.714702in}{1.787742in}}%
\pgfpathlineto{\pgfqpoint{6.717874in}{1.787307in}}%
\pgfpathlineto{\pgfqpoint{6.721046in}{1.787281in}}%
\pgfpathlineto{\pgfqpoint{6.724218in}{1.787189in}}%
\pgfpathlineto{\pgfqpoint{6.727391in}{1.786726in}}%
\pgfpathlineto{\pgfqpoint{6.730563in}{1.786479in}}%
\pgfpathlineto{\pgfqpoint{6.733735in}{1.786372in}}%
\pgfpathlineto{\pgfqpoint{6.736907in}{1.786400in}}%
\pgfpathlineto{\pgfqpoint{6.740079in}{1.785969in}}%
\pgfpathlineto{\pgfqpoint{6.743251in}{1.785519in}}%
\pgfpathlineto{\pgfqpoint{6.746423in}{1.785059in}}%
\pgfpathlineto{\pgfqpoint{6.749595in}{1.784514in}}%
\pgfpathlineto{\pgfqpoint{6.752767in}{1.784419in}}%
\pgfpathlineto{\pgfqpoint{6.755939in}{1.784168in}}%
\pgfpathlineto{\pgfqpoint{6.759111in}{1.784142in}}%
\pgfpathlineto{\pgfqpoint{6.762283in}{1.784544in}}%
\pgfpathlineto{\pgfqpoint{6.765455in}{1.784926in}}%
\pgfpathlineto{\pgfqpoint{6.768627in}{1.784863in}}%
\pgfpathlineto{\pgfqpoint{6.771799in}{1.784730in}}%
\pgfpathlineto{\pgfqpoint{6.774971in}{1.784557in}}%
\pgfpathlineto{\pgfqpoint{6.778143in}{1.784257in}}%
\pgfpathlineto{\pgfqpoint{6.781315in}{1.784419in}}%
\pgfpathlineto{\pgfqpoint{6.784487in}{1.784452in}}%
\pgfpathlineto{\pgfqpoint{6.787659in}{1.784581in}}%
\pgfpathlineto{\pgfqpoint{6.790831in}{1.784723in}}%
\pgfpathlineto{\pgfqpoint{6.794003in}{1.784698in}}%
\pgfpathlineto{\pgfqpoint{6.797175in}{1.784795in}}%
\pgfpathlineto{\pgfqpoint{6.800347in}{1.784724in}}%
\pgfpathlineto{\pgfqpoint{6.803519in}{1.784503in}}%
\pgfpathlineto{\pgfqpoint{6.806692in}{1.784261in}}%
\pgfpathlineto{\pgfqpoint{6.809864in}{1.784224in}}%
\pgfpathlineto{\pgfqpoint{6.813036in}{1.784026in}}%
\pgfpathlineto{\pgfqpoint{6.816208in}{1.783647in}}%
\pgfpathlineto{\pgfqpoint{6.819380in}{1.783099in}}%
\pgfpathlineto{\pgfqpoint{6.822552in}{1.783129in}}%
\pgfpathlineto{\pgfqpoint{6.825724in}{1.783003in}}%
\pgfpathlineto{\pgfqpoint{6.828896in}{1.782613in}}%
\pgfpathlineto{\pgfqpoint{6.832068in}{1.782662in}}%
\pgfpathlineto{\pgfqpoint{6.835240in}{1.782096in}}%
\pgfpathlineto{\pgfqpoint{6.838412in}{1.782253in}}%
\pgfpathlineto{\pgfqpoint{6.841584in}{1.782119in}}%
\pgfpathlineto{\pgfqpoint{6.844756in}{1.782282in}}%
\pgfpathlineto{\pgfqpoint{6.847928in}{1.782226in}}%
\pgfpathlineto{\pgfqpoint{6.851100in}{1.782213in}}%
\pgfpathlineto{\pgfqpoint{6.854272in}{1.782302in}}%
\pgfpathlineto{\pgfqpoint{6.857444in}{1.782066in}}%
\pgfpathlineto{\pgfqpoint{6.860616in}{1.781337in}}%
\pgfpathlineto{\pgfqpoint{6.863788in}{1.781380in}}%
\pgfpathlineto{\pgfqpoint{6.866960in}{1.781141in}}%
\pgfpathlineto{\pgfqpoint{6.870132in}{1.781293in}}%
\pgfpathlineto{\pgfqpoint{6.873304in}{1.781107in}}%
\pgfpathlineto{\pgfqpoint{6.876476in}{1.781023in}}%
\pgfpathlineto{\pgfqpoint{6.879648in}{1.781078in}}%
\pgfpathlineto{\pgfqpoint{6.882820in}{1.780946in}}%
\pgfpathlineto{\pgfqpoint{6.885993in}{1.780873in}}%
\pgfpathlineto{\pgfqpoint{6.889165in}{1.780794in}}%
\pgfpathlineto{\pgfqpoint{6.892337in}{1.780777in}}%
\pgfpathlineto{\pgfqpoint{6.895509in}{1.780442in}}%
\pgfpathlineto{\pgfqpoint{6.898681in}{1.780541in}}%
\pgfpathlineto{\pgfqpoint{6.901853in}{1.780671in}}%
\pgfpathlineto{\pgfqpoint{6.905025in}{1.780223in}}%
\pgfpathlineto{\pgfqpoint{6.908197in}{1.780315in}}%
\pgfpathlineto{\pgfqpoint{6.911369in}{1.780082in}}%
\pgfpathlineto{\pgfqpoint{6.914541in}{1.779859in}}%
\pgfpathlineto{\pgfqpoint{6.917713in}{1.780038in}}%
\pgfpathlineto{\pgfqpoint{6.920885in}{1.779663in}}%
\pgfpathlineto{\pgfqpoint{6.924057in}{1.779625in}}%
\pgfpathlineto{\pgfqpoint{6.927229in}{1.779640in}}%
\pgfpathlineto{\pgfqpoint{6.930401in}{1.779507in}}%
\pgfpathlineto{\pgfqpoint{6.933573in}{1.779595in}}%
\pgfpathlineto{\pgfqpoint{6.936745in}{1.779219in}}%
\pgfpathlineto{\pgfqpoint{6.939917in}{1.779187in}}%
\pgfpathlineto{\pgfqpoint{6.943089in}{1.779435in}}%
\pgfpathlineto{\pgfqpoint{6.946261in}{1.778973in}}%
\pgfpathlineto{\pgfqpoint{6.949433in}{1.778944in}}%
\pgfpathlineto{\pgfqpoint{6.952605in}{1.778524in}}%
\pgfpathlineto{\pgfqpoint{6.955777in}{1.778776in}}%
\pgfpathlineto{\pgfqpoint{6.958949in}{1.778930in}}%
\pgfpathlineto{\pgfqpoint{6.962122in}{1.778826in}}%
\pgfpathlineto{\pgfqpoint{6.965294in}{1.779090in}}%
\pgfpathlineto{\pgfqpoint{6.968466in}{1.778924in}}%
\pgfpathlineto{\pgfqpoint{6.971638in}{1.778855in}}%
\pgfpathlineto{\pgfqpoint{6.974810in}{1.778302in}}%
\pgfpathlineto{\pgfqpoint{6.977982in}{1.778611in}}%
\pgfpathlineto{\pgfqpoint{6.981154in}{1.778445in}}%
\pgfpathlineto{\pgfqpoint{6.984326in}{1.778573in}}%
\pgfpathlineto{\pgfqpoint{6.987498in}{1.778087in}}%
\pgfpathlineto{\pgfqpoint{6.990670in}{1.777653in}}%
\pgfpathlineto{\pgfqpoint{6.993842in}{1.777191in}}%
\pgfpathlineto{\pgfqpoint{6.997014in}{1.776753in}}%
\pgfpathlineto{\pgfqpoint{7.000186in}{1.776734in}}%
\pgfpathlineto{\pgfqpoint{7.003358in}{1.776040in}}%
\pgfpathlineto{\pgfqpoint{7.006530in}{1.776069in}}%
\pgfpathlineto{\pgfqpoint{7.009702in}{1.775791in}}%
\pgfpathlineto{\pgfqpoint{7.012874in}{1.775575in}}%
\pgfpathlineto{\pgfqpoint{7.016046in}{1.775281in}}%
\pgfpathlineto{\pgfqpoint{7.019218in}{1.774902in}}%
\pgfpathlineto{\pgfqpoint{7.022390in}{1.774804in}}%
\pgfpathlineto{\pgfqpoint{7.025562in}{1.774822in}}%
\pgfpathlineto{\pgfqpoint{7.028734in}{1.774811in}}%
\pgfpathlineto{\pgfqpoint{7.031906in}{1.774812in}}%
\pgfpathlineto{\pgfqpoint{7.035078in}{1.774819in}}%
\pgfpathlineto{\pgfqpoint{7.038250in}{1.774671in}}%
\pgfpathlineto{\pgfqpoint{7.041423in}{1.774540in}}%
\pgfpathlineto{\pgfqpoint{7.044595in}{1.774272in}}%
\pgfpathlineto{\pgfqpoint{7.047767in}{1.774149in}}%
\pgfpathlineto{\pgfqpoint{7.050939in}{1.774321in}}%
\pgfpathlineto{\pgfqpoint{7.054111in}{1.773912in}}%
\pgfpathlineto{\pgfqpoint{7.057283in}{1.773655in}}%
\pgfpathlineto{\pgfqpoint{7.060455in}{1.773683in}}%
\pgfpathlineto{\pgfqpoint{7.063627in}{1.773638in}}%
\pgfpathlineto{\pgfqpoint{7.066799in}{1.773477in}}%
\pgfpathlineto{\pgfqpoint{7.069971in}{1.773300in}}%
\pgfpathlineto{\pgfqpoint{7.073143in}{1.773247in}}%
\pgfpathlineto{\pgfqpoint{7.076315in}{1.772955in}}%
\pgfpathlineto{\pgfqpoint{7.079487in}{1.772983in}}%
\pgfpathlineto{\pgfqpoint{7.082659in}{1.772982in}}%
\pgfpathlineto{\pgfqpoint{7.085831in}{1.772874in}}%
\pgfpathlineto{\pgfqpoint{7.089003in}{1.772749in}}%
\pgfpathlineto{\pgfqpoint{7.092175in}{1.772123in}}%
\pgfpathlineto{\pgfqpoint{7.095347in}{1.771905in}}%
\pgfpathlineto{\pgfqpoint{7.098519in}{1.771400in}}%
\pgfpathlineto{\pgfqpoint{7.101691in}{1.771518in}}%
\pgfpathlineto{\pgfqpoint{7.104863in}{1.771421in}}%
\pgfpathlineto{\pgfqpoint{7.108035in}{1.771440in}}%
\pgfpathlineto{\pgfqpoint{7.111207in}{1.771133in}}%
\pgfpathlineto{\pgfqpoint{7.114379in}{1.771097in}}%
\pgfpathlineto{\pgfqpoint{7.117551in}{1.771196in}}%
\pgfpathlineto{\pgfqpoint{7.120724in}{1.771127in}}%
\pgfpathlineto{\pgfqpoint{7.123896in}{1.771121in}}%
\pgfpathlineto{\pgfqpoint{7.127068in}{1.771111in}}%
\pgfpathlineto{\pgfqpoint{7.130240in}{1.771333in}}%
\pgfpathlineto{\pgfqpoint{7.133412in}{1.771058in}}%
\pgfpathlineto{\pgfqpoint{7.136584in}{1.770700in}}%
\pgfpathlineto{\pgfqpoint{7.139756in}{1.770926in}}%
\pgfpathlineto{\pgfqpoint{7.142928in}{1.770682in}}%
\pgfpathlineto{\pgfqpoint{7.146100in}{1.770202in}}%
\pgfpathlineto{\pgfqpoint{7.149272in}{1.769943in}}%
\pgfpathlineto{\pgfqpoint{7.152444in}{1.769878in}}%
\pgfpathlineto{\pgfqpoint{7.155616in}{1.769606in}}%
\pgfpathlineto{\pgfqpoint{7.158788in}{1.769530in}}%
\pgfpathlineto{\pgfqpoint{7.161960in}{1.769023in}}%
\pgfpathlineto{\pgfqpoint{7.165132in}{1.768560in}}%
\pgfpathlineto{\pgfqpoint{7.168304in}{1.768028in}}%
\pgfpathlineto{\pgfqpoint{7.171476in}{1.767760in}}%
\pgfpathlineto{\pgfqpoint{7.174648in}{1.767637in}}%
\pgfpathlineto{\pgfqpoint{7.177820in}{1.767278in}}%
\pgfpathlineto{\pgfqpoint{7.180992in}{1.767195in}}%
\pgfpathlineto{\pgfqpoint{7.184164in}{1.767733in}}%
\pgfpathlineto{\pgfqpoint{7.187336in}{1.767757in}}%
\pgfpathlineto{\pgfqpoint{7.190508in}{1.767847in}}%
\pgfpathlineto{\pgfqpoint{7.193680in}{1.767924in}}%
\pgfpathlineto{\pgfqpoint{7.196853in}{1.768115in}}%
\pgfpathlineto{\pgfqpoint{7.200025in}{1.768006in}}%
\pgfpathlineto{\pgfqpoint{7.203197in}{1.767953in}}%
\pgfpathlineto{\pgfqpoint{7.206369in}{1.767811in}}%
\pgfpathlineto{\pgfqpoint{7.209541in}{1.768155in}}%
\pgfpathlineto{\pgfqpoint{7.212713in}{1.767620in}}%
\pgfpathlineto{\pgfqpoint{7.215885in}{1.767147in}}%
\pgfpathlineto{\pgfqpoint{7.219057in}{1.767401in}}%
\pgfpathlineto{\pgfqpoint{7.222229in}{1.766825in}}%
\pgfpathlineto{\pgfqpoint{7.225401in}{1.766645in}}%
\pgfpathlineto{\pgfqpoint{7.228573in}{1.766388in}}%
\pgfpathlineto{\pgfqpoint{7.231745in}{1.766334in}}%
\pgfpathlineto{\pgfqpoint{7.234917in}{1.765997in}}%
\pgfpathlineto{\pgfqpoint{7.238089in}{1.765985in}}%
\pgfpathlineto{\pgfqpoint{7.241261in}{1.765997in}}%
\pgfpathlineto{\pgfqpoint{7.244433in}{1.765694in}}%
\pgfpathlineto{\pgfqpoint{7.247605in}{1.765144in}}%
\pgfpathlineto{\pgfqpoint{7.250777in}{1.764836in}}%
\pgfpathlineto{\pgfqpoint{7.253949in}{1.764945in}}%
\pgfpathlineto{\pgfqpoint{7.257121in}{1.764959in}}%
\pgfpathlineto{\pgfqpoint{7.260293in}{1.764898in}}%
\pgfpathlineto{\pgfqpoint{7.263465in}{1.764459in}}%
\pgfpathlineto{\pgfqpoint{7.266637in}{1.764408in}}%
\pgfpathlineto{\pgfqpoint{7.269809in}{1.764370in}}%
\pgfpathlineto{\pgfqpoint{7.272981in}{1.764257in}}%
\pgfpathlineto{\pgfqpoint{7.276154in}{1.764023in}}%
\pgfpathlineto{\pgfqpoint{7.279326in}{1.763664in}}%
\pgfpathlineto{\pgfqpoint{7.282498in}{1.763510in}}%
\pgfpathlineto{\pgfqpoint{7.285670in}{1.762718in}}%
\pgfpathlineto{\pgfqpoint{7.288842in}{1.762509in}}%
\pgfpathlineto{\pgfqpoint{7.292014in}{1.762338in}}%
\pgfpathlineto{\pgfqpoint{7.295186in}{1.761906in}}%
\pgfpathlineto{\pgfqpoint{7.298358in}{1.761729in}}%
\pgfpathlineto{\pgfqpoint{7.301530in}{1.761440in}}%
\pgfpathlineto{\pgfqpoint{7.304702in}{1.761209in}}%
\pgfpathlineto{\pgfqpoint{7.307874in}{1.760594in}}%
\pgfpathlineto{\pgfqpoint{7.311046in}{1.760951in}}%
\pgfpathlineto{\pgfqpoint{7.314218in}{1.761168in}}%
\pgfpathlineto{\pgfqpoint{7.317390in}{1.761177in}}%
\pgfpathlineto{\pgfqpoint{7.320562in}{1.761394in}}%
\pgfpathlineto{\pgfqpoint{7.323734in}{1.761250in}}%
\pgfpathlineto{\pgfqpoint{7.326906in}{1.761230in}}%
\pgfpathlineto{\pgfqpoint{7.330078in}{1.760759in}}%
\pgfpathlineto{\pgfqpoint{7.333250in}{1.760293in}}%
\pgfpathlineto{\pgfqpoint{7.336422in}{1.760374in}}%
\pgfpathlineto{\pgfqpoint{7.339594in}{1.760444in}}%
\pgfpathlineto{\pgfqpoint{7.342766in}{1.759982in}}%
\pgfpathlineto{\pgfqpoint{7.345938in}{1.759819in}}%
\pgfpathlineto{\pgfqpoint{7.349110in}{1.759492in}}%
\pgfpathlineto{\pgfqpoint{7.352282in}{1.759223in}}%
\pgfpathlineto{\pgfqpoint{7.355455in}{1.759512in}}%
\pgfpathlineto{\pgfqpoint{7.358627in}{1.759349in}}%
\pgfpathlineto{\pgfqpoint{7.361799in}{1.759354in}}%
\pgfpathlineto{\pgfqpoint{7.364971in}{1.759500in}}%
\pgfpathlineto{\pgfqpoint{7.368143in}{1.759578in}}%
\pgfpathlineto{\pgfqpoint{7.371315in}{1.759737in}}%
\pgfpathlineto{\pgfqpoint{7.374487in}{1.759933in}}%
\pgfpathlineto{\pgfqpoint{7.377659in}{1.759850in}}%
\pgfpathlineto{\pgfqpoint{7.380831in}{1.759562in}}%
\pgfpathlineto{\pgfqpoint{7.384003in}{1.758781in}}%
\pgfpathlineto{\pgfqpoint{7.387175in}{1.758209in}}%
\pgfpathlineto{\pgfqpoint{7.390347in}{1.758252in}}%
\pgfpathlineto{\pgfqpoint{7.393519in}{1.757960in}}%
\pgfpathlineto{\pgfqpoint{7.396691in}{1.757719in}}%
\pgfpathlineto{\pgfqpoint{7.399863in}{1.757261in}}%
\pgfpathlineto{\pgfqpoint{7.403035in}{1.756840in}}%
\pgfpathlineto{\pgfqpoint{7.406207in}{1.756523in}}%
\pgfpathlineto{\pgfqpoint{7.409379in}{1.756368in}}%
\pgfpathlineto{\pgfqpoint{7.412551in}{1.756410in}}%
\pgfpathlineto{\pgfqpoint{7.415723in}{1.756131in}}%
\pgfpathlineto{\pgfqpoint{7.418895in}{1.756064in}}%
\pgfpathlineto{\pgfqpoint{7.422067in}{1.756105in}}%
\pgfpathlineto{\pgfqpoint{7.425239in}{1.756014in}}%
\pgfpathlineto{\pgfqpoint{7.428411in}{1.756011in}}%
\pgfpathlineto{\pgfqpoint{7.431584in}{1.755767in}}%
\pgfpathlineto{\pgfqpoint{7.434756in}{1.755864in}}%
\pgfpathlineto{\pgfqpoint{7.437928in}{1.755647in}}%
\pgfpathlineto{\pgfqpoint{7.441100in}{1.755481in}}%
\pgfpathlineto{\pgfqpoint{7.444272in}{1.755489in}}%
\pgfpathlineto{\pgfqpoint{7.447444in}{1.755458in}}%
\pgfpathlineto{\pgfqpoint{7.450616in}{1.755404in}}%
\pgfpathlineto{\pgfqpoint{7.453788in}{1.755360in}}%
\pgfpathlineto{\pgfqpoint{7.456960in}{1.754974in}}%
\pgfpathlineto{\pgfqpoint{7.460132in}{1.754414in}}%
\pgfpathlineto{\pgfqpoint{7.463304in}{1.754331in}}%
\pgfpathlineto{\pgfqpoint{7.466476in}{1.754385in}}%
\pgfpathlineto{\pgfqpoint{7.469648in}{1.754087in}}%
\pgfpathlineto{\pgfqpoint{7.472820in}{1.754010in}}%
\pgfpathlineto{\pgfqpoint{7.475992in}{1.753900in}}%
\pgfpathlineto{\pgfqpoint{7.479164in}{1.754011in}}%
\pgfpathlineto{\pgfqpoint{7.482336in}{1.753617in}}%
\pgfpathlineto{\pgfqpoint{7.485508in}{1.753484in}}%
\pgfpathlineto{\pgfqpoint{7.488680in}{1.753086in}}%
\pgfpathlineto{\pgfqpoint{7.491852in}{1.753195in}}%
\pgfpathlineto{\pgfqpoint{7.495024in}{1.753188in}}%
\pgfpathlineto{\pgfqpoint{7.498196in}{1.753324in}}%
\pgfpathlineto{\pgfqpoint{7.501368in}{1.753415in}}%
\pgfpathlineto{\pgfqpoint{7.504540in}{1.753475in}}%
\pgfpathlineto{\pgfqpoint{7.507712in}{1.753385in}}%
\pgfpathlineto{\pgfqpoint{7.510885in}{1.752745in}}%
\pgfpathlineto{\pgfqpoint{7.514057in}{1.752543in}}%
\pgfpathlineto{\pgfqpoint{7.517229in}{1.751913in}}%
\pgfpathlineto{\pgfqpoint{7.520401in}{1.751313in}}%
\pgfpathlineto{\pgfqpoint{7.523573in}{1.751341in}}%
\pgfpathlineto{\pgfqpoint{7.526745in}{1.750851in}}%
\pgfpathlineto{\pgfqpoint{7.529917in}{1.751091in}}%
\pgfpathlineto{\pgfqpoint{7.533089in}{1.750353in}}%
\pgfpathlineto{\pgfqpoint{7.536261in}{1.750157in}}%
\pgfpathlineto{\pgfqpoint{7.539433in}{1.750182in}}%
\pgfpathlineto{\pgfqpoint{7.542605in}{1.750320in}}%
\pgfpathlineto{\pgfqpoint{7.545777in}{1.750239in}}%
\pgfpathlineto{\pgfqpoint{7.548949in}{1.750083in}}%
\pgfpathlineto{\pgfqpoint{7.552121in}{1.750282in}}%
\pgfpathlineto{\pgfqpoint{7.555293in}{1.750674in}}%
\pgfpathlineto{\pgfqpoint{7.558465in}{1.750359in}}%
\pgfpathlineto{\pgfqpoint{7.561637in}{1.749782in}}%
\pgfpathlineto{\pgfqpoint{7.564809in}{1.750080in}}%
\pgfpathlineto{\pgfqpoint{7.567981in}{1.750133in}}%
\pgfpathlineto{\pgfqpoint{7.571153in}{1.749859in}}%
\pgfpathlineto{\pgfqpoint{7.574325in}{1.749911in}}%
\pgfpathlineto{\pgfqpoint{7.577497in}{1.750056in}}%
\pgfpathlineto{\pgfqpoint{7.580669in}{1.750102in}}%
\pgfpathlineto{\pgfqpoint{7.583841in}{1.749849in}}%
\pgfpathlineto{\pgfqpoint{7.587013in}{1.749544in}}%
\pgfpathlineto{\pgfqpoint{7.590186in}{1.749391in}}%
\pgfpathlineto{\pgfqpoint{7.593358in}{1.749197in}}%
\pgfpathlineto{\pgfqpoint{7.596530in}{1.749144in}}%
\pgfpathlineto{\pgfqpoint{7.599702in}{1.748736in}}%
\pgfpathlineto{\pgfqpoint{7.602874in}{1.748727in}}%
\pgfpathlineto{\pgfqpoint{7.606046in}{1.748959in}}%
\pgfpathlineto{\pgfqpoint{7.609218in}{1.748790in}}%
\pgfpathlineto{\pgfqpoint{7.612390in}{1.748565in}}%
\pgfpathlineto{\pgfqpoint{7.615562in}{1.747871in}}%
\pgfpathlineto{\pgfqpoint{7.618734in}{1.747142in}}%
\pgfpathlineto{\pgfqpoint{7.621906in}{1.746898in}}%
\pgfpathlineto{\pgfqpoint{7.625078in}{1.746974in}}%
\pgfpathlineto{\pgfqpoint{7.628250in}{1.747588in}}%
\pgfpathlineto{\pgfqpoint{7.631422in}{1.747452in}}%
\pgfpathlineto{\pgfqpoint{7.634594in}{1.747540in}}%
\pgfpathlineto{\pgfqpoint{7.637766in}{1.747368in}}%
\pgfpathlineto{\pgfqpoint{7.640938in}{1.747545in}}%
\pgfpathlineto{\pgfqpoint{7.644110in}{1.747317in}}%
\pgfpathlineto{\pgfqpoint{7.647282in}{1.747884in}}%
\pgfpathlineto{\pgfqpoint{7.650454in}{1.747764in}}%
\pgfpathlineto{\pgfqpoint{7.653626in}{1.747601in}}%
\pgfpathlineto{\pgfqpoint{7.656798in}{1.747576in}}%
\pgfpathlineto{\pgfqpoint{7.659970in}{1.747401in}}%
\pgfpathlineto{\pgfqpoint{7.663142in}{1.747226in}}%
\pgfpathlineto{\pgfqpoint{7.666315in}{1.747446in}}%
\pgfpathlineto{\pgfqpoint{7.669487in}{1.747692in}}%
\pgfpathlineto{\pgfqpoint{7.672659in}{1.747947in}}%
\pgfpathlineto{\pgfqpoint{7.675831in}{1.748403in}}%
\pgfpathlineto{\pgfqpoint{7.679003in}{1.748157in}}%
\pgfpathlineto{\pgfqpoint{7.682175in}{1.748237in}}%
\pgfpathlineto{\pgfqpoint{7.685347in}{1.748386in}}%
\pgfpathlineto{\pgfqpoint{7.688519in}{1.748302in}}%
\pgfpathlineto{\pgfqpoint{7.691691in}{1.748286in}}%
\pgfpathlineto{\pgfqpoint{7.694863in}{1.748270in}}%
\pgfpathlineto{\pgfqpoint{7.698035in}{1.748314in}}%
\pgfpathlineto{\pgfqpoint{7.701207in}{1.748296in}}%
\pgfpathlineto{\pgfqpoint{7.704379in}{1.748245in}}%
\pgfpathlineto{\pgfqpoint{7.707551in}{1.747877in}}%
\pgfpathlineto{\pgfqpoint{7.710723in}{1.747245in}}%
\pgfpathlineto{\pgfqpoint{7.713895in}{1.747184in}}%
\pgfpathlineto{\pgfqpoint{7.717067in}{1.747174in}}%
\pgfpathlineto{\pgfqpoint{7.720239in}{1.746819in}}%
\pgfpathlineto{\pgfqpoint{7.723411in}{1.746598in}}%
\pgfpathlineto{\pgfqpoint{7.726583in}{1.746604in}}%
\pgfpathlineto{\pgfqpoint{7.729755in}{1.746263in}}%
\pgfpathlineto{\pgfqpoint{7.732927in}{1.746086in}}%
\pgfpathlineto{\pgfqpoint{7.736099in}{1.745471in}}%
\pgfpathlineto{\pgfqpoint{7.739271in}{1.745120in}}%
\pgfpathlineto{\pgfqpoint{7.742443in}{1.744832in}}%
\pgfpathlineto{\pgfqpoint{7.745616in}{1.745361in}}%
\pgfpathlineto{\pgfqpoint{7.748788in}{1.745423in}}%
\pgfpathlineto{\pgfqpoint{7.751960in}{1.744293in}}%
\pgfpathlineto{\pgfqpoint{7.755132in}{1.744222in}}%
\pgfpathlineto{\pgfqpoint{7.758304in}{1.744399in}}%
\pgfpathlineto{\pgfqpoint{7.761476in}{1.744326in}}%
\pgfpathlineto{\pgfqpoint{7.764648in}{1.744252in}}%
\pgfpathlineto{\pgfqpoint{7.767820in}{1.744061in}}%
\pgfpathlineto{\pgfqpoint{7.770992in}{1.744400in}}%
\pgfpathlineto{\pgfqpoint{7.774164in}{1.744484in}}%
\pgfpathlineto{\pgfqpoint{7.777336in}{1.744312in}}%
\pgfpathlineto{\pgfqpoint{7.780508in}{1.744594in}}%
\pgfpathlineto{\pgfqpoint{7.783680in}{1.744552in}}%
\pgfpathlineto{\pgfqpoint{7.786852in}{1.744430in}}%
\pgfpathlineto{\pgfqpoint{7.790024in}{1.747519in}}%
\pgfpathlineto{\pgfqpoint{7.793196in}{1.750587in}}%
\pgfpathlineto{\pgfqpoint{7.796368in}{1.753674in}}%
\pgfpathlineto{\pgfqpoint{7.799540in}{1.756834in}}%
\pgfpathlineto{\pgfqpoint{7.802712in}{1.759913in}}%
\pgfpathlineto{\pgfqpoint{7.805884in}{1.763000in}}%
\pgfpathlineto{\pgfqpoint{7.809056in}{1.766116in}}%
\pgfpathlineto{\pgfqpoint{7.812228in}{1.769222in}}%
\pgfpathlineto{\pgfqpoint{7.815400in}{1.772326in}}%
\pgfpathlineto{\pgfqpoint{7.818572in}{1.775432in}}%
\pgfpathlineto{\pgfqpoint{7.821744in}{1.778513in}}%
\pgfpathlineto{\pgfqpoint{7.824917in}{1.781586in}}%
\pgfpathlineto{\pgfqpoint{7.828089in}{1.784697in}}%
\pgfpathlineto{\pgfqpoint{7.831261in}{1.787739in}}%
\pgfpathlineto{\pgfqpoint{7.834433in}{1.790743in}}%
\pgfpathlineto{\pgfqpoint{7.837605in}{1.793832in}}%
\pgfpathlineto{\pgfqpoint{7.840777in}{1.796931in}}%
\pgfpathlineto{\pgfqpoint{7.843949in}{1.800023in}}%
\pgfpathlineto{\pgfqpoint{7.847121in}{1.803063in}}%
\pgfpathlineto{\pgfqpoint{7.850293in}{1.806172in}}%
\pgfpathlineto{\pgfqpoint{7.853465in}{1.809209in}}%
\pgfpathlineto{\pgfqpoint{7.856637in}{1.812289in}}%
\pgfpathlineto{\pgfqpoint{7.859809in}{1.815382in}}%
\pgfpathlineto{\pgfqpoint{7.862981in}{1.818499in}}%
\pgfpathlineto{\pgfqpoint{7.866153in}{1.821594in}}%
\pgfpathlineto{\pgfqpoint{7.869325in}{1.824729in}}%
\pgfpathlineto{\pgfqpoint{7.872497in}{1.827779in}}%
\pgfpathlineto{\pgfqpoint{7.875669in}{1.830902in}}%
\pgfpathlineto{\pgfqpoint{7.878841in}{1.833989in}}%
\pgfpathlineto{\pgfqpoint{7.882013in}{1.837074in}}%
\pgfpathlineto{\pgfqpoint{7.885185in}{1.840055in}}%
\pgfpathlineto{\pgfqpoint{7.888357in}{1.843103in}}%
\pgfpathlineto{\pgfqpoint{7.891529in}{1.846182in}}%
\pgfpathlineto{\pgfqpoint{7.894701in}{1.849241in}}%
\pgfpathlineto{\pgfqpoint{7.897873in}{1.855590in}}%
\pgfpathlineto{\pgfqpoint{7.901046in}{1.858670in}}%
\pgfpathlineto{\pgfqpoint{7.904218in}{1.861731in}}%
\pgfpathlineto{\pgfqpoint{7.907390in}{1.864827in}}%
\pgfpathlineto{\pgfqpoint{7.910562in}{1.867901in}}%
\pgfpathlineto{\pgfqpoint{7.913734in}{1.870972in}}%
\pgfpathlineto{\pgfqpoint{7.916906in}{1.874059in}}%
\pgfpathlineto{\pgfqpoint{7.920078in}{1.877114in}}%
\pgfpathlineto{\pgfqpoint{7.923250in}{1.880168in}}%
\pgfpathlineto{\pgfqpoint{7.926422in}{1.883223in}}%
\pgfpathlineto{\pgfqpoint{7.929594in}{1.886321in}}%
\pgfpathlineto{\pgfqpoint{7.932766in}{1.889354in}}%
\pgfpathlineto{\pgfqpoint{7.935938in}{1.892420in}}%
\pgfpathlineto{\pgfqpoint{7.939110in}{1.895458in}}%
\pgfpathlineto{\pgfqpoint{7.942282in}{1.898515in}}%
\pgfpathlineto{\pgfqpoint{7.945454in}{1.901588in}}%
\pgfpathlineto{\pgfqpoint{7.948626in}{1.904665in}}%
\pgfpathlineto{\pgfqpoint{7.951798in}{1.907753in}}%
\pgfpathlineto{\pgfqpoint{7.954970in}{1.910823in}}%
\pgfpathlineto{\pgfqpoint{7.958142in}{1.913882in}}%
\pgfpathlineto{\pgfqpoint{7.961314in}{1.916939in}}%
\pgfpathlineto{\pgfqpoint{7.964486in}{1.920033in}}%
\pgfpathlineto{\pgfqpoint{7.967658in}{1.923110in}}%
\pgfpathlineto{\pgfqpoint{7.970830in}{1.926167in}}%
\pgfpathlineto{\pgfqpoint{7.974002in}{1.929293in}}%
\pgfpathlineto{\pgfqpoint{7.977174in}{1.932377in}}%
\pgfpathlineto{\pgfqpoint{7.980347in}{1.935511in}}%
\pgfpathlineto{\pgfqpoint{7.983519in}{1.938555in}}%
\pgfpathlineto{\pgfqpoint{7.986691in}{1.941613in}}%
\pgfpathlineto{\pgfqpoint{7.989863in}{1.944639in}}%
\pgfpathlineto{\pgfqpoint{7.993035in}{1.947680in}}%
\pgfpathlineto{\pgfqpoint{7.996207in}{1.950755in}}%
\pgfpathlineto{\pgfqpoint{7.999379in}{1.953797in}}%
\pgfpathlineto{\pgfqpoint{8.002551in}{1.956876in}}%
\pgfpathlineto{\pgfqpoint{8.005723in}{1.959945in}}%
\pgfpathlineto{\pgfqpoint{8.008895in}{1.963016in}}%
\pgfpathlineto{\pgfqpoint{8.012067in}{1.966070in}}%
\pgfpathlineto{\pgfqpoint{8.015239in}{1.969148in}}%
\pgfpathlineto{\pgfqpoint{8.018411in}{1.972141in}}%
\pgfpathlineto{\pgfqpoint{8.021583in}{1.975173in}}%
\pgfpathlineto{\pgfqpoint{8.024755in}{1.978232in}}%
\pgfpathlineto{\pgfqpoint{8.027927in}{1.981313in}}%
\pgfpathlineto{\pgfqpoint{8.031099in}{1.984343in}}%
\pgfpathlineto{\pgfqpoint{8.034271in}{1.987388in}}%
\pgfpathlineto{\pgfqpoint{8.037443in}{1.990476in}}%
\pgfpathlineto{\pgfqpoint{8.040615in}{1.993511in}}%
\pgfpathlineto{\pgfqpoint{8.043787in}{1.996599in}}%
\pgfpathlineto{\pgfqpoint{8.046959in}{1.999667in}}%
\pgfpathlineto{\pgfqpoint{8.050131in}{2.002739in}}%
\pgfpathlineto{\pgfqpoint{8.053303in}{2.005808in}}%
\pgfpathlineto{\pgfqpoint{8.056475in}{2.008886in}}%
\pgfpathlineto{\pgfqpoint{8.059648in}{2.011980in}}%
\pgfpathlineto{\pgfqpoint{8.062820in}{2.015045in}}%
\pgfpathlineto{\pgfqpoint{8.065992in}{2.018169in}}%
\pgfpathlineto{\pgfqpoint{8.069164in}{2.021256in}}%
\pgfpathlineto{\pgfqpoint{8.072336in}{2.024289in}}%
\pgfpathlineto{\pgfqpoint{8.075508in}{2.027332in}}%
\pgfpathlineto{\pgfqpoint{8.078680in}{2.030403in}}%
\pgfpathlineto{\pgfqpoint{8.081852in}{2.033482in}}%
\pgfpathlineto{\pgfqpoint{8.085024in}{2.036565in}}%
\pgfpathlineto{\pgfqpoint{8.088196in}{2.039695in}}%
\pgfpathlineto{\pgfqpoint{8.091368in}{2.042778in}}%
\pgfpathlineto{\pgfqpoint{8.094540in}{2.045869in}}%
\pgfpathlineto{\pgfqpoint{8.097712in}{2.048940in}}%
\pgfpathlineto{\pgfqpoint{8.100884in}{2.052032in}}%
\pgfpathlineto{\pgfqpoint{8.104056in}{2.055055in}}%
\pgfpathlineto{\pgfqpoint{8.107228in}{2.058149in}}%
\pgfpathlineto{\pgfqpoint{8.110400in}{2.061229in}}%
\pgfpathlineto{\pgfqpoint{8.113572in}{2.064293in}}%
\pgfpathlineto{\pgfqpoint{8.116744in}{2.067391in}}%
\pgfpathlineto{\pgfqpoint{8.119916in}{2.070489in}}%
\pgfpathlineto{\pgfqpoint{8.123088in}{2.073590in}}%
\pgfpathlineto{\pgfqpoint{8.126260in}{2.076655in}}%
\pgfpathlineto{\pgfqpoint{8.129432in}{2.079677in}}%
\pgfpathlineto{\pgfqpoint{8.132604in}{2.082795in}}%
\pgfpathlineto{\pgfqpoint{8.135777in}{2.085889in}}%
\pgfpathlineto{\pgfqpoint{8.138949in}{2.089009in}}%
\pgfpathlineto{\pgfqpoint{8.142121in}{2.092070in}}%
\pgfpathlineto{\pgfqpoint{8.145293in}{2.095140in}}%
\pgfpathlineto{\pgfqpoint{8.148465in}{2.098214in}}%
\pgfpathlineto{\pgfqpoint{8.151637in}{2.101271in}}%
\pgfpathlineto{\pgfqpoint{8.154809in}{2.104332in}}%
\pgfpathlineto{\pgfqpoint{8.157981in}{2.107414in}}%
\pgfpathlineto{\pgfqpoint{8.161153in}{2.110487in}}%
\pgfpathlineto{\pgfqpoint{8.164325in}{2.113547in}}%
\pgfpathlineto{\pgfqpoint{8.167497in}{2.116615in}}%
\pgfpathlineto{\pgfqpoint{8.170669in}{2.119661in}}%
\pgfpathlineto{\pgfqpoint{8.173841in}{2.123112in}}%
\pgfpathlineto{\pgfqpoint{8.177013in}{2.126220in}}%
\pgfpathlineto{\pgfqpoint{8.180185in}{2.129367in}}%
\pgfpathlineto{\pgfqpoint{8.183357in}{2.132517in}}%
\pgfpathlineto{\pgfqpoint{8.186529in}{2.135579in}}%
\pgfpathlineto{\pgfqpoint{8.189701in}{2.138686in}}%
\pgfpathlineto{\pgfqpoint{8.192873in}{2.141760in}}%
\pgfpathlineto{\pgfqpoint{8.196045in}{2.144846in}}%
\pgfpathlineto{\pgfqpoint{8.199217in}{2.147932in}}%
\pgfpathlineto{\pgfqpoint{8.202389in}{2.151052in}}%
\pgfpathlineto{\pgfqpoint{8.205561in}{2.154119in}}%
\pgfpathlineto{\pgfqpoint{8.208733in}{2.157199in}}%
\pgfpathlineto{\pgfqpoint{8.211905in}{2.160285in}}%
\pgfpathlineto{\pgfqpoint{8.215078in}{2.163342in}}%
\pgfpathlineto{\pgfqpoint{8.218250in}{2.166401in}}%
\pgfpathlineto{\pgfqpoint{8.221422in}{2.169444in}}%
\pgfpathlineto{\pgfqpoint{8.224594in}{2.172543in}}%
\pgfpathlineto{\pgfqpoint{8.227766in}{2.175674in}}%
\pgfpathlineto{\pgfqpoint{8.230938in}{2.178775in}}%
\pgfpathlineto{\pgfqpoint{8.234110in}{2.181848in}}%
\pgfpathlineto{\pgfqpoint{8.237282in}{2.184941in}}%
\pgfpathlineto{\pgfqpoint{8.240454in}{2.188033in}}%
\pgfpathlineto{\pgfqpoint{8.243626in}{2.191133in}}%
\pgfpathlineto{\pgfqpoint{8.246798in}{2.194246in}}%
\pgfpathlineto{\pgfqpoint{8.249970in}{2.197305in}}%
\pgfpathlineto{\pgfqpoint{8.253142in}{2.200380in}}%
\pgfpathlineto{\pgfqpoint{8.256314in}{2.203434in}}%
\pgfpathlineto{\pgfqpoint{8.259486in}{2.206515in}}%
\pgfpathlineto{\pgfqpoint{8.262658in}{2.209578in}}%
\pgfpathlineto{\pgfqpoint{8.265830in}{2.212652in}}%
\pgfpathlineto{\pgfqpoint{8.269002in}{2.215718in}}%
\pgfpathlineto{\pgfqpoint{8.272174in}{2.218830in}}%
\pgfpathlineto{\pgfqpoint{8.275346in}{2.221854in}}%
\pgfpathlineto{\pgfqpoint{8.278518in}{2.224920in}}%
\pgfpathlineto{\pgfqpoint{8.281690in}{2.227947in}}%
\pgfpathlineto{\pgfqpoint{8.281690in}{2.538146in}}%
\pgfpathlineto{\pgfqpoint{8.281690in}{2.538146in}}%
\pgfpathlineto{\pgfqpoint{8.278518in}{2.535004in}}%
\pgfpathlineto{\pgfqpoint{8.275346in}{2.531838in}}%
\pgfpathlineto{\pgfqpoint{8.272174in}{2.528784in}}%
\pgfpathlineto{\pgfqpoint{8.269002in}{2.525725in}}%
\pgfpathlineto{\pgfqpoint{8.265830in}{2.522667in}}%
\pgfpathlineto{\pgfqpoint{8.262658in}{2.519536in}}%
\pgfpathlineto{\pgfqpoint{8.259486in}{2.516407in}}%
\pgfpathlineto{\pgfqpoint{8.256314in}{2.513318in}}%
\pgfpathlineto{\pgfqpoint{8.253142in}{2.510358in}}%
\pgfpathlineto{\pgfqpoint{8.249970in}{2.507318in}}%
\pgfpathlineto{\pgfqpoint{8.246798in}{2.504227in}}%
\pgfpathlineto{\pgfqpoint{8.243626in}{2.501139in}}%
\pgfpathlineto{\pgfqpoint{8.240454in}{2.498117in}}%
\pgfpathlineto{\pgfqpoint{8.237282in}{2.494997in}}%
\pgfpathlineto{\pgfqpoint{8.234110in}{2.491945in}}%
\pgfpathlineto{\pgfqpoint{8.230938in}{2.488839in}}%
\pgfpathlineto{\pgfqpoint{8.227766in}{2.485882in}}%
\pgfpathlineto{\pgfqpoint{8.224594in}{2.482709in}}%
\pgfpathlineto{\pgfqpoint{8.221422in}{2.479609in}}%
\pgfpathlineto{\pgfqpoint{8.218250in}{2.476522in}}%
\pgfpathlineto{\pgfqpoint{8.215078in}{2.473443in}}%
\pgfpathlineto{\pgfqpoint{8.211905in}{2.470276in}}%
\pgfpathlineto{\pgfqpoint{8.208733in}{2.467180in}}%
\pgfpathlineto{\pgfqpoint{8.205561in}{2.464090in}}%
\pgfpathlineto{\pgfqpoint{8.202389in}{2.461029in}}%
\pgfpathlineto{\pgfqpoint{8.199217in}{2.457860in}}%
\pgfpathlineto{\pgfqpoint{8.196045in}{2.454724in}}%
\pgfpathlineto{\pgfqpoint{8.192873in}{2.451659in}}%
\pgfpathlineto{\pgfqpoint{8.189701in}{2.448561in}}%
\pgfpathlineto{\pgfqpoint{8.186529in}{2.445481in}}%
\pgfpathlineto{\pgfqpoint{8.183357in}{2.442377in}}%
\pgfpathlineto{\pgfqpoint{8.180185in}{2.439257in}}%
\pgfpathlineto{\pgfqpoint{8.177013in}{2.436226in}}%
\pgfpathlineto{\pgfqpoint{8.173841in}{2.433114in}}%
\pgfpathlineto{\pgfqpoint{8.170669in}{2.430068in}}%
\pgfpathlineto{\pgfqpoint{8.167497in}{2.426972in}}%
\pgfpathlineto{\pgfqpoint{8.164325in}{2.423625in}}%
\pgfpathlineto{\pgfqpoint{8.161153in}{2.420124in}}%
\pgfpathlineto{\pgfqpoint{8.157981in}{2.417041in}}%
\pgfpathlineto{\pgfqpoint{8.154809in}{2.413947in}}%
\pgfpathlineto{\pgfqpoint{8.151637in}{2.410831in}}%
\pgfpathlineto{\pgfqpoint{8.148465in}{2.407712in}}%
\pgfpathlineto{\pgfqpoint{8.145293in}{2.404538in}}%
\pgfpathlineto{\pgfqpoint{8.142121in}{2.401424in}}%
\pgfpathlineto{\pgfqpoint{8.138949in}{2.398272in}}%
\pgfpathlineto{\pgfqpoint{8.135777in}{2.395139in}}%
\pgfpathlineto{\pgfqpoint{8.132604in}{2.391986in}}%
\pgfpathlineto{\pgfqpoint{8.129432in}{2.388873in}}%
\pgfpathlineto{\pgfqpoint{8.126260in}{2.385931in}}%
\pgfpathlineto{\pgfqpoint{8.123088in}{2.382821in}}%
\pgfpathlineto{\pgfqpoint{8.119916in}{2.379723in}}%
\pgfpathlineto{\pgfqpoint{8.116744in}{2.376665in}}%
\pgfpathlineto{\pgfqpoint{8.113572in}{2.373558in}}%
\pgfpathlineto{\pgfqpoint{8.110400in}{2.370451in}}%
\pgfpathlineto{\pgfqpoint{8.107228in}{2.367353in}}%
\pgfpathlineto{\pgfqpoint{8.104056in}{2.364298in}}%
\pgfpathlineto{\pgfqpoint{8.100884in}{2.361206in}}%
\pgfpathlineto{\pgfqpoint{8.097712in}{2.357958in}}%
\pgfpathlineto{\pgfqpoint{8.094540in}{2.354835in}}%
\pgfpathlineto{\pgfqpoint{8.091368in}{2.351719in}}%
\pgfpathlineto{\pgfqpoint{8.088196in}{2.348669in}}%
\pgfpathlineto{\pgfqpoint{8.085024in}{2.345531in}}%
\pgfpathlineto{\pgfqpoint{8.081852in}{2.342410in}}%
\pgfpathlineto{\pgfqpoint{8.078680in}{2.339233in}}%
\pgfpathlineto{\pgfqpoint{8.075508in}{2.336113in}}%
\pgfpathlineto{\pgfqpoint{8.072336in}{2.333041in}}%
\pgfpathlineto{\pgfqpoint{8.069164in}{2.329924in}}%
\pgfpathlineto{\pgfqpoint{8.065992in}{2.326804in}}%
\pgfpathlineto{\pgfqpoint{8.062820in}{2.323688in}}%
\pgfpathlineto{\pgfqpoint{8.059648in}{2.320593in}}%
\pgfpathlineto{\pgfqpoint{8.056475in}{2.317561in}}%
\pgfpathlineto{\pgfqpoint{8.053303in}{2.314503in}}%
\pgfpathlineto{\pgfqpoint{8.050131in}{2.311442in}}%
\pgfpathlineto{\pgfqpoint{8.046959in}{2.308254in}}%
\pgfpathlineto{\pgfqpoint{8.043787in}{2.305201in}}%
\pgfpathlineto{\pgfqpoint{8.040615in}{2.302115in}}%
\pgfpathlineto{\pgfqpoint{8.037443in}{2.298982in}}%
\pgfpathlineto{\pgfqpoint{8.034271in}{2.295895in}}%
\pgfpathlineto{\pgfqpoint{8.031099in}{2.292768in}}%
\pgfpathlineto{\pgfqpoint{8.027927in}{2.289673in}}%
\pgfpathlineto{\pgfqpoint{8.024755in}{2.286638in}}%
\pgfpathlineto{\pgfqpoint{8.021583in}{2.283529in}}%
\pgfpathlineto{\pgfqpoint{8.018411in}{2.280449in}}%
\pgfpathlineto{\pgfqpoint{8.015239in}{2.277355in}}%
\pgfpathlineto{\pgfqpoint{8.012067in}{2.274301in}}%
\pgfpathlineto{\pgfqpoint{8.008895in}{2.271201in}}%
\pgfpathlineto{\pgfqpoint{8.005723in}{2.268128in}}%
\pgfpathlineto{\pgfqpoint{8.002551in}{2.265053in}}%
\pgfpathlineto{\pgfqpoint{7.999379in}{2.262198in}}%
\pgfpathlineto{\pgfqpoint{7.996207in}{2.259475in}}%
\pgfpathlineto{\pgfqpoint{7.993035in}{2.256372in}}%
\pgfpathlineto{\pgfqpoint{7.989863in}{2.253271in}}%
\pgfpathlineto{\pgfqpoint{7.986691in}{2.250188in}}%
\pgfpathlineto{\pgfqpoint{7.983519in}{2.247100in}}%
\pgfpathlineto{\pgfqpoint{7.980347in}{2.244030in}}%
\pgfpathlineto{\pgfqpoint{7.977174in}{2.240949in}}%
\pgfpathlineto{\pgfqpoint{7.974002in}{2.237836in}}%
\pgfpathlineto{\pgfqpoint{7.970830in}{2.234759in}}%
\pgfpathlineto{\pgfqpoint{7.967658in}{2.231696in}}%
\pgfpathlineto{\pgfqpoint{7.964486in}{2.228551in}}%
\pgfpathlineto{\pgfqpoint{7.961314in}{2.225442in}}%
\pgfpathlineto{\pgfqpoint{7.958142in}{2.222347in}}%
\pgfpathlineto{\pgfqpoint{7.954970in}{2.219335in}}%
\pgfpathlineto{\pgfqpoint{7.951798in}{2.216304in}}%
\pgfpathlineto{\pgfqpoint{7.948626in}{2.213194in}}%
\pgfpathlineto{\pgfqpoint{7.945454in}{2.210097in}}%
\pgfpathlineto{\pgfqpoint{7.942282in}{2.206971in}}%
\pgfpathlineto{\pgfqpoint{7.939110in}{2.203913in}}%
\pgfpathlineto{\pgfqpoint{7.935938in}{2.200805in}}%
\pgfpathlineto{\pgfqpoint{7.932766in}{2.197693in}}%
\pgfpathlineto{\pgfqpoint{7.929594in}{2.194599in}}%
\pgfpathlineto{\pgfqpoint{7.926422in}{2.191561in}}%
\pgfpathlineto{\pgfqpoint{7.923250in}{2.188490in}}%
\pgfpathlineto{\pgfqpoint{7.920078in}{2.185354in}}%
\pgfpathlineto{\pgfqpoint{7.916906in}{2.182260in}}%
\pgfpathlineto{\pgfqpoint{7.913734in}{2.179168in}}%
\pgfpathlineto{\pgfqpoint{7.910562in}{2.176124in}}%
\pgfpathlineto{\pgfqpoint{7.907390in}{2.173037in}}%
\pgfpathlineto{\pgfqpoint{7.904218in}{2.170042in}}%
\pgfpathlineto{\pgfqpoint{7.901046in}{2.166992in}}%
\pgfpathlineto{\pgfqpoint{7.897873in}{2.163940in}}%
\pgfpathlineto{\pgfqpoint{7.894701in}{2.158272in}}%
\pgfpathlineto{\pgfqpoint{7.891529in}{2.155200in}}%
\pgfpathlineto{\pgfqpoint{7.888357in}{2.152247in}}%
\pgfpathlineto{\pgfqpoint{7.885185in}{2.149149in}}%
\pgfpathlineto{\pgfqpoint{7.882013in}{2.146119in}}%
\pgfpathlineto{\pgfqpoint{7.878841in}{2.143071in}}%
\pgfpathlineto{\pgfqpoint{7.875669in}{2.139918in}}%
\pgfpathlineto{\pgfqpoint{7.872497in}{2.136832in}}%
\pgfpathlineto{\pgfqpoint{7.869325in}{2.133769in}}%
\pgfpathlineto{\pgfqpoint{7.866153in}{2.130622in}}%
\pgfpathlineto{\pgfqpoint{7.862981in}{2.127364in}}%
\pgfpathlineto{\pgfqpoint{7.859809in}{2.124228in}}%
\pgfpathlineto{\pgfqpoint{7.856637in}{2.121182in}}%
\pgfpathlineto{\pgfqpoint{7.853465in}{2.118115in}}%
\pgfpathlineto{\pgfqpoint{7.850293in}{2.115018in}}%
\pgfpathlineto{\pgfqpoint{7.847121in}{2.111980in}}%
\pgfpathlineto{\pgfqpoint{7.843949in}{2.108877in}}%
\pgfpathlineto{\pgfqpoint{7.840777in}{2.105810in}}%
\pgfpathlineto{\pgfqpoint{7.837605in}{2.102778in}}%
\pgfpathlineto{\pgfqpoint{7.834433in}{2.099668in}}%
\pgfpathlineto{\pgfqpoint{7.831261in}{2.096475in}}%
\pgfpathlineto{\pgfqpoint{7.828089in}{2.093480in}}%
\pgfpathlineto{\pgfqpoint{7.824917in}{2.090583in}}%
\pgfpathlineto{\pgfqpoint{7.821744in}{2.087430in}}%
\pgfpathlineto{\pgfqpoint{7.818572in}{2.084294in}}%
\pgfpathlineto{\pgfqpoint{7.815400in}{2.081204in}}%
\pgfpathlineto{\pgfqpoint{7.812228in}{2.078117in}}%
\pgfpathlineto{\pgfqpoint{7.809056in}{2.075052in}}%
\pgfpathlineto{\pgfqpoint{7.805884in}{2.072025in}}%
\pgfpathlineto{\pgfqpoint{7.802712in}{2.068881in}}%
\pgfpathlineto{\pgfqpoint{7.799540in}{2.065921in}}%
\pgfpathlineto{\pgfqpoint{7.796368in}{2.062862in}}%
\pgfpathlineto{\pgfqpoint{7.793196in}{2.059736in}}%
\pgfpathlineto{\pgfqpoint{7.790024in}{2.056549in}}%
\pgfpathlineto{\pgfqpoint{7.786852in}{2.053660in}}%
\pgfpathlineto{\pgfqpoint{7.783680in}{2.053610in}}%
\pgfpathlineto{\pgfqpoint{7.780508in}{2.053543in}}%
\pgfpathlineto{\pgfqpoint{7.777336in}{2.053393in}}%
\pgfpathlineto{\pgfqpoint{7.774164in}{2.053376in}}%
\pgfpathlineto{\pgfqpoint{7.770992in}{2.052908in}}%
\pgfpathlineto{\pgfqpoint{7.767820in}{2.053098in}}%
\pgfpathlineto{\pgfqpoint{7.764648in}{2.052572in}}%
\pgfpathlineto{\pgfqpoint{7.761476in}{2.052287in}}%
\pgfpathlineto{\pgfqpoint{7.758304in}{2.052186in}}%
\pgfpathlineto{\pgfqpoint{7.755132in}{2.051902in}}%
\pgfpathlineto{\pgfqpoint{7.751960in}{2.051835in}}%
\pgfpathlineto{\pgfqpoint{7.748788in}{2.051843in}}%
\pgfpathlineto{\pgfqpoint{7.745616in}{2.052062in}}%
\pgfpathlineto{\pgfqpoint{7.742443in}{2.052550in}}%
\pgfpathlineto{\pgfqpoint{7.739271in}{2.053122in}}%
\pgfpathlineto{\pgfqpoint{7.736099in}{2.053050in}}%
\pgfpathlineto{\pgfqpoint{7.732927in}{2.052660in}}%
\pgfpathlineto{\pgfqpoint{7.729755in}{2.052683in}}%
\pgfpathlineto{\pgfqpoint{7.726583in}{2.052564in}}%
\pgfpathlineto{\pgfqpoint{7.723411in}{2.052613in}}%
\pgfpathlineto{\pgfqpoint{7.720239in}{2.052344in}}%
\pgfpathlineto{\pgfqpoint{7.717067in}{2.052266in}}%
\pgfpathlineto{\pgfqpoint{7.713895in}{2.052246in}}%
\pgfpathlineto{\pgfqpoint{7.710723in}{2.051996in}}%
\pgfpathlineto{\pgfqpoint{7.707551in}{2.052183in}}%
\pgfpathlineto{\pgfqpoint{7.704379in}{2.052602in}}%
\pgfpathlineto{\pgfqpoint{7.701207in}{2.052763in}}%
\pgfpathlineto{\pgfqpoint{7.698035in}{2.053083in}}%
\pgfpathlineto{\pgfqpoint{7.694863in}{2.053071in}}%
\pgfpathlineto{\pgfqpoint{7.691691in}{2.052698in}}%
\pgfpathlineto{\pgfqpoint{7.688519in}{2.052815in}}%
\pgfpathlineto{\pgfqpoint{7.685347in}{2.052755in}}%
\pgfpathlineto{\pgfqpoint{7.682175in}{2.052339in}}%
\pgfpathlineto{\pgfqpoint{7.679003in}{2.052317in}}%
\pgfpathlineto{\pgfqpoint{7.675831in}{2.052365in}}%
\pgfpathlineto{\pgfqpoint{7.672659in}{2.052185in}}%
\pgfpathlineto{\pgfqpoint{7.669487in}{2.052204in}}%
\pgfpathlineto{\pgfqpoint{7.666315in}{2.052310in}}%
\pgfpathlineto{\pgfqpoint{7.663142in}{2.052027in}}%
\pgfpathlineto{\pgfqpoint{7.659970in}{2.052057in}}%
\pgfpathlineto{\pgfqpoint{7.656798in}{2.052297in}}%
\pgfpathlineto{\pgfqpoint{7.653626in}{2.051582in}}%
\pgfpathlineto{\pgfqpoint{7.650454in}{2.051239in}}%
\pgfpathlineto{\pgfqpoint{7.647282in}{2.051053in}}%
\pgfpathlineto{\pgfqpoint{7.644110in}{2.051082in}}%
\pgfpathlineto{\pgfqpoint{7.640938in}{2.051131in}}%
\pgfpathlineto{\pgfqpoint{7.637766in}{2.051553in}}%
\pgfpathlineto{\pgfqpoint{7.634594in}{2.051472in}}%
\pgfpathlineto{\pgfqpoint{7.631422in}{2.051671in}}%
\pgfpathlineto{\pgfqpoint{7.628250in}{2.051715in}}%
\pgfpathlineto{\pgfqpoint{7.625078in}{2.051866in}}%
\pgfpathlineto{\pgfqpoint{7.621906in}{2.052021in}}%
\pgfpathlineto{\pgfqpoint{7.618734in}{2.051674in}}%
\pgfpathlineto{\pgfqpoint{7.615562in}{2.051411in}}%
\pgfpathlineto{\pgfqpoint{7.612390in}{2.051101in}}%
\pgfpathlineto{\pgfqpoint{7.609218in}{2.050909in}}%
\pgfpathlineto{\pgfqpoint{7.606046in}{2.050905in}}%
\pgfpathlineto{\pgfqpoint{7.602874in}{2.050754in}}%
\pgfpathlineto{\pgfqpoint{7.599702in}{2.050652in}}%
\pgfpathlineto{\pgfqpoint{7.596530in}{2.050734in}}%
\pgfpathlineto{\pgfqpoint{7.593358in}{2.051035in}}%
\pgfpathlineto{\pgfqpoint{7.590186in}{2.051451in}}%
\pgfpathlineto{\pgfqpoint{7.587013in}{2.051037in}}%
\pgfpathlineto{\pgfqpoint{7.583841in}{2.050615in}}%
\pgfpathlineto{\pgfqpoint{7.580669in}{2.050885in}}%
\pgfpathlineto{\pgfqpoint{7.577497in}{2.050825in}}%
\pgfpathlineto{\pgfqpoint{7.574325in}{2.050561in}}%
\pgfpathlineto{\pgfqpoint{7.571153in}{2.050298in}}%
\pgfpathlineto{\pgfqpoint{7.567981in}{2.050431in}}%
\pgfpathlineto{\pgfqpoint{7.564809in}{2.050548in}}%
\pgfpathlineto{\pgfqpoint{7.561637in}{2.050934in}}%
\pgfpathlineto{\pgfqpoint{7.558465in}{2.050560in}}%
\pgfpathlineto{\pgfqpoint{7.555293in}{2.050419in}}%
\pgfpathlineto{\pgfqpoint{7.552121in}{2.050220in}}%
\pgfpathlineto{\pgfqpoint{7.548949in}{2.050186in}}%
\pgfpathlineto{\pgfqpoint{7.545777in}{2.049746in}}%
\pgfpathlineto{\pgfqpoint{7.542605in}{2.049818in}}%
\pgfpathlineto{\pgfqpoint{7.539433in}{2.049871in}}%
\pgfpathlineto{\pgfqpoint{7.536261in}{2.049684in}}%
\pgfpathlineto{\pgfqpoint{7.533089in}{2.049638in}}%
\pgfpathlineto{\pgfqpoint{7.529917in}{2.049887in}}%
\pgfpathlineto{\pgfqpoint{7.526745in}{2.050233in}}%
\pgfpathlineto{\pgfqpoint{7.523573in}{2.050152in}}%
\pgfpathlineto{\pgfqpoint{7.520401in}{2.050045in}}%
\pgfpathlineto{\pgfqpoint{7.517229in}{2.049660in}}%
\pgfpathlineto{\pgfqpoint{7.514057in}{2.049587in}}%
\pgfpathlineto{\pgfqpoint{7.510885in}{2.049558in}}%
\pgfpathlineto{\pgfqpoint{7.507712in}{2.049144in}}%
\pgfpathlineto{\pgfqpoint{7.504540in}{2.049087in}}%
\pgfpathlineto{\pgfqpoint{7.501368in}{2.049100in}}%
\pgfpathlineto{\pgfqpoint{7.498196in}{2.049392in}}%
\pgfpathlineto{\pgfqpoint{7.495024in}{2.049401in}}%
\pgfpathlineto{\pgfqpoint{7.491852in}{2.049260in}}%
\pgfpathlineto{\pgfqpoint{7.488680in}{2.049566in}}%
\pgfpathlineto{\pgfqpoint{7.485508in}{2.049519in}}%
\pgfpathlineto{\pgfqpoint{7.482336in}{2.049870in}}%
\pgfpathlineto{\pgfqpoint{7.479164in}{2.049874in}}%
\pgfpathlineto{\pgfqpoint{7.475992in}{2.050030in}}%
\pgfpathlineto{\pgfqpoint{7.472820in}{2.049968in}}%
\pgfpathlineto{\pgfqpoint{7.469648in}{2.049883in}}%
\pgfpathlineto{\pgfqpoint{7.466476in}{2.049489in}}%
\pgfpathlineto{\pgfqpoint{7.463304in}{2.049423in}}%
\pgfpathlineto{\pgfqpoint{7.460132in}{2.049263in}}%
\pgfpathlineto{\pgfqpoint{7.456960in}{2.049010in}}%
\pgfpathlineto{\pgfqpoint{7.453788in}{2.049404in}}%
\pgfpathlineto{\pgfqpoint{7.450616in}{2.049226in}}%
\pgfpathlineto{\pgfqpoint{7.447444in}{2.049320in}}%
\pgfpathlineto{\pgfqpoint{7.444272in}{2.049084in}}%
\pgfpathlineto{\pgfqpoint{7.441100in}{2.048644in}}%
\pgfpathlineto{\pgfqpoint{7.437928in}{2.048620in}}%
\pgfpathlineto{\pgfqpoint{7.434756in}{2.048434in}}%
\pgfpathlineto{\pgfqpoint{7.431584in}{2.048149in}}%
\pgfpathlineto{\pgfqpoint{7.428411in}{2.047548in}}%
\pgfpathlineto{\pgfqpoint{7.425239in}{2.047332in}}%
\pgfpathlineto{\pgfqpoint{7.422067in}{2.047208in}}%
\pgfpathlineto{\pgfqpoint{7.418895in}{2.047254in}}%
\pgfpathlineto{\pgfqpoint{7.415723in}{2.047079in}}%
\pgfpathlineto{\pgfqpoint{7.412551in}{2.046682in}}%
\pgfpathlineto{\pgfqpoint{7.409379in}{2.046737in}}%
\pgfpathlineto{\pgfqpoint{7.406207in}{2.047040in}}%
\pgfpathlineto{\pgfqpoint{7.403035in}{2.047061in}}%
\pgfpathlineto{\pgfqpoint{7.399863in}{2.046606in}}%
\pgfpathlineto{\pgfqpoint{7.396691in}{2.046332in}}%
\pgfpathlineto{\pgfqpoint{7.393519in}{2.046151in}}%
\pgfpathlineto{\pgfqpoint{7.390347in}{2.045971in}}%
\pgfpathlineto{\pgfqpoint{7.387175in}{2.045630in}}%
\pgfpathlineto{\pgfqpoint{7.384003in}{2.045066in}}%
\pgfpathlineto{\pgfqpoint{7.380831in}{2.045036in}}%
\pgfpathlineto{\pgfqpoint{7.377659in}{2.044152in}}%
\pgfpathlineto{\pgfqpoint{7.374487in}{2.044181in}}%
\pgfpathlineto{\pgfqpoint{7.371315in}{2.044773in}}%
\pgfpathlineto{\pgfqpoint{7.368143in}{2.044678in}}%
\pgfpathlineto{\pgfqpoint{7.364971in}{2.044688in}}%
\pgfpathlineto{\pgfqpoint{7.361799in}{2.044772in}}%
\pgfpathlineto{\pgfqpoint{7.358627in}{2.044491in}}%
\pgfpathlineto{\pgfqpoint{7.355455in}{2.044176in}}%
\pgfpathlineto{\pgfqpoint{7.352282in}{2.044261in}}%
\pgfpathlineto{\pgfqpoint{7.349110in}{2.044370in}}%
\pgfpathlineto{\pgfqpoint{7.345938in}{2.044073in}}%
\pgfpathlineto{\pgfqpoint{7.342766in}{2.044128in}}%
\pgfpathlineto{\pgfqpoint{7.339594in}{2.044181in}}%
\pgfpathlineto{\pgfqpoint{7.336422in}{2.044256in}}%
\pgfpathlineto{\pgfqpoint{7.333250in}{2.044518in}}%
\pgfpathlineto{\pgfqpoint{7.330078in}{2.044263in}}%
\pgfpathlineto{\pgfqpoint{7.326906in}{2.044301in}}%
\pgfpathlineto{\pgfqpoint{7.323734in}{2.044627in}}%
\pgfpathlineto{\pgfqpoint{7.320562in}{2.044422in}}%
\pgfpathlineto{\pgfqpoint{7.317390in}{2.044585in}}%
\pgfpathlineto{\pgfqpoint{7.314218in}{2.044826in}}%
\pgfpathlineto{\pgfqpoint{7.311046in}{2.045253in}}%
\pgfpathlineto{\pgfqpoint{7.307874in}{2.045984in}}%
\pgfpathlineto{\pgfqpoint{7.304702in}{2.046433in}}%
\pgfpathlineto{\pgfqpoint{7.301530in}{2.046381in}}%
\pgfpathlineto{\pgfqpoint{7.298358in}{2.046225in}}%
\pgfpathlineto{\pgfqpoint{7.295186in}{2.046173in}}%
\pgfpathlineto{\pgfqpoint{7.292014in}{2.045908in}}%
\pgfpathlineto{\pgfqpoint{7.288842in}{2.045762in}}%
\pgfpathlineto{\pgfqpoint{7.285670in}{2.046307in}}%
\pgfpathlineto{\pgfqpoint{7.282498in}{2.045906in}}%
\pgfpathlineto{\pgfqpoint{7.279326in}{2.045971in}}%
\pgfpathlineto{\pgfqpoint{7.276154in}{2.046007in}}%
\pgfpathlineto{\pgfqpoint{7.272981in}{2.046000in}}%
\pgfpathlineto{\pgfqpoint{7.269809in}{2.045886in}}%
\pgfpathlineto{\pgfqpoint{7.266637in}{2.045649in}}%
\pgfpathlineto{\pgfqpoint{7.263465in}{2.045741in}}%
\pgfpathlineto{\pgfqpoint{7.260293in}{2.045905in}}%
\pgfpathlineto{\pgfqpoint{7.257121in}{2.045852in}}%
\pgfpathlineto{\pgfqpoint{7.253949in}{2.046001in}}%
\pgfpathlineto{\pgfqpoint{7.250777in}{2.045949in}}%
\pgfpathlineto{\pgfqpoint{7.247605in}{2.046042in}}%
\pgfpathlineto{\pgfqpoint{7.244433in}{2.046321in}}%
\pgfpathlineto{\pgfqpoint{7.241261in}{2.045945in}}%
\pgfpathlineto{\pgfqpoint{7.238089in}{2.045861in}}%
\pgfpathlineto{\pgfqpoint{7.234917in}{2.045905in}}%
\pgfpathlineto{\pgfqpoint{7.231745in}{2.045987in}}%
\pgfpathlineto{\pgfqpoint{7.228573in}{2.045644in}}%
\pgfpathlineto{\pgfqpoint{7.225401in}{2.045158in}}%
\pgfpathlineto{\pgfqpoint{7.222229in}{2.044766in}}%
\pgfpathlineto{\pgfqpoint{7.219057in}{2.044806in}}%
\pgfpathlineto{\pgfqpoint{7.215885in}{2.045018in}}%
\pgfpathlineto{\pgfqpoint{7.212713in}{2.045014in}}%
\pgfpathlineto{\pgfqpoint{7.209541in}{2.044957in}}%
\pgfpathlineto{\pgfqpoint{7.206369in}{2.045147in}}%
\pgfpathlineto{\pgfqpoint{7.203197in}{2.045000in}}%
\pgfpathlineto{\pgfqpoint{7.200025in}{2.044796in}}%
\pgfpathlineto{\pgfqpoint{7.196853in}{2.044855in}}%
\pgfpathlineto{\pgfqpoint{7.193680in}{2.044934in}}%
\pgfpathlineto{\pgfqpoint{7.190508in}{2.044811in}}%
\pgfpathlineto{\pgfqpoint{7.187336in}{2.044427in}}%
\pgfpathlineto{\pgfqpoint{7.184164in}{2.044771in}}%
\pgfpathlineto{\pgfqpoint{7.180992in}{2.044750in}}%
\pgfpathlineto{\pgfqpoint{7.177820in}{2.044735in}}%
\pgfpathlineto{\pgfqpoint{7.174648in}{2.044642in}}%
\pgfpathlineto{\pgfqpoint{7.171476in}{2.044755in}}%
\pgfpathlineto{\pgfqpoint{7.168304in}{2.044410in}}%
\pgfpathlineto{\pgfqpoint{7.165132in}{2.043964in}}%
\pgfpathlineto{\pgfqpoint{7.161960in}{2.043659in}}%
\pgfpathlineto{\pgfqpoint{7.158788in}{2.043540in}}%
\pgfpathlineto{\pgfqpoint{7.155616in}{2.043565in}}%
\pgfpathlineto{\pgfqpoint{7.152444in}{2.043435in}}%
\pgfpathlineto{\pgfqpoint{7.149272in}{2.043332in}}%
\pgfpathlineto{\pgfqpoint{7.146100in}{2.043506in}}%
\pgfpathlineto{\pgfqpoint{7.142928in}{2.043540in}}%
\pgfpathlineto{\pgfqpoint{7.139756in}{2.043127in}}%
\pgfpathlineto{\pgfqpoint{7.136584in}{2.043210in}}%
\pgfpathlineto{\pgfqpoint{7.133412in}{2.043309in}}%
\pgfpathlineto{\pgfqpoint{7.130240in}{2.043465in}}%
\pgfpathlineto{\pgfqpoint{7.127068in}{2.043418in}}%
\pgfpathlineto{\pgfqpoint{7.123896in}{2.043467in}}%
\pgfpathlineto{\pgfqpoint{7.120724in}{2.043386in}}%
\pgfpathlineto{\pgfqpoint{7.117551in}{2.044058in}}%
\pgfpathlineto{\pgfqpoint{7.114379in}{2.044243in}}%
\pgfpathlineto{\pgfqpoint{7.111207in}{2.044176in}}%
\pgfpathlineto{\pgfqpoint{7.108035in}{2.043781in}}%
\pgfpathlineto{\pgfqpoint{7.104863in}{2.043470in}}%
\pgfpathlineto{\pgfqpoint{7.101691in}{2.043152in}}%
\pgfpathlineto{\pgfqpoint{7.098519in}{2.042926in}}%
\pgfpathlineto{\pgfqpoint{7.095347in}{2.042856in}}%
\pgfpathlineto{\pgfqpoint{7.092175in}{2.043094in}}%
\pgfpathlineto{\pgfqpoint{7.089003in}{2.043145in}}%
\pgfpathlineto{\pgfqpoint{7.085831in}{2.043153in}}%
\pgfpathlineto{\pgfqpoint{7.082659in}{2.043068in}}%
\pgfpathlineto{\pgfqpoint{7.079487in}{2.043361in}}%
\pgfpathlineto{\pgfqpoint{7.076315in}{2.043568in}}%
\pgfpathlineto{\pgfqpoint{7.073143in}{2.043380in}}%
\pgfpathlineto{\pgfqpoint{7.069971in}{2.043434in}}%
\pgfpathlineto{\pgfqpoint{7.066799in}{2.043810in}}%
\pgfpathlineto{\pgfqpoint{7.063627in}{2.043933in}}%
\pgfpathlineto{\pgfqpoint{7.060455in}{2.044368in}}%
\pgfpathlineto{\pgfqpoint{7.057283in}{2.043677in}}%
\pgfpathlineto{\pgfqpoint{7.054111in}{2.043245in}}%
\pgfpathlineto{\pgfqpoint{7.050939in}{2.043036in}}%
\pgfpathlineto{\pgfqpoint{7.047767in}{2.043110in}}%
\pgfpathlineto{\pgfqpoint{7.044595in}{2.042932in}}%
\pgfpathlineto{\pgfqpoint{7.041423in}{2.043065in}}%
\pgfpathlineto{\pgfqpoint{7.038250in}{2.042979in}}%
\pgfpathlineto{\pgfqpoint{7.035078in}{2.043112in}}%
\pgfpathlineto{\pgfqpoint{7.031906in}{2.042893in}}%
\pgfpathlineto{\pgfqpoint{7.028734in}{2.042769in}}%
\pgfpathlineto{\pgfqpoint{7.025562in}{2.042677in}}%
\pgfpathlineto{\pgfqpoint{7.022390in}{2.042718in}}%
\pgfpathlineto{\pgfqpoint{7.019218in}{2.042700in}}%
\pgfpathlineto{\pgfqpoint{7.016046in}{2.042960in}}%
\pgfpathlineto{\pgfqpoint{7.012874in}{2.042960in}}%
\pgfpathlineto{\pgfqpoint{7.009702in}{2.043403in}}%
\pgfpathlineto{\pgfqpoint{7.006530in}{2.043401in}}%
\pgfpathlineto{\pgfqpoint{7.003358in}{2.042890in}}%
\pgfpathlineto{\pgfqpoint{7.000186in}{2.042711in}}%
\pgfpathlineto{\pgfqpoint{6.997014in}{2.042624in}}%
\pgfpathlineto{\pgfqpoint{6.993842in}{2.042341in}}%
\pgfpathlineto{\pgfqpoint{6.990670in}{2.041947in}}%
\pgfpathlineto{\pgfqpoint{6.987498in}{2.041795in}}%
\pgfpathlineto{\pgfqpoint{6.984326in}{2.041927in}}%
\pgfpathlineto{\pgfqpoint{6.981154in}{2.041608in}}%
\pgfpathlineto{\pgfqpoint{6.977982in}{2.041449in}}%
\pgfpathlineto{\pgfqpoint{6.974810in}{2.041455in}}%
\pgfpathlineto{\pgfqpoint{6.971638in}{2.040480in}}%
\pgfpathlineto{\pgfqpoint{6.968466in}{2.040136in}}%
\pgfpathlineto{\pgfqpoint{6.965294in}{2.040231in}}%
\pgfpathlineto{\pgfqpoint{6.962122in}{2.040304in}}%
\pgfpathlineto{\pgfqpoint{6.958949in}{2.040039in}}%
\pgfpathlineto{\pgfqpoint{6.955777in}{2.039871in}}%
\pgfpathlineto{\pgfqpoint{6.952605in}{2.039599in}}%
\pgfpathlineto{\pgfqpoint{6.949433in}{2.038803in}}%
\pgfpathlineto{\pgfqpoint{6.946261in}{2.039184in}}%
\pgfpathlineto{\pgfqpoint{6.943089in}{2.039232in}}%
\pgfpathlineto{\pgfqpoint{6.939917in}{2.039280in}}%
\pgfpathlineto{\pgfqpoint{6.936745in}{2.039304in}}%
\pgfpathlineto{\pgfqpoint{6.933573in}{2.039258in}}%
\pgfpathlineto{\pgfqpoint{6.930401in}{2.038849in}}%
\pgfpathlineto{\pgfqpoint{6.927229in}{2.038608in}}%
\pgfpathlineto{\pgfqpoint{6.924057in}{2.038487in}}%
\pgfpathlineto{\pgfqpoint{6.920885in}{2.038418in}}%
\pgfpathlineto{\pgfqpoint{6.917713in}{2.038344in}}%
\pgfpathlineto{\pgfqpoint{6.914541in}{2.038389in}}%
\pgfpathlineto{\pgfqpoint{6.911369in}{2.038245in}}%
\pgfpathlineto{\pgfqpoint{6.908197in}{2.038707in}}%
\pgfpathlineto{\pgfqpoint{6.905025in}{2.038822in}}%
\pgfpathlineto{\pgfqpoint{6.901853in}{2.039053in}}%
\pgfpathlineto{\pgfqpoint{6.898681in}{2.039233in}}%
\pgfpathlineto{\pgfqpoint{6.895509in}{2.038690in}}%
\pgfpathlineto{\pgfqpoint{6.892337in}{2.038233in}}%
\pgfpathlineto{\pgfqpoint{6.889165in}{2.037937in}}%
\pgfpathlineto{\pgfqpoint{6.885993in}{2.038120in}}%
\pgfpathlineto{\pgfqpoint{6.882820in}{2.037888in}}%
\pgfpathlineto{\pgfqpoint{6.879648in}{2.037651in}}%
\pgfpathlineto{\pgfqpoint{6.876476in}{2.037385in}}%
\pgfpathlineto{\pgfqpoint{6.873304in}{2.037640in}}%
\pgfpathlineto{\pgfqpoint{6.870132in}{2.038016in}}%
\pgfpathlineto{\pgfqpoint{6.866960in}{2.038535in}}%
\pgfpathlineto{\pgfqpoint{6.863788in}{2.038773in}}%
\pgfpathlineto{\pgfqpoint{6.860616in}{2.038628in}}%
\pgfpathlineto{\pgfqpoint{6.857444in}{2.038330in}}%
\pgfpathlineto{\pgfqpoint{6.854272in}{2.038201in}}%
\pgfpathlineto{\pgfqpoint{6.851100in}{2.038108in}}%
\pgfpathlineto{\pgfqpoint{6.847928in}{2.037770in}}%
\pgfpathlineto{\pgfqpoint{6.844756in}{2.037879in}}%
\pgfpathlineto{\pgfqpoint{6.841584in}{2.038413in}}%
\pgfpathlineto{\pgfqpoint{6.838412in}{2.038638in}}%
\pgfpathlineto{\pgfqpoint{6.835240in}{2.038152in}}%
\pgfpathlineto{\pgfqpoint{6.832068in}{2.037811in}}%
\pgfpathlineto{\pgfqpoint{6.828896in}{2.037739in}}%
\pgfpathlineto{\pgfqpoint{6.825724in}{2.037817in}}%
\pgfpathlineto{\pgfqpoint{6.822552in}{2.037645in}}%
\pgfpathlineto{\pgfqpoint{6.819380in}{2.037580in}}%
\pgfpathlineto{\pgfqpoint{6.816208in}{2.037698in}}%
\pgfpathlineto{\pgfqpoint{6.813036in}{2.038306in}}%
\pgfpathlineto{\pgfqpoint{6.809864in}{2.038636in}}%
\pgfpathlineto{\pgfqpoint{6.806692in}{2.038444in}}%
\pgfpathlineto{\pgfqpoint{6.803519in}{2.038211in}}%
\pgfpathlineto{\pgfqpoint{6.800347in}{2.037906in}}%
\pgfpathlineto{\pgfqpoint{6.797175in}{2.038291in}}%
\pgfpathlineto{\pgfqpoint{6.794003in}{2.038288in}}%
\pgfpathlineto{\pgfqpoint{6.790831in}{2.038692in}}%
\pgfpathlineto{\pgfqpoint{6.787659in}{2.038496in}}%
\pgfpathlineto{\pgfqpoint{6.784487in}{2.038076in}}%
\pgfpathlineto{\pgfqpoint{6.781315in}{2.038136in}}%
\pgfpathlineto{\pgfqpoint{6.778143in}{2.038068in}}%
\pgfpathlineto{\pgfqpoint{6.774971in}{2.037930in}}%
\pgfpathlineto{\pgfqpoint{6.771799in}{2.037639in}}%
\pgfpathlineto{\pgfqpoint{6.768627in}{2.037659in}}%
\pgfpathlineto{\pgfqpoint{6.765455in}{2.037912in}}%
\pgfpathlineto{\pgfqpoint{6.762283in}{2.037996in}}%
\pgfpathlineto{\pgfqpoint{6.759111in}{2.037934in}}%
\pgfpathlineto{\pgfqpoint{6.755939in}{2.037955in}}%
\pgfpathlineto{\pgfqpoint{6.752767in}{2.037771in}}%
\pgfpathlineto{\pgfqpoint{6.749595in}{2.037970in}}%
\pgfpathlineto{\pgfqpoint{6.746423in}{2.037526in}}%
\pgfpathlineto{\pgfqpoint{6.743251in}{2.037617in}}%
\pgfpathlineto{\pgfqpoint{6.740079in}{2.037387in}}%
\pgfpathlineto{\pgfqpoint{6.736907in}{2.037158in}}%
\pgfpathlineto{\pgfqpoint{6.733735in}{2.036891in}}%
\pgfpathlineto{\pgfqpoint{6.730563in}{2.036732in}}%
\pgfpathlineto{\pgfqpoint{6.727391in}{2.036967in}}%
\pgfpathlineto{\pgfqpoint{6.724218in}{2.037241in}}%
\pgfpathlineto{\pgfqpoint{6.721046in}{2.037516in}}%
\pgfpathlineto{\pgfqpoint{6.717874in}{2.037222in}}%
\pgfpathlineto{\pgfqpoint{6.714702in}{2.036888in}}%
\pgfpathlineto{\pgfqpoint{6.711530in}{2.036929in}}%
\pgfpathlineto{\pgfqpoint{6.708358in}{2.037272in}}%
\pgfpathlineto{\pgfqpoint{6.705186in}{2.037320in}}%
\pgfpathlineto{\pgfqpoint{6.702014in}{2.037091in}}%
\pgfpathlineto{\pgfqpoint{6.698842in}{2.036876in}}%
\pgfpathlineto{\pgfqpoint{6.695670in}{2.036919in}}%
\pgfpathlineto{\pgfqpoint{6.692498in}{2.036707in}}%
\pgfpathlineto{\pgfqpoint{6.689326in}{2.036627in}}%
\pgfpathlineto{\pgfqpoint{6.686154in}{2.036559in}}%
\pgfpathlineto{\pgfqpoint{6.682982in}{2.036947in}}%
\pgfpathlineto{\pgfqpoint{6.679810in}{2.036328in}}%
\pgfpathlineto{\pgfqpoint{6.676638in}{2.036563in}}%
\pgfpathlineto{\pgfqpoint{6.673466in}{2.037015in}}%
\pgfpathlineto{\pgfqpoint{6.670294in}{2.036741in}}%
\pgfpathlineto{\pgfqpoint{6.667122in}{2.036506in}}%
\pgfpathlineto{\pgfqpoint{6.663950in}{2.036903in}}%
\pgfpathlineto{\pgfqpoint{6.660778in}{2.037177in}}%
\pgfpathlineto{\pgfqpoint{6.657606in}{2.036752in}}%
\pgfpathlineto{\pgfqpoint{6.654434in}{2.036475in}}%
\pgfpathlineto{\pgfqpoint{6.651262in}{2.036036in}}%
\pgfpathlineto{\pgfqpoint{6.648089in}{2.036051in}}%
\pgfpathlineto{\pgfqpoint{6.644917in}{2.035581in}}%
\pgfpathlineto{\pgfqpoint{6.641745in}{2.035779in}}%
\pgfpathlineto{\pgfqpoint{6.638573in}{2.035670in}}%
\pgfpathlineto{\pgfqpoint{6.635401in}{2.036077in}}%
\pgfpathlineto{\pgfqpoint{6.632229in}{2.036147in}}%
\pgfpathlineto{\pgfqpoint{6.629057in}{2.036042in}}%
\pgfpathlineto{\pgfqpoint{6.625885in}{2.035981in}}%
\pgfpathlineto{\pgfqpoint{6.622713in}{2.036288in}}%
\pgfpathlineto{\pgfqpoint{6.619541in}{2.036531in}}%
\pgfpathlineto{\pgfqpoint{6.616369in}{2.036294in}}%
\pgfpathlineto{\pgfqpoint{6.613197in}{2.035995in}}%
\pgfpathlineto{\pgfqpoint{6.610025in}{2.036172in}}%
\pgfpathlineto{\pgfqpoint{6.606853in}{2.036504in}}%
\pgfpathlineto{\pgfqpoint{6.603681in}{2.036667in}}%
\pgfpathlineto{\pgfqpoint{6.600509in}{2.036028in}}%
\pgfpathlineto{\pgfqpoint{6.597337in}{2.036414in}}%
\pgfpathlineto{\pgfqpoint{6.594165in}{2.036002in}}%
\pgfpathlineto{\pgfqpoint{6.590993in}{2.036128in}}%
\pgfpathlineto{\pgfqpoint{6.587821in}{2.035703in}}%
\pgfpathlineto{\pgfqpoint{6.584649in}{2.035707in}}%
\pgfpathlineto{\pgfqpoint{6.581477in}{2.035565in}}%
\pgfpathlineto{\pgfqpoint{6.578305in}{2.035420in}}%
\pgfpathlineto{\pgfqpoint{6.575133in}{2.035326in}}%
\pgfpathlineto{\pgfqpoint{6.571961in}{2.035159in}}%
\pgfpathlineto{\pgfqpoint{6.568788in}{2.034920in}}%
\pgfpathlineto{\pgfqpoint{6.565616in}{2.035269in}}%
\pgfpathlineto{\pgfqpoint{6.562444in}{2.035291in}}%
\pgfpathlineto{\pgfqpoint{6.559272in}{2.035343in}}%
\pgfpathlineto{\pgfqpoint{6.556100in}{2.035363in}}%
\pgfpathlineto{\pgfqpoint{6.552928in}{2.035665in}}%
\pgfpathlineto{\pgfqpoint{6.549756in}{2.035539in}}%
\pgfpathlineto{\pgfqpoint{6.546584in}{2.035935in}}%
\pgfpathlineto{\pgfqpoint{6.543412in}{2.036180in}}%
\pgfpathlineto{\pgfqpoint{6.540240in}{2.036037in}}%
\pgfpathlineto{\pgfqpoint{6.537068in}{2.036227in}}%
\pgfpathlineto{\pgfqpoint{6.533896in}{2.035894in}}%
\pgfpathlineto{\pgfqpoint{6.530724in}{2.035381in}}%
\pgfpathlineto{\pgfqpoint{6.527552in}{2.035676in}}%
\pgfpathlineto{\pgfqpoint{6.524380in}{2.035783in}}%
\pgfpathlineto{\pgfqpoint{6.521208in}{2.036043in}}%
\pgfpathlineto{\pgfqpoint{6.518036in}{2.036021in}}%
\pgfpathlineto{\pgfqpoint{6.514864in}{2.036240in}}%
\pgfpathlineto{\pgfqpoint{6.511692in}{2.036773in}}%
\pgfpathlineto{\pgfqpoint{6.508520in}{2.036825in}}%
\pgfpathlineto{\pgfqpoint{6.505348in}{2.036875in}}%
\pgfpathlineto{\pgfqpoint{6.502176in}{2.037121in}}%
\pgfpathlineto{\pgfqpoint{6.499004in}{2.037038in}}%
\pgfpathlineto{\pgfqpoint{6.495832in}{2.037302in}}%
\pgfpathlineto{\pgfqpoint{6.492659in}{2.037263in}}%
\pgfpathlineto{\pgfqpoint{6.489487in}{2.037045in}}%
\pgfpathlineto{\pgfqpoint{6.486315in}{2.037030in}}%
\pgfpathlineto{\pgfqpoint{6.483143in}{2.036522in}}%
\pgfpathlineto{\pgfqpoint{6.479971in}{2.036445in}}%
\pgfpathlineto{\pgfqpoint{6.476799in}{2.036525in}}%
\pgfpathlineto{\pgfqpoint{6.473627in}{2.036514in}}%
\pgfpathlineto{\pgfqpoint{6.470455in}{2.036782in}}%
\pgfpathlineto{\pgfqpoint{6.467283in}{2.037028in}}%
\pgfpathlineto{\pgfqpoint{6.464111in}{2.036904in}}%
\pgfpathlineto{\pgfqpoint{6.460939in}{2.037130in}}%
\pgfpathlineto{\pgfqpoint{6.457767in}{2.037266in}}%
\pgfpathlineto{\pgfqpoint{6.454595in}{2.037452in}}%
\pgfpathlineto{\pgfqpoint{6.451423in}{2.037314in}}%
\pgfpathlineto{\pgfqpoint{6.448251in}{2.036507in}}%
\pgfpathlineto{\pgfqpoint{6.445079in}{2.036445in}}%
\pgfpathlineto{\pgfqpoint{6.441907in}{2.036736in}}%
\pgfpathlineto{\pgfqpoint{6.438735in}{2.036864in}}%
\pgfpathlineto{\pgfqpoint{6.435563in}{2.036973in}}%
\pgfpathlineto{\pgfqpoint{6.432391in}{2.037260in}}%
\pgfpathlineto{\pgfqpoint{6.429219in}{2.037253in}}%
\pgfpathlineto{\pgfqpoint{6.426047in}{2.036517in}}%
\pgfpathlineto{\pgfqpoint{6.422875in}{2.036069in}}%
\pgfpathlineto{\pgfqpoint{6.419703in}{2.036076in}}%
\pgfpathlineto{\pgfqpoint{6.416531in}{2.036187in}}%
\pgfpathlineto{\pgfqpoint{6.413358in}{2.036213in}}%
\pgfpathlineto{\pgfqpoint{6.410186in}{2.036523in}}%
\pgfpathlineto{\pgfqpoint{6.407014in}{2.036270in}}%
\pgfpathlineto{\pgfqpoint{6.403842in}{2.035893in}}%
\pgfpathlineto{\pgfqpoint{6.400670in}{2.036088in}}%
\pgfpathlineto{\pgfqpoint{6.397498in}{2.036210in}}%
\pgfpathlineto{\pgfqpoint{6.394326in}{2.036210in}}%
\pgfpathlineto{\pgfqpoint{6.391154in}{2.036270in}}%
\pgfpathlineto{\pgfqpoint{6.387982in}{2.036404in}}%
\pgfpathlineto{\pgfqpoint{6.384810in}{2.036284in}}%
\pgfpathlineto{\pgfqpoint{6.381638in}{2.035707in}}%
\pgfpathlineto{\pgfqpoint{6.378466in}{2.035715in}}%
\pgfpathlineto{\pgfqpoint{6.375294in}{2.035606in}}%
\pgfpathlineto{\pgfqpoint{6.372122in}{2.035203in}}%
\pgfpathlineto{\pgfqpoint{6.368950in}{2.035032in}}%
\pgfpathlineto{\pgfqpoint{6.365778in}{2.034548in}}%
\pgfpathlineto{\pgfqpoint{6.362606in}{2.034408in}}%
\pgfpathlineto{\pgfqpoint{6.359434in}{2.034398in}}%
\pgfpathlineto{\pgfqpoint{6.356262in}{2.034582in}}%
\pgfpathlineto{\pgfqpoint{6.353090in}{2.034534in}}%
\pgfpathlineto{\pgfqpoint{6.349918in}{2.034633in}}%
\pgfpathlineto{\pgfqpoint{6.346746in}{2.034669in}}%
\pgfpathlineto{\pgfqpoint{6.343574in}{2.034912in}}%
\pgfpathlineto{\pgfqpoint{6.340402in}{2.034804in}}%
\pgfpathlineto{\pgfqpoint{6.337230in}{2.034353in}}%
\pgfpathlineto{\pgfqpoint{6.334057in}{2.034255in}}%
\pgfpathlineto{\pgfqpoint{6.330885in}{2.034294in}}%
\pgfpathlineto{\pgfqpoint{6.327713in}{2.034211in}}%
\pgfpathlineto{\pgfqpoint{6.324541in}{2.033972in}}%
\pgfpathlineto{\pgfqpoint{6.321369in}{2.033365in}}%
\pgfpathlineto{\pgfqpoint{6.318197in}{2.033088in}}%
\pgfpathlineto{\pgfqpoint{6.315025in}{2.033245in}}%
\pgfpathlineto{\pgfqpoint{6.311853in}{2.033028in}}%
\pgfpathlineto{\pgfqpoint{6.308681in}{2.033188in}}%
\pgfpathlineto{\pgfqpoint{6.305509in}{2.033389in}}%
\pgfpathlineto{\pgfqpoint{6.302337in}{2.033431in}}%
\pgfpathlineto{\pgfqpoint{6.299165in}{2.033455in}}%
\pgfpathlineto{\pgfqpoint{6.295993in}{2.033395in}}%
\pgfpathlineto{\pgfqpoint{6.292821in}{2.033762in}}%
\pgfpathlineto{\pgfqpoint{6.289649in}{2.033434in}}%
\pgfpathlineto{\pgfqpoint{6.286477in}{2.033117in}}%
\pgfpathlineto{\pgfqpoint{6.283305in}{2.032637in}}%
\pgfpathlineto{\pgfqpoint{6.280133in}{2.032454in}}%
\pgfpathlineto{\pgfqpoint{6.276961in}{2.032297in}}%
\pgfpathlineto{\pgfqpoint{6.273789in}{2.032398in}}%
\pgfpathlineto{\pgfqpoint{6.270617in}{2.032451in}}%
\pgfpathlineto{\pgfqpoint{6.267445in}{2.032248in}}%
\pgfpathlineto{\pgfqpoint{6.264273in}{2.031699in}}%
\pgfpathlineto{\pgfqpoint{6.261101in}{2.031797in}}%
\pgfpathlineto{\pgfqpoint{6.257928in}{2.031899in}}%
\pgfpathlineto{\pgfqpoint{6.254756in}{2.031655in}}%
\pgfpathlineto{\pgfqpoint{6.251584in}{2.031794in}}%
\pgfpathlineto{\pgfqpoint{6.248412in}{2.031696in}}%
\pgfpathlineto{\pgfqpoint{6.245240in}{2.031589in}}%
\pgfpathlineto{\pgfqpoint{6.242068in}{2.031549in}}%
\pgfpathlineto{\pgfqpoint{6.238896in}{2.031944in}}%
\pgfpathlineto{\pgfqpoint{6.235724in}{2.031846in}}%
\pgfpathlineto{\pgfqpoint{6.232552in}{2.031975in}}%
\pgfpathlineto{\pgfqpoint{6.229380in}{2.032025in}}%
\pgfpathlineto{\pgfqpoint{6.226208in}{2.031976in}}%
\pgfpathlineto{\pgfqpoint{6.223036in}{2.031791in}}%
\pgfpathlineto{\pgfqpoint{6.219864in}{2.031549in}}%
\pgfpathlineto{\pgfqpoint{6.216692in}{2.031283in}}%
\pgfpathlineto{\pgfqpoint{6.213520in}{2.031547in}}%
\pgfpathlineto{\pgfqpoint{6.210348in}{2.032197in}}%
\pgfpathlineto{\pgfqpoint{6.207176in}{2.031902in}}%
\pgfpathlineto{\pgfqpoint{6.204004in}{2.032117in}}%
\pgfpathlineto{\pgfqpoint{6.200832in}{2.031962in}}%
\pgfpathlineto{\pgfqpoint{6.197660in}{2.031668in}}%
\pgfpathlineto{\pgfqpoint{6.194488in}{2.031462in}}%
\pgfpathlineto{\pgfqpoint{6.191316in}{2.031491in}}%
\pgfpathlineto{\pgfqpoint{6.188144in}{2.031033in}}%
\pgfpathlineto{\pgfqpoint{6.184972in}{2.030786in}}%
\pgfpathlineto{\pgfqpoint{6.181800in}{2.030805in}}%
\pgfpathlineto{\pgfqpoint{6.178627in}{2.030991in}}%
\pgfpathlineto{\pgfqpoint{6.175455in}{2.031063in}}%
\pgfpathlineto{\pgfqpoint{6.172283in}{2.030701in}}%
\pgfpathlineto{\pgfqpoint{6.169111in}{2.030480in}}%
\pgfpathlineto{\pgfqpoint{6.165939in}{2.030872in}}%
\pgfpathlineto{\pgfqpoint{6.162767in}{2.031273in}}%
\pgfpathlineto{\pgfqpoint{6.159595in}{2.030881in}}%
\pgfpathlineto{\pgfqpoint{6.156423in}{2.030758in}}%
\pgfpathlineto{\pgfqpoint{6.153251in}{2.030864in}}%
\pgfpathlineto{\pgfqpoint{6.150079in}{2.031054in}}%
\pgfpathlineto{\pgfqpoint{6.146907in}{2.030866in}}%
\pgfpathlineto{\pgfqpoint{6.143735in}{2.031214in}}%
\pgfpathlineto{\pgfqpoint{6.140563in}{2.031135in}}%
\pgfpathlineto{\pgfqpoint{6.137391in}{2.031526in}}%
\pgfpathlineto{\pgfqpoint{6.134219in}{2.031741in}}%
\pgfpathlineto{\pgfqpoint{6.131047in}{2.031511in}}%
\pgfpathlineto{\pgfqpoint{6.127875in}{2.030971in}}%
\pgfpathlineto{\pgfqpoint{6.124703in}{2.031251in}}%
\pgfpathlineto{\pgfqpoint{6.121531in}{2.031833in}}%
\pgfpathlineto{\pgfqpoint{6.118359in}{2.032143in}}%
\pgfpathlineto{\pgfqpoint{6.115187in}{2.032094in}}%
\pgfpathlineto{\pgfqpoint{6.112015in}{2.031890in}}%
\pgfpathlineto{\pgfqpoint{6.108843in}{2.031834in}}%
\pgfpathlineto{\pgfqpoint{6.105671in}{2.031997in}}%
\pgfpathlineto{\pgfqpoint{6.102499in}{2.031902in}}%
\pgfpathlineto{\pgfqpoint{6.099326in}{2.031879in}}%
\pgfpathlineto{\pgfqpoint{6.096154in}{2.031918in}}%
\pgfpathlineto{\pgfqpoint{6.092982in}{2.031927in}}%
\pgfpathlineto{\pgfqpoint{6.089810in}{2.032107in}}%
\pgfpathlineto{\pgfqpoint{6.086638in}{2.032246in}}%
\pgfpathlineto{\pgfqpoint{6.083466in}{2.031879in}}%
\pgfpathlineto{\pgfqpoint{6.080294in}{2.031493in}}%
\pgfpathlineto{\pgfqpoint{6.077122in}{2.031581in}}%
\pgfpathlineto{\pgfqpoint{6.073950in}{2.031310in}}%
\pgfpathlineto{\pgfqpoint{6.070778in}{2.031270in}}%
\pgfpathlineto{\pgfqpoint{6.067606in}{2.031345in}}%
\pgfpathlineto{\pgfqpoint{6.064434in}{2.031420in}}%
\pgfpathlineto{\pgfqpoint{6.061262in}{2.031485in}}%
\pgfpathlineto{\pgfqpoint{6.058090in}{2.031910in}}%
\pgfpathlineto{\pgfqpoint{6.054918in}{2.031930in}}%
\pgfpathlineto{\pgfqpoint{6.051746in}{2.031985in}}%
\pgfpathlineto{\pgfqpoint{6.048574in}{2.031734in}}%
\pgfpathlineto{\pgfqpoint{6.045402in}{2.032155in}}%
\pgfpathlineto{\pgfqpoint{6.042230in}{2.032334in}}%
\pgfpathlineto{\pgfqpoint{6.039058in}{2.032843in}}%
\pgfpathlineto{\pgfqpoint{6.035886in}{2.032756in}}%
\pgfpathlineto{\pgfqpoint{6.032714in}{2.032981in}}%
\pgfpathlineto{\pgfqpoint{6.029542in}{2.033208in}}%
\pgfpathlineto{\pgfqpoint{6.026370in}{2.033030in}}%
\pgfpathlineto{\pgfqpoint{6.023197in}{2.032829in}}%
\pgfpathlineto{\pgfqpoint{6.020025in}{2.032466in}}%
\pgfpathlineto{\pgfqpoint{6.016853in}{2.033237in}}%
\pgfpathlineto{\pgfqpoint{6.013681in}{2.033339in}}%
\pgfpathlineto{\pgfqpoint{6.010509in}{2.033277in}}%
\pgfpathlineto{\pgfqpoint{6.007337in}{2.033384in}}%
\pgfpathlineto{\pgfqpoint{6.004165in}{2.033401in}}%
\pgfpathlineto{\pgfqpoint{6.000993in}{2.033256in}}%
\pgfpathlineto{\pgfqpoint{5.997821in}{2.033510in}}%
\pgfpathlineto{\pgfqpoint{5.994649in}{2.034003in}}%
\pgfpathlineto{\pgfqpoint{5.991477in}{2.034270in}}%
\pgfpathlineto{\pgfqpoint{5.988305in}{2.034354in}}%
\pgfpathlineto{\pgfqpoint{5.985133in}{2.034107in}}%
\pgfpathlineto{\pgfqpoint{5.981961in}{2.034096in}}%
\pgfpathlineto{\pgfqpoint{5.978789in}{2.034353in}}%
\pgfpathlineto{\pgfqpoint{5.975617in}{2.034971in}}%
\pgfpathlineto{\pgfqpoint{5.972445in}{2.034736in}}%
\pgfpathlineto{\pgfqpoint{5.969273in}{2.034832in}}%
\pgfpathlineto{\pgfqpoint{5.966101in}{2.034856in}}%
\pgfpathlineto{\pgfqpoint{5.962929in}{2.034994in}}%
\pgfpathlineto{\pgfqpoint{5.959757in}{2.035083in}}%
\pgfpathlineto{\pgfqpoint{5.956585in}{2.035399in}}%
\pgfpathlineto{\pgfqpoint{5.953413in}{2.035352in}}%
\pgfpathlineto{\pgfqpoint{5.950241in}{2.035513in}}%
\pgfpathlineto{\pgfqpoint{5.947069in}{2.035690in}}%
\pgfpathlineto{\pgfqpoint{5.943896in}{2.035279in}}%
\pgfpathlineto{\pgfqpoint{5.940724in}{2.035075in}}%
\pgfpathlineto{\pgfqpoint{5.937552in}{2.034850in}}%
\pgfpathlineto{\pgfqpoint{5.934380in}{2.034257in}}%
\pgfpathlineto{\pgfqpoint{5.931208in}{2.033941in}}%
\pgfpathlineto{\pgfqpoint{5.928036in}{2.033942in}}%
\pgfpathlineto{\pgfqpoint{5.924864in}{2.034371in}}%
\pgfpathlineto{\pgfqpoint{5.921692in}{2.034396in}}%
\pgfpathlineto{\pgfqpoint{5.918520in}{2.034379in}}%
\pgfpathlineto{\pgfqpoint{5.915348in}{2.034546in}}%
\pgfpathlineto{\pgfqpoint{5.912176in}{2.034002in}}%
\pgfpathlineto{\pgfqpoint{5.909004in}{2.033666in}}%
\pgfpathlineto{\pgfqpoint{5.905832in}{2.034068in}}%
\pgfpathlineto{\pgfqpoint{5.902660in}{2.034160in}}%
\pgfpathlineto{\pgfqpoint{5.899488in}{2.033839in}}%
\pgfpathlineto{\pgfqpoint{5.896316in}{2.033602in}}%
\pgfpathlineto{\pgfqpoint{5.893144in}{2.033812in}}%
\pgfpathlineto{\pgfqpoint{5.889972in}{2.034111in}}%
\pgfpathlineto{\pgfqpoint{5.886800in}{2.034273in}}%
\pgfpathlineto{\pgfqpoint{5.883628in}{2.033956in}}%
\pgfpathlineto{\pgfqpoint{5.880456in}{2.033648in}}%
\pgfpathlineto{\pgfqpoint{5.877284in}{2.033730in}}%
\pgfpathlineto{\pgfqpoint{5.874112in}{2.033688in}}%
\pgfpathlineto{\pgfqpoint{5.870940in}{2.033760in}}%
\pgfpathlineto{\pgfqpoint{5.867768in}{2.033521in}}%
\pgfpathlineto{\pgfqpoint{5.864595in}{2.033559in}}%
\pgfpathlineto{\pgfqpoint{5.861423in}{2.033643in}}%
\pgfpathlineto{\pgfqpoint{5.858251in}{2.033850in}}%
\pgfpathlineto{\pgfqpoint{5.855079in}{2.033921in}}%
\pgfpathlineto{\pgfqpoint{5.851907in}{2.034049in}}%
\pgfpathlineto{\pgfqpoint{5.848735in}{2.033709in}}%
\pgfpathlineto{\pgfqpoint{5.845563in}{2.033513in}}%
\pgfpathlineto{\pgfqpoint{5.842391in}{2.033635in}}%
\pgfpathlineto{\pgfqpoint{5.839219in}{2.034122in}}%
\pgfpathlineto{\pgfqpoint{5.836047in}{2.033959in}}%
\pgfpathlineto{\pgfqpoint{5.832875in}{2.034041in}}%
\pgfpathlineto{\pgfqpoint{5.829703in}{2.034011in}}%
\pgfpathlineto{\pgfqpoint{5.826531in}{2.033957in}}%
\pgfpathlineto{\pgfqpoint{5.823359in}{2.033535in}}%
\pgfpathlineto{\pgfqpoint{5.820187in}{2.033434in}}%
\pgfpathlineto{\pgfqpoint{5.817015in}{2.033039in}}%
\pgfpathlineto{\pgfqpoint{5.813843in}{2.032966in}}%
\pgfpathlineto{\pgfqpoint{5.810671in}{2.032305in}}%
\pgfpathlineto{\pgfqpoint{5.807499in}{2.032507in}}%
\pgfpathlineto{\pgfqpoint{5.804327in}{2.032342in}}%
\pgfpathlineto{\pgfqpoint{5.801155in}{2.032296in}}%
\pgfpathlineto{\pgfqpoint{5.797983in}{2.031796in}}%
\pgfpathlineto{\pgfqpoint{5.794811in}{2.032000in}}%
\pgfpathlineto{\pgfqpoint{5.791639in}{2.031930in}}%
\pgfpathlineto{\pgfqpoint{5.788466in}{2.031844in}}%
\pgfpathlineto{\pgfqpoint{5.785294in}{2.031746in}}%
\pgfpathlineto{\pgfqpoint{5.782122in}{2.032333in}}%
\pgfpathlineto{\pgfqpoint{5.778950in}{2.032672in}}%
\pgfpathlineto{\pgfqpoint{5.775778in}{2.032650in}}%
\pgfpathlineto{\pgfqpoint{5.772606in}{2.032756in}}%
\pgfpathlineto{\pgfqpoint{5.769434in}{2.032901in}}%
\pgfpathlineto{\pgfqpoint{5.766262in}{2.032363in}}%
\pgfpathlineto{\pgfqpoint{5.763090in}{2.031894in}}%
\pgfpathlineto{\pgfqpoint{5.759918in}{2.031411in}}%
\pgfpathlineto{\pgfqpoint{5.756746in}{2.031679in}}%
\pgfpathlineto{\pgfqpoint{5.753574in}{2.031300in}}%
\pgfpathlineto{\pgfqpoint{5.750402in}{2.031330in}}%
\pgfpathlineto{\pgfqpoint{5.747230in}{2.031790in}}%
\pgfpathlineto{\pgfqpoint{5.744058in}{2.032094in}}%
\pgfpathlineto{\pgfqpoint{5.740886in}{2.031966in}}%
\pgfpathlineto{\pgfqpoint{5.737714in}{2.031801in}}%
\pgfpathlineto{\pgfqpoint{5.734542in}{2.032032in}}%
\pgfpathlineto{\pgfqpoint{5.731370in}{2.031903in}}%
\pgfpathlineto{\pgfqpoint{5.728198in}{2.031757in}}%
\pgfpathlineto{\pgfqpoint{5.725026in}{2.031716in}}%
\pgfpathlineto{\pgfqpoint{5.721854in}{2.031433in}}%
\pgfpathlineto{\pgfqpoint{5.718682in}{2.031503in}}%
\pgfpathlineto{\pgfqpoint{5.715510in}{2.031433in}}%
\pgfpathlineto{\pgfqpoint{5.712338in}{2.031448in}}%
\pgfpathlineto{\pgfqpoint{5.709165in}{2.031104in}}%
\pgfpathlineto{\pgfqpoint{5.705993in}{2.031034in}}%
\pgfpathlineto{\pgfqpoint{5.702821in}{2.030796in}}%
\pgfpathlineto{\pgfqpoint{5.699649in}{2.031037in}}%
\pgfpathlineto{\pgfqpoint{5.696477in}{2.031335in}}%
\pgfpathlineto{\pgfqpoint{5.693305in}{2.030717in}}%
\pgfpathlineto{\pgfqpoint{5.690133in}{2.030666in}}%
\pgfpathlineto{\pgfqpoint{5.686961in}{2.030800in}}%
\pgfpathlineto{\pgfqpoint{5.683789in}{2.030544in}}%
\pgfpathlineto{\pgfqpoint{5.680617in}{2.030721in}}%
\pgfpathlineto{\pgfqpoint{5.677445in}{2.030510in}}%
\pgfpathlineto{\pgfqpoint{5.674273in}{2.030947in}}%
\pgfpathlineto{\pgfqpoint{5.671101in}{2.030831in}}%
\pgfpathlineto{\pgfqpoint{5.667929in}{2.030368in}}%
\pgfpathlineto{\pgfqpoint{5.664757in}{2.030393in}}%
\pgfpathlineto{\pgfqpoint{5.661585in}{2.030586in}}%
\pgfpathlineto{\pgfqpoint{5.658413in}{2.030837in}}%
\pgfpathlineto{\pgfqpoint{5.655241in}{2.030920in}}%
\pgfpathlineto{\pgfqpoint{5.652069in}{2.031277in}}%
\pgfpathlineto{\pgfqpoint{5.648897in}{2.031109in}}%
\pgfpathlineto{\pgfqpoint{5.645725in}{2.030693in}}%
\pgfpathlineto{\pgfqpoint{5.642553in}{2.030875in}}%
\pgfpathlineto{\pgfqpoint{5.639381in}{2.030986in}}%
\pgfpathlineto{\pgfqpoint{5.636209in}{2.030954in}}%
\pgfpathlineto{\pgfqpoint{5.633037in}{2.030808in}}%
\pgfpathlineto{\pgfqpoint{5.629864in}{2.030809in}}%
\pgfpathlineto{\pgfqpoint{5.626692in}{2.031318in}}%
\pgfpathlineto{\pgfqpoint{5.623520in}{2.031075in}}%
\pgfpathlineto{\pgfqpoint{5.620348in}{2.031351in}}%
\pgfpathlineto{\pgfqpoint{5.617176in}{2.031480in}}%
\pgfpathlineto{\pgfqpoint{5.614004in}{2.031538in}}%
\pgfpathlineto{\pgfqpoint{5.610832in}{2.031113in}}%
\pgfpathlineto{\pgfqpoint{5.607660in}{2.030786in}}%
\pgfpathlineto{\pgfqpoint{5.604488in}{2.030963in}}%
\pgfpathlineto{\pgfqpoint{5.601316in}{2.030829in}}%
\pgfpathlineto{\pgfqpoint{5.598144in}{2.030697in}}%
\pgfpathlineto{\pgfqpoint{5.594972in}{2.030289in}}%
\pgfpathlineto{\pgfqpoint{5.591800in}{2.030356in}}%
\pgfpathlineto{\pgfqpoint{5.588628in}{2.030124in}}%
\pgfpathlineto{\pgfqpoint{5.585456in}{2.030300in}}%
\pgfpathlineto{\pgfqpoint{5.582284in}{2.030377in}}%
\pgfpathlineto{\pgfqpoint{5.579112in}{2.031248in}}%
\pgfpathlineto{\pgfqpoint{5.575940in}{2.031480in}}%
\pgfpathlineto{\pgfqpoint{5.572768in}{2.031587in}}%
\pgfpathlineto{\pgfqpoint{5.569596in}{2.031636in}}%
\pgfpathlineto{\pgfqpoint{5.566424in}{2.031609in}}%
\pgfpathlineto{\pgfqpoint{5.563252in}{2.031780in}}%
\pgfpathlineto{\pgfqpoint{5.560080in}{2.031870in}}%
\pgfpathlineto{\pgfqpoint{5.556908in}{2.032047in}}%
\pgfpathlineto{\pgfqpoint{5.553735in}{2.031967in}}%
\pgfpathlineto{\pgfqpoint{5.550563in}{2.032062in}}%
\pgfpathlineto{\pgfqpoint{5.547391in}{2.031967in}}%
\pgfpathlineto{\pgfqpoint{5.544219in}{2.032153in}}%
\pgfpathlineto{\pgfqpoint{5.541047in}{2.032260in}}%
\pgfpathlineto{\pgfqpoint{5.537875in}{2.031966in}}%
\pgfpathlineto{\pgfqpoint{5.534703in}{2.031789in}}%
\pgfpathlineto{\pgfqpoint{5.531531in}{2.031741in}}%
\pgfpathlineto{\pgfqpoint{5.528359in}{2.031696in}}%
\pgfpathlineto{\pgfqpoint{5.525187in}{2.031981in}}%
\pgfpathlineto{\pgfqpoint{5.522015in}{2.031311in}}%
\pgfpathlineto{\pgfqpoint{5.518843in}{2.031441in}}%
\pgfpathlineto{\pgfqpoint{5.515671in}{2.032322in}}%
\pgfpathlineto{\pgfqpoint{5.512499in}{2.032353in}}%
\pgfpathlineto{\pgfqpoint{5.509327in}{2.032315in}}%
\pgfpathlineto{\pgfqpoint{5.506155in}{2.031917in}}%
\pgfpathlineto{\pgfqpoint{5.502983in}{2.031808in}}%
\pgfpathlineto{\pgfqpoint{5.499811in}{2.031987in}}%
\pgfpathlineto{\pgfqpoint{5.496639in}{2.032130in}}%
\pgfpathlineto{\pgfqpoint{5.493467in}{2.031941in}}%
\pgfpathlineto{\pgfqpoint{5.490295in}{2.031629in}}%
\pgfpathlineto{\pgfqpoint{5.487123in}{2.032004in}}%
\pgfpathlineto{\pgfqpoint{5.483951in}{2.031779in}}%
\pgfpathlineto{\pgfqpoint{5.480779in}{2.031475in}}%
\pgfpathlineto{\pgfqpoint{5.477607in}{2.030997in}}%
\pgfpathlineto{\pgfqpoint{5.474434in}{2.030805in}}%
\pgfpathlineto{\pgfqpoint{5.471262in}{2.031009in}}%
\pgfpathlineto{\pgfqpoint{5.468090in}{2.031280in}}%
\pgfpathlineto{\pgfqpoint{5.464918in}{2.030745in}}%
\pgfpathlineto{\pgfqpoint{5.461746in}{2.031238in}}%
\pgfpathlineto{\pgfqpoint{5.458574in}{2.031567in}}%
\pgfpathlineto{\pgfqpoint{5.455402in}{2.031407in}}%
\pgfpathlineto{\pgfqpoint{5.452230in}{2.031043in}}%
\pgfpathlineto{\pgfqpoint{5.449058in}{2.030512in}}%
\pgfpathlineto{\pgfqpoint{5.445886in}{2.030587in}}%
\pgfpathlineto{\pgfqpoint{5.442714in}{2.030914in}}%
\pgfpathlineto{\pgfqpoint{5.439542in}{2.030742in}}%
\pgfpathlineto{\pgfqpoint{5.436370in}{2.030813in}}%
\pgfpathlineto{\pgfqpoint{5.433198in}{2.030752in}}%
\pgfpathlineto{\pgfqpoint{5.430026in}{2.030309in}}%
\pgfpathlineto{\pgfqpoint{5.426854in}{2.030200in}}%
\pgfpathlineto{\pgfqpoint{5.423682in}{2.030580in}}%
\pgfpathlineto{\pgfqpoint{5.420510in}{2.030569in}}%
\pgfpathlineto{\pgfqpoint{5.417338in}{2.030023in}}%
\pgfpathlineto{\pgfqpoint{5.414166in}{2.030035in}}%
\pgfpathlineto{\pgfqpoint{5.410994in}{2.030354in}}%
\pgfpathlineto{\pgfqpoint{5.407822in}{2.030582in}}%
\pgfpathlineto{\pgfqpoint{5.404650in}{2.030898in}}%
\pgfpathlineto{\pgfqpoint{5.401478in}{2.031072in}}%
\pgfpathlineto{\pgfqpoint{5.398306in}{2.031384in}}%
\pgfpathlineto{\pgfqpoint{5.395133in}{2.031849in}}%
\pgfpathlineto{\pgfqpoint{5.391961in}{2.032193in}}%
\pgfpathlineto{\pgfqpoint{5.388789in}{2.031666in}}%
\pgfpathlineto{\pgfqpoint{5.385617in}{2.031880in}}%
\pgfpathlineto{\pgfqpoint{5.382445in}{2.031779in}}%
\pgfpathlineto{\pgfqpoint{5.379273in}{2.032011in}}%
\pgfpathlineto{\pgfqpoint{5.376101in}{2.031644in}}%
\pgfpathlineto{\pgfqpoint{5.372929in}{2.031351in}}%
\pgfpathlineto{\pgfqpoint{5.369757in}{2.031260in}}%
\pgfpathlineto{\pgfqpoint{5.366585in}{2.031052in}}%
\pgfpathlineto{\pgfqpoint{5.363413in}{2.030693in}}%
\pgfpathlineto{\pgfqpoint{5.360241in}{2.030093in}}%
\pgfpathlineto{\pgfqpoint{5.357069in}{2.030441in}}%
\pgfpathlineto{\pgfqpoint{5.353897in}{2.030785in}}%
\pgfpathlineto{\pgfqpoint{5.350725in}{2.031014in}}%
\pgfpathlineto{\pgfqpoint{5.347553in}{2.031118in}}%
\pgfpathlineto{\pgfqpoint{5.344381in}{2.031287in}}%
\pgfpathlineto{\pgfqpoint{5.341209in}{2.031106in}}%
\pgfpathlineto{\pgfqpoint{5.338037in}{2.030940in}}%
\pgfpathlineto{\pgfqpoint{5.334865in}{2.031008in}}%
\pgfpathlineto{\pgfqpoint{5.331693in}{2.031618in}}%
\pgfpathlineto{\pgfqpoint{5.328521in}{2.032357in}}%
\pgfpathlineto{\pgfqpoint{5.325349in}{2.032085in}}%
\pgfpathlineto{\pgfqpoint{5.322177in}{2.031955in}}%
\pgfpathlineto{\pgfqpoint{5.319004in}{2.031727in}}%
\pgfpathlineto{\pgfqpoint{5.315832in}{2.031868in}}%
\pgfpathlineto{\pgfqpoint{5.312660in}{2.031835in}}%
\pgfpathlineto{\pgfqpoint{5.309488in}{2.031151in}}%
\pgfpathlineto{\pgfqpoint{5.306316in}{2.031111in}}%
\pgfpathlineto{\pgfqpoint{5.303144in}{2.031415in}}%
\pgfpathlineto{\pgfqpoint{5.299972in}{2.031669in}}%
\pgfpathlineto{\pgfqpoint{5.296800in}{2.031787in}}%
\pgfpathlineto{\pgfqpoint{5.293628in}{2.032080in}}%
\pgfpathlineto{\pgfqpoint{5.290456in}{2.032488in}}%
\pgfpathlineto{\pgfqpoint{5.287284in}{2.032235in}}%
\pgfpathlineto{\pgfqpoint{5.284112in}{2.032012in}}%
\pgfpathlineto{\pgfqpoint{5.280940in}{2.031919in}}%
\pgfpathlineto{\pgfqpoint{5.277768in}{2.031783in}}%
\pgfpathlineto{\pgfqpoint{5.274596in}{2.032321in}}%
\pgfpathlineto{\pgfqpoint{5.271424in}{2.032696in}}%
\pgfpathlineto{\pgfqpoint{5.268252in}{2.033272in}}%
\pgfpathlineto{\pgfqpoint{5.265080in}{2.033492in}}%
\pgfpathlineto{\pgfqpoint{5.261908in}{2.033578in}}%
\pgfpathlineto{\pgfqpoint{5.258736in}{2.033071in}}%
\pgfpathlineto{\pgfqpoint{5.255564in}{2.032780in}}%
\pgfpathlineto{\pgfqpoint{5.252392in}{2.032884in}}%
\pgfpathlineto{\pgfqpoint{5.249220in}{2.032625in}}%
\pgfpathlineto{\pgfqpoint{5.246048in}{2.032855in}}%
\pgfpathlineto{\pgfqpoint{5.242876in}{2.033059in}}%
\pgfpathlineto{\pgfqpoint{5.239703in}{2.033185in}}%
\pgfpathlineto{\pgfqpoint{5.236531in}{2.033149in}}%
\pgfpathlineto{\pgfqpoint{5.233359in}{2.033125in}}%
\pgfpathlineto{\pgfqpoint{5.230187in}{2.033013in}}%
\pgfpathlineto{\pgfqpoint{5.227015in}{2.033036in}}%
\pgfpathlineto{\pgfqpoint{5.223843in}{2.033080in}}%
\pgfpathlineto{\pgfqpoint{5.220671in}{2.032911in}}%
\pgfpathlineto{\pgfqpoint{5.217499in}{2.032816in}}%
\pgfpathlineto{\pgfqpoint{5.214327in}{2.032680in}}%
\pgfpathlineto{\pgfqpoint{5.211155in}{2.032808in}}%
\pgfpathlineto{\pgfqpoint{5.207983in}{2.033398in}}%
\pgfpathlineto{\pgfqpoint{5.204811in}{2.033680in}}%
\pgfpathlineto{\pgfqpoint{5.201639in}{2.033502in}}%
\pgfpathlineto{\pgfqpoint{5.198467in}{2.033227in}}%
\pgfpathlineto{\pgfqpoint{5.195295in}{2.032856in}}%
\pgfpathlineto{\pgfqpoint{5.192123in}{2.032996in}}%
\pgfpathlineto{\pgfqpoint{5.188951in}{2.033367in}}%
\pgfpathlineto{\pgfqpoint{5.185779in}{2.033209in}}%
\pgfpathlineto{\pgfqpoint{5.182607in}{2.032510in}}%
\pgfpathlineto{\pgfqpoint{5.179435in}{2.032265in}}%
\pgfpathlineto{\pgfqpoint{5.176263in}{2.031873in}}%
\pgfpathlineto{\pgfqpoint{5.173091in}{2.031531in}}%
\pgfpathlineto{\pgfqpoint{5.169919in}{2.031592in}}%
\pgfpathlineto{\pgfqpoint{5.166747in}{2.031890in}}%
\pgfpathlineto{\pgfqpoint{5.163575in}{2.031937in}}%
\pgfpathlineto{\pgfqpoint{5.160402in}{2.031421in}}%
\pgfpathlineto{\pgfqpoint{5.157230in}{2.031223in}}%
\pgfpathlineto{\pgfqpoint{5.154058in}{2.031335in}}%
\pgfpathlineto{\pgfqpoint{5.150886in}{2.031129in}}%
\pgfpathlineto{\pgfqpoint{5.147714in}{2.030773in}}%
\pgfpathlineto{\pgfqpoint{5.144542in}{2.030774in}}%
\pgfpathlineto{\pgfqpoint{5.141370in}{2.030302in}}%
\pgfpathlineto{\pgfqpoint{5.138198in}{2.030427in}}%
\pgfpathlineto{\pgfqpoint{5.135026in}{2.030486in}}%
\pgfpathlineto{\pgfqpoint{5.131854in}{2.030487in}}%
\pgfpathlineto{\pgfqpoint{5.128682in}{2.030515in}}%
\pgfpathlineto{\pgfqpoint{5.125510in}{2.030114in}}%
\pgfpathlineto{\pgfqpoint{5.122338in}{2.030175in}}%
\pgfpathlineto{\pgfqpoint{5.119166in}{2.030403in}}%
\pgfpathlineto{\pgfqpoint{5.115994in}{2.030392in}}%
\pgfpathlineto{\pgfqpoint{5.112822in}{2.030580in}}%
\pgfpathlineto{\pgfqpoint{5.109650in}{2.030705in}}%
\pgfpathlineto{\pgfqpoint{5.106478in}{2.030377in}}%
\pgfpathlineto{\pgfqpoint{5.103306in}{2.030375in}}%
\pgfpathlineto{\pgfqpoint{5.100134in}{2.030036in}}%
\pgfpathlineto{\pgfqpoint{5.096962in}{2.029784in}}%
\pgfpathlineto{\pgfqpoint{5.093790in}{2.029716in}}%
\pgfpathlineto{\pgfqpoint{5.090618in}{2.029703in}}%
\pgfpathlineto{\pgfqpoint{5.087446in}{2.029988in}}%
\pgfpathlineto{\pgfqpoint{5.084273in}{2.029775in}}%
\pgfpathlineto{\pgfqpoint{5.081101in}{2.030222in}}%
\pgfpathlineto{\pgfqpoint{5.077929in}{2.030310in}}%
\pgfpathlineto{\pgfqpoint{5.074757in}{2.030597in}}%
\pgfpathlineto{\pgfqpoint{5.071585in}{2.030636in}}%
\pgfpathlineto{\pgfqpoint{5.068413in}{2.030634in}}%
\pgfpathlineto{\pgfqpoint{5.065241in}{2.030362in}}%
\pgfpathlineto{\pgfqpoint{5.062069in}{2.030282in}}%
\pgfpathlineto{\pgfqpoint{5.058897in}{2.030133in}}%
\pgfpathlineto{\pgfqpoint{5.055725in}{2.030170in}}%
\pgfpathlineto{\pgfqpoint{5.052553in}{2.029532in}}%
\pgfpathlineto{\pgfqpoint{5.049381in}{2.029462in}}%
\pgfpathlineto{\pgfqpoint{5.046209in}{2.029567in}}%
\pgfpathlineto{\pgfqpoint{5.043037in}{2.029603in}}%
\pgfpathlineto{\pgfqpoint{5.039865in}{2.030141in}}%
\pgfpathlineto{\pgfqpoint{5.036693in}{2.029976in}}%
\pgfpathlineto{\pgfqpoint{5.033521in}{2.029845in}}%
\pgfpathlineto{\pgfqpoint{5.030349in}{2.030073in}}%
\pgfpathlineto{\pgfqpoint{5.027177in}{2.030167in}}%
\pgfpathlineto{\pgfqpoint{5.024005in}{2.030598in}}%
\pgfpathlineto{\pgfqpoint{5.020833in}{2.030952in}}%
\pgfpathlineto{\pgfqpoint{5.017661in}{2.031567in}}%
\pgfpathlineto{\pgfqpoint{5.014489in}{2.031536in}}%
\pgfpathlineto{\pgfqpoint{5.011317in}{2.031729in}}%
\pgfpathlineto{\pgfqpoint{5.008145in}{2.031854in}}%
\pgfpathlineto{\pgfqpoint{5.004972in}{2.032322in}}%
\pgfpathlineto{\pgfqpoint{5.001800in}{2.032756in}}%
\pgfpathlineto{\pgfqpoint{4.998628in}{2.032838in}}%
\pgfpathlineto{\pgfqpoint{4.995456in}{2.032644in}}%
\pgfpathlineto{\pgfqpoint{4.992284in}{2.032237in}}%
\pgfpathlineto{\pgfqpoint{4.989112in}{2.032457in}}%
\pgfpathlineto{\pgfqpoint{4.985940in}{2.032319in}}%
\pgfpathlineto{\pgfqpoint{4.982768in}{2.032510in}}%
\pgfpathlineto{\pgfqpoint{4.979596in}{2.031960in}}%
\pgfpathlineto{\pgfqpoint{4.976424in}{2.031535in}}%
\pgfpathlineto{\pgfqpoint{4.973252in}{2.031360in}}%
\pgfpathlineto{\pgfqpoint{4.970080in}{2.031457in}}%
\pgfpathlineto{\pgfqpoint{4.966908in}{2.031712in}}%
\pgfpathlineto{\pgfqpoint{4.963736in}{2.031431in}}%
\pgfpathlineto{\pgfqpoint{4.960564in}{2.031667in}}%
\pgfpathlineto{\pgfqpoint{4.957392in}{2.032084in}}%
\pgfpathlineto{\pgfqpoint{4.954220in}{2.032511in}}%
\pgfpathlineto{\pgfqpoint{4.951048in}{2.032543in}}%
\pgfpathlineto{\pgfqpoint{4.947876in}{2.032415in}}%
\pgfpathlineto{\pgfqpoint{4.944704in}{2.032844in}}%
\pgfpathlineto{\pgfqpoint{4.941532in}{2.032500in}}%
\pgfpathlineto{\pgfqpoint{4.938360in}{2.032397in}}%
\pgfpathlineto{\pgfqpoint{4.935188in}{2.032351in}}%
\pgfpathlineto{\pgfqpoint{4.932016in}{2.032407in}}%
\pgfpathlineto{\pgfqpoint{4.928844in}{2.032349in}}%
\pgfpathlineto{\pgfqpoint{4.925671in}{2.032406in}}%
\pgfpathlineto{\pgfqpoint{4.922499in}{2.032447in}}%
\pgfpathlineto{\pgfqpoint{4.919327in}{2.032520in}}%
\pgfpathlineto{\pgfqpoint{4.916155in}{2.032570in}}%
\pgfpathlineto{\pgfqpoint{4.912983in}{2.032577in}}%
\pgfpathlineto{\pgfqpoint{4.909811in}{2.032620in}}%
\pgfpathlineto{\pgfqpoint{4.906639in}{2.032782in}}%
\pgfpathlineto{\pgfqpoint{4.903467in}{2.032893in}}%
\pgfpathlineto{\pgfqpoint{4.900295in}{2.032835in}}%
\pgfpathlineto{\pgfqpoint{4.897123in}{2.033134in}}%
\pgfpathlineto{\pgfqpoint{4.893951in}{2.033064in}}%
\pgfpathlineto{\pgfqpoint{4.890779in}{2.033727in}}%
\pgfpathlineto{\pgfqpoint{4.887607in}{2.033655in}}%
\pgfpathlineto{\pgfqpoint{4.884435in}{2.033767in}}%
\pgfpathlineto{\pgfqpoint{4.881263in}{2.034052in}}%
\pgfpathlineto{\pgfqpoint{4.878091in}{2.034330in}}%
\pgfpathlineto{\pgfqpoint{4.874919in}{2.034065in}}%
\pgfpathlineto{\pgfqpoint{4.871747in}{2.034494in}}%
\pgfpathlineto{\pgfqpoint{4.868575in}{2.034239in}}%
\pgfpathlineto{\pgfqpoint{4.865403in}{2.034124in}}%
\pgfpathlineto{\pgfqpoint{4.862231in}{2.033939in}}%
\pgfpathlineto{\pgfqpoint{4.859059in}{2.033611in}}%
\pgfpathlineto{\pgfqpoint{4.855887in}{2.033674in}}%
\pgfpathlineto{\pgfqpoint{4.852715in}{2.033709in}}%
\pgfpathlineto{\pgfqpoint{4.849542in}{2.034027in}}%
\pgfpathlineto{\pgfqpoint{4.846370in}{2.033461in}}%
\pgfpathlineto{\pgfqpoint{4.843198in}{2.033386in}}%
\pgfpathlineto{\pgfqpoint{4.840026in}{2.033147in}}%
\pgfpathlineto{\pgfqpoint{4.836854in}{2.033122in}}%
\pgfpathlineto{\pgfqpoint{4.833682in}{2.033098in}}%
\pgfpathlineto{\pgfqpoint{4.830510in}{2.033202in}}%
\pgfpathlineto{\pgfqpoint{4.827338in}{2.032754in}}%
\pgfpathlineto{\pgfqpoint{4.824166in}{2.032607in}}%
\pgfpathlineto{\pgfqpoint{4.820994in}{2.032810in}}%
\pgfpathlineto{\pgfqpoint{4.817822in}{2.032669in}}%
\pgfpathlineto{\pgfqpoint{4.814650in}{2.032408in}}%
\pgfpathlineto{\pgfqpoint{4.811478in}{2.032005in}}%
\pgfpathlineto{\pgfqpoint{4.808306in}{2.032029in}}%
\pgfpathlineto{\pgfqpoint{4.805134in}{2.031870in}}%
\pgfpathlineto{\pgfqpoint{4.801962in}{2.032013in}}%
\pgfpathlineto{\pgfqpoint{4.798790in}{2.032000in}}%
\pgfpathlineto{\pgfqpoint{4.795618in}{2.031778in}}%
\pgfpathlineto{\pgfqpoint{4.792446in}{2.032072in}}%
\pgfpathlineto{\pgfqpoint{4.789274in}{2.032127in}}%
\pgfpathlineto{\pgfqpoint{4.786102in}{2.031897in}}%
\pgfpathlineto{\pgfqpoint{4.782930in}{2.031428in}}%
\pgfpathlineto{\pgfqpoint{4.779758in}{2.031257in}}%
\pgfpathlineto{\pgfqpoint{4.776586in}{2.031420in}}%
\pgfpathlineto{\pgfqpoint{4.773414in}{2.031776in}}%
\pgfpathlineto{\pgfqpoint{4.770241in}{2.031868in}}%
\pgfpathlineto{\pgfqpoint{4.767069in}{2.031820in}}%
\pgfpathlineto{\pgfqpoint{4.763897in}{2.031782in}}%
\pgfpathlineto{\pgfqpoint{4.760725in}{2.031610in}}%
\pgfpathlineto{\pgfqpoint{4.757553in}{2.031593in}}%
\pgfpathlineto{\pgfqpoint{4.754381in}{2.031221in}}%
\pgfpathlineto{\pgfqpoint{4.751209in}{2.030879in}}%
\pgfpathlineto{\pgfqpoint{4.748037in}{2.030782in}}%
\pgfpathlineto{\pgfqpoint{4.744865in}{2.030582in}}%
\pgfpathlineto{\pgfqpoint{4.741693in}{2.030167in}}%
\pgfpathlineto{\pgfqpoint{4.738521in}{2.030538in}}%
\pgfpathlineto{\pgfqpoint{4.735349in}{2.031003in}}%
\pgfpathlineto{\pgfqpoint{4.732177in}{2.031020in}}%
\pgfpathlineto{\pgfqpoint{4.729005in}{2.030931in}}%
\pgfpathlineto{\pgfqpoint{4.725833in}{2.031427in}}%
\pgfpathlineto{\pgfqpoint{4.722661in}{2.031068in}}%
\pgfpathlineto{\pgfqpoint{4.719489in}{2.030873in}}%
\pgfpathlineto{\pgfqpoint{4.716317in}{2.030975in}}%
\pgfpathlineto{\pgfqpoint{4.713145in}{2.031159in}}%
\pgfpathlineto{\pgfqpoint{4.709973in}{2.031271in}}%
\pgfpathlineto{\pgfqpoint{4.706801in}{2.031230in}}%
\pgfpathlineto{\pgfqpoint{4.703629in}{2.031584in}}%
\pgfpathlineto{\pgfqpoint{4.700457in}{2.031598in}}%
\pgfpathlineto{\pgfqpoint{4.697285in}{2.031452in}}%
\pgfpathlineto{\pgfqpoint{4.694112in}{2.031684in}}%
\pgfpathlineto{\pgfqpoint{4.690940in}{2.031959in}}%
\pgfpathlineto{\pgfqpoint{4.687768in}{2.032290in}}%
\pgfpathlineto{\pgfqpoint{4.684596in}{2.032302in}}%
\pgfpathlineto{\pgfqpoint{4.681424in}{2.032619in}}%
\pgfpathlineto{\pgfqpoint{4.678252in}{2.032527in}}%
\pgfpathlineto{\pgfqpoint{4.675080in}{2.032606in}}%
\pgfpathlineto{\pgfqpoint{4.671908in}{2.032585in}}%
\pgfpathlineto{\pgfqpoint{4.668736in}{2.032545in}}%
\pgfpathlineto{\pgfqpoint{4.665564in}{2.032936in}}%
\pgfpathlineto{\pgfqpoint{4.662392in}{2.033239in}}%
\pgfpathlineto{\pgfqpoint{4.659220in}{2.033116in}}%
\pgfpathlineto{\pgfqpoint{4.656048in}{2.032791in}}%
\pgfpathlineto{\pgfqpoint{4.652876in}{2.032719in}}%
\pgfpathlineto{\pgfqpoint{4.649704in}{2.032924in}}%
\pgfpathlineto{\pgfqpoint{4.646532in}{2.033482in}}%
\pgfpathlineto{\pgfqpoint{4.643360in}{2.033383in}}%
\pgfpathlineto{\pgfqpoint{4.640188in}{2.033210in}}%
\pgfpathlineto{\pgfqpoint{4.637016in}{2.032877in}}%
\pgfpathlineto{\pgfqpoint{4.633844in}{2.032930in}}%
\pgfpathlineto{\pgfqpoint{4.630672in}{2.033053in}}%
\pgfpathlineto{\pgfqpoint{4.627500in}{2.033199in}}%
\pgfpathlineto{\pgfqpoint{4.624328in}{2.033249in}}%
\pgfpathlineto{\pgfqpoint{4.621156in}{2.033125in}}%
\pgfpathlineto{\pgfqpoint{4.617984in}{2.033351in}}%
\pgfpathlineto{\pgfqpoint{4.614811in}{2.033152in}}%
\pgfpathlineto{\pgfqpoint{4.611639in}{2.032981in}}%
\pgfpathlineto{\pgfqpoint{4.608467in}{2.032616in}}%
\pgfpathlineto{\pgfqpoint{4.605295in}{2.032153in}}%
\pgfpathlineto{\pgfqpoint{4.602123in}{2.031526in}}%
\pgfpathlineto{\pgfqpoint{4.598951in}{2.031324in}}%
\pgfpathlineto{\pgfqpoint{4.595779in}{2.030871in}}%
\pgfpathlineto{\pgfqpoint{4.592607in}{2.030818in}}%
\pgfpathlineto{\pgfqpoint{4.589435in}{2.030720in}}%
\pgfpathlineto{\pgfqpoint{4.586263in}{2.030749in}}%
\pgfpathlineto{\pgfqpoint{4.583091in}{2.030609in}}%
\pgfpathlineto{\pgfqpoint{4.579919in}{2.030828in}}%
\pgfpathlineto{\pgfqpoint{4.576747in}{2.031117in}}%
\pgfpathlineto{\pgfqpoint{4.573575in}{2.030787in}}%
\pgfpathlineto{\pgfqpoint{4.570403in}{2.030738in}}%
\pgfpathlineto{\pgfqpoint{4.567231in}{2.030603in}}%
\pgfpathlineto{\pgfqpoint{4.564059in}{2.031044in}}%
\pgfpathlineto{\pgfqpoint{4.560887in}{2.030640in}}%
\pgfpathlineto{\pgfqpoint{4.557715in}{2.030299in}}%
\pgfpathlineto{\pgfqpoint{4.554543in}{2.030069in}}%
\pgfpathlineto{\pgfqpoint{4.551371in}{2.029878in}}%
\pgfpathlineto{\pgfqpoint{4.548199in}{2.029709in}}%
\pgfpathlineto{\pgfqpoint{4.545027in}{2.029704in}}%
\pgfpathlineto{\pgfqpoint{4.541855in}{2.029644in}}%
\pgfpathlineto{\pgfqpoint{4.538683in}{2.029635in}}%
\pgfpathlineto{\pgfqpoint{4.535510in}{2.029989in}}%
\pgfpathlineto{\pgfqpoint{4.532338in}{2.029996in}}%
\pgfpathlineto{\pgfqpoint{4.529166in}{2.029992in}}%
\pgfpathlineto{\pgfqpoint{4.525994in}{2.030239in}}%
\pgfpathlineto{\pgfqpoint{4.522822in}{2.030034in}}%
\pgfpathlineto{\pgfqpoint{4.519650in}{2.030151in}}%
\pgfpathlineto{\pgfqpoint{4.516478in}{2.029903in}}%
\pgfpathlineto{\pgfqpoint{4.513306in}{2.030049in}}%
\pgfpathlineto{\pgfqpoint{4.510134in}{2.029882in}}%
\pgfpathlineto{\pgfqpoint{4.506962in}{2.029564in}}%
\pgfpathlineto{\pgfqpoint{4.503790in}{2.029358in}}%
\pgfpathlineto{\pgfqpoint{4.500618in}{2.029482in}}%
\pgfpathlineto{\pgfqpoint{4.497446in}{2.029430in}}%
\pgfpathlineto{\pgfqpoint{4.494274in}{2.029459in}}%
\pgfpathlineto{\pgfqpoint{4.491102in}{2.029541in}}%
\pgfpathlineto{\pgfqpoint{4.487930in}{2.029353in}}%
\pgfpathlineto{\pgfqpoint{4.484758in}{2.029401in}}%
\pgfpathlineto{\pgfqpoint{4.481586in}{2.029420in}}%
\pgfpathlineto{\pgfqpoint{4.478414in}{2.029623in}}%
\pgfpathlineto{\pgfqpoint{4.475242in}{2.029496in}}%
\pgfpathlineto{\pgfqpoint{4.472070in}{2.029159in}}%
\pgfpathlineto{\pgfqpoint{4.468898in}{2.029058in}}%
\pgfpathlineto{\pgfqpoint{4.465726in}{2.028573in}}%
\pgfpathlineto{\pgfqpoint{4.462554in}{2.028035in}}%
\pgfpathlineto{\pgfqpoint{4.459381in}{2.027729in}}%
\pgfpathlineto{\pgfqpoint{4.456209in}{2.027830in}}%
\pgfpathlineto{\pgfqpoint{4.453037in}{2.027944in}}%
\pgfpathlineto{\pgfqpoint{4.449865in}{2.027723in}}%
\pgfpathlineto{\pgfqpoint{4.446693in}{2.027826in}}%
\pgfpathlineto{\pgfqpoint{4.443521in}{2.028052in}}%
\pgfpathlineto{\pgfqpoint{4.440349in}{2.027679in}}%
\pgfpathlineto{\pgfqpoint{4.437177in}{2.027576in}}%
\pgfpathlineto{\pgfqpoint{4.434005in}{2.027647in}}%
\pgfpathlineto{\pgfqpoint{4.430833in}{2.027830in}}%
\pgfpathlineto{\pgfqpoint{4.427661in}{2.028149in}}%
\pgfpathlineto{\pgfqpoint{4.424489in}{2.028215in}}%
\pgfpathlineto{\pgfqpoint{4.421317in}{2.028184in}}%
\pgfpathlineto{\pgfqpoint{4.418145in}{2.028474in}}%
\pgfpathlineto{\pgfqpoint{4.414973in}{2.028317in}}%
\pgfpathlineto{\pgfqpoint{4.411801in}{2.028133in}}%
\pgfpathlineto{\pgfqpoint{4.408629in}{2.028236in}}%
\pgfpathlineto{\pgfqpoint{4.405457in}{2.028173in}}%
\pgfpathlineto{\pgfqpoint{4.402285in}{2.027731in}}%
\pgfpathlineto{\pgfqpoint{4.399113in}{2.027822in}}%
\pgfpathlineto{\pgfqpoint{4.395941in}{2.027221in}}%
\pgfpathlineto{\pgfqpoint{4.392769in}{2.026933in}}%
\pgfpathlineto{\pgfqpoint{4.389597in}{2.026548in}}%
\pgfpathlineto{\pgfqpoint{4.386425in}{2.026350in}}%
\pgfpathlineto{\pgfqpoint{4.383253in}{2.026231in}}%
\pgfpathlineto{\pgfqpoint{4.380080in}{2.026119in}}%
\pgfpathlineto{\pgfqpoint{4.376908in}{2.025856in}}%
\pgfpathlineto{\pgfqpoint{4.373736in}{2.025666in}}%
\pgfpathlineto{\pgfqpoint{4.370564in}{2.025480in}}%
\pgfpathlineto{\pgfqpoint{4.367392in}{2.025666in}}%
\pgfpathlineto{\pgfqpoint{4.364220in}{2.025252in}}%
\pgfpathlineto{\pgfqpoint{4.361048in}{2.025046in}}%
\pgfpathlineto{\pgfqpoint{4.357876in}{2.024668in}}%
\pgfpathlineto{\pgfqpoint{4.354704in}{2.024357in}}%
\pgfpathlineto{\pgfqpoint{4.351532in}{2.024234in}}%
\pgfpathlineto{\pgfqpoint{4.348360in}{2.024249in}}%
\pgfpathlineto{\pgfqpoint{4.345188in}{2.024054in}}%
\pgfpathlineto{\pgfqpoint{4.342016in}{2.024636in}}%
\pgfpathlineto{\pgfqpoint{4.338844in}{2.024961in}}%
\pgfpathlineto{\pgfqpoint{4.335672in}{2.024829in}}%
\pgfpathlineto{\pgfqpoint{4.332500in}{2.024826in}}%
\pgfpathlineto{\pgfqpoint{4.329328in}{2.024138in}}%
\pgfpathlineto{\pgfqpoint{4.326156in}{2.024325in}}%
\pgfpathlineto{\pgfqpoint{4.322984in}{2.024513in}}%
\pgfpathlineto{\pgfqpoint{4.319812in}{2.024550in}}%
\pgfpathlineto{\pgfqpoint{4.316640in}{2.024555in}}%
\pgfpathlineto{\pgfqpoint{4.313468in}{2.024580in}}%
\pgfpathlineto{\pgfqpoint{4.310296in}{2.024351in}}%
\pgfpathlineto{\pgfqpoint{4.307124in}{2.024568in}}%
\pgfpathlineto{\pgfqpoint{4.303952in}{2.024564in}}%
\pgfpathlineto{\pgfqpoint{4.300779in}{2.024455in}}%
\pgfpathlineto{\pgfqpoint{4.297607in}{2.024332in}}%
\pgfpathlineto{\pgfqpoint{4.294435in}{2.024423in}}%
\pgfpathlineto{\pgfqpoint{4.291263in}{2.024780in}}%
\pgfpathlineto{\pgfqpoint{4.288091in}{2.025141in}}%
\pgfpathlineto{\pgfqpoint{4.284919in}{2.025011in}}%
\pgfpathlineto{\pgfqpoint{4.281747in}{2.024845in}}%
\pgfpathlineto{\pgfqpoint{4.278575in}{2.024675in}}%
\pgfpathlineto{\pgfqpoint{4.275403in}{2.024605in}}%
\pgfpathlineto{\pgfqpoint{4.272231in}{2.024616in}}%
\pgfpathlineto{\pgfqpoint{4.269059in}{2.024616in}}%
\pgfpathlineto{\pgfqpoint{4.265887in}{2.024723in}}%
\pgfpathlineto{\pgfqpoint{4.262715in}{2.024762in}}%
\pgfpathlineto{\pgfqpoint{4.259543in}{2.024934in}}%
\pgfpathlineto{\pgfqpoint{4.256371in}{2.025130in}}%
\pgfpathlineto{\pgfqpoint{4.253199in}{2.025228in}}%
\pgfpathlineto{\pgfqpoint{4.250027in}{2.025077in}}%
\pgfpathlineto{\pgfqpoint{4.246855in}{2.025065in}}%
\pgfpathlineto{\pgfqpoint{4.243683in}{2.024958in}}%
\pgfpathlineto{\pgfqpoint{4.240511in}{2.024855in}}%
\pgfpathlineto{\pgfqpoint{4.237339in}{2.024968in}}%
\pgfpathlineto{\pgfqpoint{4.234167in}{2.024601in}}%
\pgfpathlineto{\pgfqpoint{4.230995in}{2.024411in}}%
\pgfpathlineto{\pgfqpoint{4.227823in}{2.024457in}}%
\pgfpathlineto{\pgfqpoint{4.224650in}{2.024694in}}%
\pgfpathlineto{\pgfqpoint{4.221478in}{2.024667in}}%
\pgfpathlineto{\pgfqpoint{4.218306in}{2.024616in}}%
\pgfpathlineto{\pgfqpoint{4.215134in}{2.024728in}}%
\pgfpathlineto{\pgfqpoint{4.211962in}{2.024720in}}%
\pgfpathlineto{\pgfqpoint{4.208790in}{2.025069in}}%
\pgfpathlineto{\pgfqpoint{4.205618in}{2.024951in}}%
\pgfpathlineto{\pgfqpoint{4.202446in}{2.025110in}}%
\pgfpathlineto{\pgfqpoint{4.199274in}{2.024647in}}%
\pgfpathlineto{\pgfqpoint{4.196102in}{2.024275in}}%
\pgfpathlineto{\pgfqpoint{4.192930in}{2.023906in}}%
\pgfpathlineto{\pgfqpoint{4.189758in}{2.023634in}}%
\pgfpathlineto{\pgfqpoint{4.186586in}{2.023559in}}%
\pgfpathlineto{\pgfqpoint{4.183414in}{2.023493in}}%
\pgfpathlineto{\pgfqpoint{4.180242in}{2.023468in}}%
\pgfpathlineto{\pgfqpoint{4.177070in}{2.023271in}}%
\pgfpathlineto{\pgfqpoint{4.173898in}{2.023381in}}%
\pgfpathlineto{\pgfqpoint{4.170726in}{2.023444in}}%
\pgfpathlineto{\pgfqpoint{4.167554in}{2.023421in}}%
\pgfpathlineto{\pgfqpoint{4.164382in}{2.023203in}}%
\pgfpathlineto{\pgfqpoint{4.161210in}{2.023099in}}%
\pgfpathlineto{\pgfqpoint{4.158038in}{2.023403in}}%
\pgfpathlineto{\pgfqpoint{4.154866in}{2.023365in}}%
\pgfpathlineto{\pgfqpoint{4.151694in}{2.023347in}}%
\pgfpathlineto{\pgfqpoint{4.148522in}{2.023438in}}%
\pgfpathlineto{\pgfqpoint{4.145349in}{2.023370in}}%
\pgfpathlineto{\pgfqpoint{4.142177in}{2.022968in}}%
\pgfpathlineto{\pgfqpoint{4.139005in}{2.022265in}}%
\pgfpathlineto{\pgfqpoint{4.135833in}{2.022576in}}%
\pgfpathlineto{\pgfqpoint{4.132661in}{2.021822in}}%
\pgfpathlineto{\pgfqpoint{4.129489in}{2.021384in}}%
\pgfpathlineto{\pgfqpoint{4.126317in}{2.021158in}}%
\pgfpathlineto{\pgfqpoint{4.123145in}{2.021089in}}%
\pgfpathlineto{\pgfqpoint{4.119973in}{2.021056in}}%
\pgfpathlineto{\pgfqpoint{4.116801in}{2.021181in}}%
\pgfpathlineto{\pgfqpoint{4.113629in}{2.020914in}}%
\pgfpathlineto{\pgfqpoint{4.110457in}{2.020867in}}%
\pgfpathlineto{\pgfqpoint{4.107285in}{2.021091in}}%
\pgfpathlineto{\pgfqpoint{4.104113in}{2.021114in}}%
\pgfpathlineto{\pgfqpoint{4.100941in}{2.021168in}}%
\pgfpathlineto{\pgfqpoint{4.097769in}{2.021206in}}%
\pgfpathlineto{\pgfqpoint{4.094597in}{2.021304in}}%
\pgfpathlineto{\pgfqpoint{4.091425in}{2.020631in}}%
\pgfpathlineto{\pgfqpoint{4.088253in}{2.020313in}}%
\pgfpathlineto{\pgfqpoint{4.085081in}{2.020149in}}%
\pgfpathlineto{\pgfqpoint{4.081909in}{2.020304in}}%
\pgfpathlineto{\pgfqpoint{4.078737in}{2.020351in}}%
\pgfpathlineto{\pgfqpoint{4.075565in}{2.020803in}}%
\pgfpathlineto{\pgfqpoint{4.072393in}{2.020851in}}%
\pgfpathlineto{\pgfqpoint{4.069221in}{2.020695in}}%
\pgfpathlineto{\pgfqpoint{4.066048in}{2.020812in}}%
\pgfpathlineto{\pgfqpoint{4.062876in}{2.020484in}}%
\pgfpathlineto{\pgfqpoint{4.059704in}{2.020566in}}%
\pgfpathlineto{\pgfqpoint{4.056532in}{2.020264in}}%
\pgfpathlineto{\pgfqpoint{4.053360in}{2.020156in}}%
\pgfpathlineto{\pgfqpoint{4.050188in}{2.020638in}}%
\pgfpathlineto{\pgfqpoint{4.047016in}{2.020649in}}%
\pgfpathlineto{\pgfqpoint{4.043844in}{2.020763in}}%
\pgfpathlineto{\pgfqpoint{4.040672in}{2.020951in}}%
\pgfpathlineto{\pgfqpoint{4.037500in}{2.020841in}}%
\pgfpathlineto{\pgfqpoint{4.034328in}{2.020488in}}%
\pgfpathlineto{\pgfqpoint{4.031156in}{2.020475in}}%
\pgfpathlineto{\pgfqpoint{4.027984in}{2.020609in}}%
\pgfpathlineto{\pgfqpoint{4.024812in}{2.020246in}}%
\pgfpathlineto{\pgfqpoint{4.021640in}{2.020309in}}%
\pgfpathlineto{\pgfqpoint{4.018468in}{2.020385in}}%
\pgfpathlineto{\pgfqpoint{4.015296in}{2.020494in}}%
\pgfpathlineto{\pgfqpoint{4.012124in}{2.020276in}}%
\pgfpathlineto{\pgfqpoint{4.008952in}{2.020242in}}%
\pgfpathlineto{\pgfqpoint{4.005780in}{2.019992in}}%
\pgfpathlineto{\pgfqpoint{4.002608in}{2.019642in}}%
\pgfpathlineto{\pgfqpoint{3.999436in}{2.019370in}}%
\pgfpathlineto{\pgfqpoint{3.996264in}{2.019174in}}%
\pgfpathlineto{\pgfqpoint{3.993092in}{2.018989in}}%
\pgfpathlineto{\pgfqpoint{3.989919in}{2.018749in}}%
\pgfpathlineto{\pgfqpoint{3.986747in}{2.018957in}}%
\pgfpathlineto{\pgfqpoint{3.983575in}{2.019005in}}%
\pgfpathlineto{\pgfqpoint{3.980403in}{2.019019in}}%
\pgfpathlineto{\pgfqpoint{3.977231in}{2.018700in}}%
\pgfpathlineto{\pgfqpoint{3.974059in}{2.018421in}}%
\pgfpathlineto{\pgfqpoint{3.970887in}{2.018086in}}%
\pgfpathlineto{\pgfqpoint{3.967715in}{2.017639in}}%
\pgfpathlineto{\pgfqpoint{3.964543in}{2.017327in}}%
\pgfpathlineto{\pgfqpoint{3.961371in}{2.016754in}}%
\pgfpathlineto{\pgfqpoint{3.958199in}{2.016604in}}%
\pgfpathlineto{\pgfqpoint{3.955027in}{2.016466in}}%
\pgfpathlineto{\pgfqpoint{3.951855in}{2.017279in}}%
\pgfpathlineto{\pgfqpoint{3.948683in}{2.017329in}}%
\pgfpathlineto{\pgfqpoint{3.945511in}{2.017153in}}%
\pgfpathlineto{\pgfqpoint{3.942339in}{2.017296in}}%
\pgfpathlineto{\pgfqpoint{3.939167in}{2.017380in}}%
\pgfpathlineto{\pgfqpoint{3.935995in}{2.017532in}}%
\pgfpathlineto{\pgfqpoint{3.932823in}{2.017886in}}%
\pgfpathlineto{\pgfqpoint{3.929651in}{2.018091in}}%
\pgfpathlineto{\pgfqpoint{3.926479in}{2.018549in}}%
\pgfpathlineto{\pgfqpoint{3.923307in}{2.018587in}}%
\pgfpathlineto{\pgfqpoint{3.920135in}{2.018622in}}%
\pgfpathlineto{\pgfqpoint{3.916963in}{2.018648in}}%
\pgfpathlineto{\pgfqpoint{3.913791in}{2.019014in}}%
\pgfpathlineto{\pgfqpoint{3.910618in}{2.019195in}}%
\pgfpathlineto{\pgfqpoint{3.907446in}{2.019016in}}%
\pgfpathlineto{\pgfqpoint{3.904274in}{2.019212in}}%
\pgfpathlineto{\pgfqpoint{3.901102in}{2.018570in}}%
\pgfpathlineto{\pgfqpoint{3.897930in}{2.018714in}}%
\pgfpathlineto{\pgfqpoint{3.894758in}{2.018281in}}%
\pgfpathlineto{\pgfqpoint{3.891586in}{2.018239in}}%
\pgfpathlineto{\pgfqpoint{3.888414in}{2.017753in}}%
\pgfpathlineto{\pgfqpoint{3.885242in}{2.017792in}}%
\pgfpathlineto{\pgfqpoint{3.882070in}{2.018284in}}%
\pgfpathlineto{\pgfqpoint{3.878898in}{2.018258in}}%
\pgfpathlineto{\pgfqpoint{3.875726in}{2.018338in}}%
\pgfpathlineto{\pgfqpoint{3.872554in}{2.018428in}}%
\pgfpathlineto{\pgfqpoint{3.869382in}{2.018103in}}%
\pgfpathlineto{\pgfqpoint{3.866210in}{2.018211in}}%
\pgfpathlineto{\pgfqpoint{3.863038in}{2.018579in}}%
\pgfpathlineto{\pgfqpoint{3.859866in}{2.018842in}}%
\pgfpathlineto{\pgfqpoint{3.856694in}{2.018447in}}%
\pgfpathlineto{\pgfqpoint{3.853522in}{2.018317in}}%
\pgfpathlineto{\pgfqpoint{3.850350in}{2.018316in}}%
\pgfpathlineto{\pgfqpoint{3.847178in}{2.018566in}}%
\pgfpathlineto{\pgfqpoint{3.844006in}{2.018724in}}%
\pgfpathlineto{\pgfqpoint{3.840834in}{2.018805in}}%
\pgfpathlineto{\pgfqpoint{3.837662in}{2.019081in}}%
\pgfpathlineto{\pgfqpoint{3.834490in}{2.019107in}}%
\pgfpathlineto{\pgfqpoint{3.831317in}{2.018808in}}%
\pgfpathlineto{\pgfqpoint{3.828145in}{2.018515in}}%
\pgfpathlineto{\pgfqpoint{3.824973in}{2.018428in}}%
\pgfpathlineto{\pgfqpoint{3.821801in}{2.018410in}}%
\pgfpathlineto{\pgfqpoint{3.818629in}{2.018228in}}%
\pgfpathlineto{\pgfqpoint{3.815457in}{2.017959in}}%
\pgfpathlineto{\pgfqpoint{3.812285in}{2.017743in}}%
\pgfpathlineto{\pgfqpoint{3.809113in}{2.017571in}}%
\pgfpathlineto{\pgfqpoint{3.805941in}{2.017967in}}%
\pgfpathlineto{\pgfqpoint{3.802769in}{2.018326in}}%
\pgfpathlineto{\pgfqpoint{3.799597in}{2.018427in}}%
\pgfpathlineto{\pgfqpoint{3.796425in}{2.018248in}}%
\pgfpathlineto{\pgfqpoint{3.793253in}{2.017907in}}%
\pgfpathlineto{\pgfqpoint{3.790081in}{2.017615in}}%
\pgfpathlineto{\pgfqpoint{3.786909in}{2.016955in}}%
\pgfpathlineto{\pgfqpoint{3.783737in}{2.016885in}}%
\pgfpathlineto{\pgfqpoint{3.780565in}{2.016550in}}%
\pgfpathlineto{\pgfqpoint{3.777393in}{2.016653in}}%
\pgfpathlineto{\pgfqpoint{3.774221in}{2.016580in}}%
\pgfpathlineto{\pgfqpoint{3.771049in}{2.016380in}}%
\pgfpathlineto{\pgfqpoint{3.767877in}{2.016562in}}%
\pgfpathlineto{\pgfqpoint{3.764705in}{2.016961in}}%
\pgfpathlineto{\pgfqpoint{3.761533in}{2.016479in}}%
\pgfpathlineto{\pgfqpoint{3.758361in}{2.016466in}}%
\pgfpathlineto{\pgfqpoint{3.755188in}{2.016288in}}%
\pgfpathlineto{\pgfqpoint{3.752016in}{2.015952in}}%
\pgfpathlineto{\pgfqpoint{3.748844in}{2.015636in}}%
\pgfpathlineto{\pgfqpoint{3.745672in}{2.015721in}}%
\pgfpathlineto{\pgfqpoint{3.742500in}{2.016154in}}%
\pgfpathlineto{\pgfqpoint{3.739328in}{2.016317in}}%
\pgfpathlineto{\pgfqpoint{3.736156in}{2.015898in}}%
\pgfpathlineto{\pgfqpoint{3.732984in}{2.015781in}}%
\pgfpathlineto{\pgfqpoint{3.729812in}{2.015487in}}%
\pgfpathlineto{\pgfqpoint{3.726640in}{2.015242in}}%
\pgfpathlineto{\pgfqpoint{3.723468in}{2.014514in}}%
\pgfpathlineto{\pgfqpoint{3.720296in}{2.014347in}}%
\pgfpathlineto{\pgfqpoint{3.717124in}{2.014039in}}%
\pgfpathlineto{\pgfqpoint{3.713952in}{2.013591in}}%
\pgfpathlineto{\pgfqpoint{3.710780in}{2.013083in}}%
\pgfpathlineto{\pgfqpoint{3.707608in}{2.013140in}}%
\pgfpathlineto{\pgfqpoint{3.704436in}{2.012731in}}%
\pgfpathlineto{\pgfqpoint{3.701264in}{2.012407in}}%
\pgfpathlineto{\pgfqpoint{3.698092in}{2.012463in}}%
\pgfpathlineto{\pgfqpoint{3.694920in}{2.012653in}}%
\pgfpathlineto{\pgfqpoint{3.691748in}{2.012816in}}%
\pgfpathlineto{\pgfqpoint{3.688576in}{2.013222in}}%
\pgfpathlineto{\pgfqpoint{3.685404in}{2.013680in}}%
\pgfpathlineto{\pgfqpoint{3.682232in}{2.013691in}}%
\pgfpathlineto{\pgfqpoint{3.679060in}{2.013956in}}%
\pgfpathlineto{\pgfqpoint{3.675887in}{2.013777in}}%
\pgfpathlineto{\pgfqpoint{3.672715in}{2.013796in}}%
\pgfpathlineto{\pgfqpoint{3.669543in}{2.013388in}}%
\pgfpathlineto{\pgfqpoint{3.666371in}{2.014446in}}%
\pgfpathlineto{\pgfqpoint{3.663199in}{2.014434in}}%
\pgfpathlineto{\pgfqpoint{3.660027in}{2.014600in}}%
\pgfpathlineto{\pgfqpoint{3.656855in}{2.014770in}}%
\pgfpathlineto{\pgfqpoint{3.653683in}{2.015036in}}%
\pgfpathlineto{\pgfqpoint{3.650511in}{2.015491in}}%
\pgfpathlineto{\pgfqpoint{3.647339in}{2.015597in}}%
\pgfpathlineto{\pgfqpoint{3.644167in}{2.016030in}}%
\pgfpathlineto{\pgfqpoint{3.640995in}{2.015526in}}%
\pgfpathlineto{\pgfqpoint{3.637823in}{2.015916in}}%
\pgfpathlineto{\pgfqpoint{3.634651in}{2.016156in}}%
\pgfpathlineto{\pgfqpoint{3.631479in}{2.016265in}}%
\pgfpathlineto{\pgfqpoint{3.628307in}{2.016110in}}%
\pgfpathlineto{\pgfqpoint{3.625135in}{2.015634in}}%
\pgfpathlineto{\pgfqpoint{3.621963in}{2.015080in}}%
\pgfpathlineto{\pgfqpoint{3.618791in}{2.015155in}}%
\pgfpathlineto{\pgfqpoint{3.615619in}{2.014801in}}%
\pgfpathlineto{\pgfqpoint{3.612447in}{2.014741in}}%
\pgfpathlineto{\pgfqpoint{3.609275in}{2.014686in}}%
\pgfpathlineto{\pgfqpoint{3.606103in}{2.014591in}}%
\pgfpathlineto{\pgfqpoint{3.602931in}{2.014365in}}%
\pgfpathlineto{\pgfqpoint{3.599759in}{2.014326in}}%
\pgfpathlineto{\pgfqpoint{3.596586in}{2.014792in}}%
\pgfpathlineto{\pgfqpoint{3.593414in}{2.014949in}}%
\pgfpathlineto{\pgfqpoint{3.590242in}{2.015006in}}%
\pgfpathlineto{\pgfqpoint{3.587070in}{2.015127in}}%
\pgfpathlineto{\pgfqpoint{3.583898in}{2.015215in}}%
\pgfpathlineto{\pgfqpoint{3.580726in}{2.016084in}}%
\pgfpathlineto{\pgfqpoint{3.577554in}{2.016044in}}%
\pgfpathlineto{\pgfqpoint{3.574382in}{2.015757in}}%
\pgfpathlineto{\pgfqpoint{3.571210in}{2.015636in}}%
\pgfpathlineto{\pgfqpoint{3.568038in}{2.015388in}}%
\pgfpathlineto{\pgfqpoint{3.564866in}{2.015508in}}%
\pgfpathlineto{\pgfqpoint{3.561694in}{2.015634in}}%
\pgfpathlineto{\pgfqpoint{3.558522in}{2.015556in}}%
\pgfpathlineto{\pgfqpoint{3.555350in}{2.015116in}}%
\pgfpathlineto{\pgfqpoint{3.552178in}{2.014848in}}%
\pgfpathlineto{\pgfqpoint{3.549006in}{2.014628in}}%
\pgfpathlineto{\pgfqpoint{3.545834in}{2.014579in}}%
\pgfpathlineto{\pgfqpoint{3.542662in}{2.014272in}}%
\pgfpathlineto{\pgfqpoint{3.539490in}{2.014761in}}%
\pgfpathlineto{\pgfqpoint{3.536318in}{2.014828in}}%
\pgfpathlineto{\pgfqpoint{3.533146in}{2.015055in}}%
\pgfpathlineto{\pgfqpoint{3.529974in}{2.015141in}}%
\pgfpathlineto{\pgfqpoint{3.526802in}{2.015092in}}%
\pgfpathlineto{\pgfqpoint{3.523630in}{2.015207in}}%
\pgfpathlineto{\pgfqpoint{3.520457in}{2.015390in}}%
\pgfpathlineto{\pgfqpoint{3.517285in}{2.015607in}}%
\pgfpathlineto{\pgfqpoint{3.514113in}{2.015697in}}%
\pgfpathlineto{\pgfqpoint{3.510941in}{2.015771in}}%
\pgfpathlineto{\pgfqpoint{3.507769in}{2.015627in}}%
\pgfpathlineto{\pgfqpoint{3.504597in}{2.015682in}}%
\pgfpathlineto{\pgfqpoint{3.501425in}{2.015467in}}%
\pgfpathlineto{\pgfqpoint{3.498253in}{2.016070in}}%
\pgfpathlineto{\pgfqpoint{3.495081in}{2.015531in}}%
\pgfpathlineto{\pgfqpoint{3.491909in}{2.015594in}}%
\pgfpathlineto{\pgfqpoint{3.488737in}{2.015841in}}%
\pgfpathlineto{\pgfqpoint{3.485565in}{2.015643in}}%
\pgfpathlineto{\pgfqpoint{3.482393in}{2.014947in}}%
\pgfpathlineto{\pgfqpoint{3.479221in}{2.014485in}}%
\pgfpathlineto{\pgfqpoint{3.476049in}{2.014024in}}%
\pgfpathlineto{\pgfqpoint{3.472877in}{2.013679in}}%
\pgfpathlineto{\pgfqpoint{3.469705in}{2.013959in}}%
\pgfpathlineto{\pgfqpoint{3.466533in}{2.013455in}}%
\pgfpathlineto{\pgfqpoint{3.463361in}{2.013158in}}%
\pgfpathlineto{\pgfqpoint{3.460189in}{2.013427in}}%
\pgfpathlineto{\pgfqpoint{3.457017in}{2.013434in}}%
\pgfpathlineto{\pgfqpoint{3.453845in}{2.013394in}}%
\pgfpathlineto{\pgfqpoint{3.450673in}{2.013029in}}%
\pgfpathlineto{\pgfqpoint{3.447501in}{2.013043in}}%
\pgfpathlineto{\pgfqpoint{3.444329in}{2.013355in}}%
\pgfpathlineto{\pgfqpoint{3.441156in}{2.013407in}}%
\pgfpathlineto{\pgfqpoint{3.437984in}{2.013097in}}%
\pgfpathlineto{\pgfqpoint{3.434812in}{2.013152in}}%
\pgfpathlineto{\pgfqpoint{3.431640in}{2.013163in}}%
\pgfpathlineto{\pgfqpoint{3.428468in}{2.012939in}}%
\pgfpathlineto{\pgfqpoint{3.425296in}{2.013073in}}%
\pgfpathlineto{\pgfqpoint{3.422124in}{2.013049in}}%
\pgfpathlineto{\pgfqpoint{3.418952in}{2.013130in}}%
\pgfpathlineto{\pgfqpoint{3.415780in}{2.013196in}}%
\pgfpathlineto{\pgfqpoint{3.412608in}{2.013617in}}%
\pgfpathlineto{\pgfqpoint{3.409436in}{2.013501in}}%
\pgfpathlineto{\pgfqpoint{3.406264in}{2.013519in}}%
\pgfpathlineto{\pgfqpoint{3.403092in}{2.013225in}}%
\pgfpathlineto{\pgfqpoint{3.399920in}{2.013753in}}%
\pgfpathlineto{\pgfqpoint{3.396748in}{2.013418in}}%
\pgfpathlineto{\pgfqpoint{3.393576in}{2.013886in}}%
\pgfpathlineto{\pgfqpoint{3.390404in}{2.013591in}}%
\pgfpathlineto{\pgfqpoint{3.387232in}{2.013785in}}%
\pgfpathlineto{\pgfqpoint{3.384060in}{2.013072in}}%
\pgfpathlineto{\pgfqpoint{3.380888in}{2.012624in}}%
\pgfpathlineto{\pgfqpoint{3.377716in}{2.012264in}}%
\pgfpathlineto{\pgfqpoint{3.374544in}{2.012337in}}%
\pgfpathlineto{\pgfqpoint{3.371372in}{2.012105in}}%
\pgfpathlineto{\pgfqpoint{3.368200in}{2.012253in}}%
\pgfpathlineto{\pgfqpoint{3.365028in}{2.011920in}}%
\pgfpathlineto{\pgfqpoint{3.361855in}{2.011783in}}%
\pgfpathlineto{\pgfqpoint{3.358683in}{2.011629in}}%
\pgfpathlineto{\pgfqpoint{3.355511in}{2.011373in}}%
\pgfpathlineto{\pgfqpoint{3.352339in}{2.011620in}}%
\pgfpathlineto{\pgfqpoint{3.349167in}{2.011731in}}%
\pgfpathlineto{\pgfqpoint{3.345995in}{2.011629in}}%
\pgfpathlineto{\pgfqpoint{3.342823in}{2.011330in}}%
\pgfpathlineto{\pgfqpoint{3.339651in}{2.011093in}}%
\pgfpathlineto{\pgfqpoint{3.336479in}{2.011018in}}%
\pgfpathlineto{\pgfqpoint{3.333307in}{2.011168in}}%
\pgfpathlineto{\pgfqpoint{3.330135in}{2.010895in}}%
\pgfpathlineto{\pgfqpoint{3.326963in}{2.010484in}}%
\pgfpathlineto{\pgfqpoint{3.323791in}{2.010047in}}%
\pgfpathlineto{\pgfqpoint{3.320619in}{2.009559in}}%
\pgfpathlineto{\pgfqpoint{3.317447in}{2.009108in}}%
\pgfpathlineto{\pgfqpoint{3.314275in}{2.008862in}}%
\pgfpathlineto{\pgfqpoint{3.311103in}{2.008538in}}%
\pgfpathlineto{\pgfqpoint{3.307931in}{2.008341in}}%
\pgfpathlineto{\pgfqpoint{3.304759in}{2.007944in}}%
\pgfpathlineto{\pgfqpoint{3.301587in}{2.008085in}}%
\pgfpathlineto{\pgfqpoint{3.298415in}{2.007425in}}%
\pgfpathlineto{\pgfqpoint{3.295243in}{2.008683in}}%
\pgfpathlineto{\pgfqpoint{3.292071in}{2.008824in}}%
\pgfpathlineto{\pgfqpoint{3.288899in}{2.008729in}}%
\pgfpathlineto{\pgfqpoint{3.285726in}{2.008555in}}%
\pgfpathlineto{\pgfqpoint{3.282554in}{2.008865in}}%
\pgfpathlineto{\pgfqpoint{3.279382in}{2.008259in}}%
\pgfpathlineto{\pgfqpoint{3.276210in}{2.007794in}}%
\pgfpathlineto{\pgfqpoint{3.273038in}{2.006893in}}%
\pgfpathlineto{\pgfqpoint{3.269866in}{2.007023in}}%
\pgfpathlineto{\pgfqpoint{3.266694in}{2.006823in}}%
\pgfpathlineto{\pgfqpoint{3.263522in}{2.006108in}}%
\pgfpathlineto{\pgfqpoint{3.260350in}{2.006395in}}%
\pgfpathlineto{\pgfqpoint{3.257178in}{2.006596in}}%
\pgfpathlineto{\pgfqpoint{3.254006in}{2.006581in}}%
\pgfpathlineto{\pgfqpoint{3.250834in}{2.006894in}}%
\pgfpathlineto{\pgfqpoint{3.247662in}{2.006981in}}%
\pgfpathlineto{\pgfqpoint{3.244490in}{2.007081in}}%
\pgfpathlineto{\pgfqpoint{3.241318in}{2.007575in}}%
\pgfpathlineto{\pgfqpoint{3.238146in}{2.007856in}}%
\pgfpathlineto{\pgfqpoint{3.234974in}{2.008620in}}%
\pgfpathlineto{\pgfqpoint{3.231802in}{2.008691in}}%
\pgfpathlineto{\pgfqpoint{3.228630in}{2.008045in}}%
\pgfpathlineto{\pgfqpoint{3.225458in}{2.008111in}}%
\pgfpathlineto{\pgfqpoint{3.222286in}{2.007361in}}%
\pgfpathlineto{\pgfqpoint{3.219114in}{2.007202in}}%
\pgfpathlineto{\pgfqpoint{3.215942in}{2.007155in}}%
\pgfpathlineto{\pgfqpoint{3.212770in}{2.007019in}}%
\pgfpathlineto{\pgfqpoint{3.209598in}{2.006309in}}%
\pgfpathlineto{\pgfqpoint{3.206425in}{2.007021in}}%
\pgfpathlineto{\pgfqpoint{3.203253in}{2.007220in}}%
\pgfpathlineto{\pgfqpoint{3.200081in}{2.007221in}}%
\pgfpathlineto{\pgfqpoint{3.196909in}{2.007140in}}%
\pgfpathlineto{\pgfqpoint{3.193737in}{2.007147in}}%
\pgfpathlineto{\pgfqpoint{3.190565in}{2.007322in}}%
\pgfpathlineto{\pgfqpoint{3.187393in}{2.006902in}}%
\pgfpathlineto{\pgfqpoint{3.184221in}{2.007209in}}%
\pgfpathlineto{\pgfqpoint{3.181049in}{2.006909in}}%
\pgfpathlineto{\pgfqpoint{3.177877in}{2.006523in}}%
\pgfpathlineto{\pgfqpoint{3.174705in}{2.006351in}}%
\pgfpathlineto{\pgfqpoint{3.171533in}{2.006190in}}%
\pgfpathlineto{\pgfqpoint{3.168361in}{2.005745in}}%
\pgfpathlineto{\pgfqpoint{3.165189in}{2.004989in}}%
\pgfpathlineto{\pgfqpoint{3.162017in}{2.005049in}}%
\pgfpathlineto{\pgfqpoint{3.158845in}{2.005064in}}%
\pgfpathlineto{\pgfqpoint{3.155673in}{2.004819in}}%
\pgfpathlineto{\pgfqpoint{3.152501in}{2.004812in}}%
\pgfpathlineto{\pgfqpoint{3.149329in}{2.002633in}}%
\pgfpathlineto{\pgfqpoint{3.146157in}{2.002539in}}%
\pgfpathlineto{\pgfqpoint{3.142985in}{2.002998in}}%
\pgfpathlineto{\pgfqpoint{3.139813in}{2.002805in}}%
\pgfpathlineto{\pgfqpoint{3.136641in}{2.002587in}}%
\pgfpathlineto{\pgfqpoint{3.133469in}{2.002623in}}%
\pgfpathlineto{\pgfqpoint{3.130297in}{2.002833in}}%
\pgfpathlineto{\pgfqpoint{3.127124in}{2.002840in}}%
\pgfpathlineto{\pgfqpoint{3.123952in}{2.002861in}}%
\pgfpathlineto{\pgfqpoint{3.120780in}{2.002989in}}%
\pgfpathlineto{\pgfqpoint{3.117608in}{2.003024in}}%
\pgfpathlineto{\pgfqpoint{3.114436in}{2.002948in}}%
\pgfpathlineto{\pgfqpoint{3.111264in}{2.002961in}}%
\pgfpathlineto{\pgfqpoint{3.108092in}{2.002962in}}%
\pgfpathlineto{\pgfqpoint{3.104920in}{2.002870in}}%
\pgfpathlineto{\pgfqpoint{3.101748in}{2.002809in}}%
\pgfpathlineto{\pgfqpoint{3.098576in}{2.002669in}}%
\pgfpathlineto{\pgfqpoint{3.095404in}{2.002616in}}%
\pgfpathlineto{\pgfqpoint{3.092232in}{2.002359in}}%
\pgfpathlineto{\pgfqpoint{3.089060in}{2.002281in}}%
\pgfpathlineto{\pgfqpoint{3.085888in}{2.002173in}}%
\pgfpathlineto{\pgfqpoint{3.082716in}{2.001956in}}%
\pgfpathlineto{\pgfqpoint{3.079544in}{2.001840in}}%
\pgfpathlineto{\pgfqpoint{3.076372in}{2.001891in}}%
\pgfpathlineto{\pgfqpoint{3.073200in}{2.001949in}}%
\pgfpathlineto{\pgfqpoint{3.070028in}{2.001563in}}%
\pgfpathlineto{\pgfqpoint{3.066856in}{2.001373in}}%
\pgfpathlineto{\pgfqpoint{3.063684in}{2.001483in}}%
\pgfpathlineto{\pgfqpoint{3.060512in}{2.001206in}}%
\pgfpathlineto{\pgfqpoint{3.057340in}{2.000797in}}%
\pgfpathlineto{\pgfqpoint{3.054168in}{2.000887in}}%
\pgfpathlineto{\pgfqpoint{3.050995in}{2.000767in}}%
\pgfpathlineto{\pgfqpoint{3.047823in}{2.000629in}}%
\pgfpathlineto{\pgfqpoint{3.044651in}{2.000733in}}%
\pgfpathlineto{\pgfqpoint{3.041479in}{2.000746in}}%
\pgfpathlineto{\pgfqpoint{3.038307in}{2.000850in}}%
\pgfpathlineto{\pgfqpoint{3.035135in}{2.000802in}}%
\pgfpathlineto{\pgfqpoint{3.031963in}{2.000791in}}%
\pgfpathlineto{\pgfqpoint{3.028791in}{2.000740in}}%
\pgfpathlineto{\pgfqpoint{3.025619in}{2.000684in}}%
\pgfpathlineto{\pgfqpoint{3.022447in}{2.000834in}}%
\pgfpathlineto{\pgfqpoint{3.019275in}{2.001002in}}%
\pgfpathlineto{\pgfqpoint{3.016103in}{2.001097in}}%
\pgfpathlineto{\pgfqpoint{3.012931in}{2.001013in}}%
\pgfpathlineto{\pgfqpoint{3.009759in}{2.001156in}}%
\pgfpathlineto{\pgfqpoint{3.006587in}{2.001152in}}%
\pgfpathlineto{\pgfqpoint{3.003415in}{2.001122in}}%
\pgfpathlineto{\pgfqpoint{3.000243in}{2.001071in}}%
\pgfpathlineto{\pgfqpoint{2.997071in}{2.000898in}}%
\pgfpathlineto{\pgfqpoint{2.993899in}{2.000855in}}%
\pgfpathlineto{\pgfqpoint{2.990727in}{2.000923in}}%
\pgfpathlineto{\pgfqpoint{2.987555in}{2.000531in}}%
\pgfpathlineto{\pgfqpoint{2.984383in}{2.000445in}}%
\pgfpathlineto{\pgfqpoint{2.981211in}{2.000341in}}%
\pgfpathlineto{\pgfqpoint{2.978039in}{2.000228in}}%
\pgfpathlineto{\pgfqpoint{2.974867in}{2.000114in}}%
\pgfpathlineto{\pgfqpoint{2.971694in}{1.999969in}}%
\pgfpathlineto{\pgfqpoint{2.968522in}{1.999754in}}%
\pgfpathlineto{\pgfqpoint{2.965350in}{1.999537in}}%
\pgfpathlineto{\pgfqpoint{2.962178in}{1.999677in}}%
\pgfpathlineto{\pgfqpoint{2.959006in}{1.999491in}}%
\pgfpathlineto{\pgfqpoint{2.955834in}{1.998804in}}%
\pgfpathlineto{\pgfqpoint{2.952662in}{1.998681in}}%
\pgfpathlineto{\pgfqpoint{2.949490in}{1.998703in}}%
\pgfpathlineto{\pgfqpoint{2.946318in}{1.998758in}}%
\pgfpathlineto{\pgfqpoint{2.943146in}{1.998665in}}%
\pgfpathlineto{\pgfqpoint{2.939974in}{1.998615in}}%
\pgfpathlineto{\pgfqpoint{2.936802in}{1.998702in}}%
\pgfpathlineto{\pgfqpoint{2.933630in}{1.998886in}}%
\pgfpathlineto{\pgfqpoint{2.930458in}{1.998961in}}%
\pgfpathlineto{\pgfqpoint{2.927286in}{1.999123in}}%
\pgfpathlineto{\pgfqpoint{2.924114in}{1.999005in}}%
\pgfpathlineto{\pgfqpoint{2.920942in}{1.998975in}}%
\pgfpathlineto{\pgfqpoint{2.917770in}{1.998929in}}%
\pgfpathlineto{\pgfqpoint{2.914598in}{1.998845in}}%
\pgfpathlineto{\pgfqpoint{2.911426in}{1.998847in}}%
\pgfpathlineto{\pgfqpoint{2.908254in}{1.998850in}}%
\pgfpathlineto{\pgfqpoint{2.905082in}{1.998718in}}%
\pgfpathlineto{\pgfqpoint{2.901910in}{1.998921in}}%
\pgfpathlineto{\pgfqpoint{2.898738in}{1.998795in}}%
\pgfpathlineto{\pgfqpoint{2.895565in}{1.998743in}}%
\pgfpathlineto{\pgfqpoint{2.892393in}{1.998668in}}%
\pgfpathlineto{\pgfqpoint{2.889221in}{1.998639in}}%
\pgfpathlineto{\pgfqpoint{2.886049in}{1.998232in}}%
\pgfpathlineto{\pgfqpoint{2.882877in}{1.998401in}}%
\pgfpathlineto{\pgfqpoint{2.879705in}{1.998455in}}%
\pgfpathlineto{\pgfqpoint{2.876533in}{1.998317in}}%
\pgfpathlineto{\pgfqpoint{2.873361in}{1.998122in}}%
\pgfpathlineto{\pgfqpoint{2.870189in}{1.998040in}}%
\pgfpathlineto{\pgfqpoint{2.867017in}{1.997840in}}%
\pgfpathlineto{\pgfqpoint{2.863845in}{1.997936in}}%
\pgfpathlineto{\pgfqpoint{2.860673in}{1.997998in}}%
\pgfpathlineto{\pgfqpoint{2.857501in}{1.997623in}}%
\pgfpathlineto{\pgfqpoint{2.854329in}{1.997477in}}%
\pgfpathlineto{\pgfqpoint{2.851157in}{1.997478in}}%
\pgfpathlineto{\pgfqpoint{2.847985in}{1.997609in}}%
\pgfpathlineto{\pgfqpoint{2.844813in}{1.997630in}}%
\pgfpathlineto{\pgfqpoint{2.841641in}{1.997818in}}%
\pgfpathlineto{\pgfqpoint{2.838469in}{1.997649in}}%
\pgfpathlineto{\pgfqpoint{2.835297in}{1.997566in}}%
\pgfpathlineto{\pgfqpoint{2.832125in}{1.997580in}}%
\pgfpathlineto{\pgfqpoint{2.828953in}{1.997576in}}%
\pgfpathlineto{\pgfqpoint{2.825781in}{1.997611in}}%
\pgfpathlineto{\pgfqpoint{2.822609in}{1.997559in}}%
\pgfpathlineto{\pgfqpoint{2.819437in}{1.997622in}}%
\pgfpathlineto{\pgfqpoint{2.816264in}{1.997441in}}%
\pgfpathlineto{\pgfqpoint{2.813092in}{1.997447in}}%
\pgfpathlineto{\pgfqpoint{2.809920in}{1.997436in}}%
\pgfpathlineto{\pgfqpoint{2.806748in}{1.997451in}}%
\pgfpathlineto{\pgfqpoint{2.803576in}{1.997486in}}%
\pgfpathlineto{\pgfqpoint{2.800404in}{1.997549in}}%
\pgfpathlineto{\pgfqpoint{2.797232in}{1.997450in}}%
\pgfpathlineto{\pgfqpoint{2.794060in}{1.997537in}}%
\pgfpathlineto{\pgfqpoint{2.790888in}{1.997555in}}%
\pgfpathlineto{\pgfqpoint{2.787716in}{1.997619in}}%
\pgfpathlineto{\pgfqpoint{2.784544in}{1.997613in}}%
\pgfpathlineto{\pgfqpoint{2.781372in}{1.997640in}}%
\pgfpathlineto{\pgfqpoint{2.778200in}{1.997632in}}%
\pgfpathlineto{\pgfqpoint{2.775028in}{1.997894in}}%
\pgfpathlineto{\pgfqpoint{2.771856in}{1.997865in}}%
\pgfpathlineto{\pgfqpoint{2.768684in}{1.997692in}}%
\pgfpathlineto{\pgfqpoint{2.765512in}{1.997518in}}%
\pgfpathlineto{\pgfqpoint{2.762340in}{1.997525in}}%
\pgfpathlineto{\pgfqpoint{2.759168in}{1.997347in}}%
\pgfpathlineto{\pgfqpoint{2.755996in}{1.997356in}}%
\pgfpathlineto{\pgfqpoint{2.752824in}{1.997353in}}%
\pgfpathlineto{\pgfqpoint{2.749652in}{1.997319in}}%
\pgfpathlineto{\pgfqpoint{2.746480in}{1.997304in}}%
\pgfpathlineto{\pgfqpoint{2.743308in}{1.997311in}}%
\pgfpathlineto{\pgfqpoint{2.740136in}{1.997232in}}%
\pgfpathlineto{\pgfqpoint{2.736963in}{1.997045in}}%
\pgfpathlineto{\pgfqpoint{2.733791in}{1.996996in}}%
\pgfpathlineto{\pgfqpoint{2.730619in}{1.996845in}}%
\pgfpathlineto{\pgfqpoint{2.727447in}{1.996583in}}%
\pgfpathlineto{\pgfqpoint{2.724275in}{1.996544in}}%
\pgfpathlineto{\pgfqpoint{2.721103in}{1.996426in}}%
\pgfpathlineto{\pgfqpoint{2.717931in}{1.996568in}}%
\pgfpathlineto{\pgfqpoint{2.714759in}{1.996803in}}%
\pgfpathlineto{\pgfqpoint{2.711587in}{1.996983in}}%
\pgfpathlineto{\pgfqpoint{2.708415in}{1.997014in}}%
\pgfpathlineto{\pgfqpoint{2.705243in}{1.997008in}}%
\pgfpathlineto{\pgfqpoint{2.702071in}{1.997382in}}%
\pgfpathlineto{\pgfqpoint{2.698899in}{1.997386in}}%
\pgfpathlineto{\pgfqpoint{2.695727in}{1.997533in}}%
\pgfpathlineto{\pgfqpoint{2.692555in}{1.997573in}}%
\pgfpathlineto{\pgfqpoint{2.689383in}{1.997520in}}%
\pgfpathlineto{\pgfqpoint{2.686211in}{1.997402in}}%
\pgfpathlineto{\pgfqpoint{2.683039in}{1.997450in}}%
\pgfpathlineto{\pgfqpoint{2.679867in}{1.997447in}}%
\pgfpathlineto{\pgfqpoint{2.676695in}{1.997205in}}%
\pgfpathlineto{\pgfqpoint{2.673523in}{1.997083in}}%
\pgfpathlineto{\pgfqpoint{2.670351in}{1.996998in}}%
\pgfpathlineto{\pgfqpoint{2.667179in}{1.996955in}}%
\pgfpathlineto{\pgfqpoint{2.664007in}{1.996688in}}%
\pgfpathlineto{\pgfqpoint{2.660834in}{1.996611in}}%
\pgfpathlineto{\pgfqpoint{2.657662in}{1.996364in}}%
\pgfpathlineto{\pgfqpoint{2.654490in}{1.995976in}}%
\pgfpathlineto{\pgfqpoint{2.651318in}{1.995679in}}%
\pgfpathlineto{\pgfqpoint{2.648146in}{1.995867in}}%
\pgfpathlineto{\pgfqpoint{2.644974in}{1.995809in}}%
\pgfpathlineto{\pgfqpoint{2.641802in}{1.995993in}}%
\pgfpathlineto{\pgfqpoint{2.638630in}{1.996058in}}%
\pgfpathlineto{\pgfqpoint{2.635458in}{1.995880in}}%
\pgfpathlineto{\pgfqpoint{2.632286in}{1.995840in}}%
\pgfpathlineto{\pgfqpoint{2.629114in}{1.996029in}}%
\pgfpathlineto{\pgfqpoint{2.625942in}{1.996157in}}%
\pgfpathlineto{\pgfqpoint{2.622770in}{1.995905in}}%
\pgfpathlineto{\pgfqpoint{2.619598in}{1.995725in}}%
\pgfpathlineto{\pgfqpoint{2.616426in}{1.995610in}}%
\pgfpathlineto{\pgfqpoint{2.613254in}{1.995474in}}%
\pgfpathlineto{\pgfqpoint{2.610082in}{1.995451in}}%
\pgfpathlineto{\pgfqpoint{2.606910in}{1.995474in}}%
\pgfpathlineto{\pgfqpoint{2.603738in}{1.995616in}}%
\pgfpathlineto{\pgfqpoint{2.600566in}{1.995618in}}%
\pgfpathlineto{\pgfqpoint{2.597394in}{1.995526in}}%
\pgfpathlineto{\pgfqpoint{2.594222in}{1.995655in}}%
\pgfpathlineto{\pgfqpoint{2.591050in}{1.995608in}}%
\pgfpathlineto{\pgfqpoint{2.587878in}{1.995512in}}%
\pgfpathlineto{\pgfqpoint{2.584706in}{1.995570in}}%
\pgfpathlineto{\pgfqpoint{2.581533in}{1.995469in}}%
\pgfpathlineto{\pgfqpoint{2.578361in}{1.995330in}}%
\pgfpathlineto{\pgfqpoint{2.575189in}{1.995166in}}%
\pgfpathlineto{\pgfqpoint{2.572017in}{1.995188in}}%
\pgfpathlineto{\pgfqpoint{2.568845in}{1.995209in}}%
\pgfpathlineto{\pgfqpoint{2.565673in}{1.995121in}}%
\pgfpathlineto{\pgfqpoint{2.562501in}{1.995032in}}%
\pgfpathlineto{\pgfqpoint{2.559329in}{1.994957in}}%
\pgfpathlineto{\pgfqpoint{2.556157in}{1.994989in}}%
\pgfpathlineto{\pgfqpoint{2.552985in}{1.995003in}}%
\pgfpathlineto{\pgfqpoint{2.549813in}{1.995014in}}%
\pgfpathlineto{\pgfqpoint{2.546641in}{1.995063in}}%
\pgfpathlineto{\pgfqpoint{2.543469in}{1.995202in}}%
\pgfpathlineto{\pgfqpoint{2.540297in}{1.995037in}}%
\pgfpathlineto{\pgfqpoint{2.537125in}{1.995032in}}%
\pgfpathlineto{\pgfqpoint{2.533953in}{1.994783in}}%
\pgfpathlineto{\pgfqpoint{2.530781in}{1.994863in}}%
\pgfpathlineto{\pgfqpoint{2.527609in}{1.994846in}}%
\pgfpathlineto{\pgfqpoint{2.524437in}{1.994831in}}%
\pgfpathlineto{\pgfqpoint{2.521265in}{1.995078in}}%
\pgfpathlineto{\pgfqpoint{2.518093in}{1.994987in}}%
\pgfpathlineto{\pgfqpoint{2.514921in}{1.995017in}}%
\pgfpathlineto{\pgfqpoint{2.511749in}{1.994752in}}%
\pgfpathlineto{\pgfqpoint{2.508577in}{1.994666in}}%
\pgfpathlineto{\pgfqpoint{2.505405in}{1.994724in}}%
\pgfpathlineto{\pgfqpoint{2.502232in}{1.994670in}}%
\pgfpathlineto{\pgfqpoint{2.499060in}{1.994709in}}%
\pgfpathlineto{\pgfqpoint{2.495888in}{1.994761in}}%
\pgfpathlineto{\pgfqpoint{2.492716in}{1.994680in}}%
\pgfpathlineto{\pgfqpoint{2.489544in}{1.994826in}}%
\pgfpathlineto{\pgfqpoint{2.486372in}{1.994651in}}%
\pgfpathlineto{\pgfqpoint{2.483200in}{1.994834in}}%
\pgfpathlineto{\pgfqpoint{2.480028in}{1.994849in}}%
\pgfpathlineto{\pgfqpoint{2.476856in}{1.994947in}}%
\pgfpathlineto{\pgfqpoint{2.473684in}{1.994858in}}%
\pgfpathlineto{\pgfqpoint{2.470512in}{1.994766in}}%
\pgfpathlineto{\pgfqpoint{2.467340in}{1.994903in}}%
\pgfpathlineto{\pgfqpoint{2.464168in}{1.994661in}}%
\pgfpathlineto{\pgfqpoint{2.460996in}{1.994732in}}%
\pgfpathlineto{\pgfqpoint{2.457824in}{1.994760in}}%
\pgfpathlineto{\pgfqpoint{2.454652in}{1.994804in}}%
\pgfpathlineto{\pgfqpoint{2.451480in}{1.994760in}}%
\pgfpathlineto{\pgfqpoint{2.448308in}{1.994645in}}%
\pgfpathlineto{\pgfqpoint{2.445136in}{1.994691in}}%
\pgfpathlineto{\pgfqpoint{2.441964in}{1.994577in}}%
\pgfpathlineto{\pgfqpoint{2.438792in}{1.994121in}}%
\pgfpathlineto{\pgfqpoint{2.435620in}{1.994100in}}%
\pgfpathlineto{\pgfqpoint{2.432448in}{1.994032in}}%
\pgfpathlineto{\pgfqpoint{2.429276in}{1.993916in}}%
\pgfpathlineto{\pgfqpoint{2.426103in}{1.993823in}}%
\pgfpathlineto{\pgfqpoint{2.422931in}{1.993877in}}%
\pgfpathlineto{\pgfqpoint{2.419759in}{1.993774in}}%
\pgfpathlineto{\pgfqpoint{2.416587in}{1.993783in}}%
\pgfpathlineto{\pgfqpoint{2.413415in}{1.993645in}}%
\pgfpathlineto{\pgfqpoint{2.410243in}{1.993829in}}%
\pgfpathlineto{\pgfqpoint{2.407071in}{1.993791in}}%
\pgfpathlineto{\pgfqpoint{2.403899in}{1.993742in}}%
\pgfpathlineto{\pgfqpoint{2.400727in}{1.993712in}}%
\pgfpathlineto{\pgfqpoint{2.397555in}{1.993854in}}%
\pgfpathlineto{\pgfqpoint{2.394383in}{1.993733in}}%
\pgfpathlineto{\pgfqpoint{2.391211in}{1.993859in}}%
\pgfpathlineto{\pgfqpoint{2.388039in}{1.993824in}}%
\pgfpathlineto{\pgfqpoint{2.384867in}{1.993778in}}%
\pgfpathlineto{\pgfqpoint{2.381695in}{1.993701in}}%
\pgfpathlineto{\pgfqpoint{2.378523in}{1.993634in}}%
\pgfpathlineto{\pgfqpoint{2.375351in}{1.993424in}}%
\pgfpathlineto{\pgfqpoint{2.372179in}{1.993505in}}%
\pgfpathlineto{\pgfqpoint{2.369007in}{1.993545in}}%
\pgfpathlineto{\pgfqpoint{2.365835in}{1.993511in}}%
\pgfpathlineto{\pgfqpoint{2.362663in}{1.993408in}}%
\pgfpathlineto{\pgfqpoint{2.359491in}{1.993433in}}%
\pgfpathlineto{\pgfqpoint{2.356319in}{1.993401in}}%
\pgfpathlineto{\pgfqpoint{2.353147in}{1.993263in}}%
\pgfpathlineto{\pgfqpoint{2.349975in}{1.993338in}}%
\pgfpathlineto{\pgfqpoint{2.346802in}{1.993186in}}%
\pgfpathlineto{\pgfqpoint{2.343630in}{1.993182in}}%
\pgfpathlineto{\pgfqpoint{2.340458in}{1.992945in}}%
\pgfpathlineto{\pgfqpoint{2.337286in}{1.992732in}}%
\pgfpathlineto{\pgfqpoint{2.334114in}{1.992769in}}%
\pgfpathlineto{\pgfqpoint{2.330942in}{1.992850in}}%
\pgfpathlineto{\pgfqpoint{2.327770in}{1.992870in}}%
\pgfpathlineto{\pgfqpoint{2.324598in}{1.992811in}}%
\pgfpathlineto{\pgfqpoint{2.321426in}{1.992968in}}%
\pgfpathlineto{\pgfqpoint{2.318254in}{1.992937in}}%
\pgfpathlineto{\pgfqpoint{2.315082in}{1.992933in}}%
\pgfpathlineto{\pgfqpoint{2.311910in}{1.992811in}}%
\pgfpathlineto{\pgfqpoint{2.308738in}{1.992684in}}%
\pgfpathlineto{\pgfqpoint{2.305566in}{1.992633in}}%
\pgfpathlineto{\pgfqpoint{2.302394in}{1.992558in}}%
\pgfpathlineto{\pgfqpoint{2.299222in}{1.992922in}}%
\pgfpathlineto{\pgfqpoint{2.296050in}{1.992876in}}%
\pgfpathlineto{\pgfqpoint{2.292878in}{1.992647in}}%
\pgfpathlineto{\pgfqpoint{2.289706in}{1.992608in}}%
\pgfpathlineto{\pgfqpoint{2.286534in}{1.992643in}}%
\pgfpathlineto{\pgfqpoint{2.283362in}{1.992751in}}%
\pgfpathlineto{\pgfqpoint{2.280190in}{1.993010in}}%
\pgfpathlineto{\pgfqpoint{2.277018in}{1.992860in}}%
\pgfpathlineto{\pgfqpoint{2.273846in}{1.992760in}}%
\pgfpathlineto{\pgfqpoint{2.270674in}{1.992623in}}%
\pgfpathlineto{\pgfqpoint{2.267501in}{1.992724in}}%
\pgfpathlineto{\pgfqpoint{2.264329in}{1.992951in}}%
\pgfpathlineto{\pgfqpoint{2.261157in}{1.993007in}}%
\pgfpathlineto{\pgfqpoint{2.257985in}{1.993009in}}%
\pgfpathlineto{\pgfqpoint{2.254813in}{1.993186in}}%
\pgfpathlineto{\pgfqpoint{2.251641in}{1.993229in}}%
\pgfpathlineto{\pgfqpoint{2.248469in}{1.993234in}}%
\pgfpathlineto{\pgfqpoint{2.245297in}{1.993198in}}%
\pgfpathlineto{\pgfqpoint{2.242125in}{1.993261in}}%
\pgfpathlineto{\pgfqpoint{2.238953in}{1.993060in}}%
\pgfpathlineto{\pgfqpoint{2.235781in}{1.992962in}}%
\pgfpathlineto{\pgfqpoint{2.232609in}{1.992840in}}%
\pgfpathlineto{\pgfqpoint{2.229437in}{1.992866in}}%
\pgfpathlineto{\pgfqpoint{2.226265in}{1.992818in}}%
\pgfpathlineto{\pgfqpoint{2.223093in}{1.992731in}}%
\pgfpathlineto{\pgfqpoint{2.219921in}{1.992653in}}%
\pgfpathlineto{\pgfqpoint{2.216749in}{1.992493in}}%
\pgfpathlineto{\pgfqpoint{2.213577in}{1.992420in}}%
\pgfpathlineto{\pgfqpoint{2.210405in}{1.992480in}}%
\pgfpathlineto{\pgfqpoint{2.207233in}{1.992306in}}%
\pgfpathlineto{\pgfqpoint{2.204061in}{1.992200in}}%
\pgfpathlineto{\pgfqpoint{2.200889in}{1.992322in}}%
\pgfpathlineto{\pgfqpoint{2.197717in}{1.992353in}}%
\pgfpathlineto{\pgfqpoint{2.194545in}{1.992382in}}%
\pgfpathlineto{\pgfqpoint{2.191372in}{1.992516in}}%
\pgfpathlineto{\pgfqpoint{2.188200in}{1.992617in}}%
\pgfpathlineto{\pgfqpoint{2.185028in}{1.992380in}}%
\pgfpathlineto{\pgfqpoint{2.181856in}{1.992298in}}%
\pgfpathlineto{\pgfqpoint{2.178684in}{1.992264in}}%
\pgfpathlineto{\pgfqpoint{2.175512in}{1.992262in}}%
\pgfpathlineto{\pgfqpoint{2.172340in}{1.992141in}}%
\pgfpathlineto{\pgfqpoint{2.169168in}{1.992074in}}%
\pgfpathlineto{\pgfqpoint{2.165996in}{1.991988in}}%
\pgfpathlineto{\pgfqpoint{2.162824in}{1.991771in}}%
\pgfpathlineto{\pgfqpoint{2.159652in}{1.991684in}}%
\pgfpathlineto{\pgfqpoint{2.156480in}{1.991634in}}%
\pgfpathlineto{\pgfqpoint{2.153308in}{1.991454in}}%
\pgfpathlineto{\pgfqpoint{2.150136in}{1.980914in}}%
\pgfpathlineto{\pgfqpoint{2.146964in}{1.971230in}}%
\pgfpathlineto{\pgfqpoint{2.143792in}{1.962080in}}%
\pgfpathlineto{\pgfqpoint{2.140620in}{1.947708in}}%
\pgfpathlineto{\pgfqpoint{2.137448in}{1.934034in}}%
\pgfpathlineto{\pgfqpoint{2.134276in}{1.922594in}}%
\pgfpathlineto{\pgfqpoint{2.131104in}{1.910460in}}%
\pgfpathlineto{\pgfqpoint{2.127932in}{1.898910in}}%
\pgfpathlineto{\pgfqpoint{2.124760in}{1.886772in}}%
\pgfpathlineto{\pgfqpoint{2.121588in}{1.874761in}}%
\pgfpathlineto{\pgfqpoint{2.118416in}{1.862965in}}%
\pgfpathlineto{\pgfqpoint{2.115244in}{1.851015in}}%
\pgfpathlineto{\pgfqpoint{2.112071in}{1.839574in}}%
\pgfpathlineto{\pgfqpoint{2.108899in}{1.827803in}}%
\pgfpathlineto{\pgfqpoint{2.105727in}{1.815981in}}%
\pgfpathlineto{\pgfqpoint{2.102555in}{1.804113in}}%
\pgfpathlineto{\pgfqpoint{2.099383in}{1.792019in}}%
\pgfpathlineto{\pgfqpoint{2.096211in}{1.780540in}}%
\pgfpathlineto{\pgfqpoint{2.093039in}{1.768466in}}%
\pgfpathlineto{\pgfqpoint{2.089867in}{1.756597in}}%
\pgfpathlineto{\pgfqpoint{2.086695in}{1.744817in}}%
\pgfpathlineto{\pgfqpoint{2.083523in}{1.733174in}}%
\pgfpathlineto{\pgfqpoint{2.080351in}{1.721289in}}%
\pgfpathlineto{\pgfqpoint{2.077179in}{1.709556in}}%
\pgfpathlineto{\pgfqpoint{2.074007in}{1.697857in}}%
\pgfpathlineto{\pgfqpoint{2.070835in}{1.686144in}}%
\pgfpathlineto{\pgfqpoint{2.067663in}{1.674374in}}%
\pgfpathlineto{\pgfqpoint{2.064491in}{1.662493in}}%
\pgfpathlineto{\pgfqpoint{2.061319in}{1.650463in}}%
\pgfpathlineto{\pgfqpoint{2.058147in}{1.639008in}}%
\pgfpathlineto{\pgfqpoint{2.054975in}{1.627660in}}%
\pgfpathlineto{\pgfqpoint{2.051803in}{1.616093in}}%
\pgfpathlineto{\pgfqpoint{2.048631in}{1.604546in}}%
\pgfpathlineto{\pgfqpoint{2.045459in}{1.591618in}}%
\pgfpathlineto{\pgfqpoint{2.042287in}{1.582261in}}%
\pgfpathlineto{\pgfqpoint{2.039115in}{1.569588in}}%
\pgfpathlineto{\pgfqpoint{2.035943in}{1.557354in}}%
\pgfpathlineto{\pgfqpoint{2.032770in}{1.548407in}}%
\pgfpathlineto{\pgfqpoint{2.029598in}{1.536719in}}%
\pgfpathlineto{\pgfqpoint{2.026426in}{1.526183in}}%
\pgfpathlineto{\pgfqpoint{2.023254in}{1.516844in}}%
\pgfpathlineto{\pgfqpoint{2.020082in}{1.507191in}}%
\pgfpathlineto{\pgfqpoint{2.016910in}{1.496370in}}%
\pgfpathlineto{\pgfqpoint{2.013738in}{1.485581in}}%
\pgfpathlineto{\pgfqpoint{2.010566in}{1.475397in}}%
\pgfpathlineto{\pgfqpoint{2.007394in}{1.463579in}}%
\pgfpathlineto{\pgfqpoint{2.004222in}{1.452710in}}%
\pgfpathlineto{\pgfqpoint{2.001050in}{1.442179in}}%
\pgfpathlineto{\pgfqpoint{1.997878in}{1.431385in}}%
\pgfpathlineto{\pgfqpoint{1.994706in}{1.420071in}}%
\pgfpathlineto{\pgfqpoint{1.991534in}{1.408341in}}%
\pgfpathlineto{\pgfqpoint{1.988362in}{1.396175in}}%
\pgfpathlineto{\pgfqpoint{1.985190in}{1.384646in}}%
\pgfpathlineto{\pgfqpoint{1.982018in}{1.363492in}}%
\pgfpathlineto{\pgfqpoint{1.978846in}{1.344476in}}%
\pgfpathlineto{\pgfqpoint{1.975674in}{1.320781in}}%
\pgfpathlineto{\pgfqpoint{1.972502in}{1.299134in}}%
\pgfpathlineto{\pgfqpoint{1.969330in}{1.278952in}}%
\pgfpathlineto{\pgfqpoint{1.966158in}{1.260315in}}%
\pgfpathlineto{\pgfqpoint{1.962986in}{1.241522in}}%
\pgfpathlineto{\pgfqpoint{1.959814in}{1.221156in}}%
\pgfpathlineto{\pgfqpoint{1.956641in}{1.201604in}}%
\pgfpathlineto{\pgfqpoint{1.953469in}{1.179446in}}%
\pgfpathlineto{\pgfqpoint{1.950297in}{1.152458in}}%
\pgfpathlineto{\pgfqpoint{1.947125in}{1.131054in}}%
\pgfpathlineto{\pgfqpoint{1.943953in}{1.130166in}}%
\pgfpathlineto{\pgfqpoint{1.940781in}{1.127591in}}%
\pgfpathclose%
\pgfusepath{stroke,fill}%
\end{pgfscope}%
\begin{pgfscope}%
\pgfpathrectangle{\pgfqpoint{1.623736in}{1.000625in}}{\pgfqpoint{6.975000in}{3.020000in}} %
\pgfusepath{clip}%
\pgfsetbuttcap%
\pgfsetroundjoin%
\definecolor{currentfill}{rgb}{0.505882,0.447059,0.701961}%
\pgfsetfillcolor{currentfill}%
\pgfsetfillopacity{0.200000}%
\pgfsetlinewidth{0.803000pt}%
\definecolor{currentstroke}{rgb}{0.505882,0.447059,0.701961}%
\pgfsetstrokecolor{currentstroke}%
\pgfsetstrokeopacity{0.200000}%
\pgfsetdash{}{0pt}%
\pgfpathmoveto{\pgfqpoint{1.940781in}{1.127262in}}%
\pgfpathlineto{\pgfqpoint{1.940781in}{1.126699in}}%
\pgfpathlineto{\pgfqpoint{1.943953in}{1.126818in}}%
\pgfpathlineto{\pgfqpoint{1.947125in}{1.126818in}}%
\pgfpathlineto{\pgfqpoint{1.950297in}{1.143173in}}%
\pgfpathlineto{\pgfqpoint{1.953469in}{1.167186in}}%
\pgfpathlineto{\pgfqpoint{1.956641in}{1.184195in}}%
\pgfpathlineto{\pgfqpoint{1.959814in}{1.206146in}}%
\pgfpathlineto{\pgfqpoint{1.962986in}{1.224026in}}%
\pgfpathlineto{\pgfqpoint{1.966158in}{1.244464in}}%
\pgfpathlineto{\pgfqpoint{1.969330in}{1.264109in}}%
\pgfpathlineto{\pgfqpoint{1.972502in}{1.280577in}}%
\pgfpathlineto{\pgfqpoint{1.975674in}{1.302646in}}%
\pgfpathlineto{\pgfqpoint{1.978846in}{1.324520in}}%
\pgfpathlineto{\pgfqpoint{1.982018in}{1.343295in}}%
\pgfpathlineto{\pgfqpoint{1.985190in}{1.362724in}}%
\pgfpathlineto{\pgfqpoint{1.988362in}{1.362809in}}%
\pgfpathlineto{\pgfqpoint{1.991534in}{1.362663in}}%
\pgfpathlineto{\pgfqpoint{1.994706in}{1.362462in}}%
\pgfpathlineto{\pgfqpoint{1.997878in}{1.361522in}}%
\pgfpathlineto{\pgfqpoint{2.001050in}{1.361290in}}%
\pgfpathlineto{\pgfqpoint{2.004222in}{1.361123in}}%
\pgfpathlineto{\pgfqpoint{2.007394in}{1.360954in}}%
\pgfpathlineto{\pgfqpoint{2.010566in}{1.359740in}}%
\pgfpathlineto{\pgfqpoint{2.013738in}{1.358853in}}%
\pgfpathlineto{\pgfqpoint{2.016910in}{1.357783in}}%
\pgfpathlineto{\pgfqpoint{2.020082in}{1.357902in}}%
\pgfpathlineto{\pgfqpoint{2.023254in}{1.357511in}}%
\pgfpathlineto{\pgfqpoint{2.026426in}{1.355887in}}%
\pgfpathlineto{\pgfqpoint{2.029598in}{1.354789in}}%
\pgfpathlineto{\pgfqpoint{2.032770in}{1.355226in}}%
\pgfpathlineto{\pgfqpoint{2.035943in}{1.352555in}}%
\pgfpathlineto{\pgfqpoint{2.039115in}{1.351180in}}%
\pgfpathlineto{\pgfqpoint{2.042287in}{1.350762in}}%
\pgfpathlineto{\pgfqpoint{2.045459in}{1.349553in}}%
\pgfpathlineto{\pgfqpoint{2.048631in}{1.349756in}}%
\pgfpathlineto{\pgfqpoint{2.051803in}{1.349658in}}%
\pgfpathlineto{\pgfqpoint{2.054975in}{1.349708in}}%
\pgfpathlineto{\pgfqpoint{2.058147in}{1.349545in}}%
\pgfpathlineto{\pgfqpoint{2.061319in}{1.349468in}}%
\pgfpathlineto{\pgfqpoint{2.064491in}{1.349401in}}%
\pgfpathlineto{\pgfqpoint{2.067663in}{1.349492in}}%
\pgfpathlineto{\pgfqpoint{2.070835in}{1.349640in}}%
\pgfpathlineto{\pgfqpoint{2.074007in}{1.349494in}}%
\pgfpathlineto{\pgfqpoint{2.077179in}{1.349725in}}%
\pgfpathlineto{\pgfqpoint{2.080351in}{1.349679in}}%
\pgfpathlineto{\pgfqpoint{2.083523in}{1.349628in}}%
\pgfpathlineto{\pgfqpoint{2.086695in}{1.349770in}}%
\pgfpathlineto{\pgfqpoint{2.089867in}{1.349733in}}%
\pgfpathlineto{\pgfqpoint{2.093039in}{1.350029in}}%
\pgfpathlineto{\pgfqpoint{2.096211in}{1.350185in}}%
\pgfpathlineto{\pgfqpoint{2.099383in}{1.349983in}}%
\pgfpathlineto{\pgfqpoint{2.102555in}{1.350142in}}%
\pgfpathlineto{\pgfqpoint{2.105727in}{1.350323in}}%
\pgfpathlineto{\pgfqpoint{2.108899in}{1.350540in}}%
\pgfpathlineto{\pgfqpoint{2.112071in}{1.350779in}}%
\pgfpathlineto{\pgfqpoint{2.115244in}{1.350777in}}%
\pgfpathlineto{\pgfqpoint{2.118416in}{1.350945in}}%
\pgfpathlineto{\pgfqpoint{2.121588in}{1.350933in}}%
\pgfpathlineto{\pgfqpoint{2.124760in}{1.350969in}}%
\pgfpathlineto{\pgfqpoint{2.127932in}{1.351096in}}%
\pgfpathlineto{\pgfqpoint{2.131104in}{1.350987in}}%
\pgfpathlineto{\pgfqpoint{2.134276in}{1.350900in}}%
\pgfpathlineto{\pgfqpoint{2.137448in}{1.350973in}}%
\pgfpathlineto{\pgfqpoint{2.140620in}{1.351139in}}%
\pgfpathlineto{\pgfqpoint{2.143792in}{1.351351in}}%
\pgfpathlineto{\pgfqpoint{2.146964in}{1.351437in}}%
\pgfpathlineto{\pgfqpoint{2.150136in}{1.351355in}}%
\pgfpathlineto{\pgfqpoint{2.153308in}{1.351312in}}%
\pgfpathlineto{\pgfqpoint{2.156480in}{1.351482in}}%
\pgfpathlineto{\pgfqpoint{2.159652in}{1.351768in}}%
\pgfpathlineto{\pgfqpoint{2.162824in}{1.351753in}}%
\pgfpathlineto{\pgfqpoint{2.165996in}{1.351905in}}%
\pgfpathlineto{\pgfqpoint{2.169168in}{1.351784in}}%
\pgfpathlineto{\pgfqpoint{2.172340in}{1.351781in}}%
\pgfpathlineto{\pgfqpoint{2.175512in}{1.351735in}}%
\pgfpathlineto{\pgfqpoint{2.178684in}{1.351604in}}%
\pgfpathlineto{\pgfqpoint{2.181856in}{1.351612in}}%
\pgfpathlineto{\pgfqpoint{2.185028in}{1.351754in}}%
\pgfpathlineto{\pgfqpoint{2.188200in}{1.351882in}}%
\pgfpathlineto{\pgfqpoint{2.191372in}{1.351878in}}%
\pgfpathlineto{\pgfqpoint{2.194545in}{1.351990in}}%
\pgfpathlineto{\pgfqpoint{2.197717in}{1.351861in}}%
\pgfpathlineto{\pgfqpoint{2.200889in}{1.351686in}}%
\pgfpathlineto{\pgfqpoint{2.204061in}{1.351510in}}%
\pgfpathlineto{\pgfqpoint{2.207233in}{1.351652in}}%
\pgfpathlineto{\pgfqpoint{2.210405in}{1.351917in}}%
\pgfpathlineto{\pgfqpoint{2.213577in}{1.352004in}}%
\pgfpathlineto{\pgfqpoint{2.216749in}{1.352100in}}%
\pgfpathlineto{\pgfqpoint{2.219921in}{1.352175in}}%
\pgfpathlineto{\pgfqpoint{2.223093in}{1.352170in}}%
\pgfpathlineto{\pgfqpoint{2.226265in}{1.352187in}}%
\pgfpathlineto{\pgfqpoint{2.229437in}{1.352232in}}%
\pgfpathlineto{\pgfqpoint{2.232609in}{1.352122in}}%
\pgfpathlineto{\pgfqpoint{2.235781in}{1.352321in}}%
\pgfpathlineto{\pgfqpoint{2.238953in}{1.352339in}}%
\pgfpathlineto{\pgfqpoint{2.242125in}{1.352453in}}%
\pgfpathlineto{\pgfqpoint{2.245297in}{1.352501in}}%
\pgfpathlineto{\pgfqpoint{2.248469in}{1.352580in}}%
\pgfpathlineto{\pgfqpoint{2.251641in}{1.352630in}}%
\pgfpathlineto{\pgfqpoint{2.254813in}{1.352530in}}%
\pgfpathlineto{\pgfqpoint{2.257985in}{1.352564in}}%
\pgfpathlineto{\pgfqpoint{2.261157in}{1.352360in}}%
\pgfpathlineto{\pgfqpoint{2.264329in}{1.352461in}}%
\pgfpathlineto{\pgfqpoint{2.267501in}{1.352402in}}%
\pgfpathlineto{\pgfqpoint{2.270674in}{1.352275in}}%
\pgfpathlineto{\pgfqpoint{2.273846in}{1.352460in}}%
\pgfpathlineto{\pgfqpoint{2.277018in}{1.352397in}}%
\pgfpathlineto{\pgfqpoint{2.280190in}{1.352539in}}%
\pgfpathlineto{\pgfqpoint{2.283362in}{1.352534in}}%
\pgfpathlineto{\pgfqpoint{2.286534in}{1.352603in}}%
\pgfpathlineto{\pgfqpoint{2.289706in}{1.352639in}}%
\pgfpathlineto{\pgfqpoint{2.292878in}{1.352625in}}%
\pgfpathlineto{\pgfqpoint{2.296050in}{1.352722in}}%
\pgfpathlineto{\pgfqpoint{2.299222in}{1.352937in}}%
\pgfpathlineto{\pgfqpoint{2.302394in}{1.352585in}}%
\pgfpathlineto{\pgfqpoint{2.305566in}{1.352576in}}%
\pgfpathlineto{\pgfqpoint{2.308738in}{1.352683in}}%
\pgfpathlineto{\pgfqpoint{2.311910in}{1.352632in}}%
\pgfpathlineto{\pgfqpoint{2.315082in}{1.352761in}}%
\pgfpathlineto{\pgfqpoint{2.318254in}{1.352809in}}%
\pgfpathlineto{\pgfqpoint{2.321426in}{1.352860in}}%
\pgfpathlineto{\pgfqpoint{2.324598in}{1.353013in}}%
\pgfpathlineto{\pgfqpoint{2.327770in}{1.353091in}}%
\pgfpathlineto{\pgfqpoint{2.330942in}{1.353025in}}%
\pgfpathlineto{\pgfqpoint{2.334114in}{1.353260in}}%
\pgfpathlineto{\pgfqpoint{2.337286in}{1.353234in}}%
\pgfpathlineto{\pgfqpoint{2.340458in}{1.353365in}}%
\pgfpathlineto{\pgfqpoint{2.343630in}{1.353408in}}%
\pgfpathlineto{\pgfqpoint{2.346802in}{1.353498in}}%
\pgfpathlineto{\pgfqpoint{2.349975in}{1.353495in}}%
\pgfpathlineto{\pgfqpoint{2.353147in}{1.353449in}}%
\pgfpathlineto{\pgfqpoint{2.356319in}{1.353667in}}%
\pgfpathlineto{\pgfqpoint{2.359491in}{1.353921in}}%
\pgfpathlineto{\pgfqpoint{2.362663in}{1.353815in}}%
\pgfpathlineto{\pgfqpoint{2.365835in}{1.353912in}}%
\pgfpathlineto{\pgfqpoint{2.369007in}{1.353862in}}%
\pgfpathlineto{\pgfqpoint{2.372179in}{1.353761in}}%
\pgfpathlineto{\pgfqpoint{2.375351in}{1.353875in}}%
\pgfpathlineto{\pgfqpoint{2.378523in}{1.353940in}}%
\pgfpathlineto{\pgfqpoint{2.381695in}{1.353940in}}%
\pgfpathlineto{\pgfqpoint{2.384867in}{1.353935in}}%
\pgfpathlineto{\pgfqpoint{2.388039in}{1.354079in}}%
\pgfpathlineto{\pgfqpoint{2.391211in}{1.354143in}}%
\pgfpathlineto{\pgfqpoint{2.394383in}{1.354120in}}%
\pgfpathlineto{\pgfqpoint{2.397555in}{1.354027in}}%
\pgfpathlineto{\pgfqpoint{2.400727in}{1.354025in}}%
\pgfpathlineto{\pgfqpoint{2.403899in}{1.354214in}}%
\pgfpathlineto{\pgfqpoint{2.407071in}{1.354354in}}%
\pgfpathlineto{\pgfqpoint{2.410243in}{1.354073in}}%
\pgfpathlineto{\pgfqpoint{2.413415in}{1.353990in}}%
\pgfpathlineto{\pgfqpoint{2.416587in}{1.354109in}}%
\pgfpathlineto{\pgfqpoint{2.419759in}{1.354207in}}%
\pgfpathlineto{\pgfqpoint{2.422931in}{1.354335in}}%
\pgfpathlineto{\pgfqpoint{2.426103in}{1.354343in}}%
\pgfpathlineto{\pgfqpoint{2.429276in}{1.354399in}}%
\pgfpathlineto{\pgfqpoint{2.432448in}{1.354355in}}%
\pgfpathlineto{\pgfqpoint{2.435620in}{1.354234in}}%
\pgfpathlineto{\pgfqpoint{2.438792in}{1.354382in}}%
\pgfpathlineto{\pgfqpoint{2.441964in}{1.354426in}}%
\pgfpathlineto{\pgfqpoint{2.445136in}{1.354439in}}%
\pgfpathlineto{\pgfqpoint{2.448308in}{1.354441in}}%
\pgfpathlineto{\pgfqpoint{2.451480in}{1.354507in}}%
\pgfpathlineto{\pgfqpoint{2.454652in}{1.354529in}}%
\pgfpathlineto{\pgfqpoint{2.457824in}{1.354689in}}%
\pgfpathlineto{\pgfqpoint{2.460996in}{1.354737in}}%
\pgfpathlineto{\pgfqpoint{2.464168in}{1.354739in}}%
\pgfpathlineto{\pgfqpoint{2.467340in}{1.354811in}}%
\pgfpathlineto{\pgfqpoint{2.470512in}{1.355301in}}%
\pgfpathlineto{\pgfqpoint{2.473684in}{1.355468in}}%
\pgfpathlineto{\pgfqpoint{2.476856in}{1.355607in}}%
\pgfpathlineto{\pgfqpoint{2.480028in}{1.355650in}}%
\pgfpathlineto{\pgfqpoint{2.483200in}{1.355729in}}%
\pgfpathlineto{\pgfqpoint{2.486372in}{1.355658in}}%
\pgfpathlineto{\pgfqpoint{2.489544in}{1.355463in}}%
\pgfpathlineto{\pgfqpoint{2.492716in}{1.355306in}}%
\pgfpathlineto{\pgfqpoint{2.495888in}{1.355300in}}%
\pgfpathlineto{\pgfqpoint{2.499060in}{1.355131in}}%
\pgfpathlineto{\pgfqpoint{2.502232in}{1.355165in}}%
\pgfpathlineto{\pgfqpoint{2.505405in}{1.355274in}}%
\pgfpathlineto{\pgfqpoint{2.508577in}{1.355294in}}%
\pgfpathlineto{\pgfqpoint{2.511749in}{1.355335in}}%
\pgfpathlineto{\pgfqpoint{2.514921in}{1.355294in}}%
\pgfpathlineto{\pgfqpoint{2.518093in}{1.355339in}}%
\pgfpathlineto{\pgfqpoint{2.521265in}{1.355417in}}%
\pgfpathlineto{\pgfqpoint{2.524437in}{1.355389in}}%
\pgfpathlineto{\pgfqpoint{2.527609in}{1.355410in}}%
\pgfpathlineto{\pgfqpoint{2.530781in}{1.355387in}}%
\pgfpathlineto{\pgfqpoint{2.533953in}{1.355393in}}%
\pgfpathlineto{\pgfqpoint{2.537125in}{1.355226in}}%
\pgfpathlineto{\pgfqpoint{2.540297in}{1.355325in}}%
\pgfpathlineto{\pgfqpoint{2.543469in}{1.355573in}}%
\pgfpathlineto{\pgfqpoint{2.546641in}{1.355598in}}%
\pgfpathlineto{\pgfqpoint{2.549813in}{1.355641in}}%
\pgfpathlineto{\pgfqpoint{2.552985in}{1.355724in}}%
\pgfpathlineto{\pgfqpoint{2.556157in}{1.355702in}}%
\pgfpathlineto{\pgfqpoint{2.559329in}{1.355863in}}%
\pgfpathlineto{\pgfqpoint{2.562501in}{1.355942in}}%
\pgfpathlineto{\pgfqpoint{2.565673in}{1.355701in}}%
\pgfpathlineto{\pgfqpoint{2.568845in}{1.355634in}}%
\pgfpathlineto{\pgfqpoint{2.572017in}{1.355630in}}%
\pgfpathlineto{\pgfqpoint{2.575189in}{1.355769in}}%
\pgfpathlineto{\pgfqpoint{2.578361in}{1.355831in}}%
\pgfpathlineto{\pgfqpoint{2.581533in}{1.355758in}}%
\pgfpathlineto{\pgfqpoint{2.584706in}{1.355813in}}%
\pgfpathlineto{\pgfqpoint{2.587878in}{1.355801in}}%
\pgfpathlineto{\pgfqpoint{2.591050in}{1.355949in}}%
\pgfpathlineto{\pgfqpoint{2.594222in}{1.356124in}}%
\pgfpathlineto{\pgfqpoint{2.597394in}{1.356073in}}%
\pgfpathlineto{\pgfqpoint{2.600566in}{1.355941in}}%
\pgfpathlineto{\pgfqpoint{2.603738in}{1.355999in}}%
\pgfpathlineto{\pgfqpoint{2.606910in}{1.355963in}}%
\pgfpathlineto{\pgfqpoint{2.610082in}{1.355999in}}%
\pgfpathlineto{\pgfqpoint{2.613254in}{1.355993in}}%
\pgfpathlineto{\pgfqpoint{2.616426in}{1.355880in}}%
\pgfpathlineto{\pgfqpoint{2.619598in}{1.355805in}}%
\pgfpathlineto{\pgfqpoint{2.622770in}{1.355755in}}%
\pgfpathlineto{\pgfqpoint{2.625942in}{1.355996in}}%
\pgfpathlineto{\pgfqpoint{2.629114in}{1.355815in}}%
\pgfpathlineto{\pgfqpoint{2.632286in}{1.355457in}}%
\pgfpathlineto{\pgfqpoint{2.635458in}{1.355539in}}%
\pgfpathlineto{\pgfqpoint{2.638630in}{1.355691in}}%
\pgfpathlineto{\pgfqpoint{2.641802in}{1.355650in}}%
\pgfpathlineto{\pgfqpoint{2.644974in}{1.355488in}}%
\pgfpathlineto{\pgfqpoint{2.648146in}{1.355312in}}%
\pgfpathlineto{\pgfqpoint{2.651318in}{1.355259in}}%
\pgfpathlineto{\pgfqpoint{2.654490in}{1.355481in}}%
\pgfpathlineto{\pgfqpoint{2.657662in}{1.355920in}}%
\pgfpathlineto{\pgfqpoint{2.660834in}{1.356120in}}%
\pgfpathlineto{\pgfqpoint{2.664007in}{1.356074in}}%
\pgfpathlineto{\pgfqpoint{2.667179in}{1.356457in}}%
\pgfpathlineto{\pgfqpoint{2.670351in}{1.356401in}}%
\pgfpathlineto{\pgfqpoint{2.673523in}{1.356430in}}%
\pgfpathlineto{\pgfqpoint{2.676695in}{1.356380in}}%
\pgfpathlineto{\pgfqpoint{2.679867in}{1.356438in}}%
\pgfpathlineto{\pgfqpoint{2.683039in}{1.356613in}}%
\pgfpathlineto{\pgfqpoint{2.686211in}{1.356757in}}%
\pgfpathlineto{\pgfqpoint{2.689383in}{1.356812in}}%
\pgfpathlineto{\pgfqpoint{2.692555in}{1.356808in}}%
\pgfpathlineto{\pgfqpoint{2.695727in}{1.356923in}}%
\pgfpathlineto{\pgfqpoint{2.698899in}{1.357159in}}%
\pgfpathlineto{\pgfqpoint{2.702071in}{1.357131in}}%
\pgfpathlineto{\pgfqpoint{2.705243in}{1.356920in}}%
\pgfpathlineto{\pgfqpoint{2.708415in}{1.357351in}}%
\pgfpathlineto{\pgfqpoint{2.711587in}{1.357310in}}%
\pgfpathlineto{\pgfqpoint{2.714759in}{1.357219in}}%
\pgfpathlineto{\pgfqpoint{2.717931in}{1.357015in}}%
\pgfpathlineto{\pgfqpoint{2.721103in}{1.356827in}}%
\pgfpathlineto{\pgfqpoint{2.724275in}{1.356960in}}%
\pgfpathlineto{\pgfqpoint{2.727447in}{1.356985in}}%
\pgfpathlineto{\pgfqpoint{2.730619in}{1.357007in}}%
\pgfpathlineto{\pgfqpoint{2.733791in}{1.357068in}}%
\pgfpathlineto{\pgfqpoint{2.736963in}{1.357076in}}%
\pgfpathlineto{\pgfqpoint{2.740136in}{1.357029in}}%
\pgfpathlineto{\pgfqpoint{2.743308in}{1.356976in}}%
\pgfpathlineto{\pgfqpoint{2.746480in}{1.356954in}}%
\pgfpathlineto{\pgfqpoint{2.749652in}{1.356906in}}%
\pgfpathlineto{\pgfqpoint{2.752824in}{1.356927in}}%
\pgfpathlineto{\pgfqpoint{2.755996in}{1.356992in}}%
\pgfpathlineto{\pgfqpoint{2.759168in}{1.356744in}}%
\pgfpathlineto{\pgfqpoint{2.762340in}{1.356940in}}%
\pgfpathlineto{\pgfqpoint{2.765512in}{1.357145in}}%
\pgfpathlineto{\pgfqpoint{2.768684in}{1.357146in}}%
\pgfpathlineto{\pgfqpoint{2.771856in}{1.357103in}}%
\pgfpathlineto{\pgfqpoint{2.775028in}{1.357145in}}%
\pgfpathlineto{\pgfqpoint{2.778200in}{1.356762in}}%
\pgfpathlineto{\pgfqpoint{2.781372in}{1.356641in}}%
\pgfpathlineto{\pgfqpoint{2.784544in}{1.356647in}}%
\pgfpathlineto{\pgfqpoint{2.787716in}{1.356698in}}%
\pgfpathlineto{\pgfqpoint{2.790888in}{1.356665in}}%
\pgfpathlineto{\pgfqpoint{2.794060in}{1.356640in}}%
\pgfpathlineto{\pgfqpoint{2.797232in}{1.356552in}}%
\pgfpathlineto{\pgfqpoint{2.800404in}{1.356697in}}%
\pgfpathlineto{\pgfqpoint{2.803576in}{1.356522in}}%
\pgfpathlineto{\pgfqpoint{2.806748in}{1.356560in}}%
\pgfpathlineto{\pgfqpoint{2.809920in}{1.356635in}}%
\pgfpathlineto{\pgfqpoint{2.813092in}{1.356714in}}%
\pgfpathlineto{\pgfqpoint{2.816264in}{1.356815in}}%
\pgfpathlineto{\pgfqpoint{2.819437in}{1.356974in}}%
\pgfpathlineto{\pgfqpoint{2.822609in}{1.356992in}}%
\pgfpathlineto{\pgfqpoint{2.825781in}{1.357101in}}%
\pgfpathlineto{\pgfqpoint{2.828953in}{1.357058in}}%
\pgfpathlineto{\pgfqpoint{2.832125in}{1.356945in}}%
\pgfpathlineto{\pgfqpoint{2.835297in}{1.356616in}}%
\pgfpathlineto{\pgfqpoint{2.838469in}{1.356686in}}%
\pgfpathlineto{\pgfqpoint{2.841641in}{1.356818in}}%
\pgfpathlineto{\pgfqpoint{2.844813in}{1.356575in}}%
\pgfpathlineto{\pgfqpoint{2.847985in}{1.356587in}}%
\pgfpathlineto{\pgfqpoint{2.851157in}{1.356381in}}%
\pgfpathlineto{\pgfqpoint{2.854329in}{1.356458in}}%
\pgfpathlineto{\pgfqpoint{2.857501in}{1.356474in}}%
\pgfpathlineto{\pgfqpoint{2.860673in}{1.356763in}}%
\pgfpathlineto{\pgfqpoint{2.863845in}{1.357031in}}%
\pgfpathlineto{\pgfqpoint{2.867017in}{1.357108in}}%
\pgfpathlineto{\pgfqpoint{2.870189in}{1.357327in}}%
\pgfpathlineto{\pgfqpoint{2.873361in}{1.357379in}}%
\pgfpathlineto{\pgfqpoint{2.876533in}{1.357524in}}%
\pgfpathlineto{\pgfqpoint{2.879705in}{1.357661in}}%
\pgfpathlineto{\pgfqpoint{2.882877in}{1.357703in}}%
\pgfpathlineto{\pgfqpoint{2.886049in}{1.357496in}}%
\pgfpathlineto{\pgfqpoint{2.889221in}{1.357626in}}%
\pgfpathlineto{\pgfqpoint{2.892393in}{1.357678in}}%
\pgfpathlineto{\pgfqpoint{2.895565in}{1.357671in}}%
\pgfpathlineto{\pgfqpoint{2.898738in}{1.357710in}}%
\pgfpathlineto{\pgfqpoint{2.901910in}{1.357834in}}%
\pgfpathlineto{\pgfqpoint{2.905082in}{1.357636in}}%
\pgfpathlineto{\pgfqpoint{2.908254in}{1.357647in}}%
\pgfpathlineto{\pgfqpoint{2.911426in}{1.357665in}}%
\pgfpathlineto{\pgfqpoint{2.914598in}{1.357758in}}%
\pgfpathlineto{\pgfqpoint{2.917770in}{1.357843in}}%
\pgfpathlineto{\pgfqpoint{2.920942in}{1.358047in}}%
\pgfpathlineto{\pgfqpoint{2.924114in}{1.358162in}}%
\pgfpathlineto{\pgfqpoint{2.927286in}{1.358180in}}%
\pgfpathlineto{\pgfqpoint{2.930458in}{1.358205in}}%
\pgfpathlineto{\pgfqpoint{2.933630in}{1.358166in}}%
\pgfpathlineto{\pgfqpoint{2.936802in}{1.358084in}}%
\pgfpathlineto{\pgfqpoint{2.939974in}{1.358107in}}%
\pgfpathlineto{\pgfqpoint{2.943146in}{1.358039in}}%
\pgfpathlineto{\pgfqpoint{2.946318in}{1.358111in}}%
\pgfpathlineto{\pgfqpoint{2.949490in}{1.357851in}}%
\pgfpathlineto{\pgfqpoint{2.952662in}{1.357960in}}%
\pgfpathlineto{\pgfqpoint{2.955834in}{1.358040in}}%
\pgfpathlineto{\pgfqpoint{2.959006in}{1.358050in}}%
\pgfpathlineto{\pgfqpoint{2.962178in}{1.358205in}}%
\pgfpathlineto{\pgfqpoint{2.965350in}{1.358260in}}%
\pgfpathlineto{\pgfqpoint{2.968522in}{1.358218in}}%
\pgfpathlineto{\pgfqpoint{2.971694in}{1.358102in}}%
\pgfpathlineto{\pgfqpoint{2.974867in}{1.358175in}}%
\pgfpathlineto{\pgfqpoint{2.978039in}{1.358341in}}%
\pgfpathlineto{\pgfqpoint{2.981211in}{1.358438in}}%
\pgfpathlineto{\pgfqpoint{2.984383in}{1.358412in}}%
\pgfpathlineto{\pgfqpoint{2.987555in}{1.358714in}}%
\pgfpathlineto{\pgfqpoint{2.990727in}{1.358826in}}%
\pgfpathlineto{\pgfqpoint{2.993899in}{1.358949in}}%
\pgfpathlineto{\pgfqpoint{2.997071in}{1.359172in}}%
\pgfpathlineto{\pgfqpoint{3.000243in}{1.359145in}}%
\pgfpathlineto{\pgfqpoint{3.003415in}{1.359135in}}%
\pgfpathlineto{\pgfqpoint{3.006587in}{1.359153in}}%
\pgfpathlineto{\pgfqpoint{3.009759in}{1.359261in}}%
\pgfpathlineto{\pgfqpoint{3.012931in}{1.358994in}}%
\pgfpathlineto{\pgfqpoint{3.016103in}{1.359049in}}%
\pgfpathlineto{\pgfqpoint{3.019275in}{1.359110in}}%
\pgfpathlineto{\pgfqpoint{3.022447in}{1.359202in}}%
\pgfpathlineto{\pgfqpoint{3.025619in}{1.359089in}}%
\pgfpathlineto{\pgfqpoint{3.028791in}{1.359289in}}%
\pgfpathlineto{\pgfqpoint{3.031963in}{1.359250in}}%
\pgfpathlineto{\pgfqpoint{3.035135in}{1.359197in}}%
\pgfpathlineto{\pgfqpoint{3.038307in}{1.359259in}}%
\pgfpathlineto{\pgfqpoint{3.041479in}{1.359485in}}%
\pgfpathlineto{\pgfqpoint{3.044651in}{1.359465in}}%
\pgfpathlineto{\pgfqpoint{3.047823in}{1.359414in}}%
\pgfpathlineto{\pgfqpoint{3.050995in}{1.359588in}}%
\pgfpathlineto{\pgfqpoint{3.054168in}{1.359722in}}%
\pgfpathlineto{\pgfqpoint{3.057340in}{1.359706in}}%
\pgfpathlineto{\pgfqpoint{3.060512in}{1.359679in}}%
\pgfpathlineto{\pgfqpoint{3.063684in}{1.359698in}}%
\pgfpathlineto{\pgfqpoint{3.066856in}{1.359735in}}%
\pgfpathlineto{\pgfqpoint{3.070028in}{1.359611in}}%
\pgfpathlineto{\pgfqpoint{3.073200in}{1.359583in}}%
\pgfpathlineto{\pgfqpoint{3.076372in}{1.359703in}}%
\pgfpathlineto{\pgfqpoint{3.079544in}{1.359768in}}%
\pgfpathlineto{\pgfqpoint{3.082716in}{1.359801in}}%
\pgfpathlineto{\pgfqpoint{3.085888in}{1.359942in}}%
\pgfpathlineto{\pgfqpoint{3.089060in}{1.359820in}}%
\pgfpathlineto{\pgfqpoint{3.092232in}{1.359764in}}%
\pgfpathlineto{\pgfqpoint{3.095404in}{1.359956in}}%
\pgfpathlineto{\pgfqpoint{3.098576in}{1.360118in}}%
\pgfpathlineto{\pgfqpoint{3.101748in}{1.360232in}}%
\pgfpathlineto{\pgfqpoint{3.104920in}{1.360257in}}%
\pgfpathlineto{\pgfqpoint{3.108092in}{1.360353in}}%
\pgfpathlineto{\pgfqpoint{3.111264in}{1.360468in}}%
\pgfpathlineto{\pgfqpoint{3.114436in}{1.360560in}}%
\pgfpathlineto{\pgfqpoint{3.117608in}{1.360639in}}%
\pgfpathlineto{\pgfqpoint{3.120780in}{1.360560in}}%
\pgfpathlineto{\pgfqpoint{3.123952in}{1.360601in}}%
\pgfpathlineto{\pgfqpoint{3.127124in}{1.360704in}}%
\pgfpathlineto{\pgfqpoint{3.130297in}{1.360684in}}%
\pgfpathlineto{\pgfqpoint{3.133469in}{1.360602in}}%
\pgfpathlineto{\pgfqpoint{3.136641in}{1.360835in}}%
\pgfpathlineto{\pgfqpoint{3.139813in}{1.360909in}}%
\pgfpathlineto{\pgfqpoint{3.142985in}{1.361034in}}%
\pgfpathlineto{\pgfqpoint{3.146157in}{1.360339in}}%
\pgfpathlineto{\pgfqpoint{3.149329in}{1.360482in}}%
\pgfpathlineto{\pgfqpoint{3.152501in}{1.361376in}}%
\pgfpathlineto{\pgfqpoint{3.155673in}{1.361146in}}%
\pgfpathlineto{\pgfqpoint{3.158845in}{1.361193in}}%
\pgfpathlineto{\pgfqpoint{3.162017in}{1.361256in}}%
\pgfpathlineto{\pgfqpoint{3.165189in}{1.361219in}}%
\pgfpathlineto{\pgfqpoint{3.168361in}{1.361298in}}%
\pgfpathlineto{\pgfqpoint{3.171533in}{1.361377in}}%
\pgfpathlineto{\pgfqpoint{3.174705in}{1.361259in}}%
\pgfpathlineto{\pgfqpoint{3.177877in}{1.361142in}}%
\pgfpathlineto{\pgfqpoint{3.181049in}{1.361557in}}%
\pgfpathlineto{\pgfqpoint{3.184221in}{1.361395in}}%
\pgfpathlineto{\pgfqpoint{3.187393in}{1.361974in}}%
\pgfpathlineto{\pgfqpoint{3.190565in}{1.362103in}}%
\pgfpathlineto{\pgfqpoint{3.193737in}{1.363037in}}%
\pgfpathlineto{\pgfqpoint{3.196909in}{1.364036in}}%
\pgfpathlineto{\pgfqpoint{3.200081in}{1.365471in}}%
\pgfpathlineto{\pgfqpoint{3.203253in}{1.364890in}}%
\pgfpathlineto{\pgfqpoint{3.206425in}{1.364455in}}%
\pgfpathlineto{\pgfqpoint{3.209598in}{1.363540in}}%
\pgfpathlineto{\pgfqpoint{3.212770in}{1.363366in}}%
\pgfpathlineto{\pgfqpoint{3.215942in}{1.363686in}}%
\pgfpathlineto{\pgfqpoint{3.219114in}{1.363584in}}%
\pgfpathlineto{\pgfqpoint{3.222286in}{1.363792in}}%
\pgfpathlineto{\pgfqpoint{3.225458in}{1.364004in}}%
\pgfpathlineto{\pgfqpoint{3.228630in}{1.363480in}}%
\pgfpathlineto{\pgfqpoint{3.231802in}{1.363504in}}%
\pgfpathlineto{\pgfqpoint{3.234974in}{1.363204in}}%
\pgfpathlineto{\pgfqpoint{3.238146in}{1.362269in}}%
\pgfpathlineto{\pgfqpoint{3.241318in}{1.362862in}}%
\pgfpathlineto{\pgfqpoint{3.244490in}{1.361619in}}%
\pgfpathlineto{\pgfqpoint{3.247662in}{1.361799in}}%
\pgfpathlineto{\pgfqpoint{3.250834in}{1.361963in}}%
\pgfpathlineto{\pgfqpoint{3.254006in}{1.362085in}}%
\pgfpathlineto{\pgfqpoint{3.257178in}{1.362045in}}%
\pgfpathlineto{\pgfqpoint{3.260350in}{1.361617in}}%
\pgfpathlineto{\pgfqpoint{3.263522in}{1.361108in}}%
\pgfpathlineto{\pgfqpoint{3.266694in}{1.360734in}}%
\pgfpathlineto{\pgfqpoint{3.269866in}{1.360880in}}%
\pgfpathlineto{\pgfqpoint{3.273038in}{1.360666in}}%
\pgfpathlineto{\pgfqpoint{3.276210in}{1.361034in}}%
\pgfpathlineto{\pgfqpoint{3.279382in}{1.361313in}}%
\pgfpathlineto{\pgfqpoint{3.282554in}{1.361319in}}%
\pgfpathlineto{\pgfqpoint{3.285726in}{1.360989in}}%
\pgfpathlineto{\pgfqpoint{3.288899in}{1.361114in}}%
\pgfpathlineto{\pgfqpoint{3.292071in}{1.361273in}}%
\pgfpathlineto{\pgfqpoint{3.295243in}{1.360818in}}%
\pgfpathlineto{\pgfqpoint{3.298415in}{1.360226in}}%
\pgfpathlineto{\pgfqpoint{3.301587in}{1.359865in}}%
\pgfpathlineto{\pgfqpoint{3.304759in}{1.359438in}}%
\pgfpathlineto{\pgfqpoint{3.307931in}{1.359339in}}%
\pgfpathlineto{\pgfqpoint{3.311103in}{1.359174in}}%
\pgfpathlineto{\pgfqpoint{3.314275in}{1.359813in}}%
\pgfpathlineto{\pgfqpoint{3.317447in}{1.360108in}}%
\pgfpathlineto{\pgfqpoint{3.320619in}{1.360661in}}%
\pgfpathlineto{\pgfqpoint{3.323791in}{1.361092in}}%
\pgfpathlineto{\pgfqpoint{3.326963in}{1.360989in}}%
\pgfpathlineto{\pgfqpoint{3.330135in}{1.361184in}}%
\pgfpathlineto{\pgfqpoint{3.333307in}{1.361298in}}%
\pgfpathlineto{\pgfqpoint{3.336479in}{1.362014in}}%
\pgfpathlineto{\pgfqpoint{3.339651in}{1.362241in}}%
\pgfpathlineto{\pgfqpoint{3.342823in}{1.362946in}}%
\pgfpathlineto{\pgfqpoint{3.345995in}{1.363244in}}%
\pgfpathlineto{\pgfqpoint{3.349167in}{1.362866in}}%
\pgfpathlineto{\pgfqpoint{3.352339in}{1.362291in}}%
\pgfpathlineto{\pgfqpoint{3.355511in}{1.362282in}}%
\pgfpathlineto{\pgfqpoint{3.358683in}{1.362601in}}%
\pgfpathlineto{\pgfqpoint{3.361855in}{1.362670in}}%
\pgfpathlineto{\pgfqpoint{3.365028in}{1.362733in}}%
\pgfpathlineto{\pgfqpoint{3.368200in}{1.363898in}}%
\pgfpathlineto{\pgfqpoint{3.371372in}{1.363719in}}%
\pgfpathlineto{\pgfqpoint{3.374544in}{1.363884in}}%
\pgfpathlineto{\pgfqpoint{3.377716in}{1.363864in}}%
\pgfpathlineto{\pgfqpoint{3.380888in}{1.364563in}}%
\pgfpathlineto{\pgfqpoint{3.384060in}{1.364811in}}%
\pgfpathlineto{\pgfqpoint{3.387232in}{1.364979in}}%
\pgfpathlineto{\pgfqpoint{3.390404in}{1.364566in}}%
\pgfpathlineto{\pgfqpoint{3.393576in}{1.364753in}}%
\pgfpathlineto{\pgfqpoint{3.396748in}{1.364358in}}%
\pgfpathlineto{\pgfqpoint{3.399920in}{1.364666in}}%
\pgfpathlineto{\pgfqpoint{3.403092in}{1.363648in}}%
\pgfpathlineto{\pgfqpoint{3.406264in}{1.363925in}}%
\pgfpathlineto{\pgfqpoint{3.409436in}{1.364131in}}%
\pgfpathlineto{\pgfqpoint{3.412608in}{1.364023in}}%
\pgfpathlineto{\pgfqpoint{3.415780in}{1.363871in}}%
\pgfpathlineto{\pgfqpoint{3.418952in}{1.363872in}}%
\pgfpathlineto{\pgfqpoint{3.422124in}{1.364329in}}%
\pgfpathlineto{\pgfqpoint{3.425296in}{1.364438in}}%
\pgfpathlineto{\pgfqpoint{3.428468in}{1.363909in}}%
\pgfpathlineto{\pgfqpoint{3.431640in}{1.363989in}}%
\pgfpathlineto{\pgfqpoint{3.434812in}{1.363581in}}%
\pgfpathlineto{\pgfqpoint{3.437984in}{1.363479in}}%
\pgfpathlineto{\pgfqpoint{3.441156in}{1.363516in}}%
\pgfpathlineto{\pgfqpoint{3.444329in}{1.363403in}}%
\pgfpathlineto{\pgfqpoint{3.447501in}{1.363315in}}%
\pgfpathlineto{\pgfqpoint{3.450673in}{1.363109in}}%
\pgfpathlineto{\pgfqpoint{3.453845in}{1.363282in}}%
\pgfpathlineto{\pgfqpoint{3.457017in}{1.363113in}}%
\pgfpathlineto{\pgfqpoint{3.460189in}{1.363458in}}%
\pgfpathlineto{\pgfqpoint{3.463361in}{1.363541in}}%
\pgfpathlineto{\pgfqpoint{3.466533in}{1.363501in}}%
\pgfpathlineto{\pgfqpoint{3.469705in}{1.363358in}}%
\pgfpathlineto{\pgfqpoint{3.472877in}{1.363513in}}%
\pgfpathlineto{\pgfqpoint{3.476049in}{1.363361in}}%
\pgfpathlineto{\pgfqpoint{3.479221in}{1.363442in}}%
\pgfpathlineto{\pgfqpoint{3.482393in}{1.363345in}}%
\pgfpathlineto{\pgfqpoint{3.485565in}{1.363397in}}%
\pgfpathlineto{\pgfqpoint{3.488737in}{1.363441in}}%
\pgfpathlineto{\pgfqpoint{3.491909in}{1.363523in}}%
\pgfpathlineto{\pgfqpoint{3.495081in}{1.363455in}}%
\pgfpathlineto{\pgfqpoint{3.498253in}{1.363569in}}%
\pgfpathlineto{\pgfqpoint{3.501425in}{1.362763in}}%
\pgfpathlineto{\pgfqpoint{3.504597in}{1.363040in}}%
\pgfpathlineto{\pgfqpoint{3.507769in}{1.363629in}}%
\pgfpathlineto{\pgfqpoint{3.510941in}{1.363568in}}%
\pgfpathlineto{\pgfqpoint{3.514113in}{1.363304in}}%
\pgfpathlineto{\pgfqpoint{3.517285in}{1.363537in}}%
\pgfpathlineto{\pgfqpoint{3.520457in}{1.363055in}}%
\pgfpathlineto{\pgfqpoint{3.523630in}{1.363569in}}%
\pgfpathlineto{\pgfqpoint{3.526802in}{1.363692in}}%
\pgfpathlineto{\pgfqpoint{3.529974in}{1.363946in}}%
\pgfpathlineto{\pgfqpoint{3.533146in}{1.363869in}}%
\pgfpathlineto{\pgfqpoint{3.536318in}{1.363506in}}%
\pgfpathlineto{\pgfqpoint{3.539490in}{1.363851in}}%
\pgfpathlineto{\pgfqpoint{3.542662in}{1.363446in}}%
\pgfpathlineto{\pgfqpoint{3.545834in}{1.363576in}}%
\pgfpathlineto{\pgfqpoint{3.549006in}{1.363685in}}%
\pgfpathlineto{\pgfqpoint{3.552178in}{1.363808in}}%
\pgfpathlineto{\pgfqpoint{3.555350in}{1.364061in}}%
\pgfpathlineto{\pgfqpoint{3.558522in}{1.363950in}}%
\pgfpathlineto{\pgfqpoint{3.561694in}{1.363696in}}%
\pgfpathlineto{\pgfqpoint{3.564866in}{1.363490in}}%
\pgfpathlineto{\pgfqpoint{3.568038in}{1.363686in}}%
\pgfpathlineto{\pgfqpoint{3.571210in}{1.364081in}}%
\pgfpathlineto{\pgfqpoint{3.574382in}{1.364118in}}%
\pgfpathlineto{\pgfqpoint{3.577554in}{1.364601in}}%
\pgfpathlineto{\pgfqpoint{3.580726in}{1.364130in}}%
\pgfpathlineto{\pgfqpoint{3.583898in}{1.363514in}}%
\pgfpathlineto{\pgfqpoint{3.587070in}{1.363571in}}%
\pgfpathlineto{\pgfqpoint{3.590242in}{1.363967in}}%
\pgfpathlineto{\pgfqpoint{3.593414in}{1.363971in}}%
\pgfpathlineto{\pgfqpoint{3.596586in}{1.363966in}}%
\pgfpathlineto{\pgfqpoint{3.599759in}{1.363664in}}%
\pgfpathlineto{\pgfqpoint{3.602931in}{1.363680in}}%
\pgfpathlineto{\pgfqpoint{3.606103in}{1.363718in}}%
\pgfpathlineto{\pgfqpoint{3.609275in}{1.363392in}}%
\pgfpathlineto{\pgfqpoint{3.612447in}{1.363466in}}%
\pgfpathlineto{\pgfqpoint{3.615619in}{1.363411in}}%
\pgfpathlineto{\pgfqpoint{3.618791in}{1.363515in}}%
\pgfpathlineto{\pgfqpoint{3.621963in}{1.363674in}}%
\pgfpathlineto{\pgfqpoint{3.625135in}{1.364392in}}%
\pgfpathlineto{\pgfqpoint{3.628307in}{1.364484in}}%
\pgfpathlineto{\pgfqpoint{3.631479in}{1.364497in}}%
\pgfpathlineto{\pgfqpoint{3.634651in}{1.364741in}}%
\pgfpathlineto{\pgfqpoint{3.637823in}{1.365058in}}%
\pgfpathlineto{\pgfqpoint{3.640995in}{1.364593in}}%
\pgfpathlineto{\pgfqpoint{3.644167in}{1.364492in}}%
\pgfpathlineto{\pgfqpoint{3.647339in}{1.364173in}}%
\pgfpathlineto{\pgfqpoint{3.650511in}{1.364416in}}%
\pgfpathlineto{\pgfqpoint{3.653683in}{1.364176in}}%
\pgfpathlineto{\pgfqpoint{3.656855in}{1.364135in}}%
\pgfpathlineto{\pgfqpoint{3.660027in}{1.363674in}}%
\pgfpathlineto{\pgfqpoint{3.663199in}{1.363257in}}%
\pgfpathlineto{\pgfqpoint{3.666371in}{1.363328in}}%
\pgfpathlineto{\pgfqpoint{3.669543in}{1.363225in}}%
\pgfpathlineto{\pgfqpoint{3.672715in}{1.363605in}}%
\pgfpathlineto{\pgfqpoint{3.675887in}{1.363713in}}%
\pgfpathlineto{\pgfqpoint{3.679060in}{1.363704in}}%
\pgfpathlineto{\pgfqpoint{3.682232in}{1.363893in}}%
\pgfpathlineto{\pgfqpoint{3.685404in}{1.363734in}}%
\pgfpathlineto{\pgfqpoint{3.688576in}{1.363212in}}%
\pgfpathlineto{\pgfqpoint{3.691748in}{1.362794in}}%
\pgfpathlineto{\pgfqpoint{3.694920in}{1.363325in}}%
\pgfpathlineto{\pgfqpoint{3.698092in}{1.363002in}}%
\pgfpathlineto{\pgfqpoint{3.701264in}{1.362764in}}%
\pgfpathlineto{\pgfqpoint{3.704436in}{1.362728in}}%
\pgfpathlineto{\pgfqpoint{3.707608in}{1.362956in}}%
\pgfpathlineto{\pgfqpoint{3.710780in}{1.362707in}}%
\pgfpathlineto{\pgfqpoint{3.713952in}{1.362703in}}%
\pgfpathlineto{\pgfqpoint{3.717124in}{1.362935in}}%
\pgfpathlineto{\pgfqpoint{3.720296in}{1.363264in}}%
\pgfpathlineto{\pgfqpoint{3.723468in}{1.363401in}}%
\pgfpathlineto{\pgfqpoint{3.726640in}{1.363278in}}%
\pgfpathlineto{\pgfqpoint{3.729812in}{1.363336in}}%
\pgfpathlineto{\pgfqpoint{3.732984in}{1.363282in}}%
\pgfpathlineto{\pgfqpoint{3.736156in}{1.363039in}}%
\pgfpathlineto{\pgfqpoint{3.739328in}{1.363345in}}%
\pgfpathlineto{\pgfqpoint{3.742500in}{1.363282in}}%
\pgfpathlineto{\pgfqpoint{3.745672in}{1.362714in}}%
\pgfpathlineto{\pgfqpoint{3.748844in}{1.362651in}}%
\pgfpathlineto{\pgfqpoint{3.752016in}{1.362209in}}%
\pgfpathlineto{\pgfqpoint{3.755188in}{1.362141in}}%
\pgfpathlineto{\pgfqpoint{3.758361in}{1.361376in}}%
\pgfpathlineto{\pgfqpoint{3.761533in}{1.361125in}}%
\pgfpathlineto{\pgfqpoint{3.764705in}{1.361203in}}%
\pgfpathlineto{\pgfqpoint{3.767877in}{1.360926in}}%
\pgfpathlineto{\pgfqpoint{3.771049in}{1.360304in}}%
\pgfpathlineto{\pgfqpoint{3.774221in}{1.360212in}}%
\pgfpathlineto{\pgfqpoint{3.777393in}{1.360332in}}%
\pgfpathlineto{\pgfqpoint{3.780565in}{1.359998in}}%
\pgfpathlineto{\pgfqpoint{3.783737in}{1.359925in}}%
\pgfpathlineto{\pgfqpoint{3.786909in}{1.359603in}}%
\pgfpathlineto{\pgfqpoint{3.790081in}{1.359544in}}%
\pgfpathlineto{\pgfqpoint{3.793253in}{1.359541in}}%
\pgfpathlineto{\pgfqpoint{3.796425in}{1.359090in}}%
\pgfpathlineto{\pgfqpoint{3.799597in}{1.358976in}}%
\pgfpathlineto{\pgfqpoint{3.802769in}{1.358812in}}%
\pgfpathlineto{\pgfqpoint{3.805941in}{1.358290in}}%
\pgfpathlineto{\pgfqpoint{3.809113in}{1.358087in}}%
\pgfpathlineto{\pgfqpoint{3.812285in}{1.357908in}}%
\pgfpathlineto{\pgfqpoint{3.815457in}{1.357941in}}%
\pgfpathlineto{\pgfqpoint{3.818629in}{1.357917in}}%
\pgfpathlineto{\pgfqpoint{3.821801in}{1.358172in}}%
\pgfpathlineto{\pgfqpoint{3.824973in}{1.358161in}}%
\pgfpathlineto{\pgfqpoint{3.828145in}{1.357947in}}%
\pgfpathlineto{\pgfqpoint{3.831317in}{1.358172in}}%
\pgfpathlineto{\pgfqpoint{3.834490in}{1.358694in}}%
\pgfpathlineto{\pgfqpoint{3.837662in}{1.358500in}}%
\pgfpathlineto{\pgfqpoint{3.840834in}{1.358696in}}%
\pgfpathlineto{\pgfqpoint{3.844006in}{1.358485in}}%
\pgfpathlineto{\pgfqpoint{3.847178in}{1.358288in}}%
\pgfpathlineto{\pgfqpoint{3.850350in}{1.357944in}}%
\pgfpathlineto{\pgfqpoint{3.853522in}{1.357583in}}%
\pgfpathlineto{\pgfqpoint{3.856694in}{1.357626in}}%
\pgfpathlineto{\pgfqpoint{3.859866in}{1.357797in}}%
\pgfpathlineto{\pgfqpoint{3.863038in}{1.357805in}}%
\pgfpathlineto{\pgfqpoint{3.866210in}{1.357489in}}%
\pgfpathlineto{\pgfqpoint{3.869382in}{1.357865in}}%
\pgfpathlineto{\pgfqpoint{3.872554in}{1.358104in}}%
\pgfpathlineto{\pgfqpoint{3.875726in}{1.358433in}}%
\pgfpathlineto{\pgfqpoint{3.878898in}{1.358576in}}%
\pgfpathlineto{\pgfqpoint{3.882070in}{1.358384in}}%
\pgfpathlineto{\pgfqpoint{3.885242in}{1.357859in}}%
\pgfpathlineto{\pgfqpoint{3.888414in}{1.358437in}}%
\pgfpathlineto{\pgfqpoint{3.891586in}{1.358634in}}%
\pgfpathlineto{\pgfqpoint{3.894758in}{1.358520in}}%
\pgfpathlineto{\pgfqpoint{3.897930in}{1.358570in}}%
\pgfpathlineto{\pgfqpoint{3.901102in}{1.358617in}}%
\pgfpathlineto{\pgfqpoint{3.904274in}{1.359048in}}%
\pgfpathlineto{\pgfqpoint{3.907446in}{1.358946in}}%
\pgfpathlineto{\pgfqpoint{3.910618in}{1.359204in}}%
\pgfpathlineto{\pgfqpoint{3.913791in}{1.358770in}}%
\pgfpathlineto{\pgfqpoint{3.916963in}{1.358677in}}%
\pgfpathlineto{\pgfqpoint{3.920135in}{1.358392in}}%
\pgfpathlineto{\pgfqpoint{3.923307in}{1.358511in}}%
\pgfpathlineto{\pgfqpoint{3.926479in}{1.358608in}}%
\pgfpathlineto{\pgfqpoint{3.929651in}{1.358710in}}%
\pgfpathlineto{\pgfqpoint{3.932823in}{1.358712in}}%
\pgfpathlineto{\pgfqpoint{3.935995in}{1.358404in}}%
\pgfpathlineto{\pgfqpoint{3.939167in}{1.357809in}}%
\pgfpathlineto{\pgfqpoint{3.942339in}{1.357947in}}%
\pgfpathlineto{\pgfqpoint{3.945511in}{1.358122in}}%
\pgfpathlineto{\pgfqpoint{3.948683in}{1.358284in}}%
\pgfpathlineto{\pgfqpoint{3.951855in}{1.358190in}}%
\pgfpathlineto{\pgfqpoint{3.955027in}{1.357991in}}%
\pgfpathlineto{\pgfqpoint{3.958199in}{1.358197in}}%
\pgfpathlineto{\pgfqpoint{3.961371in}{1.357863in}}%
\pgfpathlineto{\pgfqpoint{3.964543in}{1.357893in}}%
\pgfpathlineto{\pgfqpoint{3.967715in}{1.357851in}}%
\pgfpathlineto{\pgfqpoint{3.970887in}{1.357684in}}%
\pgfpathlineto{\pgfqpoint{3.974059in}{1.357599in}}%
\pgfpathlineto{\pgfqpoint{3.977231in}{1.357236in}}%
\pgfpathlineto{\pgfqpoint{3.980403in}{1.357214in}}%
\pgfpathlineto{\pgfqpoint{3.983575in}{1.357188in}}%
\pgfpathlineto{\pgfqpoint{3.986747in}{1.357119in}}%
\pgfpathlineto{\pgfqpoint{3.989919in}{1.356839in}}%
\pgfpathlineto{\pgfqpoint{3.993092in}{1.357235in}}%
\pgfpathlineto{\pgfqpoint{3.996264in}{1.357248in}}%
\pgfpathlineto{\pgfqpoint{3.999436in}{1.357439in}}%
\pgfpathlineto{\pgfqpoint{4.002608in}{1.357671in}}%
\pgfpathlineto{\pgfqpoint{4.005780in}{1.357811in}}%
\pgfpathlineto{\pgfqpoint{4.008952in}{1.358023in}}%
\pgfpathlineto{\pgfqpoint{4.012124in}{1.358178in}}%
\pgfpathlineto{\pgfqpoint{4.015296in}{1.358003in}}%
\pgfpathlineto{\pgfqpoint{4.018468in}{1.358395in}}%
\pgfpathlineto{\pgfqpoint{4.021640in}{1.358475in}}%
\pgfpathlineto{\pgfqpoint{4.024812in}{1.358666in}}%
\pgfpathlineto{\pgfqpoint{4.027984in}{1.358986in}}%
\pgfpathlineto{\pgfqpoint{4.031156in}{1.359323in}}%
\pgfpathlineto{\pgfqpoint{4.034328in}{1.359181in}}%
\pgfpathlineto{\pgfqpoint{4.037500in}{1.359252in}}%
\pgfpathlineto{\pgfqpoint{4.040672in}{1.359315in}}%
\pgfpathlineto{\pgfqpoint{4.043844in}{1.359218in}}%
\pgfpathlineto{\pgfqpoint{4.047016in}{1.359590in}}%
\pgfpathlineto{\pgfqpoint{4.050188in}{1.359833in}}%
\pgfpathlineto{\pgfqpoint{4.053360in}{1.359264in}}%
\pgfpathlineto{\pgfqpoint{4.056532in}{1.359127in}}%
\pgfpathlineto{\pgfqpoint{4.059704in}{1.359055in}}%
\pgfpathlineto{\pgfqpoint{4.062876in}{1.359014in}}%
\pgfpathlineto{\pgfqpoint{4.066048in}{1.359117in}}%
\pgfpathlineto{\pgfqpoint{4.069221in}{1.359327in}}%
\pgfpathlineto{\pgfqpoint{4.072393in}{1.359481in}}%
\pgfpathlineto{\pgfqpoint{4.075565in}{1.359660in}}%
\pgfpathlineto{\pgfqpoint{4.078737in}{1.359928in}}%
\pgfpathlineto{\pgfqpoint{4.081909in}{1.360356in}}%
\pgfpathlineto{\pgfqpoint{4.085081in}{1.360528in}}%
\pgfpathlineto{\pgfqpoint{4.088253in}{1.360611in}}%
\pgfpathlineto{\pgfqpoint{4.091425in}{1.360171in}}%
\pgfpathlineto{\pgfqpoint{4.094597in}{1.360295in}}%
\pgfpathlineto{\pgfqpoint{4.097769in}{1.360432in}}%
\pgfpathlineto{\pgfqpoint{4.100941in}{1.360580in}}%
\pgfpathlineto{\pgfqpoint{4.104113in}{1.360246in}}%
\pgfpathlineto{\pgfqpoint{4.107285in}{1.360544in}}%
\pgfpathlineto{\pgfqpoint{4.110457in}{1.360551in}}%
\pgfpathlineto{\pgfqpoint{4.113629in}{1.361118in}}%
\pgfpathlineto{\pgfqpoint{4.116801in}{1.361534in}}%
\pgfpathlineto{\pgfqpoint{4.119973in}{1.362010in}}%
\pgfpathlineto{\pgfqpoint{4.123145in}{1.362524in}}%
\pgfpathlineto{\pgfqpoint{4.126317in}{1.362928in}}%
\pgfpathlineto{\pgfqpoint{4.129489in}{1.362911in}}%
\pgfpathlineto{\pgfqpoint{4.132661in}{1.362960in}}%
\pgfpathlineto{\pgfqpoint{4.135833in}{1.362406in}}%
\pgfpathlineto{\pgfqpoint{4.139005in}{1.362314in}}%
\pgfpathlineto{\pgfqpoint{4.142177in}{1.362574in}}%
\pgfpathlineto{\pgfqpoint{4.145349in}{1.362979in}}%
\pgfpathlineto{\pgfqpoint{4.148522in}{1.363076in}}%
\pgfpathlineto{\pgfqpoint{4.151694in}{1.363407in}}%
\pgfpathlineto{\pgfqpoint{4.154866in}{1.363214in}}%
\pgfpathlineto{\pgfqpoint{4.158038in}{1.363767in}}%
\pgfpathlineto{\pgfqpoint{4.161210in}{1.363861in}}%
\pgfpathlineto{\pgfqpoint{4.164382in}{1.364255in}}%
\pgfpathlineto{\pgfqpoint{4.167554in}{1.364876in}}%
\pgfpathlineto{\pgfqpoint{4.170726in}{1.364943in}}%
\pgfpathlineto{\pgfqpoint{4.173898in}{1.365062in}}%
\pgfpathlineto{\pgfqpoint{4.177070in}{1.365254in}}%
\pgfpathlineto{\pgfqpoint{4.180242in}{1.364972in}}%
\pgfpathlineto{\pgfqpoint{4.183414in}{1.364744in}}%
\pgfpathlineto{\pgfqpoint{4.186586in}{1.364428in}}%
\pgfpathlineto{\pgfqpoint{4.189758in}{1.364594in}}%
\pgfpathlineto{\pgfqpoint{4.192930in}{1.364569in}}%
\pgfpathlineto{\pgfqpoint{4.196102in}{1.364785in}}%
\pgfpathlineto{\pgfqpoint{4.199274in}{1.365047in}}%
\pgfpathlineto{\pgfqpoint{4.202446in}{1.365300in}}%
\pgfpathlineto{\pgfqpoint{4.205618in}{1.365038in}}%
\pgfpathlineto{\pgfqpoint{4.208790in}{1.364915in}}%
\pgfpathlineto{\pgfqpoint{4.211962in}{1.363795in}}%
\pgfpathlineto{\pgfqpoint{4.215134in}{1.363480in}}%
\pgfpathlineto{\pgfqpoint{4.218306in}{1.363288in}}%
\pgfpathlineto{\pgfqpoint{4.221478in}{1.363386in}}%
\pgfpathlineto{\pgfqpoint{4.224650in}{1.363395in}}%
\pgfpathlineto{\pgfqpoint{4.227823in}{1.363138in}}%
\pgfpathlineto{\pgfqpoint{4.230995in}{1.363505in}}%
\pgfpathlineto{\pgfqpoint{4.234167in}{1.363798in}}%
\pgfpathlineto{\pgfqpoint{4.237339in}{1.363954in}}%
\pgfpathlineto{\pgfqpoint{4.240511in}{1.363831in}}%
\pgfpathlineto{\pgfqpoint{4.243683in}{1.363842in}}%
\pgfpathlineto{\pgfqpoint{4.246855in}{1.363642in}}%
\pgfpathlineto{\pgfqpoint{4.250027in}{1.363659in}}%
\pgfpathlineto{\pgfqpoint{4.253199in}{1.363362in}}%
\pgfpathlineto{\pgfqpoint{4.256371in}{1.363822in}}%
\pgfpathlineto{\pgfqpoint{4.259543in}{1.364067in}}%
\pgfpathlineto{\pgfqpoint{4.262715in}{1.364217in}}%
\pgfpathlineto{\pgfqpoint{4.265887in}{1.364368in}}%
\pgfpathlineto{\pgfqpoint{4.269059in}{1.364720in}}%
\pgfpathlineto{\pgfqpoint{4.272231in}{1.364999in}}%
\pgfpathlineto{\pgfqpoint{4.275403in}{1.364997in}}%
\pgfpathlineto{\pgfqpoint{4.278575in}{1.365170in}}%
\pgfpathlineto{\pgfqpoint{4.281747in}{1.364970in}}%
\pgfpathlineto{\pgfqpoint{4.284919in}{1.365424in}}%
\pgfpathlineto{\pgfqpoint{4.288091in}{1.365088in}}%
\pgfpathlineto{\pgfqpoint{4.291263in}{1.365043in}}%
\pgfpathlineto{\pgfqpoint{4.294435in}{1.364999in}}%
\pgfpathlineto{\pgfqpoint{4.297607in}{1.365225in}}%
\pgfpathlineto{\pgfqpoint{4.300779in}{1.365786in}}%
\pgfpathlineto{\pgfqpoint{4.303952in}{1.366145in}}%
\pgfpathlineto{\pgfqpoint{4.307124in}{1.366172in}}%
\pgfpathlineto{\pgfqpoint{4.310296in}{1.366136in}}%
\pgfpathlineto{\pgfqpoint{4.313468in}{1.366369in}}%
\pgfpathlineto{\pgfqpoint{4.316640in}{1.365940in}}%
\pgfpathlineto{\pgfqpoint{4.319812in}{1.366089in}}%
\pgfpathlineto{\pgfqpoint{4.322984in}{1.365815in}}%
\pgfpathlineto{\pgfqpoint{4.326156in}{1.366013in}}%
\pgfpathlineto{\pgfqpoint{4.329328in}{1.366241in}}%
\pgfpathlineto{\pgfqpoint{4.332500in}{1.366758in}}%
\pgfpathlineto{\pgfqpoint{4.335672in}{1.366967in}}%
\pgfpathlineto{\pgfqpoint{4.338844in}{1.367477in}}%
\pgfpathlineto{\pgfqpoint{4.342016in}{1.367585in}}%
\pgfpathlineto{\pgfqpoint{4.345188in}{1.366963in}}%
\pgfpathlineto{\pgfqpoint{4.348360in}{1.366520in}}%
\pgfpathlineto{\pgfqpoint{4.351532in}{1.366549in}}%
\pgfpathlineto{\pgfqpoint{4.354704in}{1.366233in}}%
\pgfpathlineto{\pgfqpoint{4.357876in}{1.365719in}}%
\pgfpathlineto{\pgfqpoint{4.361048in}{1.366087in}}%
\pgfpathlineto{\pgfqpoint{4.364220in}{1.366240in}}%
\pgfpathlineto{\pgfqpoint{4.367392in}{1.366095in}}%
\pgfpathlineto{\pgfqpoint{4.370564in}{1.365568in}}%
\pgfpathlineto{\pgfqpoint{4.373736in}{1.366031in}}%
\pgfpathlineto{\pgfqpoint{4.376908in}{1.366072in}}%
\pgfpathlineto{\pgfqpoint{4.380080in}{1.365947in}}%
\pgfpathlineto{\pgfqpoint{4.383253in}{1.366105in}}%
\pgfpathlineto{\pgfqpoint{4.386425in}{1.366316in}}%
\pgfpathlineto{\pgfqpoint{4.389597in}{1.366222in}}%
\pgfpathlineto{\pgfqpoint{4.392769in}{1.366083in}}%
\pgfpathlineto{\pgfqpoint{4.395941in}{1.365817in}}%
\pgfpathlineto{\pgfqpoint{4.399113in}{1.365820in}}%
\pgfpathlineto{\pgfqpoint{4.402285in}{1.365685in}}%
\pgfpathlineto{\pgfqpoint{4.405457in}{1.365843in}}%
\pgfpathlineto{\pgfqpoint{4.408629in}{1.365927in}}%
\pgfpathlineto{\pgfqpoint{4.411801in}{1.366107in}}%
\pgfpathlineto{\pgfqpoint{4.414973in}{1.365993in}}%
\pgfpathlineto{\pgfqpoint{4.418145in}{1.366176in}}%
\pgfpathlineto{\pgfqpoint{4.421317in}{1.366190in}}%
\pgfpathlineto{\pgfqpoint{4.424489in}{1.365945in}}%
\pgfpathlineto{\pgfqpoint{4.427661in}{1.366002in}}%
\pgfpathlineto{\pgfqpoint{4.430833in}{1.365235in}}%
\pgfpathlineto{\pgfqpoint{4.434005in}{1.365963in}}%
\pgfpathlineto{\pgfqpoint{4.437177in}{1.366709in}}%
\pgfpathlineto{\pgfqpoint{4.440349in}{1.367034in}}%
\pgfpathlineto{\pgfqpoint{4.443521in}{1.366649in}}%
\pgfpathlineto{\pgfqpoint{4.446693in}{1.366133in}}%
\pgfpathlineto{\pgfqpoint{4.449865in}{1.365805in}}%
\pgfpathlineto{\pgfqpoint{4.453037in}{1.366080in}}%
\pgfpathlineto{\pgfqpoint{4.456209in}{1.366167in}}%
\pgfpathlineto{\pgfqpoint{4.459381in}{1.366216in}}%
\pgfpathlineto{\pgfqpoint{4.462554in}{1.366101in}}%
\pgfpathlineto{\pgfqpoint{4.465726in}{1.366013in}}%
\pgfpathlineto{\pgfqpoint{4.468898in}{1.365934in}}%
\pgfpathlineto{\pgfqpoint{4.472070in}{1.365843in}}%
\pgfpathlineto{\pgfqpoint{4.475242in}{1.365508in}}%
\pgfpathlineto{\pgfqpoint{4.478414in}{1.365516in}}%
\pgfpathlineto{\pgfqpoint{4.481586in}{1.365242in}}%
\pgfpathlineto{\pgfqpoint{4.484758in}{1.365213in}}%
\pgfpathlineto{\pgfqpoint{4.487930in}{1.365248in}}%
\pgfpathlineto{\pgfqpoint{4.491102in}{1.365225in}}%
\pgfpathlineto{\pgfqpoint{4.494274in}{1.364701in}}%
\pgfpathlineto{\pgfqpoint{4.497446in}{1.364568in}}%
\pgfpathlineto{\pgfqpoint{4.500618in}{1.364820in}}%
\pgfpathlineto{\pgfqpoint{4.503790in}{1.364656in}}%
\pgfpathlineto{\pgfqpoint{4.506962in}{1.364667in}}%
\pgfpathlineto{\pgfqpoint{4.510134in}{1.364893in}}%
\pgfpathlineto{\pgfqpoint{4.513306in}{1.365069in}}%
\pgfpathlineto{\pgfqpoint{4.516478in}{1.364928in}}%
\pgfpathlineto{\pgfqpoint{4.519650in}{1.365108in}}%
\pgfpathlineto{\pgfqpoint{4.522822in}{1.364969in}}%
\pgfpathlineto{\pgfqpoint{4.525994in}{1.365028in}}%
\pgfpathlineto{\pgfqpoint{4.529166in}{1.364686in}}%
\pgfpathlineto{\pgfqpoint{4.532338in}{1.364738in}}%
\pgfpathlineto{\pgfqpoint{4.535510in}{1.365176in}}%
\pgfpathlineto{\pgfqpoint{4.538683in}{1.364953in}}%
\pgfpathlineto{\pgfqpoint{4.541855in}{1.364870in}}%
\pgfpathlineto{\pgfqpoint{4.545027in}{1.365178in}}%
\pgfpathlineto{\pgfqpoint{4.548199in}{1.365051in}}%
\pgfpathlineto{\pgfqpoint{4.551371in}{1.364862in}}%
\pgfpathlineto{\pgfqpoint{4.554543in}{1.364746in}}%
\pgfpathlineto{\pgfqpoint{4.557715in}{1.364916in}}%
\pgfpathlineto{\pgfqpoint{4.560887in}{1.365097in}}%
\pgfpathlineto{\pgfqpoint{4.564059in}{1.364920in}}%
\pgfpathlineto{\pgfqpoint{4.567231in}{1.364643in}}%
\pgfpathlineto{\pgfqpoint{4.570403in}{1.365056in}}%
\pgfpathlineto{\pgfqpoint{4.573575in}{1.364945in}}%
\pgfpathlineto{\pgfqpoint{4.576747in}{1.364695in}}%
\pgfpathlineto{\pgfqpoint{4.579919in}{1.364661in}}%
\pgfpathlineto{\pgfqpoint{4.583091in}{1.365070in}}%
\pgfpathlineto{\pgfqpoint{4.586263in}{1.365131in}}%
\pgfpathlineto{\pgfqpoint{4.589435in}{1.365219in}}%
\pgfpathlineto{\pgfqpoint{4.592607in}{1.365323in}}%
\pgfpathlineto{\pgfqpoint{4.595779in}{1.365635in}}%
\pgfpathlineto{\pgfqpoint{4.598951in}{1.365924in}}%
\pgfpathlineto{\pgfqpoint{4.602123in}{1.365914in}}%
\pgfpathlineto{\pgfqpoint{4.605295in}{1.365586in}}%
\pgfpathlineto{\pgfqpoint{4.608467in}{1.365698in}}%
\pgfpathlineto{\pgfqpoint{4.611639in}{1.365864in}}%
\pgfpathlineto{\pgfqpoint{4.614811in}{1.365383in}}%
\pgfpathlineto{\pgfqpoint{4.617984in}{1.365345in}}%
\pgfpathlineto{\pgfqpoint{4.621156in}{1.365714in}}%
\pgfpathlineto{\pgfqpoint{4.624328in}{1.365868in}}%
\pgfpathlineto{\pgfqpoint{4.627500in}{1.365765in}}%
\pgfpathlineto{\pgfqpoint{4.630672in}{1.365457in}}%
\pgfpathlineto{\pgfqpoint{4.633844in}{1.365196in}}%
\pgfpathlineto{\pgfqpoint{4.637016in}{1.365198in}}%
\pgfpathlineto{\pgfqpoint{4.640188in}{1.365118in}}%
\pgfpathlineto{\pgfqpoint{4.643360in}{1.364708in}}%
\pgfpathlineto{\pgfqpoint{4.646532in}{1.365242in}}%
\pgfpathlineto{\pgfqpoint{4.649704in}{1.365540in}}%
\pgfpathlineto{\pgfqpoint{4.652876in}{1.365679in}}%
\pgfpathlineto{\pgfqpoint{4.656048in}{1.365647in}}%
\pgfpathlineto{\pgfqpoint{4.659220in}{1.365784in}}%
\pgfpathlineto{\pgfqpoint{4.662392in}{1.366135in}}%
\pgfpathlineto{\pgfqpoint{4.665564in}{1.366197in}}%
\pgfpathlineto{\pgfqpoint{4.668736in}{1.366676in}}%
\pgfpathlineto{\pgfqpoint{4.671908in}{1.367418in}}%
\pgfpathlineto{\pgfqpoint{4.675080in}{1.367674in}}%
\pgfpathlineto{\pgfqpoint{4.678252in}{1.367768in}}%
\pgfpathlineto{\pgfqpoint{4.681424in}{1.368341in}}%
\pgfpathlineto{\pgfqpoint{4.684596in}{1.367952in}}%
\pgfpathlineto{\pgfqpoint{4.687768in}{1.368012in}}%
\pgfpathlineto{\pgfqpoint{4.690940in}{1.368407in}}%
\pgfpathlineto{\pgfqpoint{4.694112in}{1.368086in}}%
\pgfpathlineto{\pgfqpoint{4.697285in}{1.367910in}}%
\pgfpathlineto{\pgfqpoint{4.700457in}{1.367601in}}%
\pgfpathlineto{\pgfqpoint{4.703629in}{1.367576in}}%
\pgfpathlineto{\pgfqpoint{4.706801in}{1.368046in}}%
\pgfpathlineto{\pgfqpoint{4.709973in}{1.368279in}}%
\pgfpathlineto{\pgfqpoint{4.713145in}{1.368224in}}%
\pgfpathlineto{\pgfqpoint{4.716317in}{1.368175in}}%
\pgfpathlineto{\pgfqpoint{4.719489in}{1.367644in}}%
\pgfpathlineto{\pgfqpoint{4.722661in}{1.367787in}}%
\pgfpathlineto{\pgfqpoint{4.725833in}{1.368186in}}%
\pgfpathlineto{\pgfqpoint{4.729005in}{1.367781in}}%
\pgfpathlineto{\pgfqpoint{4.732177in}{1.367963in}}%
\pgfpathlineto{\pgfqpoint{4.735349in}{1.367713in}}%
\pgfpathlineto{\pgfqpoint{4.738521in}{1.367283in}}%
\pgfpathlineto{\pgfqpoint{4.741693in}{1.366662in}}%
\pgfpathlineto{\pgfqpoint{4.744865in}{1.367326in}}%
\pgfpathlineto{\pgfqpoint{4.748037in}{1.367875in}}%
\pgfpathlineto{\pgfqpoint{4.751209in}{1.368137in}}%
\pgfpathlineto{\pgfqpoint{4.754381in}{1.367853in}}%
\pgfpathlineto{\pgfqpoint{4.757553in}{1.368026in}}%
\pgfpathlineto{\pgfqpoint{4.760725in}{1.368761in}}%
\pgfpathlineto{\pgfqpoint{4.763897in}{1.368914in}}%
\pgfpathlineto{\pgfqpoint{4.767069in}{1.368475in}}%
\pgfpathlineto{\pgfqpoint{4.770241in}{1.368257in}}%
\pgfpathlineto{\pgfqpoint{4.773414in}{1.368582in}}%
\pgfpathlineto{\pgfqpoint{4.776586in}{1.368405in}}%
\pgfpathlineto{\pgfqpoint{4.779758in}{1.368714in}}%
\pgfpathlineto{\pgfqpoint{4.782930in}{1.368764in}}%
\pgfpathlineto{\pgfqpoint{4.786102in}{1.369303in}}%
\pgfpathlineto{\pgfqpoint{4.789274in}{1.369029in}}%
\pgfpathlineto{\pgfqpoint{4.792446in}{1.368599in}}%
\pgfpathlineto{\pgfqpoint{4.795618in}{1.367753in}}%
\pgfpathlineto{\pgfqpoint{4.798790in}{1.368008in}}%
\pgfpathlineto{\pgfqpoint{4.801962in}{1.367817in}}%
\pgfpathlineto{\pgfqpoint{4.805134in}{1.367848in}}%
\pgfpathlineto{\pgfqpoint{4.808306in}{1.367914in}}%
\pgfpathlineto{\pgfqpoint{4.811478in}{1.368343in}}%
\pgfpathlineto{\pgfqpoint{4.814650in}{1.368777in}}%
\pgfpathlineto{\pgfqpoint{4.817822in}{1.368836in}}%
\pgfpathlineto{\pgfqpoint{4.820994in}{1.368455in}}%
\pgfpathlineto{\pgfqpoint{4.824166in}{1.368380in}}%
\pgfpathlineto{\pgfqpoint{4.827338in}{1.368238in}}%
\pgfpathlineto{\pgfqpoint{4.830510in}{1.368089in}}%
\pgfpathlineto{\pgfqpoint{4.833682in}{1.368139in}}%
\pgfpathlineto{\pgfqpoint{4.836854in}{1.368422in}}%
\pgfpathlineto{\pgfqpoint{4.840026in}{1.368141in}}%
\pgfpathlineto{\pgfqpoint{4.843198in}{1.368419in}}%
\pgfpathlineto{\pgfqpoint{4.846370in}{1.368500in}}%
\pgfpathlineto{\pgfqpoint{4.849542in}{1.368668in}}%
\pgfpathlineto{\pgfqpoint{4.852715in}{1.369028in}}%
\pgfpathlineto{\pgfqpoint{4.855887in}{1.369065in}}%
\pgfpathlineto{\pgfqpoint{4.859059in}{1.368887in}}%
\pgfpathlineto{\pgfqpoint{4.862231in}{1.369274in}}%
\pgfpathlineto{\pgfqpoint{4.865403in}{1.369785in}}%
\pgfpathlineto{\pgfqpoint{4.868575in}{1.369786in}}%
\pgfpathlineto{\pgfqpoint{4.871747in}{1.370141in}}%
\pgfpathlineto{\pgfqpoint{4.874919in}{1.370156in}}%
\pgfpathlineto{\pgfqpoint{4.878091in}{1.370093in}}%
\pgfpathlineto{\pgfqpoint{4.881263in}{1.370113in}}%
\pgfpathlineto{\pgfqpoint{4.884435in}{1.370077in}}%
\pgfpathlineto{\pgfqpoint{4.887607in}{1.370013in}}%
\pgfpathlineto{\pgfqpoint{4.890779in}{1.369914in}}%
\pgfpathlineto{\pgfqpoint{4.893951in}{1.370034in}}%
\pgfpathlineto{\pgfqpoint{4.897123in}{1.370003in}}%
\pgfpathlineto{\pgfqpoint{4.900295in}{1.370500in}}%
\pgfpathlineto{\pgfqpoint{4.903467in}{1.370432in}}%
\pgfpathlineto{\pgfqpoint{4.906639in}{1.370541in}}%
\pgfpathlineto{\pgfqpoint{4.909811in}{1.370507in}}%
\pgfpathlineto{\pgfqpoint{4.912983in}{1.370794in}}%
\pgfpathlineto{\pgfqpoint{4.916155in}{1.371067in}}%
\pgfpathlineto{\pgfqpoint{4.919327in}{1.370943in}}%
\pgfpathlineto{\pgfqpoint{4.922499in}{1.370783in}}%
\pgfpathlineto{\pgfqpoint{4.925671in}{1.370534in}}%
\pgfpathlineto{\pgfqpoint{4.928844in}{1.370645in}}%
\pgfpathlineto{\pgfqpoint{4.932016in}{1.370811in}}%
\pgfpathlineto{\pgfqpoint{4.935188in}{1.371165in}}%
\pgfpathlineto{\pgfqpoint{4.938360in}{1.371409in}}%
\pgfpathlineto{\pgfqpoint{4.941532in}{1.371490in}}%
\pgfpathlineto{\pgfqpoint{4.944704in}{1.371961in}}%
\pgfpathlineto{\pgfqpoint{4.947876in}{1.372403in}}%
\pgfpathlineto{\pgfqpoint{4.951048in}{1.373111in}}%
\pgfpathlineto{\pgfqpoint{4.954220in}{1.373219in}}%
\pgfpathlineto{\pgfqpoint{4.957392in}{1.372947in}}%
\pgfpathlineto{\pgfqpoint{4.960564in}{1.372971in}}%
\pgfpathlineto{\pgfqpoint{4.963736in}{1.372851in}}%
\pgfpathlineto{\pgfqpoint{4.966908in}{1.372934in}}%
\pgfpathlineto{\pgfqpoint{4.970080in}{1.373078in}}%
\pgfpathlineto{\pgfqpoint{4.973252in}{1.373088in}}%
\pgfpathlineto{\pgfqpoint{4.976424in}{1.373391in}}%
\pgfpathlineto{\pgfqpoint{4.979596in}{1.373701in}}%
\pgfpathlineto{\pgfqpoint{4.982768in}{1.373493in}}%
\pgfpathlineto{\pgfqpoint{4.985940in}{1.373149in}}%
\pgfpathlineto{\pgfqpoint{4.989112in}{1.372899in}}%
\pgfpathlineto{\pgfqpoint{4.992284in}{1.372213in}}%
\pgfpathlineto{\pgfqpoint{4.995456in}{1.372388in}}%
\pgfpathlineto{\pgfqpoint{4.998628in}{1.372424in}}%
\pgfpathlineto{\pgfqpoint{5.001800in}{1.372670in}}%
\pgfpathlineto{\pgfqpoint{5.004972in}{1.372857in}}%
\pgfpathlineto{\pgfqpoint{5.008145in}{1.372277in}}%
\pgfpathlineto{\pgfqpoint{5.011317in}{1.371685in}}%
\pgfpathlineto{\pgfqpoint{5.014489in}{1.371606in}}%
\pgfpathlineto{\pgfqpoint{5.017661in}{1.371634in}}%
\pgfpathlineto{\pgfqpoint{5.020833in}{1.371393in}}%
\pgfpathlineto{\pgfqpoint{5.024005in}{1.371568in}}%
\pgfpathlineto{\pgfqpoint{5.027177in}{1.371595in}}%
\pgfpathlineto{\pgfqpoint{5.030349in}{1.371560in}}%
\pgfpathlineto{\pgfqpoint{5.033521in}{1.371298in}}%
\pgfpathlineto{\pgfqpoint{5.036693in}{1.371754in}}%
\pgfpathlineto{\pgfqpoint{5.039865in}{1.371932in}}%
\pgfpathlineto{\pgfqpoint{5.043037in}{1.371928in}}%
\pgfpathlineto{\pgfqpoint{5.046209in}{1.372269in}}%
\pgfpathlineto{\pgfqpoint{5.049381in}{1.372524in}}%
\pgfpathlineto{\pgfqpoint{5.052553in}{1.372287in}}%
\pgfpathlineto{\pgfqpoint{5.055725in}{1.371993in}}%
\pgfpathlineto{\pgfqpoint{5.058897in}{1.371882in}}%
\pgfpathlineto{\pgfqpoint{5.062069in}{1.372114in}}%
\pgfpathlineto{\pgfqpoint{5.065241in}{1.372008in}}%
\pgfpathlineto{\pgfqpoint{5.068413in}{1.372378in}}%
\pgfpathlineto{\pgfqpoint{5.071585in}{1.372832in}}%
\pgfpathlineto{\pgfqpoint{5.074757in}{1.373057in}}%
\pgfpathlineto{\pgfqpoint{5.077929in}{1.372860in}}%
\pgfpathlineto{\pgfqpoint{5.081101in}{1.373468in}}%
\pgfpathlineto{\pgfqpoint{5.084273in}{1.373253in}}%
\pgfpathlineto{\pgfqpoint{5.087446in}{1.373073in}}%
\pgfpathlineto{\pgfqpoint{5.090618in}{1.372964in}}%
\pgfpathlineto{\pgfqpoint{5.093790in}{1.373191in}}%
\pgfpathlineto{\pgfqpoint{5.096962in}{1.373192in}}%
\pgfpathlineto{\pgfqpoint{5.100134in}{1.373200in}}%
\pgfpathlineto{\pgfqpoint{5.103306in}{1.373249in}}%
\pgfpathlineto{\pgfqpoint{5.106478in}{1.373365in}}%
\pgfpathlineto{\pgfqpoint{5.109650in}{1.373462in}}%
\pgfpathlineto{\pgfqpoint{5.112822in}{1.373540in}}%
\pgfpathlineto{\pgfqpoint{5.115994in}{1.373494in}}%
\pgfpathlineto{\pgfqpoint{5.119166in}{1.373386in}}%
\pgfpathlineto{\pgfqpoint{5.122338in}{1.373641in}}%
\pgfpathlineto{\pgfqpoint{5.125510in}{1.373268in}}%
\pgfpathlineto{\pgfqpoint{5.128682in}{1.373337in}}%
\pgfpathlineto{\pgfqpoint{5.131854in}{1.373388in}}%
\pgfpathlineto{\pgfqpoint{5.135026in}{1.373577in}}%
\pgfpathlineto{\pgfqpoint{5.138198in}{1.373836in}}%
\pgfpathlineto{\pgfqpoint{5.141370in}{1.374385in}}%
\pgfpathlineto{\pgfqpoint{5.144542in}{1.374205in}}%
\pgfpathlineto{\pgfqpoint{5.147714in}{1.374057in}}%
\pgfpathlineto{\pgfqpoint{5.150886in}{1.374071in}}%
\pgfpathlineto{\pgfqpoint{5.154058in}{1.374367in}}%
\pgfpathlineto{\pgfqpoint{5.157230in}{1.374438in}}%
\pgfpathlineto{\pgfqpoint{5.160402in}{1.374101in}}%
\pgfpathlineto{\pgfqpoint{5.163575in}{1.373712in}}%
\pgfpathlineto{\pgfqpoint{5.166747in}{1.373925in}}%
\pgfpathlineto{\pgfqpoint{5.169919in}{1.373593in}}%
\pgfpathlineto{\pgfqpoint{5.173091in}{1.373178in}}%
\pgfpathlineto{\pgfqpoint{5.176263in}{1.373034in}}%
\pgfpathlineto{\pgfqpoint{5.179435in}{1.373027in}}%
\pgfpathlineto{\pgfqpoint{5.182607in}{1.372994in}}%
\pgfpathlineto{\pgfqpoint{5.185779in}{1.373082in}}%
\pgfpathlineto{\pgfqpoint{5.188951in}{1.373184in}}%
\pgfpathlineto{\pgfqpoint{5.192123in}{1.373165in}}%
\pgfpathlineto{\pgfqpoint{5.195295in}{1.372577in}}%
\pgfpathlineto{\pgfqpoint{5.198467in}{1.372509in}}%
\pgfpathlineto{\pgfqpoint{5.201639in}{1.372692in}}%
\pgfpathlineto{\pgfqpoint{5.204811in}{1.372894in}}%
\pgfpathlineto{\pgfqpoint{5.207983in}{1.373234in}}%
\pgfpathlineto{\pgfqpoint{5.211155in}{1.373861in}}%
\pgfpathlineto{\pgfqpoint{5.214327in}{1.374310in}}%
\pgfpathlineto{\pgfqpoint{5.217499in}{1.374709in}}%
\pgfpathlineto{\pgfqpoint{5.220671in}{1.375000in}}%
\pgfpathlineto{\pgfqpoint{5.223843in}{1.374950in}}%
\pgfpathlineto{\pgfqpoint{5.227015in}{1.374905in}}%
\pgfpathlineto{\pgfqpoint{5.230187in}{1.374985in}}%
\pgfpathlineto{\pgfqpoint{5.233359in}{1.375072in}}%
\pgfpathlineto{\pgfqpoint{5.236531in}{1.375189in}}%
\pgfpathlineto{\pgfqpoint{5.239703in}{1.375299in}}%
\pgfpathlineto{\pgfqpoint{5.242876in}{1.375130in}}%
\pgfpathlineto{\pgfqpoint{5.246048in}{1.374736in}}%
\pgfpathlineto{\pgfqpoint{5.249220in}{1.374219in}}%
\pgfpathlineto{\pgfqpoint{5.252392in}{1.374283in}}%
\pgfpathlineto{\pgfqpoint{5.255564in}{1.374403in}}%
\pgfpathlineto{\pgfqpoint{5.258736in}{1.374414in}}%
\pgfpathlineto{\pgfqpoint{5.261908in}{1.374344in}}%
\pgfpathlineto{\pgfqpoint{5.265080in}{1.374352in}}%
\pgfpathlineto{\pgfqpoint{5.268252in}{1.374269in}}%
\pgfpathlineto{\pgfqpoint{5.271424in}{1.374041in}}%
\pgfpathlineto{\pgfqpoint{5.274596in}{1.373730in}}%
\pgfpathlineto{\pgfqpoint{5.277768in}{1.373025in}}%
\pgfpathlineto{\pgfqpoint{5.280940in}{1.372842in}}%
\pgfpathlineto{\pgfqpoint{5.284112in}{1.373292in}}%
\pgfpathlineto{\pgfqpoint{5.287284in}{1.373397in}}%
\pgfpathlineto{\pgfqpoint{5.290456in}{1.373422in}}%
\pgfpathlineto{\pgfqpoint{5.293628in}{1.373102in}}%
\pgfpathlineto{\pgfqpoint{5.296800in}{1.373029in}}%
\pgfpathlineto{\pgfqpoint{5.299972in}{1.373280in}}%
\pgfpathlineto{\pgfqpoint{5.303144in}{1.373149in}}%
\pgfpathlineto{\pgfqpoint{5.306316in}{1.373006in}}%
\pgfpathlineto{\pgfqpoint{5.309488in}{1.372996in}}%
\pgfpathlineto{\pgfqpoint{5.312660in}{1.373403in}}%
\pgfpathlineto{\pgfqpoint{5.315832in}{1.374049in}}%
\pgfpathlineto{\pgfqpoint{5.319004in}{1.374529in}}%
\pgfpathlineto{\pgfqpoint{5.322177in}{1.374445in}}%
\pgfpathlineto{\pgfqpoint{5.325349in}{1.374317in}}%
\pgfpathlineto{\pgfqpoint{5.328521in}{1.374577in}}%
\pgfpathlineto{\pgfqpoint{5.331693in}{1.374740in}}%
\pgfpathlineto{\pgfqpoint{5.334865in}{1.375206in}}%
\pgfpathlineto{\pgfqpoint{5.338037in}{1.374942in}}%
\pgfpathlineto{\pgfqpoint{5.341209in}{1.375195in}}%
\pgfpathlineto{\pgfqpoint{5.344381in}{1.375054in}}%
\pgfpathlineto{\pgfqpoint{5.347553in}{1.375136in}}%
\pgfpathlineto{\pgfqpoint{5.350725in}{1.375268in}}%
\pgfpathlineto{\pgfqpoint{5.353897in}{1.375090in}}%
\pgfpathlineto{\pgfqpoint{5.357069in}{1.375111in}}%
\pgfpathlineto{\pgfqpoint{5.360241in}{1.374143in}}%
\pgfpathlineto{\pgfqpoint{5.363413in}{1.374423in}}%
\pgfpathlineto{\pgfqpoint{5.366585in}{1.374335in}}%
\pgfpathlineto{\pgfqpoint{5.369757in}{1.374630in}}%
\pgfpathlineto{\pgfqpoint{5.372929in}{1.374826in}}%
\pgfpathlineto{\pgfqpoint{5.376101in}{1.374990in}}%
\pgfpathlineto{\pgfqpoint{5.379273in}{1.374958in}}%
\pgfpathlineto{\pgfqpoint{5.382445in}{1.374532in}}%
\pgfpathlineto{\pgfqpoint{5.385617in}{1.375047in}}%
\pgfpathlineto{\pgfqpoint{5.388789in}{1.374941in}}%
\pgfpathlineto{\pgfqpoint{5.391961in}{1.374750in}}%
\pgfpathlineto{\pgfqpoint{5.395133in}{1.374988in}}%
\pgfpathlineto{\pgfqpoint{5.398306in}{1.375463in}}%
\pgfpathlineto{\pgfqpoint{5.401478in}{1.375950in}}%
\pgfpathlineto{\pgfqpoint{5.404650in}{1.376062in}}%
\pgfpathlineto{\pgfqpoint{5.407822in}{1.376214in}}%
\pgfpathlineto{\pgfqpoint{5.410994in}{1.376455in}}%
\pgfpathlineto{\pgfqpoint{5.414166in}{1.376682in}}%
\pgfpathlineto{\pgfqpoint{5.417338in}{1.376949in}}%
\pgfpathlineto{\pgfqpoint{5.420510in}{1.377318in}}%
\pgfpathlineto{\pgfqpoint{5.423682in}{1.377594in}}%
\pgfpathlineto{\pgfqpoint{5.426854in}{1.377690in}}%
\pgfpathlineto{\pgfqpoint{5.430026in}{1.378306in}}%
\pgfpathlineto{\pgfqpoint{5.433198in}{1.378439in}}%
\pgfpathlineto{\pgfqpoint{5.436370in}{1.378677in}}%
\pgfpathlineto{\pgfqpoint{5.439542in}{1.378540in}}%
\pgfpathlineto{\pgfqpoint{5.442714in}{1.378516in}}%
\pgfpathlineto{\pgfqpoint{5.445886in}{1.377974in}}%
\pgfpathlineto{\pgfqpoint{5.449058in}{1.378073in}}%
\pgfpathlineto{\pgfqpoint{5.452230in}{1.378075in}}%
\pgfpathlineto{\pgfqpoint{5.455402in}{1.378259in}}%
\pgfpathlineto{\pgfqpoint{5.458574in}{1.378232in}}%
\pgfpathlineto{\pgfqpoint{5.461746in}{1.378064in}}%
\pgfpathlineto{\pgfqpoint{5.464918in}{1.377985in}}%
\pgfpathlineto{\pgfqpoint{5.468090in}{1.378117in}}%
\pgfpathlineto{\pgfqpoint{5.471262in}{1.378152in}}%
\pgfpathlineto{\pgfqpoint{5.474434in}{1.377928in}}%
\pgfpathlineto{\pgfqpoint{5.477607in}{1.377550in}}%
\pgfpathlineto{\pgfqpoint{5.480779in}{1.377333in}}%
\pgfpathlineto{\pgfqpoint{5.483951in}{1.377511in}}%
\pgfpathlineto{\pgfqpoint{5.487123in}{1.377306in}}%
\pgfpathlineto{\pgfqpoint{5.490295in}{1.377157in}}%
\pgfpathlineto{\pgfqpoint{5.493467in}{1.377028in}}%
\pgfpathlineto{\pgfqpoint{5.496639in}{1.376771in}}%
\pgfpathlineto{\pgfqpoint{5.499811in}{1.377136in}}%
\pgfpathlineto{\pgfqpoint{5.502983in}{1.376945in}}%
\pgfpathlineto{\pgfqpoint{5.506155in}{1.377018in}}%
\pgfpathlineto{\pgfqpoint{5.509327in}{1.376376in}}%
\pgfpathlineto{\pgfqpoint{5.512499in}{1.376471in}}%
\pgfpathlineto{\pgfqpoint{5.515671in}{1.376343in}}%
\pgfpathlineto{\pgfqpoint{5.518843in}{1.375377in}}%
\pgfpathlineto{\pgfqpoint{5.522015in}{1.375120in}}%
\pgfpathlineto{\pgfqpoint{5.525187in}{1.375052in}}%
\pgfpathlineto{\pgfqpoint{5.528359in}{1.375585in}}%
\pgfpathlineto{\pgfqpoint{5.531531in}{1.376012in}}%
\pgfpathlineto{\pgfqpoint{5.534703in}{1.376053in}}%
\pgfpathlineto{\pgfqpoint{5.537875in}{1.375786in}}%
\pgfpathlineto{\pgfqpoint{5.541047in}{1.375557in}}%
\pgfpathlineto{\pgfqpoint{5.544219in}{1.375450in}}%
\pgfpathlineto{\pgfqpoint{5.547391in}{1.374857in}}%
\pgfpathlineto{\pgfqpoint{5.550563in}{1.374838in}}%
\pgfpathlineto{\pgfqpoint{5.553735in}{1.374889in}}%
\pgfpathlineto{\pgfqpoint{5.556908in}{1.374869in}}%
\pgfpathlineto{\pgfqpoint{5.560080in}{1.374764in}}%
\pgfpathlineto{\pgfqpoint{5.563252in}{1.375011in}}%
\pgfpathlineto{\pgfqpoint{5.566424in}{1.375310in}}%
\pgfpathlineto{\pgfqpoint{5.569596in}{1.375272in}}%
\pgfpathlineto{\pgfqpoint{5.572768in}{1.375370in}}%
\pgfpathlineto{\pgfqpoint{5.575940in}{1.375145in}}%
\pgfpathlineto{\pgfqpoint{5.579112in}{1.374952in}}%
\pgfpathlineto{\pgfqpoint{5.582284in}{1.374677in}}%
\pgfpathlineto{\pgfqpoint{5.585456in}{1.374844in}}%
\pgfpathlineto{\pgfqpoint{5.588628in}{1.375407in}}%
\pgfpathlineto{\pgfqpoint{5.591800in}{1.375329in}}%
\pgfpathlineto{\pgfqpoint{5.594972in}{1.374898in}}%
\pgfpathlineto{\pgfqpoint{5.598144in}{1.374848in}}%
\pgfpathlineto{\pgfqpoint{5.601316in}{1.375316in}}%
\pgfpathlineto{\pgfqpoint{5.604488in}{1.375260in}}%
\pgfpathlineto{\pgfqpoint{5.607660in}{1.375066in}}%
\pgfpathlineto{\pgfqpoint{5.610832in}{1.375469in}}%
\pgfpathlineto{\pgfqpoint{5.614004in}{1.375897in}}%
\pgfpathlineto{\pgfqpoint{5.617176in}{1.375784in}}%
\pgfpathlineto{\pgfqpoint{5.620348in}{1.375752in}}%
\pgfpathlineto{\pgfqpoint{5.623520in}{1.376090in}}%
\pgfpathlineto{\pgfqpoint{5.626692in}{1.375503in}}%
\pgfpathlineto{\pgfqpoint{5.629864in}{1.374812in}}%
\pgfpathlineto{\pgfqpoint{5.633037in}{1.374140in}}%
\pgfpathlineto{\pgfqpoint{5.636209in}{1.373779in}}%
\pgfpathlineto{\pgfqpoint{5.639381in}{1.373552in}}%
\pgfpathlineto{\pgfqpoint{5.642553in}{1.373530in}}%
\pgfpathlineto{\pgfqpoint{5.645725in}{1.373381in}}%
\pgfpathlineto{\pgfqpoint{5.648897in}{1.373247in}}%
\pgfpathlineto{\pgfqpoint{5.652069in}{1.372954in}}%
\pgfpathlineto{\pgfqpoint{5.655241in}{1.372547in}}%
\pgfpathlineto{\pgfqpoint{5.658413in}{1.372334in}}%
\pgfpathlineto{\pgfqpoint{5.661585in}{1.372323in}}%
\pgfpathlineto{\pgfqpoint{5.664757in}{1.371964in}}%
\pgfpathlineto{\pgfqpoint{5.667929in}{1.371498in}}%
\pgfpathlineto{\pgfqpoint{5.671101in}{1.371341in}}%
\pgfpathlineto{\pgfqpoint{5.674273in}{1.371373in}}%
\pgfpathlineto{\pgfqpoint{5.677445in}{1.371451in}}%
\pgfpathlineto{\pgfqpoint{5.680617in}{1.371653in}}%
\pgfpathlineto{\pgfqpoint{5.683789in}{1.371854in}}%
\pgfpathlineto{\pgfqpoint{5.686961in}{1.371949in}}%
\pgfpathlineto{\pgfqpoint{5.690133in}{1.372091in}}%
\pgfpathlineto{\pgfqpoint{5.693305in}{1.372314in}}%
\pgfpathlineto{\pgfqpoint{5.696477in}{1.372532in}}%
\pgfpathlineto{\pgfqpoint{5.699649in}{1.371925in}}%
\pgfpathlineto{\pgfqpoint{5.702821in}{1.371540in}}%
\pgfpathlineto{\pgfqpoint{5.705993in}{1.371892in}}%
\pgfpathlineto{\pgfqpoint{5.709165in}{1.371848in}}%
\pgfpathlineto{\pgfqpoint{5.712338in}{1.371754in}}%
\pgfpathlineto{\pgfqpoint{5.715510in}{1.371599in}}%
\pgfpathlineto{\pgfqpoint{5.718682in}{1.371491in}}%
\pgfpathlineto{\pgfqpoint{5.721854in}{1.371269in}}%
\pgfpathlineto{\pgfqpoint{5.725026in}{1.370752in}}%
\pgfpathlineto{\pgfqpoint{5.728198in}{1.370835in}}%
\pgfpathlineto{\pgfqpoint{5.731370in}{1.371090in}}%
\pgfpathlineto{\pgfqpoint{5.734542in}{1.371462in}}%
\pgfpathlineto{\pgfqpoint{5.737714in}{1.371485in}}%
\pgfpathlineto{\pgfqpoint{5.740886in}{1.371597in}}%
\pgfpathlineto{\pgfqpoint{5.744058in}{1.371411in}}%
\pgfpathlineto{\pgfqpoint{5.747230in}{1.370690in}}%
\pgfpathlineto{\pgfqpoint{5.750402in}{1.370869in}}%
\pgfpathlineto{\pgfqpoint{5.753574in}{1.371090in}}%
\pgfpathlineto{\pgfqpoint{5.756746in}{1.370819in}}%
\pgfpathlineto{\pgfqpoint{5.759918in}{1.370889in}}%
\pgfpathlineto{\pgfqpoint{5.763090in}{1.371354in}}%
\pgfpathlineto{\pgfqpoint{5.766262in}{1.371896in}}%
\pgfpathlineto{\pgfqpoint{5.769434in}{1.371967in}}%
\pgfpathlineto{\pgfqpoint{5.772606in}{1.372032in}}%
\pgfpathlineto{\pgfqpoint{5.775778in}{1.371567in}}%
\pgfpathlineto{\pgfqpoint{5.778950in}{1.371357in}}%
\pgfpathlineto{\pgfqpoint{5.782122in}{1.370539in}}%
\pgfpathlineto{\pgfqpoint{5.785294in}{1.370169in}}%
\pgfpathlineto{\pgfqpoint{5.788466in}{1.370453in}}%
\pgfpathlineto{\pgfqpoint{5.791639in}{1.370884in}}%
\pgfpathlineto{\pgfqpoint{5.794811in}{1.371239in}}%
\pgfpathlineto{\pgfqpoint{5.797983in}{1.370837in}}%
\pgfpathlineto{\pgfqpoint{5.801155in}{1.371392in}}%
\pgfpathlineto{\pgfqpoint{5.804327in}{1.371513in}}%
\pgfpathlineto{\pgfqpoint{5.807499in}{1.371535in}}%
\pgfpathlineto{\pgfqpoint{5.810671in}{1.371186in}}%
\pgfpathlineto{\pgfqpoint{5.813843in}{1.371301in}}%
\pgfpathlineto{\pgfqpoint{5.817015in}{1.371312in}}%
\pgfpathlineto{\pgfqpoint{5.820187in}{1.371237in}}%
\pgfpathlineto{\pgfqpoint{5.823359in}{1.371334in}}%
\pgfpathlineto{\pgfqpoint{5.826531in}{1.371573in}}%
\pgfpathlineto{\pgfqpoint{5.829703in}{1.371118in}}%
\pgfpathlineto{\pgfqpoint{5.832875in}{1.370605in}}%
\pgfpathlineto{\pgfqpoint{5.836047in}{1.370111in}}%
\pgfpathlineto{\pgfqpoint{5.839219in}{1.369571in}}%
\pgfpathlineto{\pgfqpoint{5.842391in}{1.369360in}}%
\pgfpathlineto{\pgfqpoint{5.845563in}{1.369587in}}%
\pgfpathlineto{\pgfqpoint{5.848735in}{1.369268in}}%
\pgfpathlineto{\pgfqpoint{5.851907in}{1.369153in}}%
\pgfpathlineto{\pgfqpoint{5.855079in}{1.369434in}}%
\pgfpathlineto{\pgfqpoint{5.858251in}{1.369894in}}%
\pgfpathlineto{\pgfqpoint{5.861423in}{1.369982in}}%
\pgfpathlineto{\pgfqpoint{5.864595in}{1.370275in}}%
\pgfpathlineto{\pgfqpoint{5.867768in}{1.370329in}}%
\pgfpathlineto{\pgfqpoint{5.870940in}{1.370128in}}%
\pgfpathlineto{\pgfqpoint{5.874112in}{1.370124in}}%
\pgfpathlineto{\pgfqpoint{5.877284in}{1.369981in}}%
\pgfpathlineto{\pgfqpoint{5.880456in}{1.370168in}}%
\pgfpathlineto{\pgfqpoint{5.883628in}{1.370224in}}%
\pgfpathlineto{\pgfqpoint{5.886800in}{1.370086in}}%
\pgfpathlineto{\pgfqpoint{5.889972in}{1.369966in}}%
\pgfpathlineto{\pgfqpoint{5.893144in}{1.369898in}}%
\pgfpathlineto{\pgfqpoint{5.896316in}{1.369713in}}%
\pgfpathlineto{\pgfqpoint{5.899488in}{1.369471in}}%
\pgfpathlineto{\pgfqpoint{5.902660in}{1.369364in}}%
\pgfpathlineto{\pgfqpoint{5.905832in}{1.369416in}}%
\pgfpathlineto{\pgfqpoint{5.909004in}{1.369237in}}%
\pgfpathlineto{\pgfqpoint{5.912176in}{1.369714in}}%
\pgfpathlineto{\pgfqpoint{5.915348in}{1.369876in}}%
\pgfpathlineto{\pgfqpoint{5.918520in}{1.370093in}}%
\pgfpathlineto{\pgfqpoint{5.921692in}{1.370348in}}%
\pgfpathlineto{\pgfqpoint{5.924864in}{1.370036in}}%
\pgfpathlineto{\pgfqpoint{5.928036in}{1.369793in}}%
\pgfpathlineto{\pgfqpoint{5.931208in}{1.369707in}}%
\pgfpathlineto{\pgfqpoint{5.934380in}{1.369553in}}%
\pgfpathlineto{\pgfqpoint{5.937552in}{1.369434in}}%
\pgfpathlineto{\pgfqpoint{5.940724in}{1.369743in}}%
\pgfpathlineto{\pgfqpoint{5.943896in}{1.369973in}}%
\pgfpathlineto{\pgfqpoint{5.947069in}{1.369843in}}%
\pgfpathlineto{\pgfqpoint{5.950241in}{1.370020in}}%
\pgfpathlineto{\pgfqpoint{5.953413in}{1.370340in}}%
\pgfpathlineto{\pgfqpoint{5.956585in}{1.370291in}}%
\pgfpathlineto{\pgfqpoint{5.959757in}{1.369702in}}%
\pgfpathlineto{\pgfqpoint{5.962929in}{1.369258in}}%
\pgfpathlineto{\pgfqpoint{5.966101in}{1.368689in}}%
\pgfpathlineto{\pgfqpoint{5.969273in}{1.369027in}}%
\pgfpathlineto{\pgfqpoint{5.972445in}{1.368894in}}%
\pgfpathlineto{\pgfqpoint{5.975617in}{1.368968in}}%
\pgfpathlineto{\pgfqpoint{5.978789in}{1.368626in}}%
\pgfpathlineto{\pgfqpoint{5.981961in}{1.368648in}}%
\pgfpathlineto{\pgfqpoint{5.985133in}{1.368809in}}%
\pgfpathlineto{\pgfqpoint{5.988305in}{1.368739in}}%
\pgfpathlineto{\pgfqpoint{5.991477in}{1.368141in}}%
\pgfpathlineto{\pgfqpoint{5.994649in}{1.368099in}}%
\pgfpathlineto{\pgfqpoint{5.997821in}{1.367558in}}%
\pgfpathlineto{\pgfqpoint{6.000993in}{1.367807in}}%
\pgfpathlineto{\pgfqpoint{6.004165in}{1.368433in}}%
\pgfpathlineto{\pgfqpoint{6.007337in}{1.368582in}}%
\pgfpathlineto{\pgfqpoint{6.010509in}{1.368593in}}%
\pgfpathlineto{\pgfqpoint{6.013681in}{1.369125in}}%
\pgfpathlineto{\pgfqpoint{6.016853in}{1.369186in}}%
\pgfpathlineto{\pgfqpoint{6.020025in}{1.369191in}}%
\pgfpathlineto{\pgfqpoint{6.023197in}{1.369559in}}%
\pgfpathlineto{\pgfqpoint{6.026370in}{1.369781in}}%
\pgfpathlineto{\pgfqpoint{6.029542in}{1.369829in}}%
\pgfpathlineto{\pgfqpoint{6.032714in}{1.369972in}}%
\pgfpathlineto{\pgfqpoint{6.035886in}{1.369700in}}%
\pgfpathlineto{\pgfqpoint{6.039058in}{1.369676in}}%
\pgfpathlineto{\pgfqpoint{6.042230in}{1.369125in}}%
\pgfpathlineto{\pgfqpoint{6.045402in}{1.369279in}}%
\pgfpathlineto{\pgfqpoint{6.048574in}{1.368747in}}%
\pgfpathlineto{\pgfqpoint{6.051746in}{1.368462in}}%
\pgfpathlineto{\pgfqpoint{6.054918in}{1.368518in}}%
\pgfpathlineto{\pgfqpoint{6.058090in}{1.368682in}}%
\pgfpathlineto{\pgfqpoint{6.061262in}{1.368316in}}%
\pgfpathlineto{\pgfqpoint{6.064434in}{1.368004in}}%
\pgfpathlineto{\pgfqpoint{6.067606in}{1.367837in}}%
\pgfpathlineto{\pgfqpoint{6.070778in}{1.367114in}}%
\pgfpathlineto{\pgfqpoint{6.073950in}{1.367244in}}%
\pgfpathlineto{\pgfqpoint{6.077122in}{1.367315in}}%
\pgfpathlineto{\pgfqpoint{6.080294in}{1.367362in}}%
\pgfpathlineto{\pgfqpoint{6.083466in}{1.367106in}}%
\pgfpathlineto{\pgfqpoint{6.086638in}{1.366920in}}%
\pgfpathlineto{\pgfqpoint{6.089810in}{1.366608in}}%
\pgfpathlineto{\pgfqpoint{6.092982in}{1.366953in}}%
\pgfpathlineto{\pgfqpoint{6.096154in}{1.367007in}}%
\pgfpathlineto{\pgfqpoint{6.099326in}{1.367065in}}%
\pgfpathlineto{\pgfqpoint{6.102499in}{1.367086in}}%
\pgfpathlineto{\pgfqpoint{6.105671in}{1.367172in}}%
\pgfpathlineto{\pgfqpoint{6.108843in}{1.367026in}}%
\pgfpathlineto{\pgfqpoint{6.112015in}{1.367138in}}%
\pgfpathlineto{\pgfqpoint{6.115187in}{1.367196in}}%
\pgfpathlineto{\pgfqpoint{6.118359in}{1.367307in}}%
\pgfpathlineto{\pgfqpoint{6.121531in}{1.366629in}}%
\pgfpathlineto{\pgfqpoint{6.124703in}{1.366135in}}%
\pgfpathlineto{\pgfqpoint{6.127875in}{1.365917in}}%
\pgfpathlineto{\pgfqpoint{6.131047in}{1.366130in}}%
\pgfpathlineto{\pgfqpoint{6.134219in}{1.366233in}}%
\pgfpathlineto{\pgfqpoint{6.137391in}{1.366669in}}%
\pgfpathlineto{\pgfqpoint{6.140563in}{1.366176in}}%
\pgfpathlineto{\pgfqpoint{6.143735in}{1.365864in}}%
\pgfpathlineto{\pgfqpoint{6.146907in}{1.365163in}}%
\pgfpathlineto{\pgfqpoint{6.150079in}{1.365671in}}%
\pgfpathlineto{\pgfqpoint{6.153251in}{1.365676in}}%
\pgfpathlineto{\pgfqpoint{6.156423in}{1.366011in}}%
\pgfpathlineto{\pgfqpoint{6.159595in}{1.366077in}}%
\pgfpathlineto{\pgfqpoint{6.162767in}{1.366042in}}%
\pgfpathlineto{\pgfqpoint{6.165939in}{1.366278in}}%
\pgfpathlineto{\pgfqpoint{6.169111in}{1.366027in}}%
\pgfpathlineto{\pgfqpoint{6.172283in}{1.366202in}}%
\pgfpathlineto{\pgfqpoint{6.175455in}{1.366878in}}%
\pgfpathlineto{\pgfqpoint{6.178627in}{1.366806in}}%
\pgfpathlineto{\pgfqpoint{6.181800in}{1.366592in}}%
\pgfpathlineto{\pgfqpoint{6.184972in}{1.366664in}}%
\pgfpathlineto{\pgfqpoint{6.188144in}{1.366853in}}%
\pgfpathlineto{\pgfqpoint{6.191316in}{1.367158in}}%
\pgfpathlineto{\pgfqpoint{6.194488in}{1.366843in}}%
\pgfpathlineto{\pgfqpoint{6.197660in}{1.367048in}}%
\pgfpathlineto{\pgfqpoint{6.200832in}{1.367510in}}%
\pgfpathlineto{\pgfqpoint{6.204004in}{1.367496in}}%
\pgfpathlineto{\pgfqpoint{6.207176in}{1.367055in}}%
\pgfpathlineto{\pgfqpoint{6.210348in}{1.366716in}}%
\pgfpathlineto{\pgfqpoint{6.213520in}{1.366516in}}%
\pgfpathlineto{\pgfqpoint{6.216692in}{1.366716in}}%
\pgfpathlineto{\pgfqpoint{6.219864in}{1.366948in}}%
\pgfpathlineto{\pgfqpoint{6.223036in}{1.367070in}}%
\pgfpathlineto{\pgfqpoint{6.226208in}{1.367089in}}%
\pgfpathlineto{\pgfqpoint{6.229380in}{1.367412in}}%
\pgfpathlineto{\pgfqpoint{6.232552in}{1.367552in}}%
\pgfpathlineto{\pgfqpoint{6.235724in}{1.367436in}}%
\pgfpathlineto{\pgfqpoint{6.238896in}{1.367049in}}%
\pgfpathlineto{\pgfqpoint{6.242068in}{1.366929in}}%
\pgfpathlineto{\pgfqpoint{6.245240in}{1.366981in}}%
\pgfpathlineto{\pgfqpoint{6.248412in}{1.367108in}}%
\pgfpathlineto{\pgfqpoint{6.251584in}{1.366835in}}%
\pgfpathlineto{\pgfqpoint{6.254756in}{1.366910in}}%
\pgfpathlineto{\pgfqpoint{6.257928in}{1.367164in}}%
\pgfpathlineto{\pgfqpoint{6.261101in}{1.367287in}}%
\pgfpathlineto{\pgfqpoint{6.264273in}{1.367273in}}%
\pgfpathlineto{\pgfqpoint{6.267445in}{1.367938in}}%
\pgfpathlineto{\pgfqpoint{6.270617in}{1.367987in}}%
\pgfpathlineto{\pgfqpoint{6.273789in}{1.367978in}}%
\pgfpathlineto{\pgfqpoint{6.276961in}{1.368467in}}%
\pgfpathlineto{\pgfqpoint{6.280133in}{1.368630in}}%
\pgfpathlineto{\pgfqpoint{6.283305in}{1.368367in}}%
\pgfpathlineto{\pgfqpoint{6.286477in}{1.368232in}}%
\pgfpathlineto{\pgfqpoint{6.289649in}{1.368283in}}%
\pgfpathlineto{\pgfqpoint{6.292821in}{1.368080in}}%
\pgfpathlineto{\pgfqpoint{6.295993in}{1.367941in}}%
\pgfpathlineto{\pgfqpoint{6.299165in}{1.368040in}}%
\pgfpathlineto{\pgfqpoint{6.302337in}{1.368141in}}%
\pgfpathlineto{\pgfqpoint{6.305509in}{1.368120in}}%
\pgfpathlineto{\pgfqpoint{6.308681in}{1.367718in}}%
\pgfpathlineto{\pgfqpoint{6.311853in}{1.367615in}}%
\pgfpathlineto{\pgfqpoint{6.315025in}{1.367283in}}%
\pgfpathlineto{\pgfqpoint{6.318197in}{1.367155in}}%
\pgfpathlineto{\pgfqpoint{6.321369in}{1.367040in}}%
\pgfpathlineto{\pgfqpoint{6.324541in}{1.366944in}}%
\pgfpathlineto{\pgfqpoint{6.327713in}{1.367232in}}%
\pgfpathlineto{\pgfqpoint{6.330885in}{1.367210in}}%
\pgfpathlineto{\pgfqpoint{6.334057in}{1.367440in}}%
\pgfpathlineto{\pgfqpoint{6.337230in}{1.368046in}}%
\pgfpathlineto{\pgfqpoint{6.340402in}{1.367969in}}%
\pgfpathlineto{\pgfqpoint{6.343574in}{1.367704in}}%
\pgfpathlineto{\pgfqpoint{6.346746in}{1.367441in}}%
\pgfpathlineto{\pgfqpoint{6.349918in}{1.367510in}}%
\pgfpathlineto{\pgfqpoint{6.353090in}{1.367441in}}%
\pgfpathlineto{\pgfqpoint{6.356262in}{1.367513in}}%
\pgfpathlineto{\pgfqpoint{6.359434in}{1.367047in}}%
\pgfpathlineto{\pgfqpoint{6.362606in}{1.366809in}}%
\pgfpathlineto{\pgfqpoint{6.365778in}{1.366934in}}%
\pgfpathlineto{\pgfqpoint{6.368950in}{1.367460in}}%
\pgfpathlineto{\pgfqpoint{6.372122in}{1.366896in}}%
\pgfpathlineto{\pgfqpoint{6.375294in}{1.367187in}}%
\pgfpathlineto{\pgfqpoint{6.378466in}{1.366826in}}%
\pgfpathlineto{\pgfqpoint{6.381638in}{1.366950in}}%
\pgfpathlineto{\pgfqpoint{6.384810in}{1.367305in}}%
\pgfpathlineto{\pgfqpoint{6.387982in}{1.367282in}}%
\pgfpathlineto{\pgfqpoint{6.391154in}{1.367158in}}%
\pgfpathlineto{\pgfqpoint{6.394326in}{1.366816in}}%
\pgfpathlineto{\pgfqpoint{6.397498in}{1.366699in}}%
\pgfpathlineto{\pgfqpoint{6.400670in}{1.366431in}}%
\pgfpathlineto{\pgfqpoint{6.403842in}{1.366677in}}%
\pgfpathlineto{\pgfqpoint{6.407014in}{1.366345in}}%
\pgfpathlineto{\pgfqpoint{6.410186in}{1.366183in}}%
\pgfpathlineto{\pgfqpoint{6.413358in}{1.366184in}}%
\pgfpathlineto{\pgfqpoint{6.416531in}{1.367000in}}%
\pgfpathlineto{\pgfqpoint{6.419703in}{1.366517in}}%
\pgfpathlineto{\pgfqpoint{6.422875in}{1.366468in}}%
\pgfpathlineto{\pgfqpoint{6.426047in}{1.366507in}}%
\pgfpathlineto{\pgfqpoint{6.429219in}{1.366204in}}%
\pgfpathlineto{\pgfqpoint{6.432391in}{1.366358in}}%
\pgfpathlineto{\pgfqpoint{6.435563in}{1.366284in}}%
\pgfpathlineto{\pgfqpoint{6.438735in}{1.366285in}}%
\pgfpathlineto{\pgfqpoint{6.441907in}{1.365788in}}%
\pgfpathlineto{\pgfqpoint{6.445079in}{1.365666in}}%
\pgfpathlineto{\pgfqpoint{6.448251in}{1.365743in}}%
\pgfpathlineto{\pgfqpoint{6.451423in}{1.365255in}}%
\pgfpathlineto{\pgfqpoint{6.454595in}{1.365257in}}%
\pgfpathlineto{\pgfqpoint{6.457767in}{1.365811in}}%
\pgfpathlineto{\pgfqpoint{6.460939in}{1.366378in}}%
\pgfpathlineto{\pgfqpoint{6.464111in}{1.366695in}}%
\pgfpathlineto{\pgfqpoint{6.467283in}{1.367126in}}%
\pgfpathlineto{\pgfqpoint{6.470455in}{1.366858in}}%
\pgfpathlineto{\pgfqpoint{6.473627in}{1.366740in}}%
\pgfpathlineto{\pgfqpoint{6.476799in}{1.367030in}}%
\pgfpathlineto{\pgfqpoint{6.479971in}{1.367125in}}%
\pgfpathlineto{\pgfqpoint{6.483143in}{1.367437in}}%
\pgfpathlineto{\pgfqpoint{6.486315in}{1.367862in}}%
\pgfpathlineto{\pgfqpoint{6.489487in}{1.367718in}}%
\pgfpathlineto{\pgfqpoint{6.492659in}{1.368006in}}%
\pgfpathlineto{\pgfqpoint{6.495832in}{1.368028in}}%
\pgfpathlineto{\pgfqpoint{6.499004in}{1.368046in}}%
\pgfpathlineto{\pgfqpoint{6.502176in}{1.367836in}}%
\pgfpathlineto{\pgfqpoint{6.505348in}{1.368032in}}%
\pgfpathlineto{\pgfqpoint{6.508520in}{1.368291in}}%
\pgfpathlineto{\pgfqpoint{6.511692in}{1.368279in}}%
\pgfpathlineto{\pgfqpoint{6.514864in}{1.368442in}}%
\pgfpathlineto{\pgfqpoint{6.518036in}{1.367821in}}%
\pgfpathlineto{\pgfqpoint{6.521208in}{1.367400in}}%
\pgfpathlineto{\pgfqpoint{6.524380in}{1.367322in}}%
\pgfpathlineto{\pgfqpoint{6.527552in}{1.367300in}}%
\pgfpathlineto{\pgfqpoint{6.530724in}{1.367437in}}%
\pgfpathlineto{\pgfqpoint{6.533896in}{1.368204in}}%
\pgfpathlineto{\pgfqpoint{6.537068in}{1.368763in}}%
\pgfpathlineto{\pgfqpoint{6.540240in}{1.368607in}}%
\pgfpathlineto{\pgfqpoint{6.543412in}{1.368282in}}%
\pgfpathlineto{\pgfqpoint{6.546584in}{1.367843in}}%
\pgfpathlineto{\pgfqpoint{6.549756in}{1.367393in}}%
\pgfpathlineto{\pgfqpoint{6.552928in}{1.367143in}}%
\pgfpathlineto{\pgfqpoint{6.556100in}{1.367202in}}%
\pgfpathlineto{\pgfqpoint{6.559272in}{1.367158in}}%
\pgfpathlineto{\pgfqpoint{6.562444in}{1.366633in}}%
\pgfpathlineto{\pgfqpoint{6.565616in}{1.367063in}}%
\pgfpathlineto{\pgfqpoint{6.568788in}{1.366759in}}%
\pgfpathlineto{\pgfqpoint{6.571961in}{1.366719in}}%
\pgfpathlineto{\pgfqpoint{6.575133in}{1.367112in}}%
\pgfpathlineto{\pgfqpoint{6.578305in}{1.367372in}}%
\pgfpathlineto{\pgfqpoint{6.581477in}{1.367343in}}%
\pgfpathlineto{\pgfqpoint{6.584649in}{1.367987in}}%
\pgfpathlineto{\pgfqpoint{6.587821in}{1.368453in}}%
\pgfpathlineto{\pgfqpoint{6.590993in}{1.368769in}}%
\pgfpathlineto{\pgfqpoint{6.594165in}{1.368969in}}%
\pgfpathlineto{\pgfqpoint{6.597337in}{1.368971in}}%
\pgfpathlineto{\pgfqpoint{6.600509in}{1.368384in}}%
\pgfpathlineto{\pgfqpoint{6.603681in}{1.368316in}}%
\pgfpathlineto{\pgfqpoint{6.606853in}{1.367969in}}%
\pgfpathlineto{\pgfqpoint{6.610025in}{1.367456in}}%
\pgfpathlineto{\pgfqpoint{6.613197in}{1.366411in}}%
\pgfpathlineto{\pgfqpoint{6.616369in}{1.366360in}}%
\pgfpathlineto{\pgfqpoint{6.619541in}{1.366844in}}%
\pgfpathlineto{\pgfqpoint{6.622713in}{1.367189in}}%
\pgfpathlineto{\pgfqpoint{6.625885in}{1.367639in}}%
\pgfpathlineto{\pgfqpoint{6.629057in}{1.367452in}}%
\pgfpathlineto{\pgfqpoint{6.632229in}{1.367505in}}%
\pgfpathlineto{\pgfqpoint{6.635401in}{1.367772in}}%
\pgfpathlineto{\pgfqpoint{6.638573in}{1.367767in}}%
\pgfpathlineto{\pgfqpoint{6.641745in}{1.368055in}}%
\pgfpathlineto{\pgfqpoint{6.644917in}{1.368366in}}%
\pgfpathlineto{\pgfqpoint{6.648089in}{1.368617in}}%
\pgfpathlineto{\pgfqpoint{6.651262in}{1.368542in}}%
\pgfpathlineto{\pgfqpoint{6.654434in}{1.368497in}}%
\pgfpathlineto{\pgfqpoint{6.657606in}{1.368351in}}%
\pgfpathlineto{\pgfqpoint{6.660778in}{1.368068in}}%
\pgfpathlineto{\pgfqpoint{6.663950in}{1.368013in}}%
\pgfpathlineto{\pgfqpoint{6.667122in}{1.367846in}}%
\pgfpathlineto{\pgfqpoint{6.670294in}{1.367830in}}%
\pgfpathlineto{\pgfqpoint{6.673466in}{1.367891in}}%
\pgfpathlineto{\pgfqpoint{6.676638in}{1.368328in}}%
\pgfpathlineto{\pgfqpoint{6.679810in}{1.368410in}}%
\pgfpathlineto{\pgfqpoint{6.682982in}{1.368785in}}%
\pgfpathlineto{\pgfqpoint{6.686154in}{1.368647in}}%
\pgfpathlineto{\pgfqpoint{6.689326in}{1.368331in}}%
\pgfpathlineto{\pgfqpoint{6.692498in}{1.368463in}}%
\pgfpathlineto{\pgfqpoint{6.695670in}{1.369002in}}%
\pgfpathlineto{\pgfqpoint{6.698842in}{1.368948in}}%
\pgfpathlineto{\pgfqpoint{6.702014in}{1.368939in}}%
\pgfpathlineto{\pgfqpoint{6.705186in}{1.368832in}}%
\pgfpathlineto{\pgfqpoint{6.708358in}{1.369772in}}%
\pgfpathlineto{\pgfqpoint{6.711530in}{1.369020in}}%
\pgfpathlineto{\pgfqpoint{6.714702in}{1.368878in}}%
\pgfpathlineto{\pgfqpoint{6.717874in}{1.368801in}}%
\pgfpathlineto{\pgfqpoint{6.721046in}{1.368704in}}%
\pgfpathlineto{\pgfqpoint{6.724218in}{1.368349in}}%
\pgfpathlineto{\pgfqpoint{6.727391in}{1.367870in}}%
\pgfpathlineto{\pgfqpoint{6.730563in}{1.367570in}}%
\pgfpathlineto{\pgfqpoint{6.733735in}{1.367397in}}%
\pgfpathlineto{\pgfqpoint{6.736907in}{1.367656in}}%
\pgfpathlineto{\pgfqpoint{6.740079in}{1.367490in}}%
\pgfpathlineto{\pgfqpoint{6.743251in}{1.367510in}}%
\pgfpathlineto{\pgfqpoint{6.746423in}{1.366892in}}%
\pgfpathlineto{\pgfqpoint{6.749595in}{1.366723in}}%
\pgfpathlineto{\pgfqpoint{6.752767in}{1.366630in}}%
\pgfpathlineto{\pgfqpoint{6.755939in}{1.366439in}}%
\pgfpathlineto{\pgfqpoint{6.759111in}{1.366788in}}%
\pgfpathlineto{\pgfqpoint{6.762283in}{1.367300in}}%
\pgfpathlineto{\pgfqpoint{6.765455in}{1.367644in}}%
\pgfpathlineto{\pgfqpoint{6.768627in}{1.367752in}}%
\pgfpathlineto{\pgfqpoint{6.771799in}{1.367773in}}%
\pgfpathlineto{\pgfqpoint{6.774971in}{1.367923in}}%
\pgfpathlineto{\pgfqpoint{6.778143in}{1.368160in}}%
\pgfpathlineto{\pgfqpoint{6.781315in}{1.368371in}}%
\pgfpathlineto{\pgfqpoint{6.784487in}{1.368559in}}%
\pgfpathlineto{\pgfqpoint{6.787659in}{1.368601in}}%
\pgfpathlineto{\pgfqpoint{6.790831in}{1.368679in}}%
\pgfpathlineto{\pgfqpoint{6.794003in}{1.368843in}}%
\pgfpathlineto{\pgfqpoint{6.797175in}{1.369036in}}%
\pgfpathlineto{\pgfqpoint{6.800347in}{1.368629in}}%
\pgfpathlineto{\pgfqpoint{6.803519in}{1.368540in}}%
\pgfpathlineto{\pgfqpoint{6.806692in}{1.368791in}}%
\pgfpathlineto{\pgfqpoint{6.809864in}{1.369000in}}%
\pgfpathlineto{\pgfqpoint{6.813036in}{1.368896in}}%
\pgfpathlineto{\pgfqpoint{6.816208in}{1.368486in}}%
\pgfpathlineto{\pgfqpoint{6.819380in}{1.368529in}}%
\pgfpathlineto{\pgfqpoint{6.822552in}{1.368686in}}%
\pgfpathlineto{\pgfqpoint{6.825724in}{1.368435in}}%
\pgfpathlineto{\pgfqpoint{6.828896in}{1.368642in}}%
\pgfpathlineto{\pgfqpoint{6.832068in}{1.369216in}}%
\pgfpathlineto{\pgfqpoint{6.835240in}{1.369149in}}%
\pgfpathlineto{\pgfqpoint{6.838412in}{1.369387in}}%
\pgfpathlineto{\pgfqpoint{6.841584in}{1.369194in}}%
\pgfpathlineto{\pgfqpoint{6.844756in}{1.369443in}}%
\pgfpathlineto{\pgfqpoint{6.847928in}{1.369396in}}%
\pgfpathlineto{\pgfqpoint{6.851100in}{1.369509in}}%
\pgfpathlineto{\pgfqpoint{6.854272in}{1.369409in}}%
\pgfpathlineto{\pgfqpoint{6.857444in}{1.369242in}}%
\pgfpathlineto{\pgfqpoint{6.860616in}{1.368890in}}%
\pgfpathlineto{\pgfqpoint{6.863788in}{1.368802in}}%
\pgfpathlineto{\pgfqpoint{6.866960in}{1.368681in}}%
\pgfpathlineto{\pgfqpoint{6.870132in}{1.368644in}}%
\pgfpathlineto{\pgfqpoint{6.873304in}{1.368299in}}%
\pgfpathlineto{\pgfqpoint{6.876476in}{1.367784in}}%
\pgfpathlineto{\pgfqpoint{6.879648in}{1.368078in}}%
\pgfpathlineto{\pgfqpoint{6.882820in}{1.367854in}}%
\pgfpathlineto{\pgfqpoint{6.885993in}{1.368007in}}%
\pgfpathlineto{\pgfqpoint{6.889165in}{1.367960in}}%
\pgfpathlineto{\pgfqpoint{6.892337in}{1.368600in}}%
\pgfpathlineto{\pgfqpoint{6.895509in}{1.368344in}}%
\pgfpathlineto{\pgfqpoint{6.898681in}{1.368638in}}%
\pgfpathlineto{\pgfqpoint{6.901853in}{1.368691in}}%
\pgfpathlineto{\pgfqpoint{6.905025in}{1.368199in}}%
\pgfpathlineto{\pgfqpoint{6.908197in}{1.368418in}}%
\pgfpathlineto{\pgfqpoint{6.911369in}{1.367778in}}%
\pgfpathlineto{\pgfqpoint{6.914541in}{1.367942in}}%
\pgfpathlineto{\pgfqpoint{6.917713in}{1.368120in}}%
\pgfpathlineto{\pgfqpoint{6.920885in}{1.368077in}}%
\pgfpathlineto{\pgfqpoint{6.924057in}{1.367887in}}%
\pgfpathlineto{\pgfqpoint{6.927229in}{1.367961in}}%
\pgfpathlineto{\pgfqpoint{6.930401in}{1.368225in}}%
\pgfpathlineto{\pgfqpoint{6.933573in}{1.368676in}}%
\pgfpathlineto{\pgfqpoint{6.936745in}{1.368362in}}%
\pgfpathlineto{\pgfqpoint{6.939917in}{1.368463in}}%
\pgfpathlineto{\pgfqpoint{6.943089in}{1.368345in}}%
\pgfpathlineto{\pgfqpoint{6.946261in}{1.368388in}}%
\pgfpathlineto{\pgfqpoint{6.949433in}{1.368356in}}%
\pgfpathlineto{\pgfqpoint{6.952605in}{1.367994in}}%
\pgfpathlineto{\pgfqpoint{6.955777in}{1.367820in}}%
\pgfpathlineto{\pgfqpoint{6.958949in}{1.367923in}}%
\pgfpathlineto{\pgfqpoint{6.962122in}{1.367790in}}%
\pgfpathlineto{\pgfqpoint{6.965294in}{1.367874in}}%
\pgfpathlineto{\pgfqpoint{6.968466in}{1.367823in}}%
\pgfpathlineto{\pgfqpoint{6.971638in}{1.368075in}}%
\pgfpathlineto{\pgfqpoint{6.974810in}{1.368142in}}%
\pgfpathlineto{\pgfqpoint{6.977982in}{1.368116in}}%
\pgfpathlineto{\pgfqpoint{6.981154in}{1.368135in}}%
\pgfpathlineto{\pgfqpoint{6.984326in}{1.368342in}}%
\pgfpathlineto{\pgfqpoint{6.987498in}{1.368183in}}%
\pgfpathlineto{\pgfqpoint{6.990670in}{1.368068in}}%
\pgfpathlineto{\pgfqpoint{6.993842in}{1.368002in}}%
\pgfpathlineto{\pgfqpoint{6.997014in}{1.367949in}}%
\pgfpathlineto{\pgfqpoint{7.000186in}{1.367495in}}%
\pgfpathlineto{\pgfqpoint{7.003358in}{1.367354in}}%
\pgfpathlineto{\pgfqpoint{7.006530in}{1.367634in}}%
\pgfpathlineto{\pgfqpoint{7.009702in}{1.367949in}}%
\pgfpathlineto{\pgfqpoint{7.012874in}{1.367842in}}%
\pgfpathlineto{\pgfqpoint{7.016046in}{1.368046in}}%
\pgfpathlineto{\pgfqpoint{7.019218in}{1.367757in}}%
\pgfpathlineto{\pgfqpoint{7.022390in}{1.367769in}}%
\pgfpathlineto{\pgfqpoint{7.025562in}{1.367884in}}%
\pgfpathlineto{\pgfqpoint{7.028734in}{1.367873in}}%
\pgfpathlineto{\pgfqpoint{7.031906in}{1.367758in}}%
\pgfpathlineto{\pgfqpoint{7.035078in}{1.367990in}}%
\pgfpathlineto{\pgfqpoint{7.038250in}{1.368287in}}%
\pgfpathlineto{\pgfqpoint{7.041423in}{1.368328in}}%
\pgfpathlineto{\pgfqpoint{7.044595in}{1.368428in}}%
\pgfpathlineto{\pgfqpoint{7.047767in}{1.368752in}}%
\pgfpathlineto{\pgfqpoint{7.050939in}{1.368806in}}%
\pgfpathlineto{\pgfqpoint{7.054111in}{1.368205in}}%
\pgfpathlineto{\pgfqpoint{7.057283in}{1.368364in}}%
\pgfpathlineto{\pgfqpoint{7.060455in}{1.368500in}}%
\pgfpathlineto{\pgfqpoint{7.063627in}{1.368624in}}%
\pgfpathlineto{\pgfqpoint{7.066799in}{1.368446in}}%
\pgfpathlineto{\pgfqpoint{7.069971in}{1.368297in}}%
\pgfpathlineto{\pgfqpoint{7.073143in}{1.368309in}}%
\pgfpathlineto{\pgfqpoint{7.076315in}{1.368157in}}%
\pgfpathlineto{\pgfqpoint{7.079487in}{1.368268in}}%
\pgfpathlineto{\pgfqpoint{7.082659in}{1.368291in}}%
\pgfpathlineto{\pgfqpoint{7.085831in}{1.368477in}}%
\pgfpathlineto{\pgfqpoint{7.089003in}{1.368710in}}%
\pgfpathlineto{\pgfqpoint{7.092175in}{1.369049in}}%
\pgfpathlineto{\pgfqpoint{7.095347in}{1.368615in}}%
\pgfpathlineto{\pgfqpoint{7.098519in}{1.368383in}}%
\pgfpathlineto{\pgfqpoint{7.101691in}{1.368457in}}%
\pgfpathlineto{\pgfqpoint{7.104863in}{1.368445in}}%
\pgfpathlineto{\pgfqpoint{7.108035in}{1.368674in}}%
\pgfpathlineto{\pgfqpoint{7.111207in}{1.368454in}}%
\pgfpathlineto{\pgfqpoint{7.114379in}{1.368749in}}%
\pgfpathlineto{\pgfqpoint{7.117551in}{1.368828in}}%
\pgfpathlineto{\pgfqpoint{7.120724in}{1.368572in}}%
\pgfpathlineto{\pgfqpoint{7.123896in}{1.368770in}}%
\pgfpathlineto{\pgfqpoint{7.127068in}{1.368530in}}%
\pgfpathlineto{\pgfqpoint{7.130240in}{1.368594in}}%
\pgfpathlineto{\pgfqpoint{7.133412in}{1.368493in}}%
\pgfpathlineto{\pgfqpoint{7.136584in}{1.367636in}}%
\pgfpathlineto{\pgfqpoint{7.139756in}{1.367905in}}%
\pgfpathlineto{\pgfqpoint{7.142928in}{1.368165in}}%
\pgfpathlineto{\pgfqpoint{7.146100in}{1.367761in}}%
\pgfpathlineto{\pgfqpoint{7.149272in}{1.368484in}}%
\pgfpathlineto{\pgfqpoint{7.152444in}{1.368379in}}%
\pgfpathlineto{\pgfqpoint{7.155616in}{1.368081in}}%
\pgfpathlineto{\pgfqpoint{7.158788in}{1.368277in}}%
\pgfpathlineto{\pgfqpoint{7.161960in}{1.367847in}}%
\pgfpathlineto{\pgfqpoint{7.165132in}{1.367817in}}%
\pgfpathlineto{\pgfqpoint{7.168304in}{1.367853in}}%
\pgfpathlineto{\pgfqpoint{7.171476in}{1.368271in}}%
\pgfpathlineto{\pgfqpoint{7.174648in}{1.368394in}}%
\pgfpathlineto{\pgfqpoint{7.177820in}{1.368439in}}%
\pgfpathlineto{\pgfqpoint{7.180992in}{1.368197in}}%
\pgfpathlineto{\pgfqpoint{7.184164in}{1.367952in}}%
\pgfpathlineto{\pgfqpoint{7.187336in}{1.368192in}}%
\pgfpathlineto{\pgfqpoint{7.190508in}{1.368273in}}%
\pgfpathlineto{\pgfqpoint{7.193680in}{1.368538in}}%
\pgfpathlineto{\pgfqpoint{7.196853in}{1.368216in}}%
\pgfpathlineto{\pgfqpoint{7.200025in}{1.368272in}}%
\pgfpathlineto{\pgfqpoint{7.203197in}{1.368323in}}%
\pgfpathlineto{\pgfqpoint{7.206369in}{1.368528in}}%
\pgfpathlineto{\pgfqpoint{7.209541in}{1.368852in}}%
\pgfpathlineto{\pgfqpoint{7.212713in}{1.368226in}}%
\pgfpathlineto{\pgfqpoint{7.215885in}{1.367827in}}%
\pgfpathlineto{\pgfqpoint{7.219057in}{1.367717in}}%
\pgfpathlineto{\pgfqpoint{7.222229in}{1.367431in}}%
\pgfpathlineto{\pgfqpoint{7.225401in}{1.367668in}}%
\pgfpathlineto{\pgfqpoint{7.228573in}{1.367454in}}%
\pgfpathlineto{\pgfqpoint{7.231745in}{1.367696in}}%
\pgfpathlineto{\pgfqpoint{7.234917in}{1.368312in}}%
\pgfpathlineto{\pgfqpoint{7.238089in}{1.368746in}}%
\pgfpathlineto{\pgfqpoint{7.241261in}{1.368525in}}%
\pgfpathlineto{\pgfqpoint{7.244433in}{1.368419in}}%
\pgfpathlineto{\pgfqpoint{7.247605in}{1.367824in}}%
\pgfpathlineto{\pgfqpoint{7.250777in}{1.368129in}}%
\pgfpathlineto{\pgfqpoint{7.253949in}{1.368089in}}%
\pgfpathlineto{\pgfqpoint{7.257121in}{1.367755in}}%
\pgfpathlineto{\pgfqpoint{7.260293in}{1.368166in}}%
\pgfpathlineto{\pgfqpoint{7.263465in}{1.368271in}}%
\pgfpathlineto{\pgfqpoint{7.266637in}{1.367978in}}%
\pgfpathlineto{\pgfqpoint{7.269809in}{1.368210in}}%
\pgfpathlineto{\pgfqpoint{7.272981in}{1.368211in}}%
\pgfpathlineto{\pgfqpoint{7.276154in}{1.368270in}}%
\pgfpathlineto{\pgfqpoint{7.279326in}{1.368243in}}%
\pgfpathlineto{\pgfqpoint{7.282498in}{1.368593in}}%
\pgfpathlineto{\pgfqpoint{7.285670in}{1.368631in}}%
\pgfpathlineto{\pgfqpoint{7.288842in}{1.368716in}}%
\pgfpathlineto{\pgfqpoint{7.292014in}{1.369095in}}%
\pgfpathlineto{\pgfqpoint{7.295186in}{1.369165in}}%
\pgfpathlineto{\pgfqpoint{7.298358in}{1.369627in}}%
\pgfpathlineto{\pgfqpoint{7.301530in}{1.369992in}}%
\pgfpathlineto{\pgfqpoint{7.304702in}{1.370207in}}%
\pgfpathlineto{\pgfqpoint{7.307874in}{1.369349in}}%
\pgfpathlineto{\pgfqpoint{7.311046in}{1.368612in}}%
\pgfpathlineto{\pgfqpoint{7.314218in}{1.368277in}}%
\pgfpathlineto{\pgfqpoint{7.317390in}{1.367841in}}%
\pgfpathlineto{\pgfqpoint{7.320562in}{1.367779in}}%
\pgfpathlineto{\pgfqpoint{7.323734in}{1.368050in}}%
\pgfpathlineto{\pgfqpoint{7.326906in}{1.367919in}}%
\pgfpathlineto{\pgfqpoint{7.330078in}{1.367031in}}%
\pgfpathlineto{\pgfqpoint{7.333250in}{1.366753in}}%
\pgfpathlineto{\pgfqpoint{7.336422in}{1.366972in}}%
\pgfpathlineto{\pgfqpoint{7.339594in}{1.366935in}}%
\pgfpathlineto{\pgfqpoint{7.342766in}{1.366732in}}%
\pgfpathlineto{\pgfqpoint{7.345938in}{1.366792in}}%
\pgfpathlineto{\pgfqpoint{7.349110in}{1.366816in}}%
\pgfpathlineto{\pgfqpoint{7.352282in}{1.367365in}}%
\pgfpathlineto{\pgfqpoint{7.355455in}{1.367596in}}%
\pgfpathlineto{\pgfqpoint{7.358627in}{1.368093in}}%
\pgfpathlineto{\pgfqpoint{7.361799in}{1.368315in}}%
\pgfpathlineto{\pgfqpoint{7.364971in}{1.368369in}}%
\pgfpathlineto{\pgfqpoint{7.368143in}{1.368318in}}%
\pgfpathlineto{\pgfqpoint{7.371315in}{1.368400in}}%
\pgfpathlineto{\pgfqpoint{7.374487in}{1.368428in}}%
\pgfpathlineto{\pgfqpoint{7.377659in}{1.368233in}}%
\pgfpathlineto{\pgfqpoint{7.380831in}{1.368209in}}%
\pgfpathlineto{\pgfqpoint{7.384003in}{1.367739in}}%
\pgfpathlineto{\pgfqpoint{7.387175in}{1.367666in}}%
\pgfpathlineto{\pgfqpoint{7.390347in}{1.367752in}}%
\pgfpathlineto{\pgfqpoint{7.393519in}{1.367491in}}%
\pgfpathlineto{\pgfqpoint{7.396691in}{1.367270in}}%
\pgfpathlineto{\pgfqpoint{7.399863in}{1.367287in}}%
\pgfpathlineto{\pgfqpoint{7.403035in}{1.367205in}}%
\pgfpathlineto{\pgfqpoint{7.406207in}{1.367002in}}%
\pgfpathlineto{\pgfqpoint{7.409379in}{1.367142in}}%
\pgfpathlineto{\pgfqpoint{7.412551in}{1.366582in}}%
\pgfpathlineto{\pgfqpoint{7.415723in}{1.366290in}}%
\pgfpathlineto{\pgfqpoint{7.418895in}{1.366224in}}%
\pgfpathlineto{\pgfqpoint{7.422067in}{1.366271in}}%
\pgfpathlineto{\pgfqpoint{7.425239in}{1.366390in}}%
\pgfpathlineto{\pgfqpoint{7.428411in}{1.366550in}}%
\pgfpathlineto{\pgfqpoint{7.431584in}{1.366386in}}%
\pgfpathlineto{\pgfqpoint{7.434756in}{1.366512in}}%
\pgfpathlineto{\pgfqpoint{7.437928in}{1.366421in}}%
\pgfpathlineto{\pgfqpoint{7.441100in}{1.366305in}}%
\pgfpathlineto{\pgfqpoint{7.444272in}{1.366289in}}%
\pgfpathlineto{\pgfqpoint{7.447444in}{1.366563in}}%
\pgfpathlineto{\pgfqpoint{7.450616in}{1.367267in}}%
\pgfpathlineto{\pgfqpoint{7.453788in}{1.367339in}}%
\pgfpathlineto{\pgfqpoint{7.456960in}{1.366915in}}%
\pgfpathlineto{\pgfqpoint{7.460132in}{1.366978in}}%
\pgfpathlineto{\pgfqpoint{7.463304in}{1.366923in}}%
\pgfpathlineto{\pgfqpoint{7.466476in}{1.366875in}}%
\pgfpathlineto{\pgfqpoint{7.469648in}{1.366635in}}%
\pgfpathlineto{\pgfqpoint{7.472820in}{1.366790in}}%
\pgfpathlineto{\pgfqpoint{7.475992in}{1.366934in}}%
\pgfpathlineto{\pgfqpoint{7.479164in}{1.366834in}}%
\pgfpathlineto{\pgfqpoint{7.482336in}{1.367191in}}%
\pgfpathlineto{\pgfqpoint{7.485508in}{1.367407in}}%
\pgfpathlineto{\pgfqpoint{7.488680in}{1.367339in}}%
\pgfpathlineto{\pgfqpoint{7.491852in}{1.367267in}}%
\pgfpathlineto{\pgfqpoint{7.495024in}{1.367552in}}%
\pgfpathlineto{\pgfqpoint{7.498196in}{1.367432in}}%
\pgfpathlineto{\pgfqpoint{7.501368in}{1.367296in}}%
\pgfpathlineto{\pgfqpoint{7.504540in}{1.368102in}}%
\pgfpathlineto{\pgfqpoint{7.507712in}{1.367985in}}%
\pgfpathlineto{\pgfqpoint{7.510885in}{1.367421in}}%
\pgfpathlineto{\pgfqpoint{7.514057in}{1.367130in}}%
\pgfpathlineto{\pgfqpoint{7.517229in}{1.366828in}}%
\pgfpathlineto{\pgfqpoint{7.520401in}{1.366570in}}%
\pgfpathlineto{\pgfqpoint{7.523573in}{1.366746in}}%
\pgfpathlineto{\pgfqpoint{7.526745in}{1.366753in}}%
\pgfpathlineto{\pgfqpoint{7.529917in}{1.366612in}}%
\pgfpathlineto{\pgfqpoint{7.533089in}{1.366457in}}%
\pgfpathlineto{\pgfqpoint{7.536261in}{1.366664in}}%
\pgfpathlineto{\pgfqpoint{7.539433in}{1.366997in}}%
\pgfpathlineto{\pgfqpoint{7.542605in}{1.367090in}}%
\pgfpathlineto{\pgfqpoint{7.545777in}{1.367006in}}%
\pgfpathlineto{\pgfqpoint{7.548949in}{1.367083in}}%
\pgfpathlineto{\pgfqpoint{7.552121in}{1.367190in}}%
\pgfpathlineto{\pgfqpoint{7.555293in}{1.367698in}}%
\pgfpathlineto{\pgfqpoint{7.558465in}{1.367437in}}%
\pgfpathlineto{\pgfqpoint{7.561637in}{1.367029in}}%
\pgfpathlineto{\pgfqpoint{7.564809in}{1.366874in}}%
\pgfpathlineto{\pgfqpoint{7.567981in}{1.366701in}}%
\pgfpathlineto{\pgfqpoint{7.571153in}{1.366598in}}%
\pgfpathlineto{\pgfqpoint{7.574325in}{1.366873in}}%
\pgfpathlineto{\pgfqpoint{7.577497in}{1.367102in}}%
\pgfpathlineto{\pgfqpoint{7.580669in}{1.367495in}}%
\pgfpathlineto{\pgfqpoint{7.583841in}{1.368106in}}%
\pgfpathlineto{\pgfqpoint{7.587013in}{1.368127in}}%
\pgfpathlineto{\pgfqpoint{7.590186in}{1.367834in}}%
\pgfpathlineto{\pgfqpoint{7.593358in}{1.368082in}}%
\pgfpathlineto{\pgfqpoint{7.596530in}{1.367651in}}%
\pgfpathlineto{\pgfqpoint{7.599702in}{1.367307in}}%
\pgfpathlineto{\pgfqpoint{7.602874in}{1.367531in}}%
\pgfpathlineto{\pgfqpoint{7.606046in}{1.367528in}}%
\pgfpathlineto{\pgfqpoint{7.609218in}{1.367665in}}%
\pgfpathlineto{\pgfqpoint{7.612390in}{1.367576in}}%
\pgfpathlineto{\pgfqpoint{7.615562in}{1.367485in}}%
\pgfpathlineto{\pgfqpoint{7.618734in}{1.366964in}}%
\pgfpathlineto{\pgfqpoint{7.621906in}{1.367320in}}%
\pgfpathlineto{\pgfqpoint{7.625078in}{1.367489in}}%
\pgfpathlineto{\pgfqpoint{7.628250in}{1.367622in}}%
\pgfpathlineto{\pgfqpoint{7.631422in}{1.367789in}}%
\pgfpathlineto{\pgfqpoint{7.634594in}{1.367836in}}%
\pgfpathlineto{\pgfqpoint{7.637766in}{1.367368in}}%
\pgfpathlineto{\pgfqpoint{7.640938in}{1.367128in}}%
\pgfpathlineto{\pgfqpoint{7.644110in}{1.367277in}}%
\pgfpathlineto{\pgfqpoint{7.647282in}{1.367321in}}%
\pgfpathlineto{\pgfqpoint{7.650454in}{1.367078in}}%
\pgfpathlineto{\pgfqpoint{7.653626in}{1.367015in}}%
\pgfpathlineto{\pgfqpoint{7.656798in}{1.366859in}}%
\pgfpathlineto{\pgfqpoint{7.659970in}{1.367056in}}%
\pgfpathlineto{\pgfqpoint{7.663142in}{1.367113in}}%
\pgfpathlineto{\pgfqpoint{7.666315in}{1.367021in}}%
\pgfpathlineto{\pgfqpoint{7.669487in}{1.367358in}}%
\pgfpathlineto{\pgfqpoint{7.672659in}{1.367533in}}%
\pgfpathlineto{\pgfqpoint{7.675831in}{1.367845in}}%
\pgfpathlineto{\pgfqpoint{7.679003in}{1.368206in}}%
\pgfpathlineto{\pgfqpoint{7.682175in}{1.368420in}}%
\pgfpathlineto{\pgfqpoint{7.685347in}{1.368522in}}%
\pgfpathlineto{\pgfqpoint{7.688519in}{1.368515in}}%
\pgfpathlineto{\pgfqpoint{7.691691in}{1.368590in}}%
\pgfpathlineto{\pgfqpoint{7.694863in}{1.368712in}}%
\pgfpathlineto{\pgfqpoint{7.698035in}{1.368890in}}%
\pgfpathlineto{\pgfqpoint{7.701207in}{1.369134in}}%
\pgfpathlineto{\pgfqpoint{7.704379in}{1.369617in}}%
\pgfpathlineto{\pgfqpoint{7.707551in}{1.369210in}}%
\pgfpathlineto{\pgfqpoint{7.710723in}{1.368929in}}%
\pgfpathlineto{\pgfqpoint{7.713895in}{1.369021in}}%
\pgfpathlineto{\pgfqpoint{7.717067in}{1.369082in}}%
\pgfpathlineto{\pgfqpoint{7.720239in}{1.368991in}}%
\pgfpathlineto{\pgfqpoint{7.723411in}{1.369615in}}%
\pgfpathlineto{\pgfqpoint{7.726583in}{1.369942in}}%
\pgfpathlineto{\pgfqpoint{7.729755in}{1.370075in}}%
\pgfpathlineto{\pgfqpoint{7.732927in}{1.369881in}}%
\pgfpathlineto{\pgfqpoint{7.736099in}{1.369612in}}%
\pgfpathlineto{\pgfqpoint{7.739271in}{1.369860in}}%
\pgfpathlineto{\pgfqpoint{7.742443in}{1.369458in}}%
\pgfpathlineto{\pgfqpoint{7.745616in}{1.369155in}}%
\pgfpathlineto{\pgfqpoint{7.748788in}{1.369218in}}%
\pgfpathlineto{\pgfqpoint{7.751960in}{1.368296in}}%
\pgfpathlineto{\pgfqpoint{7.755132in}{1.368341in}}%
\pgfpathlineto{\pgfqpoint{7.758304in}{1.368536in}}%
\pgfpathlineto{\pgfqpoint{7.761476in}{1.368973in}}%
\pgfpathlineto{\pgfqpoint{7.764648in}{1.368561in}}%
\pgfpathlineto{\pgfqpoint{7.767820in}{1.368531in}}%
\pgfpathlineto{\pgfqpoint{7.770992in}{1.368933in}}%
\pgfpathlineto{\pgfqpoint{7.774164in}{1.369544in}}%
\pgfpathlineto{\pgfqpoint{7.777336in}{1.369691in}}%
\pgfpathlineto{\pgfqpoint{7.780508in}{1.369765in}}%
\pgfpathlineto{\pgfqpoint{7.783680in}{1.370045in}}%
\pgfpathlineto{\pgfqpoint{7.786852in}{1.369426in}}%
\pgfpathlineto{\pgfqpoint{7.790024in}{1.372236in}}%
\pgfpathlineto{\pgfqpoint{7.793196in}{1.375419in}}%
\pgfpathlineto{\pgfqpoint{7.796368in}{1.378505in}}%
\pgfpathlineto{\pgfqpoint{7.799540in}{1.381681in}}%
\pgfpathlineto{\pgfqpoint{7.802712in}{1.384815in}}%
\pgfpathlineto{\pgfqpoint{7.805884in}{1.387995in}}%
\pgfpathlineto{\pgfqpoint{7.809056in}{1.391085in}}%
\pgfpathlineto{\pgfqpoint{7.812228in}{1.394087in}}%
\pgfpathlineto{\pgfqpoint{7.815400in}{1.397270in}}%
\pgfpathlineto{\pgfqpoint{7.818572in}{1.400469in}}%
\pgfpathlineto{\pgfqpoint{7.821744in}{1.403727in}}%
\pgfpathlineto{\pgfqpoint{7.824917in}{1.406954in}}%
\pgfpathlineto{\pgfqpoint{7.828089in}{1.410063in}}%
\pgfpathlineto{\pgfqpoint{7.831261in}{1.413144in}}%
\pgfpathlineto{\pgfqpoint{7.834433in}{1.416281in}}%
\pgfpathlineto{\pgfqpoint{7.837605in}{1.419429in}}%
\pgfpathlineto{\pgfqpoint{7.840777in}{1.422702in}}%
\pgfpathlineto{\pgfqpoint{7.843949in}{1.426055in}}%
\pgfpathlineto{\pgfqpoint{7.847121in}{1.429230in}}%
\pgfpathlineto{\pgfqpoint{7.850293in}{1.432356in}}%
\pgfpathlineto{\pgfqpoint{7.853465in}{1.435521in}}%
\pgfpathlineto{\pgfqpoint{7.856637in}{1.438723in}}%
\pgfpathlineto{\pgfqpoint{7.859809in}{1.441872in}}%
\pgfpathlineto{\pgfqpoint{7.862981in}{1.445058in}}%
\pgfpathlineto{\pgfqpoint{7.866153in}{1.448243in}}%
\pgfpathlineto{\pgfqpoint{7.869325in}{1.451388in}}%
\pgfpathlineto{\pgfqpoint{7.872497in}{1.454555in}}%
\pgfpathlineto{\pgfqpoint{7.875669in}{1.457713in}}%
\pgfpathlineto{\pgfqpoint{7.878841in}{1.460866in}}%
\pgfpathlineto{\pgfqpoint{7.882013in}{1.464074in}}%
\pgfpathlineto{\pgfqpoint{7.885185in}{1.467155in}}%
\pgfpathlineto{\pgfqpoint{7.888357in}{1.470307in}}%
\pgfpathlineto{\pgfqpoint{7.891529in}{1.473377in}}%
\pgfpathlineto{\pgfqpoint{7.894701in}{1.476547in}}%
\pgfpathlineto{\pgfqpoint{7.897873in}{1.479240in}}%
\pgfpathlineto{\pgfqpoint{7.901046in}{1.482360in}}%
\pgfpathlineto{\pgfqpoint{7.904218in}{1.485496in}}%
\pgfpathlineto{\pgfqpoint{7.907390in}{1.488624in}}%
\pgfpathlineto{\pgfqpoint{7.910562in}{1.491783in}}%
\pgfpathlineto{\pgfqpoint{7.913734in}{1.494917in}}%
\pgfpathlineto{\pgfqpoint{7.916906in}{1.498097in}}%
\pgfpathlineto{\pgfqpoint{7.920078in}{1.501274in}}%
\pgfpathlineto{\pgfqpoint{7.923250in}{1.504446in}}%
\pgfpathlineto{\pgfqpoint{7.926422in}{1.507621in}}%
\pgfpathlineto{\pgfqpoint{7.929594in}{1.510764in}}%
\pgfpathlineto{\pgfqpoint{7.932766in}{1.513916in}}%
\pgfpathlineto{\pgfqpoint{7.935938in}{1.517104in}}%
\pgfpathlineto{\pgfqpoint{7.939110in}{1.520263in}}%
\pgfpathlineto{\pgfqpoint{7.942282in}{1.523367in}}%
\pgfpathlineto{\pgfqpoint{7.945454in}{1.526540in}}%
\pgfpathlineto{\pgfqpoint{7.948626in}{1.529693in}}%
\pgfpathlineto{\pgfqpoint{7.951798in}{1.532932in}}%
\pgfpathlineto{\pgfqpoint{7.954970in}{1.536079in}}%
\pgfpathlineto{\pgfqpoint{7.958142in}{1.539181in}}%
\pgfpathlineto{\pgfqpoint{7.961314in}{1.542362in}}%
\pgfpathlineto{\pgfqpoint{7.964486in}{1.545569in}}%
\pgfpathlineto{\pgfqpoint{7.967658in}{1.548692in}}%
\pgfpathlineto{\pgfqpoint{7.970830in}{1.551858in}}%
\pgfpathlineto{\pgfqpoint{7.974002in}{1.555003in}}%
\pgfpathlineto{\pgfqpoint{7.977174in}{1.558157in}}%
\pgfpathlineto{\pgfqpoint{7.980347in}{1.561326in}}%
\pgfpathlineto{\pgfqpoint{7.983519in}{1.564531in}}%
\pgfpathlineto{\pgfqpoint{7.986691in}{1.567780in}}%
\pgfpathlineto{\pgfqpoint{7.989863in}{1.570846in}}%
\pgfpathlineto{\pgfqpoint{7.993035in}{1.573976in}}%
\pgfpathlineto{\pgfqpoint{7.996207in}{1.577128in}}%
\pgfpathlineto{\pgfqpoint{7.999379in}{1.580100in}}%
\pgfpathlineto{\pgfqpoint{8.002551in}{1.583055in}}%
\pgfpathlineto{\pgfqpoint{8.005723in}{1.586218in}}%
\pgfpathlineto{\pgfqpoint{8.008895in}{1.589356in}}%
\pgfpathlineto{\pgfqpoint{8.012067in}{1.592535in}}%
\pgfpathlineto{\pgfqpoint{8.015239in}{1.595686in}}%
\pgfpathlineto{\pgfqpoint{8.018411in}{1.598750in}}%
\pgfpathlineto{\pgfqpoint{8.021583in}{1.601909in}}%
\pgfpathlineto{\pgfqpoint{8.024755in}{1.605080in}}%
\pgfpathlineto{\pgfqpoint{8.027927in}{1.608198in}}%
\pgfpathlineto{\pgfqpoint{8.031099in}{1.611171in}}%
\pgfpathlineto{\pgfqpoint{8.034271in}{1.614347in}}%
\pgfpathlineto{\pgfqpoint{8.037443in}{1.617467in}}%
\pgfpathlineto{\pgfqpoint{8.040615in}{1.620637in}}%
\pgfpathlineto{\pgfqpoint{8.043787in}{1.623733in}}%
\pgfpathlineto{\pgfqpoint{8.046959in}{1.626814in}}%
\pgfpathlineto{\pgfqpoint{8.050131in}{1.629971in}}%
\pgfpathlineto{\pgfqpoint{8.053303in}{1.633160in}}%
\pgfpathlineto{\pgfqpoint{8.056475in}{1.636298in}}%
\pgfpathlineto{\pgfqpoint{8.059648in}{1.639394in}}%
\pgfpathlineto{\pgfqpoint{8.062820in}{1.642578in}}%
\pgfpathlineto{\pgfqpoint{8.065992in}{1.646010in}}%
\pgfpathlineto{\pgfqpoint{8.069164in}{1.649194in}}%
\pgfpathlineto{\pgfqpoint{8.072336in}{1.652401in}}%
\pgfpathlineto{\pgfqpoint{8.075508in}{1.655574in}}%
\pgfpathlineto{\pgfqpoint{8.078680in}{1.658756in}}%
\pgfpathlineto{\pgfqpoint{8.081852in}{1.661952in}}%
\pgfpathlineto{\pgfqpoint{8.085024in}{1.665171in}}%
\pgfpathlineto{\pgfqpoint{8.088196in}{1.668345in}}%
\pgfpathlineto{\pgfqpoint{8.091368in}{1.671463in}}%
\pgfpathlineto{\pgfqpoint{8.094540in}{1.674611in}}%
\pgfpathlineto{\pgfqpoint{8.097712in}{1.677791in}}%
\pgfpathlineto{\pgfqpoint{8.100884in}{1.680935in}}%
\pgfpathlineto{\pgfqpoint{8.104056in}{1.684059in}}%
\pgfpathlineto{\pgfqpoint{8.107228in}{1.687184in}}%
\pgfpathlineto{\pgfqpoint{8.110400in}{1.690321in}}%
\pgfpathlineto{\pgfqpoint{8.113572in}{1.693484in}}%
\pgfpathlineto{\pgfqpoint{8.116744in}{1.696661in}}%
\pgfpathlineto{\pgfqpoint{8.119916in}{1.699826in}}%
\pgfpathlineto{\pgfqpoint{8.123088in}{1.702974in}}%
\pgfpathlineto{\pgfqpoint{8.126260in}{1.706131in}}%
\pgfpathlineto{\pgfqpoint{8.129432in}{1.709376in}}%
\pgfpathlineto{\pgfqpoint{8.132604in}{1.712483in}}%
\pgfpathlineto{\pgfqpoint{8.135777in}{1.715646in}}%
\pgfpathlineto{\pgfqpoint{8.138949in}{1.718793in}}%
\pgfpathlineto{\pgfqpoint{8.142121in}{1.722009in}}%
\pgfpathlineto{\pgfqpoint{8.145293in}{1.725186in}}%
\pgfpathlineto{\pgfqpoint{8.148465in}{1.728353in}}%
\pgfpathlineto{\pgfqpoint{8.151637in}{1.731511in}}%
\pgfpathlineto{\pgfqpoint{8.154809in}{1.734677in}}%
\pgfpathlineto{\pgfqpoint{8.157981in}{1.737767in}}%
\pgfpathlineto{\pgfqpoint{8.161153in}{1.740917in}}%
\pgfpathlineto{\pgfqpoint{8.164325in}{1.744175in}}%
\pgfpathlineto{\pgfqpoint{8.167497in}{1.747332in}}%
\pgfpathlineto{\pgfqpoint{8.170669in}{1.750463in}}%
\pgfpathlineto{\pgfqpoint{8.173841in}{1.753546in}}%
\pgfpathlineto{\pgfqpoint{8.177013in}{1.756697in}}%
\pgfpathlineto{\pgfqpoint{8.180185in}{1.759771in}}%
\pgfpathlineto{\pgfqpoint{8.183357in}{1.762935in}}%
\pgfpathlineto{\pgfqpoint{8.186529in}{1.766090in}}%
\pgfpathlineto{\pgfqpoint{8.189701in}{1.769299in}}%
\pgfpathlineto{\pgfqpoint{8.192873in}{1.772452in}}%
\pgfpathlineto{\pgfqpoint{8.196045in}{1.775535in}}%
\pgfpathlineto{\pgfqpoint{8.199217in}{1.778712in}}%
\pgfpathlineto{\pgfqpoint{8.202389in}{1.781877in}}%
\pgfpathlineto{\pgfqpoint{8.205561in}{1.785007in}}%
\pgfpathlineto{\pgfqpoint{8.208733in}{1.788244in}}%
\pgfpathlineto{\pgfqpoint{8.211905in}{1.791402in}}%
\pgfpathlineto{\pgfqpoint{8.215078in}{1.794679in}}%
\pgfpathlineto{\pgfqpoint{8.218250in}{1.797792in}}%
\pgfpathlineto{\pgfqpoint{8.221422in}{1.800924in}}%
\pgfpathlineto{\pgfqpoint{8.224594in}{1.804134in}}%
\pgfpathlineto{\pgfqpoint{8.227766in}{1.807278in}}%
\pgfpathlineto{\pgfqpoint{8.230938in}{1.810452in}}%
\pgfpathlineto{\pgfqpoint{8.234110in}{1.813637in}}%
\pgfpathlineto{\pgfqpoint{8.237282in}{1.816728in}}%
\pgfpathlineto{\pgfqpoint{8.240454in}{1.819876in}}%
\pgfpathlineto{\pgfqpoint{8.243626in}{1.823001in}}%
\pgfpathlineto{\pgfqpoint{8.246798in}{1.826128in}}%
\pgfpathlineto{\pgfqpoint{8.249970in}{1.829225in}}%
\pgfpathlineto{\pgfqpoint{8.253142in}{1.832361in}}%
\pgfpathlineto{\pgfqpoint{8.256314in}{1.835434in}}%
\pgfpathlineto{\pgfqpoint{8.259486in}{1.838567in}}%
\pgfpathlineto{\pgfqpoint{8.262658in}{1.841718in}}%
\pgfpathlineto{\pgfqpoint{8.265830in}{1.844887in}}%
\pgfpathlineto{\pgfqpoint{8.269002in}{1.848054in}}%
\pgfpathlineto{\pgfqpoint{8.272174in}{1.851204in}}%
\pgfpathlineto{\pgfqpoint{8.275346in}{1.854272in}}%
\pgfpathlineto{\pgfqpoint{8.278518in}{1.857449in}}%
\pgfpathlineto{\pgfqpoint{8.281690in}{1.860595in}}%
\pgfpathlineto{\pgfqpoint{8.281690in}{1.992654in}}%
\pgfpathlineto{\pgfqpoint{8.281690in}{1.992654in}}%
\pgfpathlineto{\pgfqpoint{8.278518in}{1.989467in}}%
\pgfpathlineto{\pgfqpoint{8.275346in}{1.986273in}}%
\pgfpathlineto{\pgfqpoint{8.272174in}{1.983081in}}%
\pgfpathlineto{\pgfqpoint{8.269002in}{1.979908in}}%
\pgfpathlineto{\pgfqpoint{8.265830in}{1.976804in}}%
\pgfpathlineto{\pgfqpoint{8.262658in}{1.973636in}}%
\pgfpathlineto{\pgfqpoint{8.259486in}{1.970474in}}%
\pgfpathlineto{\pgfqpoint{8.256314in}{1.967299in}}%
\pgfpathlineto{\pgfqpoint{8.253142in}{1.964106in}}%
\pgfpathlineto{\pgfqpoint{8.249970in}{1.960947in}}%
\pgfpathlineto{\pgfqpoint{8.246798in}{1.957792in}}%
\pgfpathlineto{\pgfqpoint{8.243626in}{1.954610in}}%
\pgfpathlineto{\pgfqpoint{8.240454in}{1.951476in}}%
\pgfpathlineto{\pgfqpoint{8.237282in}{1.948282in}}%
\pgfpathlineto{\pgfqpoint{8.234110in}{1.945083in}}%
\pgfpathlineto{\pgfqpoint{8.230938in}{1.941903in}}%
\pgfpathlineto{\pgfqpoint{8.227766in}{1.938701in}}%
\pgfpathlineto{\pgfqpoint{8.224594in}{1.935544in}}%
\pgfpathlineto{\pgfqpoint{8.221422in}{1.932359in}}%
\pgfpathlineto{\pgfqpoint{8.218250in}{1.929187in}}%
\pgfpathlineto{\pgfqpoint{8.215078in}{1.926010in}}%
\pgfpathlineto{\pgfqpoint{8.211905in}{1.922828in}}%
\pgfpathlineto{\pgfqpoint{8.208733in}{1.919685in}}%
\pgfpathlineto{\pgfqpoint{8.205561in}{1.916520in}}%
\pgfpathlineto{\pgfqpoint{8.202389in}{1.913350in}}%
\pgfpathlineto{\pgfqpoint{8.199217in}{1.910207in}}%
\pgfpathlineto{\pgfqpoint{8.196045in}{1.907049in}}%
\pgfpathlineto{\pgfqpoint{8.192873in}{1.903856in}}%
\pgfpathlineto{\pgfqpoint{8.189701in}{1.900677in}}%
\pgfpathlineto{\pgfqpoint{8.186529in}{1.897513in}}%
\pgfpathlineto{\pgfqpoint{8.183357in}{1.894342in}}%
\pgfpathlineto{\pgfqpoint{8.180185in}{1.891172in}}%
\pgfpathlineto{\pgfqpoint{8.177013in}{1.887964in}}%
\pgfpathlineto{\pgfqpoint{8.173841in}{1.884774in}}%
\pgfpathlineto{\pgfqpoint{8.170669in}{1.881506in}}%
\pgfpathlineto{\pgfqpoint{8.167497in}{1.878331in}}%
\pgfpathlineto{\pgfqpoint{8.164325in}{1.875142in}}%
\pgfpathlineto{\pgfqpoint{8.161153in}{1.871965in}}%
\pgfpathlineto{\pgfqpoint{8.157981in}{1.868783in}}%
\pgfpathlineto{\pgfqpoint{8.154809in}{1.865525in}}%
\pgfpathlineto{\pgfqpoint{8.151637in}{1.862336in}}%
\pgfpathlineto{\pgfqpoint{8.148465in}{1.859169in}}%
\pgfpathlineto{\pgfqpoint{8.145293in}{1.855976in}}%
\pgfpathlineto{\pgfqpoint{8.142121in}{1.852734in}}%
\pgfpathlineto{\pgfqpoint{8.138949in}{1.849562in}}%
\pgfpathlineto{\pgfqpoint{8.135777in}{1.846389in}}%
\pgfpathlineto{\pgfqpoint{8.132604in}{1.843231in}}%
\pgfpathlineto{\pgfqpoint{8.129432in}{1.840057in}}%
\pgfpathlineto{\pgfqpoint{8.126260in}{1.837218in}}%
\pgfpathlineto{\pgfqpoint{8.123088in}{1.834011in}}%
\pgfpathlineto{\pgfqpoint{8.119916in}{1.830814in}}%
\pgfpathlineto{\pgfqpoint{8.116744in}{1.827621in}}%
\pgfpathlineto{\pgfqpoint{8.113572in}{1.824429in}}%
\pgfpathlineto{\pgfqpoint{8.110400in}{1.821269in}}%
\pgfpathlineto{\pgfqpoint{8.107228in}{1.818074in}}%
\pgfpathlineto{\pgfqpoint{8.104056in}{1.814888in}}%
\pgfpathlineto{\pgfqpoint{8.100884in}{1.811699in}}%
\pgfpathlineto{\pgfqpoint{8.097712in}{1.808493in}}%
\pgfpathlineto{\pgfqpoint{8.094540in}{1.805352in}}%
\pgfpathlineto{\pgfqpoint{8.091368in}{1.802087in}}%
\pgfpathlineto{\pgfqpoint{8.088196in}{1.798902in}}%
\pgfpathlineto{\pgfqpoint{8.085024in}{1.795712in}}%
\pgfpathlineto{\pgfqpoint{8.081852in}{1.792542in}}%
\pgfpathlineto{\pgfqpoint{8.078680in}{1.789371in}}%
\pgfpathlineto{\pgfqpoint{8.075508in}{1.786157in}}%
\pgfpathlineto{\pgfqpoint{8.072336in}{1.782984in}}%
\pgfpathlineto{\pgfqpoint{8.069164in}{1.779817in}}%
\pgfpathlineto{\pgfqpoint{8.065992in}{1.776635in}}%
\pgfpathlineto{\pgfqpoint{8.062820in}{1.773403in}}%
\pgfpathlineto{\pgfqpoint{8.059648in}{1.770206in}}%
\pgfpathlineto{\pgfqpoint{8.056475in}{1.767025in}}%
\pgfpathlineto{\pgfqpoint{8.053303in}{1.763828in}}%
\pgfpathlineto{\pgfqpoint{8.050131in}{1.760650in}}%
\pgfpathlineto{\pgfqpoint{8.046959in}{1.757477in}}%
\pgfpathlineto{\pgfqpoint{8.043787in}{1.754225in}}%
\pgfpathlineto{\pgfqpoint{8.040615in}{1.750972in}}%
\pgfpathlineto{\pgfqpoint{8.037443in}{1.747797in}}%
\pgfpathlineto{\pgfqpoint{8.034271in}{1.744632in}}%
\pgfpathlineto{\pgfqpoint{8.031099in}{1.741431in}}%
\pgfpathlineto{\pgfqpoint{8.027927in}{1.737685in}}%
\pgfpathlineto{\pgfqpoint{8.024755in}{1.734506in}}%
\pgfpathlineto{\pgfqpoint{8.021583in}{1.731348in}}%
\pgfpathlineto{\pgfqpoint{8.018411in}{1.728191in}}%
\pgfpathlineto{\pgfqpoint{8.015239in}{1.725008in}}%
\pgfpathlineto{\pgfqpoint{8.012067in}{1.721861in}}%
\pgfpathlineto{\pgfqpoint{8.008895in}{1.718653in}}%
\pgfpathlineto{\pgfqpoint{8.005723in}{1.715496in}}%
\pgfpathlineto{\pgfqpoint{8.002551in}{1.712343in}}%
\pgfpathlineto{\pgfqpoint{7.999379in}{1.709213in}}%
\pgfpathlineto{\pgfqpoint{7.996207in}{1.706238in}}%
\pgfpathlineto{\pgfqpoint{7.993035in}{1.703096in}}%
\pgfpathlineto{\pgfqpoint{7.989863in}{1.699917in}}%
\pgfpathlineto{\pgfqpoint{7.986691in}{1.696653in}}%
\pgfpathlineto{\pgfqpoint{7.983519in}{1.693498in}}%
\pgfpathlineto{\pgfqpoint{7.980347in}{1.690363in}}%
\pgfpathlineto{\pgfqpoint{7.977174in}{1.687189in}}%
\pgfpathlineto{\pgfqpoint{7.974002in}{1.684014in}}%
\pgfpathlineto{\pgfqpoint{7.970830in}{1.680852in}}%
\pgfpathlineto{\pgfqpoint{7.967658in}{1.677689in}}%
\pgfpathlineto{\pgfqpoint{7.964486in}{1.674499in}}%
\pgfpathlineto{\pgfqpoint{7.961314in}{1.671333in}}%
\pgfpathlineto{\pgfqpoint{7.958142in}{1.668155in}}%
\pgfpathlineto{\pgfqpoint{7.954970in}{1.664967in}}%
\pgfpathlineto{\pgfqpoint{7.951798in}{1.661827in}}%
\pgfpathlineto{\pgfqpoint{7.948626in}{1.658669in}}%
\pgfpathlineto{\pgfqpoint{7.945454in}{1.655478in}}%
\pgfpathlineto{\pgfqpoint{7.942282in}{1.652293in}}%
\pgfpathlineto{\pgfqpoint{7.939110in}{1.649102in}}%
\pgfpathlineto{\pgfqpoint{7.935938in}{1.645949in}}%
\pgfpathlineto{\pgfqpoint{7.932766in}{1.642772in}}%
\pgfpathlineto{\pgfqpoint{7.929594in}{1.639599in}}%
\pgfpathlineto{\pgfqpoint{7.926422in}{1.636434in}}%
\pgfpathlineto{\pgfqpoint{7.923250in}{1.633308in}}%
\pgfpathlineto{\pgfqpoint{7.920078in}{1.630158in}}%
\pgfpathlineto{\pgfqpoint{7.916906in}{1.626952in}}%
\pgfpathlineto{\pgfqpoint{7.913734in}{1.623765in}}%
\pgfpathlineto{\pgfqpoint{7.910562in}{1.620592in}}%
\pgfpathlineto{\pgfqpoint{7.907390in}{1.617421in}}%
\pgfpathlineto{\pgfqpoint{7.904218in}{1.614253in}}%
\pgfpathlineto{\pgfqpoint{7.901046in}{1.611062in}}%
\pgfpathlineto{\pgfqpoint{7.897873in}{1.607886in}}%
\pgfpathlineto{\pgfqpoint{7.894701in}{1.603840in}}%
\pgfpathlineto{\pgfqpoint{7.891529in}{1.600677in}}%
\pgfpathlineto{\pgfqpoint{7.888357in}{1.597513in}}%
\pgfpathlineto{\pgfqpoint{7.885185in}{1.594429in}}%
\pgfpathlineto{\pgfqpoint{7.882013in}{1.591260in}}%
\pgfpathlineto{\pgfqpoint{7.878841in}{1.588138in}}%
\pgfpathlineto{\pgfqpoint{7.875669in}{1.584978in}}%
\pgfpathlineto{\pgfqpoint{7.872497in}{1.581813in}}%
\pgfpathlineto{\pgfqpoint{7.869325in}{1.578636in}}%
\pgfpathlineto{\pgfqpoint{7.866153in}{1.575490in}}%
\pgfpathlineto{\pgfqpoint{7.862981in}{1.572260in}}%
\pgfpathlineto{\pgfqpoint{7.859809in}{1.569075in}}%
\pgfpathlineto{\pgfqpoint{7.856637in}{1.565843in}}%
\pgfpathlineto{\pgfqpoint{7.853465in}{1.562680in}}%
\pgfpathlineto{\pgfqpoint{7.850293in}{1.559517in}}%
\pgfpathlineto{\pgfqpoint{7.847121in}{1.556360in}}%
\pgfpathlineto{\pgfqpoint{7.843949in}{1.553180in}}%
\pgfpathlineto{\pgfqpoint{7.840777in}{1.549950in}}%
\pgfpathlineto{\pgfqpoint{7.837605in}{1.546813in}}%
\pgfpathlineto{\pgfqpoint{7.834433in}{1.543631in}}%
\pgfpathlineto{\pgfqpoint{7.831261in}{1.540473in}}%
\pgfpathlineto{\pgfqpoint{7.828089in}{1.537347in}}%
\pgfpathlineto{\pgfqpoint{7.824917in}{1.534267in}}%
\pgfpathlineto{\pgfqpoint{7.821744in}{1.531073in}}%
\pgfpathlineto{\pgfqpoint{7.818572in}{1.527921in}}%
\pgfpathlineto{\pgfqpoint{7.815400in}{1.524751in}}%
\pgfpathlineto{\pgfqpoint{7.812228in}{1.521580in}}%
\pgfpathlineto{\pgfqpoint{7.809056in}{1.517780in}}%
\pgfpathlineto{\pgfqpoint{7.805884in}{1.514525in}}%
\pgfpathlineto{\pgfqpoint{7.802712in}{1.511315in}}%
\pgfpathlineto{\pgfqpoint{7.799540in}{1.508205in}}%
\pgfpathlineto{\pgfqpoint{7.796368in}{1.505034in}}%
\pgfpathlineto{\pgfqpoint{7.793196in}{1.501802in}}%
\pgfpathlineto{\pgfqpoint{7.790024in}{1.498652in}}%
\pgfpathlineto{\pgfqpoint{7.786852in}{1.494806in}}%
\pgfpathlineto{\pgfqpoint{7.783680in}{1.494373in}}%
\pgfpathlineto{\pgfqpoint{7.780508in}{1.494734in}}%
\pgfpathlineto{\pgfqpoint{7.777336in}{1.494830in}}%
\pgfpathlineto{\pgfqpoint{7.774164in}{1.494921in}}%
\pgfpathlineto{\pgfqpoint{7.770992in}{1.494961in}}%
\pgfpathlineto{\pgfqpoint{7.767820in}{1.494976in}}%
\pgfpathlineto{\pgfqpoint{7.764648in}{1.494551in}}%
\pgfpathlineto{\pgfqpoint{7.761476in}{1.494425in}}%
\pgfpathlineto{\pgfqpoint{7.758304in}{1.494641in}}%
\pgfpathlineto{\pgfqpoint{7.755132in}{1.494473in}}%
\pgfpathlineto{\pgfqpoint{7.751960in}{1.494839in}}%
\pgfpathlineto{\pgfqpoint{7.748788in}{1.495039in}}%
\pgfpathlineto{\pgfqpoint{7.745616in}{1.495410in}}%
\pgfpathlineto{\pgfqpoint{7.742443in}{1.495239in}}%
\pgfpathlineto{\pgfqpoint{7.739271in}{1.495555in}}%
\pgfpathlineto{\pgfqpoint{7.736099in}{1.495737in}}%
\pgfpathlineto{\pgfqpoint{7.732927in}{1.495415in}}%
\pgfpathlineto{\pgfqpoint{7.729755in}{1.495487in}}%
\pgfpathlineto{\pgfqpoint{7.726583in}{1.495673in}}%
\pgfpathlineto{\pgfqpoint{7.723411in}{1.495349in}}%
\pgfpathlineto{\pgfqpoint{7.720239in}{1.495286in}}%
\pgfpathlineto{\pgfqpoint{7.717067in}{1.495274in}}%
\pgfpathlineto{\pgfqpoint{7.713895in}{1.495033in}}%
\pgfpathlineto{\pgfqpoint{7.710723in}{1.494914in}}%
\pgfpathlineto{\pgfqpoint{7.707551in}{1.494696in}}%
\pgfpathlineto{\pgfqpoint{7.704379in}{1.494565in}}%
\pgfpathlineto{\pgfqpoint{7.701207in}{1.494873in}}%
\pgfpathlineto{\pgfqpoint{7.698035in}{1.495080in}}%
\pgfpathlineto{\pgfqpoint{7.694863in}{1.495012in}}%
\pgfpathlineto{\pgfqpoint{7.691691in}{1.494440in}}%
\pgfpathlineto{\pgfqpoint{7.688519in}{1.494077in}}%
\pgfpathlineto{\pgfqpoint{7.685347in}{1.493798in}}%
\pgfpathlineto{\pgfqpoint{7.682175in}{1.493961in}}%
\pgfpathlineto{\pgfqpoint{7.679003in}{1.493804in}}%
\pgfpathlineto{\pgfqpoint{7.675831in}{1.494093in}}%
\pgfpathlineto{\pgfqpoint{7.672659in}{1.493951in}}%
\pgfpathlineto{\pgfqpoint{7.669487in}{1.493771in}}%
\pgfpathlineto{\pgfqpoint{7.666315in}{1.493577in}}%
\pgfpathlineto{\pgfqpoint{7.663142in}{1.493758in}}%
\pgfpathlineto{\pgfqpoint{7.659970in}{1.493623in}}%
\pgfpathlineto{\pgfqpoint{7.656798in}{1.493650in}}%
\pgfpathlineto{\pgfqpoint{7.653626in}{1.493147in}}%
\pgfpathlineto{\pgfqpoint{7.650454in}{1.493260in}}%
\pgfpathlineto{\pgfqpoint{7.647282in}{1.493024in}}%
\pgfpathlineto{\pgfqpoint{7.644110in}{1.492397in}}%
\pgfpathlineto{\pgfqpoint{7.640938in}{1.492583in}}%
\pgfpathlineto{\pgfqpoint{7.637766in}{1.492390in}}%
\pgfpathlineto{\pgfqpoint{7.634594in}{1.491937in}}%
\pgfpathlineto{\pgfqpoint{7.631422in}{1.492294in}}%
\pgfpathlineto{\pgfqpoint{7.628250in}{1.492104in}}%
\pgfpathlineto{\pgfqpoint{7.625078in}{1.491677in}}%
\pgfpathlineto{\pgfqpoint{7.621906in}{1.491353in}}%
\pgfpathlineto{\pgfqpoint{7.618734in}{1.491399in}}%
\pgfpathlineto{\pgfqpoint{7.615562in}{1.491007in}}%
\pgfpathlineto{\pgfqpoint{7.612390in}{1.490264in}}%
\pgfpathlineto{\pgfqpoint{7.609218in}{1.489573in}}%
\pgfpathlineto{\pgfqpoint{7.606046in}{1.489339in}}%
\pgfpathlineto{\pgfqpoint{7.602874in}{1.489549in}}%
\pgfpathlineto{\pgfqpoint{7.599702in}{1.489512in}}%
\pgfpathlineto{\pgfqpoint{7.596530in}{1.489387in}}%
\pgfpathlineto{\pgfqpoint{7.593358in}{1.489187in}}%
\pgfpathlineto{\pgfqpoint{7.590186in}{1.489471in}}%
\pgfpathlineto{\pgfqpoint{7.587013in}{1.489022in}}%
\pgfpathlineto{\pgfqpoint{7.583841in}{1.488964in}}%
\pgfpathlineto{\pgfqpoint{7.580669in}{1.489252in}}%
\pgfpathlineto{\pgfqpoint{7.577497in}{1.489009in}}%
\pgfpathlineto{\pgfqpoint{7.574325in}{1.489116in}}%
\pgfpathlineto{\pgfqpoint{7.571153in}{1.489039in}}%
\pgfpathlineto{\pgfqpoint{7.567981in}{1.489045in}}%
\pgfpathlineto{\pgfqpoint{7.564809in}{1.489037in}}%
\pgfpathlineto{\pgfqpoint{7.561637in}{1.488769in}}%
\pgfpathlineto{\pgfqpoint{7.558465in}{1.488677in}}%
\pgfpathlineto{\pgfqpoint{7.555293in}{1.488535in}}%
\pgfpathlineto{\pgfqpoint{7.552121in}{1.488510in}}%
\pgfpathlineto{\pgfqpoint{7.548949in}{1.488509in}}%
\pgfpathlineto{\pgfqpoint{7.545777in}{1.488728in}}%
\pgfpathlineto{\pgfqpoint{7.542605in}{1.488708in}}%
\pgfpathlineto{\pgfqpoint{7.539433in}{1.488293in}}%
\pgfpathlineto{\pgfqpoint{7.536261in}{1.488279in}}%
\pgfpathlineto{\pgfqpoint{7.533089in}{1.488252in}}%
\pgfpathlineto{\pgfqpoint{7.529917in}{1.488165in}}%
\pgfpathlineto{\pgfqpoint{7.526745in}{1.487963in}}%
\pgfpathlineto{\pgfqpoint{7.523573in}{1.488080in}}%
\pgfpathlineto{\pgfqpoint{7.520401in}{1.488141in}}%
\pgfpathlineto{\pgfqpoint{7.517229in}{1.488055in}}%
\pgfpathlineto{\pgfqpoint{7.514057in}{1.488036in}}%
\pgfpathlineto{\pgfqpoint{7.510885in}{1.488283in}}%
\pgfpathlineto{\pgfqpoint{7.507712in}{1.488183in}}%
\pgfpathlineto{\pgfqpoint{7.504540in}{1.488177in}}%
\pgfpathlineto{\pgfqpoint{7.501368in}{1.488034in}}%
\pgfpathlineto{\pgfqpoint{7.498196in}{1.488057in}}%
\pgfpathlineto{\pgfqpoint{7.495024in}{1.487886in}}%
\pgfpathlineto{\pgfqpoint{7.491852in}{1.488081in}}%
\pgfpathlineto{\pgfqpoint{7.488680in}{1.487891in}}%
\pgfpathlineto{\pgfqpoint{7.485508in}{1.487875in}}%
\pgfpathlineto{\pgfqpoint{7.482336in}{1.488276in}}%
\pgfpathlineto{\pgfqpoint{7.479164in}{1.488351in}}%
\pgfpathlineto{\pgfqpoint{7.475992in}{1.488214in}}%
\pgfpathlineto{\pgfqpoint{7.472820in}{1.488233in}}%
\pgfpathlineto{\pgfqpoint{7.469648in}{1.488314in}}%
\pgfpathlineto{\pgfqpoint{7.466476in}{1.488582in}}%
\pgfpathlineto{\pgfqpoint{7.463304in}{1.488431in}}%
\pgfpathlineto{\pgfqpoint{7.460132in}{1.488416in}}%
\pgfpathlineto{\pgfqpoint{7.456960in}{1.488658in}}%
\pgfpathlineto{\pgfqpoint{7.453788in}{1.488581in}}%
\pgfpathlineto{\pgfqpoint{7.450616in}{1.488677in}}%
\pgfpathlineto{\pgfqpoint{7.447444in}{1.488644in}}%
\pgfpathlineto{\pgfqpoint{7.444272in}{1.488211in}}%
\pgfpathlineto{\pgfqpoint{7.441100in}{1.487552in}}%
\pgfpathlineto{\pgfqpoint{7.437928in}{1.487424in}}%
\pgfpathlineto{\pgfqpoint{7.434756in}{1.487109in}}%
\pgfpathlineto{\pgfqpoint{7.431584in}{1.486858in}}%
\pgfpathlineto{\pgfqpoint{7.428411in}{1.486478in}}%
\pgfpathlineto{\pgfqpoint{7.425239in}{1.486790in}}%
\pgfpathlineto{\pgfqpoint{7.422067in}{1.486619in}}%
\pgfpathlineto{\pgfqpoint{7.418895in}{1.486645in}}%
\pgfpathlineto{\pgfqpoint{7.415723in}{1.486513in}}%
\pgfpathlineto{\pgfqpoint{7.412551in}{1.486140in}}%
\pgfpathlineto{\pgfqpoint{7.409379in}{1.485468in}}%
\pgfpathlineto{\pgfqpoint{7.406207in}{1.485074in}}%
\pgfpathlineto{\pgfqpoint{7.403035in}{1.485214in}}%
\pgfpathlineto{\pgfqpoint{7.399863in}{1.485419in}}%
\pgfpathlineto{\pgfqpoint{7.396691in}{1.485514in}}%
\pgfpathlineto{\pgfqpoint{7.393519in}{1.485301in}}%
\pgfpathlineto{\pgfqpoint{7.390347in}{1.485322in}}%
\pgfpathlineto{\pgfqpoint{7.387175in}{1.485694in}}%
\pgfpathlineto{\pgfqpoint{7.384003in}{1.485477in}}%
\pgfpathlineto{\pgfqpoint{7.380831in}{1.485434in}}%
\pgfpathlineto{\pgfqpoint{7.377659in}{1.484866in}}%
\pgfpathlineto{\pgfqpoint{7.374487in}{1.484714in}}%
\pgfpathlineto{\pgfqpoint{7.371315in}{1.484708in}}%
\pgfpathlineto{\pgfqpoint{7.368143in}{1.484209in}}%
\pgfpathlineto{\pgfqpoint{7.364971in}{1.483979in}}%
\pgfpathlineto{\pgfqpoint{7.361799in}{1.483725in}}%
\pgfpathlineto{\pgfqpoint{7.358627in}{1.483359in}}%
\pgfpathlineto{\pgfqpoint{7.355455in}{1.483561in}}%
\pgfpathlineto{\pgfqpoint{7.352282in}{1.483426in}}%
\pgfpathlineto{\pgfqpoint{7.349110in}{1.483630in}}%
\pgfpathlineto{\pgfqpoint{7.345938in}{1.483618in}}%
\pgfpathlineto{\pgfqpoint{7.342766in}{1.484114in}}%
\pgfpathlineto{\pgfqpoint{7.339594in}{1.484078in}}%
\pgfpathlineto{\pgfqpoint{7.336422in}{1.483954in}}%
\pgfpathlineto{\pgfqpoint{7.333250in}{1.484042in}}%
\pgfpathlineto{\pgfqpoint{7.330078in}{1.483462in}}%
\pgfpathlineto{\pgfqpoint{7.326906in}{1.483236in}}%
\pgfpathlineto{\pgfqpoint{7.323734in}{1.483460in}}%
\pgfpathlineto{\pgfqpoint{7.320562in}{1.483165in}}%
\pgfpathlineto{\pgfqpoint{7.317390in}{1.482977in}}%
\pgfpathlineto{\pgfqpoint{7.314218in}{1.482911in}}%
\pgfpathlineto{\pgfqpoint{7.311046in}{1.482756in}}%
\pgfpathlineto{\pgfqpoint{7.307874in}{1.482712in}}%
\pgfpathlineto{\pgfqpoint{7.304702in}{1.482604in}}%
\pgfpathlineto{\pgfqpoint{7.301530in}{1.482780in}}%
\pgfpathlineto{\pgfqpoint{7.298358in}{1.482618in}}%
\pgfpathlineto{\pgfqpoint{7.295186in}{1.482593in}}%
\pgfpathlineto{\pgfqpoint{7.292014in}{1.482635in}}%
\pgfpathlineto{\pgfqpoint{7.288842in}{1.482503in}}%
\pgfpathlineto{\pgfqpoint{7.285670in}{1.482506in}}%
\pgfpathlineto{\pgfqpoint{7.282498in}{1.482620in}}%
\pgfpathlineto{\pgfqpoint{7.279326in}{1.482859in}}%
\pgfpathlineto{\pgfqpoint{7.276154in}{1.483059in}}%
\pgfpathlineto{\pgfqpoint{7.272981in}{1.482991in}}%
\pgfpathlineto{\pgfqpoint{7.269809in}{1.482801in}}%
\pgfpathlineto{\pgfqpoint{7.266637in}{1.483076in}}%
\pgfpathlineto{\pgfqpoint{7.263465in}{1.482866in}}%
\pgfpathlineto{\pgfqpoint{7.260293in}{1.482762in}}%
\pgfpathlineto{\pgfqpoint{7.257121in}{1.482499in}}%
\pgfpathlineto{\pgfqpoint{7.253949in}{1.482227in}}%
\pgfpathlineto{\pgfqpoint{7.250777in}{1.482274in}}%
\pgfpathlineto{\pgfqpoint{7.247605in}{1.482162in}}%
\pgfpathlineto{\pgfqpoint{7.244433in}{1.481718in}}%
\pgfpathlineto{\pgfqpoint{7.241261in}{1.481001in}}%
\pgfpathlineto{\pgfqpoint{7.238089in}{1.480458in}}%
\pgfpathlineto{\pgfqpoint{7.234917in}{1.480019in}}%
\pgfpathlineto{\pgfqpoint{7.231745in}{1.480314in}}%
\pgfpathlineto{\pgfqpoint{7.228573in}{1.480111in}}%
\pgfpathlineto{\pgfqpoint{7.225401in}{1.480383in}}%
\pgfpathlineto{\pgfqpoint{7.222229in}{1.480112in}}%
\pgfpathlineto{\pgfqpoint{7.219057in}{1.480260in}}%
\pgfpathlineto{\pgfqpoint{7.215885in}{1.480465in}}%
\pgfpathlineto{\pgfqpoint{7.212713in}{1.479865in}}%
\pgfpathlineto{\pgfqpoint{7.209541in}{1.479923in}}%
\pgfpathlineto{\pgfqpoint{7.206369in}{1.480050in}}%
\pgfpathlineto{\pgfqpoint{7.203197in}{1.480115in}}%
\pgfpathlineto{\pgfqpoint{7.200025in}{1.479955in}}%
\pgfpathlineto{\pgfqpoint{7.196853in}{1.479919in}}%
\pgfpathlineto{\pgfqpoint{7.193680in}{1.479540in}}%
\pgfpathlineto{\pgfqpoint{7.190508in}{1.479026in}}%
\pgfpathlineto{\pgfqpoint{7.187336in}{1.478840in}}%
\pgfpathlineto{\pgfqpoint{7.184164in}{1.478968in}}%
\pgfpathlineto{\pgfqpoint{7.180992in}{1.478312in}}%
\pgfpathlineto{\pgfqpoint{7.177820in}{1.477719in}}%
\pgfpathlineto{\pgfqpoint{7.174648in}{1.477498in}}%
\pgfpathlineto{\pgfqpoint{7.171476in}{1.477227in}}%
\pgfpathlineto{\pgfqpoint{7.168304in}{1.477118in}}%
\pgfpathlineto{\pgfqpoint{7.165132in}{1.477227in}}%
\pgfpathlineto{\pgfqpoint{7.161960in}{1.477031in}}%
\pgfpathlineto{\pgfqpoint{7.158788in}{1.476933in}}%
\pgfpathlineto{\pgfqpoint{7.155616in}{1.477033in}}%
\pgfpathlineto{\pgfqpoint{7.152444in}{1.476571in}}%
\pgfpathlineto{\pgfqpoint{7.149272in}{1.476694in}}%
\pgfpathlineto{\pgfqpoint{7.146100in}{1.477244in}}%
\pgfpathlineto{\pgfqpoint{7.142928in}{1.476974in}}%
\pgfpathlineto{\pgfqpoint{7.139756in}{1.476730in}}%
\pgfpathlineto{\pgfqpoint{7.136584in}{1.476560in}}%
\pgfpathlineto{\pgfqpoint{7.133412in}{1.476240in}}%
\pgfpathlineto{\pgfqpoint{7.130240in}{1.476053in}}%
\pgfpathlineto{\pgfqpoint{7.127068in}{1.475973in}}%
\pgfpathlineto{\pgfqpoint{7.123896in}{1.476057in}}%
\pgfpathlineto{\pgfqpoint{7.120724in}{1.475904in}}%
\pgfpathlineto{\pgfqpoint{7.117551in}{1.476180in}}%
\pgfpathlineto{\pgfqpoint{7.114379in}{1.476112in}}%
\pgfpathlineto{\pgfqpoint{7.111207in}{1.475839in}}%
\pgfpathlineto{\pgfqpoint{7.108035in}{1.475940in}}%
\pgfpathlineto{\pgfqpoint{7.104863in}{1.476189in}}%
\pgfpathlineto{\pgfqpoint{7.101691in}{1.475937in}}%
\pgfpathlineto{\pgfqpoint{7.098519in}{1.475952in}}%
\pgfpathlineto{\pgfqpoint{7.095347in}{1.476136in}}%
\pgfpathlineto{\pgfqpoint{7.092175in}{1.475961in}}%
\pgfpathlineto{\pgfqpoint{7.089003in}{1.476009in}}%
\pgfpathlineto{\pgfqpoint{7.085831in}{1.475961in}}%
\pgfpathlineto{\pgfqpoint{7.082659in}{1.475883in}}%
\pgfpathlineto{\pgfqpoint{7.079487in}{1.475944in}}%
\pgfpathlineto{\pgfqpoint{7.076315in}{1.475840in}}%
\pgfpathlineto{\pgfqpoint{7.073143in}{1.475774in}}%
\pgfpathlineto{\pgfqpoint{7.069971in}{1.475856in}}%
\pgfpathlineto{\pgfqpoint{7.066799in}{1.475841in}}%
\pgfpathlineto{\pgfqpoint{7.063627in}{1.475471in}}%
\pgfpathlineto{\pgfqpoint{7.060455in}{1.476099in}}%
\pgfpathlineto{\pgfqpoint{7.057283in}{1.476288in}}%
\pgfpathlineto{\pgfqpoint{7.054111in}{1.476226in}}%
\pgfpathlineto{\pgfqpoint{7.050939in}{1.476146in}}%
\pgfpathlineto{\pgfqpoint{7.047767in}{1.476664in}}%
\pgfpathlineto{\pgfqpoint{7.044595in}{1.476700in}}%
\pgfpathlineto{\pgfqpoint{7.041423in}{1.476562in}}%
\pgfpathlineto{\pgfqpoint{7.038250in}{1.477160in}}%
\pgfpathlineto{\pgfqpoint{7.035078in}{1.476888in}}%
\pgfpathlineto{\pgfqpoint{7.031906in}{1.476770in}}%
\pgfpathlineto{\pgfqpoint{7.028734in}{1.476240in}}%
\pgfpathlineto{\pgfqpoint{7.025562in}{1.476231in}}%
\pgfpathlineto{\pgfqpoint{7.022390in}{1.476100in}}%
\pgfpathlineto{\pgfqpoint{7.019218in}{1.476159in}}%
\pgfpathlineto{\pgfqpoint{7.016046in}{1.476342in}}%
\pgfpathlineto{\pgfqpoint{7.012874in}{1.476644in}}%
\pgfpathlineto{\pgfqpoint{7.009702in}{1.476550in}}%
\pgfpathlineto{\pgfqpoint{7.006530in}{1.476693in}}%
\pgfpathlineto{\pgfqpoint{7.003358in}{1.476659in}}%
\pgfpathlineto{\pgfqpoint{7.000186in}{1.476600in}}%
\pgfpathlineto{\pgfqpoint{6.997014in}{1.476220in}}%
\pgfpathlineto{\pgfqpoint{6.993842in}{1.476245in}}%
\pgfpathlineto{\pgfqpoint{6.990670in}{1.476296in}}%
\pgfpathlineto{\pgfqpoint{6.987498in}{1.476598in}}%
\pgfpathlineto{\pgfqpoint{6.984326in}{1.476861in}}%
\pgfpathlineto{\pgfqpoint{6.981154in}{1.476807in}}%
\pgfpathlineto{\pgfqpoint{6.977982in}{1.476459in}}%
\pgfpathlineto{\pgfqpoint{6.974810in}{1.476577in}}%
\pgfpathlineto{\pgfqpoint{6.971638in}{1.476470in}}%
\pgfpathlineto{\pgfqpoint{6.968466in}{1.476944in}}%
\pgfpathlineto{\pgfqpoint{6.965294in}{1.477077in}}%
\pgfpathlineto{\pgfqpoint{6.962122in}{1.476767in}}%
\pgfpathlineto{\pgfqpoint{6.958949in}{1.476439in}}%
\pgfpathlineto{\pgfqpoint{6.955777in}{1.476457in}}%
\pgfpathlineto{\pgfqpoint{6.952605in}{1.476188in}}%
\pgfpathlineto{\pgfqpoint{6.949433in}{1.475949in}}%
\pgfpathlineto{\pgfqpoint{6.946261in}{1.475843in}}%
\pgfpathlineto{\pgfqpoint{6.943089in}{1.475862in}}%
\pgfpathlineto{\pgfqpoint{6.939917in}{1.475528in}}%
\pgfpathlineto{\pgfqpoint{6.936745in}{1.475792in}}%
\pgfpathlineto{\pgfqpoint{6.933573in}{1.475814in}}%
\pgfpathlineto{\pgfqpoint{6.930401in}{1.475968in}}%
\pgfpathlineto{\pgfqpoint{6.927229in}{1.476128in}}%
\pgfpathlineto{\pgfqpoint{6.924057in}{1.476082in}}%
\pgfpathlineto{\pgfqpoint{6.920885in}{1.476061in}}%
\pgfpathlineto{\pgfqpoint{6.917713in}{1.475999in}}%
\pgfpathlineto{\pgfqpoint{6.914541in}{1.475960in}}%
\pgfpathlineto{\pgfqpoint{6.911369in}{1.475898in}}%
\pgfpathlineto{\pgfqpoint{6.908197in}{1.475751in}}%
\pgfpathlineto{\pgfqpoint{6.905025in}{1.475647in}}%
\pgfpathlineto{\pgfqpoint{6.901853in}{1.475563in}}%
\pgfpathlineto{\pgfqpoint{6.898681in}{1.475833in}}%
\pgfpathlineto{\pgfqpoint{6.895509in}{1.475742in}}%
\pgfpathlineto{\pgfqpoint{6.892337in}{1.475081in}}%
\pgfpathlineto{\pgfqpoint{6.889165in}{1.474544in}}%
\pgfpathlineto{\pgfqpoint{6.885993in}{1.474570in}}%
\pgfpathlineto{\pgfqpoint{6.882820in}{1.474656in}}%
\pgfpathlineto{\pgfqpoint{6.879648in}{1.474305in}}%
\pgfpathlineto{\pgfqpoint{6.876476in}{1.474622in}}%
\pgfpathlineto{\pgfqpoint{6.873304in}{1.474675in}}%
\pgfpathlineto{\pgfqpoint{6.870132in}{1.474190in}}%
\pgfpathlineto{\pgfqpoint{6.866960in}{1.474196in}}%
\pgfpathlineto{\pgfqpoint{6.863788in}{1.474705in}}%
\pgfpathlineto{\pgfqpoint{6.860616in}{1.474692in}}%
\pgfpathlineto{\pgfqpoint{6.857444in}{1.474749in}}%
\pgfpathlineto{\pgfqpoint{6.854272in}{1.474778in}}%
\pgfpathlineto{\pgfqpoint{6.851100in}{1.474808in}}%
\pgfpathlineto{\pgfqpoint{6.847928in}{1.474888in}}%
\pgfpathlineto{\pgfqpoint{6.844756in}{1.474808in}}%
\pgfpathlineto{\pgfqpoint{6.841584in}{1.474757in}}%
\pgfpathlineto{\pgfqpoint{6.838412in}{1.474787in}}%
\pgfpathlineto{\pgfqpoint{6.835240in}{1.474499in}}%
\pgfpathlineto{\pgfqpoint{6.832068in}{1.474736in}}%
\pgfpathlineto{\pgfqpoint{6.828896in}{1.474742in}}%
\pgfpathlineto{\pgfqpoint{6.825724in}{1.474877in}}%
\pgfpathlineto{\pgfqpoint{6.822552in}{1.474400in}}%
\pgfpathlineto{\pgfqpoint{6.819380in}{1.474339in}}%
\pgfpathlineto{\pgfqpoint{6.816208in}{1.474996in}}%
\pgfpathlineto{\pgfqpoint{6.813036in}{1.475177in}}%
\pgfpathlineto{\pgfqpoint{6.809864in}{1.475516in}}%
\pgfpathlineto{\pgfqpoint{6.806692in}{1.475569in}}%
\pgfpathlineto{\pgfqpoint{6.803519in}{1.475223in}}%
\pgfpathlineto{\pgfqpoint{6.800347in}{1.474998in}}%
\pgfpathlineto{\pgfqpoint{6.797175in}{1.474930in}}%
\pgfpathlineto{\pgfqpoint{6.794003in}{1.474714in}}%
\pgfpathlineto{\pgfqpoint{6.790831in}{1.474621in}}%
\pgfpathlineto{\pgfqpoint{6.787659in}{1.473966in}}%
\pgfpathlineto{\pgfqpoint{6.784487in}{1.473455in}}%
\pgfpathlineto{\pgfqpoint{6.781315in}{1.473125in}}%
\pgfpathlineto{\pgfqpoint{6.778143in}{1.472757in}}%
\pgfpathlineto{\pgfqpoint{6.774971in}{1.472950in}}%
\pgfpathlineto{\pgfqpoint{6.771799in}{1.472335in}}%
\pgfpathlineto{\pgfqpoint{6.768627in}{1.472146in}}%
\pgfpathlineto{\pgfqpoint{6.765455in}{1.472150in}}%
\pgfpathlineto{\pgfqpoint{6.762283in}{1.472055in}}%
\pgfpathlineto{\pgfqpoint{6.759111in}{1.472136in}}%
\pgfpathlineto{\pgfqpoint{6.755939in}{1.472052in}}%
\pgfpathlineto{\pgfqpoint{6.752767in}{1.471746in}}%
\pgfpathlineto{\pgfqpoint{6.749595in}{1.471507in}}%
\pgfpathlineto{\pgfqpoint{6.746423in}{1.471399in}}%
\pgfpathlineto{\pgfqpoint{6.743251in}{1.471303in}}%
\pgfpathlineto{\pgfqpoint{6.740079in}{1.471463in}}%
\pgfpathlineto{\pgfqpoint{6.736907in}{1.471495in}}%
\pgfpathlineto{\pgfqpoint{6.733735in}{1.471408in}}%
\pgfpathlineto{\pgfqpoint{6.730563in}{1.471477in}}%
\pgfpathlineto{\pgfqpoint{6.727391in}{1.471343in}}%
\pgfpathlineto{\pgfqpoint{6.724218in}{1.471066in}}%
\pgfpathlineto{\pgfqpoint{6.721046in}{1.470841in}}%
\pgfpathlineto{\pgfqpoint{6.717874in}{1.470580in}}%
\pgfpathlineto{\pgfqpoint{6.714702in}{1.470581in}}%
\pgfpathlineto{\pgfqpoint{6.711530in}{1.470451in}}%
\pgfpathlineto{\pgfqpoint{6.708358in}{1.470464in}}%
\pgfpathlineto{\pgfqpoint{6.705186in}{1.470390in}}%
\pgfpathlineto{\pgfqpoint{6.702014in}{1.470633in}}%
\pgfpathlineto{\pgfqpoint{6.698842in}{1.470675in}}%
\pgfpathlineto{\pgfqpoint{6.695670in}{1.470873in}}%
\pgfpathlineto{\pgfqpoint{6.692498in}{1.470912in}}%
\pgfpathlineto{\pgfqpoint{6.689326in}{1.470763in}}%
\pgfpathlineto{\pgfqpoint{6.686154in}{1.470810in}}%
\pgfpathlineto{\pgfqpoint{6.682982in}{1.470912in}}%
\pgfpathlineto{\pgfqpoint{6.679810in}{1.470700in}}%
\pgfpathlineto{\pgfqpoint{6.676638in}{1.470883in}}%
\pgfpathlineto{\pgfqpoint{6.673466in}{1.471301in}}%
\pgfpathlineto{\pgfqpoint{6.670294in}{1.471583in}}%
\pgfpathlineto{\pgfqpoint{6.667122in}{1.471727in}}%
\pgfpathlineto{\pgfqpoint{6.663950in}{1.471852in}}%
\pgfpathlineto{\pgfqpoint{6.660778in}{1.472185in}}%
\pgfpathlineto{\pgfqpoint{6.657606in}{1.472412in}}%
\pgfpathlineto{\pgfqpoint{6.654434in}{1.472374in}}%
\pgfpathlineto{\pgfqpoint{6.651262in}{1.472350in}}%
\pgfpathlineto{\pgfqpoint{6.648089in}{1.471963in}}%
\pgfpathlineto{\pgfqpoint{6.644917in}{1.472084in}}%
\pgfpathlineto{\pgfqpoint{6.641745in}{1.472064in}}%
\pgfpathlineto{\pgfqpoint{6.638573in}{1.472182in}}%
\pgfpathlineto{\pgfqpoint{6.635401in}{1.472411in}}%
\pgfpathlineto{\pgfqpoint{6.632229in}{1.472303in}}%
\pgfpathlineto{\pgfqpoint{6.629057in}{1.472108in}}%
\pgfpathlineto{\pgfqpoint{6.625885in}{1.472099in}}%
\pgfpathlineto{\pgfqpoint{6.622713in}{1.472048in}}%
\pgfpathlineto{\pgfqpoint{6.619541in}{1.471910in}}%
\pgfpathlineto{\pgfqpoint{6.616369in}{1.471959in}}%
\pgfpathlineto{\pgfqpoint{6.613197in}{1.471162in}}%
\pgfpathlineto{\pgfqpoint{6.610025in}{1.470580in}}%
\pgfpathlineto{\pgfqpoint{6.606853in}{1.470540in}}%
\pgfpathlineto{\pgfqpoint{6.603681in}{1.470216in}}%
\pgfpathlineto{\pgfqpoint{6.600509in}{1.469994in}}%
\pgfpathlineto{\pgfqpoint{6.597337in}{1.469869in}}%
\pgfpathlineto{\pgfqpoint{6.594165in}{1.469638in}}%
\pgfpathlineto{\pgfqpoint{6.590993in}{1.469233in}}%
\pgfpathlineto{\pgfqpoint{6.587821in}{1.469297in}}%
\pgfpathlineto{\pgfqpoint{6.584649in}{1.469521in}}%
\pgfpathlineto{\pgfqpoint{6.581477in}{1.469578in}}%
\pgfpathlineto{\pgfqpoint{6.578305in}{1.469654in}}%
\pgfpathlineto{\pgfqpoint{6.575133in}{1.469419in}}%
\pgfpathlineto{\pgfqpoint{6.571961in}{1.469533in}}%
\pgfpathlineto{\pgfqpoint{6.568788in}{1.469340in}}%
\pgfpathlineto{\pgfqpoint{6.565616in}{1.469181in}}%
\pgfpathlineto{\pgfqpoint{6.562444in}{1.469480in}}%
\pgfpathlineto{\pgfqpoint{6.559272in}{1.469020in}}%
\pgfpathlineto{\pgfqpoint{6.556100in}{1.469094in}}%
\pgfpathlineto{\pgfqpoint{6.552928in}{1.468952in}}%
\pgfpathlineto{\pgfqpoint{6.549756in}{1.468599in}}%
\pgfpathlineto{\pgfqpoint{6.546584in}{1.468396in}}%
\pgfpathlineto{\pgfqpoint{6.543412in}{1.468475in}}%
\pgfpathlineto{\pgfqpoint{6.540240in}{1.468382in}}%
\pgfpathlineto{\pgfqpoint{6.537068in}{1.468725in}}%
\pgfpathlineto{\pgfqpoint{6.533896in}{1.469084in}}%
\pgfpathlineto{\pgfqpoint{6.530724in}{1.469159in}}%
\pgfpathlineto{\pgfqpoint{6.527552in}{1.469203in}}%
\pgfpathlineto{\pgfqpoint{6.524380in}{1.469257in}}%
\pgfpathlineto{\pgfqpoint{6.521208in}{1.469326in}}%
\pgfpathlineto{\pgfqpoint{6.518036in}{1.469240in}}%
\pgfpathlineto{\pgfqpoint{6.514864in}{1.469362in}}%
\pgfpathlineto{\pgfqpoint{6.511692in}{1.469977in}}%
\pgfpathlineto{\pgfqpoint{6.508520in}{1.469175in}}%
\pgfpathlineto{\pgfqpoint{6.505348in}{1.468827in}}%
\pgfpathlineto{\pgfqpoint{6.502176in}{1.468935in}}%
\pgfpathlineto{\pgfqpoint{6.499004in}{1.468694in}}%
\pgfpathlineto{\pgfqpoint{6.495832in}{1.468626in}}%
\pgfpathlineto{\pgfqpoint{6.492659in}{1.469016in}}%
\pgfpathlineto{\pgfqpoint{6.489487in}{1.468922in}}%
\pgfpathlineto{\pgfqpoint{6.486315in}{1.469193in}}%
\pgfpathlineto{\pgfqpoint{6.483143in}{1.469131in}}%
\pgfpathlineto{\pgfqpoint{6.479971in}{1.469072in}}%
\pgfpathlineto{\pgfqpoint{6.476799in}{1.469102in}}%
\pgfpathlineto{\pgfqpoint{6.473627in}{1.469176in}}%
\pgfpathlineto{\pgfqpoint{6.470455in}{1.468719in}}%
\pgfpathlineto{\pgfqpoint{6.467283in}{1.468931in}}%
\pgfpathlineto{\pgfqpoint{6.464111in}{1.469033in}}%
\pgfpathlineto{\pgfqpoint{6.460939in}{1.469250in}}%
\pgfpathlineto{\pgfqpoint{6.457767in}{1.469064in}}%
\pgfpathlineto{\pgfqpoint{6.454595in}{1.469071in}}%
\pgfpathlineto{\pgfqpoint{6.451423in}{1.469367in}}%
\pgfpathlineto{\pgfqpoint{6.448251in}{1.468595in}}%
\pgfpathlineto{\pgfqpoint{6.445079in}{1.468126in}}%
\pgfpathlineto{\pgfqpoint{6.441907in}{1.467985in}}%
\pgfpathlineto{\pgfqpoint{6.438735in}{1.468065in}}%
\pgfpathlineto{\pgfqpoint{6.435563in}{1.467982in}}%
\pgfpathlineto{\pgfqpoint{6.432391in}{1.467819in}}%
\pgfpathlineto{\pgfqpoint{6.429219in}{1.468110in}}%
\pgfpathlineto{\pgfqpoint{6.426047in}{1.467646in}}%
\pgfpathlineto{\pgfqpoint{6.422875in}{1.467486in}}%
\pgfpathlineto{\pgfqpoint{6.419703in}{1.467535in}}%
\pgfpathlineto{\pgfqpoint{6.416531in}{1.467261in}}%
\pgfpathlineto{\pgfqpoint{6.413358in}{1.467675in}}%
\pgfpathlineto{\pgfqpoint{6.410186in}{1.467928in}}%
\pgfpathlineto{\pgfqpoint{6.407014in}{1.467867in}}%
\pgfpathlineto{\pgfqpoint{6.403842in}{1.467768in}}%
\pgfpathlineto{\pgfqpoint{6.400670in}{1.468070in}}%
\pgfpathlineto{\pgfqpoint{6.397498in}{1.467768in}}%
\pgfpathlineto{\pgfqpoint{6.394326in}{1.467710in}}%
\pgfpathlineto{\pgfqpoint{6.391154in}{1.467855in}}%
\pgfpathlineto{\pgfqpoint{6.387982in}{1.468108in}}%
\pgfpathlineto{\pgfqpoint{6.384810in}{1.467544in}}%
\pgfpathlineto{\pgfqpoint{6.381638in}{1.467252in}}%
\pgfpathlineto{\pgfqpoint{6.378466in}{1.467088in}}%
\pgfpathlineto{\pgfqpoint{6.375294in}{1.467014in}}%
\pgfpathlineto{\pgfqpoint{6.372122in}{1.467035in}}%
\pgfpathlineto{\pgfqpoint{6.368950in}{1.467148in}}%
\pgfpathlineto{\pgfqpoint{6.365778in}{1.466812in}}%
\pgfpathlineto{\pgfqpoint{6.362606in}{1.466917in}}%
\pgfpathlineto{\pgfqpoint{6.359434in}{1.466744in}}%
\pgfpathlineto{\pgfqpoint{6.356262in}{1.466671in}}%
\pgfpathlineto{\pgfqpoint{6.353090in}{1.466466in}}%
\pgfpathlineto{\pgfqpoint{6.349918in}{1.466487in}}%
\pgfpathlineto{\pgfqpoint{6.346746in}{1.466651in}}%
\pgfpathlineto{\pgfqpoint{6.343574in}{1.466119in}}%
\pgfpathlineto{\pgfqpoint{6.340402in}{1.465935in}}%
\pgfpathlineto{\pgfqpoint{6.337230in}{1.465976in}}%
\pgfpathlineto{\pgfqpoint{6.334057in}{1.466134in}}%
\pgfpathlineto{\pgfqpoint{6.330885in}{1.466012in}}%
\pgfpathlineto{\pgfqpoint{6.327713in}{1.465760in}}%
\pgfpathlineto{\pgfqpoint{6.324541in}{1.465666in}}%
\pgfpathlineto{\pgfqpoint{6.321369in}{1.465605in}}%
\pgfpathlineto{\pgfqpoint{6.318197in}{1.465631in}}%
\pgfpathlineto{\pgfqpoint{6.315025in}{1.465582in}}%
\pgfpathlineto{\pgfqpoint{6.311853in}{1.465309in}}%
\pgfpathlineto{\pgfqpoint{6.308681in}{1.465317in}}%
\pgfpathlineto{\pgfqpoint{6.305509in}{1.465357in}}%
\pgfpathlineto{\pgfqpoint{6.302337in}{1.465266in}}%
\pgfpathlineto{\pgfqpoint{6.299165in}{1.465264in}}%
\pgfpathlineto{\pgfqpoint{6.295993in}{1.465203in}}%
\pgfpathlineto{\pgfqpoint{6.292821in}{1.465300in}}%
\pgfpathlineto{\pgfqpoint{6.289649in}{1.465071in}}%
\pgfpathlineto{\pgfqpoint{6.286477in}{1.465114in}}%
\pgfpathlineto{\pgfqpoint{6.283305in}{1.465204in}}%
\pgfpathlineto{\pgfqpoint{6.280133in}{1.464769in}}%
\pgfpathlineto{\pgfqpoint{6.276961in}{1.464941in}}%
\pgfpathlineto{\pgfqpoint{6.273789in}{1.464877in}}%
\pgfpathlineto{\pgfqpoint{6.270617in}{1.464942in}}%
\pgfpathlineto{\pgfqpoint{6.267445in}{1.465154in}}%
\pgfpathlineto{\pgfqpoint{6.264273in}{1.465048in}}%
\pgfpathlineto{\pgfqpoint{6.261101in}{1.465061in}}%
\pgfpathlineto{\pgfqpoint{6.257928in}{1.465016in}}%
\pgfpathlineto{\pgfqpoint{6.254756in}{1.464928in}}%
\pgfpathlineto{\pgfqpoint{6.251584in}{1.464714in}}%
\pgfpathlineto{\pgfqpoint{6.248412in}{1.464763in}}%
\pgfpathlineto{\pgfqpoint{6.245240in}{1.464775in}}%
\pgfpathlineto{\pgfqpoint{6.242068in}{1.464887in}}%
\pgfpathlineto{\pgfqpoint{6.238896in}{1.465242in}}%
\pgfpathlineto{\pgfqpoint{6.235724in}{1.465400in}}%
\pgfpathlineto{\pgfqpoint{6.232552in}{1.465294in}}%
\pgfpathlineto{\pgfqpoint{6.229380in}{1.465284in}}%
\pgfpathlineto{\pgfqpoint{6.226208in}{1.465154in}}%
\pgfpathlineto{\pgfqpoint{6.223036in}{1.464878in}}%
\pgfpathlineto{\pgfqpoint{6.219864in}{1.465146in}}%
\pgfpathlineto{\pgfqpoint{6.216692in}{1.465248in}}%
\pgfpathlineto{\pgfqpoint{6.213520in}{1.465463in}}%
\pgfpathlineto{\pgfqpoint{6.210348in}{1.465458in}}%
\pgfpathlineto{\pgfqpoint{6.207176in}{1.465179in}}%
\pgfpathlineto{\pgfqpoint{6.204004in}{1.464987in}}%
\pgfpathlineto{\pgfqpoint{6.200832in}{1.465297in}}%
\pgfpathlineto{\pgfqpoint{6.197660in}{1.464884in}}%
\pgfpathlineto{\pgfqpoint{6.194488in}{1.464417in}}%
\pgfpathlineto{\pgfqpoint{6.191316in}{1.464377in}}%
\pgfpathlineto{\pgfqpoint{6.188144in}{1.463841in}}%
\pgfpathlineto{\pgfqpoint{6.184972in}{1.463366in}}%
\pgfpathlineto{\pgfqpoint{6.181800in}{1.463350in}}%
\pgfpathlineto{\pgfqpoint{6.178627in}{1.463133in}}%
\pgfpathlineto{\pgfqpoint{6.175455in}{1.462648in}}%
\pgfpathlineto{\pgfqpoint{6.172283in}{1.462560in}}%
\pgfpathlineto{\pgfqpoint{6.169111in}{1.462698in}}%
\pgfpathlineto{\pgfqpoint{6.165939in}{1.462621in}}%
\pgfpathlineto{\pgfqpoint{6.162767in}{1.462416in}}%
\pgfpathlineto{\pgfqpoint{6.159595in}{1.462635in}}%
\pgfpathlineto{\pgfqpoint{6.156423in}{1.462499in}}%
\pgfpathlineto{\pgfqpoint{6.153251in}{1.462695in}}%
\pgfpathlineto{\pgfqpoint{6.150079in}{1.462622in}}%
\pgfpathlineto{\pgfqpoint{6.146907in}{1.462632in}}%
\pgfpathlineto{\pgfqpoint{6.143735in}{1.462494in}}%
\pgfpathlineto{\pgfqpoint{6.140563in}{1.461881in}}%
\pgfpathlineto{\pgfqpoint{6.137391in}{1.461387in}}%
\pgfpathlineto{\pgfqpoint{6.134219in}{1.461184in}}%
\pgfpathlineto{\pgfqpoint{6.131047in}{1.461102in}}%
\pgfpathlineto{\pgfqpoint{6.127875in}{1.461063in}}%
\pgfpathlineto{\pgfqpoint{6.124703in}{1.461275in}}%
\pgfpathlineto{\pgfqpoint{6.121531in}{1.461147in}}%
\pgfpathlineto{\pgfqpoint{6.118359in}{1.461223in}}%
\pgfpathlineto{\pgfqpoint{6.115187in}{1.461221in}}%
\pgfpathlineto{\pgfqpoint{6.112015in}{1.461263in}}%
\pgfpathlineto{\pgfqpoint{6.108843in}{1.461411in}}%
\pgfpathlineto{\pgfqpoint{6.105671in}{1.461632in}}%
\pgfpathlineto{\pgfqpoint{6.102499in}{1.461536in}}%
\pgfpathlineto{\pgfqpoint{6.099326in}{1.461648in}}%
\pgfpathlineto{\pgfqpoint{6.096154in}{1.461633in}}%
\pgfpathlineto{\pgfqpoint{6.092982in}{1.461634in}}%
\pgfpathlineto{\pgfqpoint{6.089810in}{1.461853in}}%
\pgfpathlineto{\pgfqpoint{6.086638in}{1.462290in}}%
\pgfpathlineto{\pgfqpoint{6.083466in}{1.461997in}}%
\pgfpathlineto{\pgfqpoint{6.080294in}{1.462313in}}%
\pgfpathlineto{\pgfqpoint{6.077122in}{1.461897in}}%
\pgfpathlineto{\pgfqpoint{6.073950in}{1.461873in}}%
\pgfpathlineto{\pgfqpoint{6.070778in}{1.461680in}}%
\pgfpathlineto{\pgfqpoint{6.067606in}{1.461636in}}%
\pgfpathlineto{\pgfqpoint{6.064434in}{1.461300in}}%
\pgfpathlineto{\pgfqpoint{6.061262in}{1.461236in}}%
\pgfpathlineto{\pgfqpoint{6.058090in}{1.460919in}}%
\pgfpathlineto{\pgfqpoint{6.054918in}{1.460856in}}%
\pgfpathlineto{\pgfqpoint{6.051746in}{1.460604in}}%
\pgfpathlineto{\pgfqpoint{6.048574in}{1.460458in}}%
\pgfpathlineto{\pgfqpoint{6.045402in}{1.460104in}}%
\pgfpathlineto{\pgfqpoint{6.042230in}{1.459886in}}%
\pgfpathlineto{\pgfqpoint{6.039058in}{1.459806in}}%
\pgfpathlineto{\pgfqpoint{6.035886in}{1.459546in}}%
\pgfpathlineto{\pgfqpoint{6.032714in}{1.459431in}}%
\pgfpathlineto{\pgfqpoint{6.029542in}{1.459281in}}%
\pgfpathlineto{\pgfqpoint{6.026370in}{1.458832in}}%
\pgfpathlineto{\pgfqpoint{6.023197in}{1.458765in}}%
\pgfpathlineto{\pgfqpoint{6.020025in}{1.458722in}}%
\pgfpathlineto{\pgfqpoint{6.016853in}{1.458639in}}%
\pgfpathlineto{\pgfqpoint{6.013681in}{1.458075in}}%
\pgfpathlineto{\pgfqpoint{6.010509in}{1.458004in}}%
\pgfpathlineto{\pgfqpoint{6.007337in}{1.457970in}}%
\pgfpathlineto{\pgfqpoint{6.004165in}{1.457800in}}%
\pgfpathlineto{\pgfqpoint{6.000993in}{1.457739in}}%
\pgfpathlineto{\pgfqpoint{5.997821in}{1.457694in}}%
\pgfpathlineto{\pgfqpoint{5.994649in}{1.457893in}}%
\pgfpathlineto{\pgfqpoint{5.991477in}{1.458169in}}%
\pgfpathlineto{\pgfqpoint{5.988305in}{1.458074in}}%
\pgfpathlineto{\pgfqpoint{5.985133in}{1.458056in}}%
\pgfpathlineto{\pgfqpoint{5.981961in}{1.458001in}}%
\pgfpathlineto{\pgfqpoint{5.978789in}{1.457118in}}%
\pgfpathlineto{\pgfqpoint{5.975617in}{1.456561in}}%
\pgfpathlineto{\pgfqpoint{5.972445in}{1.456531in}}%
\pgfpathlineto{\pgfqpoint{5.969273in}{1.456384in}}%
\pgfpathlineto{\pgfqpoint{5.966101in}{1.456442in}}%
\pgfpathlineto{\pgfqpoint{5.962929in}{1.456088in}}%
\pgfpathlineto{\pgfqpoint{5.959757in}{1.455531in}}%
\pgfpathlineto{\pgfqpoint{5.956585in}{1.455299in}}%
\pgfpathlineto{\pgfqpoint{5.953413in}{1.455139in}}%
\pgfpathlineto{\pgfqpoint{5.950241in}{1.455116in}}%
\pgfpathlineto{\pgfqpoint{5.947069in}{1.455195in}}%
\pgfpathlineto{\pgfqpoint{5.943896in}{1.455250in}}%
\pgfpathlineto{\pgfqpoint{5.940724in}{1.455280in}}%
\pgfpathlineto{\pgfqpoint{5.937552in}{1.455682in}}%
\pgfpathlineto{\pgfqpoint{5.934380in}{1.455922in}}%
\pgfpathlineto{\pgfqpoint{5.931208in}{1.455867in}}%
\pgfpathlineto{\pgfqpoint{5.928036in}{1.455937in}}%
\pgfpathlineto{\pgfqpoint{5.924864in}{1.455719in}}%
\pgfpathlineto{\pgfqpoint{5.921692in}{1.455934in}}%
\pgfpathlineto{\pgfqpoint{5.918520in}{1.455971in}}%
\pgfpathlineto{\pgfqpoint{5.915348in}{1.455828in}}%
\pgfpathlineto{\pgfqpoint{5.912176in}{1.455417in}}%
\pgfpathlineto{\pgfqpoint{5.909004in}{1.455708in}}%
\pgfpathlineto{\pgfqpoint{5.905832in}{1.455675in}}%
\pgfpathlineto{\pgfqpoint{5.902660in}{1.455822in}}%
\pgfpathlineto{\pgfqpoint{5.899488in}{1.455398in}}%
\pgfpathlineto{\pgfqpoint{5.896316in}{1.455073in}}%
\pgfpathlineto{\pgfqpoint{5.893144in}{1.455070in}}%
\pgfpathlineto{\pgfqpoint{5.889972in}{1.454805in}}%
\pgfpathlineto{\pgfqpoint{5.886800in}{1.454421in}}%
\pgfpathlineto{\pgfqpoint{5.883628in}{1.454820in}}%
\pgfpathlineto{\pgfqpoint{5.880456in}{1.455037in}}%
\pgfpathlineto{\pgfqpoint{5.877284in}{1.455660in}}%
\pgfpathlineto{\pgfqpoint{5.874112in}{1.455530in}}%
\pgfpathlineto{\pgfqpoint{5.870940in}{1.455407in}}%
\pgfpathlineto{\pgfqpoint{5.867768in}{1.454909in}}%
\pgfpathlineto{\pgfqpoint{5.864595in}{1.454684in}}%
\pgfpathlineto{\pgfqpoint{5.861423in}{1.454483in}}%
\pgfpathlineto{\pgfqpoint{5.858251in}{1.454305in}}%
\pgfpathlineto{\pgfqpoint{5.855079in}{1.454365in}}%
\pgfpathlineto{\pgfqpoint{5.851907in}{1.454403in}}%
\pgfpathlineto{\pgfqpoint{5.848735in}{1.454226in}}%
\pgfpathlineto{\pgfqpoint{5.845563in}{1.454587in}}%
\pgfpathlineto{\pgfqpoint{5.842391in}{1.454992in}}%
\pgfpathlineto{\pgfqpoint{5.839219in}{1.455108in}}%
\pgfpathlineto{\pgfqpoint{5.836047in}{1.455319in}}%
\pgfpathlineto{\pgfqpoint{5.832875in}{1.455146in}}%
\pgfpathlineto{\pgfqpoint{5.829703in}{1.455032in}}%
\pgfpathlineto{\pgfqpoint{5.826531in}{1.454532in}}%
\pgfpathlineto{\pgfqpoint{5.823359in}{1.454432in}}%
\pgfpathlineto{\pgfqpoint{5.820187in}{1.454496in}}%
\pgfpathlineto{\pgfqpoint{5.817015in}{1.454360in}}%
\pgfpathlineto{\pgfqpoint{5.813843in}{1.454228in}}%
\pgfpathlineto{\pgfqpoint{5.810671in}{1.453948in}}%
\pgfpathlineto{\pgfqpoint{5.807499in}{1.453970in}}%
\pgfpathlineto{\pgfqpoint{5.804327in}{1.453729in}}%
\pgfpathlineto{\pgfqpoint{5.801155in}{1.453333in}}%
\pgfpathlineto{\pgfqpoint{5.797983in}{1.452824in}}%
\pgfpathlineto{\pgfqpoint{5.794811in}{1.452559in}}%
\pgfpathlineto{\pgfqpoint{5.791639in}{1.452082in}}%
\pgfpathlineto{\pgfqpoint{5.788466in}{1.452055in}}%
\pgfpathlineto{\pgfqpoint{5.785294in}{1.452262in}}%
\pgfpathlineto{\pgfqpoint{5.782122in}{1.451814in}}%
\pgfpathlineto{\pgfqpoint{5.778950in}{1.451360in}}%
\pgfpathlineto{\pgfqpoint{5.775778in}{1.451122in}}%
\pgfpathlineto{\pgfqpoint{5.772606in}{1.451152in}}%
\pgfpathlineto{\pgfqpoint{5.769434in}{1.451486in}}%
\pgfpathlineto{\pgfqpoint{5.766262in}{1.451553in}}%
\pgfpathlineto{\pgfqpoint{5.763090in}{1.451361in}}%
\pgfpathlineto{\pgfqpoint{5.759918in}{1.451443in}}%
\pgfpathlineto{\pgfqpoint{5.756746in}{1.451506in}}%
\pgfpathlineto{\pgfqpoint{5.753574in}{1.451124in}}%
\pgfpathlineto{\pgfqpoint{5.750402in}{1.451089in}}%
\pgfpathlineto{\pgfqpoint{5.747230in}{1.451044in}}%
\pgfpathlineto{\pgfqpoint{5.744058in}{1.451000in}}%
\pgfpathlineto{\pgfqpoint{5.740886in}{1.451151in}}%
\pgfpathlineto{\pgfqpoint{5.737714in}{1.451035in}}%
\pgfpathlineto{\pgfqpoint{5.734542in}{1.451145in}}%
\pgfpathlineto{\pgfqpoint{5.731370in}{1.451125in}}%
\pgfpathlineto{\pgfqpoint{5.728198in}{1.451077in}}%
\pgfpathlineto{\pgfqpoint{5.725026in}{1.451032in}}%
\pgfpathlineto{\pgfqpoint{5.721854in}{1.450052in}}%
\pgfpathlineto{\pgfqpoint{5.718682in}{1.449821in}}%
\pgfpathlineto{\pgfqpoint{5.715510in}{1.450032in}}%
\pgfpathlineto{\pgfqpoint{5.712338in}{1.450100in}}%
\pgfpathlineto{\pgfqpoint{5.709165in}{1.450112in}}%
\pgfpathlineto{\pgfqpoint{5.705993in}{1.450222in}}%
\pgfpathlineto{\pgfqpoint{5.702821in}{1.450298in}}%
\pgfpathlineto{\pgfqpoint{5.699649in}{1.449901in}}%
\pgfpathlineto{\pgfqpoint{5.696477in}{1.449744in}}%
\pgfpathlineto{\pgfqpoint{5.693305in}{1.449681in}}%
\pgfpathlineto{\pgfqpoint{5.690133in}{1.449535in}}%
\pgfpathlineto{\pgfqpoint{5.686961in}{1.449263in}}%
\pgfpathlineto{\pgfqpoint{5.683789in}{1.448511in}}%
\pgfpathlineto{\pgfqpoint{5.680617in}{1.448372in}}%
\pgfpathlineto{\pgfqpoint{5.677445in}{1.448125in}}%
\pgfpathlineto{\pgfqpoint{5.674273in}{1.448084in}}%
\pgfpathlineto{\pgfqpoint{5.671101in}{1.448122in}}%
\pgfpathlineto{\pgfqpoint{5.667929in}{1.447585in}}%
\pgfpathlineto{\pgfqpoint{5.664757in}{1.447573in}}%
\pgfpathlineto{\pgfqpoint{5.661585in}{1.447828in}}%
\pgfpathlineto{\pgfqpoint{5.658413in}{1.447755in}}%
\pgfpathlineto{\pgfqpoint{5.655241in}{1.447690in}}%
\pgfpathlineto{\pgfqpoint{5.652069in}{1.447302in}}%
\pgfpathlineto{\pgfqpoint{5.648897in}{1.447161in}}%
\pgfpathlineto{\pgfqpoint{5.645725in}{1.446914in}}%
\pgfpathlineto{\pgfqpoint{5.642553in}{1.446706in}}%
\pgfpathlineto{\pgfqpoint{5.639381in}{1.447054in}}%
\pgfpathlineto{\pgfqpoint{5.636209in}{1.446747in}}%
\pgfpathlineto{\pgfqpoint{5.633037in}{1.446461in}}%
\pgfpathlineto{\pgfqpoint{5.629864in}{1.446234in}}%
\pgfpathlineto{\pgfqpoint{5.626692in}{1.445960in}}%
\pgfpathlineto{\pgfqpoint{5.623520in}{1.446126in}}%
\pgfpathlineto{\pgfqpoint{5.620348in}{1.445744in}}%
\pgfpathlineto{\pgfqpoint{5.617176in}{1.445818in}}%
\pgfpathlineto{\pgfqpoint{5.614004in}{1.445863in}}%
\pgfpathlineto{\pgfqpoint{5.610832in}{1.445913in}}%
\pgfpathlineto{\pgfqpoint{5.607660in}{1.446052in}}%
\pgfpathlineto{\pgfqpoint{5.604488in}{1.445517in}}%
\pgfpathlineto{\pgfqpoint{5.601316in}{1.444497in}}%
\pgfpathlineto{\pgfqpoint{5.598144in}{1.444292in}}%
\pgfpathlineto{\pgfqpoint{5.594972in}{1.444344in}}%
\pgfpathlineto{\pgfqpoint{5.591800in}{1.444537in}}%
\pgfpathlineto{\pgfqpoint{5.588628in}{1.444280in}}%
\pgfpathlineto{\pgfqpoint{5.585456in}{1.444475in}}%
\pgfpathlineto{\pgfqpoint{5.582284in}{1.444389in}}%
\pgfpathlineto{\pgfqpoint{5.579112in}{1.444483in}}%
\pgfpathlineto{\pgfqpoint{5.575940in}{1.444365in}}%
\pgfpathlineto{\pgfqpoint{5.572768in}{1.444189in}}%
\pgfpathlineto{\pgfqpoint{5.569596in}{1.444115in}}%
\pgfpathlineto{\pgfqpoint{5.566424in}{1.443825in}}%
\pgfpathlineto{\pgfqpoint{5.563252in}{1.444006in}}%
\pgfpathlineto{\pgfqpoint{5.560080in}{1.444206in}}%
\pgfpathlineto{\pgfqpoint{5.556908in}{1.444330in}}%
\pgfpathlineto{\pgfqpoint{5.553735in}{1.444293in}}%
\pgfpathlineto{\pgfqpoint{5.550563in}{1.444469in}}%
\pgfpathlineto{\pgfqpoint{5.547391in}{1.444177in}}%
\pgfpathlineto{\pgfqpoint{5.544219in}{1.443797in}}%
\pgfpathlineto{\pgfqpoint{5.541047in}{1.443699in}}%
\pgfpathlineto{\pgfqpoint{5.537875in}{1.443679in}}%
\pgfpathlineto{\pgfqpoint{5.534703in}{1.443980in}}%
\pgfpathlineto{\pgfqpoint{5.531531in}{1.444053in}}%
\pgfpathlineto{\pgfqpoint{5.528359in}{1.444176in}}%
\pgfpathlineto{\pgfqpoint{5.525187in}{1.444280in}}%
\pgfpathlineto{\pgfqpoint{5.522015in}{1.443939in}}%
\pgfpathlineto{\pgfqpoint{5.518843in}{1.444120in}}%
\pgfpathlineto{\pgfqpoint{5.515671in}{1.443606in}}%
\pgfpathlineto{\pgfqpoint{5.512499in}{1.443775in}}%
\pgfpathlineto{\pgfqpoint{5.509327in}{1.443303in}}%
\pgfpathlineto{\pgfqpoint{5.506155in}{1.442816in}}%
\pgfpathlineto{\pgfqpoint{5.502983in}{1.442738in}}%
\pgfpathlineto{\pgfqpoint{5.499811in}{1.442576in}}%
\pgfpathlineto{\pgfqpoint{5.496639in}{1.442716in}}%
\pgfpathlineto{\pgfqpoint{5.493467in}{1.442660in}}%
\pgfpathlineto{\pgfqpoint{5.490295in}{1.442650in}}%
\pgfpathlineto{\pgfqpoint{5.487123in}{1.442524in}}%
\pgfpathlineto{\pgfqpoint{5.483951in}{1.442252in}}%
\pgfpathlineto{\pgfqpoint{5.480779in}{1.441671in}}%
\pgfpathlineto{\pgfqpoint{5.477607in}{1.441252in}}%
\pgfpathlineto{\pgfqpoint{5.474434in}{1.441173in}}%
\pgfpathlineto{\pgfqpoint{5.471262in}{1.441250in}}%
\pgfpathlineto{\pgfqpoint{5.468090in}{1.441251in}}%
\pgfpathlineto{\pgfqpoint{5.464918in}{1.441355in}}%
\pgfpathlineto{\pgfqpoint{5.461746in}{1.441352in}}%
\pgfpathlineto{\pgfqpoint{5.458574in}{1.441275in}}%
\pgfpathlineto{\pgfqpoint{5.455402in}{1.440655in}}%
\pgfpathlineto{\pgfqpoint{5.452230in}{1.440581in}}%
\pgfpathlineto{\pgfqpoint{5.449058in}{1.440621in}}%
\pgfpathlineto{\pgfqpoint{5.445886in}{1.440657in}}%
\pgfpathlineto{\pgfqpoint{5.442714in}{1.440604in}}%
\pgfpathlineto{\pgfqpoint{5.439542in}{1.440327in}}%
\pgfpathlineto{\pgfqpoint{5.436370in}{1.440341in}}%
\pgfpathlineto{\pgfqpoint{5.433198in}{1.440438in}}%
\pgfpathlineto{\pgfqpoint{5.430026in}{1.440219in}}%
\pgfpathlineto{\pgfqpoint{5.426854in}{1.440357in}}%
\pgfpathlineto{\pgfqpoint{5.423682in}{1.440419in}}%
\pgfpathlineto{\pgfqpoint{5.420510in}{1.440268in}}%
\pgfpathlineto{\pgfqpoint{5.417338in}{1.439970in}}%
\pgfpathlineto{\pgfqpoint{5.414166in}{1.440367in}}%
\pgfpathlineto{\pgfqpoint{5.410994in}{1.440541in}}%
\pgfpathlineto{\pgfqpoint{5.407822in}{1.440741in}}%
\pgfpathlineto{\pgfqpoint{5.404650in}{1.440852in}}%
\pgfpathlineto{\pgfqpoint{5.401478in}{1.440835in}}%
\pgfpathlineto{\pgfqpoint{5.398306in}{1.441321in}}%
\pgfpathlineto{\pgfqpoint{5.395133in}{1.441410in}}%
\pgfpathlineto{\pgfqpoint{5.391961in}{1.441182in}}%
\pgfpathlineto{\pgfqpoint{5.388789in}{1.441250in}}%
\pgfpathlineto{\pgfqpoint{5.385617in}{1.440863in}}%
\pgfpathlineto{\pgfqpoint{5.382445in}{1.440808in}}%
\pgfpathlineto{\pgfqpoint{5.379273in}{1.441037in}}%
\pgfpathlineto{\pgfqpoint{5.376101in}{1.440961in}}%
\pgfpathlineto{\pgfqpoint{5.372929in}{1.440852in}}%
\pgfpathlineto{\pgfqpoint{5.369757in}{1.440853in}}%
\pgfpathlineto{\pgfqpoint{5.366585in}{1.440930in}}%
\pgfpathlineto{\pgfqpoint{5.363413in}{1.440988in}}%
\pgfpathlineto{\pgfqpoint{5.360241in}{1.440934in}}%
\pgfpathlineto{\pgfqpoint{5.357069in}{1.439902in}}%
\pgfpathlineto{\pgfqpoint{5.353897in}{1.439491in}}%
\pgfpathlineto{\pgfqpoint{5.350725in}{1.439546in}}%
\pgfpathlineto{\pgfqpoint{5.347553in}{1.439852in}}%
\pgfpathlineto{\pgfqpoint{5.344381in}{1.440244in}}%
\pgfpathlineto{\pgfqpoint{5.341209in}{1.440063in}}%
\pgfpathlineto{\pgfqpoint{5.338037in}{1.439833in}}%
\pgfpathlineto{\pgfqpoint{5.334865in}{1.439793in}}%
\pgfpathlineto{\pgfqpoint{5.331693in}{1.440224in}}%
\pgfpathlineto{\pgfqpoint{5.328521in}{1.440081in}}%
\pgfpathlineto{\pgfqpoint{5.325349in}{1.439527in}}%
\pgfpathlineto{\pgfqpoint{5.322177in}{1.439401in}}%
\pgfpathlineto{\pgfqpoint{5.319004in}{1.439199in}}%
\pgfpathlineto{\pgfqpoint{5.315832in}{1.438797in}}%
\pgfpathlineto{\pgfqpoint{5.312660in}{1.438258in}}%
\pgfpathlineto{\pgfqpoint{5.309488in}{1.437552in}}%
\pgfpathlineto{\pgfqpoint{5.306316in}{1.437477in}}%
\pgfpathlineto{\pgfqpoint{5.303144in}{1.437134in}}%
\pgfpathlineto{\pgfqpoint{5.299972in}{1.436970in}}%
\pgfpathlineto{\pgfqpoint{5.296800in}{1.436798in}}%
\pgfpathlineto{\pgfqpoint{5.293628in}{1.436905in}}%
\pgfpathlineto{\pgfqpoint{5.290456in}{1.436692in}}%
\pgfpathlineto{\pgfqpoint{5.287284in}{1.436420in}}%
\pgfpathlineto{\pgfqpoint{5.284112in}{1.436415in}}%
\pgfpathlineto{\pgfqpoint{5.280940in}{1.436633in}}%
\pgfpathlineto{\pgfqpoint{5.277768in}{1.436545in}}%
\pgfpathlineto{\pgfqpoint{5.274596in}{1.436522in}}%
\pgfpathlineto{\pgfqpoint{5.271424in}{1.436256in}}%
\pgfpathlineto{\pgfqpoint{5.268252in}{1.436152in}}%
\pgfpathlineto{\pgfqpoint{5.265080in}{1.435689in}}%
\pgfpathlineto{\pgfqpoint{5.261908in}{1.435793in}}%
\pgfpathlineto{\pgfqpoint{5.258736in}{1.435777in}}%
\pgfpathlineto{\pgfqpoint{5.255564in}{1.435303in}}%
\pgfpathlineto{\pgfqpoint{5.252392in}{1.435149in}}%
\pgfpathlineto{\pgfqpoint{5.249220in}{1.434978in}}%
\pgfpathlineto{\pgfqpoint{5.246048in}{1.435151in}}%
\pgfpathlineto{\pgfqpoint{5.242876in}{1.435009in}}%
\pgfpathlineto{\pgfqpoint{5.239703in}{1.434935in}}%
\pgfpathlineto{\pgfqpoint{5.236531in}{1.434537in}}%
\pgfpathlineto{\pgfqpoint{5.233359in}{1.434295in}}%
\pgfpathlineto{\pgfqpoint{5.230187in}{1.433907in}}%
\pgfpathlineto{\pgfqpoint{5.227015in}{1.433738in}}%
\pgfpathlineto{\pgfqpoint{5.223843in}{1.433545in}}%
\pgfpathlineto{\pgfqpoint{5.220671in}{1.433945in}}%
\pgfpathlineto{\pgfqpoint{5.217499in}{1.434162in}}%
\pgfpathlineto{\pgfqpoint{5.214327in}{1.434172in}}%
\pgfpathlineto{\pgfqpoint{5.211155in}{1.434437in}}%
\pgfpathlineto{\pgfqpoint{5.207983in}{1.435446in}}%
\pgfpathlineto{\pgfqpoint{5.204811in}{1.435452in}}%
\pgfpathlineto{\pgfqpoint{5.201639in}{1.435566in}}%
\pgfpathlineto{\pgfqpoint{5.198467in}{1.435926in}}%
\pgfpathlineto{\pgfqpoint{5.195295in}{1.435948in}}%
\pgfpathlineto{\pgfqpoint{5.192123in}{1.436139in}}%
\pgfpathlineto{\pgfqpoint{5.188951in}{1.436346in}}%
\pgfpathlineto{\pgfqpoint{5.185779in}{1.436647in}}%
\pgfpathlineto{\pgfqpoint{5.182607in}{1.436230in}}%
\pgfpathlineto{\pgfqpoint{5.179435in}{1.436174in}}%
\pgfpathlineto{\pgfqpoint{5.176263in}{1.436219in}}%
\pgfpathlineto{\pgfqpoint{5.173091in}{1.436544in}}%
\pgfpathlineto{\pgfqpoint{5.169919in}{1.437103in}}%
\pgfpathlineto{\pgfqpoint{5.166747in}{1.436811in}}%
\pgfpathlineto{\pgfqpoint{5.163575in}{1.436971in}}%
\pgfpathlineto{\pgfqpoint{5.160402in}{1.437134in}}%
\pgfpathlineto{\pgfqpoint{5.157230in}{1.437161in}}%
\pgfpathlineto{\pgfqpoint{5.154058in}{1.437161in}}%
\pgfpathlineto{\pgfqpoint{5.150886in}{1.437029in}}%
\pgfpathlineto{\pgfqpoint{5.147714in}{1.436967in}}%
\pgfpathlineto{\pgfqpoint{5.144542in}{1.436616in}}%
\pgfpathlineto{\pgfqpoint{5.141370in}{1.436604in}}%
\pgfpathlineto{\pgfqpoint{5.138198in}{1.436608in}}%
\pgfpathlineto{\pgfqpoint{5.135026in}{1.436846in}}%
\pgfpathlineto{\pgfqpoint{5.131854in}{1.436843in}}%
\pgfpathlineto{\pgfqpoint{5.128682in}{1.436751in}}%
\pgfpathlineto{\pgfqpoint{5.125510in}{1.436430in}}%
\pgfpathlineto{\pgfqpoint{5.122338in}{1.436361in}}%
\pgfpathlineto{\pgfqpoint{5.119166in}{1.436192in}}%
\pgfpathlineto{\pgfqpoint{5.115994in}{1.436209in}}%
\pgfpathlineto{\pgfqpoint{5.112822in}{1.436273in}}%
\pgfpathlineto{\pgfqpoint{5.109650in}{1.436346in}}%
\pgfpathlineto{\pgfqpoint{5.106478in}{1.436030in}}%
\pgfpathlineto{\pgfqpoint{5.103306in}{1.435626in}}%
\pgfpathlineto{\pgfqpoint{5.100134in}{1.435543in}}%
\pgfpathlineto{\pgfqpoint{5.096962in}{1.435095in}}%
\pgfpathlineto{\pgfqpoint{5.093790in}{1.434913in}}%
\pgfpathlineto{\pgfqpoint{5.090618in}{1.434780in}}%
\pgfpathlineto{\pgfqpoint{5.087446in}{1.435092in}}%
\pgfpathlineto{\pgfqpoint{5.084273in}{1.434940in}}%
\pgfpathlineto{\pgfqpoint{5.081101in}{1.434715in}}%
\pgfpathlineto{\pgfqpoint{5.077929in}{1.434951in}}%
\pgfpathlineto{\pgfqpoint{5.074757in}{1.434918in}}%
\pgfpathlineto{\pgfqpoint{5.071585in}{1.435089in}}%
\pgfpathlineto{\pgfqpoint{5.068413in}{1.435280in}}%
\pgfpathlineto{\pgfqpoint{5.065241in}{1.435304in}}%
\pgfpathlineto{\pgfqpoint{5.062069in}{1.435360in}}%
\pgfpathlineto{\pgfqpoint{5.058897in}{1.435481in}}%
\pgfpathlineto{\pgfqpoint{5.055725in}{1.434902in}}%
\pgfpathlineto{\pgfqpoint{5.052553in}{1.434685in}}%
\pgfpathlineto{\pgfqpoint{5.049381in}{1.434388in}}%
\pgfpathlineto{\pgfqpoint{5.046209in}{1.434968in}}%
\pgfpathlineto{\pgfqpoint{5.043037in}{1.434999in}}%
\pgfpathlineto{\pgfqpoint{5.039865in}{1.434463in}}%
\pgfpathlineto{\pgfqpoint{5.036693in}{1.434413in}}%
\pgfpathlineto{\pgfqpoint{5.033521in}{1.434366in}}%
\pgfpathlineto{\pgfqpoint{5.030349in}{1.434258in}}%
\pgfpathlineto{\pgfqpoint{5.027177in}{1.434293in}}%
\pgfpathlineto{\pgfqpoint{5.024005in}{1.434385in}}%
\pgfpathlineto{\pgfqpoint{5.020833in}{1.434823in}}%
\pgfpathlineto{\pgfqpoint{5.017661in}{1.434894in}}%
\pgfpathlineto{\pgfqpoint{5.014489in}{1.435236in}}%
\pgfpathlineto{\pgfqpoint{5.011317in}{1.435205in}}%
\pgfpathlineto{\pgfqpoint{5.008145in}{1.434988in}}%
\pgfpathlineto{\pgfqpoint{5.004972in}{1.434287in}}%
\pgfpathlineto{\pgfqpoint{5.001800in}{1.434065in}}%
\pgfpathlineto{\pgfqpoint{4.998628in}{1.433932in}}%
\pgfpathlineto{\pgfqpoint{4.995456in}{1.433594in}}%
\pgfpathlineto{\pgfqpoint{4.992284in}{1.433672in}}%
\pgfpathlineto{\pgfqpoint{4.989112in}{1.433113in}}%
\pgfpathlineto{\pgfqpoint{4.985940in}{1.432898in}}%
\pgfpathlineto{\pgfqpoint{4.982768in}{1.432711in}}%
\pgfpathlineto{\pgfqpoint{4.979596in}{1.432531in}}%
\pgfpathlineto{\pgfqpoint{4.976424in}{1.432266in}}%
\pgfpathlineto{\pgfqpoint{4.973252in}{1.432453in}}%
\pgfpathlineto{\pgfqpoint{4.970080in}{1.431934in}}%
\pgfpathlineto{\pgfqpoint{4.966908in}{1.432103in}}%
\pgfpathlineto{\pgfqpoint{4.963736in}{1.431903in}}%
\pgfpathlineto{\pgfqpoint{4.960564in}{1.432181in}}%
\pgfpathlineto{\pgfqpoint{4.957392in}{1.432370in}}%
\pgfpathlineto{\pgfqpoint{4.954220in}{1.432651in}}%
\pgfpathlineto{\pgfqpoint{4.951048in}{1.432841in}}%
\pgfpathlineto{\pgfqpoint{4.947876in}{1.433033in}}%
\pgfpathlineto{\pgfqpoint{4.944704in}{1.433065in}}%
\pgfpathlineto{\pgfqpoint{4.941532in}{1.433092in}}%
\pgfpathlineto{\pgfqpoint{4.938360in}{1.433148in}}%
\pgfpathlineto{\pgfqpoint{4.935188in}{1.433106in}}%
\pgfpathlineto{\pgfqpoint{4.932016in}{1.433268in}}%
\pgfpathlineto{\pgfqpoint{4.928844in}{1.433333in}}%
\pgfpathlineto{\pgfqpoint{4.925671in}{1.433226in}}%
\pgfpathlineto{\pgfqpoint{4.922499in}{1.432997in}}%
\pgfpathlineto{\pgfqpoint{4.919327in}{1.432669in}}%
\pgfpathlineto{\pgfqpoint{4.916155in}{1.432695in}}%
\pgfpathlineto{\pgfqpoint{4.912983in}{1.432962in}}%
\pgfpathlineto{\pgfqpoint{4.909811in}{1.433221in}}%
\pgfpathlineto{\pgfqpoint{4.906639in}{1.433410in}}%
\pgfpathlineto{\pgfqpoint{4.903467in}{1.432920in}}%
\pgfpathlineto{\pgfqpoint{4.900295in}{1.432437in}}%
\pgfpathlineto{\pgfqpoint{4.897123in}{1.432702in}}%
\pgfpathlineto{\pgfqpoint{4.893951in}{1.432492in}}%
\pgfpathlineto{\pgfqpoint{4.890779in}{1.432212in}}%
\pgfpathlineto{\pgfqpoint{4.887607in}{1.432161in}}%
\pgfpathlineto{\pgfqpoint{4.884435in}{1.432167in}}%
\pgfpathlineto{\pgfqpoint{4.881263in}{1.431932in}}%
\pgfpathlineto{\pgfqpoint{4.878091in}{1.431567in}}%
\pgfpathlineto{\pgfqpoint{4.874919in}{1.430647in}}%
\pgfpathlineto{\pgfqpoint{4.871747in}{1.431052in}}%
\pgfpathlineto{\pgfqpoint{4.868575in}{1.430867in}}%
\pgfpathlineto{\pgfqpoint{4.865403in}{1.430382in}}%
\pgfpathlineto{\pgfqpoint{4.862231in}{1.430219in}}%
\pgfpathlineto{\pgfqpoint{4.859059in}{1.430203in}}%
\pgfpathlineto{\pgfqpoint{4.855887in}{1.429773in}}%
\pgfpathlineto{\pgfqpoint{4.852715in}{1.429716in}}%
\pgfpathlineto{\pgfqpoint{4.849542in}{1.429720in}}%
\pgfpathlineto{\pgfqpoint{4.846370in}{1.429332in}}%
\pgfpathlineto{\pgfqpoint{4.843198in}{1.429301in}}%
\pgfpathlineto{\pgfqpoint{4.840026in}{1.428993in}}%
\pgfpathlineto{\pgfqpoint{4.836854in}{1.428905in}}%
\pgfpathlineto{\pgfqpoint{4.833682in}{1.429137in}}%
\pgfpathlineto{\pgfqpoint{4.830510in}{1.428982in}}%
\pgfpathlineto{\pgfqpoint{4.827338in}{1.428866in}}%
\pgfpathlineto{\pgfqpoint{4.824166in}{1.428736in}}%
\pgfpathlineto{\pgfqpoint{4.820994in}{1.428945in}}%
\pgfpathlineto{\pgfqpoint{4.817822in}{1.429010in}}%
\pgfpathlineto{\pgfqpoint{4.814650in}{1.428887in}}%
\pgfpathlineto{\pgfqpoint{4.811478in}{1.429081in}}%
\pgfpathlineto{\pgfqpoint{4.808306in}{1.428900in}}%
\pgfpathlineto{\pgfqpoint{4.805134in}{1.428985in}}%
\pgfpathlineto{\pgfqpoint{4.801962in}{1.429030in}}%
\pgfpathlineto{\pgfqpoint{4.798790in}{1.429091in}}%
\pgfpathlineto{\pgfqpoint{4.795618in}{1.428912in}}%
\pgfpathlineto{\pgfqpoint{4.792446in}{1.428605in}}%
\pgfpathlineto{\pgfqpoint{4.789274in}{1.428211in}}%
\pgfpathlineto{\pgfqpoint{4.786102in}{1.427889in}}%
\pgfpathlineto{\pgfqpoint{4.782930in}{1.428006in}}%
\pgfpathlineto{\pgfqpoint{4.779758in}{1.428151in}}%
\pgfpathlineto{\pgfqpoint{4.776586in}{1.428438in}}%
\pgfpathlineto{\pgfqpoint{4.773414in}{1.428419in}}%
\pgfpathlineto{\pgfqpoint{4.770241in}{1.428488in}}%
\pgfpathlineto{\pgfqpoint{4.767069in}{1.428317in}}%
\pgfpathlineto{\pgfqpoint{4.763897in}{1.428545in}}%
\pgfpathlineto{\pgfqpoint{4.760725in}{1.428195in}}%
\pgfpathlineto{\pgfqpoint{4.757553in}{1.428269in}}%
\pgfpathlineto{\pgfqpoint{4.754381in}{1.428190in}}%
\pgfpathlineto{\pgfqpoint{4.751209in}{1.428130in}}%
\pgfpathlineto{\pgfqpoint{4.748037in}{1.428094in}}%
\pgfpathlineto{\pgfqpoint{4.744865in}{1.428133in}}%
\pgfpathlineto{\pgfqpoint{4.741693in}{1.428167in}}%
\pgfpathlineto{\pgfqpoint{4.738521in}{1.428015in}}%
\pgfpathlineto{\pgfqpoint{4.735349in}{1.428100in}}%
\pgfpathlineto{\pgfqpoint{4.732177in}{1.427747in}}%
\pgfpathlineto{\pgfqpoint{4.729005in}{1.427873in}}%
\pgfpathlineto{\pgfqpoint{4.725833in}{1.428024in}}%
\pgfpathlineto{\pgfqpoint{4.722661in}{1.428019in}}%
\pgfpathlineto{\pgfqpoint{4.719489in}{1.427972in}}%
\pgfpathlineto{\pgfqpoint{4.716317in}{1.427907in}}%
\pgfpathlineto{\pgfqpoint{4.713145in}{1.428009in}}%
\pgfpathlineto{\pgfqpoint{4.709973in}{1.428045in}}%
\pgfpathlineto{\pgfqpoint{4.706801in}{1.427596in}}%
\pgfpathlineto{\pgfqpoint{4.703629in}{1.427850in}}%
\pgfpathlineto{\pgfqpoint{4.700457in}{1.427732in}}%
\pgfpathlineto{\pgfqpoint{4.697285in}{1.427938in}}%
\pgfpathlineto{\pgfqpoint{4.694112in}{1.427814in}}%
\pgfpathlineto{\pgfqpoint{4.690940in}{1.427590in}}%
\pgfpathlineto{\pgfqpoint{4.687768in}{1.427950in}}%
\pgfpathlineto{\pgfqpoint{4.684596in}{1.428019in}}%
\pgfpathlineto{\pgfqpoint{4.681424in}{1.427259in}}%
\pgfpathlineto{\pgfqpoint{4.678252in}{1.427328in}}%
\pgfpathlineto{\pgfqpoint{4.675080in}{1.427318in}}%
\pgfpathlineto{\pgfqpoint{4.671908in}{1.427360in}}%
\pgfpathlineto{\pgfqpoint{4.668736in}{1.427465in}}%
\pgfpathlineto{\pgfqpoint{4.665564in}{1.427420in}}%
\pgfpathlineto{\pgfqpoint{4.662392in}{1.427014in}}%
\pgfpathlineto{\pgfqpoint{4.659220in}{1.427048in}}%
\pgfpathlineto{\pgfqpoint{4.656048in}{1.426828in}}%
\pgfpathlineto{\pgfqpoint{4.652876in}{1.427115in}}%
\pgfpathlineto{\pgfqpoint{4.649704in}{1.427017in}}%
\pgfpathlineto{\pgfqpoint{4.646532in}{1.427165in}}%
\pgfpathlineto{\pgfqpoint{4.643360in}{1.426993in}}%
\pgfpathlineto{\pgfqpoint{4.640188in}{1.426938in}}%
\pgfpathlineto{\pgfqpoint{4.637016in}{1.426771in}}%
\pgfpathlineto{\pgfqpoint{4.633844in}{1.426535in}}%
\pgfpathlineto{\pgfqpoint{4.630672in}{1.426298in}}%
\pgfpathlineto{\pgfqpoint{4.627500in}{1.426126in}}%
\pgfpathlineto{\pgfqpoint{4.624328in}{1.426253in}}%
\pgfpathlineto{\pgfqpoint{4.621156in}{1.426368in}}%
\pgfpathlineto{\pgfqpoint{4.617984in}{1.426490in}}%
\pgfpathlineto{\pgfqpoint{4.614811in}{1.426338in}}%
\pgfpathlineto{\pgfqpoint{4.611639in}{1.425999in}}%
\pgfpathlineto{\pgfqpoint{4.608467in}{1.425917in}}%
\pgfpathlineto{\pgfqpoint{4.605295in}{1.425989in}}%
\pgfpathlineto{\pgfqpoint{4.602123in}{1.425943in}}%
\pgfpathlineto{\pgfqpoint{4.598951in}{1.425921in}}%
\pgfpathlineto{\pgfqpoint{4.595779in}{1.425538in}}%
\pgfpathlineto{\pgfqpoint{4.592607in}{1.425220in}}%
\pgfpathlineto{\pgfqpoint{4.589435in}{1.425161in}}%
\pgfpathlineto{\pgfqpoint{4.586263in}{1.425026in}}%
\pgfpathlineto{\pgfqpoint{4.583091in}{1.425134in}}%
\pgfpathlineto{\pgfqpoint{4.579919in}{1.425429in}}%
\pgfpathlineto{\pgfqpoint{4.576747in}{1.425728in}}%
\pgfpathlineto{\pgfqpoint{4.573575in}{1.426072in}}%
\pgfpathlineto{\pgfqpoint{4.570403in}{1.425988in}}%
\pgfpathlineto{\pgfqpoint{4.567231in}{1.426057in}}%
\pgfpathlineto{\pgfqpoint{4.564059in}{1.426359in}}%
\pgfpathlineto{\pgfqpoint{4.560887in}{1.426191in}}%
\pgfpathlineto{\pgfqpoint{4.557715in}{1.425773in}}%
\pgfpathlineto{\pgfqpoint{4.554543in}{1.425335in}}%
\pgfpathlineto{\pgfqpoint{4.551371in}{1.425136in}}%
\pgfpathlineto{\pgfqpoint{4.548199in}{1.424750in}}%
\pgfpathlineto{\pgfqpoint{4.545027in}{1.424591in}}%
\pgfpathlineto{\pgfqpoint{4.541855in}{1.424593in}}%
\pgfpathlineto{\pgfqpoint{4.538683in}{1.424502in}}%
\pgfpathlineto{\pgfqpoint{4.535510in}{1.424357in}}%
\pgfpathlineto{\pgfqpoint{4.532338in}{1.424526in}}%
\pgfpathlineto{\pgfqpoint{4.529166in}{1.424748in}}%
\pgfpathlineto{\pgfqpoint{4.525994in}{1.424340in}}%
\pgfpathlineto{\pgfqpoint{4.522822in}{1.424032in}}%
\pgfpathlineto{\pgfqpoint{4.519650in}{1.423631in}}%
\pgfpathlineto{\pgfqpoint{4.516478in}{1.423372in}}%
\pgfpathlineto{\pgfqpoint{4.513306in}{1.423642in}}%
\pgfpathlineto{\pgfqpoint{4.510134in}{1.423564in}}%
\pgfpathlineto{\pgfqpoint{4.506962in}{1.423541in}}%
\pgfpathlineto{\pgfqpoint{4.503790in}{1.423315in}}%
\pgfpathlineto{\pgfqpoint{4.500618in}{1.422936in}}%
\pgfpathlineto{\pgfqpoint{4.497446in}{1.422668in}}%
\pgfpathlineto{\pgfqpoint{4.494274in}{1.422652in}}%
\pgfpathlineto{\pgfqpoint{4.491102in}{1.422556in}}%
\pgfpathlineto{\pgfqpoint{4.487930in}{1.422452in}}%
\pgfpathlineto{\pgfqpoint{4.484758in}{1.422452in}}%
\pgfpathlineto{\pgfqpoint{4.481586in}{1.422438in}}%
\pgfpathlineto{\pgfqpoint{4.478414in}{1.422288in}}%
\pgfpathlineto{\pgfqpoint{4.475242in}{1.421854in}}%
\pgfpathlineto{\pgfqpoint{4.472070in}{1.421513in}}%
\pgfpathlineto{\pgfqpoint{4.468898in}{1.421071in}}%
\pgfpathlineto{\pgfqpoint{4.465726in}{1.420428in}}%
\pgfpathlineto{\pgfqpoint{4.462554in}{1.420228in}}%
\pgfpathlineto{\pgfqpoint{4.459381in}{1.420271in}}%
\pgfpathlineto{\pgfqpoint{4.456209in}{1.420259in}}%
\pgfpathlineto{\pgfqpoint{4.453037in}{1.420219in}}%
\pgfpathlineto{\pgfqpoint{4.449865in}{1.420459in}}%
\pgfpathlineto{\pgfqpoint{4.446693in}{1.420454in}}%
\pgfpathlineto{\pgfqpoint{4.443521in}{1.420497in}}%
\pgfpathlineto{\pgfqpoint{4.440349in}{1.420074in}}%
\pgfpathlineto{\pgfqpoint{4.437177in}{1.419935in}}%
\pgfpathlineto{\pgfqpoint{4.434005in}{1.419911in}}%
\pgfpathlineto{\pgfqpoint{4.430833in}{1.420182in}}%
\pgfpathlineto{\pgfqpoint{4.427661in}{1.419497in}}%
\pgfpathlineto{\pgfqpoint{4.424489in}{1.419244in}}%
\pgfpathlineto{\pgfqpoint{4.421317in}{1.419096in}}%
\pgfpathlineto{\pgfqpoint{4.418145in}{1.418876in}}%
\pgfpathlineto{\pgfqpoint{4.414973in}{1.418972in}}%
\pgfpathlineto{\pgfqpoint{4.411801in}{1.418813in}}%
\pgfpathlineto{\pgfqpoint{4.408629in}{1.418868in}}%
\pgfpathlineto{\pgfqpoint{4.405457in}{1.418924in}}%
\pgfpathlineto{\pgfqpoint{4.402285in}{1.418921in}}%
\pgfpathlineto{\pgfqpoint{4.399113in}{1.418652in}}%
\pgfpathlineto{\pgfqpoint{4.395941in}{1.418643in}}%
\pgfpathlineto{\pgfqpoint{4.392769in}{1.418424in}}%
\pgfpathlineto{\pgfqpoint{4.389597in}{1.418147in}}%
\pgfpathlineto{\pgfqpoint{4.386425in}{1.418031in}}%
\pgfpathlineto{\pgfqpoint{4.383253in}{1.418056in}}%
\pgfpathlineto{\pgfqpoint{4.380080in}{1.418052in}}%
\pgfpathlineto{\pgfqpoint{4.376908in}{1.418022in}}%
\pgfpathlineto{\pgfqpoint{4.373736in}{1.417823in}}%
\pgfpathlineto{\pgfqpoint{4.370564in}{1.417918in}}%
\pgfpathlineto{\pgfqpoint{4.367392in}{1.417995in}}%
\pgfpathlineto{\pgfqpoint{4.364220in}{1.417693in}}%
\pgfpathlineto{\pgfqpoint{4.361048in}{1.417747in}}%
\pgfpathlineto{\pgfqpoint{4.357876in}{1.417771in}}%
\pgfpathlineto{\pgfqpoint{4.354704in}{1.417766in}}%
\pgfpathlineto{\pgfqpoint{4.351532in}{1.417543in}}%
\pgfpathlineto{\pgfqpoint{4.348360in}{1.417644in}}%
\pgfpathlineto{\pgfqpoint{4.345188in}{1.417487in}}%
\pgfpathlineto{\pgfqpoint{4.342016in}{1.417197in}}%
\pgfpathlineto{\pgfqpoint{4.338844in}{1.417095in}}%
\pgfpathlineto{\pgfqpoint{4.335672in}{1.416963in}}%
\pgfpathlineto{\pgfqpoint{4.332500in}{1.417103in}}%
\pgfpathlineto{\pgfqpoint{4.329328in}{1.417222in}}%
\pgfpathlineto{\pgfqpoint{4.326156in}{1.417263in}}%
\pgfpathlineto{\pgfqpoint{4.322984in}{1.417100in}}%
\pgfpathlineto{\pgfqpoint{4.319812in}{1.416518in}}%
\pgfpathlineto{\pgfqpoint{4.316640in}{1.416485in}}%
\pgfpathlineto{\pgfqpoint{4.313468in}{1.416065in}}%
\pgfpathlineto{\pgfqpoint{4.310296in}{1.415998in}}%
\pgfpathlineto{\pgfqpoint{4.307124in}{1.416110in}}%
\pgfpathlineto{\pgfqpoint{4.303952in}{1.416310in}}%
\pgfpathlineto{\pgfqpoint{4.300779in}{1.416779in}}%
\pgfpathlineto{\pgfqpoint{4.297607in}{1.416892in}}%
\pgfpathlineto{\pgfqpoint{4.294435in}{1.416880in}}%
\pgfpathlineto{\pgfqpoint{4.291263in}{1.417352in}}%
\pgfpathlineto{\pgfqpoint{4.288091in}{1.417125in}}%
\pgfpathlineto{\pgfqpoint{4.284919in}{1.416860in}}%
\pgfpathlineto{\pgfqpoint{4.281747in}{1.416551in}}%
\pgfpathlineto{\pgfqpoint{4.278575in}{1.416751in}}%
\pgfpathlineto{\pgfqpoint{4.275403in}{1.416792in}}%
\pgfpathlineto{\pgfqpoint{4.272231in}{1.416782in}}%
\pgfpathlineto{\pgfqpoint{4.269059in}{1.416970in}}%
\pgfpathlineto{\pgfqpoint{4.265887in}{1.416464in}}%
\pgfpathlineto{\pgfqpoint{4.262715in}{1.416345in}}%
\pgfpathlineto{\pgfqpoint{4.259543in}{1.416570in}}%
\pgfpathlineto{\pgfqpoint{4.256371in}{1.416410in}}%
\pgfpathlineto{\pgfqpoint{4.253199in}{1.415809in}}%
\pgfpathlineto{\pgfqpoint{4.250027in}{1.415333in}}%
\pgfpathlineto{\pgfqpoint{4.246855in}{1.415048in}}%
\pgfpathlineto{\pgfqpoint{4.243683in}{1.414510in}}%
\pgfpathlineto{\pgfqpoint{4.240511in}{1.414638in}}%
\pgfpathlineto{\pgfqpoint{4.237339in}{1.414668in}}%
\pgfpathlineto{\pgfqpoint{4.234167in}{1.414583in}}%
\pgfpathlineto{\pgfqpoint{4.230995in}{1.414284in}}%
\pgfpathlineto{\pgfqpoint{4.227823in}{1.414058in}}%
\pgfpathlineto{\pgfqpoint{4.224650in}{1.414398in}}%
\pgfpathlineto{\pgfqpoint{4.221478in}{1.414259in}}%
\pgfpathlineto{\pgfqpoint{4.218306in}{1.414450in}}%
\pgfpathlineto{\pgfqpoint{4.215134in}{1.414242in}}%
\pgfpathlineto{\pgfqpoint{4.211962in}{1.414407in}}%
\pgfpathlineto{\pgfqpoint{4.208790in}{1.414544in}}%
\pgfpathlineto{\pgfqpoint{4.205618in}{1.414446in}}%
\pgfpathlineto{\pgfqpoint{4.202446in}{1.414366in}}%
\pgfpathlineto{\pgfqpoint{4.199274in}{1.413820in}}%
\pgfpathlineto{\pgfqpoint{4.196102in}{1.413275in}}%
\pgfpathlineto{\pgfqpoint{4.192930in}{1.412988in}}%
\pgfpathlineto{\pgfqpoint{4.189758in}{1.412511in}}%
\pgfpathlineto{\pgfqpoint{4.186586in}{1.412443in}}%
\pgfpathlineto{\pgfqpoint{4.183414in}{1.412443in}}%
\pgfpathlineto{\pgfqpoint{4.180242in}{1.412258in}}%
\pgfpathlineto{\pgfqpoint{4.177070in}{1.411911in}}%
\pgfpathlineto{\pgfqpoint{4.173898in}{1.411546in}}%
\pgfpathlineto{\pgfqpoint{4.170726in}{1.411730in}}%
\pgfpathlineto{\pgfqpoint{4.167554in}{1.411339in}}%
\pgfpathlineto{\pgfqpoint{4.164382in}{1.411114in}}%
\pgfpathlineto{\pgfqpoint{4.161210in}{1.411168in}}%
\pgfpathlineto{\pgfqpoint{4.158038in}{1.410977in}}%
\pgfpathlineto{\pgfqpoint{4.154866in}{1.411087in}}%
\pgfpathlineto{\pgfqpoint{4.151694in}{1.410849in}}%
\pgfpathlineto{\pgfqpoint{4.148522in}{1.410925in}}%
\pgfpathlineto{\pgfqpoint{4.145349in}{1.410737in}}%
\pgfpathlineto{\pgfqpoint{4.142177in}{1.410709in}}%
\pgfpathlineto{\pgfqpoint{4.139005in}{1.410620in}}%
\pgfpathlineto{\pgfqpoint{4.135833in}{1.410618in}}%
\pgfpathlineto{\pgfqpoint{4.132661in}{1.410694in}}%
\pgfpathlineto{\pgfqpoint{4.129489in}{1.410753in}}%
\pgfpathlineto{\pgfqpoint{4.126317in}{1.410856in}}%
\pgfpathlineto{\pgfqpoint{4.123145in}{1.410859in}}%
\pgfpathlineto{\pgfqpoint{4.119973in}{1.410699in}}%
\pgfpathlineto{\pgfqpoint{4.116801in}{1.410189in}}%
\pgfpathlineto{\pgfqpoint{4.113629in}{1.409976in}}%
\pgfpathlineto{\pgfqpoint{4.110457in}{1.409943in}}%
\pgfpathlineto{\pgfqpoint{4.107285in}{1.409776in}}%
\pgfpathlineto{\pgfqpoint{4.104113in}{1.409720in}}%
\pgfpathlineto{\pgfqpoint{4.100941in}{1.409525in}}%
\pgfpathlineto{\pgfqpoint{4.097769in}{1.409038in}}%
\pgfpathlineto{\pgfqpoint{4.094597in}{1.408819in}}%
\pgfpathlineto{\pgfqpoint{4.091425in}{1.408465in}}%
\pgfpathlineto{\pgfqpoint{4.088253in}{1.408296in}}%
\pgfpathlineto{\pgfqpoint{4.085081in}{1.408108in}}%
\pgfpathlineto{\pgfqpoint{4.081909in}{1.408203in}}%
\pgfpathlineto{\pgfqpoint{4.078737in}{1.407993in}}%
\pgfpathlineto{\pgfqpoint{4.075565in}{1.407818in}}%
\pgfpathlineto{\pgfqpoint{4.072393in}{1.407704in}}%
\pgfpathlineto{\pgfqpoint{4.069221in}{1.407836in}}%
\pgfpathlineto{\pgfqpoint{4.066048in}{1.407745in}}%
\pgfpathlineto{\pgfqpoint{4.062876in}{1.407707in}}%
\pgfpathlineto{\pgfqpoint{4.059704in}{1.408088in}}%
\pgfpathlineto{\pgfqpoint{4.056532in}{1.408033in}}%
\pgfpathlineto{\pgfqpoint{4.053360in}{1.408035in}}%
\pgfpathlineto{\pgfqpoint{4.050188in}{1.407803in}}%
\pgfpathlineto{\pgfqpoint{4.047016in}{1.407709in}}%
\pgfpathlineto{\pgfqpoint{4.043844in}{1.407947in}}%
\pgfpathlineto{\pgfqpoint{4.040672in}{1.408195in}}%
\pgfpathlineto{\pgfqpoint{4.037500in}{1.408030in}}%
\pgfpathlineto{\pgfqpoint{4.034328in}{1.408098in}}%
\pgfpathlineto{\pgfqpoint{4.031156in}{1.408178in}}%
\pgfpathlineto{\pgfqpoint{4.027984in}{1.408285in}}%
\pgfpathlineto{\pgfqpoint{4.024812in}{1.408014in}}%
\pgfpathlineto{\pgfqpoint{4.021640in}{1.407719in}}%
\pgfpathlineto{\pgfqpoint{4.018468in}{1.407648in}}%
\pgfpathlineto{\pgfqpoint{4.015296in}{1.407501in}}%
\pgfpathlineto{\pgfqpoint{4.012124in}{1.407165in}}%
\pgfpathlineto{\pgfqpoint{4.008952in}{1.406893in}}%
\pgfpathlineto{\pgfqpoint{4.005780in}{1.406753in}}%
\pgfpathlineto{\pgfqpoint{4.002608in}{1.406515in}}%
\pgfpathlineto{\pgfqpoint{3.999436in}{1.406162in}}%
\pgfpathlineto{\pgfqpoint{3.996264in}{1.405858in}}%
\pgfpathlineto{\pgfqpoint{3.993092in}{1.405538in}}%
\pgfpathlineto{\pgfqpoint{3.989919in}{1.405289in}}%
\pgfpathlineto{\pgfqpoint{3.986747in}{1.405028in}}%
\pgfpathlineto{\pgfqpoint{3.983575in}{1.405025in}}%
\pgfpathlineto{\pgfqpoint{3.980403in}{1.405026in}}%
\pgfpathlineto{\pgfqpoint{3.977231in}{1.405194in}}%
\pgfpathlineto{\pgfqpoint{3.974059in}{1.405159in}}%
\pgfpathlineto{\pgfqpoint{3.970887in}{1.405122in}}%
\pgfpathlineto{\pgfqpoint{3.967715in}{1.405212in}}%
\pgfpathlineto{\pgfqpoint{3.964543in}{1.404972in}}%
\pgfpathlineto{\pgfqpoint{3.961371in}{1.404830in}}%
\pgfpathlineto{\pgfqpoint{3.958199in}{1.404773in}}%
\pgfpathlineto{\pgfqpoint{3.955027in}{1.404709in}}%
\pgfpathlineto{\pgfqpoint{3.951855in}{1.404919in}}%
\pgfpathlineto{\pgfqpoint{3.948683in}{1.404916in}}%
\pgfpathlineto{\pgfqpoint{3.945511in}{1.404864in}}%
\pgfpathlineto{\pgfqpoint{3.942339in}{1.404594in}}%
\pgfpathlineto{\pgfqpoint{3.939167in}{1.404802in}}%
\pgfpathlineto{\pgfqpoint{3.935995in}{1.404665in}}%
\pgfpathlineto{\pgfqpoint{3.932823in}{1.404879in}}%
\pgfpathlineto{\pgfqpoint{3.929651in}{1.405113in}}%
\pgfpathlineto{\pgfqpoint{3.926479in}{1.405350in}}%
\pgfpathlineto{\pgfqpoint{3.923307in}{1.405408in}}%
\pgfpathlineto{\pgfqpoint{3.920135in}{1.405375in}}%
\pgfpathlineto{\pgfqpoint{3.916963in}{1.405405in}}%
\pgfpathlineto{\pgfqpoint{3.913791in}{1.405178in}}%
\pgfpathlineto{\pgfqpoint{3.910618in}{1.405540in}}%
\pgfpathlineto{\pgfqpoint{3.907446in}{1.405586in}}%
\pgfpathlineto{\pgfqpoint{3.904274in}{1.405172in}}%
\pgfpathlineto{\pgfqpoint{3.901102in}{1.404882in}}%
\pgfpathlineto{\pgfqpoint{3.897930in}{1.404868in}}%
\pgfpathlineto{\pgfqpoint{3.894758in}{1.404880in}}%
\pgfpathlineto{\pgfqpoint{3.891586in}{1.404419in}}%
\pgfpathlineto{\pgfqpoint{3.888414in}{1.404364in}}%
\pgfpathlineto{\pgfqpoint{3.885242in}{1.404499in}}%
\pgfpathlineto{\pgfqpoint{3.882070in}{1.404249in}}%
\pgfpathlineto{\pgfqpoint{3.878898in}{1.404702in}}%
\pgfpathlineto{\pgfqpoint{3.875726in}{1.405212in}}%
\pgfpathlineto{\pgfqpoint{3.872554in}{1.405638in}}%
\pgfpathlineto{\pgfqpoint{3.869382in}{1.405572in}}%
\pgfpathlineto{\pgfqpoint{3.866210in}{1.405802in}}%
\pgfpathlineto{\pgfqpoint{3.863038in}{1.405734in}}%
\pgfpathlineto{\pgfqpoint{3.859866in}{1.405715in}}%
\pgfpathlineto{\pgfqpoint{3.856694in}{1.405906in}}%
\pgfpathlineto{\pgfqpoint{3.853522in}{1.405389in}}%
\pgfpathlineto{\pgfqpoint{3.850350in}{1.404891in}}%
\pgfpathlineto{\pgfqpoint{3.847178in}{1.404852in}}%
\pgfpathlineto{\pgfqpoint{3.844006in}{1.404752in}}%
\pgfpathlineto{\pgfqpoint{3.840834in}{1.404884in}}%
\pgfpathlineto{\pgfqpoint{3.837662in}{1.404833in}}%
\pgfpathlineto{\pgfqpoint{3.834490in}{1.404825in}}%
\pgfpathlineto{\pgfqpoint{3.831317in}{1.405274in}}%
\pgfpathlineto{\pgfqpoint{3.828145in}{1.404885in}}%
\pgfpathlineto{\pgfqpoint{3.824973in}{1.404885in}}%
\pgfpathlineto{\pgfqpoint{3.821801in}{1.404850in}}%
\pgfpathlineto{\pgfqpoint{3.818629in}{1.404886in}}%
\pgfpathlineto{\pgfqpoint{3.815457in}{1.404854in}}%
\pgfpathlineto{\pgfqpoint{3.812285in}{1.404826in}}%
\pgfpathlineto{\pgfqpoint{3.809113in}{1.404815in}}%
\pgfpathlineto{\pgfqpoint{3.805941in}{1.404754in}}%
\pgfpathlineto{\pgfqpoint{3.802769in}{1.404675in}}%
\pgfpathlineto{\pgfqpoint{3.799597in}{1.404549in}}%
\pgfpathlineto{\pgfqpoint{3.796425in}{1.404381in}}%
\pgfpathlineto{\pgfqpoint{3.793253in}{1.404118in}}%
\pgfpathlineto{\pgfqpoint{3.790081in}{1.404080in}}%
\pgfpathlineto{\pgfqpoint{3.786909in}{1.404018in}}%
\pgfpathlineto{\pgfqpoint{3.783737in}{1.403782in}}%
\pgfpathlineto{\pgfqpoint{3.780565in}{1.403620in}}%
\pgfpathlineto{\pgfqpoint{3.777393in}{1.403589in}}%
\pgfpathlineto{\pgfqpoint{3.774221in}{1.403653in}}%
\pgfpathlineto{\pgfqpoint{3.771049in}{1.403586in}}%
\pgfpathlineto{\pgfqpoint{3.767877in}{1.403737in}}%
\pgfpathlineto{\pgfqpoint{3.764705in}{1.403620in}}%
\pgfpathlineto{\pgfqpoint{3.761533in}{1.403797in}}%
\pgfpathlineto{\pgfqpoint{3.758361in}{1.403878in}}%
\pgfpathlineto{\pgfqpoint{3.755188in}{1.403814in}}%
\pgfpathlineto{\pgfqpoint{3.752016in}{1.403703in}}%
\pgfpathlineto{\pgfqpoint{3.748844in}{1.403329in}}%
\pgfpathlineto{\pgfqpoint{3.745672in}{1.403850in}}%
\pgfpathlineto{\pgfqpoint{3.742500in}{1.403284in}}%
\pgfpathlineto{\pgfqpoint{3.739328in}{1.402695in}}%
\pgfpathlineto{\pgfqpoint{3.736156in}{1.402282in}}%
\pgfpathlineto{\pgfqpoint{3.732984in}{1.401452in}}%
\pgfpathlineto{\pgfqpoint{3.729812in}{1.400769in}}%
\pgfpathlineto{\pgfqpoint{3.726640in}{1.400933in}}%
\pgfpathlineto{\pgfqpoint{3.723468in}{1.400851in}}%
\pgfpathlineto{\pgfqpoint{3.720296in}{1.400788in}}%
\pgfpathlineto{\pgfqpoint{3.717124in}{1.400940in}}%
\pgfpathlineto{\pgfqpoint{3.713952in}{1.400922in}}%
\pgfpathlineto{\pgfqpoint{3.710780in}{1.400939in}}%
\pgfpathlineto{\pgfqpoint{3.707608in}{1.401041in}}%
\pgfpathlineto{\pgfqpoint{3.704436in}{1.401010in}}%
\pgfpathlineto{\pgfqpoint{3.701264in}{1.400948in}}%
\pgfpathlineto{\pgfqpoint{3.698092in}{1.400694in}}%
\pgfpathlineto{\pgfqpoint{3.694920in}{1.400588in}}%
\pgfpathlineto{\pgfqpoint{3.691748in}{1.400535in}}%
\pgfpathlineto{\pgfqpoint{3.688576in}{1.401184in}}%
\pgfpathlineto{\pgfqpoint{3.685404in}{1.400830in}}%
\pgfpathlineto{\pgfqpoint{3.682232in}{1.400769in}}%
\pgfpathlineto{\pgfqpoint{3.679060in}{1.400788in}}%
\pgfpathlineto{\pgfqpoint{3.675887in}{1.400346in}}%
\pgfpathlineto{\pgfqpoint{3.672715in}{1.400399in}}%
\pgfpathlineto{\pgfqpoint{3.669543in}{1.400577in}}%
\pgfpathlineto{\pgfqpoint{3.666371in}{1.401080in}}%
\pgfpathlineto{\pgfqpoint{3.663199in}{1.401284in}}%
\pgfpathlineto{\pgfqpoint{3.660027in}{1.401360in}}%
\pgfpathlineto{\pgfqpoint{3.656855in}{1.401303in}}%
\pgfpathlineto{\pgfqpoint{3.653683in}{1.400945in}}%
\pgfpathlineto{\pgfqpoint{3.650511in}{1.400622in}}%
\pgfpathlineto{\pgfqpoint{3.647339in}{1.400462in}}%
\pgfpathlineto{\pgfqpoint{3.644167in}{1.400213in}}%
\pgfpathlineto{\pgfqpoint{3.640995in}{1.399516in}}%
\pgfpathlineto{\pgfqpoint{3.637823in}{1.399513in}}%
\pgfpathlineto{\pgfqpoint{3.634651in}{1.399715in}}%
\pgfpathlineto{\pgfqpoint{3.631479in}{1.400074in}}%
\pgfpathlineto{\pgfqpoint{3.628307in}{1.400018in}}%
\pgfpathlineto{\pgfqpoint{3.625135in}{1.399917in}}%
\pgfpathlineto{\pgfqpoint{3.621963in}{1.399775in}}%
\pgfpathlineto{\pgfqpoint{3.618791in}{1.399655in}}%
\pgfpathlineto{\pgfqpoint{3.615619in}{1.399659in}}%
\pgfpathlineto{\pgfqpoint{3.612447in}{1.399586in}}%
\pgfpathlineto{\pgfqpoint{3.609275in}{1.399640in}}%
\pgfpathlineto{\pgfqpoint{3.606103in}{1.399318in}}%
\pgfpathlineto{\pgfqpoint{3.602931in}{1.399106in}}%
\pgfpathlineto{\pgfqpoint{3.599759in}{1.399184in}}%
\pgfpathlineto{\pgfqpoint{3.596586in}{1.399000in}}%
\pgfpathlineto{\pgfqpoint{3.593414in}{1.399042in}}%
\pgfpathlineto{\pgfqpoint{3.590242in}{1.398453in}}%
\pgfpathlineto{\pgfqpoint{3.587070in}{1.398377in}}%
\pgfpathlineto{\pgfqpoint{3.583898in}{1.398257in}}%
\pgfpathlineto{\pgfqpoint{3.580726in}{1.398355in}}%
\pgfpathlineto{\pgfqpoint{3.577554in}{1.398395in}}%
\pgfpathlineto{\pgfqpoint{3.574382in}{1.398495in}}%
\pgfpathlineto{\pgfqpoint{3.571210in}{1.398594in}}%
\pgfpathlineto{\pgfqpoint{3.568038in}{1.398646in}}%
\pgfpathlineto{\pgfqpoint{3.564866in}{1.398119in}}%
\pgfpathlineto{\pgfqpoint{3.561694in}{1.398379in}}%
\pgfpathlineto{\pgfqpoint{3.558522in}{1.398608in}}%
\pgfpathlineto{\pgfqpoint{3.555350in}{1.398549in}}%
\pgfpathlineto{\pgfqpoint{3.552178in}{1.398456in}}%
\pgfpathlineto{\pgfqpoint{3.549006in}{1.398829in}}%
\pgfpathlineto{\pgfqpoint{3.545834in}{1.398902in}}%
\pgfpathlineto{\pgfqpoint{3.542662in}{1.398987in}}%
\pgfpathlineto{\pgfqpoint{3.539490in}{1.398548in}}%
\pgfpathlineto{\pgfqpoint{3.536318in}{1.398263in}}%
\pgfpathlineto{\pgfqpoint{3.533146in}{1.398401in}}%
\pgfpathlineto{\pgfqpoint{3.529974in}{1.398263in}}%
\pgfpathlineto{\pgfqpoint{3.526802in}{1.398450in}}%
\pgfpathlineto{\pgfqpoint{3.523630in}{1.398700in}}%
\pgfpathlineto{\pgfqpoint{3.520457in}{1.398942in}}%
\pgfpathlineto{\pgfqpoint{3.517285in}{1.399120in}}%
\pgfpathlineto{\pgfqpoint{3.514113in}{1.398847in}}%
\pgfpathlineto{\pgfqpoint{3.510941in}{1.398719in}}%
\pgfpathlineto{\pgfqpoint{3.507769in}{1.398421in}}%
\pgfpathlineto{\pgfqpoint{3.504597in}{1.398606in}}%
\pgfpathlineto{\pgfqpoint{3.501425in}{1.398359in}}%
\pgfpathlineto{\pgfqpoint{3.498253in}{1.398033in}}%
\pgfpathlineto{\pgfqpoint{3.495081in}{1.398155in}}%
\pgfpathlineto{\pgfqpoint{3.491909in}{1.398313in}}%
\pgfpathlineto{\pgfqpoint{3.488737in}{1.398131in}}%
\pgfpathlineto{\pgfqpoint{3.485565in}{1.398208in}}%
\pgfpathlineto{\pgfqpoint{3.482393in}{1.398166in}}%
\pgfpathlineto{\pgfqpoint{3.479221in}{1.397912in}}%
\pgfpathlineto{\pgfqpoint{3.476049in}{1.397686in}}%
\pgfpathlineto{\pgfqpoint{3.472877in}{1.397208in}}%
\pgfpathlineto{\pgfqpoint{3.469705in}{1.397330in}}%
\pgfpathlineto{\pgfqpoint{3.466533in}{1.397349in}}%
\pgfpathlineto{\pgfqpoint{3.463361in}{1.397543in}}%
\pgfpathlineto{\pgfqpoint{3.460189in}{1.397681in}}%
\pgfpathlineto{\pgfqpoint{3.457017in}{1.397431in}}%
\pgfpathlineto{\pgfqpoint{3.453845in}{1.397578in}}%
\pgfpathlineto{\pgfqpoint{3.450673in}{1.397543in}}%
\pgfpathlineto{\pgfqpoint{3.447501in}{1.397406in}}%
\pgfpathlineto{\pgfqpoint{3.444329in}{1.397405in}}%
\pgfpathlineto{\pgfqpoint{3.441156in}{1.397543in}}%
\pgfpathlineto{\pgfqpoint{3.437984in}{1.397628in}}%
\pgfpathlineto{\pgfqpoint{3.434812in}{1.397637in}}%
\pgfpathlineto{\pgfqpoint{3.431640in}{1.397575in}}%
\pgfpathlineto{\pgfqpoint{3.428468in}{1.396987in}}%
\pgfpathlineto{\pgfqpoint{3.425296in}{1.396192in}}%
\pgfpathlineto{\pgfqpoint{3.422124in}{1.395625in}}%
\pgfpathlineto{\pgfqpoint{3.418952in}{1.395926in}}%
\pgfpathlineto{\pgfqpoint{3.415780in}{1.395401in}}%
\pgfpathlineto{\pgfqpoint{3.412608in}{1.395066in}}%
\pgfpathlineto{\pgfqpoint{3.409436in}{1.394967in}}%
\pgfpathlineto{\pgfqpoint{3.406264in}{1.394796in}}%
\pgfpathlineto{\pgfqpoint{3.403092in}{1.394739in}}%
\pgfpathlineto{\pgfqpoint{3.399920in}{1.393864in}}%
\pgfpathlineto{\pgfqpoint{3.396748in}{1.393630in}}%
\pgfpathlineto{\pgfqpoint{3.393576in}{1.393423in}}%
\pgfpathlineto{\pgfqpoint{3.390404in}{1.393124in}}%
\pgfpathlineto{\pgfqpoint{3.387232in}{1.392712in}}%
\pgfpathlineto{\pgfqpoint{3.384060in}{1.392562in}}%
\pgfpathlineto{\pgfqpoint{3.380888in}{1.392590in}}%
\pgfpathlineto{\pgfqpoint{3.377716in}{1.392280in}}%
\pgfpathlineto{\pgfqpoint{3.374544in}{1.392088in}}%
\pgfpathlineto{\pgfqpoint{3.371372in}{1.390884in}}%
\pgfpathlineto{\pgfqpoint{3.368200in}{1.390954in}}%
\pgfpathlineto{\pgfqpoint{3.365028in}{1.390629in}}%
\pgfpathlineto{\pgfqpoint{3.361855in}{1.390411in}}%
\pgfpathlineto{\pgfqpoint{3.358683in}{1.390435in}}%
\pgfpathlineto{\pgfqpoint{3.355511in}{1.390433in}}%
\pgfpathlineto{\pgfqpoint{3.352339in}{1.390602in}}%
\pgfpathlineto{\pgfqpoint{3.349167in}{1.389884in}}%
\pgfpathlineto{\pgfqpoint{3.345995in}{1.390246in}}%
\pgfpathlineto{\pgfqpoint{3.342823in}{1.390220in}}%
\pgfpathlineto{\pgfqpoint{3.339651in}{1.390188in}}%
\pgfpathlineto{\pgfqpoint{3.336479in}{1.390382in}}%
\pgfpathlineto{\pgfqpoint{3.333307in}{1.390071in}}%
\pgfpathlineto{\pgfqpoint{3.330135in}{1.390015in}}%
\pgfpathlineto{\pgfqpoint{3.326963in}{1.389813in}}%
\pgfpathlineto{\pgfqpoint{3.323791in}{1.389650in}}%
\pgfpathlineto{\pgfqpoint{3.320619in}{1.389623in}}%
\pgfpathlineto{\pgfqpoint{3.317447in}{1.389684in}}%
\pgfpathlineto{\pgfqpoint{3.314275in}{1.389923in}}%
\pgfpathlineto{\pgfqpoint{3.311103in}{1.390189in}}%
\pgfpathlineto{\pgfqpoint{3.307931in}{1.390534in}}%
\pgfpathlineto{\pgfqpoint{3.304759in}{1.390437in}}%
\pgfpathlineto{\pgfqpoint{3.301587in}{1.390300in}}%
\pgfpathlineto{\pgfqpoint{3.298415in}{1.390170in}}%
\pgfpathlineto{\pgfqpoint{3.295243in}{1.390042in}}%
\pgfpathlineto{\pgfqpoint{3.292071in}{1.389837in}}%
\pgfpathlineto{\pgfqpoint{3.288899in}{1.388968in}}%
\pgfpathlineto{\pgfqpoint{3.285726in}{1.388820in}}%
\pgfpathlineto{\pgfqpoint{3.282554in}{1.388771in}}%
\pgfpathlineto{\pgfqpoint{3.279382in}{1.388270in}}%
\pgfpathlineto{\pgfqpoint{3.276210in}{1.388151in}}%
\pgfpathlineto{\pgfqpoint{3.273038in}{1.388006in}}%
\pgfpathlineto{\pgfqpoint{3.269866in}{1.387904in}}%
\pgfpathlineto{\pgfqpoint{3.266694in}{1.388139in}}%
\pgfpathlineto{\pgfqpoint{3.263522in}{1.386734in}}%
\pgfpathlineto{\pgfqpoint{3.260350in}{1.386877in}}%
\pgfpathlineto{\pgfqpoint{3.257178in}{1.386722in}}%
\pgfpathlineto{\pgfqpoint{3.254006in}{1.386775in}}%
\pgfpathlineto{\pgfqpoint{3.250834in}{1.387288in}}%
\pgfpathlineto{\pgfqpoint{3.247662in}{1.387333in}}%
\pgfpathlineto{\pgfqpoint{3.244490in}{1.387329in}}%
\pgfpathlineto{\pgfqpoint{3.241318in}{1.388032in}}%
\pgfpathlineto{\pgfqpoint{3.238146in}{1.388066in}}%
\pgfpathlineto{\pgfqpoint{3.234974in}{1.387707in}}%
\pgfpathlineto{\pgfqpoint{3.231802in}{1.387803in}}%
\pgfpathlineto{\pgfqpoint{3.228630in}{1.386503in}}%
\pgfpathlineto{\pgfqpoint{3.225458in}{1.385836in}}%
\pgfpathlineto{\pgfqpoint{3.222286in}{1.385777in}}%
\pgfpathlineto{\pgfqpoint{3.219114in}{1.386055in}}%
\pgfpathlineto{\pgfqpoint{3.215942in}{1.386386in}}%
\pgfpathlineto{\pgfqpoint{3.212770in}{1.386312in}}%
\pgfpathlineto{\pgfqpoint{3.209598in}{1.386132in}}%
\pgfpathlineto{\pgfqpoint{3.206425in}{1.384757in}}%
\pgfpathlineto{\pgfqpoint{3.203253in}{1.384758in}}%
\pgfpathlineto{\pgfqpoint{3.200081in}{1.384156in}}%
\pgfpathlineto{\pgfqpoint{3.196909in}{1.383292in}}%
\pgfpathlineto{\pgfqpoint{3.193737in}{1.383516in}}%
\pgfpathlineto{\pgfqpoint{3.190565in}{1.383298in}}%
\pgfpathlineto{\pgfqpoint{3.187393in}{1.383699in}}%
\pgfpathlineto{\pgfqpoint{3.184221in}{1.382899in}}%
\pgfpathlineto{\pgfqpoint{3.181049in}{1.383017in}}%
\pgfpathlineto{\pgfqpoint{3.177877in}{1.382314in}}%
\pgfpathlineto{\pgfqpoint{3.174705in}{1.381557in}}%
\pgfpathlineto{\pgfqpoint{3.171533in}{1.381344in}}%
\pgfpathlineto{\pgfqpoint{3.168361in}{1.381246in}}%
\pgfpathlineto{\pgfqpoint{3.165189in}{1.381384in}}%
\pgfpathlineto{\pgfqpoint{3.162017in}{1.381485in}}%
\pgfpathlineto{\pgfqpoint{3.158845in}{1.381628in}}%
\pgfpathlineto{\pgfqpoint{3.155673in}{1.381732in}}%
\pgfpathlineto{\pgfqpoint{3.152501in}{1.381756in}}%
\pgfpathlineto{\pgfqpoint{3.149329in}{1.380272in}}%
\pgfpathlineto{\pgfqpoint{3.146157in}{1.380267in}}%
\pgfpathlineto{\pgfqpoint{3.142985in}{1.379818in}}%
\pgfpathlineto{\pgfqpoint{3.139813in}{1.379821in}}%
\pgfpathlineto{\pgfqpoint{3.136641in}{1.379575in}}%
\pgfpathlineto{\pgfqpoint{3.133469in}{1.379572in}}%
\pgfpathlineto{\pgfqpoint{3.130297in}{1.379640in}}%
\pgfpathlineto{\pgfqpoint{3.127124in}{1.379665in}}%
\pgfpathlineto{\pgfqpoint{3.123952in}{1.379620in}}%
\pgfpathlineto{\pgfqpoint{3.120780in}{1.379407in}}%
\pgfpathlineto{\pgfqpoint{3.117608in}{1.379484in}}%
\pgfpathlineto{\pgfqpoint{3.114436in}{1.379482in}}%
\pgfpathlineto{\pgfqpoint{3.111264in}{1.379521in}}%
\pgfpathlineto{\pgfqpoint{3.108092in}{1.379429in}}%
\pgfpathlineto{\pgfqpoint{3.104920in}{1.379758in}}%
\pgfpathlineto{\pgfqpoint{3.101748in}{1.379774in}}%
\pgfpathlineto{\pgfqpoint{3.098576in}{1.379614in}}%
\pgfpathlineto{\pgfqpoint{3.095404in}{1.379690in}}%
\pgfpathlineto{\pgfqpoint{3.092232in}{1.379961in}}%
\pgfpathlineto{\pgfqpoint{3.089060in}{1.379962in}}%
\pgfpathlineto{\pgfqpoint{3.085888in}{1.379909in}}%
\pgfpathlineto{\pgfqpoint{3.082716in}{1.379830in}}%
\pgfpathlineto{\pgfqpoint{3.079544in}{1.380041in}}%
\pgfpathlineto{\pgfqpoint{3.076372in}{1.380081in}}%
\pgfpathlineto{\pgfqpoint{3.073200in}{1.380381in}}%
\pgfpathlineto{\pgfqpoint{3.070028in}{1.380450in}}%
\pgfpathlineto{\pgfqpoint{3.066856in}{1.380379in}}%
\pgfpathlineto{\pgfqpoint{3.063684in}{1.380465in}}%
\pgfpathlineto{\pgfqpoint{3.060512in}{1.380565in}}%
\pgfpathlineto{\pgfqpoint{3.057340in}{1.380498in}}%
\pgfpathlineto{\pgfqpoint{3.054168in}{1.380427in}}%
\pgfpathlineto{\pgfqpoint{3.050995in}{1.380584in}}%
\pgfpathlineto{\pgfqpoint{3.047823in}{1.380557in}}%
\pgfpathlineto{\pgfqpoint{3.044651in}{1.380436in}}%
\pgfpathlineto{\pgfqpoint{3.041479in}{1.380456in}}%
\pgfpathlineto{\pgfqpoint{3.038307in}{1.380460in}}%
\pgfpathlineto{\pgfqpoint{3.035135in}{1.380345in}}%
\pgfpathlineto{\pgfqpoint{3.031963in}{1.380354in}}%
\pgfpathlineto{\pgfqpoint{3.028791in}{1.380323in}}%
\pgfpathlineto{\pgfqpoint{3.025619in}{1.380406in}}%
\pgfpathlineto{\pgfqpoint{3.022447in}{1.380486in}}%
\pgfpathlineto{\pgfqpoint{3.019275in}{1.380655in}}%
\pgfpathlineto{\pgfqpoint{3.016103in}{1.380528in}}%
\pgfpathlineto{\pgfqpoint{3.012931in}{1.380560in}}%
\pgfpathlineto{\pgfqpoint{3.009759in}{1.380466in}}%
\pgfpathlineto{\pgfqpoint{3.006587in}{1.380422in}}%
\pgfpathlineto{\pgfqpoint{3.003415in}{1.380295in}}%
\pgfpathlineto{\pgfqpoint{3.000243in}{1.380286in}}%
\pgfpathlineto{\pgfqpoint{2.997071in}{1.380200in}}%
\pgfpathlineto{\pgfqpoint{2.993899in}{1.380142in}}%
\pgfpathlineto{\pgfqpoint{2.990727in}{1.380169in}}%
\pgfpathlineto{\pgfqpoint{2.987555in}{1.380162in}}%
\pgfpathlineto{\pgfqpoint{2.984383in}{1.380086in}}%
\pgfpathlineto{\pgfqpoint{2.981211in}{1.380002in}}%
\pgfpathlineto{\pgfqpoint{2.978039in}{1.380077in}}%
\pgfpathlineto{\pgfqpoint{2.974867in}{1.380241in}}%
\pgfpathlineto{\pgfqpoint{2.971694in}{1.380409in}}%
\pgfpathlineto{\pgfqpoint{2.968522in}{1.380162in}}%
\pgfpathlineto{\pgfqpoint{2.965350in}{1.380021in}}%
\pgfpathlineto{\pgfqpoint{2.962178in}{1.380135in}}%
\pgfpathlineto{\pgfqpoint{2.959006in}{1.380198in}}%
\pgfpathlineto{\pgfqpoint{2.955834in}{1.378710in}}%
\pgfpathlineto{\pgfqpoint{2.952662in}{1.378447in}}%
\pgfpathlineto{\pgfqpoint{2.949490in}{1.378282in}}%
\pgfpathlineto{\pgfqpoint{2.946318in}{1.378299in}}%
\pgfpathlineto{\pgfqpoint{2.943146in}{1.378295in}}%
\pgfpathlineto{\pgfqpoint{2.939974in}{1.378079in}}%
\pgfpathlineto{\pgfqpoint{2.936802in}{1.378041in}}%
\pgfpathlineto{\pgfqpoint{2.933630in}{1.378095in}}%
\pgfpathlineto{\pgfqpoint{2.930458in}{1.377998in}}%
\pgfpathlineto{\pgfqpoint{2.927286in}{1.377785in}}%
\pgfpathlineto{\pgfqpoint{2.924114in}{1.377442in}}%
\pgfpathlineto{\pgfqpoint{2.920942in}{1.377519in}}%
\pgfpathlineto{\pgfqpoint{2.917770in}{1.377614in}}%
\pgfpathlineto{\pgfqpoint{2.914598in}{1.377519in}}%
\pgfpathlineto{\pgfqpoint{2.911426in}{1.377443in}}%
\pgfpathlineto{\pgfqpoint{2.908254in}{1.377463in}}%
\pgfpathlineto{\pgfqpoint{2.905082in}{1.377406in}}%
\pgfpathlineto{\pgfqpoint{2.901910in}{1.377253in}}%
\pgfpathlineto{\pgfqpoint{2.898738in}{1.377232in}}%
\pgfpathlineto{\pgfqpoint{2.895565in}{1.377323in}}%
\pgfpathlineto{\pgfqpoint{2.892393in}{1.377332in}}%
\pgfpathlineto{\pgfqpoint{2.889221in}{1.377266in}}%
\pgfpathlineto{\pgfqpoint{2.886049in}{1.377174in}}%
\pgfpathlineto{\pgfqpoint{2.882877in}{1.377241in}}%
\pgfpathlineto{\pgfqpoint{2.879705in}{1.377287in}}%
\pgfpathlineto{\pgfqpoint{2.876533in}{1.377395in}}%
\pgfpathlineto{\pgfqpoint{2.873361in}{1.377451in}}%
\pgfpathlineto{\pgfqpoint{2.870189in}{1.377533in}}%
\pgfpathlineto{\pgfqpoint{2.867017in}{1.377762in}}%
\pgfpathlineto{\pgfqpoint{2.863845in}{1.377873in}}%
\pgfpathlineto{\pgfqpoint{2.860673in}{1.377704in}}%
\pgfpathlineto{\pgfqpoint{2.857501in}{1.377657in}}%
\pgfpathlineto{\pgfqpoint{2.854329in}{1.377650in}}%
\pgfpathlineto{\pgfqpoint{2.851157in}{1.377470in}}%
\pgfpathlineto{\pgfqpoint{2.847985in}{1.377456in}}%
\pgfpathlineto{\pgfqpoint{2.844813in}{1.377489in}}%
\pgfpathlineto{\pgfqpoint{2.841641in}{1.377257in}}%
\pgfpathlineto{\pgfqpoint{2.838469in}{1.377331in}}%
\pgfpathlineto{\pgfqpoint{2.835297in}{1.377321in}}%
\pgfpathlineto{\pgfqpoint{2.832125in}{1.377040in}}%
\pgfpathlineto{\pgfqpoint{2.828953in}{1.376952in}}%
\pgfpathlineto{\pgfqpoint{2.825781in}{1.376948in}}%
\pgfpathlineto{\pgfqpoint{2.822609in}{1.376924in}}%
\pgfpathlineto{\pgfqpoint{2.819437in}{1.377005in}}%
\pgfpathlineto{\pgfqpoint{2.816264in}{1.377007in}}%
\pgfpathlineto{\pgfqpoint{2.813092in}{1.376968in}}%
\pgfpathlineto{\pgfqpoint{2.809920in}{1.376898in}}%
\pgfpathlineto{\pgfqpoint{2.806748in}{1.376908in}}%
\pgfpathlineto{\pgfqpoint{2.803576in}{1.376960in}}%
\pgfpathlineto{\pgfqpoint{2.800404in}{1.376974in}}%
\pgfpathlineto{\pgfqpoint{2.797232in}{1.376659in}}%
\pgfpathlineto{\pgfqpoint{2.794060in}{1.376444in}}%
\pgfpathlineto{\pgfqpoint{2.790888in}{1.376391in}}%
\pgfpathlineto{\pgfqpoint{2.787716in}{1.376589in}}%
\pgfpathlineto{\pgfqpoint{2.784544in}{1.376474in}}%
\pgfpathlineto{\pgfqpoint{2.781372in}{1.376398in}}%
\pgfpathlineto{\pgfqpoint{2.778200in}{1.376391in}}%
\pgfpathlineto{\pgfqpoint{2.775028in}{1.376493in}}%
\pgfpathlineto{\pgfqpoint{2.771856in}{1.376324in}}%
\pgfpathlineto{\pgfqpoint{2.768684in}{1.376292in}}%
\pgfpathlineto{\pgfqpoint{2.765512in}{1.376305in}}%
\pgfpathlineto{\pgfqpoint{2.762340in}{1.376505in}}%
\pgfpathlineto{\pgfqpoint{2.759168in}{1.376611in}}%
\pgfpathlineto{\pgfqpoint{2.755996in}{1.376676in}}%
\pgfpathlineto{\pgfqpoint{2.752824in}{1.376729in}}%
\pgfpathlineto{\pgfqpoint{2.749652in}{1.376644in}}%
\pgfpathlineto{\pgfqpoint{2.746480in}{1.376724in}}%
\pgfpathlineto{\pgfqpoint{2.743308in}{1.376681in}}%
\pgfpathlineto{\pgfqpoint{2.740136in}{1.376786in}}%
\pgfpathlineto{\pgfqpoint{2.736963in}{1.376779in}}%
\pgfpathlineto{\pgfqpoint{2.733791in}{1.376953in}}%
\pgfpathlineto{\pgfqpoint{2.730619in}{1.377097in}}%
\pgfpathlineto{\pgfqpoint{2.727447in}{1.376955in}}%
\pgfpathlineto{\pgfqpoint{2.724275in}{1.376895in}}%
\pgfpathlineto{\pgfqpoint{2.721103in}{1.376809in}}%
\pgfpathlineto{\pgfqpoint{2.717931in}{1.376859in}}%
\pgfpathlineto{\pgfqpoint{2.714759in}{1.376897in}}%
\pgfpathlineto{\pgfqpoint{2.711587in}{1.376464in}}%
\pgfpathlineto{\pgfqpoint{2.708415in}{1.376422in}}%
\pgfpathlineto{\pgfqpoint{2.705243in}{1.376434in}}%
\pgfpathlineto{\pgfqpoint{2.702071in}{1.376493in}}%
\pgfpathlineto{\pgfqpoint{2.698899in}{1.376669in}}%
\pgfpathlineto{\pgfqpoint{2.695727in}{1.376848in}}%
\pgfpathlineto{\pgfqpoint{2.692555in}{1.376830in}}%
\pgfpathlineto{\pgfqpoint{2.689383in}{1.376686in}}%
\pgfpathlineto{\pgfqpoint{2.686211in}{1.376883in}}%
\pgfpathlineto{\pgfqpoint{2.683039in}{1.376911in}}%
\pgfpathlineto{\pgfqpoint{2.679867in}{1.376855in}}%
\pgfpathlineto{\pgfqpoint{2.676695in}{1.376834in}}%
\pgfpathlineto{\pgfqpoint{2.673523in}{1.376634in}}%
\pgfpathlineto{\pgfqpoint{2.670351in}{1.376748in}}%
\pgfpathlineto{\pgfqpoint{2.667179in}{1.376708in}}%
\pgfpathlineto{\pgfqpoint{2.664007in}{1.376556in}}%
\pgfpathlineto{\pgfqpoint{2.660834in}{1.376381in}}%
\pgfpathlineto{\pgfqpoint{2.657662in}{1.376419in}}%
\pgfpathlineto{\pgfqpoint{2.654490in}{1.376421in}}%
\pgfpathlineto{\pgfqpoint{2.651318in}{1.376557in}}%
\pgfpathlineto{\pgfqpoint{2.648146in}{1.376299in}}%
\pgfpathlineto{\pgfqpoint{2.644974in}{1.376316in}}%
\pgfpathlineto{\pgfqpoint{2.641802in}{1.376229in}}%
\pgfpathlineto{\pgfqpoint{2.638630in}{1.376328in}}%
\pgfpathlineto{\pgfqpoint{2.635458in}{1.376265in}}%
\pgfpathlineto{\pgfqpoint{2.632286in}{1.376418in}}%
\pgfpathlineto{\pgfqpoint{2.629114in}{1.376350in}}%
\pgfpathlineto{\pgfqpoint{2.625942in}{1.376363in}}%
\pgfpathlineto{\pgfqpoint{2.622770in}{1.376289in}}%
\pgfpathlineto{\pgfqpoint{2.619598in}{1.376297in}}%
\pgfpathlineto{\pgfqpoint{2.616426in}{1.376389in}}%
\pgfpathlineto{\pgfqpoint{2.613254in}{1.376287in}}%
\pgfpathlineto{\pgfqpoint{2.610082in}{1.376243in}}%
\pgfpathlineto{\pgfqpoint{2.606910in}{1.376388in}}%
\pgfpathlineto{\pgfqpoint{2.603738in}{1.376294in}}%
\pgfpathlineto{\pgfqpoint{2.600566in}{1.376235in}}%
\pgfpathlineto{\pgfqpoint{2.597394in}{1.376184in}}%
\pgfpathlineto{\pgfqpoint{2.594222in}{1.376340in}}%
\pgfpathlineto{\pgfqpoint{2.591050in}{1.376336in}}%
\pgfpathlineto{\pgfqpoint{2.587878in}{1.376543in}}%
\pgfpathlineto{\pgfqpoint{2.584706in}{1.376529in}}%
\pgfpathlineto{\pgfqpoint{2.581533in}{1.376648in}}%
\pgfpathlineto{\pgfqpoint{2.578361in}{1.376591in}}%
\pgfpathlineto{\pgfqpoint{2.575189in}{1.376463in}}%
\pgfpathlineto{\pgfqpoint{2.572017in}{1.376500in}}%
\pgfpathlineto{\pgfqpoint{2.568845in}{1.376363in}}%
\pgfpathlineto{\pgfqpoint{2.565673in}{1.376078in}}%
\pgfpathlineto{\pgfqpoint{2.562501in}{1.375972in}}%
\pgfpathlineto{\pgfqpoint{2.559329in}{1.375779in}}%
\pgfpathlineto{\pgfqpoint{2.556157in}{1.375906in}}%
\pgfpathlineto{\pgfqpoint{2.552985in}{1.376050in}}%
\pgfpathlineto{\pgfqpoint{2.549813in}{1.376090in}}%
\pgfpathlineto{\pgfqpoint{2.546641in}{1.376058in}}%
\pgfpathlineto{\pgfqpoint{2.543469in}{1.376027in}}%
\pgfpathlineto{\pgfqpoint{2.540297in}{1.375955in}}%
\pgfpathlineto{\pgfqpoint{2.537125in}{1.375954in}}%
\pgfpathlineto{\pgfqpoint{2.533953in}{1.375915in}}%
\pgfpathlineto{\pgfqpoint{2.530781in}{1.375869in}}%
\pgfpathlineto{\pgfqpoint{2.527609in}{1.375769in}}%
\pgfpathlineto{\pgfqpoint{2.524437in}{1.375772in}}%
\pgfpathlineto{\pgfqpoint{2.521265in}{1.375797in}}%
\pgfpathlineto{\pgfqpoint{2.518093in}{1.375799in}}%
\pgfpathlineto{\pgfqpoint{2.514921in}{1.375648in}}%
\pgfpathlineto{\pgfqpoint{2.511749in}{1.375665in}}%
\pgfpathlineto{\pgfqpoint{2.508577in}{1.375662in}}%
\pgfpathlineto{\pgfqpoint{2.505405in}{1.375522in}}%
\pgfpathlineto{\pgfqpoint{2.502232in}{1.375604in}}%
\pgfpathlineto{\pgfqpoint{2.499060in}{1.375495in}}%
\pgfpathlineto{\pgfqpoint{2.495888in}{1.375407in}}%
\pgfpathlineto{\pgfqpoint{2.492716in}{1.375421in}}%
\pgfpathlineto{\pgfqpoint{2.489544in}{1.375377in}}%
\pgfpathlineto{\pgfqpoint{2.486372in}{1.375115in}}%
\pgfpathlineto{\pgfqpoint{2.483200in}{1.374939in}}%
\pgfpathlineto{\pgfqpoint{2.480028in}{1.374975in}}%
\pgfpathlineto{\pgfqpoint{2.476856in}{1.374904in}}%
\pgfpathlineto{\pgfqpoint{2.473684in}{1.375010in}}%
\pgfpathlineto{\pgfqpoint{2.470512in}{1.374724in}}%
\pgfpathlineto{\pgfqpoint{2.467340in}{1.374824in}}%
\pgfpathlineto{\pgfqpoint{2.464168in}{1.374788in}}%
\pgfpathlineto{\pgfqpoint{2.460996in}{1.374715in}}%
\pgfpathlineto{\pgfqpoint{2.457824in}{1.374581in}}%
\pgfpathlineto{\pgfqpoint{2.454652in}{1.374700in}}%
\pgfpathlineto{\pgfqpoint{2.451480in}{1.374793in}}%
\pgfpathlineto{\pgfqpoint{2.448308in}{1.374696in}}%
\pgfpathlineto{\pgfqpoint{2.445136in}{1.374615in}}%
\pgfpathlineto{\pgfqpoint{2.441964in}{1.374727in}}%
\pgfpathlineto{\pgfqpoint{2.438792in}{1.374806in}}%
\pgfpathlineto{\pgfqpoint{2.435620in}{1.374812in}}%
\pgfpathlineto{\pgfqpoint{2.432448in}{1.374765in}}%
\pgfpathlineto{\pgfqpoint{2.429276in}{1.374834in}}%
\pgfpathlineto{\pgfqpoint{2.426103in}{1.375089in}}%
\pgfpathlineto{\pgfqpoint{2.422931in}{1.374946in}}%
\pgfpathlineto{\pgfqpoint{2.419759in}{1.374839in}}%
\pgfpathlineto{\pgfqpoint{2.416587in}{1.374891in}}%
\pgfpathlineto{\pgfqpoint{2.413415in}{1.374932in}}%
\pgfpathlineto{\pgfqpoint{2.410243in}{1.374940in}}%
\pgfpathlineto{\pgfqpoint{2.407071in}{1.374889in}}%
\pgfpathlineto{\pgfqpoint{2.403899in}{1.374878in}}%
\pgfpathlineto{\pgfqpoint{2.400727in}{1.375016in}}%
\pgfpathlineto{\pgfqpoint{2.397555in}{1.374855in}}%
\pgfpathlineto{\pgfqpoint{2.394383in}{1.374778in}}%
\pgfpathlineto{\pgfqpoint{2.391211in}{1.374922in}}%
\pgfpathlineto{\pgfqpoint{2.388039in}{1.374657in}}%
\pgfpathlineto{\pgfqpoint{2.384867in}{1.374642in}}%
\pgfpathlineto{\pgfqpoint{2.381695in}{1.374689in}}%
\pgfpathlineto{\pgfqpoint{2.378523in}{1.374731in}}%
\pgfpathlineto{\pgfqpoint{2.375351in}{1.374662in}}%
\pgfpathlineto{\pgfqpoint{2.372179in}{1.374591in}}%
\pgfpathlineto{\pgfqpoint{2.369007in}{1.374555in}}%
\pgfpathlineto{\pgfqpoint{2.365835in}{1.374530in}}%
\pgfpathlineto{\pgfqpoint{2.362663in}{1.374752in}}%
\pgfpathlineto{\pgfqpoint{2.359491in}{1.374942in}}%
\pgfpathlineto{\pgfqpoint{2.356319in}{1.374847in}}%
\pgfpathlineto{\pgfqpoint{2.353147in}{1.374856in}}%
\pgfpathlineto{\pgfqpoint{2.349975in}{1.374789in}}%
\pgfpathlineto{\pgfqpoint{2.346802in}{1.374886in}}%
\pgfpathlineto{\pgfqpoint{2.343630in}{1.375003in}}%
\pgfpathlineto{\pgfqpoint{2.340458in}{1.375108in}}%
\pgfpathlineto{\pgfqpoint{2.337286in}{1.375203in}}%
\pgfpathlineto{\pgfqpoint{2.334114in}{1.375068in}}%
\pgfpathlineto{\pgfqpoint{2.330942in}{1.375248in}}%
\pgfpathlineto{\pgfqpoint{2.327770in}{1.374990in}}%
\pgfpathlineto{\pgfqpoint{2.324598in}{1.375179in}}%
\pgfpathlineto{\pgfqpoint{2.321426in}{1.374776in}}%
\pgfpathlineto{\pgfqpoint{2.318254in}{1.374893in}}%
\pgfpathlineto{\pgfqpoint{2.315082in}{1.375000in}}%
\pgfpathlineto{\pgfqpoint{2.311910in}{1.375099in}}%
\pgfpathlineto{\pgfqpoint{2.308738in}{1.375245in}}%
\pgfpathlineto{\pgfqpoint{2.305566in}{1.375236in}}%
\pgfpathlineto{\pgfqpoint{2.302394in}{1.375307in}}%
\pgfpathlineto{\pgfqpoint{2.299222in}{1.375081in}}%
\pgfpathlineto{\pgfqpoint{2.296050in}{1.374911in}}%
\pgfpathlineto{\pgfqpoint{2.292878in}{1.374568in}}%
\pgfpathlineto{\pgfqpoint{2.289706in}{1.374489in}}%
\pgfpathlineto{\pgfqpoint{2.286534in}{1.374513in}}%
\pgfpathlineto{\pgfqpoint{2.283362in}{1.374261in}}%
\pgfpathlineto{\pgfqpoint{2.280190in}{1.374097in}}%
\pgfpathlineto{\pgfqpoint{2.277018in}{1.374207in}}%
\pgfpathlineto{\pgfqpoint{2.273846in}{1.374217in}}%
\pgfpathlineto{\pgfqpoint{2.270674in}{1.374367in}}%
\pgfpathlineto{\pgfqpoint{2.267501in}{1.374344in}}%
\pgfpathlineto{\pgfqpoint{2.264329in}{1.374330in}}%
\pgfpathlineto{\pgfqpoint{2.261157in}{1.374074in}}%
\pgfpathlineto{\pgfqpoint{2.257985in}{1.373933in}}%
\pgfpathlineto{\pgfqpoint{2.254813in}{1.374012in}}%
\pgfpathlineto{\pgfqpoint{2.251641in}{1.374055in}}%
\pgfpathlineto{\pgfqpoint{2.248469in}{1.374266in}}%
\pgfpathlineto{\pgfqpoint{2.245297in}{1.374420in}}%
\pgfpathlineto{\pgfqpoint{2.242125in}{1.374370in}}%
\pgfpathlineto{\pgfqpoint{2.238953in}{1.374363in}}%
\pgfpathlineto{\pgfqpoint{2.235781in}{1.374367in}}%
\pgfpathlineto{\pgfqpoint{2.232609in}{1.374506in}}%
\pgfpathlineto{\pgfqpoint{2.229437in}{1.374491in}}%
\pgfpathlineto{\pgfqpoint{2.226265in}{1.374414in}}%
\pgfpathlineto{\pgfqpoint{2.223093in}{1.374446in}}%
\pgfpathlineto{\pgfqpoint{2.219921in}{1.374532in}}%
\pgfpathlineto{\pgfqpoint{2.216749in}{1.374587in}}%
\pgfpathlineto{\pgfqpoint{2.213577in}{1.374536in}}%
\pgfpathlineto{\pgfqpoint{2.210405in}{1.374384in}}%
\pgfpathlineto{\pgfqpoint{2.207233in}{1.374298in}}%
\pgfpathlineto{\pgfqpoint{2.204061in}{1.374320in}}%
\pgfpathlineto{\pgfqpoint{2.200889in}{1.374372in}}%
\pgfpathlineto{\pgfqpoint{2.197717in}{1.374332in}}%
\pgfpathlineto{\pgfqpoint{2.194545in}{1.374622in}}%
\pgfpathlineto{\pgfqpoint{2.191372in}{1.374433in}}%
\pgfpathlineto{\pgfqpoint{2.188200in}{1.374551in}}%
\pgfpathlineto{\pgfqpoint{2.185028in}{1.374651in}}%
\pgfpathlineto{\pgfqpoint{2.181856in}{1.374733in}}%
\pgfpathlineto{\pgfqpoint{2.178684in}{1.374494in}}%
\pgfpathlineto{\pgfqpoint{2.175512in}{1.374245in}}%
\pgfpathlineto{\pgfqpoint{2.172340in}{1.374123in}}%
\pgfpathlineto{\pgfqpoint{2.169168in}{1.374074in}}%
\pgfpathlineto{\pgfqpoint{2.165996in}{1.373677in}}%
\pgfpathlineto{\pgfqpoint{2.162824in}{1.373646in}}%
\pgfpathlineto{\pgfqpoint{2.159652in}{1.373708in}}%
\pgfpathlineto{\pgfqpoint{2.156480in}{1.373876in}}%
\pgfpathlineto{\pgfqpoint{2.153308in}{1.373900in}}%
\pgfpathlineto{\pgfqpoint{2.150136in}{1.373842in}}%
\pgfpathlineto{\pgfqpoint{2.146964in}{1.373864in}}%
\pgfpathlineto{\pgfqpoint{2.143792in}{1.373987in}}%
\pgfpathlineto{\pgfqpoint{2.140620in}{1.373893in}}%
\pgfpathlineto{\pgfqpoint{2.137448in}{1.373910in}}%
\pgfpathlineto{\pgfqpoint{2.134276in}{1.373921in}}%
\pgfpathlineto{\pgfqpoint{2.131104in}{1.373541in}}%
\pgfpathlineto{\pgfqpoint{2.127932in}{1.373579in}}%
\pgfpathlineto{\pgfqpoint{2.124760in}{1.373683in}}%
\pgfpathlineto{\pgfqpoint{2.121588in}{1.373721in}}%
\pgfpathlineto{\pgfqpoint{2.118416in}{1.373735in}}%
\pgfpathlineto{\pgfqpoint{2.115244in}{1.373901in}}%
\pgfpathlineto{\pgfqpoint{2.112071in}{1.373689in}}%
\pgfpathlineto{\pgfqpoint{2.108899in}{1.373722in}}%
\pgfpathlineto{\pgfqpoint{2.105727in}{1.373845in}}%
\pgfpathlineto{\pgfqpoint{2.102555in}{1.373956in}}%
\pgfpathlineto{\pgfqpoint{2.099383in}{1.374127in}}%
\pgfpathlineto{\pgfqpoint{2.096211in}{1.374269in}}%
\pgfpathlineto{\pgfqpoint{2.093039in}{1.374354in}}%
\pgfpathlineto{\pgfqpoint{2.089867in}{1.374324in}}%
\pgfpathlineto{\pgfqpoint{2.086695in}{1.374398in}}%
\pgfpathlineto{\pgfqpoint{2.083523in}{1.374416in}}%
\pgfpathlineto{\pgfqpoint{2.080351in}{1.374606in}}%
\pgfpathlineto{\pgfqpoint{2.077179in}{1.374499in}}%
\pgfpathlineto{\pgfqpoint{2.074007in}{1.374658in}}%
\pgfpathlineto{\pgfqpoint{2.070835in}{1.374643in}}%
\pgfpathlineto{\pgfqpoint{2.067663in}{1.374806in}}%
\pgfpathlineto{\pgfqpoint{2.064491in}{1.374964in}}%
\pgfpathlineto{\pgfqpoint{2.061319in}{1.374830in}}%
\pgfpathlineto{\pgfqpoint{2.058147in}{1.374787in}}%
\pgfpathlineto{\pgfqpoint{2.054975in}{1.374686in}}%
\pgfpathlineto{\pgfqpoint{2.051803in}{1.374789in}}%
\pgfpathlineto{\pgfqpoint{2.048631in}{1.374784in}}%
\pgfpathlineto{\pgfqpoint{2.045459in}{1.374506in}}%
\pgfpathlineto{\pgfqpoint{2.042287in}{1.374618in}}%
\pgfpathlineto{\pgfqpoint{2.039115in}{1.374451in}}%
\pgfpathlineto{\pgfqpoint{2.035943in}{1.372852in}}%
\pgfpathlineto{\pgfqpoint{2.032770in}{1.371928in}}%
\pgfpathlineto{\pgfqpoint{2.029598in}{1.372236in}}%
\pgfpathlineto{\pgfqpoint{2.026426in}{1.372155in}}%
\pgfpathlineto{\pgfqpoint{2.023254in}{1.373210in}}%
\pgfpathlineto{\pgfqpoint{2.020082in}{1.373095in}}%
\pgfpathlineto{\pgfqpoint{2.016910in}{1.373931in}}%
\pgfpathlineto{\pgfqpoint{2.013738in}{1.373576in}}%
\pgfpathlineto{\pgfqpoint{2.010566in}{1.373187in}}%
\pgfpathlineto{\pgfqpoint{2.007394in}{1.371080in}}%
\pgfpathlineto{\pgfqpoint{2.004222in}{1.371659in}}%
\pgfpathlineto{\pgfqpoint{2.001050in}{1.371005in}}%
\pgfpathlineto{\pgfqpoint{1.997878in}{1.371058in}}%
\pgfpathlineto{\pgfqpoint{1.994706in}{1.370875in}}%
\pgfpathlineto{\pgfqpoint{1.991534in}{1.370669in}}%
\pgfpathlineto{\pgfqpoint{1.988362in}{1.370680in}}%
\pgfpathlineto{\pgfqpoint{1.985190in}{1.371117in}}%
\pgfpathlineto{\pgfqpoint{1.982018in}{1.349879in}}%
\pgfpathlineto{\pgfqpoint{1.978846in}{1.329654in}}%
\pgfpathlineto{\pgfqpoint{1.975674in}{1.311541in}}%
\pgfpathlineto{\pgfqpoint{1.972502in}{1.291787in}}%
\pgfpathlineto{\pgfqpoint{1.969330in}{1.272301in}}%
\pgfpathlineto{\pgfqpoint{1.966158in}{1.256558in}}%
\pgfpathlineto{\pgfqpoint{1.962986in}{1.232604in}}%
\pgfpathlineto{\pgfqpoint{1.959814in}{1.212179in}}%
\pgfpathlineto{\pgfqpoint{1.956641in}{1.192896in}}%
\pgfpathlineto{\pgfqpoint{1.953469in}{1.171081in}}%
\pgfpathlineto{\pgfqpoint{1.950297in}{1.145452in}}%
\pgfpathlineto{\pgfqpoint{1.947125in}{1.126818in}}%
\pgfpathlineto{\pgfqpoint{1.943953in}{1.126818in}}%
\pgfpathlineto{\pgfqpoint{1.940781in}{1.127262in}}%
\pgfpathclose%
\pgfusepath{stroke,fill}%
\end{pgfscope}%
\begin{pgfscope}%
\pgfpathrectangle{\pgfqpoint{1.623736in}{1.000625in}}{\pgfqpoint{6.975000in}{3.020000in}} %
\pgfusepath{clip}%
\pgfsetbuttcap%
\pgfsetroundjoin%
\definecolor{currentfill}{rgb}{0.576471,0.470588,0.376471}%
\pgfsetfillcolor{currentfill}%
\pgfsetfillopacity{0.200000}%
\pgfsetlinewidth{0.803000pt}%
\definecolor{currentstroke}{rgb}{0.576471,0.470588,0.376471}%
\pgfsetstrokecolor{currentstroke}%
\pgfsetstrokeopacity{0.200000}%
\pgfsetdash{}{0pt}%
\pgfpathmoveto{\pgfqpoint{1.940781in}{1.129362in}}%
\pgfpathlineto{\pgfqpoint{1.940781in}{1.127380in}}%
\pgfpathlineto{\pgfqpoint{1.943953in}{1.126824in}}%
\pgfpathlineto{\pgfqpoint{1.947125in}{1.126875in}}%
\pgfpathlineto{\pgfqpoint{1.950297in}{1.143521in}}%
\pgfpathlineto{\pgfqpoint{1.953469in}{1.168813in}}%
\pgfpathlineto{\pgfqpoint{1.956641in}{1.191671in}}%
\pgfpathlineto{\pgfqpoint{1.959814in}{1.193123in}}%
\pgfpathlineto{\pgfqpoint{1.962986in}{1.191640in}}%
\pgfpathlineto{\pgfqpoint{1.966158in}{1.192679in}}%
\pgfpathlineto{\pgfqpoint{1.969330in}{1.191443in}}%
\pgfpathlineto{\pgfqpoint{1.972502in}{1.189994in}}%
\pgfpathlineto{\pgfqpoint{1.975674in}{1.190782in}}%
\pgfpathlineto{\pgfqpoint{1.978846in}{1.193208in}}%
\pgfpathlineto{\pgfqpoint{1.982018in}{1.192658in}}%
\pgfpathlineto{\pgfqpoint{1.985190in}{1.213953in}}%
\pgfpathlineto{\pgfqpoint{1.988362in}{1.214355in}}%
\pgfpathlineto{\pgfqpoint{1.991534in}{1.216530in}}%
\pgfpathlineto{\pgfqpoint{1.994706in}{1.216134in}}%
\pgfpathlineto{\pgfqpoint{1.997878in}{1.215634in}}%
\pgfpathlineto{\pgfqpoint{2.001050in}{1.215451in}}%
\pgfpathlineto{\pgfqpoint{2.004222in}{1.215395in}}%
\pgfpathlineto{\pgfqpoint{2.007394in}{1.215059in}}%
\pgfpathlineto{\pgfqpoint{2.010566in}{1.214176in}}%
\pgfpathlineto{\pgfqpoint{2.013738in}{1.213627in}}%
\pgfpathlineto{\pgfqpoint{2.016910in}{1.213211in}}%
\pgfpathlineto{\pgfqpoint{2.020082in}{1.213534in}}%
\pgfpathlineto{\pgfqpoint{2.023254in}{1.213362in}}%
\pgfpathlineto{\pgfqpoint{2.026426in}{1.212782in}}%
\pgfpathlineto{\pgfqpoint{2.029598in}{1.211504in}}%
\pgfpathlineto{\pgfqpoint{2.032770in}{1.211843in}}%
\pgfpathlineto{\pgfqpoint{2.035943in}{1.209925in}}%
\pgfpathlineto{\pgfqpoint{2.039115in}{1.209271in}}%
\pgfpathlineto{\pgfqpoint{2.042287in}{1.207225in}}%
\pgfpathlineto{\pgfqpoint{2.045459in}{1.208312in}}%
\pgfpathlineto{\pgfqpoint{2.048631in}{1.208889in}}%
\pgfpathlineto{\pgfqpoint{2.051803in}{1.208811in}}%
\pgfpathlineto{\pgfqpoint{2.054975in}{1.208788in}}%
\pgfpathlineto{\pgfqpoint{2.058147in}{1.208666in}}%
\pgfpathlineto{\pgfqpoint{2.061319in}{1.208537in}}%
\pgfpathlineto{\pgfqpoint{2.064491in}{1.208491in}}%
\pgfpathlineto{\pgfqpoint{2.067663in}{1.208508in}}%
\pgfpathlineto{\pgfqpoint{2.070835in}{1.208600in}}%
\pgfpathlineto{\pgfqpoint{2.074007in}{1.208580in}}%
\pgfpathlineto{\pgfqpoint{2.077179in}{1.208533in}}%
\pgfpathlineto{\pgfqpoint{2.080351in}{1.208649in}}%
\pgfpathlineto{\pgfqpoint{2.083523in}{1.208694in}}%
\pgfpathlineto{\pgfqpoint{2.086695in}{1.208827in}}%
\pgfpathlineto{\pgfqpoint{2.089867in}{1.208742in}}%
\pgfpathlineto{\pgfqpoint{2.093039in}{1.208815in}}%
\pgfpathlineto{\pgfqpoint{2.096211in}{1.208884in}}%
\pgfpathlineto{\pgfqpoint{2.099383in}{1.208578in}}%
\pgfpathlineto{\pgfqpoint{2.102555in}{1.208619in}}%
\pgfpathlineto{\pgfqpoint{2.105727in}{1.208514in}}%
\pgfpathlineto{\pgfqpoint{2.108899in}{1.208721in}}%
\pgfpathlineto{\pgfqpoint{2.112071in}{1.208701in}}%
\pgfpathlineto{\pgfqpoint{2.115244in}{1.208703in}}%
\pgfpathlineto{\pgfqpoint{2.118416in}{1.208902in}}%
\pgfpathlineto{\pgfqpoint{2.121588in}{1.209003in}}%
\pgfpathlineto{\pgfqpoint{2.124760in}{1.209004in}}%
\pgfpathlineto{\pgfqpoint{2.127932in}{1.209000in}}%
\pgfpathlineto{\pgfqpoint{2.131104in}{1.208833in}}%
\pgfpathlineto{\pgfqpoint{2.134276in}{1.208986in}}%
\pgfpathlineto{\pgfqpoint{2.137448in}{1.208924in}}%
\pgfpathlineto{\pgfqpoint{2.140620in}{1.208929in}}%
\pgfpathlineto{\pgfqpoint{2.143792in}{1.209114in}}%
\pgfpathlineto{\pgfqpoint{2.146964in}{1.209335in}}%
\pgfpathlineto{\pgfqpoint{2.150136in}{1.209427in}}%
\pgfpathlineto{\pgfqpoint{2.153308in}{1.209569in}}%
\pgfpathlineto{\pgfqpoint{2.156480in}{1.209610in}}%
\pgfpathlineto{\pgfqpoint{2.159652in}{1.209573in}}%
\pgfpathlineto{\pgfqpoint{2.162824in}{1.209566in}}%
\pgfpathlineto{\pgfqpoint{2.165996in}{1.209625in}}%
\pgfpathlineto{\pgfqpoint{2.169168in}{1.209615in}}%
\pgfpathlineto{\pgfqpoint{2.172340in}{1.209722in}}%
\pgfpathlineto{\pgfqpoint{2.175512in}{1.209625in}}%
\pgfpathlineto{\pgfqpoint{2.178684in}{1.209457in}}%
\pgfpathlineto{\pgfqpoint{2.181856in}{1.209552in}}%
\pgfpathlineto{\pgfqpoint{2.185028in}{1.209615in}}%
\pgfpathlineto{\pgfqpoint{2.188200in}{1.209617in}}%
\pgfpathlineto{\pgfqpoint{2.191372in}{1.209634in}}%
\pgfpathlineto{\pgfqpoint{2.194545in}{1.209658in}}%
\pgfpathlineto{\pgfqpoint{2.197717in}{1.209798in}}%
\pgfpathlineto{\pgfqpoint{2.200889in}{1.209890in}}%
\pgfpathlineto{\pgfqpoint{2.204061in}{1.209790in}}%
\pgfpathlineto{\pgfqpoint{2.207233in}{1.209686in}}%
\pgfpathlineto{\pgfqpoint{2.210405in}{1.209777in}}%
\pgfpathlineto{\pgfqpoint{2.213577in}{1.209779in}}%
\pgfpathlineto{\pgfqpoint{2.216749in}{1.209852in}}%
\pgfpathlineto{\pgfqpoint{2.219921in}{1.209920in}}%
\pgfpathlineto{\pgfqpoint{2.223093in}{1.210198in}}%
\pgfpathlineto{\pgfqpoint{2.226265in}{1.210269in}}%
\pgfpathlineto{\pgfqpoint{2.229437in}{1.210434in}}%
\pgfpathlineto{\pgfqpoint{2.232609in}{1.210364in}}%
\pgfpathlineto{\pgfqpoint{2.235781in}{1.210422in}}%
\pgfpathlineto{\pgfqpoint{2.238953in}{1.210396in}}%
\pgfpathlineto{\pgfqpoint{2.242125in}{1.210311in}}%
\pgfpathlineto{\pgfqpoint{2.245297in}{1.210323in}}%
\pgfpathlineto{\pgfqpoint{2.248469in}{1.210400in}}%
\pgfpathlineto{\pgfqpoint{2.251641in}{1.210596in}}%
\pgfpathlineto{\pgfqpoint{2.254813in}{1.210615in}}%
\pgfpathlineto{\pgfqpoint{2.257985in}{1.210475in}}%
\pgfpathlineto{\pgfqpoint{2.261157in}{1.210377in}}%
\pgfpathlineto{\pgfqpoint{2.264329in}{1.210459in}}%
\pgfpathlineto{\pgfqpoint{2.267501in}{1.210321in}}%
\pgfpathlineto{\pgfqpoint{2.270674in}{1.210147in}}%
\pgfpathlineto{\pgfqpoint{2.273846in}{1.210147in}}%
\pgfpathlineto{\pgfqpoint{2.277018in}{1.210163in}}%
\pgfpathlineto{\pgfqpoint{2.280190in}{1.210206in}}%
\pgfpathlineto{\pgfqpoint{2.283362in}{1.210083in}}%
\pgfpathlineto{\pgfqpoint{2.286534in}{1.209957in}}%
\pgfpathlineto{\pgfqpoint{2.289706in}{1.210016in}}%
\pgfpathlineto{\pgfqpoint{2.292878in}{1.210081in}}%
\pgfpathlineto{\pgfqpoint{2.296050in}{1.210161in}}%
\pgfpathlineto{\pgfqpoint{2.299222in}{1.210276in}}%
\pgfpathlineto{\pgfqpoint{2.302394in}{1.209920in}}%
\pgfpathlineto{\pgfqpoint{2.305566in}{1.209953in}}%
\pgfpathlineto{\pgfqpoint{2.308738in}{1.209920in}}%
\pgfpathlineto{\pgfqpoint{2.311910in}{1.209835in}}%
\pgfpathlineto{\pgfqpoint{2.315082in}{1.209757in}}%
\pgfpathlineto{\pgfqpoint{2.318254in}{1.209997in}}%
\pgfpathlineto{\pgfqpoint{2.321426in}{1.209892in}}%
\pgfpathlineto{\pgfqpoint{2.324598in}{1.209810in}}%
\pgfpathlineto{\pgfqpoint{2.327770in}{1.210092in}}%
\pgfpathlineto{\pgfqpoint{2.330942in}{1.210110in}}%
\pgfpathlineto{\pgfqpoint{2.334114in}{1.210128in}}%
\pgfpathlineto{\pgfqpoint{2.337286in}{1.209997in}}%
\pgfpathlineto{\pgfqpoint{2.340458in}{1.210102in}}%
\pgfpathlineto{\pgfqpoint{2.343630in}{1.210155in}}%
\pgfpathlineto{\pgfqpoint{2.346802in}{1.210124in}}%
\pgfpathlineto{\pgfqpoint{2.349975in}{1.209988in}}%
\pgfpathlineto{\pgfqpoint{2.353147in}{1.209947in}}%
\pgfpathlineto{\pgfqpoint{2.356319in}{1.210084in}}%
\pgfpathlineto{\pgfqpoint{2.359491in}{1.210105in}}%
\pgfpathlineto{\pgfqpoint{2.362663in}{1.210044in}}%
\pgfpathlineto{\pgfqpoint{2.365835in}{1.210206in}}%
\pgfpathlineto{\pgfqpoint{2.369007in}{1.210126in}}%
\pgfpathlineto{\pgfqpoint{2.372179in}{1.210075in}}%
\pgfpathlineto{\pgfqpoint{2.375351in}{1.209961in}}%
\pgfpathlineto{\pgfqpoint{2.378523in}{1.209966in}}%
\pgfpathlineto{\pgfqpoint{2.381695in}{1.210008in}}%
\pgfpathlineto{\pgfqpoint{2.384867in}{1.210083in}}%
\pgfpathlineto{\pgfqpoint{2.388039in}{1.210100in}}%
\pgfpathlineto{\pgfqpoint{2.391211in}{1.209999in}}%
\pgfpathlineto{\pgfqpoint{2.394383in}{1.209848in}}%
\pgfpathlineto{\pgfqpoint{2.397555in}{1.209876in}}%
\pgfpathlineto{\pgfqpoint{2.400727in}{1.209816in}}%
\pgfpathlineto{\pgfqpoint{2.403899in}{1.209797in}}%
\pgfpathlineto{\pgfqpoint{2.407071in}{1.209754in}}%
\pgfpathlineto{\pgfqpoint{2.410243in}{1.209814in}}%
\pgfpathlineto{\pgfqpoint{2.413415in}{1.209705in}}%
\pgfpathlineto{\pgfqpoint{2.416587in}{1.209826in}}%
\pgfpathlineto{\pgfqpoint{2.419759in}{1.209814in}}%
\pgfpathlineto{\pgfqpoint{2.422931in}{1.210223in}}%
\pgfpathlineto{\pgfqpoint{2.426103in}{1.210225in}}%
\pgfpathlineto{\pgfqpoint{2.429276in}{1.210277in}}%
\pgfpathlineto{\pgfqpoint{2.432448in}{1.210345in}}%
\pgfpathlineto{\pgfqpoint{2.435620in}{1.210468in}}%
\pgfpathlineto{\pgfqpoint{2.438792in}{1.210753in}}%
\pgfpathlineto{\pgfqpoint{2.441964in}{1.210791in}}%
\pgfpathlineto{\pgfqpoint{2.445136in}{1.210778in}}%
\pgfpathlineto{\pgfqpoint{2.448308in}{1.210770in}}%
\pgfpathlineto{\pgfqpoint{2.451480in}{1.210847in}}%
\pgfpathlineto{\pgfqpoint{2.454652in}{1.210823in}}%
\pgfpathlineto{\pgfqpoint{2.457824in}{1.210807in}}%
\pgfpathlineto{\pgfqpoint{2.460996in}{1.210915in}}%
\pgfpathlineto{\pgfqpoint{2.464168in}{1.210892in}}%
\pgfpathlineto{\pgfqpoint{2.467340in}{1.211065in}}%
\pgfpathlineto{\pgfqpoint{2.470512in}{1.211104in}}%
\pgfpathlineto{\pgfqpoint{2.473684in}{1.211194in}}%
\pgfpathlineto{\pgfqpoint{2.476856in}{1.211346in}}%
\pgfpathlineto{\pgfqpoint{2.480028in}{1.211226in}}%
\pgfpathlineto{\pgfqpoint{2.483200in}{1.211236in}}%
\pgfpathlineto{\pgfqpoint{2.486372in}{1.211293in}}%
\pgfpathlineto{\pgfqpoint{2.489544in}{1.211240in}}%
\pgfpathlineto{\pgfqpoint{2.492716in}{1.211141in}}%
\pgfpathlineto{\pgfqpoint{2.495888in}{1.211168in}}%
\pgfpathlineto{\pgfqpoint{2.499060in}{1.211265in}}%
\pgfpathlineto{\pgfqpoint{2.502232in}{1.211125in}}%
\pgfpathlineto{\pgfqpoint{2.505405in}{1.211051in}}%
\pgfpathlineto{\pgfqpoint{2.508577in}{1.211032in}}%
\pgfpathlineto{\pgfqpoint{2.511749in}{1.211008in}}%
\pgfpathlineto{\pgfqpoint{2.514921in}{1.211095in}}%
\pgfpathlineto{\pgfqpoint{2.518093in}{1.211031in}}%
\pgfpathlineto{\pgfqpoint{2.521265in}{1.211185in}}%
\pgfpathlineto{\pgfqpoint{2.524437in}{1.211115in}}%
\pgfpathlineto{\pgfqpoint{2.527609in}{1.211052in}}%
\pgfpathlineto{\pgfqpoint{2.530781in}{1.210987in}}%
\pgfpathlineto{\pgfqpoint{2.533953in}{1.211093in}}%
\pgfpathlineto{\pgfqpoint{2.537125in}{1.211187in}}%
\pgfpathlineto{\pgfqpoint{2.540297in}{1.211597in}}%
\pgfpathlineto{\pgfqpoint{2.543469in}{1.211679in}}%
\pgfpathlineto{\pgfqpoint{2.546641in}{1.211636in}}%
\pgfpathlineto{\pgfqpoint{2.549813in}{1.211668in}}%
\pgfpathlineto{\pgfqpoint{2.552985in}{1.211519in}}%
\pgfpathlineto{\pgfqpoint{2.556157in}{1.211474in}}%
\pgfpathlineto{\pgfqpoint{2.559329in}{1.211539in}}%
\pgfpathlineto{\pgfqpoint{2.562501in}{1.211597in}}%
\pgfpathlineto{\pgfqpoint{2.565673in}{1.211418in}}%
\pgfpathlineto{\pgfqpoint{2.568845in}{1.211486in}}%
\pgfpathlineto{\pgfqpoint{2.572017in}{1.211530in}}%
\pgfpathlineto{\pgfqpoint{2.575189in}{1.211493in}}%
\pgfpathlineto{\pgfqpoint{2.578361in}{1.211441in}}%
\pgfpathlineto{\pgfqpoint{2.581533in}{1.211573in}}%
\pgfpathlineto{\pgfqpoint{2.584706in}{1.211620in}}%
\pgfpathlineto{\pgfqpoint{2.587878in}{1.211704in}}%
\pgfpathlineto{\pgfqpoint{2.591050in}{1.211559in}}%
\pgfpathlineto{\pgfqpoint{2.594222in}{1.211547in}}%
\pgfpathlineto{\pgfqpoint{2.597394in}{1.211531in}}%
\pgfpathlineto{\pgfqpoint{2.600566in}{1.211525in}}%
\pgfpathlineto{\pgfqpoint{2.603738in}{1.211483in}}%
\pgfpathlineto{\pgfqpoint{2.606910in}{1.211314in}}%
\pgfpathlineto{\pgfqpoint{2.610082in}{1.211615in}}%
\pgfpathlineto{\pgfqpoint{2.613254in}{1.211562in}}%
\pgfpathlineto{\pgfqpoint{2.616426in}{1.211475in}}%
\pgfpathlineto{\pgfqpoint{2.619598in}{1.211354in}}%
\pgfpathlineto{\pgfqpoint{2.622770in}{1.211370in}}%
\pgfpathlineto{\pgfqpoint{2.625942in}{1.211311in}}%
\pgfpathlineto{\pgfqpoint{2.629114in}{1.211236in}}%
\pgfpathlineto{\pgfqpoint{2.632286in}{1.210960in}}%
\pgfpathlineto{\pgfqpoint{2.635458in}{1.210954in}}%
\pgfpathlineto{\pgfqpoint{2.638630in}{1.210936in}}%
\pgfpathlineto{\pgfqpoint{2.641802in}{1.210858in}}%
\pgfpathlineto{\pgfqpoint{2.644974in}{1.210667in}}%
\pgfpathlineto{\pgfqpoint{2.648146in}{1.210465in}}%
\pgfpathlineto{\pgfqpoint{2.651318in}{1.210389in}}%
\pgfpathlineto{\pgfqpoint{2.654490in}{1.210704in}}%
\pgfpathlineto{\pgfqpoint{2.657662in}{1.210892in}}%
\pgfpathlineto{\pgfqpoint{2.660834in}{1.210863in}}%
\pgfpathlineto{\pgfqpoint{2.664007in}{1.210841in}}%
\pgfpathlineto{\pgfqpoint{2.667179in}{1.210876in}}%
\pgfpathlineto{\pgfqpoint{2.670351in}{1.210663in}}%
\pgfpathlineto{\pgfqpoint{2.673523in}{1.210583in}}%
\pgfpathlineto{\pgfqpoint{2.676695in}{1.210666in}}%
\pgfpathlineto{\pgfqpoint{2.679867in}{1.210698in}}%
\pgfpathlineto{\pgfqpoint{2.683039in}{1.210725in}}%
\pgfpathlineto{\pgfqpoint{2.686211in}{1.210712in}}%
\pgfpathlineto{\pgfqpoint{2.689383in}{1.210654in}}%
\pgfpathlineto{\pgfqpoint{2.692555in}{1.210624in}}%
\pgfpathlineto{\pgfqpoint{2.695727in}{1.210608in}}%
\pgfpathlineto{\pgfqpoint{2.698899in}{1.210593in}}%
\pgfpathlineto{\pgfqpoint{2.702071in}{1.210710in}}%
\pgfpathlineto{\pgfqpoint{2.705243in}{1.210470in}}%
\pgfpathlineto{\pgfqpoint{2.708415in}{1.210471in}}%
\pgfpathlineto{\pgfqpoint{2.711587in}{1.210415in}}%
\pgfpathlineto{\pgfqpoint{2.714759in}{1.210512in}}%
\pgfpathlineto{\pgfqpoint{2.717931in}{1.210326in}}%
\pgfpathlineto{\pgfqpoint{2.721103in}{1.210228in}}%
\pgfpathlineto{\pgfqpoint{2.724275in}{1.210227in}}%
\pgfpathlineto{\pgfqpoint{2.727447in}{1.210295in}}%
\pgfpathlineto{\pgfqpoint{2.730619in}{1.210473in}}%
\pgfpathlineto{\pgfqpoint{2.733791in}{1.210757in}}%
\pgfpathlineto{\pgfqpoint{2.736963in}{1.210790in}}%
\pgfpathlineto{\pgfqpoint{2.740136in}{1.210728in}}%
\pgfpathlineto{\pgfqpoint{2.743308in}{1.210699in}}%
\pgfpathlineto{\pgfqpoint{2.746480in}{1.210790in}}%
\pgfpathlineto{\pgfqpoint{2.749652in}{1.210808in}}%
\pgfpathlineto{\pgfqpoint{2.752824in}{1.210787in}}%
\pgfpathlineto{\pgfqpoint{2.755996in}{1.210659in}}%
\pgfpathlineto{\pgfqpoint{2.759168in}{1.210592in}}%
\pgfpathlineto{\pgfqpoint{2.762340in}{1.210863in}}%
\pgfpathlineto{\pgfqpoint{2.765512in}{1.211052in}}%
\pgfpathlineto{\pgfqpoint{2.768684in}{1.210876in}}%
\pgfpathlineto{\pgfqpoint{2.771856in}{1.210874in}}%
\pgfpathlineto{\pgfqpoint{2.775028in}{1.210931in}}%
\pgfpathlineto{\pgfqpoint{2.778200in}{1.210902in}}%
\pgfpathlineto{\pgfqpoint{2.781372in}{1.210615in}}%
\pgfpathlineto{\pgfqpoint{2.784544in}{1.210601in}}%
\pgfpathlineto{\pgfqpoint{2.787716in}{1.210584in}}%
\pgfpathlineto{\pgfqpoint{2.790888in}{1.210657in}}%
\pgfpathlineto{\pgfqpoint{2.794060in}{1.210778in}}%
\pgfpathlineto{\pgfqpoint{2.797232in}{1.210963in}}%
\pgfpathlineto{\pgfqpoint{2.800404in}{1.210992in}}%
\pgfpathlineto{\pgfqpoint{2.803576in}{1.211065in}}%
\pgfpathlineto{\pgfqpoint{2.806748in}{1.211160in}}%
\pgfpathlineto{\pgfqpoint{2.809920in}{1.211280in}}%
\pgfpathlineto{\pgfqpoint{2.813092in}{1.211200in}}%
\pgfpathlineto{\pgfqpoint{2.816264in}{1.211230in}}%
\pgfpathlineto{\pgfqpoint{2.819437in}{1.211460in}}%
\pgfpathlineto{\pgfqpoint{2.822609in}{1.211477in}}%
\pgfpathlineto{\pgfqpoint{2.825781in}{1.211557in}}%
\pgfpathlineto{\pgfqpoint{2.828953in}{1.211889in}}%
\pgfpathlineto{\pgfqpoint{2.832125in}{1.211842in}}%
\pgfpathlineto{\pgfqpoint{2.835297in}{1.211647in}}%
\pgfpathlineto{\pgfqpoint{2.838469in}{1.211632in}}%
\pgfpathlineto{\pgfqpoint{2.841641in}{1.211839in}}%
\pgfpathlineto{\pgfqpoint{2.844813in}{1.211838in}}%
\pgfpathlineto{\pgfqpoint{2.847985in}{1.211770in}}%
\pgfpathlineto{\pgfqpoint{2.851157in}{1.211516in}}%
\pgfpathlineto{\pgfqpoint{2.854329in}{1.211493in}}%
\pgfpathlineto{\pgfqpoint{2.857501in}{1.211597in}}%
\pgfpathlineto{\pgfqpoint{2.860673in}{1.211769in}}%
\pgfpathlineto{\pgfqpoint{2.863845in}{1.211645in}}%
\pgfpathlineto{\pgfqpoint{2.867017in}{1.211527in}}%
\pgfpathlineto{\pgfqpoint{2.870189in}{1.211831in}}%
\pgfpathlineto{\pgfqpoint{2.873361in}{1.211778in}}%
\pgfpathlineto{\pgfqpoint{2.876533in}{1.211918in}}%
\pgfpathlineto{\pgfqpoint{2.879705in}{1.212102in}}%
\pgfpathlineto{\pgfqpoint{2.882877in}{1.212118in}}%
\pgfpathlineto{\pgfqpoint{2.886049in}{1.211907in}}%
\pgfpathlineto{\pgfqpoint{2.889221in}{1.212017in}}%
\pgfpathlineto{\pgfqpoint{2.892393in}{1.211900in}}%
\pgfpathlineto{\pgfqpoint{2.895565in}{1.212122in}}%
\pgfpathlineto{\pgfqpoint{2.898738in}{1.212252in}}%
\pgfpathlineto{\pgfqpoint{2.901910in}{1.212454in}}%
\pgfpathlineto{\pgfqpoint{2.905082in}{1.212252in}}%
\pgfpathlineto{\pgfqpoint{2.908254in}{1.212164in}}%
\pgfpathlineto{\pgfqpoint{2.911426in}{1.212005in}}%
\pgfpathlineto{\pgfqpoint{2.914598in}{1.212022in}}%
\pgfpathlineto{\pgfqpoint{2.917770in}{1.212194in}}%
\pgfpathlineto{\pgfqpoint{2.920942in}{1.212160in}}%
\pgfpathlineto{\pgfqpoint{2.924114in}{1.212237in}}%
\pgfpathlineto{\pgfqpoint{2.927286in}{1.212308in}}%
\pgfpathlineto{\pgfqpoint{2.930458in}{1.212373in}}%
\pgfpathlineto{\pgfqpoint{2.933630in}{1.212502in}}%
\pgfpathlineto{\pgfqpoint{2.936802in}{1.212491in}}%
\pgfpathlineto{\pgfqpoint{2.939974in}{1.212496in}}%
\pgfpathlineto{\pgfqpoint{2.943146in}{1.212363in}}%
\pgfpathlineto{\pgfqpoint{2.946318in}{1.212550in}}%
\pgfpathlineto{\pgfqpoint{2.949490in}{1.212564in}}%
\pgfpathlineto{\pgfqpoint{2.952662in}{1.212467in}}%
\pgfpathlineto{\pgfqpoint{2.955834in}{1.212654in}}%
\pgfpathlineto{\pgfqpoint{2.959006in}{1.213302in}}%
\pgfpathlineto{\pgfqpoint{2.962178in}{1.213302in}}%
\pgfpathlineto{\pgfqpoint{2.965350in}{1.213126in}}%
\pgfpathlineto{\pgfqpoint{2.968522in}{1.213283in}}%
\pgfpathlineto{\pgfqpoint{2.971694in}{1.213525in}}%
\pgfpathlineto{\pgfqpoint{2.974867in}{1.213630in}}%
\pgfpathlineto{\pgfqpoint{2.978039in}{1.213686in}}%
\pgfpathlineto{\pgfqpoint{2.981211in}{1.213925in}}%
\pgfpathlineto{\pgfqpoint{2.984383in}{1.214035in}}%
\pgfpathlineto{\pgfqpoint{2.987555in}{1.214332in}}%
\pgfpathlineto{\pgfqpoint{2.990727in}{1.214398in}}%
\pgfpathlineto{\pgfqpoint{2.993899in}{1.214458in}}%
\pgfpathlineto{\pgfqpoint{2.997071in}{1.214634in}}%
\pgfpathlineto{\pgfqpoint{3.000243in}{1.214772in}}%
\pgfpathlineto{\pgfqpoint{3.003415in}{1.214801in}}%
\pgfpathlineto{\pgfqpoint{3.006587in}{1.215068in}}%
\pgfpathlineto{\pgfqpoint{3.009759in}{1.215177in}}%
\pgfpathlineto{\pgfqpoint{3.012931in}{1.215091in}}%
\pgfpathlineto{\pgfqpoint{3.016103in}{1.215019in}}%
\pgfpathlineto{\pgfqpoint{3.019275in}{1.214886in}}%
\pgfpathlineto{\pgfqpoint{3.022447in}{1.214787in}}%
\pgfpathlineto{\pgfqpoint{3.025619in}{1.214847in}}%
\pgfpathlineto{\pgfqpoint{3.028791in}{1.214930in}}%
\pgfpathlineto{\pgfqpoint{3.031963in}{1.214967in}}%
\pgfpathlineto{\pgfqpoint{3.035135in}{1.214930in}}%
\pgfpathlineto{\pgfqpoint{3.038307in}{1.214920in}}%
\pgfpathlineto{\pgfqpoint{3.041479in}{1.214781in}}%
\pgfpathlineto{\pgfqpoint{3.044651in}{1.214809in}}%
\pgfpathlineto{\pgfqpoint{3.047823in}{1.214785in}}%
\pgfpathlineto{\pgfqpoint{3.050995in}{1.214744in}}%
\pgfpathlineto{\pgfqpoint{3.054168in}{1.214708in}}%
\pgfpathlineto{\pgfqpoint{3.057340in}{1.214965in}}%
\pgfpathlineto{\pgfqpoint{3.060512in}{1.215153in}}%
\pgfpathlineto{\pgfqpoint{3.063684in}{1.215119in}}%
\pgfpathlineto{\pgfqpoint{3.066856in}{1.215014in}}%
\pgfpathlineto{\pgfqpoint{3.070028in}{1.214967in}}%
\pgfpathlineto{\pgfqpoint{3.073200in}{1.214852in}}%
\pgfpathlineto{\pgfqpoint{3.076372in}{1.214702in}}%
\pgfpathlineto{\pgfqpoint{3.079544in}{1.214796in}}%
\pgfpathlineto{\pgfqpoint{3.082716in}{1.214902in}}%
\pgfpathlineto{\pgfqpoint{3.085888in}{1.214949in}}%
\pgfpathlineto{\pgfqpoint{3.089060in}{1.214889in}}%
\pgfpathlineto{\pgfqpoint{3.092232in}{1.214972in}}%
\pgfpathlineto{\pgfqpoint{3.095404in}{1.215126in}}%
\pgfpathlineto{\pgfqpoint{3.098576in}{1.215077in}}%
\pgfpathlineto{\pgfqpoint{3.101748in}{1.215108in}}%
\pgfpathlineto{\pgfqpoint{3.104920in}{1.215145in}}%
\pgfpathlineto{\pgfqpoint{3.108092in}{1.215243in}}%
\pgfpathlineto{\pgfqpoint{3.111264in}{1.215521in}}%
\pgfpathlineto{\pgfqpoint{3.114436in}{1.215472in}}%
\pgfpathlineto{\pgfqpoint{3.117608in}{1.215590in}}%
\pgfpathlineto{\pgfqpoint{3.120780in}{1.215491in}}%
\pgfpathlineto{\pgfqpoint{3.123952in}{1.215661in}}%
\pgfpathlineto{\pgfqpoint{3.127124in}{1.215713in}}%
\pgfpathlineto{\pgfqpoint{3.130297in}{1.215598in}}%
\pgfpathlineto{\pgfqpoint{3.133469in}{1.215425in}}%
\pgfpathlineto{\pgfqpoint{3.136641in}{1.215616in}}%
\pgfpathlineto{\pgfqpoint{3.139813in}{1.215853in}}%
\pgfpathlineto{\pgfqpoint{3.142985in}{1.216196in}}%
\pgfpathlineto{\pgfqpoint{3.146157in}{1.215558in}}%
\pgfpathlineto{\pgfqpoint{3.149329in}{1.215419in}}%
\pgfpathlineto{\pgfqpoint{3.152501in}{1.215303in}}%
\pgfpathlineto{\pgfqpoint{3.155673in}{1.215486in}}%
\pgfpathlineto{\pgfqpoint{3.158845in}{1.215656in}}%
\pgfpathlineto{\pgfqpoint{3.162017in}{1.215688in}}%
\pgfpathlineto{\pgfqpoint{3.165189in}{1.215427in}}%
\pgfpathlineto{\pgfqpoint{3.168361in}{1.216033in}}%
\pgfpathlineto{\pgfqpoint{3.171533in}{1.216022in}}%
\pgfpathlineto{\pgfqpoint{3.174705in}{1.216041in}}%
\pgfpathlineto{\pgfqpoint{3.177877in}{1.215768in}}%
\pgfpathlineto{\pgfqpoint{3.181049in}{1.215481in}}%
\pgfpathlineto{\pgfqpoint{3.184221in}{1.215610in}}%
\pgfpathlineto{\pgfqpoint{3.187393in}{1.215719in}}%
\pgfpathlineto{\pgfqpoint{3.190565in}{1.215667in}}%
\pgfpathlineto{\pgfqpoint{3.193737in}{1.215628in}}%
\pgfpathlineto{\pgfqpoint{3.196909in}{1.215446in}}%
\pgfpathlineto{\pgfqpoint{3.200081in}{1.215598in}}%
\pgfpathlineto{\pgfqpoint{3.203253in}{1.215411in}}%
\pgfpathlineto{\pgfqpoint{3.206425in}{1.215316in}}%
\pgfpathlineto{\pgfqpoint{3.209598in}{1.214354in}}%
\pgfpathlineto{\pgfqpoint{3.212770in}{1.213994in}}%
\pgfpathlineto{\pgfqpoint{3.215942in}{1.214058in}}%
\pgfpathlineto{\pgfqpoint{3.219114in}{1.213773in}}%
\pgfpathlineto{\pgfqpoint{3.222286in}{1.213726in}}%
\pgfpathlineto{\pgfqpoint{3.225458in}{1.214258in}}%
\pgfpathlineto{\pgfqpoint{3.228630in}{1.213961in}}%
\pgfpathlineto{\pgfqpoint{3.231802in}{1.213358in}}%
\pgfpathlineto{\pgfqpoint{3.234974in}{1.213173in}}%
\pgfpathlineto{\pgfqpoint{3.238146in}{1.212631in}}%
\pgfpathlineto{\pgfqpoint{3.241318in}{1.212459in}}%
\pgfpathlineto{\pgfqpoint{3.244490in}{1.211859in}}%
\pgfpathlineto{\pgfqpoint{3.247662in}{1.211638in}}%
\pgfpathlineto{\pgfqpoint{3.250834in}{1.211592in}}%
\pgfpathlineto{\pgfqpoint{3.254006in}{1.211314in}}%
\pgfpathlineto{\pgfqpoint{3.257178in}{1.210865in}}%
\pgfpathlineto{\pgfqpoint{3.260350in}{1.210532in}}%
\pgfpathlineto{\pgfqpoint{3.263522in}{1.210018in}}%
\pgfpathlineto{\pgfqpoint{3.266694in}{1.210202in}}%
\pgfpathlineto{\pgfqpoint{3.269866in}{1.210118in}}%
\pgfpathlineto{\pgfqpoint{3.273038in}{1.209740in}}%
\pgfpathlineto{\pgfqpoint{3.276210in}{1.209959in}}%
\pgfpathlineto{\pgfqpoint{3.279382in}{1.210445in}}%
\pgfpathlineto{\pgfqpoint{3.282554in}{1.210834in}}%
\pgfpathlineto{\pgfqpoint{3.285726in}{1.210524in}}%
\pgfpathlineto{\pgfqpoint{3.288899in}{1.210517in}}%
\pgfpathlineto{\pgfqpoint{3.292071in}{1.210899in}}%
\pgfpathlineto{\pgfqpoint{3.295243in}{1.210914in}}%
\pgfpathlineto{\pgfqpoint{3.298415in}{1.210826in}}%
\pgfpathlineto{\pgfqpoint{3.301587in}{1.211373in}}%
\pgfpathlineto{\pgfqpoint{3.304759in}{1.211266in}}%
\pgfpathlineto{\pgfqpoint{3.307931in}{1.211677in}}%
\pgfpathlineto{\pgfqpoint{3.311103in}{1.211326in}}%
\pgfpathlineto{\pgfqpoint{3.314275in}{1.211387in}}%
\pgfpathlineto{\pgfqpoint{3.317447in}{1.211317in}}%
\pgfpathlineto{\pgfqpoint{3.320619in}{1.211414in}}%
\pgfpathlineto{\pgfqpoint{3.323791in}{1.211640in}}%
\pgfpathlineto{\pgfqpoint{3.326963in}{1.211318in}}%
\pgfpathlineto{\pgfqpoint{3.330135in}{1.211868in}}%
\pgfpathlineto{\pgfqpoint{3.333307in}{1.211766in}}%
\pgfpathlineto{\pgfqpoint{3.336479in}{1.211581in}}%
\pgfpathlineto{\pgfqpoint{3.339651in}{1.211779in}}%
\pgfpathlineto{\pgfqpoint{3.342823in}{1.211776in}}%
\pgfpathlineto{\pgfqpoint{3.345995in}{1.211731in}}%
\pgfpathlineto{\pgfqpoint{3.349167in}{1.211468in}}%
\pgfpathlineto{\pgfqpoint{3.352339in}{1.210868in}}%
\pgfpathlineto{\pgfqpoint{3.355511in}{1.210624in}}%
\pgfpathlineto{\pgfqpoint{3.358683in}{1.210835in}}%
\pgfpathlineto{\pgfqpoint{3.361855in}{1.210626in}}%
\pgfpathlineto{\pgfqpoint{3.365028in}{1.210332in}}%
\pgfpathlineto{\pgfqpoint{3.368200in}{1.210453in}}%
\pgfpathlineto{\pgfqpoint{3.371372in}{1.210240in}}%
\pgfpathlineto{\pgfqpoint{3.374544in}{1.210364in}}%
\pgfpathlineto{\pgfqpoint{3.377716in}{1.210381in}}%
\pgfpathlineto{\pgfqpoint{3.380888in}{1.210661in}}%
\pgfpathlineto{\pgfqpoint{3.384060in}{1.211037in}}%
\pgfpathlineto{\pgfqpoint{3.387232in}{1.211063in}}%
\pgfpathlineto{\pgfqpoint{3.390404in}{1.211004in}}%
\pgfpathlineto{\pgfqpoint{3.393576in}{1.210977in}}%
\pgfpathlineto{\pgfqpoint{3.396748in}{1.210545in}}%
\pgfpathlineto{\pgfqpoint{3.399920in}{1.210271in}}%
\pgfpathlineto{\pgfqpoint{3.403092in}{1.209655in}}%
\pgfpathlineto{\pgfqpoint{3.406264in}{1.209784in}}%
\pgfpathlineto{\pgfqpoint{3.409436in}{1.209857in}}%
\pgfpathlineto{\pgfqpoint{3.412608in}{1.209901in}}%
\pgfpathlineto{\pgfqpoint{3.415780in}{1.209575in}}%
\pgfpathlineto{\pgfqpoint{3.418952in}{1.209299in}}%
\pgfpathlineto{\pgfqpoint{3.422124in}{1.209595in}}%
\pgfpathlineto{\pgfqpoint{3.425296in}{1.209461in}}%
\pgfpathlineto{\pgfqpoint{3.428468in}{1.209168in}}%
\pgfpathlineto{\pgfqpoint{3.431640in}{1.209270in}}%
\pgfpathlineto{\pgfqpoint{3.434812in}{1.208978in}}%
\pgfpathlineto{\pgfqpoint{3.437984in}{1.208754in}}%
\pgfpathlineto{\pgfqpoint{3.441156in}{1.208925in}}%
\pgfpathlineto{\pgfqpoint{3.444329in}{1.209035in}}%
\pgfpathlineto{\pgfqpoint{3.447501in}{1.208794in}}%
\pgfpathlineto{\pgfqpoint{3.450673in}{1.208784in}}%
\pgfpathlineto{\pgfqpoint{3.453845in}{1.208625in}}%
\pgfpathlineto{\pgfqpoint{3.457017in}{1.208623in}}%
\pgfpathlineto{\pgfqpoint{3.460189in}{1.208741in}}%
\pgfpathlineto{\pgfqpoint{3.463361in}{1.208502in}}%
\pgfpathlineto{\pgfqpoint{3.466533in}{1.208068in}}%
\pgfpathlineto{\pgfqpoint{3.469705in}{1.208154in}}%
\pgfpathlineto{\pgfqpoint{3.472877in}{1.208164in}}%
\pgfpathlineto{\pgfqpoint{3.476049in}{1.208529in}}%
\pgfpathlineto{\pgfqpoint{3.479221in}{1.208449in}}%
\pgfpathlineto{\pgfqpoint{3.482393in}{1.208550in}}%
\pgfpathlineto{\pgfqpoint{3.485565in}{1.208637in}}%
\pgfpathlineto{\pgfqpoint{3.488737in}{1.208779in}}%
\pgfpathlineto{\pgfqpoint{3.491909in}{1.208646in}}%
\pgfpathlineto{\pgfqpoint{3.495081in}{1.208608in}}%
\pgfpathlineto{\pgfqpoint{3.498253in}{1.209019in}}%
\pgfpathlineto{\pgfqpoint{3.501425in}{1.208498in}}%
\pgfpathlineto{\pgfqpoint{3.504597in}{1.208697in}}%
\pgfpathlineto{\pgfqpoint{3.507769in}{1.208669in}}%
\pgfpathlineto{\pgfqpoint{3.510941in}{1.208718in}}%
\pgfpathlineto{\pgfqpoint{3.514113in}{1.208739in}}%
\pgfpathlineto{\pgfqpoint{3.517285in}{1.208607in}}%
\pgfpathlineto{\pgfqpoint{3.520457in}{1.207931in}}%
\pgfpathlineto{\pgfqpoint{3.523630in}{1.207635in}}%
\pgfpathlineto{\pgfqpoint{3.526802in}{1.207520in}}%
\pgfpathlineto{\pgfqpoint{3.529974in}{1.207803in}}%
\pgfpathlineto{\pgfqpoint{3.533146in}{1.207438in}}%
\pgfpathlineto{\pgfqpoint{3.536318in}{1.207192in}}%
\pgfpathlineto{\pgfqpoint{3.539490in}{1.207294in}}%
\pgfpathlineto{\pgfqpoint{3.542662in}{1.206753in}}%
\pgfpathlineto{\pgfqpoint{3.545834in}{1.207060in}}%
\pgfpathlineto{\pgfqpoint{3.549006in}{1.207287in}}%
\pgfpathlineto{\pgfqpoint{3.552178in}{1.207335in}}%
\pgfpathlineto{\pgfqpoint{3.555350in}{1.207253in}}%
\pgfpathlineto{\pgfqpoint{3.558522in}{1.207488in}}%
\pgfpathlineto{\pgfqpoint{3.561694in}{1.207526in}}%
\pgfpathlineto{\pgfqpoint{3.564866in}{1.207532in}}%
\pgfpathlineto{\pgfqpoint{3.568038in}{1.208036in}}%
\pgfpathlineto{\pgfqpoint{3.571210in}{1.207972in}}%
\pgfpathlineto{\pgfqpoint{3.574382in}{1.207445in}}%
\pgfpathlineto{\pgfqpoint{3.577554in}{1.207418in}}%
\pgfpathlineto{\pgfqpoint{3.580726in}{1.207335in}}%
\pgfpathlineto{\pgfqpoint{3.583898in}{1.206716in}}%
\pgfpathlineto{\pgfqpoint{3.587070in}{1.206670in}}%
\pgfpathlineto{\pgfqpoint{3.590242in}{1.206663in}}%
\pgfpathlineto{\pgfqpoint{3.593414in}{1.207087in}}%
\pgfpathlineto{\pgfqpoint{3.596586in}{1.206964in}}%
\pgfpathlineto{\pgfqpoint{3.599759in}{1.206294in}}%
\pgfpathlineto{\pgfqpoint{3.602931in}{1.206082in}}%
\pgfpathlineto{\pgfqpoint{3.606103in}{1.206092in}}%
\pgfpathlineto{\pgfqpoint{3.609275in}{1.206542in}}%
\pgfpathlineto{\pgfqpoint{3.612447in}{1.206608in}}%
\pgfpathlineto{\pgfqpoint{3.615619in}{1.206639in}}%
\pgfpathlineto{\pgfqpoint{3.618791in}{1.206444in}}%
\pgfpathlineto{\pgfqpoint{3.621963in}{1.206205in}}%
\pgfpathlineto{\pgfqpoint{3.625135in}{1.206360in}}%
\pgfpathlineto{\pgfqpoint{3.628307in}{1.206100in}}%
\pgfpathlineto{\pgfqpoint{3.631479in}{1.205753in}}%
\pgfpathlineto{\pgfqpoint{3.634651in}{1.205674in}}%
\pgfpathlineto{\pgfqpoint{3.637823in}{1.205321in}}%
\pgfpathlineto{\pgfqpoint{3.640995in}{1.204866in}}%
\pgfpathlineto{\pgfqpoint{3.644167in}{1.204589in}}%
\pgfpathlineto{\pgfqpoint{3.647339in}{1.204012in}}%
\pgfpathlineto{\pgfqpoint{3.650511in}{1.203833in}}%
\pgfpathlineto{\pgfqpoint{3.653683in}{1.203275in}}%
\pgfpathlineto{\pgfqpoint{3.656855in}{1.202588in}}%
\pgfpathlineto{\pgfqpoint{3.660027in}{1.202387in}}%
\pgfpathlineto{\pgfqpoint{3.663199in}{1.202106in}}%
\pgfpathlineto{\pgfqpoint{3.666371in}{1.202175in}}%
\pgfpathlineto{\pgfqpoint{3.669543in}{1.201161in}}%
\pgfpathlineto{\pgfqpoint{3.672715in}{1.201125in}}%
\pgfpathlineto{\pgfqpoint{3.675887in}{1.201293in}}%
\pgfpathlineto{\pgfqpoint{3.679060in}{1.201146in}}%
\pgfpathlineto{\pgfqpoint{3.682232in}{1.201131in}}%
\pgfpathlineto{\pgfqpoint{3.685404in}{1.201079in}}%
\pgfpathlineto{\pgfqpoint{3.688576in}{1.200725in}}%
\pgfpathlineto{\pgfqpoint{3.691748in}{1.200659in}}%
\pgfpathlineto{\pgfqpoint{3.694920in}{1.200540in}}%
\pgfpathlineto{\pgfqpoint{3.698092in}{1.200272in}}%
\pgfpathlineto{\pgfqpoint{3.701264in}{1.199845in}}%
\pgfpathlineto{\pgfqpoint{3.704436in}{1.199894in}}%
\pgfpathlineto{\pgfqpoint{3.707608in}{1.199438in}}%
\pgfpathlineto{\pgfqpoint{3.710780in}{1.199125in}}%
\pgfpathlineto{\pgfqpoint{3.713952in}{1.199258in}}%
\pgfpathlineto{\pgfqpoint{3.717124in}{1.199580in}}%
\pgfpathlineto{\pgfqpoint{3.720296in}{1.199445in}}%
\pgfpathlineto{\pgfqpoint{3.723468in}{1.199236in}}%
\pgfpathlineto{\pgfqpoint{3.726640in}{1.199085in}}%
\pgfpathlineto{\pgfqpoint{3.729812in}{1.199051in}}%
\pgfpathlineto{\pgfqpoint{3.732984in}{1.199103in}}%
\pgfpathlineto{\pgfqpoint{3.736156in}{1.198877in}}%
\pgfpathlineto{\pgfqpoint{3.739328in}{1.199122in}}%
\pgfpathlineto{\pgfqpoint{3.742500in}{1.198583in}}%
\pgfpathlineto{\pgfqpoint{3.745672in}{1.198633in}}%
\pgfpathlineto{\pgfqpoint{3.748844in}{1.198351in}}%
\pgfpathlineto{\pgfqpoint{3.752016in}{1.198080in}}%
\pgfpathlineto{\pgfqpoint{3.755188in}{1.198305in}}%
\pgfpathlineto{\pgfqpoint{3.758361in}{1.198142in}}%
\pgfpathlineto{\pgfqpoint{3.761533in}{1.198000in}}%
\pgfpathlineto{\pgfqpoint{3.764705in}{1.197999in}}%
\pgfpathlineto{\pgfqpoint{3.767877in}{1.197462in}}%
\pgfpathlineto{\pgfqpoint{3.771049in}{1.197192in}}%
\pgfpathlineto{\pgfqpoint{3.774221in}{1.196890in}}%
\pgfpathlineto{\pgfqpoint{3.777393in}{1.196809in}}%
\pgfpathlineto{\pgfqpoint{3.780565in}{1.196559in}}%
\pgfpathlineto{\pgfqpoint{3.783737in}{1.196453in}}%
\pgfpathlineto{\pgfqpoint{3.786909in}{1.196049in}}%
\pgfpathlineto{\pgfqpoint{3.790081in}{1.195926in}}%
\pgfpathlineto{\pgfqpoint{3.793253in}{1.195991in}}%
\pgfpathlineto{\pgfqpoint{3.796425in}{1.195938in}}%
\pgfpathlineto{\pgfqpoint{3.799597in}{1.196020in}}%
\pgfpathlineto{\pgfqpoint{3.802769in}{1.195807in}}%
\pgfpathlineto{\pgfqpoint{3.805941in}{1.195414in}}%
\pgfpathlineto{\pgfqpoint{3.809113in}{1.195030in}}%
\pgfpathlineto{\pgfqpoint{3.812285in}{1.195243in}}%
\pgfpathlineto{\pgfqpoint{3.815457in}{1.195592in}}%
\pgfpathlineto{\pgfqpoint{3.818629in}{1.195583in}}%
\pgfpathlineto{\pgfqpoint{3.821801in}{1.195431in}}%
\pgfpathlineto{\pgfqpoint{3.824973in}{1.195177in}}%
\pgfpathlineto{\pgfqpoint{3.828145in}{1.195268in}}%
\pgfpathlineto{\pgfqpoint{3.831317in}{1.195185in}}%
\pgfpathlineto{\pgfqpoint{3.834490in}{1.195335in}}%
\pgfpathlineto{\pgfqpoint{3.837662in}{1.195013in}}%
\pgfpathlineto{\pgfqpoint{3.840834in}{1.194790in}}%
\pgfpathlineto{\pgfqpoint{3.844006in}{1.194448in}}%
\pgfpathlineto{\pgfqpoint{3.847178in}{1.194389in}}%
\pgfpathlineto{\pgfqpoint{3.850350in}{1.194027in}}%
\pgfpathlineto{\pgfqpoint{3.853522in}{1.193788in}}%
\pgfpathlineto{\pgfqpoint{3.856694in}{1.193720in}}%
\pgfpathlineto{\pgfqpoint{3.859866in}{1.193639in}}%
\pgfpathlineto{\pgfqpoint{3.863038in}{1.193309in}}%
\pgfpathlineto{\pgfqpoint{3.866210in}{1.192795in}}%
\pgfpathlineto{\pgfqpoint{3.869382in}{1.192642in}}%
\pgfpathlineto{\pgfqpoint{3.872554in}{1.192574in}}%
\pgfpathlineto{\pgfqpoint{3.875726in}{1.192497in}}%
\pgfpathlineto{\pgfqpoint{3.878898in}{1.192510in}}%
\pgfpathlineto{\pgfqpoint{3.882070in}{1.192338in}}%
\pgfpathlineto{\pgfqpoint{3.885242in}{1.191786in}}%
\pgfpathlineto{\pgfqpoint{3.888414in}{1.191805in}}%
\pgfpathlineto{\pgfqpoint{3.891586in}{1.191970in}}%
\pgfpathlineto{\pgfqpoint{3.894758in}{1.191875in}}%
\pgfpathlineto{\pgfqpoint{3.897930in}{1.191815in}}%
\pgfpathlineto{\pgfqpoint{3.901102in}{1.192037in}}%
\pgfpathlineto{\pgfqpoint{3.904274in}{1.192427in}}%
\pgfpathlineto{\pgfqpoint{3.907446in}{1.192335in}}%
\pgfpathlineto{\pgfqpoint{3.910618in}{1.192576in}}%
\pgfpathlineto{\pgfqpoint{3.913791in}{1.192314in}}%
\pgfpathlineto{\pgfqpoint{3.916963in}{1.192103in}}%
\pgfpathlineto{\pgfqpoint{3.920135in}{1.191849in}}%
\pgfpathlineto{\pgfqpoint{3.923307in}{1.191867in}}%
\pgfpathlineto{\pgfqpoint{3.926479in}{1.191881in}}%
\pgfpathlineto{\pgfqpoint{3.929651in}{1.191497in}}%
\pgfpathlineto{\pgfqpoint{3.932823in}{1.191248in}}%
\pgfpathlineto{\pgfqpoint{3.935995in}{1.190740in}}%
\pgfpathlineto{\pgfqpoint{3.939167in}{1.190775in}}%
\pgfpathlineto{\pgfqpoint{3.942339in}{1.190658in}}%
\pgfpathlineto{\pgfqpoint{3.945511in}{1.190467in}}%
\pgfpathlineto{\pgfqpoint{3.948683in}{1.190484in}}%
\pgfpathlineto{\pgfqpoint{3.951855in}{1.190133in}}%
\pgfpathlineto{\pgfqpoint{3.955027in}{1.189244in}}%
\pgfpathlineto{\pgfqpoint{3.958199in}{1.189227in}}%
\pgfpathlineto{\pgfqpoint{3.961371in}{1.188825in}}%
\pgfpathlineto{\pgfqpoint{3.964543in}{1.188848in}}%
\pgfpathlineto{\pgfqpoint{3.967715in}{1.188728in}}%
\pgfpathlineto{\pgfqpoint{3.970887in}{1.188678in}}%
\pgfpathlineto{\pgfqpoint{3.974059in}{1.188767in}}%
\pgfpathlineto{\pgfqpoint{3.977231in}{1.188559in}}%
\pgfpathlineto{\pgfqpoint{3.980403in}{1.188798in}}%
\pgfpathlineto{\pgfqpoint{3.983575in}{1.188497in}}%
\pgfpathlineto{\pgfqpoint{3.986747in}{1.188339in}}%
\pgfpathlineto{\pgfqpoint{3.989919in}{1.188281in}}%
\pgfpathlineto{\pgfqpoint{3.993092in}{1.188326in}}%
\pgfpathlineto{\pgfqpoint{3.996264in}{1.188462in}}%
\pgfpathlineto{\pgfqpoint{3.999436in}{1.188322in}}%
\pgfpathlineto{\pgfqpoint{4.002608in}{1.188492in}}%
\pgfpathlineto{\pgfqpoint{4.005780in}{1.188845in}}%
\pgfpathlineto{\pgfqpoint{4.008952in}{1.188890in}}%
\pgfpathlineto{\pgfqpoint{4.012124in}{1.188789in}}%
\pgfpathlineto{\pgfqpoint{4.015296in}{1.188514in}}%
\pgfpathlineto{\pgfqpoint{4.018468in}{1.188648in}}%
\pgfpathlineto{\pgfqpoint{4.021640in}{1.188371in}}%
\pgfpathlineto{\pgfqpoint{4.024812in}{1.188357in}}%
\pgfpathlineto{\pgfqpoint{4.027984in}{1.188402in}}%
\pgfpathlineto{\pgfqpoint{4.031156in}{1.188297in}}%
\pgfpathlineto{\pgfqpoint{4.034328in}{1.188183in}}%
\pgfpathlineto{\pgfqpoint{4.037500in}{1.188009in}}%
\pgfpathlineto{\pgfqpoint{4.040672in}{1.187902in}}%
\pgfpathlineto{\pgfqpoint{4.043844in}{1.187851in}}%
\pgfpathlineto{\pgfqpoint{4.047016in}{1.187673in}}%
\pgfpathlineto{\pgfqpoint{4.050188in}{1.187738in}}%
\pgfpathlineto{\pgfqpoint{4.053360in}{1.187391in}}%
\pgfpathlineto{\pgfqpoint{4.056532in}{1.187169in}}%
\pgfpathlineto{\pgfqpoint{4.059704in}{1.187306in}}%
\pgfpathlineto{\pgfqpoint{4.062876in}{1.187309in}}%
\pgfpathlineto{\pgfqpoint{4.066048in}{1.187208in}}%
\pgfpathlineto{\pgfqpoint{4.069221in}{1.186998in}}%
\pgfpathlineto{\pgfqpoint{4.072393in}{1.186914in}}%
\pgfpathlineto{\pgfqpoint{4.075565in}{1.186556in}}%
\pgfpathlineto{\pgfqpoint{4.078737in}{1.186060in}}%
\pgfpathlineto{\pgfqpoint{4.081909in}{1.185855in}}%
\pgfpathlineto{\pgfqpoint{4.085081in}{1.185476in}}%
\pgfpathlineto{\pgfqpoint{4.088253in}{1.185437in}}%
\pgfpathlineto{\pgfqpoint{4.091425in}{1.185054in}}%
\pgfpathlineto{\pgfqpoint{4.094597in}{1.185220in}}%
\pgfpathlineto{\pgfqpoint{4.097769in}{1.185203in}}%
\pgfpathlineto{\pgfqpoint{4.100941in}{1.185240in}}%
\pgfpathlineto{\pgfqpoint{4.104113in}{1.185118in}}%
\pgfpathlineto{\pgfqpoint{4.107285in}{1.184947in}}%
\pgfpathlineto{\pgfqpoint{4.110457in}{1.184367in}}%
\pgfpathlineto{\pgfqpoint{4.113629in}{1.184404in}}%
\pgfpathlineto{\pgfqpoint{4.116801in}{1.184389in}}%
\pgfpathlineto{\pgfqpoint{4.119973in}{1.184623in}}%
\pgfpathlineto{\pgfqpoint{4.123145in}{1.184575in}}%
\pgfpathlineto{\pgfqpoint{4.126317in}{1.184489in}}%
\pgfpathlineto{\pgfqpoint{4.129489in}{1.184322in}}%
\pgfpathlineto{\pgfqpoint{4.132661in}{1.184271in}}%
\pgfpathlineto{\pgfqpoint{4.135833in}{1.184195in}}%
\pgfpathlineto{\pgfqpoint{4.139005in}{1.184028in}}%
\pgfpathlineto{\pgfqpoint{4.142177in}{1.184024in}}%
\pgfpathlineto{\pgfqpoint{4.145349in}{1.184194in}}%
\pgfpathlineto{\pgfqpoint{4.148522in}{1.184472in}}%
\pgfpathlineto{\pgfqpoint{4.151694in}{1.184350in}}%
\pgfpathlineto{\pgfqpoint{4.154866in}{1.184303in}}%
\pgfpathlineto{\pgfqpoint{4.158038in}{1.184315in}}%
\pgfpathlineto{\pgfqpoint{4.161210in}{1.184790in}}%
\pgfpathlineto{\pgfqpoint{4.164382in}{1.185088in}}%
\pgfpathlineto{\pgfqpoint{4.167554in}{1.185196in}}%
\pgfpathlineto{\pgfqpoint{4.170726in}{1.185203in}}%
\pgfpathlineto{\pgfqpoint{4.173898in}{1.184889in}}%
\pgfpathlineto{\pgfqpoint{4.177070in}{1.184848in}}%
\pgfpathlineto{\pgfqpoint{4.180242in}{1.184731in}}%
\pgfpathlineto{\pgfqpoint{4.183414in}{1.184653in}}%
\pgfpathlineto{\pgfqpoint{4.186586in}{1.184361in}}%
\pgfpathlineto{\pgfqpoint{4.189758in}{1.184412in}}%
\pgfpathlineto{\pgfqpoint{4.192930in}{1.184589in}}%
\pgfpathlineto{\pgfqpoint{4.196102in}{1.184661in}}%
\pgfpathlineto{\pgfqpoint{4.199274in}{1.184648in}}%
\pgfpathlineto{\pgfqpoint{4.202446in}{1.184615in}}%
\pgfpathlineto{\pgfqpoint{4.205618in}{1.185056in}}%
\pgfpathlineto{\pgfqpoint{4.208790in}{1.184997in}}%
\pgfpathlineto{\pgfqpoint{4.211962in}{1.184360in}}%
\pgfpathlineto{\pgfqpoint{4.215134in}{1.183888in}}%
\pgfpathlineto{\pgfqpoint{4.218306in}{1.183715in}}%
\pgfpathlineto{\pgfqpoint{4.221478in}{1.183736in}}%
\pgfpathlineto{\pgfqpoint{4.224650in}{1.183966in}}%
\pgfpathlineto{\pgfqpoint{4.227823in}{1.183821in}}%
\pgfpathlineto{\pgfqpoint{4.230995in}{1.184354in}}%
\pgfpathlineto{\pgfqpoint{4.234167in}{1.184558in}}%
\pgfpathlineto{\pgfqpoint{4.237339in}{1.184469in}}%
\pgfpathlineto{\pgfqpoint{4.240511in}{1.184265in}}%
\pgfpathlineto{\pgfqpoint{4.243683in}{1.184066in}}%
\pgfpathlineto{\pgfqpoint{4.246855in}{1.184081in}}%
\pgfpathlineto{\pgfqpoint{4.250027in}{1.184261in}}%
\pgfpathlineto{\pgfqpoint{4.253199in}{1.184293in}}%
\pgfpathlineto{\pgfqpoint{4.256371in}{1.184442in}}%
\pgfpathlineto{\pgfqpoint{4.259543in}{1.184008in}}%
\pgfpathlineto{\pgfqpoint{4.262715in}{1.183799in}}%
\pgfpathlineto{\pgfqpoint{4.265887in}{1.183779in}}%
\pgfpathlineto{\pgfqpoint{4.269059in}{1.184100in}}%
\pgfpathlineto{\pgfqpoint{4.272231in}{1.184303in}}%
\pgfpathlineto{\pgfqpoint{4.275403in}{1.183880in}}%
\pgfpathlineto{\pgfqpoint{4.278575in}{1.183766in}}%
\pgfpathlineto{\pgfqpoint{4.281747in}{1.183602in}}%
\pgfpathlineto{\pgfqpoint{4.284919in}{1.183657in}}%
\pgfpathlineto{\pgfqpoint{4.288091in}{1.183609in}}%
\pgfpathlineto{\pgfqpoint{4.291263in}{1.183500in}}%
\pgfpathlineto{\pgfqpoint{4.294435in}{1.183459in}}%
\pgfpathlineto{\pgfqpoint{4.297607in}{1.183445in}}%
\pgfpathlineto{\pgfqpoint{4.300779in}{1.183470in}}%
\pgfpathlineto{\pgfqpoint{4.303952in}{1.183238in}}%
\pgfpathlineto{\pgfqpoint{4.307124in}{1.183031in}}%
\pgfpathlineto{\pgfqpoint{4.310296in}{1.183178in}}%
\pgfpathlineto{\pgfqpoint{4.313468in}{1.183171in}}%
\pgfpathlineto{\pgfqpoint{4.316640in}{1.182961in}}%
\pgfpathlineto{\pgfqpoint{4.319812in}{1.183085in}}%
\pgfpathlineto{\pgfqpoint{4.322984in}{1.183020in}}%
\pgfpathlineto{\pgfqpoint{4.326156in}{1.182739in}}%
\pgfpathlineto{\pgfqpoint{4.329328in}{1.182459in}}%
\pgfpathlineto{\pgfqpoint{4.332500in}{1.182700in}}%
\pgfpathlineto{\pgfqpoint{4.335672in}{1.182583in}}%
\pgfpathlineto{\pgfqpoint{4.338844in}{1.182673in}}%
\pgfpathlineto{\pgfqpoint{4.342016in}{1.182545in}}%
\pgfpathlineto{\pgfqpoint{4.345188in}{1.181835in}}%
\pgfpathlineto{\pgfqpoint{4.348360in}{1.182067in}}%
\pgfpathlineto{\pgfqpoint{4.351532in}{1.181813in}}%
\pgfpathlineto{\pgfqpoint{4.354704in}{1.181773in}}%
\pgfpathlineto{\pgfqpoint{4.357876in}{1.181501in}}%
\pgfpathlineto{\pgfqpoint{4.361048in}{1.181622in}}%
\pgfpathlineto{\pgfqpoint{4.364220in}{1.181371in}}%
\pgfpathlineto{\pgfqpoint{4.367392in}{1.181135in}}%
\pgfpathlineto{\pgfqpoint{4.370564in}{1.180454in}}%
\pgfpathlineto{\pgfqpoint{4.373736in}{1.180528in}}%
\pgfpathlineto{\pgfqpoint{4.376908in}{1.180419in}}%
\pgfpathlineto{\pgfqpoint{4.380080in}{1.180532in}}%
\pgfpathlineto{\pgfqpoint{4.383253in}{1.180757in}}%
\pgfpathlineto{\pgfqpoint{4.386425in}{1.180956in}}%
\pgfpathlineto{\pgfqpoint{4.389597in}{1.180803in}}%
\pgfpathlineto{\pgfqpoint{4.392769in}{1.180812in}}%
\pgfpathlineto{\pgfqpoint{4.395941in}{1.180849in}}%
\pgfpathlineto{\pgfqpoint{4.399113in}{1.181023in}}%
\pgfpathlineto{\pgfqpoint{4.402285in}{1.180935in}}%
\pgfpathlineto{\pgfqpoint{4.405457in}{1.180926in}}%
\pgfpathlineto{\pgfqpoint{4.408629in}{1.180682in}}%
\pgfpathlineto{\pgfqpoint{4.411801in}{1.180618in}}%
\pgfpathlineto{\pgfqpoint{4.414973in}{1.180508in}}%
\pgfpathlineto{\pgfqpoint{4.418145in}{1.180368in}}%
\pgfpathlineto{\pgfqpoint{4.421317in}{1.179997in}}%
\pgfpathlineto{\pgfqpoint{4.424489in}{1.179486in}}%
\pgfpathlineto{\pgfqpoint{4.427661in}{1.179271in}}%
\pgfpathlineto{\pgfqpoint{4.430833in}{1.178329in}}%
\pgfpathlineto{\pgfqpoint{4.434005in}{1.178979in}}%
\pgfpathlineto{\pgfqpoint{4.437177in}{1.178844in}}%
\pgfpathlineto{\pgfqpoint{4.440349in}{1.178963in}}%
\pgfpathlineto{\pgfqpoint{4.443521in}{1.178587in}}%
\pgfpathlineto{\pgfqpoint{4.446693in}{1.178331in}}%
\pgfpathlineto{\pgfqpoint{4.449865in}{1.177760in}}%
\pgfpathlineto{\pgfqpoint{4.453037in}{1.177756in}}%
\pgfpathlineto{\pgfqpoint{4.456209in}{1.177582in}}%
\pgfpathlineto{\pgfqpoint{4.459381in}{1.177437in}}%
\pgfpathlineto{\pgfqpoint{4.462554in}{1.177294in}}%
\pgfpathlineto{\pgfqpoint{4.465726in}{1.177356in}}%
\pgfpathlineto{\pgfqpoint{4.468898in}{1.177459in}}%
\pgfpathlineto{\pgfqpoint{4.472070in}{1.177423in}}%
\pgfpathlineto{\pgfqpoint{4.475242in}{1.177217in}}%
\pgfpathlineto{\pgfqpoint{4.478414in}{1.176933in}}%
\pgfpathlineto{\pgfqpoint{4.481586in}{1.176432in}}%
\pgfpathlineto{\pgfqpoint{4.484758in}{1.176422in}}%
\pgfpathlineto{\pgfqpoint{4.487930in}{1.176317in}}%
\pgfpathlineto{\pgfqpoint{4.491102in}{1.176170in}}%
\pgfpathlineto{\pgfqpoint{4.494274in}{1.175748in}}%
\pgfpathlineto{\pgfqpoint{4.497446in}{1.175380in}}%
\pgfpathlineto{\pgfqpoint{4.500618in}{1.175238in}}%
\pgfpathlineto{\pgfqpoint{4.503790in}{1.175185in}}%
\pgfpathlineto{\pgfqpoint{4.506962in}{1.175199in}}%
\pgfpathlineto{\pgfqpoint{4.510134in}{1.175038in}}%
\pgfpathlineto{\pgfqpoint{4.513306in}{1.175026in}}%
\pgfpathlineto{\pgfqpoint{4.516478in}{1.174961in}}%
\pgfpathlineto{\pgfqpoint{4.519650in}{1.175041in}}%
\pgfpathlineto{\pgfqpoint{4.522822in}{1.175106in}}%
\pgfpathlineto{\pgfqpoint{4.525994in}{1.175086in}}%
\pgfpathlineto{\pgfqpoint{4.529166in}{1.174879in}}%
\pgfpathlineto{\pgfqpoint{4.532338in}{1.175066in}}%
\pgfpathlineto{\pgfqpoint{4.535510in}{1.175056in}}%
\pgfpathlineto{\pgfqpoint{4.538683in}{1.174773in}}%
\pgfpathlineto{\pgfqpoint{4.541855in}{1.174413in}}%
\pgfpathlineto{\pgfqpoint{4.545027in}{1.174627in}}%
\pgfpathlineto{\pgfqpoint{4.548199in}{1.174399in}}%
\pgfpathlineto{\pgfqpoint{4.551371in}{1.174482in}}%
\pgfpathlineto{\pgfqpoint{4.554543in}{1.174384in}}%
\pgfpathlineto{\pgfqpoint{4.557715in}{1.174544in}}%
\pgfpathlineto{\pgfqpoint{4.560887in}{1.174754in}}%
\pgfpathlineto{\pgfqpoint{4.564059in}{1.174528in}}%
\pgfpathlineto{\pgfqpoint{4.567231in}{1.174031in}}%
\pgfpathlineto{\pgfqpoint{4.570403in}{1.173844in}}%
\pgfpathlineto{\pgfqpoint{4.573575in}{1.173667in}}%
\pgfpathlineto{\pgfqpoint{4.576747in}{1.173696in}}%
\pgfpathlineto{\pgfqpoint{4.579919in}{1.173378in}}%
\pgfpathlineto{\pgfqpoint{4.583091in}{1.173238in}}%
\pgfpathlineto{\pgfqpoint{4.586263in}{1.173076in}}%
\pgfpathlineto{\pgfqpoint{4.589435in}{1.173183in}}%
\pgfpathlineto{\pgfqpoint{4.592607in}{1.172954in}}%
\pgfpathlineto{\pgfqpoint{4.595779in}{1.172943in}}%
\pgfpathlineto{\pgfqpoint{4.598951in}{1.173016in}}%
\pgfpathlineto{\pgfqpoint{4.602123in}{1.173001in}}%
\pgfpathlineto{\pgfqpoint{4.605295in}{1.172841in}}%
\pgfpathlineto{\pgfqpoint{4.608467in}{1.173007in}}%
\pgfpathlineto{\pgfqpoint{4.611639in}{1.173262in}}%
\pgfpathlineto{\pgfqpoint{4.614811in}{1.173378in}}%
\pgfpathlineto{\pgfqpoint{4.617984in}{1.173367in}}%
\pgfpathlineto{\pgfqpoint{4.621156in}{1.173345in}}%
\pgfpathlineto{\pgfqpoint{4.624328in}{1.173274in}}%
\pgfpathlineto{\pgfqpoint{4.627500in}{1.172974in}}%
\pgfpathlineto{\pgfqpoint{4.630672in}{1.172652in}}%
\pgfpathlineto{\pgfqpoint{4.633844in}{1.172260in}}%
\pgfpathlineto{\pgfqpoint{4.637016in}{1.172135in}}%
\pgfpathlineto{\pgfqpoint{4.640188in}{1.171985in}}%
\pgfpathlineto{\pgfqpoint{4.643360in}{1.171695in}}%
\pgfpathlineto{\pgfqpoint{4.646532in}{1.171795in}}%
\pgfpathlineto{\pgfqpoint{4.649704in}{1.171535in}}%
\pgfpathlineto{\pgfqpoint{4.652876in}{1.171182in}}%
\pgfpathlineto{\pgfqpoint{4.656048in}{1.170720in}}%
\pgfpathlineto{\pgfqpoint{4.659220in}{1.171113in}}%
\pgfpathlineto{\pgfqpoint{4.662392in}{1.171014in}}%
\pgfpathlineto{\pgfqpoint{4.665564in}{1.170503in}}%
\pgfpathlineto{\pgfqpoint{4.668736in}{1.170139in}}%
\pgfpathlineto{\pgfqpoint{4.671908in}{1.170128in}}%
\pgfpathlineto{\pgfqpoint{4.675080in}{1.170080in}}%
\pgfpathlineto{\pgfqpoint{4.678252in}{1.169484in}}%
\pgfpathlineto{\pgfqpoint{4.681424in}{1.169654in}}%
\pgfpathlineto{\pgfqpoint{4.684596in}{1.169201in}}%
\pgfpathlineto{\pgfqpoint{4.687768in}{1.169457in}}%
\pgfpathlineto{\pgfqpoint{4.690940in}{1.169376in}}%
\pgfpathlineto{\pgfqpoint{4.694112in}{1.168650in}}%
\pgfpathlineto{\pgfqpoint{4.697285in}{1.167946in}}%
\pgfpathlineto{\pgfqpoint{4.700457in}{1.167790in}}%
\pgfpathlineto{\pgfqpoint{4.703629in}{1.167563in}}%
\pgfpathlineto{\pgfqpoint{4.706801in}{1.167018in}}%
\pgfpathlineto{\pgfqpoint{4.709973in}{1.167007in}}%
\pgfpathlineto{\pgfqpoint{4.713145in}{1.167000in}}%
\pgfpathlineto{\pgfqpoint{4.716317in}{1.166832in}}%
\pgfpathlineto{\pgfqpoint{4.719489in}{1.166266in}}%
\pgfpathlineto{\pgfqpoint{4.722661in}{1.166395in}}%
\pgfpathlineto{\pgfqpoint{4.725833in}{1.166183in}}%
\pgfpathlineto{\pgfqpoint{4.729005in}{1.165597in}}%
\pgfpathlineto{\pgfqpoint{4.732177in}{1.165440in}}%
\pgfpathlineto{\pgfqpoint{4.735349in}{1.165519in}}%
\pgfpathlineto{\pgfqpoint{4.738521in}{1.165110in}}%
\pgfpathlineto{\pgfqpoint{4.741693in}{1.164643in}}%
\pgfpathlineto{\pgfqpoint{4.744865in}{1.164775in}}%
\pgfpathlineto{\pgfqpoint{4.748037in}{1.164843in}}%
\pgfpathlineto{\pgfqpoint{4.751209in}{1.164771in}}%
\pgfpathlineto{\pgfqpoint{4.754381in}{1.164648in}}%
\pgfpathlineto{\pgfqpoint{4.757553in}{1.164638in}}%
\pgfpathlineto{\pgfqpoint{4.760725in}{1.164677in}}%
\pgfpathlineto{\pgfqpoint{4.763897in}{1.165006in}}%
\pgfpathlineto{\pgfqpoint{4.767069in}{1.164796in}}%
\pgfpathlineto{\pgfqpoint{4.770241in}{1.164340in}}%
\pgfpathlineto{\pgfqpoint{4.773414in}{1.164110in}}%
\pgfpathlineto{\pgfqpoint{4.776586in}{1.163792in}}%
\pgfpathlineto{\pgfqpoint{4.779758in}{1.163763in}}%
\pgfpathlineto{\pgfqpoint{4.782930in}{1.163544in}}%
\pgfpathlineto{\pgfqpoint{4.786102in}{1.163647in}}%
\pgfpathlineto{\pgfqpoint{4.789274in}{1.163583in}}%
\pgfpathlineto{\pgfqpoint{4.792446in}{1.163049in}}%
\pgfpathlineto{\pgfqpoint{4.795618in}{1.162330in}}%
\pgfpathlineto{\pgfqpoint{4.798790in}{1.162294in}}%
\pgfpathlineto{\pgfqpoint{4.801962in}{1.162211in}}%
\pgfpathlineto{\pgfqpoint{4.805134in}{1.162224in}}%
\pgfpathlineto{\pgfqpoint{4.808306in}{1.162332in}}%
\pgfpathlineto{\pgfqpoint{4.811478in}{1.162331in}}%
\pgfpathlineto{\pgfqpoint{4.814650in}{1.162442in}}%
\pgfpathlineto{\pgfqpoint{4.817822in}{1.162303in}}%
\pgfpathlineto{\pgfqpoint{4.820994in}{1.161880in}}%
\pgfpathlineto{\pgfqpoint{4.824166in}{1.161473in}}%
\pgfpathlineto{\pgfqpoint{4.827338in}{1.161238in}}%
\pgfpathlineto{\pgfqpoint{4.830510in}{1.161190in}}%
\pgfpathlineto{\pgfqpoint{4.833682in}{1.160854in}}%
\pgfpathlineto{\pgfqpoint{4.836854in}{1.160858in}}%
\pgfpathlineto{\pgfqpoint{4.840026in}{1.160644in}}%
\pgfpathlineto{\pgfqpoint{4.843198in}{1.160433in}}%
\pgfpathlineto{\pgfqpoint{4.846370in}{1.160352in}}%
\pgfpathlineto{\pgfqpoint{4.849542in}{1.160380in}}%
\pgfpathlineto{\pgfqpoint{4.852715in}{1.160220in}}%
\pgfpathlineto{\pgfqpoint{4.855887in}{1.160015in}}%
\pgfpathlineto{\pgfqpoint{4.859059in}{1.159995in}}%
\pgfpathlineto{\pgfqpoint{4.862231in}{1.160148in}}%
\pgfpathlineto{\pgfqpoint{4.865403in}{1.160025in}}%
\pgfpathlineto{\pgfqpoint{4.868575in}{1.159822in}}%
\pgfpathlineto{\pgfqpoint{4.871747in}{1.159825in}}%
\pgfpathlineto{\pgfqpoint{4.874919in}{1.159582in}}%
\pgfpathlineto{\pgfqpoint{4.878091in}{1.159796in}}%
\pgfpathlineto{\pgfqpoint{4.881263in}{1.159660in}}%
\pgfpathlineto{\pgfqpoint{4.884435in}{1.159236in}}%
\pgfpathlineto{\pgfqpoint{4.887607in}{1.159195in}}%
\pgfpathlineto{\pgfqpoint{4.890779in}{1.159274in}}%
\pgfpathlineto{\pgfqpoint{4.893951in}{1.158646in}}%
\pgfpathlineto{\pgfqpoint{4.897123in}{1.158331in}}%
\pgfpathlineto{\pgfqpoint{4.900295in}{1.158167in}}%
\pgfpathlineto{\pgfqpoint{4.903467in}{1.157972in}}%
\pgfpathlineto{\pgfqpoint{4.906639in}{1.157854in}}%
\pgfpathlineto{\pgfqpoint{4.909811in}{1.157644in}}%
\pgfpathlineto{\pgfqpoint{4.912983in}{1.157620in}}%
\pgfpathlineto{\pgfqpoint{4.916155in}{1.157831in}}%
\pgfpathlineto{\pgfqpoint{4.919327in}{1.157779in}}%
\pgfpathlineto{\pgfqpoint{4.922499in}{1.157609in}}%
\pgfpathlineto{\pgfqpoint{4.925671in}{1.157387in}}%
\pgfpathlineto{\pgfqpoint{4.928844in}{1.157524in}}%
\pgfpathlineto{\pgfqpoint{4.932016in}{1.157200in}}%
\pgfpathlineto{\pgfqpoint{4.935188in}{1.157090in}}%
\pgfpathlineto{\pgfqpoint{4.938360in}{1.157188in}}%
\pgfpathlineto{\pgfqpoint{4.941532in}{1.157183in}}%
\pgfpathlineto{\pgfqpoint{4.944704in}{1.157311in}}%
\pgfpathlineto{\pgfqpoint{4.947876in}{1.157087in}}%
\pgfpathlineto{\pgfqpoint{4.951048in}{1.156928in}}%
\pgfpathlineto{\pgfqpoint{4.954220in}{1.156901in}}%
\pgfpathlineto{\pgfqpoint{4.957392in}{1.156518in}}%
\pgfpathlineto{\pgfqpoint{4.960564in}{1.156572in}}%
\pgfpathlineto{\pgfqpoint{4.963736in}{1.156316in}}%
\pgfpathlineto{\pgfqpoint{4.966908in}{1.156409in}}%
\pgfpathlineto{\pgfqpoint{4.970080in}{1.156217in}}%
\pgfpathlineto{\pgfqpoint{4.973252in}{1.155635in}}%
\pgfpathlineto{\pgfqpoint{4.976424in}{1.155433in}}%
\pgfpathlineto{\pgfqpoint{4.979596in}{1.155855in}}%
\pgfpathlineto{\pgfqpoint{4.982768in}{1.155881in}}%
\pgfpathlineto{\pgfqpoint{4.985940in}{1.155586in}}%
\pgfpathlineto{\pgfqpoint{4.989112in}{1.155556in}}%
\pgfpathlineto{\pgfqpoint{4.992284in}{1.155320in}}%
\pgfpathlineto{\pgfqpoint{4.995456in}{1.155605in}}%
\pgfpathlineto{\pgfqpoint{4.998628in}{1.155839in}}%
\pgfpathlineto{\pgfqpoint{5.001800in}{1.155722in}}%
\pgfpathlineto{\pgfqpoint{5.004972in}{1.155388in}}%
\pgfpathlineto{\pgfqpoint{5.008145in}{1.154658in}}%
\pgfpathlineto{\pgfqpoint{5.011317in}{1.154394in}}%
\pgfpathlineto{\pgfqpoint{5.014489in}{1.154088in}}%
\pgfpathlineto{\pgfqpoint{5.017661in}{1.153573in}}%
\pgfpathlineto{\pgfqpoint{5.020833in}{1.153055in}}%
\pgfpathlineto{\pgfqpoint{5.024005in}{1.152657in}}%
\pgfpathlineto{\pgfqpoint{5.027177in}{1.152274in}}%
\pgfpathlineto{\pgfqpoint{5.030349in}{1.152495in}}%
\pgfpathlineto{\pgfqpoint{5.033521in}{1.152178in}}%
\pgfpathlineto{\pgfqpoint{5.036693in}{1.151911in}}%
\pgfpathlineto{\pgfqpoint{5.039865in}{1.151900in}}%
\pgfpathlineto{\pgfqpoint{5.043037in}{1.151441in}}%
\pgfpathlineto{\pgfqpoint{5.046209in}{1.151748in}}%
\pgfpathlineto{\pgfqpoint{5.049381in}{1.151658in}}%
\pgfpathlineto{\pgfqpoint{5.052553in}{1.151546in}}%
\pgfpathlineto{\pgfqpoint{5.055725in}{1.151777in}}%
\pgfpathlineto{\pgfqpoint{5.058897in}{1.151868in}}%
\pgfpathlineto{\pgfqpoint{5.062069in}{1.151954in}}%
\pgfpathlineto{\pgfqpoint{5.065241in}{1.152056in}}%
\pgfpathlineto{\pgfqpoint{5.068413in}{1.152388in}}%
\pgfpathlineto{\pgfqpoint{5.071585in}{1.152321in}}%
\pgfpathlineto{\pgfqpoint{5.074757in}{1.152195in}}%
\pgfpathlineto{\pgfqpoint{5.077929in}{1.152309in}}%
\pgfpathlineto{\pgfqpoint{5.081101in}{1.152383in}}%
\pgfpathlineto{\pgfqpoint{5.084273in}{1.151947in}}%
\pgfpathlineto{\pgfqpoint{5.087446in}{1.152227in}}%
\pgfpathlineto{\pgfqpoint{5.090618in}{1.151771in}}%
\pgfpathlineto{\pgfqpoint{5.093790in}{1.151781in}}%
\pgfpathlineto{\pgfqpoint{5.096962in}{1.152118in}}%
\pgfpathlineto{\pgfqpoint{5.100134in}{1.152040in}}%
\pgfpathlineto{\pgfqpoint{5.103306in}{1.152215in}}%
\pgfpathlineto{\pgfqpoint{5.106478in}{1.152629in}}%
\pgfpathlineto{\pgfqpoint{5.109650in}{1.152675in}}%
\pgfpathlineto{\pgfqpoint{5.112822in}{1.152636in}}%
\pgfpathlineto{\pgfqpoint{5.115994in}{1.152554in}}%
\pgfpathlineto{\pgfqpoint{5.119166in}{1.152374in}}%
\pgfpathlineto{\pgfqpoint{5.122338in}{1.152343in}}%
\pgfpathlineto{\pgfqpoint{5.125510in}{1.152111in}}%
\pgfpathlineto{\pgfqpoint{5.128682in}{1.152343in}}%
\pgfpathlineto{\pgfqpoint{5.131854in}{1.152708in}}%
\pgfpathlineto{\pgfqpoint{5.135026in}{1.152432in}}%
\pgfpathlineto{\pgfqpoint{5.138198in}{1.152642in}}%
\pgfpathlineto{\pgfqpoint{5.141370in}{1.152598in}}%
\pgfpathlineto{\pgfqpoint{5.144542in}{1.152561in}}%
\pgfpathlineto{\pgfqpoint{5.147714in}{1.152586in}}%
\pgfpathlineto{\pgfqpoint{5.150886in}{1.152634in}}%
\pgfpathlineto{\pgfqpoint{5.154058in}{1.152581in}}%
\pgfpathlineto{\pgfqpoint{5.157230in}{1.152504in}}%
\pgfpathlineto{\pgfqpoint{5.160402in}{1.152422in}}%
\pgfpathlineto{\pgfqpoint{5.163575in}{1.152196in}}%
\pgfpathlineto{\pgfqpoint{5.166747in}{1.151977in}}%
\pgfpathlineto{\pgfqpoint{5.169919in}{1.151487in}}%
\pgfpathlineto{\pgfqpoint{5.173091in}{1.151320in}}%
\pgfpathlineto{\pgfqpoint{5.176263in}{1.151496in}}%
\pgfpathlineto{\pgfqpoint{5.179435in}{1.151429in}}%
\pgfpathlineto{\pgfqpoint{5.182607in}{1.151628in}}%
\pgfpathlineto{\pgfqpoint{5.185779in}{1.151528in}}%
\pgfpathlineto{\pgfqpoint{5.188951in}{1.151391in}}%
\pgfpathlineto{\pgfqpoint{5.192123in}{1.151412in}}%
\pgfpathlineto{\pgfqpoint{5.195295in}{1.151132in}}%
\pgfpathlineto{\pgfqpoint{5.198467in}{1.150651in}}%
\pgfpathlineto{\pgfqpoint{5.201639in}{1.150764in}}%
\pgfpathlineto{\pgfqpoint{5.204811in}{1.150525in}}%
\pgfpathlineto{\pgfqpoint{5.207983in}{1.150251in}}%
\pgfpathlineto{\pgfqpoint{5.211155in}{1.149778in}}%
\pgfpathlineto{\pgfqpoint{5.214327in}{1.149685in}}%
\pgfpathlineto{\pgfqpoint{5.217499in}{1.149567in}}%
\pgfpathlineto{\pgfqpoint{5.220671in}{1.149821in}}%
\pgfpathlineto{\pgfqpoint{5.223843in}{1.149699in}}%
\pgfpathlineto{\pgfqpoint{5.227015in}{1.149525in}}%
\pgfpathlineto{\pgfqpoint{5.230187in}{1.149523in}}%
\pgfpathlineto{\pgfqpoint{5.233359in}{1.149263in}}%
\pgfpathlineto{\pgfqpoint{5.236531in}{1.149348in}}%
\pgfpathlineto{\pgfqpoint{5.239703in}{1.149024in}}%
\pgfpathlineto{\pgfqpoint{5.242876in}{1.148599in}}%
\pgfpathlineto{\pgfqpoint{5.246048in}{1.148279in}}%
\pgfpathlineto{\pgfqpoint{5.249220in}{1.147945in}}%
\pgfpathlineto{\pgfqpoint{5.252392in}{1.148009in}}%
\pgfpathlineto{\pgfqpoint{5.255564in}{1.147774in}}%
\pgfpathlineto{\pgfqpoint{5.258736in}{1.147556in}}%
\pgfpathlineto{\pgfqpoint{5.261908in}{1.147400in}}%
\pgfpathlineto{\pgfqpoint{5.265080in}{1.147600in}}%
\pgfpathlineto{\pgfqpoint{5.268252in}{1.147393in}}%
\pgfpathlineto{\pgfqpoint{5.271424in}{1.146690in}}%
\pgfpathlineto{\pgfqpoint{5.274596in}{1.146315in}}%
\pgfpathlineto{\pgfqpoint{5.277768in}{1.145452in}}%
\pgfpathlineto{\pgfqpoint{5.280940in}{1.145132in}}%
\pgfpathlineto{\pgfqpoint{5.284112in}{1.145109in}}%
\pgfpathlineto{\pgfqpoint{5.287284in}{1.145023in}}%
\pgfpathlineto{\pgfqpoint{5.290456in}{1.145263in}}%
\pgfpathlineto{\pgfqpoint{5.293628in}{1.145030in}}%
\pgfpathlineto{\pgfqpoint{5.296800in}{1.145028in}}%
\pgfpathlineto{\pgfqpoint{5.299972in}{1.145052in}}%
\pgfpathlineto{\pgfqpoint{5.303144in}{1.144682in}}%
\pgfpathlineto{\pgfqpoint{5.306316in}{1.144844in}}%
\pgfpathlineto{\pgfqpoint{5.309488in}{1.145096in}}%
\pgfpathlineto{\pgfqpoint{5.312660in}{1.145124in}}%
\pgfpathlineto{\pgfqpoint{5.315832in}{1.145065in}}%
\pgfpathlineto{\pgfqpoint{5.319004in}{1.144919in}}%
\pgfpathlineto{\pgfqpoint{5.322177in}{1.144754in}}%
\pgfpathlineto{\pgfqpoint{5.325349in}{1.144717in}}%
\pgfpathlineto{\pgfqpoint{5.328521in}{1.144759in}}%
\pgfpathlineto{\pgfqpoint{5.331693in}{1.144031in}}%
\pgfpathlineto{\pgfqpoint{5.334865in}{1.143843in}}%
\pgfpathlineto{\pgfqpoint{5.338037in}{1.143806in}}%
\pgfpathlineto{\pgfqpoint{5.341209in}{1.143830in}}%
\pgfpathlineto{\pgfqpoint{5.344381in}{1.143638in}}%
\pgfpathlineto{\pgfqpoint{5.347553in}{1.143740in}}%
\pgfpathlineto{\pgfqpoint{5.350725in}{1.143713in}}%
\pgfpathlineto{\pgfqpoint{5.353897in}{1.143533in}}%
\pgfpathlineto{\pgfqpoint{5.357069in}{1.143750in}}%
\pgfpathlineto{\pgfqpoint{5.360241in}{1.143000in}}%
\pgfpathlineto{\pgfqpoint{5.363413in}{1.143347in}}%
\pgfpathlineto{\pgfqpoint{5.366585in}{1.143613in}}%
\pgfpathlineto{\pgfqpoint{5.369757in}{1.143863in}}%
\pgfpathlineto{\pgfqpoint{5.372929in}{1.143869in}}%
\pgfpathlineto{\pgfqpoint{5.376101in}{1.144215in}}%
\pgfpathlineto{\pgfqpoint{5.379273in}{1.144025in}}%
\pgfpathlineto{\pgfqpoint{5.382445in}{1.143833in}}%
\pgfpathlineto{\pgfqpoint{5.385617in}{1.143833in}}%
\pgfpathlineto{\pgfqpoint{5.388789in}{1.143387in}}%
\pgfpathlineto{\pgfqpoint{5.391961in}{1.143397in}}%
\pgfpathlineto{\pgfqpoint{5.395133in}{1.143422in}}%
\pgfpathlineto{\pgfqpoint{5.398306in}{1.143524in}}%
\pgfpathlineto{\pgfqpoint{5.401478in}{1.143271in}}%
\pgfpathlineto{\pgfqpoint{5.404650in}{1.143136in}}%
\pgfpathlineto{\pgfqpoint{5.407822in}{1.142668in}}%
\pgfpathlineto{\pgfqpoint{5.410994in}{1.142173in}}%
\pgfpathlineto{\pgfqpoint{5.414166in}{1.142151in}}%
\pgfpathlineto{\pgfqpoint{5.417338in}{1.142153in}}%
\pgfpathlineto{\pgfqpoint{5.420510in}{1.142201in}}%
\pgfpathlineto{\pgfqpoint{5.423682in}{1.142208in}}%
\pgfpathlineto{\pgfqpoint{5.426854in}{1.142215in}}%
\pgfpathlineto{\pgfqpoint{5.430026in}{1.142223in}}%
\pgfpathlineto{\pgfqpoint{5.433198in}{1.142245in}}%
\pgfpathlineto{\pgfqpoint{5.436370in}{1.142242in}}%
\pgfpathlineto{\pgfqpoint{5.439542in}{1.142280in}}%
\pgfpathlineto{\pgfqpoint{5.442714in}{1.142317in}}%
\pgfpathlineto{\pgfqpoint{5.445886in}{1.142317in}}%
\pgfpathlineto{\pgfqpoint{5.449058in}{1.142328in}}%
\pgfpathlineto{\pgfqpoint{5.452230in}{1.142328in}}%
\pgfpathlineto{\pgfqpoint{5.455402in}{1.142334in}}%
\pgfpathlineto{\pgfqpoint{5.458574in}{1.142337in}}%
\pgfpathlineto{\pgfqpoint{5.461746in}{1.142381in}}%
\pgfpathlineto{\pgfqpoint{5.464918in}{1.142423in}}%
\pgfpathlineto{\pgfqpoint{5.468090in}{1.142441in}}%
\pgfpathlineto{\pgfqpoint{5.471262in}{1.142443in}}%
\pgfpathlineto{\pgfqpoint{5.474434in}{1.142471in}}%
\pgfpathlineto{\pgfqpoint{5.477607in}{1.142498in}}%
\pgfpathlineto{\pgfqpoint{5.480779in}{1.142503in}}%
\pgfpathlineto{\pgfqpoint{5.483951in}{1.142504in}}%
\pgfpathlineto{\pgfqpoint{5.487123in}{1.142513in}}%
\pgfpathlineto{\pgfqpoint{5.490295in}{1.142514in}}%
\pgfpathlineto{\pgfqpoint{5.493467in}{1.142511in}}%
\pgfpathlineto{\pgfqpoint{5.496639in}{1.142521in}}%
\pgfpathlineto{\pgfqpoint{5.499811in}{1.142520in}}%
\pgfpathlineto{\pgfqpoint{5.502983in}{1.142535in}}%
\pgfpathlineto{\pgfqpoint{5.506155in}{1.142576in}}%
\pgfpathlineto{\pgfqpoint{5.509327in}{1.142601in}}%
\pgfpathlineto{\pgfqpoint{5.512499in}{1.142594in}}%
\pgfpathlineto{\pgfqpoint{5.515671in}{1.142574in}}%
\pgfpathlineto{\pgfqpoint{5.518843in}{1.142647in}}%
\pgfpathlineto{\pgfqpoint{5.522015in}{1.142624in}}%
\pgfpathlineto{\pgfqpoint{5.525187in}{1.142629in}}%
\pgfpathlineto{\pgfqpoint{5.528359in}{1.142617in}}%
\pgfpathlineto{\pgfqpoint{5.531531in}{1.142631in}}%
\pgfpathlineto{\pgfqpoint{5.534703in}{1.142615in}}%
\pgfpathlineto{\pgfqpoint{5.537875in}{1.142630in}}%
\pgfpathlineto{\pgfqpoint{5.541047in}{1.142616in}}%
\pgfpathlineto{\pgfqpoint{5.544219in}{1.142632in}}%
\pgfpathlineto{\pgfqpoint{5.547391in}{1.142617in}}%
\pgfpathlineto{\pgfqpoint{5.550563in}{1.142633in}}%
\pgfpathlineto{\pgfqpoint{5.553735in}{1.142656in}}%
\pgfpathlineto{\pgfqpoint{5.556908in}{1.142668in}}%
\pgfpathlineto{\pgfqpoint{5.560080in}{1.142662in}}%
\pgfpathlineto{\pgfqpoint{5.563252in}{1.142651in}}%
\pgfpathlineto{\pgfqpoint{5.566424in}{1.142656in}}%
\pgfpathlineto{\pgfqpoint{5.569596in}{1.142670in}}%
\pgfpathlineto{\pgfqpoint{5.572768in}{1.142685in}}%
\pgfpathlineto{\pgfqpoint{5.575940in}{1.142733in}}%
\pgfpathlineto{\pgfqpoint{5.579112in}{1.142762in}}%
\pgfpathlineto{\pgfqpoint{5.582284in}{1.142744in}}%
\pgfpathlineto{\pgfqpoint{5.585456in}{1.142738in}}%
\pgfpathlineto{\pgfqpoint{5.588628in}{1.142751in}}%
\pgfpathlineto{\pgfqpoint{5.591800in}{1.142767in}}%
\pgfpathlineto{\pgfqpoint{5.594972in}{1.142760in}}%
\pgfpathlineto{\pgfqpoint{5.598144in}{1.142758in}}%
\pgfpathlineto{\pgfqpoint{5.601316in}{1.142737in}}%
\pgfpathlineto{\pgfqpoint{5.604488in}{1.142776in}}%
\pgfpathlineto{\pgfqpoint{5.607660in}{1.142829in}}%
\pgfpathlineto{\pgfqpoint{5.610832in}{1.142831in}}%
\pgfpathlineto{\pgfqpoint{5.614004in}{1.142861in}}%
\pgfpathlineto{\pgfqpoint{5.617176in}{1.142859in}}%
\pgfpathlineto{\pgfqpoint{5.620348in}{1.142858in}}%
\pgfpathlineto{\pgfqpoint{5.623520in}{1.142878in}}%
\pgfpathlineto{\pgfqpoint{5.626692in}{1.142892in}}%
\pgfpathlineto{\pgfqpoint{5.629864in}{1.142911in}}%
\pgfpathlineto{\pgfqpoint{5.633037in}{1.142911in}}%
\pgfpathlineto{\pgfqpoint{5.636209in}{1.142931in}}%
\pgfpathlineto{\pgfqpoint{5.639381in}{1.142919in}}%
\pgfpathlineto{\pgfqpoint{5.642553in}{1.142882in}}%
\pgfpathlineto{\pgfqpoint{5.645725in}{1.142921in}}%
\pgfpathlineto{\pgfqpoint{5.648897in}{1.142938in}}%
\pgfpathlineto{\pgfqpoint{5.652069in}{1.142928in}}%
\pgfpathlineto{\pgfqpoint{5.655241in}{1.142948in}}%
\pgfpathlineto{\pgfqpoint{5.658413in}{1.142974in}}%
\pgfpathlineto{\pgfqpoint{5.661585in}{1.142962in}}%
\pgfpathlineto{\pgfqpoint{5.664757in}{1.142950in}}%
\pgfpathlineto{\pgfqpoint{5.667929in}{1.142934in}}%
\pgfpathlineto{\pgfqpoint{5.671101in}{1.142972in}}%
\pgfpathlineto{\pgfqpoint{5.674273in}{1.142975in}}%
\pgfpathlineto{\pgfqpoint{5.677445in}{1.142990in}}%
\pgfpathlineto{\pgfqpoint{5.680617in}{1.143028in}}%
\pgfpathlineto{\pgfqpoint{5.683789in}{1.143042in}}%
\pgfpathlineto{\pgfqpoint{5.686961in}{1.143037in}}%
\pgfpathlineto{\pgfqpoint{5.690133in}{1.143042in}}%
\pgfpathlineto{\pgfqpoint{5.693305in}{1.143067in}}%
\pgfpathlineto{\pgfqpoint{5.696477in}{1.143066in}}%
\pgfpathlineto{\pgfqpoint{5.699649in}{1.143074in}}%
\pgfpathlineto{\pgfqpoint{5.702821in}{1.143084in}}%
\pgfpathlineto{\pgfqpoint{5.705993in}{1.143088in}}%
\pgfpathlineto{\pgfqpoint{5.709165in}{1.143092in}}%
\pgfpathlineto{\pgfqpoint{5.712338in}{1.143089in}}%
\pgfpathlineto{\pgfqpoint{5.715510in}{1.143101in}}%
\pgfpathlineto{\pgfqpoint{5.718682in}{1.143060in}}%
\pgfpathlineto{\pgfqpoint{5.721854in}{1.143076in}}%
\pgfpathlineto{\pgfqpoint{5.725026in}{1.143103in}}%
\pgfpathlineto{\pgfqpoint{5.728198in}{1.143158in}}%
\pgfpathlineto{\pgfqpoint{5.731370in}{1.143174in}}%
\pgfpathlineto{\pgfqpoint{5.734542in}{1.143159in}}%
\pgfpathlineto{\pgfqpoint{5.737714in}{1.143163in}}%
\pgfpathlineto{\pgfqpoint{5.740886in}{1.143193in}}%
\pgfpathlineto{\pgfqpoint{5.744058in}{1.143229in}}%
\pgfpathlineto{\pgfqpoint{5.747230in}{1.143214in}}%
\pgfpathlineto{\pgfqpoint{5.750402in}{1.143233in}}%
\pgfpathlineto{\pgfqpoint{5.753574in}{1.143247in}}%
\pgfpathlineto{\pgfqpoint{5.756746in}{1.143271in}}%
\pgfpathlineto{\pgfqpoint{5.759918in}{1.143264in}}%
\pgfpathlineto{\pgfqpoint{5.763090in}{1.143277in}}%
\pgfpathlineto{\pgfqpoint{5.766262in}{1.143311in}}%
\pgfpathlineto{\pgfqpoint{5.769434in}{1.143309in}}%
\pgfpathlineto{\pgfqpoint{5.772606in}{1.143304in}}%
\pgfpathlineto{\pgfqpoint{5.775778in}{1.143317in}}%
\pgfpathlineto{\pgfqpoint{5.778950in}{1.143353in}}%
\pgfpathlineto{\pgfqpoint{5.782122in}{1.143350in}}%
\pgfpathlineto{\pgfqpoint{5.785294in}{1.143354in}}%
\pgfpathlineto{\pgfqpoint{5.788466in}{1.143335in}}%
\pgfpathlineto{\pgfqpoint{5.791639in}{1.143336in}}%
\pgfpathlineto{\pgfqpoint{5.794811in}{1.143348in}}%
\pgfpathlineto{\pgfqpoint{5.797983in}{1.143344in}}%
\pgfpathlineto{\pgfqpoint{5.801155in}{1.143399in}}%
\pgfpathlineto{\pgfqpoint{5.804327in}{1.143391in}}%
\pgfpathlineto{\pgfqpoint{5.807499in}{1.143406in}}%
\pgfpathlineto{\pgfqpoint{5.810671in}{1.143375in}}%
\pgfpathlineto{\pgfqpoint{5.813843in}{1.143409in}}%
\pgfpathlineto{\pgfqpoint{5.817015in}{1.143429in}}%
\pgfpathlineto{\pgfqpoint{5.820187in}{1.143431in}}%
\pgfpathlineto{\pgfqpoint{5.823359in}{1.143428in}}%
\pgfpathlineto{\pgfqpoint{5.826531in}{1.143435in}}%
\pgfpathlineto{\pgfqpoint{5.829703in}{1.143442in}}%
\pgfpathlineto{\pgfqpoint{5.832875in}{1.143465in}}%
\pgfpathlineto{\pgfqpoint{5.836047in}{1.143445in}}%
\pgfpathlineto{\pgfqpoint{5.839219in}{1.143442in}}%
\pgfpathlineto{\pgfqpoint{5.842391in}{1.143427in}}%
\pgfpathlineto{\pgfqpoint{5.845563in}{1.143417in}}%
\pgfpathlineto{\pgfqpoint{5.848735in}{1.143395in}}%
\pgfpathlineto{\pgfqpoint{5.851907in}{1.143382in}}%
\pgfpathlineto{\pgfqpoint{5.855079in}{1.143377in}}%
\pgfpathlineto{\pgfqpoint{5.858251in}{1.143393in}}%
\pgfpathlineto{\pgfqpoint{5.861423in}{1.143437in}}%
\pgfpathlineto{\pgfqpoint{5.864595in}{1.143459in}}%
\pgfpathlineto{\pgfqpoint{5.867768in}{1.143460in}}%
\pgfpathlineto{\pgfqpoint{5.870940in}{1.143499in}}%
\pgfpathlineto{\pgfqpoint{5.874112in}{1.143533in}}%
\pgfpathlineto{\pgfqpoint{5.877284in}{1.143534in}}%
\pgfpathlineto{\pgfqpoint{5.880456in}{1.143496in}}%
\pgfpathlineto{\pgfqpoint{5.883628in}{1.143497in}}%
\pgfpathlineto{\pgfqpoint{5.886800in}{1.143455in}}%
\pgfpathlineto{\pgfqpoint{5.889972in}{1.143470in}}%
\pgfpathlineto{\pgfqpoint{5.893144in}{1.143479in}}%
\pgfpathlineto{\pgfqpoint{5.896316in}{1.143483in}}%
\pgfpathlineto{\pgfqpoint{5.899488in}{1.143520in}}%
\pgfpathlineto{\pgfqpoint{5.902660in}{1.143526in}}%
\pgfpathlineto{\pgfqpoint{5.905832in}{1.143540in}}%
\pgfpathlineto{\pgfqpoint{5.909004in}{1.143571in}}%
\pgfpathlineto{\pgfqpoint{5.912176in}{1.143582in}}%
\pgfpathlineto{\pgfqpoint{5.915348in}{1.143589in}}%
\pgfpathlineto{\pgfqpoint{5.918520in}{1.143622in}}%
\pgfpathlineto{\pgfqpoint{5.921692in}{1.143636in}}%
\pgfpathlineto{\pgfqpoint{5.924864in}{1.143625in}}%
\pgfpathlineto{\pgfqpoint{5.928036in}{1.143628in}}%
\pgfpathlineto{\pgfqpoint{5.931208in}{1.143631in}}%
\pgfpathlineto{\pgfqpoint{5.934380in}{1.143660in}}%
\pgfpathlineto{\pgfqpoint{5.937552in}{1.143677in}}%
\pgfpathlineto{\pgfqpoint{5.940724in}{1.143669in}}%
\pgfpathlineto{\pgfqpoint{5.943896in}{1.143703in}}%
\pgfpathlineto{\pgfqpoint{5.947069in}{1.143719in}}%
\pgfpathlineto{\pgfqpoint{5.950241in}{1.143720in}}%
\pgfpathlineto{\pgfqpoint{5.953413in}{1.143718in}}%
\pgfpathlineto{\pgfqpoint{5.956585in}{1.143722in}}%
\pgfpathlineto{\pgfqpoint{5.959757in}{1.143714in}}%
\pgfpathlineto{\pgfqpoint{5.962929in}{1.143709in}}%
\pgfpathlineto{\pgfqpoint{5.966101in}{1.143694in}}%
\pgfpathlineto{\pgfqpoint{5.969273in}{1.143719in}}%
\pgfpathlineto{\pgfqpoint{5.972445in}{1.143734in}}%
\pgfpathlineto{\pgfqpoint{5.975617in}{1.143719in}}%
\pgfpathlineto{\pgfqpoint{5.978789in}{1.143734in}}%
\pgfpathlineto{\pgfqpoint{5.981961in}{1.143734in}}%
\pgfpathlineto{\pgfqpoint{5.985133in}{1.143758in}}%
\pgfpathlineto{\pgfqpoint{5.988305in}{1.143730in}}%
\pgfpathlineto{\pgfqpoint{5.991477in}{1.143691in}}%
\pgfpathlineto{\pgfqpoint{5.994649in}{1.143722in}}%
\pgfpathlineto{\pgfqpoint{5.997821in}{1.143649in}}%
\pgfpathlineto{\pgfqpoint{6.000993in}{1.143642in}}%
\pgfpathlineto{\pgfqpoint{6.004165in}{1.143649in}}%
\pgfpathlineto{\pgfqpoint{6.007337in}{1.143692in}}%
\pgfpathlineto{\pgfqpoint{6.010509in}{1.143700in}}%
\pgfpathlineto{\pgfqpoint{6.013681in}{1.143744in}}%
\pgfpathlineto{\pgfqpoint{6.016853in}{1.143767in}}%
\pgfpathlineto{\pgfqpoint{6.020025in}{1.143759in}}%
\pgfpathlineto{\pgfqpoint{6.023197in}{1.143756in}}%
\pgfpathlineto{\pgfqpoint{6.026370in}{1.143779in}}%
\pgfpathlineto{\pgfqpoint{6.029542in}{1.143817in}}%
\pgfpathlineto{\pgfqpoint{6.032714in}{1.143840in}}%
\pgfpathlineto{\pgfqpoint{6.035886in}{1.143853in}}%
\pgfpathlineto{\pgfqpoint{6.039058in}{1.143872in}}%
\pgfpathlineto{\pgfqpoint{6.042230in}{1.143886in}}%
\pgfpathlineto{\pgfqpoint{6.045402in}{1.143886in}}%
\pgfpathlineto{\pgfqpoint{6.048574in}{1.143889in}}%
\pgfpathlineto{\pgfqpoint{6.051746in}{1.143898in}}%
\pgfpathlineto{\pgfqpoint{6.054918in}{1.143917in}}%
\pgfpathlineto{\pgfqpoint{6.058090in}{1.143932in}}%
\pgfpathlineto{\pgfqpoint{6.061262in}{1.143955in}}%
\pgfpathlineto{\pgfqpoint{6.064434in}{1.143934in}}%
\pgfpathlineto{\pgfqpoint{6.067606in}{1.143944in}}%
\pgfpathlineto{\pgfqpoint{6.070778in}{1.143963in}}%
\pgfpathlineto{\pgfqpoint{6.073950in}{1.143990in}}%
\pgfpathlineto{\pgfqpoint{6.077122in}{1.144014in}}%
\pgfpathlineto{\pgfqpoint{6.080294in}{1.144027in}}%
\pgfpathlineto{\pgfqpoint{6.083466in}{1.144050in}}%
\pgfpathlineto{\pgfqpoint{6.086638in}{1.144024in}}%
\pgfpathlineto{\pgfqpoint{6.089810in}{1.144008in}}%
\pgfpathlineto{\pgfqpoint{6.092982in}{1.143971in}}%
\pgfpathlineto{\pgfqpoint{6.096154in}{1.143988in}}%
\pgfpathlineto{\pgfqpoint{6.099326in}{1.143996in}}%
\pgfpathlineto{\pgfqpoint{6.102499in}{1.143975in}}%
\pgfpathlineto{\pgfqpoint{6.105671in}{1.143960in}}%
\pgfpathlineto{\pgfqpoint{6.108843in}{1.143940in}}%
\pgfpathlineto{\pgfqpoint{6.112015in}{1.143949in}}%
\pgfpathlineto{\pgfqpoint{6.115187in}{1.143977in}}%
\pgfpathlineto{\pgfqpoint{6.118359in}{1.144018in}}%
\pgfpathlineto{\pgfqpoint{6.121531in}{1.144011in}}%
\pgfpathlineto{\pgfqpoint{6.124703in}{1.144035in}}%
\pgfpathlineto{\pgfqpoint{6.127875in}{1.144054in}}%
\pgfpathlineto{\pgfqpoint{6.131047in}{1.144053in}}%
\pgfpathlineto{\pgfqpoint{6.134219in}{1.144055in}}%
\pgfpathlineto{\pgfqpoint{6.137391in}{1.144055in}}%
\pgfpathlineto{\pgfqpoint{6.140563in}{1.144058in}}%
\pgfpathlineto{\pgfqpoint{6.143735in}{1.144057in}}%
\pgfpathlineto{\pgfqpoint{6.146907in}{1.144034in}}%
\pgfpathlineto{\pgfqpoint{6.150079in}{1.144020in}}%
\pgfpathlineto{\pgfqpoint{6.153251in}{1.144010in}}%
\pgfpathlineto{\pgfqpoint{6.156423in}{1.144012in}}%
\pgfpathlineto{\pgfqpoint{6.159595in}{1.144012in}}%
\pgfpathlineto{\pgfqpoint{6.162767in}{1.144005in}}%
\pgfpathlineto{\pgfqpoint{6.165939in}{1.144001in}}%
\pgfpathlineto{\pgfqpoint{6.169111in}{1.144009in}}%
\pgfpathlineto{\pgfqpoint{6.172283in}{1.144052in}}%
\pgfpathlineto{\pgfqpoint{6.175455in}{1.144073in}}%
\pgfpathlineto{\pgfqpoint{6.178627in}{1.144046in}}%
\pgfpathlineto{\pgfqpoint{6.181800in}{1.144025in}}%
\pgfpathlineto{\pgfqpoint{6.184972in}{1.144009in}}%
\pgfpathlineto{\pgfqpoint{6.188144in}{1.144027in}}%
\pgfpathlineto{\pgfqpoint{6.191316in}{1.144031in}}%
\pgfpathlineto{\pgfqpoint{6.194488in}{1.144004in}}%
\pgfpathlineto{\pgfqpoint{6.197660in}{1.144048in}}%
\pgfpathlineto{\pgfqpoint{6.200832in}{1.144115in}}%
\pgfpathlineto{\pgfqpoint{6.204004in}{1.144104in}}%
\pgfpathlineto{\pgfqpoint{6.207176in}{1.144103in}}%
\pgfpathlineto{\pgfqpoint{6.210348in}{1.144165in}}%
\pgfpathlineto{\pgfqpoint{6.213520in}{1.144159in}}%
\pgfpathlineto{\pgfqpoint{6.216692in}{1.144164in}}%
\pgfpathlineto{\pgfqpoint{6.219864in}{1.144212in}}%
\pgfpathlineto{\pgfqpoint{6.223036in}{1.144206in}}%
\pgfpathlineto{\pgfqpoint{6.226208in}{1.144206in}}%
\pgfpathlineto{\pgfqpoint{6.229380in}{1.144234in}}%
\pgfpathlineto{\pgfqpoint{6.232552in}{1.144285in}}%
\pgfpathlineto{\pgfqpoint{6.235724in}{1.144312in}}%
\pgfpathlineto{\pgfqpoint{6.238896in}{1.144323in}}%
\pgfpathlineto{\pgfqpoint{6.242068in}{1.144295in}}%
\pgfpathlineto{\pgfqpoint{6.245240in}{1.144303in}}%
\pgfpathlineto{\pgfqpoint{6.248412in}{1.144291in}}%
\pgfpathlineto{\pgfqpoint{6.251584in}{1.144326in}}%
\pgfpathlineto{\pgfqpoint{6.254756in}{1.144343in}}%
\pgfpathlineto{\pgfqpoint{6.257928in}{1.144365in}}%
\pgfpathlineto{\pgfqpoint{6.261101in}{1.144392in}}%
\pgfpathlineto{\pgfqpoint{6.264273in}{1.144393in}}%
\pgfpathlineto{\pgfqpoint{6.267445in}{1.144401in}}%
\pgfpathlineto{\pgfqpoint{6.270617in}{1.144372in}}%
\pgfpathlineto{\pgfqpoint{6.273789in}{1.144351in}}%
\pgfpathlineto{\pgfqpoint{6.276961in}{1.144380in}}%
\pgfpathlineto{\pgfqpoint{6.280133in}{1.144377in}}%
\pgfpathlineto{\pgfqpoint{6.283305in}{1.144375in}}%
\pgfpathlineto{\pgfqpoint{6.286477in}{1.144349in}}%
\pgfpathlineto{\pgfqpoint{6.289649in}{1.144369in}}%
\pgfpathlineto{\pgfqpoint{6.292821in}{1.144367in}}%
\pgfpathlineto{\pgfqpoint{6.295993in}{1.144377in}}%
\pgfpathlineto{\pgfqpoint{6.299165in}{1.144394in}}%
\pgfpathlineto{\pgfqpoint{6.302337in}{1.144378in}}%
\pgfpathlineto{\pgfqpoint{6.305509in}{1.144370in}}%
\pgfpathlineto{\pgfqpoint{6.308681in}{1.144352in}}%
\pgfpathlineto{\pgfqpoint{6.311853in}{1.144381in}}%
\pgfpathlineto{\pgfqpoint{6.315025in}{1.144367in}}%
\pgfpathlineto{\pgfqpoint{6.318197in}{1.144428in}}%
\pgfpathlineto{\pgfqpoint{6.321369in}{1.144429in}}%
\pgfpathlineto{\pgfqpoint{6.324541in}{1.144455in}}%
\pgfpathlineto{\pgfqpoint{6.327713in}{1.144464in}}%
\pgfpathlineto{\pgfqpoint{6.330885in}{1.144498in}}%
\pgfpathlineto{\pgfqpoint{6.334057in}{1.144494in}}%
\pgfpathlineto{\pgfqpoint{6.337230in}{1.144552in}}%
\pgfpathlineto{\pgfqpoint{6.340402in}{1.144565in}}%
\pgfpathlineto{\pgfqpoint{6.343574in}{1.144593in}}%
\pgfpathlineto{\pgfqpoint{6.346746in}{1.144607in}}%
\pgfpathlineto{\pgfqpoint{6.349918in}{1.144603in}}%
\pgfpathlineto{\pgfqpoint{6.353090in}{1.144605in}}%
\pgfpathlineto{\pgfqpoint{6.356262in}{1.144617in}}%
\pgfpathlineto{\pgfqpoint{6.359434in}{1.144607in}}%
\pgfpathlineto{\pgfqpoint{6.362606in}{1.144603in}}%
\pgfpathlineto{\pgfqpoint{6.365778in}{1.144626in}}%
\pgfpathlineto{\pgfqpoint{6.368950in}{1.144643in}}%
\pgfpathlineto{\pgfqpoint{6.372122in}{1.144639in}}%
\pgfpathlineto{\pgfqpoint{6.375294in}{1.144638in}}%
\pgfpathlineto{\pgfqpoint{6.378466in}{1.144659in}}%
\pgfpathlineto{\pgfqpoint{6.381638in}{1.144709in}}%
\pgfpathlineto{\pgfqpoint{6.384810in}{1.144698in}}%
\pgfpathlineto{\pgfqpoint{6.387982in}{1.144704in}}%
\pgfpathlineto{\pgfqpoint{6.391154in}{1.144681in}}%
\pgfpathlineto{\pgfqpoint{6.394326in}{1.144688in}}%
\pgfpathlineto{\pgfqpoint{6.397498in}{1.144722in}}%
\pgfpathlineto{\pgfqpoint{6.400670in}{1.144712in}}%
\pgfpathlineto{\pgfqpoint{6.403842in}{1.144739in}}%
\pgfpathlineto{\pgfqpoint{6.407014in}{1.144784in}}%
\pgfpathlineto{\pgfqpoint{6.410186in}{1.144763in}}%
\pgfpathlineto{\pgfqpoint{6.413358in}{1.144730in}}%
\pgfpathlineto{\pgfqpoint{6.416531in}{1.144716in}}%
\pgfpathlineto{\pgfqpoint{6.419703in}{1.144708in}}%
\pgfpathlineto{\pgfqpoint{6.422875in}{1.144723in}}%
\pgfpathlineto{\pgfqpoint{6.426047in}{1.144719in}}%
\pgfpathlineto{\pgfqpoint{6.429219in}{1.144676in}}%
\pgfpathlineto{\pgfqpoint{6.432391in}{1.144670in}}%
\pgfpathlineto{\pgfqpoint{6.435563in}{1.144698in}}%
\pgfpathlineto{\pgfqpoint{6.438735in}{1.144734in}}%
\pgfpathlineto{\pgfqpoint{6.441907in}{1.144721in}}%
\pgfpathlineto{\pgfqpoint{6.445079in}{1.144729in}}%
\pgfpathlineto{\pgfqpoint{6.448251in}{1.144763in}}%
\pgfpathlineto{\pgfqpoint{6.451423in}{1.144710in}}%
\pgfpathlineto{\pgfqpoint{6.454595in}{1.144707in}}%
\pgfpathlineto{\pgfqpoint{6.457767in}{1.144742in}}%
\pgfpathlineto{\pgfqpoint{6.460939in}{1.144753in}}%
\pgfpathlineto{\pgfqpoint{6.464111in}{1.144762in}}%
\pgfpathlineto{\pgfqpoint{6.467283in}{1.144766in}}%
\pgfpathlineto{\pgfqpoint{6.470455in}{1.144764in}}%
\pgfpathlineto{\pgfqpoint{6.473627in}{1.144781in}}%
\pgfpathlineto{\pgfqpoint{6.476799in}{1.144791in}}%
\pgfpathlineto{\pgfqpoint{6.479971in}{1.144782in}}%
\pgfpathlineto{\pgfqpoint{6.483143in}{1.144827in}}%
\pgfpathlineto{\pgfqpoint{6.486315in}{1.144828in}}%
\pgfpathlineto{\pgfqpoint{6.489487in}{1.144896in}}%
\pgfpathlineto{\pgfqpoint{6.492659in}{1.144878in}}%
\pgfpathlineto{\pgfqpoint{6.495832in}{1.144868in}}%
\pgfpathlineto{\pgfqpoint{6.499004in}{1.144888in}}%
\pgfpathlineto{\pgfqpoint{6.502176in}{1.144871in}}%
\pgfpathlineto{\pgfqpoint{6.505348in}{1.144887in}}%
\pgfpathlineto{\pgfqpoint{6.508520in}{1.144858in}}%
\pgfpathlineto{\pgfqpoint{6.511692in}{1.144860in}}%
\pgfpathlineto{\pgfqpoint{6.514864in}{1.144851in}}%
\pgfpathlineto{\pgfqpoint{6.518036in}{1.144821in}}%
\pgfpathlineto{\pgfqpoint{6.521208in}{1.144840in}}%
\pgfpathlineto{\pgfqpoint{6.524380in}{1.144841in}}%
\pgfpathlineto{\pgfqpoint{6.527552in}{1.144827in}}%
\pgfpathlineto{\pgfqpoint{6.530724in}{1.144806in}}%
\pgfpathlineto{\pgfqpoint{6.533896in}{1.144816in}}%
\pgfpathlineto{\pgfqpoint{6.537068in}{1.144800in}}%
\pgfpathlineto{\pgfqpoint{6.540240in}{1.144782in}}%
\pgfpathlineto{\pgfqpoint{6.543412in}{1.144771in}}%
\pgfpathlineto{\pgfqpoint{6.546584in}{1.144736in}}%
\pgfpathlineto{\pgfqpoint{6.549756in}{1.144759in}}%
\pgfpathlineto{\pgfqpoint{6.552928in}{1.144753in}}%
\pgfpathlineto{\pgfqpoint{6.556100in}{1.144764in}}%
\pgfpathlineto{\pgfqpoint{6.559272in}{1.144759in}}%
\pgfpathlineto{\pgfqpoint{6.562444in}{1.144758in}}%
\pgfpathlineto{\pgfqpoint{6.565616in}{1.144787in}}%
\pgfpathlineto{\pgfqpoint{6.568788in}{1.144785in}}%
\pgfpathlineto{\pgfqpoint{6.571961in}{1.144797in}}%
\pgfpathlineto{\pgfqpoint{6.575133in}{1.144806in}}%
\pgfpathlineto{\pgfqpoint{6.578305in}{1.144786in}}%
\pgfpathlineto{\pgfqpoint{6.581477in}{1.144781in}}%
\pgfpathlineto{\pgfqpoint{6.584649in}{1.144817in}}%
\pgfpathlineto{\pgfqpoint{6.587821in}{1.144850in}}%
\pgfpathlineto{\pgfqpoint{6.590993in}{1.144850in}}%
\pgfpathlineto{\pgfqpoint{6.594165in}{1.144862in}}%
\pgfpathlineto{\pgfqpoint{6.597337in}{1.144851in}}%
\pgfpathlineto{\pgfqpoint{6.600509in}{1.144793in}}%
\pgfpathlineto{\pgfqpoint{6.603681in}{1.144762in}}%
\pgfpathlineto{\pgfqpoint{6.606853in}{1.144778in}}%
\pgfpathlineto{\pgfqpoint{6.610025in}{1.144788in}}%
\pgfpathlineto{\pgfqpoint{6.613197in}{1.144711in}}%
\pgfpathlineto{\pgfqpoint{6.616369in}{1.144680in}}%
\pgfpathlineto{\pgfqpoint{6.619541in}{1.144700in}}%
\pgfpathlineto{\pgfqpoint{6.622713in}{1.144747in}}%
\pgfpathlineto{\pgfqpoint{6.625885in}{1.144836in}}%
\pgfpathlineto{\pgfqpoint{6.629057in}{1.144823in}}%
\pgfpathlineto{\pgfqpoint{6.632229in}{1.144861in}}%
\pgfpathlineto{\pgfqpoint{6.635401in}{1.144957in}}%
\pgfpathlineto{\pgfqpoint{6.638573in}{1.144978in}}%
\pgfpathlineto{\pgfqpoint{6.641745in}{1.144972in}}%
\pgfpathlineto{\pgfqpoint{6.644917in}{1.144993in}}%
\pgfpathlineto{\pgfqpoint{6.648089in}{1.144959in}}%
\pgfpathlineto{\pgfqpoint{6.651262in}{1.145005in}}%
\pgfpathlineto{\pgfqpoint{6.654434in}{1.145009in}}%
\pgfpathlineto{\pgfqpoint{6.657606in}{1.145001in}}%
\pgfpathlineto{\pgfqpoint{6.660778in}{1.144964in}}%
\pgfpathlineto{\pgfqpoint{6.663950in}{1.144936in}}%
\pgfpathlineto{\pgfqpoint{6.667122in}{1.144947in}}%
\pgfpathlineto{\pgfqpoint{6.670294in}{1.144994in}}%
\pgfpathlineto{\pgfqpoint{6.673466in}{1.145025in}}%
\pgfpathlineto{\pgfqpoint{6.676638in}{1.145020in}}%
\pgfpathlineto{\pgfqpoint{6.679810in}{1.145034in}}%
\pgfpathlineto{\pgfqpoint{6.682982in}{1.145029in}}%
\pgfpathlineto{\pgfqpoint{6.686154in}{1.145045in}}%
\pgfpathlineto{\pgfqpoint{6.689326in}{1.145052in}}%
\pgfpathlineto{\pgfqpoint{6.692498in}{1.145145in}}%
\pgfpathlineto{\pgfqpoint{6.695670in}{1.145223in}}%
\pgfpathlineto{\pgfqpoint{6.698842in}{1.145220in}}%
\pgfpathlineto{\pgfqpoint{6.702014in}{1.145209in}}%
\pgfpathlineto{\pgfqpoint{6.705186in}{1.145198in}}%
\pgfpathlineto{\pgfqpoint{6.708358in}{1.145192in}}%
\pgfpathlineto{\pgfqpoint{6.711530in}{1.145194in}}%
\pgfpathlineto{\pgfqpoint{6.714702in}{1.145291in}}%
\pgfpathlineto{\pgfqpoint{6.717874in}{1.145294in}}%
\pgfpathlineto{\pgfqpoint{6.721046in}{1.145324in}}%
\pgfpathlineto{\pgfqpoint{6.724218in}{1.145318in}}%
\pgfpathlineto{\pgfqpoint{6.727391in}{1.145289in}}%
\pgfpathlineto{\pgfqpoint{6.730563in}{1.145307in}}%
\pgfpathlineto{\pgfqpoint{6.733735in}{1.145284in}}%
\pgfpathlineto{\pgfqpoint{6.736907in}{1.145319in}}%
\pgfpathlineto{\pgfqpoint{6.740079in}{1.145273in}}%
\pgfpathlineto{\pgfqpoint{6.743251in}{1.145246in}}%
\pgfpathlineto{\pgfqpoint{6.746423in}{1.145199in}}%
\pgfpathlineto{\pgfqpoint{6.749595in}{1.145175in}}%
\pgfpathlineto{\pgfqpoint{6.752767in}{1.145181in}}%
\pgfpathlineto{\pgfqpoint{6.755939in}{1.145207in}}%
\pgfpathlineto{\pgfqpoint{6.759111in}{1.145185in}}%
\pgfpathlineto{\pgfqpoint{6.762283in}{1.145205in}}%
\pgfpathlineto{\pgfqpoint{6.765455in}{1.145243in}}%
\pgfpathlineto{\pgfqpoint{6.768627in}{1.145250in}}%
\pgfpathlineto{\pgfqpoint{6.771799in}{1.145255in}}%
\pgfpathlineto{\pgfqpoint{6.774971in}{1.145253in}}%
\pgfpathlineto{\pgfqpoint{6.778143in}{1.145187in}}%
\pgfpathlineto{\pgfqpoint{6.781315in}{1.145207in}}%
\pgfpathlineto{\pgfqpoint{6.784487in}{1.145199in}}%
\pgfpathlineto{\pgfqpoint{6.787659in}{1.145220in}}%
\pgfpathlineto{\pgfqpoint{6.790831in}{1.145247in}}%
\pgfpathlineto{\pgfqpoint{6.794003in}{1.145255in}}%
\pgfpathlineto{\pgfqpoint{6.797175in}{1.145251in}}%
\pgfpathlineto{\pgfqpoint{6.800347in}{1.145246in}}%
\pgfpathlineto{\pgfqpoint{6.803519in}{1.145234in}}%
\pgfpathlineto{\pgfqpoint{6.806692in}{1.145224in}}%
\pgfpathlineto{\pgfqpoint{6.809864in}{1.145266in}}%
\pgfpathlineto{\pgfqpoint{6.813036in}{1.145295in}}%
\pgfpathlineto{\pgfqpoint{6.816208in}{1.145296in}}%
\pgfpathlineto{\pgfqpoint{6.819380in}{1.145263in}}%
\pgfpathlineto{\pgfqpoint{6.822552in}{1.145277in}}%
\pgfpathlineto{\pgfqpoint{6.825724in}{1.145277in}}%
\pgfpathlineto{\pgfqpoint{6.828896in}{1.145315in}}%
\pgfpathlineto{\pgfqpoint{6.832068in}{1.145326in}}%
\pgfpathlineto{\pgfqpoint{6.835240in}{1.145312in}}%
\pgfpathlineto{\pgfqpoint{6.838412in}{1.145348in}}%
\pgfpathlineto{\pgfqpoint{6.841584in}{1.145318in}}%
\pgfpathlineto{\pgfqpoint{6.844756in}{1.145375in}}%
\pgfpathlineto{\pgfqpoint{6.847928in}{1.145394in}}%
\pgfpathlineto{\pgfqpoint{6.851100in}{1.145401in}}%
\pgfpathlineto{\pgfqpoint{6.854272in}{1.145433in}}%
\pgfpathlineto{\pgfqpoint{6.857444in}{1.145430in}}%
\pgfpathlineto{\pgfqpoint{6.860616in}{1.145393in}}%
\pgfpathlineto{\pgfqpoint{6.863788in}{1.145415in}}%
\pgfpathlineto{\pgfqpoint{6.866960in}{1.145450in}}%
\pgfpathlineto{\pgfqpoint{6.870132in}{1.145504in}}%
\pgfpathlineto{\pgfqpoint{6.873304in}{1.145513in}}%
\pgfpathlineto{\pgfqpoint{6.876476in}{1.145523in}}%
\pgfpathlineto{\pgfqpoint{6.879648in}{1.145581in}}%
\pgfpathlineto{\pgfqpoint{6.882820in}{1.145569in}}%
\pgfpathlineto{\pgfqpoint{6.885993in}{1.145569in}}%
\pgfpathlineto{\pgfqpoint{6.889165in}{1.145580in}}%
\pgfpathlineto{\pgfqpoint{6.892337in}{1.145624in}}%
\pgfpathlineto{\pgfqpoint{6.895509in}{1.145590in}}%
\pgfpathlineto{\pgfqpoint{6.898681in}{1.145633in}}%
\pgfpathlineto{\pgfqpoint{6.901853in}{1.145668in}}%
\pgfpathlineto{\pgfqpoint{6.905025in}{1.145655in}}%
\pgfpathlineto{\pgfqpoint{6.908197in}{1.145722in}}%
\pgfpathlineto{\pgfqpoint{6.911369in}{1.145713in}}%
\pgfpathlineto{\pgfqpoint{6.914541in}{1.145723in}}%
\pgfpathlineto{\pgfqpoint{6.917713in}{1.145724in}}%
\pgfpathlineto{\pgfqpoint{6.920885in}{1.145755in}}%
\pgfpathlineto{\pgfqpoint{6.924057in}{1.145744in}}%
\pgfpathlineto{\pgfqpoint{6.927229in}{1.145761in}}%
\pgfpathlineto{\pgfqpoint{6.930401in}{1.145779in}}%
\pgfpathlineto{\pgfqpoint{6.933573in}{1.145792in}}%
\pgfpathlineto{\pgfqpoint{6.936745in}{1.145785in}}%
\pgfpathlineto{\pgfqpoint{6.939917in}{1.145801in}}%
\pgfpathlineto{\pgfqpoint{6.943089in}{1.145781in}}%
\pgfpathlineto{\pgfqpoint{6.946261in}{1.145794in}}%
\pgfpathlineto{\pgfqpoint{6.949433in}{1.145818in}}%
\pgfpathlineto{\pgfqpoint{6.952605in}{1.145755in}}%
\pgfpathlineto{\pgfqpoint{6.955777in}{1.145764in}}%
\pgfpathlineto{\pgfqpoint{6.958949in}{1.145805in}}%
\pgfpathlineto{\pgfqpoint{6.962122in}{1.145840in}}%
\pgfpathlineto{\pgfqpoint{6.965294in}{1.145860in}}%
\pgfpathlineto{\pgfqpoint{6.968466in}{1.145837in}}%
\pgfpathlineto{\pgfqpoint{6.971638in}{1.145854in}}%
\pgfpathlineto{\pgfqpoint{6.974810in}{1.145821in}}%
\pgfpathlineto{\pgfqpoint{6.977982in}{1.145845in}}%
\pgfpathlineto{\pgfqpoint{6.981154in}{1.145844in}}%
\pgfpathlineto{\pgfqpoint{6.984326in}{1.145872in}}%
\pgfpathlineto{\pgfqpoint{6.987498in}{1.145873in}}%
\pgfpathlineto{\pgfqpoint{6.990670in}{1.145857in}}%
\pgfpathlineto{\pgfqpoint{6.993842in}{1.145795in}}%
\pgfpathlineto{\pgfqpoint{6.997014in}{1.145765in}}%
\pgfpathlineto{\pgfqpoint{7.000186in}{1.145747in}}%
\pgfpathlineto{\pgfqpoint{7.003358in}{1.145698in}}%
\pgfpathlineto{\pgfqpoint{7.006530in}{1.145721in}}%
\pgfpathlineto{\pgfqpoint{7.009702in}{1.145696in}}%
\pgfpathlineto{\pgfqpoint{7.012874in}{1.145756in}}%
\pgfpathlineto{\pgfqpoint{7.016046in}{1.145707in}}%
\pgfpathlineto{\pgfqpoint{7.019218in}{1.145667in}}%
\pgfpathlineto{\pgfqpoint{7.022390in}{1.145685in}}%
\pgfpathlineto{\pgfqpoint{7.025562in}{1.145682in}}%
\pgfpathlineto{\pgfqpoint{7.028734in}{1.145705in}}%
\pgfpathlineto{\pgfqpoint{7.031906in}{1.145731in}}%
\pgfpathlineto{\pgfqpoint{7.035078in}{1.145743in}}%
\pgfpathlineto{\pgfqpoint{7.038250in}{1.145738in}}%
\pgfpathlineto{\pgfqpoint{7.041423in}{1.145735in}}%
\pgfpathlineto{\pgfqpoint{7.044595in}{1.145726in}}%
\pgfpathlineto{\pgfqpoint{7.047767in}{1.145735in}}%
\pgfpathlineto{\pgfqpoint{7.050939in}{1.145752in}}%
\pgfpathlineto{\pgfqpoint{7.054111in}{1.145724in}}%
\pgfpathlineto{\pgfqpoint{7.057283in}{1.145732in}}%
\pgfpathlineto{\pgfqpoint{7.060455in}{1.145721in}}%
\pgfpathlineto{\pgfqpoint{7.063627in}{1.145735in}}%
\pgfpathlineto{\pgfqpoint{7.066799in}{1.145741in}}%
\pgfpathlineto{\pgfqpoint{7.069971in}{1.145729in}}%
\pgfpathlineto{\pgfqpoint{7.073143in}{1.145743in}}%
\pgfpathlineto{\pgfqpoint{7.076315in}{1.145752in}}%
\pgfpathlineto{\pgfqpoint{7.079487in}{1.145762in}}%
\pgfpathlineto{\pgfqpoint{7.082659in}{1.145771in}}%
\pgfpathlineto{\pgfqpoint{7.085831in}{1.145778in}}%
\pgfpathlineto{\pgfqpoint{7.089003in}{1.145740in}}%
\pgfpathlineto{\pgfqpoint{7.092175in}{1.145742in}}%
\pgfpathlineto{\pgfqpoint{7.095347in}{1.145722in}}%
\pgfpathlineto{\pgfqpoint{7.098519in}{1.145663in}}%
\pgfpathlineto{\pgfqpoint{7.101691in}{1.145687in}}%
\pgfpathlineto{\pgfqpoint{7.104863in}{1.145704in}}%
\pgfpathlineto{\pgfqpoint{7.108035in}{1.145712in}}%
\pgfpathlineto{\pgfqpoint{7.111207in}{1.145698in}}%
\pgfpathlineto{\pgfqpoint{7.114379in}{1.145742in}}%
\pgfpathlineto{\pgfqpoint{7.117551in}{1.145779in}}%
\pgfpathlineto{\pgfqpoint{7.120724in}{1.145780in}}%
\pgfpathlineto{\pgfqpoint{7.123896in}{1.145849in}}%
\pgfpathlineto{\pgfqpoint{7.127068in}{1.145885in}}%
\pgfpathlineto{\pgfqpoint{7.130240in}{1.145903in}}%
\pgfpathlineto{\pgfqpoint{7.133412in}{1.145911in}}%
\pgfpathlineto{\pgfqpoint{7.136584in}{1.145873in}}%
\pgfpathlineto{\pgfqpoint{7.139756in}{1.145889in}}%
\pgfpathlineto{\pgfqpoint{7.142928in}{1.145865in}}%
\pgfpathlineto{\pgfqpoint{7.146100in}{1.145822in}}%
\pgfpathlineto{\pgfqpoint{7.149272in}{1.145831in}}%
\pgfpathlineto{\pgfqpoint{7.152444in}{1.145868in}}%
\pgfpathlineto{\pgfqpoint{7.155616in}{1.145816in}}%
\pgfpathlineto{\pgfqpoint{7.158788in}{1.145822in}}%
\pgfpathlineto{\pgfqpoint{7.161960in}{1.145781in}}%
\pgfpathlineto{\pgfqpoint{7.165132in}{1.145724in}}%
\pgfpathlineto{\pgfqpoint{7.168304in}{1.145729in}}%
\pgfpathlineto{\pgfqpoint{7.171476in}{1.145736in}}%
\pgfpathlineto{\pgfqpoint{7.174648in}{1.145799in}}%
\pgfpathlineto{\pgfqpoint{7.177820in}{1.145774in}}%
\pgfpathlineto{\pgfqpoint{7.180992in}{1.145765in}}%
\pgfpathlineto{\pgfqpoint{7.184164in}{1.145779in}}%
\pgfpathlineto{\pgfqpoint{7.187336in}{1.145801in}}%
\pgfpathlineto{\pgfqpoint{7.190508in}{1.145807in}}%
\pgfpathlineto{\pgfqpoint{7.193680in}{1.145860in}}%
\pgfpathlineto{\pgfqpoint{7.196853in}{1.145823in}}%
\pgfpathlineto{\pgfqpoint{7.200025in}{1.145855in}}%
\pgfpathlineto{\pgfqpoint{7.203197in}{1.145885in}}%
\pgfpathlineto{\pgfqpoint{7.206369in}{1.145947in}}%
\pgfpathlineto{\pgfqpoint{7.209541in}{1.145981in}}%
\pgfpathlineto{\pgfqpoint{7.212713in}{1.145955in}}%
\pgfpathlineto{\pgfqpoint{7.215885in}{1.145882in}}%
\pgfpathlineto{\pgfqpoint{7.219057in}{1.145860in}}%
\pgfpathlineto{\pgfqpoint{7.222229in}{1.145798in}}%
\pgfpathlineto{\pgfqpoint{7.225401in}{1.145805in}}%
\pgfpathlineto{\pgfqpoint{7.228573in}{1.145754in}}%
\pgfpathlineto{\pgfqpoint{7.231745in}{1.145760in}}%
\pgfpathlineto{\pgfqpoint{7.234917in}{1.145735in}}%
\pgfpathlineto{\pgfqpoint{7.238089in}{1.145769in}}%
\pgfpathlineto{\pgfqpoint{7.241261in}{1.145818in}}%
\pgfpathlineto{\pgfqpoint{7.244433in}{1.145787in}}%
\pgfpathlineto{\pgfqpoint{7.247605in}{1.145741in}}%
\pgfpathlineto{\pgfqpoint{7.250777in}{1.145749in}}%
\pgfpathlineto{\pgfqpoint{7.253949in}{1.145770in}}%
\pgfpathlineto{\pgfqpoint{7.257121in}{1.145770in}}%
\pgfpathlineto{\pgfqpoint{7.260293in}{1.145778in}}%
\pgfpathlineto{\pgfqpoint{7.263465in}{1.145807in}}%
\pgfpathlineto{\pgfqpoint{7.266637in}{1.145876in}}%
\pgfpathlineto{\pgfqpoint{7.269809in}{1.145915in}}%
\pgfpathlineto{\pgfqpoint{7.272981in}{1.145965in}}%
\pgfpathlineto{\pgfqpoint{7.276154in}{1.145956in}}%
\pgfpathlineto{\pgfqpoint{7.279326in}{1.145947in}}%
\pgfpathlineto{\pgfqpoint{7.282498in}{1.145966in}}%
\pgfpathlineto{\pgfqpoint{7.285670in}{1.145975in}}%
\pgfpathlineto{\pgfqpoint{7.288842in}{1.146068in}}%
\pgfpathlineto{\pgfqpoint{7.292014in}{1.146075in}}%
\pgfpathlineto{\pgfqpoint{7.295186in}{1.146099in}}%
\pgfpathlineto{\pgfqpoint{7.298358in}{1.146112in}}%
\pgfpathlineto{\pgfqpoint{7.301530in}{1.146112in}}%
\pgfpathlineto{\pgfqpoint{7.304702in}{1.146105in}}%
\pgfpathlineto{\pgfqpoint{7.307874in}{1.146058in}}%
\pgfpathlineto{\pgfqpoint{7.311046in}{1.146113in}}%
\pgfpathlineto{\pgfqpoint{7.314218in}{1.146084in}}%
\pgfpathlineto{\pgfqpoint{7.317390in}{1.146134in}}%
\pgfpathlineto{\pgfqpoint{7.320562in}{1.146153in}}%
\pgfpathlineto{\pgfqpoint{7.323734in}{1.146156in}}%
\pgfpathlineto{\pgfqpoint{7.326906in}{1.146150in}}%
\pgfpathlineto{\pgfqpoint{7.330078in}{1.146107in}}%
\pgfpathlineto{\pgfqpoint{7.333250in}{1.146057in}}%
\pgfpathlineto{\pgfqpoint{7.336422in}{1.146056in}}%
\pgfpathlineto{\pgfqpoint{7.339594in}{1.146083in}}%
\pgfpathlineto{\pgfqpoint{7.342766in}{1.146052in}}%
\pgfpathlineto{\pgfqpoint{7.345938in}{1.146142in}}%
\pgfpathlineto{\pgfqpoint{7.349110in}{1.146155in}}%
\pgfpathlineto{\pgfqpoint{7.352282in}{1.146147in}}%
\pgfpathlineto{\pgfqpoint{7.355455in}{1.146174in}}%
\pgfpathlineto{\pgfqpoint{7.358627in}{1.146152in}}%
\pgfpathlineto{\pgfqpoint{7.361799in}{1.146176in}}%
\pgfpathlineto{\pgfqpoint{7.364971in}{1.146258in}}%
\pgfpathlineto{\pgfqpoint{7.368143in}{1.146305in}}%
\pgfpathlineto{\pgfqpoint{7.371315in}{1.146334in}}%
\pgfpathlineto{\pgfqpoint{7.374487in}{1.146399in}}%
\pgfpathlineto{\pgfqpoint{7.377659in}{1.146419in}}%
\pgfpathlineto{\pgfqpoint{7.380831in}{1.146424in}}%
\pgfpathlineto{\pgfqpoint{7.384003in}{1.146365in}}%
\pgfpathlineto{\pgfqpoint{7.387175in}{1.146321in}}%
\pgfpathlineto{\pgfqpoint{7.390347in}{1.146319in}}%
\pgfpathlineto{\pgfqpoint{7.393519in}{1.146292in}}%
\pgfpathlineto{\pgfqpoint{7.396691in}{1.146313in}}%
\pgfpathlineto{\pgfqpoint{7.399863in}{1.146327in}}%
\pgfpathlineto{\pgfqpoint{7.403035in}{1.146286in}}%
\pgfpathlineto{\pgfqpoint{7.406207in}{1.146307in}}%
\pgfpathlineto{\pgfqpoint{7.409379in}{1.146302in}}%
\pgfpathlineto{\pgfqpoint{7.412551in}{1.146255in}}%
\pgfpathlineto{\pgfqpoint{7.415723in}{1.146264in}}%
\pgfpathlineto{\pgfqpoint{7.418895in}{1.146289in}}%
\pgfpathlineto{\pgfqpoint{7.422067in}{1.146303in}}%
\pgfpathlineto{\pgfqpoint{7.425239in}{1.146348in}}%
\pgfpathlineto{\pgfqpoint{7.428411in}{1.146329in}}%
\pgfpathlineto{\pgfqpoint{7.431584in}{1.146310in}}%
\pgfpathlineto{\pgfqpoint{7.434756in}{1.146332in}}%
\pgfpathlineto{\pgfqpoint{7.437928in}{1.146307in}}%
\pgfpathlineto{\pgfqpoint{7.441100in}{1.146303in}}%
\pgfpathlineto{\pgfqpoint{7.444272in}{1.146307in}}%
\pgfpathlineto{\pgfqpoint{7.447444in}{1.146273in}}%
\pgfpathlineto{\pgfqpoint{7.450616in}{1.146297in}}%
\pgfpathlineto{\pgfqpoint{7.453788in}{1.146309in}}%
\pgfpathlineto{\pgfqpoint{7.456960in}{1.146286in}}%
\pgfpathlineto{\pgfqpoint{7.460132in}{1.146322in}}%
\pgfpathlineto{\pgfqpoint{7.463304in}{1.146369in}}%
\pgfpathlineto{\pgfqpoint{7.466476in}{1.146389in}}%
\pgfpathlineto{\pgfqpoint{7.469648in}{1.146353in}}%
\pgfpathlineto{\pgfqpoint{7.472820in}{1.146346in}}%
\pgfpathlineto{\pgfqpoint{7.475992in}{1.146356in}}%
\pgfpathlineto{\pgfqpoint{7.479164in}{1.146366in}}%
\pgfpathlineto{\pgfqpoint{7.482336in}{1.146360in}}%
\pgfpathlineto{\pgfqpoint{7.485508in}{1.146343in}}%
\pgfpathlineto{\pgfqpoint{7.488680in}{1.146330in}}%
\pgfpathlineto{\pgfqpoint{7.491852in}{1.146326in}}%
\pgfpathlineto{\pgfqpoint{7.495024in}{1.146347in}}%
\pgfpathlineto{\pgfqpoint{7.498196in}{1.146348in}}%
\pgfpathlineto{\pgfqpoint{7.501368in}{1.146349in}}%
\pgfpathlineto{\pgfqpoint{7.504540in}{1.146425in}}%
\pgfpathlineto{\pgfqpoint{7.507712in}{1.146434in}}%
\pgfpathlineto{\pgfqpoint{7.510885in}{1.146391in}}%
\pgfpathlineto{\pgfqpoint{7.514057in}{1.146441in}}%
\pgfpathlineto{\pgfqpoint{7.517229in}{1.146434in}}%
\pgfpathlineto{\pgfqpoint{7.520401in}{1.146350in}}%
\pgfpathlineto{\pgfqpoint{7.523573in}{1.146324in}}%
\pgfpathlineto{\pgfqpoint{7.526745in}{1.146272in}}%
\pgfpathlineto{\pgfqpoint{7.529917in}{1.146283in}}%
\pgfpathlineto{\pgfqpoint{7.533089in}{1.146264in}}%
\pgfpathlineto{\pgfqpoint{7.536261in}{1.146304in}}%
\pgfpathlineto{\pgfqpoint{7.539433in}{1.146349in}}%
\pgfpathlineto{\pgfqpoint{7.542605in}{1.146362in}}%
\pgfpathlineto{\pgfqpoint{7.545777in}{1.146400in}}%
\pgfpathlineto{\pgfqpoint{7.548949in}{1.146403in}}%
\pgfpathlineto{\pgfqpoint{7.552121in}{1.146411in}}%
\pgfpathlineto{\pgfqpoint{7.555293in}{1.146457in}}%
\pgfpathlineto{\pgfqpoint{7.558465in}{1.146500in}}%
\pgfpathlineto{\pgfqpoint{7.561637in}{1.146422in}}%
\pgfpathlineto{\pgfqpoint{7.564809in}{1.146426in}}%
\pgfpathlineto{\pgfqpoint{7.567981in}{1.146447in}}%
\pgfpathlineto{\pgfqpoint{7.571153in}{1.146476in}}%
\pgfpathlineto{\pgfqpoint{7.574325in}{1.146518in}}%
\pgfpathlineto{\pgfqpoint{7.577497in}{1.146580in}}%
\pgfpathlineto{\pgfqpoint{7.580669in}{1.146632in}}%
\pgfpathlineto{\pgfqpoint{7.583841in}{1.146643in}}%
\pgfpathlineto{\pgfqpoint{7.587013in}{1.146612in}}%
\pgfpathlineto{\pgfqpoint{7.590186in}{1.146681in}}%
\pgfpathlineto{\pgfqpoint{7.593358in}{1.146675in}}%
\pgfpathlineto{\pgfqpoint{7.596530in}{1.146657in}}%
\pgfpathlineto{\pgfqpoint{7.599702in}{1.146618in}}%
\pgfpathlineto{\pgfqpoint{7.602874in}{1.146626in}}%
\pgfpathlineto{\pgfqpoint{7.606046in}{1.146679in}}%
\pgfpathlineto{\pgfqpoint{7.609218in}{1.146731in}}%
\pgfpathlineto{\pgfqpoint{7.612390in}{1.146763in}}%
\pgfpathlineto{\pgfqpoint{7.615562in}{1.146709in}}%
\pgfpathlineto{\pgfqpoint{7.618734in}{1.146645in}}%
\pgfpathlineto{\pgfqpoint{7.621906in}{1.146634in}}%
\pgfpathlineto{\pgfqpoint{7.625078in}{1.146687in}}%
\pgfpathlineto{\pgfqpoint{7.628250in}{1.146682in}}%
\pgfpathlineto{\pgfqpoint{7.631422in}{1.146667in}}%
\pgfpathlineto{\pgfqpoint{7.634594in}{1.146625in}}%
\pgfpathlineto{\pgfqpoint{7.637766in}{1.146593in}}%
\pgfpathlineto{\pgfqpoint{7.640938in}{1.146636in}}%
\pgfpathlineto{\pgfqpoint{7.644110in}{1.146628in}}%
\pgfpathlineto{\pgfqpoint{7.647282in}{1.146589in}}%
\pgfpathlineto{\pgfqpoint{7.650454in}{1.146573in}}%
\pgfpathlineto{\pgfqpoint{7.653626in}{1.146536in}}%
\pgfpathlineto{\pgfqpoint{7.656798in}{1.146630in}}%
\pgfpathlineto{\pgfqpoint{7.659970in}{1.146642in}}%
\pgfpathlineto{\pgfqpoint{7.663142in}{1.146640in}}%
\pgfpathlineto{\pgfqpoint{7.666315in}{1.146597in}}%
\pgfpathlineto{\pgfqpoint{7.669487in}{1.146657in}}%
\pgfpathlineto{\pgfqpoint{7.672659in}{1.146705in}}%
\pgfpathlineto{\pgfqpoint{7.675831in}{1.146737in}}%
\pgfpathlineto{\pgfqpoint{7.679003in}{1.146776in}}%
\pgfpathlineto{\pgfqpoint{7.682175in}{1.146794in}}%
\pgfpathlineto{\pgfqpoint{7.685347in}{1.146810in}}%
\pgfpathlineto{\pgfqpoint{7.688519in}{1.146826in}}%
\pgfpathlineto{\pgfqpoint{7.691691in}{1.146823in}}%
\pgfpathlineto{\pgfqpoint{7.694863in}{1.146836in}}%
\pgfpathlineto{\pgfqpoint{7.698035in}{1.146895in}}%
\pgfpathlineto{\pgfqpoint{7.701207in}{1.146965in}}%
\pgfpathlineto{\pgfqpoint{7.704379in}{1.146972in}}%
\pgfpathlineto{\pgfqpoint{7.707551in}{1.146962in}}%
\pgfpathlineto{\pgfqpoint{7.710723in}{1.146899in}}%
\pgfpathlineto{\pgfqpoint{7.713895in}{1.146957in}}%
\pgfpathlineto{\pgfqpoint{7.717067in}{1.146956in}}%
\pgfpathlineto{\pgfqpoint{7.720239in}{1.146909in}}%
\pgfpathlineto{\pgfqpoint{7.723411in}{1.146974in}}%
\pgfpathlineto{\pgfqpoint{7.726583in}{1.146969in}}%
\pgfpathlineto{\pgfqpoint{7.729755in}{1.146953in}}%
\pgfpathlineto{\pgfqpoint{7.732927in}{1.146994in}}%
\pgfpathlineto{\pgfqpoint{7.736099in}{1.146941in}}%
\pgfpathlineto{\pgfqpoint{7.739271in}{1.146901in}}%
\pgfpathlineto{\pgfqpoint{7.742443in}{1.146877in}}%
\pgfpathlineto{\pgfqpoint{7.745616in}{1.146942in}}%
\pgfpathlineto{\pgfqpoint{7.748788in}{1.146926in}}%
\pgfpathlineto{\pgfqpoint{7.751960in}{1.146804in}}%
\pgfpathlineto{\pgfqpoint{7.755132in}{1.146781in}}%
\pgfpathlineto{\pgfqpoint{7.758304in}{1.146801in}}%
\pgfpathlineto{\pgfqpoint{7.761476in}{1.146826in}}%
\pgfpathlineto{\pgfqpoint{7.764648in}{1.146822in}}%
\pgfpathlineto{\pgfqpoint{7.767820in}{1.146834in}}%
\pgfpathlineto{\pgfqpoint{7.770992in}{1.146834in}}%
\pgfpathlineto{\pgfqpoint{7.774164in}{1.146836in}}%
\pgfpathlineto{\pgfqpoint{7.777336in}{1.146850in}}%
\pgfpathlineto{\pgfqpoint{7.780508in}{1.146885in}}%
\pgfpathlineto{\pgfqpoint{7.783680in}{1.146933in}}%
\pgfpathlineto{\pgfqpoint{7.786852in}{1.146921in}}%
\pgfpathlineto{\pgfqpoint{7.790024in}{1.147236in}}%
\pgfpathlineto{\pgfqpoint{7.793196in}{1.147532in}}%
\pgfpathlineto{\pgfqpoint{7.796368in}{1.147833in}}%
\pgfpathlineto{\pgfqpoint{7.799540in}{1.148136in}}%
\pgfpathlineto{\pgfqpoint{7.802712in}{1.148441in}}%
\pgfpathlineto{\pgfqpoint{7.805884in}{1.148747in}}%
\pgfpathlineto{\pgfqpoint{7.809056in}{1.149045in}}%
\pgfpathlineto{\pgfqpoint{7.812228in}{1.149342in}}%
\pgfpathlineto{\pgfqpoint{7.815400in}{1.149641in}}%
\pgfpathlineto{\pgfqpoint{7.818572in}{1.149939in}}%
\pgfpathlineto{\pgfqpoint{7.821744in}{1.150232in}}%
\pgfpathlineto{\pgfqpoint{7.824917in}{1.150528in}}%
\pgfpathlineto{\pgfqpoint{7.828089in}{1.150831in}}%
\pgfpathlineto{\pgfqpoint{7.831261in}{1.151122in}}%
\pgfpathlineto{\pgfqpoint{7.834433in}{1.151411in}}%
\pgfpathlineto{\pgfqpoint{7.837605in}{1.151710in}}%
\pgfpathlineto{\pgfqpoint{7.840777in}{1.152006in}}%
\pgfpathlineto{\pgfqpoint{7.843949in}{1.152303in}}%
\pgfpathlineto{\pgfqpoint{7.847121in}{1.152596in}}%
\pgfpathlineto{\pgfqpoint{7.850293in}{1.152891in}}%
\pgfpathlineto{\pgfqpoint{7.853465in}{1.153185in}}%
\pgfpathlineto{\pgfqpoint{7.856637in}{1.153480in}}%
\pgfpathlineto{\pgfqpoint{7.859809in}{1.153775in}}%
\pgfpathlineto{\pgfqpoint{7.862981in}{1.154070in}}%
\pgfpathlineto{\pgfqpoint{7.866153in}{1.154369in}}%
\pgfpathlineto{\pgfqpoint{7.869325in}{1.154659in}}%
\pgfpathlineto{\pgfqpoint{7.872497in}{1.154955in}}%
\pgfpathlineto{\pgfqpoint{7.875669in}{1.155245in}}%
\pgfpathlineto{\pgfqpoint{7.878841in}{1.155538in}}%
\pgfpathlineto{\pgfqpoint{7.882013in}{1.155829in}}%
\pgfpathlineto{\pgfqpoint{7.885185in}{1.156111in}}%
\pgfpathlineto{\pgfqpoint{7.888357in}{1.157085in}}%
\pgfpathlineto{\pgfqpoint{7.891529in}{1.159938in}}%
\pgfpathlineto{\pgfqpoint{7.894701in}{1.162936in}}%
\pgfpathlineto{\pgfqpoint{7.897873in}{1.167541in}}%
\pgfpathlineto{\pgfqpoint{7.901046in}{1.170472in}}%
\pgfpathlineto{\pgfqpoint{7.904218in}{1.173441in}}%
\pgfpathlineto{\pgfqpoint{7.907390in}{1.176342in}}%
\pgfpathlineto{\pgfqpoint{7.910562in}{1.179360in}}%
\pgfpathlineto{\pgfqpoint{7.913734in}{1.182326in}}%
\pgfpathlineto{\pgfqpoint{7.916906in}{1.185341in}}%
\pgfpathlineto{\pgfqpoint{7.920078in}{1.188346in}}%
\pgfpathlineto{\pgfqpoint{7.923250in}{1.191375in}}%
\pgfpathlineto{\pgfqpoint{7.926422in}{1.194402in}}%
\pgfpathlineto{\pgfqpoint{7.929594in}{1.197363in}}%
\pgfpathlineto{\pgfqpoint{7.932766in}{1.200369in}}%
\pgfpathlineto{\pgfqpoint{7.935938in}{1.203374in}}%
\pgfpathlineto{\pgfqpoint{7.939110in}{1.206374in}}%
\pgfpathlineto{\pgfqpoint{7.942282in}{1.209388in}}%
\pgfpathlineto{\pgfqpoint{7.945454in}{1.212397in}}%
\pgfpathlineto{\pgfqpoint{7.948626in}{1.215382in}}%
\pgfpathlineto{\pgfqpoint{7.951798in}{1.218411in}}%
\pgfpathlineto{\pgfqpoint{7.954970in}{1.221356in}}%
\pgfpathlineto{\pgfqpoint{7.958142in}{1.224260in}}%
\pgfpathlineto{\pgfqpoint{7.961314in}{1.227232in}}%
\pgfpathlineto{\pgfqpoint{7.964486in}{1.230328in}}%
\pgfpathlineto{\pgfqpoint{7.967658in}{1.233304in}}%
\pgfpathlineto{\pgfqpoint{7.970830in}{1.236287in}}%
\pgfpathlineto{\pgfqpoint{7.974002in}{1.239299in}}%
\pgfpathlineto{\pgfqpoint{7.977174in}{1.242321in}}%
\pgfpathlineto{\pgfqpoint{7.980347in}{1.245332in}}%
\pgfpathlineto{\pgfqpoint{7.983519in}{1.248324in}}%
\pgfpathlineto{\pgfqpoint{7.986691in}{1.251368in}}%
\pgfpathlineto{\pgfqpoint{7.989863in}{1.254409in}}%
\pgfpathlineto{\pgfqpoint{7.993035in}{1.257412in}}%
\pgfpathlineto{\pgfqpoint{7.996207in}{1.260451in}}%
\pgfpathlineto{\pgfqpoint{7.999379in}{1.263109in}}%
\pgfpathlineto{\pgfqpoint{8.002551in}{1.265847in}}%
\pgfpathlineto{\pgfqpoint{8.005723in}{1.268851in}}%
\pgfpathlineto{\pgfqpoint{8.008895in}{1.271825in}}%
\pgfpathlineto{\pgfqpoint{8.012067in}{1.274782in}}%
\pgfpathlineto{\pgfqpoint{8.015239in}{1.277736in}}%
\pgfpathlineto{\pgfqpoint{8.018411in}{1.280719in}}%
\pgfpathlineto{\pgfqpoint{8.021583in}{1.283686in}}%
\pgfpathlineto{\pgfqpoint{8.024755in}{1.286683in}}%
\pgfpathlineto{\pgfqpoint{8.027927in}{1.289616in}}%
\pgfpathlineto{\pgfqpoint{8.031099in}{1.292602in}}%
\pgfpathlineto{\pgfqpoint{8.034271in}{1.295617in}}%
\pgfpathlineto{\pgfqpoint{8.037443in}{1.298627in}}%
\pgfpathlineto{\pgfqpoint{8.040615in}{1.301654in}}%
\pgfpathlineto{\pgfqpoint{8.043787in}{1.304673in}}%
\pgfpathlineto{\pgfqpoint{8.046959in}{1.307673in}}%
\pgfpathlineto{\pgfqpoint{8.050131in}{1.310691in}}%
\pgfpathlineto{\pgfqpoint{8.053303in}{1.313658in}}%
\pgfpathlineto{\pgfqpoint{8.056475in}{1.316653in}}%
\pgfpathlineto{\pgfqpoint{8.059648in}{1.319588in}}%
\pgfpathlineto{\pgfqpoint{8.062820in}{1.322555in}}%
\pgfpathlineto{\pgfqpoint{8.065992in}{1.325557in}}%
\pgfpathlineto{\pgfqpoint{8.069164in}{1.328602in}}%
\pgfpathlineto{\pgfqpoint{8.072336in}{1.331609in}}%
\pgfpathlineto{\pgfqpoint{8.075508in}{1.334567in}}%
\pgfpathlineto{\pgfqpoint{8.078680in}{1.337609in}}%
\pgfpathlineto{\pgfqpoint{8.081852in}{1.340668in}}%
\pgfpathlineto{\pgfqpoint{8.085024in}{1.343716in}}%
\pgfpathlineto{\pgfqpoint{8.088196in}{1.346726in}}%
\pgfpathlineto{\pgfqpoint{8.091368in}{1.349685in}}%
\pgfpathlineto{\pgfqpoint{8.094540in}{1.352762in}}%
\pgfpathlineto{\pgfqpoint{8.097712in}{1.355844in}}%
\pgfpathlineto{\pgfqpoint{8.100884in}{1.358905in}}%
\pgfpathlineto{\pgfqpoint{8.104056in}{1.361869in}}%
\pgfpathlineto{\pgfqpoint{8.107228in}{1.364821in}}%
\pgfpathlineto{\pgfqpoint{8.110400in}{1.367815in}}%
\pgfpathlineto{\pgfqpoint{8.113572in}{1.370839in}}%
\pgfpathlineto{\pgfqpoint{8.116744in}{1.373860in}}%
\pgfpathlineto{\pgfqpoint{8.119916in}{1.376816in}}%
\pgfpathlineto{\pgfqpoint{8.123088in}{1.379848in}}%
\pgfpathlineto{\pgfqpoint{8.126260in}{1.382848in}}%
\pgfpathlineto{\pgfqpoint{8.129432in}{1.385832in}}%
\pgfpathlineto{\pgfqpoint{8.132604in}{1.388754in}}%
\pgfpathlineto{\pgfqpoint{8.135777in}{1.391752in}}%
\pgfpathlineto{\pgfqpoint{8.138949in}{1.394754in}}%
\pgfpathlineto{\pgfqpoint{8.142121in}{1.397734in}}%
\pgfpathlineto{\pgfqpoint{8.145293in}{1.400854in}}%
\pgfpathlineto{\pgfqpoint{8.148465in}{1.403888in}}%
\pgfpathlineto{\pgfqpoint{8.151637in}{1.406905in}}%
\pgfpathlineto{\pgfqpoint{8.154809in}{1.409913in}}%
\pgfpathlineto{\pgfqpoint{8.157981in}{1.412953in}}%
\pgfpathlineto{\pgfqpoint{8.161153in}{1.415932in}}%
\pgfpathlineto{\pgfqpoint{8.164325in}{1.418913in}}%
\pgfpathlineto{\pgfqpoint{8.167497in}{1.421961in}}%
\pgfpathlineto{\pgfqpoint{8.170669in}{1.424955in}}%
\pgfpathlineto{\pgfqpoint{8.173841in}{1.427996in}}%
\pgfpathlineto{\pgfqpoint{8.177013in}{1.431009in}}%
\pgfpathlineto{\pgfqpoint{8.180185in}{1.433875in}}%
\pgfpathlineto{\pgfqpoint{8.183357in}{1.436915in}}%
\pgfpathlineto{\pgfqpoint{8.186529in}{1.439950in}}%
\pgfpathlineto{\pgfqpoint{8.189701in}{1.443009in}}%
\pgfpathlineto{\pgfqpoint{8.192873in}{1.446019in}}%
\pgfpathlineto{\pgfqpoint{8.196045in}{1.448958in}}%
\pgfpathlineto{\pgfqpoint{8.199217in}{1.452002in}}%
\pgfpathlineto{\pgfqpoint{8.202389in}{1.455042in}}%
\pgfpathlineto{\pgfqpoint{8.205561in}{1.458015in}}%
\pgfpathlineto{\pgfqpoint{8.208733in}{1.461036in}}%
\pgfpathlineto{\pgfqpoint{8.211905in}{1.464062in}}%
\pgfpathlineto{\pgfqpoint{8.215078in}{1.467154in}}%
\pgfpathlineto{\pgfqpoint{8.218250in}{1.470128in}}%
\pgfpathlineto{\pgfqpoint{8.221422in}{1.473105in}}%
\pgfpathlineto{\pgfqpoint{8.224594in}{1.476108in}}%
\pgfpathlineto{\pgfqpoint{8.227766in}{1.479119in}}%
\pgfpathlineto{\pgfqpoint{8.230938in}{1.481957in}}%
\pgfpathlineto{\pgfqpoint{8.234110in}{1.484955in}}%
\pgfpathlineto{\pgfqpoint{8.237282in}{1.487881in}}%
\pgfpathlineto{\pgfqpoint{8.240454in}{1.490908in}}%
\pgfpathlineto{\pgfqpoint{8.243626in}{1.493882in}}%
\pgfpathlineto{\pgfqpoint{8.246798in}{1.496882in}}%
\pgfpathlineto{\pgfqpoint{8.249970in}{1.499847in}}%
\pgfpathlineto{\pgfqpoint{8.253142in}{1.502785in}}%
\pgfpathlineto{\pgfqpoint{8.256314in}{1.505670in}}%
\pgfpathlineto{\pgfqpoint{8.259486in}{1.508672in}}%
\pgfpathlineto{\pgfqpoint{8.262658in}{1.511730in}}%
\pgfpathlineto{\pgfqpoint{8.265830in}{1.514781in}}%
\pgfpathlineto{\pgfqpoint{8.269002in}{1.517764in}}%
\pgfpathlineto{\pgfqpoint{8.272174in}{1.520725in}}%
\pgfpathlineto{\pgfqpoint{8.275346in}{1.523666in}}%
\pgfpathlineto{\pgfqpoint{8.278518in}{1.526708in}}%
\pgfpathlineto{\pgfqpoint{8.281690in}{1.529732in}}%
\pgfpathlineto{\pgfqpoint{8.281690in}{1.938557in}}%
\pgfpathlineto{\pgfqpoint{8.281690in}{1.938557in}}%
\pgfpathlineto{\pgfqpoint{8.278518in}{1.935414in}}%
\pgfpathlineto{\pgfqpoint{8.275346in}{1.932257in}}%
\pgfpathlineto{\pgfqpoint{8.272174in}{1.929103in}}%
\pgfpathlineto{\pgfqpoint{8.269002in}{1.925931in}}%
\pgfpathlineto{\pgfqpoint{8.265830in}{1.922794in}}%
\pgfpathlineto{\pgfqpoint{8.262658in}{1.919638in}}%
\pgfpathlineto{\pgfqpoint{8.259486in}{1.916492in}}%
\pgfpathlineto{\pgfqpoint{8.256314in}{1.913328in}}%
\pgfpathlineto{\pgfqpoint{8.253142in}{1.910130in}}%
\pgfpathlineto{\pgfqpoint{8.249970in}{1.906954in}}%
\pgfpathlineto{\pgfqpoint{8.246798in}{1.903809in}}%
\pgfpathlineto{\pgfqpoint{8.243626in}{1.900629in}}%
\pgfpathlineto{\pgfqpoint{8.240454in}{1.897473in}}%
\pgfpathlineto{\pgfqpoint{8.237282in}{1.894292in}}%
\pgfpathlineto{\pgfqpoint{8.234110in}{1.891063in}}%
\pgfpathlineto{\pgfqpoint{8.230938in}{1.887877in}}%
\pgfpathlineto{\pgfqpoint{8.227766in}{1.884635in}}%
\pgfpathlineto{\pgfqpoint{8.224594in}{1.881501in}}%
\pgfpathlineto{\pgfqpoint{8.221422in}{1.878339in}}%
\pgfpathlineto{\pgfqpoint{8.218250in}{1.875163in}}%
\pgfpathlineto{\pgfqpoint{8.215078in}{1.871973in}}%
\pgfpathlineto{\pgfqpoint{8.211905in}{1.868799in}}%
\pgfpathlineto{\pgfqpoint{8.208733in}{1.865672in}}%
\pgfpathlineto{\pgfqpoint{8.205561in}{1.862520in}}%
\pgfpathlineto{\pgfqpoint{8.202389in}{1.859345in}}%
\pgfpathlineto{\pgfqpoint{8.199217in}{1.856181in}}%
\pgfpathlineto{\pgfqpoint{8.196045in}{1.853006in}}%
\pgfpathlineto{\pgfqpoint{8.192873in}{1.849784in}}%
\pgfpathlineto{\pgfqpoint{8.189701in}{1.846609in}}%
\pgfpathlineto{\pgfqpoint{8.186529in}{1.843421in}}%
\pgfpathlineto{\pgfqpoint{8.183357in}{1.840252in}}%
\pgfpathlineto{\pgfqpoint{8.180185in}{1.837097in}}%
\pgfpathlineto{\pgfqpoint{8.177013in}{1.833881in}}%
\pgfpathlineto{\pgfqpoint{8.173841in}{1.830713in}}%
\pgfpathlineto{\pgfqpoint{8.170669in}{1.827540in}}%
\pgfpathlineto{\pgfqpoint{8.167497in}{1.824329in}}%
\pgfpathlineto{\pgfqpoint{8.164325in}{1.821170in}}%
\pgfpathlineto{\pgfqpoint{8.161153in}{1.817992in}}%
\pgfpathlineto{\pgfqpoint{8.157981in}{1.814806in}}%
\pgfpathlineto{\pgfqpoint{8.154809in}{1.811685in}}%
\pgfpathlineto{\pgfqpoint{8.151637in}{1.808506in}}%
\pgfpathlineto{\pgfqpoint{8.148465in}{1.805290in}}%
\pgfpathlineto{\pgfqpoint{8.145293in}{1.802102in}}%
\pgfpathlineto{\pgfqpoint{8.142121in}{1.798967in}}%
\pgfpathlineto{\pgfqpoint{8.138949in}{1.795792in}}%
\pgfpathlineto{\pgfqpoint{8.135777in}{1.792616in}}%
\pgfpathlineto{\pgfqpoint{8.132604in}{1.789455in}}%
\pgfpathlineto{\pgfqpoint{8.129432in}{1.786253in}}%
\pgfpathlineto{\pgfqpoint{8.126260in}{1.783286in}}%
\pgfpathlineto{\pgfqpoint{8.123088in}{1.780080in}}%
\pgfpathlineto{\pgfqpoint{8.119916in}{1.776910in}}%
\pgfpathlineto{\pgfqpoint{8.116744in}{1.773712in}}%
\pgfpathlineto{\pgfqpoint{8.113572in}{1.770543in}}%
\pgfpathlineto{\pgfqpoint{8.110400in}{1.767392in}}%
\pgfpathlineto{\pgfqpoint{8.107228in}{1.764207in}}%
\pgfpathlineto{\pgfqpoint{8.104056in}{1.761025in}}%
\pgfpathlineto{\pgfqpoint{8.100884in}{1.757878in}}%
\pgfpathlineto{\pgfqpoint{8.097712in}{1.754642in}}%
\pgfpathlineto{\pgfqpoint{8.094540in}{1.751534in}}%
\pgfpathlineto{\pgfqpoint{8.091368in}{1.748356in}}%
\pgfpathlineto{\pgfqpoint{8.088196in}{1.745167in}}%
\pgfpathlineto{\pgfqpoint{8.085024in}{1.741973in}}%
\pgfpathlineto{\pgfqpoint{8.081852in}{1.738793in}}%
\pgfpathlineto{\pgfqpoint{8.078680in}{1.735618in}}%
\pgfpathlineto{\pgfqpoint{8.075508in}{1.732429in}}%
\pgfpathlineto{\pgfqpoint{8.072336in}{1.729253in}}%
\pgfpathlineto{\pgfqpoint{8.069164in}{1.726095in}}%
\pgfpathlineto{\pgfqpoint{8.065992in}{1.722930in}}%
\pgfpathlineto{\pgfqpoint{8.062820in}{1.719741in}}%
\pgfpathlineto{\pgfqpoint{8.059648in}{1.716555in}}%
\pgfpathlineto{\pgfqpoint{8.056475in}{1.713323in}}%
\pgfpathlineto{\pgfqpoint{8.053303in}{1.710134in}}%
\pgfpathlineto{\pgfqpoint{8.050131in}{1.706952in}}%
\pgfpathlineto{\pgfqpoint{8.046959in}{1.703783in}}%
\pgfpathlineto{\pgfqpoint{8.043787in}{1.700654in}}%
\pgfpathlineto{\pgfqpoint{8.040615in}{1.697509in}}%
\pgfpathlineto{\pgfqpoint{8.037443in}{1.694343in}}%
\pgfpathlineto{\pgfqpoint{8.034271in}{1.691042in}}%
\pgfpathlineto{\pgfqpoint{8.031099in}{1.687861in}}%
\pgfpathlineto{\pgfqpoint{8.027927in}{1.684709in}}%
\pgfpathlineto{\pgfqpoint{8.024755in}{1.681510in}}%
\pgfpathlineto{\pgfqpoint{8.021583in}{1.678294in}}%
\pgfpathlineto{\pgfqpoint{8.018411in}{1.675135in}}%
\pgfpathlineto{\pgfqpoint{8.015239in}{1.671639in}}%
\pgfpathlineto{\pgfqpoint{8.012067in}{1.668298in}}%
\pgfpathlineto{\pgfqpoint{8.008895in}{1.665120in}}%
\pgfpathlineto{\pgfqpoint{8.005723in}{1.661890in}}%
\pgfpathlineto{\pgfqpoint{8.002551in}{1.658733in}}%
\pgfpathlineto{\pgfqpoint{7.999379in}{1.655445in}}%
\pgfpathlineto{\pgfqpoint{7.996207in}{1.652327in}}%
\pgfpathlineto{\pgfqpoint{7.993035in}{1.649212in}}%
\pgfpathlineto{\pgfqpoint{7.989863in}{1.646060in}}%
\pgfpathlineto{\pgfqpoint{7.986691in}{1.642966in}}%
\pgfpathlineto{\pgfqpoint{7.983519in}{1.639783in}}%
\pgfpathlineto{\pgfqpoint{7.980347in}{1.636618in}}%
\pgfpathlineto{\pgfqpoint{7.977174in}{1.633445in}}%
\pgfpathlineto{\pgfqpoint{7.974002in}{1.630267in}}%
\pgfpathlineto{\pgfqpoint{7.970830in}{1.627101in}}%
\pgfpathlineto{\pgfqpoint{7.967658in}{1.623871in}}%
\pgfpathlineto{\pgfqpoint{7.964486in}{1.620683in}}%
\pgfpathlineto{\pgfqpoint{7.961314in}{1.617486in}}%
\pgfpathlineto{\pgfqpoint{7.958142in}{1.614324in}}%
\pgfpathlineto{\pgfqpoint{7.954970in}{1.611128in}}%
\pgfpathlineto{\pgfqpoint{7.951798in}{1.607980in}}%
\pgfpathlineto{\pgfqpoint{7.948626in}{1.604817in}}%
\pgfpathlineto{\pgfqpoint{7.945454in}{1.601616in}}%
\pgfpathlineto{\pgfqpoint{7.942282in}{1.598447in}}%
\pgfpathlineto{\pgfqpoint{7.939110in}{1.595265in}}%
\pgfpathlineto{\pgfqpoint{7.935938in}{1.592118in}}%
\pgfpathlineto{\pgfqpoint{7.932766in}{1.588948in}}%
\pgfpathlineto{\pgfqpoint{7.929594in}{1.585806in}}%
\pgfpathlineto{\pgfqpoint{7.926422in}{1.582620in}}%
\pgfpathlineto{\pgfqpoint{7.923250in}{1.579489in}}%
\pgfpathlineto{\pgfqpoint{7.920078in}{1.576349in}}%
\pgfpathlineto{\pgfqpoint{7.916906in}{1.573173in}}%
\pgfpathlineto{\pgfqpoint{7.913734in}{1.570015in}}%
\pgfpathlineto{\pgfqpoint{7.910562in}{1.566863in}}%
\pgfpathlineto{\pgfqpoint{7.907390in}{1.563709in}}%
\pgfpathlineto{\pgfqpoint{7.904218in}{1.560529in}}%
\pgfpathlineto{\pgfqpoint{7.901046in}{1.557351in}}%
\pgfpathlineto{\pgfqpoint{7.897873in}{1.554155in}}%
\pgfpathlineto{\pgfqpoint{7.894701in}{1.551511in}}%
\pgfpathlineto{\pgfqpoint{7.891529in}{1.548304in}}%
\pgfpathlineto{\pgfqpoint{7.888357in}{1.545080in}}%
\pgfpathlineto{\pgfqpoint{7.885185in}{1.541451in}}%
\pgfpathlineto{\pgfqpoint{7.882013in}{1.537624in}}%
\pgfpathlineto{\pgfqpoint{7.878841in}{1.533760in}}%
\pgfpathlineto{\pgfqpoint{7.875669in}{1.529867in}}%
\pgfpathlineto{\pgfqpoint{7.872497in}{1.526009in}}%
\pgfpathlineto{\pgfqpoint{7.869325in}{1.522104in}}%
\pgfpathlineto{\pgfqpoint{7.866153in}{1.518244in}}%
\pgfpathlineto{\pgfqpoint{7.862981in}{1.514288in}}%
\pgfpathlineto{\pgfqpoint{7.859809in}{1.510380in}}%
\pgfpathlineto{\pgfqpoint{7.856637in}{1.506422in}}%
\pgfpathlineto{\pgfqpoint{7.853465in}{1.502537in}}%
\pgfpathlineto{\pgfqpoint{7.850293in}{1.498632in}}%
\pgfpathlineto{\pgfqpoint{7.847121in}{1.494752in}}%
\pgfpathlineto{\pgfqpoint{7.843949in}{1.490861in}}%
\pgfpathlineto{\pgfqpoint{7.840777in}{1.486959in}}%
\pgfpathlineto{\pgfqpoint{7.837605in}{1.483090in}}%
\pgfpathlineto{\pgfqpoint{7.834433in}{1.479179in}}%
\pgfpathlineto{\pgfqpoint{7.831261in}{1.475315in}}%
\pgfpathlineto{\pgfqpoint{7.828089in}{1.471508in}}%
\pgfpathlineto{\pgfqpoint{7.824917in}{1.467654in}}%
\pgfpathlineto{\pgfqpoint{7.821744in}{1.463728in}}%
\pgfpathlineto{\pgfqpoint{7.818572in}{1.459821in}}%
\pgfpathlineto{\pgfqpoint{7.815400in}{1.455928in}}%
\pgfpathlineto{\pgfqpoint{7.812228in}{1.452040in}}%
\pgfpathlineto{\pgfqpoint{7.809056in}{1.448185in}}%
\pgfpathlineto{\pgfqpoint{7.805884in}{1.444294in}}%
\pgfpathlineto{\pgfqpoint{7.802712in}{1.440339in}}%
\pgfpathlineto{\pgfqpoint{7.799540in}{1.436473in}}%
\pgfpathlineto{\pgfqpoint{7.796368in}{1.432592in}}%
\pgfpathlineto{\pgfqpoint{7.793196in}{1.428688in}}%
\pgfpathlineto{\pgfqpoint{7.790024in}{1.424823in}}%
\pgfpathlineto{\pgfqpoint{7.786852in}{1.420866in}}%
\pgfpathlineto{\pgfqpoint{7.783680in}{1.420716in}}%
\pgfpathlineto{\pgfqpoint{7.780508in}{1.421013in}}%
\pgfpathlineto{\pgfqpoint{7.777336in}{1.420577in}}%
\pgfpathlineto{\pgfqpoint{7.774164in}{1.420614in}}%
\pgfpathlineto{\pgfqpoint{7.770992in}{1.420488in}}%
\pgfpathlineto{\pgfqpoint{7.767820in}{1.420622in}}%
\pgfpathlineto{\pgfqpoint{7.764648in}{1.420780in}}%
\pgfpathlineto{\pgfqpoint{7.761476in}{1.420648in}}%
\pgfpathlineto{\pgfqpoint{7.758304in}{1.420300in}}%
\pgfpathlineto{\pgfqpoint{7.755132in}{1.420012in}}%
\pgfpathlineto{\pgfqpoint{7.751960in}{1.420117in}}%
\pgfpathlineto{\pgfqpoint{7.748788in}{1.420874in}}%
\pgfpathlineto{\pgfqpoint{7.745616in}{1.421140in}}%
\pgfpathlineto{\pgfqpoint{7.742443in}{1.420891in}}%
\pgfpathlineto{\pgfqpoint{7.739271in}{1.421025in}}%
\pgfpathlineto{\pgfqpoint{7.736099in}{1.421413in}}%
\pgfpathlineto{\pgfqpoint{7.732927in}{1.421563in}}%
\pgfpathlineto{\pgfqpoint{7.729755in}{1.421482in}}%
\pgfpathlineto{\pgfqpoint{7.726583in}{1.421726in}}%
\pgfpathlineto{\pgfqpoint{7.723411in}{1.421556in}}%
\pgfpathlineto{\pgfqpoint{7.720239in}{1.421442in}}%
\pgfpathlineto{\pgfqpoint{7.717067in}{1.421580in}}%
\pgfpathlineto{\pgfqpoint{7.713895in}{1.421604in}}%
\pgfpathlineto{\pgfqpoint{7.710723in}{1.421207in}}%
\pgfpathlineto{\pgfqpoint{7.707551in}{1.421178in}}%
\pgfpathlineto{\pgfqpoint{7.704379in}{1.421199in}}%
\pgfpathlineto{\pgfqpoint{7.701207in}{1.420933in}}%
\pgfpathlineto{\pgfqpoint{7.698035in}{1.420785in}}%
\pgfpathlineto{\pgfqpoint{7.694863in}{1.420581in}}%
\pgfpathlineto{\pgfqpoint{7.691691in}{1.420458in}}%
\pgfpathlineto{\pgfqpoint{7.688519in}{1.420164in}}%
\pgfpathlineto{\pgfqpoint{7.685347in}{1.419697in}}%
\pgfpathlineto{\pgfqpoint{7.682175in}{1.419581in}}%
\pgfpathlineto{\pgfqpoint{7.679003in}{1.419269in}}%
\pgfpathlineto{\pgfqpoint{7.675831in}{1.419040in}}%
\pgfpathlineto{\pgfqpoint{7.672659in}{1.418757in}}%
\pgfpathlineto{\pgfqpoint{7.669487in}{1.418443in}}%
\pgfpathlineto{\pgfqpoint{7.666315in}{1.418161in}}%
\pgfpathlineto{\pgfqpoint{7.663142in}{1.418300in}}%
\pgfpathlineto{\pgfqpoint{7.659970in}{1.418246in}}%
\pgfpathlineto{\pgfqpoint{7.656798in}{1.418330in}}%
\pgfpathlineto{\pgfqpoint{7.653626in}{1.417382in}}%
\pgfpathlineto{\pgfqpoint{7.650454in}{1.417420in}}%
\pgfpathlineto{\pgfqpoint{7.647282in}{1.417275in}}%
\pgfpathlineto{\pgfqpoint{7.644110in}{1.417176in}}%
\pgfpathlineto{\pgfqpoint{7.640938in}{1.417263in}}%
\pgfpathlineto{\pgfqpoint{7.637766in}{1.416990in}}%
\pgfpathlineto{\pgfqpoint{7.634594in}{1.417075in}}%
\pgfpathlineto{\pgfqpoint{7.631422in}{1.417146in}}%
\pgfpathlineto{\pgfqpoint{7.628250in}{1.416831in}}%
\pgfpathlineto{\pgfqpoint{7.625078in}{1.416545in}}%
\pgfpathlineto{\pgfqpoint{7.621906in}{1.415802in}}%
\pgfpathlineto{\pgfqpoint{7.618734in}{1.415752in}}%
\pgfpathlineto{\pgfqpoint{7.615562in}{1.416038in}}%
\pgfpathlineto{\pgfqpoint{7.612390in}{1.416428in}}%
\pgfpathlineto{\pgfqpoint{7.609218in}{1.416293in}}%
\pgfpathlineto{\pgfqpoint{7.606046in}{1.415845in}}%
\pgfpathlineto{\pgfqpoint{7.602874in}{1.415749in}}%
\pgfpathlineto{\pgfqpoint{7.599702in}{1.415652in}}%
\pgfpathlineto{\pgfqpoint{7.596530in}{1.415849in}}%
\pgfpathlineto{\pgfqpoint{7.593358in}{1.415677in}}%
\pgfpathlineto{\pgfqpoint{7.590186in}{1.415662in}}%
\pgfpathlineto{\pgfqpoint{7.587013in}{1.415376in}}%
\pgfpathlineto{\pgfqpoint{7.583841in}{1.415642in}}%
\pgfpathlineto{\pgfqpoint{7.580669in}{1.415458in}}%
\pgfpathlineto{\pgfqpoint{7.577497in}{1.415152in}}%
\pgfpathlineto{\pgfqpoint{7.574325in}{1.414913in}}%
\pgfpathlineto{\pgfqpoint{7.571153in}{1.414604in}}%
\pgfpathlineto{\pgfqpoint{7.567981in}{1.414442in}}%
\pgfpathlineto{\pgfqpoint{7.564809in}{1.414222in}}%
\pgfpathlineto{\pgfqpoint{7.561637in}{1.414163in}}%
\pgfpathlineto{\pgfqpoint{7.558465in}{1.414554in}}%
\pgfpathlineto{\pgfqpoint{7.555293in}{1.414443in}}%
\pgfpathlineto{\pgfqpoint{7.552121in}{1.414192in}}%
\pgfpathlineto{\pgfqpoint{7.548949in}{1.414082in}}%
\pgfpathlineto{\pgfqpoint{7.545777in}{1.414084in}}%
\pgfpathlineto{\pgfqpoint{7.542605in}{1.413550in}}%
\pgfpathlineto{\pgfqpoint{7.539433in}{1.413400in}}%
\pgfpathlineto{\pgfqpoint{7.536261in}{1.413697in}}%
\pgfpathlineto{\pgfqpoint{7.533089in}{1.413079in}}%
\pgfpathlineto{\pgfqpoint{7.529917in}{1.413343in}}%
\pgfpathlineto{\pgfqpoint{7.526745in}{1.413057in}}%
\pgfpathlineto{\pgfqpoint{7.523573in}{1.413137in}}%
\pgfpathlineto{\pgfqpoint{7.520401in}{1.413208in}}%
\pgfpathlineto{\pgfqpoint{7.517229in}{1.413542in}}%
\pgfpathlineto{\pgfqpoint{7.514057in}{1.413517in}}%
\pgfpathlineto{\pgfqpoint{7.510885in}{1.413361in}}%
\pgfpathlineto{\pgfqpoint{7.507712in}{1.413525in}}%
\pgfpathlineto{\pgfqpoint{7.504540in}{1.413426in}}%
\pgfpathlineto{\pgfqpoint{7.501368in}{1.413102in}}%
\pgfpathlineto{\pgfqpoint{7.498196in}{1.413293in}}%
\pgfpathlineto{\pgfqpoint{7.495024in}{1.413325in}}%
\pgfpathlineto{\pgfqpoint{7.491852in}{1.413076in}}%
\pgfpathlineto{\pgfqpoint{7.488680in}{1.412514in}}%
\pgfpathlineto{\pgfqpoint{7.485508in}{1.412086in}}%
\pgfpathlineto{\pgfqpoint{7.482336in}{1.412404in}}%
\pgfpathlineto{\pgfqpoint{7.479164in}{1.412010in}}%
\pgfpathlineto{\pgfqpoint{7.475992in}{1.411443in}}%
\pgfpathlineto{\pgfqpoint{7.472820in}{1.410985in}}%
\pgfpathlineto{\pgfqpoint{7.469648in}{1.410971in}}%
\pgfpathlineto{\pgfqpoint{7.466476in}{1.410895in}}%
\pgfpathlineto{\pgfqpoint{7.463304in}{1.410428in}}%
\pgfpathlineto{\pgfqpoint{7.460132in}{1.410413in}}%
\pgfpathlineto{\pgfqpoint{7.456960in}{1.410010in}}%
\pgfpathlineto{\pgfqpoint{7.453788in}{1.410262in}}%
\pgfpathlineto{\pgfqpoint{7.450616in}{1.410032in}}%
\pgfpathlineto{\pgfqpoint{7.447444in}{1.409620in}}%
\pgfpathlineto{\pgfqpoint{7.444272in}{1.409527in}}%
\pgfpathlineto{\pgfqpoint{7.441100in}{1.409028in}}%
\pgfpathlineto{\pgfqpoint{7.437928in}{1.408989in}}%
\pgfpathlineto{\pgfqpoint{7.434756in}{1.408956in}}%
\pgfpathlineto{\pgfqpoint{7.431584in}{1.408758in}}%
\pgfpathlineto{\pgfqpoint{7.428411in}{1.408537in}}%
\pgfpathlineto{\pgfqpoint{7.425239in}{1.408608in}}%
\pgfpathlineto{\pgfqpoint{7.422067in}{1.408558in}}%
\pgfpathlineto{\pgfqpoint{7.418895in}{1.408479in}}%
\pgfpathlineto{\pgfqpoint{7.415723in}{1.408309in}}%
\pgfpathlineto{\pgfqpoint{7.412551in}{1.407798in}}%
\pgfpathlineto{\pgfqpoint{7.409379in}{1.407641in}}%
\pgfpathlineto{\pgfqpoint{7.406207in}{1.407889in}}%
\pgfpathlineto{\pgfqpoint{7.403035in}{1.408084in}}%
\pgfpathlineto{\pgfqpoint{7.399863in}{1.408229in}}%
\pgfpathlineto{\pgfqpoint{7.396691in}{1.408062in}}%
\pgfpathlineto{\pgfqpoint{7.393519in}{1.408008in}}%
\pgfpathlineto{\pgfqpoint{7.390347in}{1.408168in}}%
\pgfpathlineto{\pgfqpoint{7.387175in}{1.408346in}}%
\pgfpathlineto{\pgfqpoint{7.384003in}{1.408106in}}%
\pgfpathlineto{\pgfqpoint{7.380831in}{1.408448in}}%
\pgfpathlineto{\pgfqpoint{7.377659in}{1.408396in}}%
\pgfpathlineto{\pgfqpoint{7.374487in}{1.408247in}}%
\pgfpathlineto{\pgfqpoint{7.371315in}{1.407739in}}%
\pgfpathlineto{\pgfqpoint{7.368143in}{1.407496in}}%
\pgfpathlineto{\pgfqpoint{7.364971in}{1.407284in}}%
\pgfpathlineto{\pgfqpoint{7.361799in}{1.406475in}}%
\pgfpathlineto{\pgfqpoint{7.358627in}{1.406272in}}%
\pgfpathlineto{\pgfqpoint{7.355455in}{1.406513in}}%
\pgfpathlineto{\pgfqpoint{7.352282in}{1.406303in}}%
\pgfpathlineto{\pgfqpoint{7.349110in}{1.406454in}}%
\pgfpathlineto{\pgfqpoint{7.345938in}{1.406278in}}%
\pgfpathlineto{\pgfqpoint{7.342766in}{1.405995in}}%
\pgfpathlineto{\pgfqpoint{7.339594in}{1.406191in}}%
\pgfpathlineto{\pgfqpoint{7.336422in}{1.406110in}}%
\pgfpathlineto{\pgfqpoint{7.333250in}{1.406182in}}%
\pgfpathlineto{\pgfqpoint{7.330078in}{1.406354in}}%
\pgfpathlineto{\pgfqpoint{7.326906in}{1.406576in}}%
\pgfpathlineto{\pgfqpoint{7.323734in}{1.406422in}}%
\pgfpathlineto{\pgfqpoint{7.320562in}{1.406158in}}%
\pgfpathlineto{\pgfqpoint{7.317390in}{1.405546in}}%
\pgfpathlineto{\pgfqpoint{7.314218in}{1.404866in}}%
\pgfpathlineto{\pgfqpoint{7.311046in}{1.404951in}}%
\pgfpathlineto{\pgfqpoint{7.307874in}{1.404201in}}%
\pgfpathlineto{\pgfqpoint{7.304702in}{1.404523in}}%
\pgfpathlineto{\pgfqpoint{7.301530in}{1.404260in}}%
\pgfpathlineto{\pgfqpoint{7.298358in}{1.404153in}}%
\pgfpathlineto{\pgfqpoint{7.295186in}{1.403949in}}%
\pgfpathlineto{\pgfqpoint{7.292014in}{1.403843in}}%
\pgfpathlineto{\pgfqpoint{7.288842in}{1.403385in}}%
\pgfpathlineto{\pgfqpoint{7.285670in}{1.402783in}}%
\pgfpathlineto{\pgfqpoint{7.282498in}{1.402900in}}%
\pgfpathlineto{\pgfqpoint{7.279326in}{1.402697in}}%
\pgfpathlineto{\pgfqpoint{7.276154in}{1.402929in}}%
\pgfpathlineto{\pgfqpoint{7.272981in}{1.402939in}}%
\pgfpathlineto{\pgfqpoint{7.269809in}{1.402728in}}%
\pgfpathlineto{\pgfqpoint{7.266637in}{1.402409in}}%
\pgfpathlineto{\pgfqpoint{7.263465in}{1.401956in}}%
\pgfpathlineto{\pgfqpoint{7.260293in}{1.401726in}}%
\pgfpathlineto{\pgfqpoint{7.257121in}{1.401687in}}%
\pgfpathlineto{\pgfqpoint{7.253949in}{1.401628in}}%
\pgfpathlineto{\pgfqpoint{7.250777in}{1.401443in}}%
\pgfpathlineto{\pgfqpoint{7.247605in}{1.401267in}}%
\pgfpathlineto{\pgfqpoint{7.244433in}{1.401656in}}%
\pgfpathlineto{\pgfqpoint{7.241261in}{1.401744in}}%
\pgfpathlineto{\pgfqpoint{7.238089in}{1.401441in}}%
\pgfpathlineto{\pgfqpoint{7.234917in}{1.401396in}}%
\pgfpathlineto{\pgfqpoint{7.231745in}{1.401603in}}%
\pgfpathlineto{\pgfqpoint{7.228573in}{1.401484in}}%
\pgfpathlineto{\pgfqpoint{7.225401in}{1.401853in}}%
\pgfpathlineto{\pgfqpoint{7.222229in}{1.401568in}}%
\pgfpathlineto{\pgfqpoint{7.219057in}{1.401923in}}%
\pgfpathlineto{\pgfqpoint{7.215885in}{1.401946in}}%
\pgfpathlineto{\pgfqpoint{7.212713in}{1.402281in}}%
\pgfpathlineto{\pgfqpoint{7.209541in}{1.402398in}}%
\pgfpathlineto{\pgfqpoint{7.206369in}{1.402212in}}%
\pgfpathlineto{\pgfqpoint{7.203197in}{1.401950in}}%
\pgfpathlineto{\pgfqpoint{7.200025in}{1.401588in}}%
\pgfpathlineto{\pgfqpoint{7.196853in}{1.401113in}}%
\pgfpathlineto{\pgfqpoint{7.193680in}{1.401333in}}%
\pgfpathlineto{\pgfqpoint{7.190508in}{1.401014in}}%
\pgfpathlineto{\pgfqpoint{7.187336in}{1.400576in}}%
\pgfpathlineto{\pgfqpoint{7.184164in}{1.400282in}}%
\pgfpathlineto{\pgfqpoint{7.180992in}{1.400016in}}%
\pgfpathlineto{\pgfqpoint{7.177820in}{1.399981in}}%
\pgfpathlineto{\pgfqpoint{7.174648in}{1.399937in}}%
\pgfpathlineto{\pgfqpoint{7.171476in}{1.399490in}}%
\pgfpathlineto{\pgfqpoint{7.168304in}{1.399411in}}%
\pgfpathlineto{\pgfqpoint{7.165132in}{1.399486in}}%
\pgfpathlineto{\pgfqpoint{7.161960in}{1.399568in}}%
\pgfpathlineto{\pgfqpoint{7.158788in}{1.399885in}}%
\pgfpathlineto{\pgfqpoint{7.155616in}{1.399637in}}%
\pgfpathlineto{\pgfqpoint{7.152444in}{1.399889in}}%
\pgfpathlineto{\pgfqpoint{7.149272in}{1.399564in}}%
\pgfpathlineto{\pgfqpoint{7.146100in}{1.399784in}}%
\pgfpathlineto{\pgfqpoint{7.142928in}{1.399971in}}%
\pgfpathlineto{\pgfqpoint{7.139756in}{1.400049in}}%
\pgfpathlineto{\pgfqpoint{7.136584in}{1.400124in}}%
\pgfpathlineto{\pgfqpoint{7.133412in}{1.400109in}}%
\pgfpathlineto{\pgfqpoint{7.130240in}{1.400113in}}%
\pgfpathlineto{\pgfqpoint{7.127068in}{1.399662in}}%
\pgfpathlineto{\pgfqpoint{7.123896in}{1.399360in}}%
\pgfpathlineto{\pgfqpoint{7.120724in}{1.398838in}}%
\pgfpathlineto{\pgfqpoint{7.117551in}{1.399083in}}%
\pgfpathlineto{\pgfqpoint{7.114379in}{1.398937in}}%
\pgfpathlineto{\pgfqpoint{7.111207in}{1.398659in}}%
\pgfpathlineto{\pgfqpoint{7.108035in}{1.398839in}}%
\pgfpathlineto{\pgfqpoint{7.104863in}{1.398771in}}%
\pgfpathlineto{\pgfqpoint{7.101691in}{1.398720in}}%
\pgfpathlineto{\pgfqpoint{7.098519in}{1.398407in}}%
\pgfpathlineto{\pgfqpoint{7.095347in}{1.398474in}}%
\pgfpathlineto{\pgfqpoint{7.092175in}{1.398599in}}%
\pgfpathlineto{\pgfqpoint{7.089003in}{1.398576in}}%
\pgfpathlineto{\pgfqpoint{7.085831in}{1.398847in}}%
\pgfpathlineto{\pgfqpoint{7.082659in}{1.398735in}}%
\pgfpathlineto{\pgfqpoint{7.079487in}{1.398756in}}%
\pgfpathlineto{\pgfqpoint{7.076315in}{1.398462in}}%
\pgfpathlineto{\pgfqpoint{7.073143in}{1.397965in}}%
\pgfpathlineto{\pgfqpoint{7.069971in}{1.397645in}}%
\pgfpathlineto{\pgfqpoint{7.066799in}{1.397781in}}%
\pgfpathlineto{\pgfqpoint{7.063627in}{1.397509in}}%
\pgfpathlineto{\pgfqpoint{7.060455in}{1.397925in}}%
\pgfpathlineto{\pgfqpoint{7.057283in}{1.398202in}}%
\pgfpathlineto{\pgfqpoint{7.054111in}{1.397984in}}%
\pgfpathlineto{\pgfqpoint{7.050939in}{1.398394in}}%
\pgfpathlineto{\pgfqpoint{7.047767in}{1.398488in}}%
\pgfpathlineto{\pgfqpoint{7.044595in}{1.398402in}}%
\pgfpathlineto{\pgfqpoint{7.041423in}{1.397882in}}%
\pgfpathlineto{\pgfqpoint{7.038250in}{1.398217in}}%
\pgfpathlineto{\pgfqpoint{7.035078in}{1.398203in}}%
\pgfpathlineto{\pgfqpoint{7.031906in}{1.398004in}}%
\pgfpathlineto{\pgfqpoint{7.028734in}{1.397754in}}%
\pgfpathlineto{\pgfqpoint{7.025562in}{1.397767in}}%
\pgfpathlineto{\pgfqpoint{7.022390in}{1.397900in}}%
\pgfpathlineto{\pgfqpoint{7.019218in}{1.397990in}}%
\pgfpathlineto{\pgfqpoint{7.016046in}{1.397969in}}%
\pgfpathlineto{\pgfqpoint{7.012874in}{1.398249in}}%
\pgfpathlineto{\pgfqpoint{7.009702in}{1.397996in}}%
\pgfpathlineto{\pgfqpoint{7.006530in}{1.397957in}}%
\pgfpathlineto{\pgfqpoint{7.003358in}{1.397695in}}%
\pgfpathlineto{\pgfqpoint{7.000186in}{1.397799in}}%
\pgfpathlineto{\pgfqpoint{6.997014in}{1.397698in}}%
\pgfpathlineto{\pgfqpoint{6.993842in}{1.398037in}}%
\pgfpathlineto{\pgfqpoint{6.990670in}{1.398330in}}%
\pgfpathlineto{\pgfqpoint{6.987498in}{1.398378in}}%
\pgfpathlineto{\pgfqpoint{6.984326in}{1.398111in}}%
\pgfpathlineto{\pgfqpoint{6.981154in}{1.397737in}}%
\pgfpathlineto{\pgfqpoint{6.977982in}{1.397704in}}%
\pgfpathlineto{\pgfqpoint{6.974810in}{1.397612in}}%
\pgfpathlineto{\pgfqpoint{6.971638in}{1.397368in}}%
\pgfpathlineto{\pgfqpoint{6.968466in}{1.397593in}}%
\pgfpathlineto{\pgfqpoint{6.965294in}{1.397776in}}%
\pgfpathlineto{\pgfqpoint{6.962122in}{1.397220in}}%
\pgfpathlineto{\pgfqpoint{6.958949in}{1.396556in}}%
\pgfpathlineto{\pgfqpoint{6.955777in}{1.396203in}}%
\pgfpathlineto{\pgfqpoint{6.952605in}{1.396122in}}%
\pgfpathlineto{\pgfqpoint{6.949433in}{1.396491in}}%
\pgfpathlineto{\pgfqpoint{6.946261in}{1.396387in}}%
\pgfpathlineto{\pgfqpoint{6.943089in}{1.396216in}}%
\pgfpathlineto{\pgfqpoint{6.939917in}{1.395988in}}%
\pgfpathlineto{\pgfqpoint{6.936745in}{1.396128in}}%
\pgfpathlineto{\pgfqpoint{6.933573in}{1.396128in}}%
\pgfpathlineto{\pgfqpoint{6.930401in}{1.395981in}}%
\pgfpathlineto{\pgfqpoint{6.927229in}{1.395886in}}%
\pgfpathlineto{\pgfqpoint{6.924057in}{1.395667in}}%
\pgfpathlineto{\pgfqpoint{6.920885in}{1.395755in}}%
\pgfpathlineto{\pgfqpoint{6.917713in}{1.395174in}}%
\pgfpathlineto{\pgfqpoint{6.914541in}{1.395310in}}%
\pgfpathlineto{\pgfqpoint{6.911369in}{1.395077in}}%
\pgfpathlineto{\pgfqpoint{6.908197in}{1.395122in}}%
\pgfpathlineto{\pgfqpoint{6.905025in}{1.394451in}}%
\pgfpathlineto{\pgfqpoint{6.901853in}{1.394244in}}%
\pgfpathlineto{\pgfqpoint{6.898681in}{1.393914in}}%
\pgfpathlineto{\pgfqpoint{6.895509in}{1.393550in}}%
\pgfpathlineto{\pgfqpoint{6.892337in}{1.393411in}}%
\pgfpathlineto{\pgfqpoint{6.889165in}{1.393020in}}%
\pgfpathlineto{\pgfqpoint{6.885993in}{1.392974in}}%
\pgfpathlineto{\pgfqpoint{6.882820in}{1.392719in}}%
\pgfpathlineto{\pgfqpoint{6.879648in}{1.392907in}}%
\pgfpathlineto{\pgfqpoint{6.876476in}{1.392311in}}%
\pgfpathlineto{\pgfqpoint{6.873304in}{1.392050in}}%
\pgfpathlineto{\pgfqpoint{6.870132in}{1.391880in}}%
\pgfpathlineto{\pgfqpoint{6.866960in}{1.391456in}}%
\pgfpathlineto{\pgfqpoint{6.863788in}{1.391508in}}%
\pgfpathlineto{\pgfqpoint{6.860616in}{1.391318in}}%
\pgfpathlineto{\pgfqpoint{6.857444in}{1.391261in}}%
\pgfpathlineto{\pgfqpoint{6.854272in}{1.391236in}}%
\pgfpathlineto{\pgfqpoint{6.851100in}{1.390879in}}%
\pgfpathlineto{\pgfqpoint{6.847928in}{1.390501in}}%
\pgfpathlineto{\pgfqpoint{6.844756in}{1.390377in}}%
\pgfpathlineto{\pgfqpoint{6.841584in}{1.389952in}}%
\pgfpathlineto{\pgfqpoint{6.838412in}{1.390254in}}%
\pgfpathlineto{\pgfqpoint{6.835240in}{1.389525in}}%
\pgfpathlineto{\pgfqpoint{6.832068in}{1.389836in}}%
\pgfpathlineto{\pgfqpoint{6.828896in}{1.389677in}}%
\pgfpathlineto{\pgfqpoint{6.825724in}{1.389256in}}%
\pgfpathlineto{\pgfqpoint{6.822552in}{1.389106in}}%
\pgfpathlineto{\pgfqpoint{6.819380in}{1.389118in}}%
\pgfpathlineto{\pgfqpoint{6.816208in}{1.389455in}}%
\pgfpathlineto{\pgfqpoint{6.813036in}{1.389189in}}%
\pgfpathlineto{\pgfqpoint{6.809864in}{1.389166in}}%
\pgfpathlineto{\pgfqpoint{6.806692in}{1.388826in}}%
\pgfpathlineto{\pgfqpoint{6.803519in}{1.388796in}}%
\pgfpathlineto{\pgfqpoint{6.800347in}{1.388765in}}%
\pgfpathlineto{\pgfqpoint{6.797175in}{1.388652in}}%
\pgfpathlineto{\pgfqpoint{6.794003in}{1.388316in}}%
\pgfpathlineto{\pgfqpoint{6.790831in}{1.388603in}}%
\pgfpathlineto{\pgfqpoint{6.787659in}{1.388452in}}%
\pgfpathlineto{\pgfqpoint{6.784487in}{1.387821in}}%
\pgfpathlineto{\pgfqpoint{6.781315in}{1.387934in}}%
\pgfpathlineto{\pgfqpoint{6.778143in}{1.387712in}}%
\pgfpathlineto{\pgfqpoint{6.774971in}{1.388105in}}%
\pgfpathlineto{\pgfqpoint{6.771799in}{1.388144in}}%
\pgfpathlineto{\pgfqpoint{6.768627in}{1.387968in}}%
\pgfpathlineto{\pgfqpoint{6.765455in}{1.388040in}}%
\pgfpathlineto{\pgfqpoint{6.762283in}{1.387521in}}%
\pgfpathlineto{\pgfqpoint{6.759111in}{1.387211in}}%
\pgfpathlineto{\pgfqpoint{6.755939in}{1.387362in}}%
\pgfpathlineto{\pgfqpoint{6.752767in}{1.387254in}}%
\pgfpathlineto{\pgfqpoint{6.749595in}{1.387127in}}%
\pgfpathlineto{\pgfqpoint{6.746423in}{1.387317in}}%
\pgfpathlineto{\pgfqpoint{6.743251in}{1.387586in}}%
\pgfpathlineto{\pgfqpoint{6.740079in}{1.387796in}}%
\pgfpathlineto{\pgfqpoint{6.736907in}{1.388152in}}%
\pgfpathlineto{\pgfqpoint{6.733735in}{1.387863in}}%
\pgfpathlineto{\pgfqpoint{6.730563in}{1.388137in}}%
\pgfpathlineto{\pgfqpoint{6.727391in}{1.387888in}}%
\pgfpathlineto{\pgfqpoint{6.724218in}{1.388167in}}%
\pgfpathlineto{\pgfqpoint{6.721046in}{1.388197in}}%
\pgfpathlineto{\pgfqpoint{6.717874in}{1.387945in}}%
\pgfpathlineto{\pgfqpoint{6.714702in}{1.387757in}}%
\pgfpathlineto{\pgfqpoint{6.711530in}{1.386554in}}%
\pgfpathlineto{\pgfqpoint{6.708358in}{1.386510in}}%
\pgfpathlineto{\pgfqpoint{6.705186in}{1.386542in}}%
\pgfpathlineto{\pgfqpoint{6.702014in}{1.386606in}}%
\pgfpathlineto{\pgfqpoint{6.698842in}{1.386717in}}%
\pgfpathlineto{\pgfqpoint{6.695670in}{1.386776in}}%
\pgfpathlineto{\pgfqpoint{6.692498in}{1.385853in}}%
\pgfpathlineto{\pgfqpoint{6.689326in}{1.384798in}}%
\pgfpathlineto{\pgfqpoint{6.686154in}{1.384727in}}%
\pgfpathlineto{\pgfqpoint{6.682982in}{1.384622in}}%
\pgfpathlineto{\pgfqpoint{6.679810in}{1.384748in}}%
\pgfpathlineto{\pgfqpoint{6.676638in}{1.384744in}}%
\pgfpathlineto{\pgfqpoint{6.673466in}{1.384896in}}%
\pgfpathlineto{\pgfqpoint{6.670294in}{1.384376in}}%
\pgfpathlineto{\pgfqpoint{6.667122in}{1.383952in}}%
\pgfpathlineto{\pgfqpoint{6.663950in}{1.383964in}}%
\pgfpathlineto{\pgfqpoint{6.660778in}{1.384357in}}%
\pgfpathlineto{\pgfqpoint{6.657606in}{1.384789in}}%
\pgfpathlineto{\pgfqpoint{6.654434in}{1.384720in}}%
\pgfpathlineto{\pgfqpoint{6.651262in}{1.384843in}}%
\pgfpathlineto{\pgfqpoint{6.648089in}{1.384219in}}%
\pgfpathlineto{\pgfqpoint{6.644917in}{1.384529in}}%
\pgfpathlineto{\pgfqpoint{6.641745in}{1.384543in}}%
\pgfpathlineto{\pgfqpoint{6.638573in}{1.384760in}}%
\pgfpathlineto{\pgfqpoint{6.635401in}{1.384270in}}%
\pgfpathlineto{\pgfqpoint{6.632229in}{1.383516in}}%
\pgfpathlineto{\pgfqpoint{6.629057in}{1.383125in}}%
\pgfpathlineto{\pgfqpoint{6.625885in}{1.383418in}}%
\pgfpathlineto{\pgfqpoint{6.622713in}{1.382525in}}%
\pgfpathlineto{\pgfqpoint{6.619541in}{1.382214in}}%
\pgfpathlineto{\pgfqpoint{6.616369in}{1.382125in}}%
\pgfpathlineto{\pgfqpoint{6.613197in}{1.382158in}}%
\pgfpathlineto{\pgfqpoint{6.610025in}{1.382677in}}%
\pgfpathlineto{\pgfqpoint{6.606853in}{1.382552in}}%
\pgfpathlineto{\pgfqpoint{6.603681in}{1.382250in}}%
\pgfpathlineto{\pgfqpoint{6.600509in}{1.382294in}}%
\pgfpathlineto{\pgfqpoint{6.597337in}{1.382819in}}%
\pgfpathlineto{\pgfqpoint{6.594165in}{1.382853in}}%
\pgfpathlineto{\pgfqpoint{6.590993in}{1.382543in}}%
\pgfpathlineto{\pgfqpoint{6.587821in}{1.382535in}}%
\pgfpathlineto{\pgfqpoint{6.584649in}{1.382363in}}%
\pgfpathlineto{\pgfqpoint{6.581477in}{1.382016in}}%
\pgfpathlineto{\pgfqpoint{6.578305in}{1.382012in}}%
\pgfpathlineto{\pgfqpoint{6.575133in}{1.382048in}}%
\pgfpathlineto{\pgfqpoint{6.571961in}{1.381912in}}%
\pgfpathlineto{\pgfqpoint{6.568788in}{1.381900in}}%
\pgfpathlineto{\pgfqpoint{6.565616in}{1.381775in}}%
\pgfpathlineto{\pgfqpoint{6.562444in}{1.381658in}}%
\pgfpathlineto{\pgfqpoint{6.559272in}{1.381835in}}%
\pgfpathlineto{\pgfqpoint{6.556100in}{1.382012in}}%
\pgfpathlineto{\pgfqpoint{6.552928in}{1.382020in}}%
\pgfpathlineto{\pgfqpoint{6.549756in}{1.381865in}}%
\pgfpathlineto{\pgfqpoint{6.546584in}{1.381585in}}%
\pgfpathlineto{\pgfqpoint{6.543412in}{1.381876in}}%
\pgfpathlineto{\pgfqpoint{6.540240in}{1.381656in}}%
\pgfpathlineto{\pgfqpoint{6.537068in}{1.381960in}}%
\pgfpathlineto{\pgfqpoint{6.533896in}{1.382293in}}%
\pgfpathlineto{\pgfqpoint{6.530724in}{1.382445in}}%
\pgfpathlineto{\pgfqpoint{6.527552in}{1.382650in}}%
\pgfpathlineto{\pgfqpoint{6.524380in}{1.382783in}}%
\pgfpathlineto{\pgfqpoint{6.521208in}{1.382776in}}%
\pgfpathlineto{\pgfqpoint{6.518036in}{1.382468in}}%
\pgfpathlineto{\pgfqpoint{6.514864in}{1.382874in}}%
\pgfpathlineto{\pgfqpoint{6.511692in}{1.382815in}}%
\pgfpathlineto{\pgfqpoint{6.508520in}{1.382375in}}%
\pgfpathlineto{\pgfqpoint{6.505348in}{1.382460in}}%
\pgfpathlineto{\pgfqpoint{6.502176in}{1.382295in}}%
\pgfpathlineto{\pgfqpoint{6.499004in}{1.382494in}}%
\pgfpathlineto{\pgfqpoint{6.495832in}{1.382476in}}%
\pgfpathlineto{\pgfqpoint{6.492659in}{1.382548in}}%
\pgfpathlineto{\pgfqpoint{6.489487in}{1.382586in}}%
\pgfpathlineto{\pgfqpoint{6.486315in}{1.381970in}}%
\pgfpathlineto{\pgfqpoint{6.483143in}{1.381794in}}%
\pgfpathlineto{\pgfqpoint{6.479971in}{1.381412in}}%
\pgfpathlineto{\pgfqpoint{6.476799in}{1.381303in}}%
\pgfpathlineto{\pgfqpoint{6.473627in}{1.380916in}}%
\pgfpathlineto{\pgfqpoint{6.470455in}{1.380651in}}%
\pgfpathlineto{\pgfqpoint{6.467283in}{1.380705in}}%
\pgfpathlineto{\pgfqpoint{6.464111in}{1.380460in}}%
\pgfpathlineto{\pgfqpoint{6.460939in}{1.380468in}}%
\pgfpathlineto{\pgfqpoint{6.457767in}{1.380317in}}%
\pgfpathlineto{\pgfqpoint{6.454595in}{1.380056in}}%
\pgfpathlineto{\pgfqpoint{6.451423in}{1.380274in}}%
\pgfpathlineto{\pgfqpoint{6.448251in}{1.380588in}}%
\pgfpathlineto{\pgfqpoint{6.445079in}{1.380271in}}%
\pgfpathlineto{\pgfqpoint{6.441907in}{1.379950in}}%
\pgfpathlineto{\pgfqpoint{6.438735in}{1.380091in}}%
\pgfpathlineto{\pgfqpoint{6.435563in}{1.379441in}}%
\pgfpathlineto{\pgfqpoint{6.432391in}{1.379331in}}%
\pgfpathlineto{\pgfqpoint{6.429219in}{1.379226in}}%
\pgfpathlineto{\pgfqpoint{6.426047in}{1.379230in}}%
\pgfpathlineto{\pgfqpoint{6.422875in}{1.379045in}}%
\pgfpathlineto{\pgfqpoint{6.419703in}{1.379020in}}%
\pgfpathlineto{\pgfqpoint{6.416531in}{1.378767in}}%
\pgfpathlineto{\pgfqpoint{6.413358in}{1.379035in}}%
\pgfpathlineto{\pgfqpoint{6.410186in}{1.379252in}}%
\pgfpathlineto{\pgfqpoint{6.407014in}{1.379537in}}%
\pgfpathlineto{\pgfqpoint{6.403842in}{1.379032in}}%
\pgfpathlineto{\pgfqpoint{6.400670in}{1.378922in}}%
\pgfpathlineto{\pgfqpoint{6.397498in}{1.378914in}}%
\pgfpathlineto{\pgfqpoint{6.394326in}{1.378589in}}%
\pgfpathlineto{\pgfqpoint{6.391154in}{1.378407in}}%
\pgfpathlineto{\pgfqpoint{6.387982in}{1.378586in}}%
\pgfpathlineto{\pgfqpoint{6.384810in}{1.378502in}}%
\pgfpathlineto{\pgfqpoint{6.381638in}{1.378206in}}%
\pgfpathlineto{\pgfqpoint{6.378466in}{1.377411in}}%
\pgfpathlineto{\pgfqpoint{6.375294in}{1.377175in}}%
\pgfpathlineto{\pgfqpoint{6.372122in}{1.377213in}}%
\pgfpathlineto{\pgfqpoint{6.368950in}{1.377215in}}%
\pgfpathlineto{\pgfqpoint{6.365778in}{1.377020in}}%
\pgfpathlineto{\pgfqpoint{6.362606in}{1.376737in}}%
\pgfpathlineto{\pgfqpoint{6.359434in}{1.376833in}}%
\pgfpathlineto{\pgfqpoint{6.356262in}{1.377012in}}%
\pgfpathlineto{\pgfqpoint{6.353090in}{1.376839in}}%
\pgfpathlineto{\pgfqpoint{6.349918in}{1.376863in}}%
\pgfpathlineto{\pgfqpoint{6.346746in}{1.376956in}}%
\pgfpathlineto{\pgfqpoint{6.343574in}{1.376739in}}%
\pgfpathlineto{\pgfqpoint{6.340402in}{1.376463in}}%
\pgfpathlineto{\pgfqpoint{6.337230in}{1.376278in}}%
\pgfpathlineto{\pgfqpoint{6.334057in}{1.375541in}}%
\pgfpathlineto{\pgfqpoint{6.330885in}{1.375487in}}%
\pgfpathlineto{\pgfqpoint{6.327713in}{1.375217in}}%
\pgfpathlineto{\pgfqpoint{6.324541in}{1.375078in}}%
\pgfpathlineto{\pgfqpoint{6.321369in}{1.374583in}}%
\pgfpathlineto{\pgfqpoint{6.318197in}{1.374594in}}%
\pgfpathlineto{\pgfqpoint{6.315025in}{1.374091in}}%
\pgfpathlineto{\pgfqpoint{6.311853in}{1.373800in}}%
\pgfpathlineto{\pgfqpoint{6.308681in}{1.373540in}}%
\pgfpathlineto{\pgfqpoint{6.305509in}{1.373564in}}%
\pgfpathlineto{\pgfqpoint{6.302337in}{1.373655in}}%
\pgfpathlineto{\pgfqpoint{6.299165in}{1.373732in}}%
\pgfpathlineto{\pgfqpoint{6.295993in}{1.373621in}}%
\pgfpathlineto{\pgfqpoint{6.292821in}{1.373481in}}%
\pgfpathlineto{\pgfqpoint{6.289649in}{1.373497in}}%
\pgfpathlineto{\pgfqpoint{6.286477in}{1.373095in}}%
\pgfpathlineto{\pgfqpoint{6.283305in}{1.373116in}}%
\pgfpathlineto{\pgfqpoint{6.280133in}{1.373137in}}%
\pgfpathlineto{\pgfqpoint{6.276961in}{1.373046in}}%
\pgfpathlineto{\pgfqpoint{6.273789in}{1.372751in}}%
\pgfpathlineto{\pgfqpoint{6.270617in}{1.372850in}}%
\pgfpathlineto{\pgfqpoint{6.267445in}{1.373018in}}%
\pgfpathlineto{\pgfqpoint{6.264273in}{1.372884in}}%
\pgfpathlineto{\pgfqpoint{6.261101in}{1.372649in}}%
\pgfpathlineto{\pgfqpoint{6.257928in}{1.372263in}}%
\pgfpathlineto{\pgfqpoint{6.254756in}{1.371928in}}%
\pgfpathlineto{\pgfqpoint{6.251584in}{1.371603in}}%
\pgfpathlineto{\pgfqpoint{6.248412in}{1.371243in}}%
\pgfpathlineto{\pgfqpoint{6.245240in}{1.371353in}}%
\pgfpathlineto{\pgfqpoint{6.242068in}{1.371261in}}%
\pgfpathlineto{\pgfqpoint{6.238896in}{1.371719in}}%
\pgfpathlineto{\pgfqpoint{6.235724in}{1.371581in}}%
\pgfpathlineto{\pgfqpoint{6.232552in}{1.371329in}}%
\pgfpathlineto{\pgfqpoint{6.229380in}{1.370516in}}%
\pgfpathlineto{\pgfqpoint{6.226208in}{1.370097in}}%
\pgfpathlineto{\pgfqpoint{6.223036in}{1.370113in}}%
\pgfpathlineto{\pgfqpoint{6.219864in}{1.370156in}}%
\pgfpathlineto{\pgfqpoint{6.216692in}{1.369438in}}%
\pgfpathlineto{\pgfqpoint{6.213520in}{1.369502in}}%
\pgfpathlineto{\pgfqpoint{6.210348in}{1.369654in}}%
\pgfpathlineto{\pgfqpoint{6.207176in}{1.368858in}}%
\pgfpathlineto{\pgfqpoint{6.204004in}{1.368789in}}%
\pgfpathlineto{\pgfqpoint{6.200832in}{1.369022in}}%
\pgfpathlineto{\pgfqpoint{6.197660in}{1.368357in}}%
\pgfpathlineto{\pgfqpoint{6.194488in}{1.368017in}}%
\pgfpathlineto{\pgfqpoint{6.191316in}{1.368316in}}%
\pgfpathlineto{\pgfqpoint{6.188144in}{1.368195in}}%
\pgfpathlineto{\pgfqpoint{6.184972in}{1.367987in}}%
\pgfpathlineto{\pgfqpoint{6.181800in}{1.368153in}}%
\pgfpathlineto{\pgfqpoint{6.178627in}{1.368479in}}%
\pgfpathlineto{\pgfqpoint{6.175455in}{1.368712in}}%
\pgfpathlineto{\pgfqpoint{6.172283in}{1.368319in}}%
\pgfpathlineto{\pgfqpoint{6.169111in}{1.367847in}}%
\pgfpathlineto{\pgfqpoint{6.165939in}{1.367683in}}%
\pgfpathlineto{\pgfqpoint{6.162767in}{1.367854in}}%
\pgfpathlineto{\pgfqpoint{6.159595in}{1.367869in}}%
\pgfpathlineto{\pgfqpoint{6.156423in}{1.367724in}}%
\pgfpathlineto{\pgfqpoint{6.153251in}{1.367896in}}%
\pgfpathlineto{\pgfqpoint{6.150079in}{1.368007in}}%
\pgfpathlineto{\pgfqpoint{6.146907in}{1.368072in}}%
\pgfpathlineto{\pgfqpoint{6.143735in}{1.368054in}}%
\pgfpathlineto{\pgfqpoint{6.140563in}{1.368004in}}%
\pgfpathlineto{\pgfqpoint{6.137391in}{1.367898in}}%
\pgfpathlineto{\pgfqpoint{6.134219in}{1.367986in}}%
\pgfpathlineto{\pgfqpoint{6.131047in}{1.367956in}}%
\pgfpathlineto{\pgfqpoint{6.127875in}{1.367884in}}%
\pgfpathlineto{\pgfqpoint{6.124703in}{1.367582in}}%
\pgfpathlineto{\pgfqpoint{6.121531in}{1.367381in}}%
\pgfpathlineto{\pgfqpoint{6.118359in}{1.367379in}}%
\pgfpathlineto{\pgfqpoint{6.115187in}{1.367002in}}%
\pgfpathlineto{\pgfqpoint{6.112015in}{1.366677in}}%
\pgfpathlineto{\pgfqpoint{6.108843in}{1.366590in}}%
\pgfpathlineto{\pgfqpoint{6.105671in}{1.366665in}}%
\pgfpathlineto{\pgfqpoint{6.102499in}{1.366783in}}%
\pgfpathlineto{\pgfqpoint{6.099326in}{1.367112in}}%
\pgfpathlineto{\pgfqpoint{6.096154in}{1.367007in}}%
\pgfpathlineto{\pgfqpoint{6.092982in}{1.366784in}}%
\pgfpathlineto{\pgfqpoint{6.089810in}{1.367292in}}%
\pgfpathlineto{\pgfqpoint{6.086638in}{1.367530in}}%
\pgfpathlineto{\pgfqpoint{6.083466in}{1.367558in}}%
\pgfpathlineto{\pgfqpoint{6.080294in}{1.367195in}}%
\pgfpathlineto{\pgfqpoint{6.077122in}{1.367105in}}%
\pgfpathlineto{\pgfqpoint{6.073950in}{1.366745in}}%
\pgfpathlineto{\pgfqpoint{6.070778in}{1.366201in}}%
\pgfpathlineto{\pgfqpoint{6.067606in}{1.365952in}}%
\pgfpathlineto{\pgfqpoint{6.064434in}{1.365895in}}%
\pgfpathlineto{\pgfqpoint{6.061262in}{1.366052in}}%
\pgfpathlineto{\pgfqpoint{6.058090in}{1.365791in}}%
\pgfpathlineto{\pgfqpoint{6.054918in}{1.365513in}}%
\pgfpathlineto{\pgfqpoint{6.051746in}{1.365203in}}%
\pgfpathlineto{\pgfqpoint{6.048574in}{1.364992in}}%
\pgfpathlineto{\pgfqpoint{6.045402in}{1.364992in}}%
\pgfpathlineto{\pgfqpoint{6.042230in}{1.365037in}}%
\pgfpathlineto{\pgfqpoint{6.039058in}{1.364887in}}%
\pgfpathlineto{\pgfqpoint{6.035886in}{1.364545in}}%
\pgfpathlineto{\pgfqpoint{6.032714in}{1.364314in}}%
\pgfpathlineto{\pgfqpoint{6.029542in}{1.364034in}}%
\pgfpathlineto{\pgfqpoint{6.026370in}{1.363432in}}%
\pgfpathlineto{\pgfqpoint{6.023197in}{1.363043in}}%
\pgfpathlineto{\pgfqpoint{6.020025in}{1.363033in}}%
\pgfpathlineto{\pgfqpoint{6.016853in}{1.363193in}}%
\pgfpathlineto{\pgfqpoint{6.013681in}{1.362817in}}%
\pgfpathlineto{\pgfqpoint{6.010509in}{1.362095in}}%
\pgfpathlineto{\pgfqpoint{6.007337in}{1.361998in}}%
\pgfpathlineto{\pgfqpoint{6.004165in}{1.361445in}}%
\pgfpathlineto{\pgfqpoint{6.000993in}{1.361302in}}%
\pgfpathlineto{\pgfqpoint{5.997821in}{1.361458in}}%
\pgfpathlineto{\pgfqpoint{5.994649in}{1.362412in}}%
\pgfpathlineto{\pgfqpoint{5.991477in}{1.361990in}}%
\pgfpathlineto{\pgfqpoint{5.988305in}{1.362465in}}%
\pgfpathlineto{\pgfqpoint{5.985133in}{1.362807in}}%
\pgfpathlineto{\pgfqpoint{5.981961in}{1.362454in}}%
\pgfpathlineto{\pgfqpoint{5.978789in}{1.362445in}}%
\pgfpathlineto{\pgfqpoint{5.975617in}{1.362263in}}%
\pgfpathlineto{\pgfqpoint{5.972445in}{1.362434in}}%
\pgfpathlineto{\pgfqpoint{5.969273in}{1.362235in}}%
\pgfpathlineto{\pgfqpoint{5.966101in}{1.361875in}}%
\pgfpathlineto{\pgfqpoint{5.962929in}{1.362082in}}%
\pgfpathlineto{\pgfqpoint{5.959757in}{1.362142in}}%
\pgfpathlineto{\pgfqpoint{5.956585in}{1.362263in}}%
\pgfpathlineto{\pgfqpoint{5.953413in}{1.362201in}}%
\pgfpathlineto{\pgfqpoint{5.950241in}{1.362235in}}%
\pgfpathlineto{\pgfqpoint{5.947069in}{1.362223in}}%
\pgfpathlineto{\pgfqpoint{5.943896in}{1.361991in}}%
\pgfpathlineto{\pgfqpoint{5.940724in}{1.361526in}}%
\pgfpathlineto{\pgfqpoint{5.937552in}{1.361653in}}%
\pgfpathlineto{\pgfqpoint{5.934380in}{1.361415in}}%
\pgfpathlineto{\pgfqpoint{5.931208in}{1.361017in}}%
\pgfpathlineto{\pgfqpoint{5.928036in}{1.360984in}}%
\pgfpathlineto{\pgfqpoint{5.924864in}{1.360926in}}%
\pgfpathlineto{\pgfqpoint{5.921692in}{1.361063in}}%
\pgfpathlineto{\pgfqpoint{5.918520in}{1.360863in}}%
\pgfpathlineto{\pgfqpoint{5.915348in}{1.360410in}}%
\pgfpathlineto{\pgfqpoint{5.912176in}{1.360314in}}%
\pgfpathlineto{\pgfqpoint{5.909004in}{1.360157in}}%
\pgfpathlineto{\pgfqpoint{5.905832in}{1.359732in}}%
\pgfpathlineto{\pgfqpoint{5.902660in}{1.359531in}}%
\pgfpathlineto{\pgfqpoint{5.899488in}{1.359449in}}%
\pgfpathlineto{\pgfqpoint{5.896316in}{1.358935in}}%
\pgfpathlineto{\pgfqpoint{5.893144in}{1.358872in}}%
\pgfpathlineto{\pgfqpoint{5.889972in}{1.358759in}}%
\pgfpathlineto{\pgfqpoint{5.886800in}{1.358570in}}%
\pgfpathlineto{\pgfqpoint{5.883628in}{1.359143in}}%
\pgfpathlineto{\pgfqpoint{5.880456in}{1.359120in}}%
\pgfpathlineto{\pgfqpoint{5.877284in}{1.359652in}}%
\pgfpathlineto{\pgfqpoint{5.874112in}{1.359640in}}%
\pgfpathlineto{\pgfqpoint{5.870940in}{1.359165in}}%
\pgfpathlineto{\pgfqpoint{5.867768in}{1.358642in}}%
\pgfpathlineto{\pgfqpoint{5.864595in}{1.358625in}}%
\pgfpathlineto{\pgfqpoint{5.861423in}{1.358340in}}%
\pgfpathlineto{\pgfqpoint{5.858251in}{1.357741in}}%
\pgfpathlineto{\pgfqpoint{5.855079in}{1.357524in}}%
\pgfpathlineto{\pgfqpoint{5.851907in}{1.357619in}}%
\pgfpathlineto{\pgfqpoint{5.848735in}{1.357754in}}%
\pgfpathlineto{\pgfqpoint{5.845563in}{1.358027in}}%
\pgfpathlineto{\pgfqpoint{5.842391in}{1.358159in}}%
\pgfpathlineto{\pgfqpoint{5.839219in}{1.358382in}}%
\pgfpathlineto{\pgfqpoint{5.836047in}{1.358408in}}%
\pgfpathlineto{\pgfqpoint{5.832875in}{1.358691in}}%
\pgfpathlineto{\pgfqpoint{5.829703in}{1.358410in}}%
\pgfpathlineto{\pgfqpoint{5.826531in}{1.358360in}}%
\pgfpathlineto{\pgfqpoint{5.823359in}{1.358254in}}%
\pgfpathlineto{\pgfqpoint{5.820187in}{1.358318in}}%
\pgfpathlineto{\pgfqpoint{5.817015in}{1.358320in}}%
\pgfpathlineto{\pgfqpoint{5.813843in}{1.358049in}}%
\pgfpathlineto{\pgfqpoint{5.810671in}{1.357532in}}%
\pgfpathlineto{\pgfqpoint{5.807499in}{1.357990in}}%
\pgfpathlineto{\pgfqpoint{5.804327in}{1.357823in}}%
\pgfpathlineto{\pgfqpoint{5.801155in}{1.357920in}}%
\pgfpathlineto{\pgfqpoint{5.797983in}{1.357207in}}%
\pgfpathlineto{\pgfqpoint{5.794811in}{1.357323in}}%
\pgfpathlineto{\pgfqpoint{5.791639in}{1.357174in}}%
\pgfpathlineto{\pgfqpoint{5.788466in}{1.357224in}}%
\pgfpathlineto{\pgfqpoint{5.785294in}{1.357508in}}%
\pgfpathlineto{\pgfqpoint{5.782122in}{1.357499in}}%
\pgfpathlineto{\pgfqpoint{5.778950in}{1.357635in}}%
\pgfpathlineto{\pgfqpoint{5.775778in}{1.357348in}}%
\pgfpathlineto{\pgfqpoint{5.772606in}{1.357380in}}%
\pgfpathlineto{\pgfqpoint{5.769434in}{1.357452in}}%
\pgfpathlineto{\pgfqpoint{5.766262in}{1.357448in}}%
\pgfpathlineto{\pgfqpoint{5.763090in}{1.357110in}}%
\pgfpathlineto{\pgfqpoint{5.759918in}{1.357020in}}%
\pgfpathlineto{\pgfqpoint{5.756746in}{1.356901in}}%
\pgfpathlineto{\pgfqpoint{5.753574in}{1.356595in}}%
\pgfpathlineto{\pgfqpoint{5.750402in}{1.356307in}}%
\pgfpathlineto{\pgfqpoint{5.747230in}{1.355897in}}%
\pgfpathlineto{\pgfqpoint{5.744058in}{1.356120in}}%
\pgfpathlineto{\pgfqpoint{5.740886in}{1.355790in}}%
\pgfpathlineto{\pgfqpoint{5.737714in}{1.355561in}}%
\pgfpathlineto{\pgfqpoint{5.734542in}{1.355402in}}%
\pgfpathlineto{\pgfqpoint{5.731370in}{1.355531in}}%
\pgfpathlineto{\pgfqpoint{5.728198in}{1.355374in}}%
\pgfpathlineto{\pgfqpoint{5.725026in}{1.354694in}}%
\pgfpathlineto{\pgfqpoint{5.721854in}{1.354802in}}%
\pgfpathlineto{\pgfqpoint{5.718682in}{1.354450in}}%
\pgfpathlineto{\pgfqpoint{5.715510in}{1.354931in}}%
\pgfpathlineto{\pgfqpoint{5.712338in}{1.354922in}}%
\pgfpathlineto{\pgfqpoint{5.709165in}{1.354948in}}%
\pgfpathlineto{\pgfqpoint{5.705993in}{1.354940in}}%
\pgfpathlineto{\pgfqpoint{5.702821in}{1.354888in}}%
\pgfpathlineto{\pgfqpoint{5.699649in}{1.354699in}}%
\pgfpathlineto{\pgfqpoint{5.696477in}{1.354738in}}%
\pgfpathlineto{\pgfqpoint{5.693305in}{1.354875in}}%
\pgfpathlineto{\pgfqpoint{5.690133in}{1.354477in}}%
\pgfpathlineto{\pgfqpoint{5.686961in}{1.354366in}}%
\pgfpathlineto{\pgfqpoint{5.683789in}{1.354526in}}%
\pgfpathlineto{\pgfqpoint{5.680617in}{1.354320in}}%
\pgfpathlineto{\pgfqpoint{5.677445in}{1.354113in}}%
\pgfpathlineto{\pgfqpoint{5.674273in}{1.353827in}}%
\pgfpathlineto{\pgfqpoint{5.671101in}{1.353800in}}%
\pgfpathlineto{\pgfqpoint{5.667929in}{1.353684in}}%
\pgfpathlineto{\pgfqpoint{5.664757in}{1.354062in}}%
\pgfpathlineto{\pgfqpoint{5.661585in}{1.354432in}}%
\pgfpathlineto{\pgfqpoint{5.658413in}{1.354502in}}%
\pgfpathlineto{\pgfqpoint{5.655241in}{1.354682in}}%
\pgfpathlineto{\pgfqpoint{5.652069in}{1.354044in}}%
\pgfpathlineto{\pgfqpoint{5.648897in}{1.354112in}}%
\pgfpathlineto{\pgfqpoint{5.645725in}{1.353906in}}%
\pgfpathlineto{\pgfqpoint{5.642553in}{1.353331in}}%
\pgfpathlineto{\pgfqpoint{5.639381in}{1.353632in}}%
\pgfpathlineto{\pgfqpoint{5.636209in}{1.353701in}}%
\pgfpathlineto{\pgfqpoint{5.633037in}{1.353745in}}%
\pgfpathlineto{\pgfqpoint{5.629864in}{1.353975in}}%
\pgfpathlineto{\pgfqpoint{5.626692in}{1.353880in}}%
\pgfpathlineto{\pgfqpoint{5.623520in}{1.353313in}}%
\pgfpathlineto{\pgfqpoint{5.620348in}{1.352987in}}%
\pgfpathlineto{\pgfqpoint{5.617176in}{1.353221in}}%
\pgfpathlineto{\pgfqpoint{5.614004in}{1.353446in}}%
\pgfpathlineto{\pgfqpoint{5.610832in}{1.353360in}}%
\pgfpathlineto{\pgfqpoint{5.607660in}{1.353368in}}%
\pgfpathlineto{\pgfqpoint{5.604488in}{1.352525in}}%
\pgfpathlineto{\pgfqpoint{5.601316in}{1.352239in}}%
\pgfpathlineto{\pgfqpoint{5.598144in}{1.352879in}}%
\pgfpathlineto{\pgfqpoint{5.594972in}{1.353221in}}%
\pgfpathlineto{\pgfqpoint{5.591800in}{1.353512in}}%
\pgfpathlineto{\pgfqpoint{5.588628in}{1.353331in}}%
\pgfpathlineto{\pgfqpoint{5.585456in}{1.352817in}}%
\pgfpathlineto{\pgfqpoint{5.582284in}{1.352715in}}%
\pgfpathlineto{\pgfqpoint{5.579112in}{1.352847in}}%
\pgfpathlineto{\pgfqpoint{5.575940in}{1.352663in}}%
\pgfpathlineto{\pgfqpoint{5.572768in}{1.352276in}}%
\pgfpathlineto{\pgfqpoint{5.569596in}{1.352304in}}%
\pgfpathlineto{\pgfqpoint{5.566424in}{1.352044in}}%
\pgfpathlineto{\pgfqpoint{5.563252in}{1.352259in}}%
\pgfpathlineto{\pgfqpoint{5.560080in}{1.352288in}}%
\pgfpathlineto{\pgfqpoint{5.556908in}{1.352315in}}%
\pgfpathlineto{\pgfqpoint{5.553735in}{1.352289in}}%
\pgfpathlineto{\pgfqpoint{5.550563in}{1.352214in}}%
\pgfpathlineto{\pgfqpoint{5.547391in}{1.351988in}}%
\pgfpathlineto{\pgfqpoint{5.544219in}{1.352203in}}%
\pgfpathlineto{\pgfqpoint{5.541047in}{1.352094in}}%
\pgfpathlineto{\pgfqpoint{5.537875in}{1.352781in}}%
\pgfpathlineto{\pgfqpoint{5.534703in}{1.352633in}}%
\pgfpathlineto{\pgfqpoint{5.531531in}{1.352736in}}%
\pgfpathlineto{\pgfqpoint{5.528359in}{1.352557in}}%
\pgfpathlineto{\pgfqpoint{5.525187in}{1.352432in}}%
\pgfpathlineto{\pgfqpoint{5.522015in}{1.352143in}}%
\pgfpathlineto{\pgfqpoint{5.518843in}{1.352355in}}%
\pgfpathlineto{\pgfqpoint{5.515671in}{1.351878in}}%
\pgfpathlineto{\pgfqpoint{5.512499in}{1.352109in}}%
\pgfpathlineto{\pgfqpoint{5.509327in}{1.352333in}}%
\pgfpathlineto{\pgfqpoint{5.506155in}{1.352529in}}%
\pgfpathlineto{\pgfqpoint{5.502983in}{1.352287in}}%
\pgfpathlineto{\pgfqpoint{5.499811in}{1.351744in}}%
\pgfpathlineto{\pgfqpoint{5.496639in}{1.351806in}}%
\pgfpathlineto{\pgfqpoint{5.493467in}{1.351641in}}%
\pgfpathlineto{\pgfqpoint{5.490295in}{1.351681in}}%
\pgfpathlineto{\pgfqpoint{5.487123in}{1.351926in}}%
\pgfpathlineto{\pgfqpoint{5.483951in}{1.352139in}}%
\pgfpathlineto{\pgfqpoint{5.480779in}{1.352152in}}%
\pgfpathlineto{\pgfqpoint{5.477607in}{1.352457in}}%
\pgfpathlineto{\pgfqpoint{5.474434in}{1.352635in}}%
\pgfpathlineto{\pgfqpoint{5.471262in}{1.352509in}}%
\pgfpathlineto{\pgfqpoint{5.468090in}{1.352334in}}%
\pgfpathlineto{\pgfqpoint{5.464918in}{1.351932in}}%
\pgfpathlineto{\pgfqpoint{5.461746in}{1.351471in}}%
\pgfpathlineto{\pgfqpoint{5.458574in}{1.351177in}}%
\pgfpathlineto{\pgfqpoint{5.455402in}{1.350909in}}%
\pgfpathlineto{\pgfqpoint{5.452230in}{1.350765in}}%
\pgfpathlineto{\pgfqpoint{5.449058in}{1.350651in}}%
\pgfpathlineto{\pgfqpoint{5.445886in}{1.350148in}}%
\pgfpathlineto{\pgfqpoint{5.442714in}{1.350012in}}%
\pgfpathlineto{\pgfqpoint{5.439542in}{1.349626in}}%
\pgfpathlineto{\pgfqpoint{5.436370in}{1.349136in}}%
\pgfpathlineto{\pgfqpoint{5.433198in}{1.348995in}}%
\pgfpathlineto{\pgfqpoint{5.430026in}{1.349160in}}%
\pgfpathlineto{\pgfqpoint{5.426854in}{1.349124in}}%
\pgfpathlineto{\pgfqpoint{5.423682in}{1.349029in}}%
\pgfpathlineto{\pgfqpoint{5.420510in}{1.348819in}}%
\pgfpathlineto{\pgfqpoint{5.417338in}{1.348462in}}%
\pgfpathlineto{\pgfqpoint{5.414166in}{1.348412in}}%
\pgfpathlineto{\pgfqpoint{5.410994in}{1.348634in}}%
\pgfpathlineto{\pgfqpoint{5.407822in}{1.348513in}}%
\pgfpathlineto{\pgfqpoint{5.404650in}{1.348069in}}%
\pgfpathlineto{\pgfqpoint{5.401478in}{1.348062in}}%
\pgfpathlineto{\pgfqpoint{5.398306in}{1.348121in}}%
\pgfpathlineto{\pgfqpoint{5.395133in}{1.347935in}}%
\pgfpathlineto{\pgfqpoint{5.391961in}{1.347707in}}%
\pgfpathlineto{\pgfqpoint{5.388789in}{1.347747in}}%
\pgfpathlineto{\pgfqpoint{5.385617in}{1.347350in}}%
\pgfpathlineto{\pgfqpoint{5.382445in}{1.347287in}}%
\pgfpathlineto{\pgfqpoint{5.379273in}{1.347339in}}%
\pgfpathlineto{\pgfqpoint{5.376101in}{1.347038in}}%
\pgfpathlineto{\pgfqpoint{5.372929in}{1.347170in}}%
\pgfpathlineto{\pgfqpoint{5.369757in}{1.346580in}}%
\pgfpathlineto{\pgfqpoint{5.366585in}{1.346316in}}%
\pgfpathlineto{\pgfqpoint{5.363413in}{1.346797in}}%
\pgfpathlineto{\pgfqpoint{5.360241in}{1.347134in}}%
\pgfpathlineto{\pgfqpoint{5.357069in}{1.346056in}}%
\pgfpathlineto{\pgfqpoint{5.353897in}{1.345680in}}%
\pgfpathlineto{\pgfqpoint{5.350725in}{1.345439in}}%
\pgfpathlineto{\pgfqpoint{5.347553in}{1.345265in}}%
\pgfpathlineto{\pgfqpoint{5.344381in}{1.345391in}}%
\pgfpathlineto{\pgfqpoint{5.341209in}{1.345326in}}%
\pgfpathlineto{\pgfqpoint{5.338037in}{1.344977in}}%
\pgfpathlineto{\pgfqpoint{5.334865in}{1.344359in}}%
\pgfpathlineto{\pgfqpoint{5.331693in}{1.343888in}}%
\pgfpathlineto{\pgfqpoint{5.328521in}{1.343251in}}%
\pgfpathlineto{\pgfqpoint{5.325349in}{1.343198in}}%
\pgfpathlineto{\pgfqpoint{5.322177in}{1.343318in}}%
\pgfpathlineto{\pgfqpoint{5.319004in}{1.343337in}}%
\pgfpathlineto{\pgfqpoint{5.315832in}{1.342940in}}%
\pgfpathlineto{\pgfqpoint{5.312660in}{1.343036in}}%
\pgfpathlineto{\pgfqpoint{5.309488in}{1.341983in}}%
\pgfpathlineto{\pgfqpoint{5.306316in}{1.341669in}}%
\pgfpathlineto{\pgfqpoint{5.303144in}{1.341212in}}%
\pgfpathlineto{\pgfqpoint{5.299972in}{1.340897in}}%
\pgfpathlineto{\pgfqpoint{5.296800in}{1.340728in}}%
\pgfpathlineto{\pgfqpoint{5.293628in}{1.340680in}}%
\pgfpathlineto{\pgfqpoint{5.290456in}{1.340347in}}%
\pgfpathlineto{\pgfqpoint{5.287284in}{1.340237in}}%
\pgfpathlineto{\pgfqpoint{5.284112in}{1.340406in}}%
\pgfpathlineto{\pgfqpoint{5.280940in}{1.340418in}}%
\pgfpathlineto{\pgfqpoint{5.277768in}{1.340344in}}%
\pgfpathlineto{\pgfqpoint{5.274596in}{1.340215in}}%
\pgfpathlineto{\pgfqpoint{5.271424in}{1.340070in}}%
\pgfpathlineto{\pgfqpoint{5.268252in}{1.339972in}}%
\pgfpathlineto{\pgfqpoint{5.265080in}{1.339854in}}%
\pgfpathlineto{\pgfqpoint{5.261908in}{1.339617in}}%
\pgfpathlineto{\pgfqpoint{5.258736in}{1.339461in}}%
\pgfpathlineto{\pgfqpoint{5.255564in}{1.338934in}}%
\pgfpathlineto{\pgfqpoint{5.252392in}{1.338599in}}%
\pgfpathlineto{\pgfqpoint{5.249220in}{1.338184in}}%
\pgfpathlineto{\pgfqpoint{5.246048in}{1.338077in}}%
\pgfpathlineto{\pgfqpoint{5.242876in}{1.337820in}}%
\pgfpathlineto{\pgfqpoint{5.239703in}{1.337641in}}%
\pgfpathlineto{\pgfqpoint{5.236531in}{1.337023in}}%
\pgfpathlineto{\pgfqpoint{5.233359in}{1.337222in}}%
\pgfpathlineto{\pgfqpoint{5.230187in}{1.337326in}}%
\pgfpathlineto{\pgfqpoint{5.227015in}{1.337355in}}%
\pgfpathlineto{\pgfqpoint{5.223843in}{1.337353in}}%
\pgfpathlineto{\pgfqpoint{5.220671in}{1.337764in}}%
\pgfpathlineto{\pgfqpoint{5.217499in}{1.337953in}}%
\pgfpathlineto{\pgfqpoint{5.214327in}{1.338122in}}%
\pgfpathlineto{\pgfqpoint{5.211155in}{1.337979in}}%
\pgfpathlineto{\pgfqpoint{5.207983in}{1.338035in}}%
\pgfpathlineto{\pgfqpoint{5.204811in}{1.337574in}}%
\pgfpathlineto{\pgfqpoint{5.201639in}{1.337228in}}%
\pgfpathlineto{\pgfqpoint{5.198467in}{1.337647in}}%
\pgfpathlineto{\pgfqpoint{5.195295in}{1.337283in}}%
\pgfpathlineto{\pgfqpoint{5.192123in}{1.337563in}}%
\pgfpathlineto{\pgfqpoint{5.188951in}{1.337870in}}%
\pgfpathlineto{\pgfqpoint{5.185779in}{1.337833in}}%
\pgfpathlineto{\pgfqpoint{5.182607in}{1.337717in}}%
\pgfpathlineto{\pgfqpoint{5.179435in}{1.337894in}}%
\pgfpathlineto{\pgfqpoint{5.176263in}{1.337795in}}%
\pgfpathlineto{\pgfqpoint{5.173091in}{1.337859in}}%
\pgfpathlineto{\pgfqpoint{5.169919in}{1.338306in}}%
\pgfpathlineto{\pgfqpoint{5.166747in}{1.337974in}}%
\pgfpathlineto{\pgfqpoint{5.163575in}{1.337762in}}%
\pgfpathlineto{\pgfqpoint{5.160402in}{1.337829in}}%
\pgfpathlineto{\pgfqpoint{5.157230in}{1.337418in}}%
\pgfpathlineto{\pgfqpoint{5.154058in}{1.337247in}}%
\pgfpathlineto{\pgfqpoint{5.150886in}{1.337192in}}%
\pgfpathlineto{\pgfqpoint{5.147714in}{1.337072in}}%
\pgfpathlineto{\pgfqpoint{5.144542in}{1.337287in}}%
\pgfpathlineto{\pgfqpoint{5.141370in}{1.337230in}}%
\pgfpathlineto{\pgfqpoint{5.138198in}{1.337313in}}%
\pgfpathlineto{\pgfqpoint{5.135026in}{1.337529in}}%
\pgfpathlineto{\pgfqpoint{5.131854in}{1.337416in}}%
\pgfpathlineto{\pgfqpoint{5.128682in}{1.337199in}}%
\pgfpathlineto{\pgfqpoint{5.125510in}{1.336940in}}%
\pgfpathlineto{\pgfqpoint{5.122338in}{1.336484in}}%
\pgfpathlineto{\pgfqpoint{5.119166in}{1.336517in}}%
\pgfpathlineto{\pgfqpoint{5.115994in}{1.336572in}}%
\pgfpathlineto{\pgfqpoint{5.112822in}{1.336541in}}%
\pgfpathlineto{\pgfqpoint{5.109650in}{1.336877in}}%
\pgfpathlineto{\pgfqpoint{5.106478in}{1.336744in}}%
\pgfpathlineto{\pgfqpoint{5.103306in}{1.336399in}}%
\pgfpathlineto{\pgfqpoint{5.100134in}{1.336164in}}%
\pgfpathlineto{\pgfqpoint{5.096962in}{1.336153in}}%
\pgfpathlineto{\pgfqpoint{5.093790in}{1.336295in}}%
\pgfpathlineto{\pgfqpoint{5.090618in}{1.336335in}}%
\pgfpathlineto{\pgfqpoint{5.087446in}{1.336251in}}%
\pgfpathlineto{\pgfqpoint{5.084273in}{1.336352in}}%
\pgfpathlineto{\pgfqpoint{5.081101in}{1.336243in}}%
\pgfpathlineto{\pgfqpoint{5.077929in}{1.335936in}}%
\pgfpathlineto{\pgfqpoint{5.074757in}{1.336093in}}%
\pgfpathlineto{\pgfqpoint{5.071585in}{1.335733in}}%
\pgfpathlineto{\pgfqpoint{5.068413in}{1.335635in}}%
\pgfpathlineto{\pgfqpoint{5.065241in}{1.335417in}}%
\pgfpathlineto{\pgfqpoint{5.062069in}{1.335266in}}%
\pgfpathlineto{\pgfqpoint{5.058897in}{1.334983in}}%
\pgfpathlineto{\pgfqpoint{5.055725in}{1.334863in}}%
\pgfpathlineto{\pgfqpoint{5.052553in}{1.335006in}}%
\pgfpathlineto{\pgfqpoint{5.049381in}{1.335312in}}%
\pgfpathlineto{\pgfqpoint{5.046209in}{1.335850in}}%
\pgfpathlineto{\pgfqpoint{5.043037in}{1.335684in}}%
\pgfpathlineto{\pgfqpoint{5.039865in}{1.335496in}}%
\pgfpathlineto{\pgfqpoint{5.036693in}{1.335234in}}%
\pgfpathlineto{\pgfqpoint{5.033521in}{1.334889in}}%
\pgfpathlineto{\pgfqpoint{5.030349in}{1.334764in}}%
\pgfpathlineto{\pgfqpoint{5.027177in}{1.334393in}}%
\pgfpathlineto{\pgfqpoint{5.024005in}{1.334039in}}%
\pgfpathlineto{\pgfqpoint{5.020833in}{1.334118in}}%
\pgfpathlineto{\pgfqpoint{5.017661in}{1.334168in}}%
\pgfpathlineto{\pgfqpoint{5.014489in}{1.333970in}}%
\pgfpathlineto{\pgfqpoint{5.011317in}{1.333908in}}%
\pgfpathlineto{\pgfqpoint{5.008145in}{1.333764in}}%
\pgfpathlineto{\pgfqpoint{5.004972in}{1.333946in}}%
\pgfpathlineto{\pgfqpoint{5.001800in}{1.333797in}}%
\pgfpathlineto{\pgfqpoint{4.998628in}{1.333630in}}%
\pgfpathlineto{\pgfqpoint{4.995456in}{1.333770in}}%
\pgfpathlineto{\pgfqpoint{4.992284in}{1.333845in}}%
\pgfpathlineto{\pgfqpoint{4.989112in}{1.333022in}}%
\pgfpathlineto{\pgfqpoint{4.985940in}{1.333367in}}%
\pgfpathlineto{\pgfqpoint{4.982768in}{1.332923in}}%
\pgfpathlineto{\pgfqpoint{4.979596in}{1.332713in}}%
\pgfpathlineto{\pgfqpoint{4.976424in}{1.332694in}}%
\pgfpathlineto{\pgfqpoint{4.973252in}{1.332508in}}%
\pgfpathlineto{\pgfqpoint{4.970080in}{1.332385in}}%
\pgfpathlineto{\pgfqpoint{4.966908in}{1.332125in}}%
\pgfpathlineto{\pgfqpoint{4.963736in}{1.331259in}}%
\pgfpathlineto{\pgfqpoint{4.960564in}{1.330934in}}%
\pgfpathlineto{\pgfqpoint{4.957392in}{1.330955in}}%
\pgfpathlineto{\pgfqpoint{4.954220in}{1.330291in}}%
\pgfpathlineto{\pgfqpoint{4.951048in}{1.330206in}}%
\pgfpathlineto{\pgfqpoint{4.947876in}{1.330280in}}%
\pgfpathlineto{\pgfqpoint{4.944704in}{1.330119in}}%
\pgfpathlineto{\pgfqpoint{4.941532in}{1.330048in}}%
\pgfpathlineto{\pgfqpoint{4.938360in}{1.330021in}}%
\pgfpathlineto{\pgfqpoint{4.935188in}{1.329830in}}%
\pgfpathlineto{\pgfqpoint{4.932016in}{1.329713in}}%
\pgfpathlineto{\pgfqpoint{4.928844in}{1.329248in}}%
\pgfpathlineto{\pgfqpoint{4.925671in}{1.329358in}}%
\pgfpathlineto{\pgfqpoint{4.922499in}{1.329206in}}%
\pgfpathlineto{\pgfqpoint{4.919327in}{1.329333in}}%
\pgfpathlineto{\pgfqpoint{4.916155in}{1.329541in}}%
\pgfpathlineto{\pgfqpoint{4.912983in}{1.329531in}}%
\pgfpathlineto{\pgfqpoint{4.909811in}{1.329486in}}%
\pgfpathlineto{\pgfqpoint{4.906639in}{1.329298in}}%
\pgfpathlineto{\pgfqpoint{4.903467in}{1.329033in}}%
\pgfpathlineto{\pgfqpoint{4.900295in}{1.328665in}}%
\pgfpathlineto{\pgfqpoint{4.897123in}{1.328360in}}%
\pgfpathlineto{\pgfqpoint{4.893951in}{1.327691in}}%
\pgfpathlineto{\pgfqpoint{4.890779in}{1.327300in}}%
\pgfpathlineto{\pgfqpoint{4.887607in}{1.327420in}}%
\pgfpathlineto{\pgfqpoint{4.884435in}{1.326976in}}%
\pgfpathlineto{\pgfqpoint{4.881263in}{1.326965in}}%
\pgfpathlineto{\pgfqpoint{4.878091in}{1.326813in}}%
\pgfpathlineto{\pgfqpoint{4.874919in}{1.326917in}}%
\pgfpathlineto{\pgfqpoint{4.871747in}{1.327010in}}%
\pgfpathlineto{\pgfqpoint{4.868575in}{1.326935in}}%
\pgfpathlineto{\pgfqpoint{4.865403in}{1.326886in}}%
\pgfpathlineto{\pgfqpoint{4.862231in}{1.326363in}}%
\pgfpathlineto{\pgfqpoint{4.859059in}{1.326512in}}%
\pgfpathlineto{\pgfqpoint{4.855887in}{1.326416in}}%
\pgfpathlineto{\pgfqpoint{4.852715in}{1.325874in}}%
\pgfpathlineto{\pgfqpoint{4.849542in}{1.325853in}}%
\pgfpathlineto{\pgfqpoint{4.846370in}{1.325485in}}%
\pgfpathlineto{\pgfqpoint{4.843198in}{1.325155in}}%
\pgfpathlineto{\pgfqpoint{4.840026in}{1.324744in}}%
\pgfpathlineto{\pgfqpoint{4.836854in}{1.324782in}}%
\pgfpathlineto{\pgfqpoint{4.833682in}{1.325055in}}%
\pgfpathlineto{\pgfqpoint{4.830510in}{1.324751in}}%
\pgfpathlineto{\pgfqpoint{4.827338in}{1.324377in}}%
\pgfpathlineto{\pgfqpoint{4.824166in}{1.324139in}}%
\pgfpathlineto{\pgfqpoint{4.820994in}{1.323838in}}%
\pgfpathlineto{\pgfqpoint{4.817822in}{1.323580in}}%
\pgfpathlineto{\pgfqpoint{4.814650in}{1.323112in}}%
\pgfpathlineto{\pgfqpoint{4.811478in}{1.323216in}}%
\pgfpathlineto{\pgfqpoint{4.808306in}{1.322698in}}%
\pgfpathlineto{\pgfqpoint{4.805134in}{1.322693in}}%
\pgfpathlineto{\pgfqpoint{4.801962in}{1.322604in}}%
\pgfpathlineto{\pgfqpoint{4.798790in}{1.322542in}}%
\pgfpathlineto{\pgfqpoint{4.795618in}{1.322592in}}%
\pgfpathlineto{\pgfqpoint{4.792446in}{1.322480in}}%
\pgfpathlineto{\pgfqpoint{4.789274in}{1.321770in}}%
\pgfpathlineto{\pgfqpoint{4.786102in}{1.321766in}}%
\pgfpathlineto{\pgfqpoint{4.782930in}{1.322069in}}%
\pgfpathlineto{\pgfqpoint{4.779758in}{1.322615in}}%
\pgfpathlineto{\pgfqpoint{4.776586in}{1.322658in}}%
\pgfpathlineto{\pgfqpoint{4.773414in}{1.322576in}}%
\pgfpathlineto{\pgfqpoint{4.770241in}{1.322089in}}%
\pgfpathlineto{\pgfqpoint{4.767069in}{1.321741in}}%
\pgfpathlineto{\pgfqpoint{4.763897in}{1.321573in}}%
\pgfpathlineto{\pgfqpoint{4.760725in}{1.321489in}}%
\pgfpathlineto{\pgfqpoint{4.757553in}{1.321653in}}%
\pgfpathlineto{\pgfqpoint{4.754381in}{1.321422in}}%
\pgfpathlineto{\pgfqpoint{4.751209in}{1.321166in}}%
\pgfpathlineto{\pgfqpoint{4.748037in}{1.320472in}}%
\pgfpathlineto{\pgfqpoint{4.744865in}{1.320276in}}%
\pgfpathlineto{\pgfqpoint{4.741693in}{1.320070in}}%
\pgfpathlineto{\pgfqpoint{4.738521in}{1.319363in}}%
\pgfpathlineto{\pgfqpoint{4.735349in}{1.318945in}}%
\pgfpathlineto{\pgfqpoint{4.732177in}{1.318778in}}%
\pgfpathlineto{\pgfqpoint{4.729005in}{1.319143in}}%
\pgfpathlineto{\pgfqpoint{4.725833in}{1.318786in}}%
\pgfpathlineto{\pgfqpoint{4.722661in}{1.318644in}}%
\pgfpathlineto{\pgfqpoint{4.719489in}{1.318858in}}%
\pgfpathlineto{\pgfqpoint{4.716317in}{1.318842in}}%
\pgfpathlineto{\pgfqpoint{4.713145in}{1.319051in}}%
\pgfpathlineto{\pgfqpoint{4.709973in}{1.319143in}}%
\pgfpathlineto{\pgfqpoint{4.706801in}{1.319077in}}%
\pgfpathlineto{\pgfqpoint{4.703629in}{1.318900in}}%
\pgfpathlineto{\pgfqpoint{4.700457in}{1.318858in}}%
\pgfpathlineto{\pgfqpoint{4.697285in}{1.318750in}}%
\pgfpathlineto{\pgfqpoint{4.694112in}{1.318242in}}%
\pgfpathlineto{\pgfqpoint{4.690940in}{1.317927in}}%
\pgfpathlineto{\pgfqpoint{4.687768in}{1.317912in}}%
\pgfpathlineto{\pgfqpoint{4.684596in}{1.317503in}}%
\pgfpathlineto{\pgfqpoint{4.681424in}{1.317055in}}%
\pgfpathlineto{\pgfqpoint{4.678252in}{1.317131in}}%
\pgfpathlineto{\pgfqpoint{4.675080in}{1.317071in}}%
\pgfpathlineto{\pgfqpoint{4.671908in}{1.316481in}}%
\pgfpathlineto{\pgfqpoint{4.668736in}{1.316499in}}%
\pgfpathlineto{\pgfqpoint{4.665564in}{1.316081in}}%
\pgfpathlineto{\pgfqpoint{4.662392in}{1.315775in}}%
\pgfpathlineto{\pgfqpoint{4.659220in}{1.315241in}}%
\pgfpathlineto{\pgfqpoint{4.656048in}{1.315140in}}%
\pgfpathlineto{\pgfqpoint{4.652876in}{1.315404in}}%
\pgfpathlineto{\pgfqpoint{4.649704in}{1.315775in}}%
\pgfpathlineto{\pgfqpoint{4.646532in}{1.315262in}}%
\pgfpathlineto{\pgfqpoint{4.643360in}{1.315265in}}%
\pgfpathlineto{\pgfqpoint{4.640188in}{1.315296in}}%
\pgfpathlineto{\pgfqpoint{4.637016in}{1.315310in}}%
\pgfpathlineto{\pgfqpoint{4.633844in}{1.315268in}}%
\pgfpathlineto{\pgfqpoint{4.630672in}{1.315255in}}%
\pgfpathlineto{\pgfqpoint{4.627500in}{1.315269in}}%
\pgfpathlineto{\pgfqpoint{4.624328in}{1.315128in}}%
\pgfpathlineto{\pgfqpoint{4.621156in}{1.315147in}}%
\pgfpathlineto{\pgfqpoint{4.617984in}{1.315394in}}%
\pgfpathlineto{\pgfqpoint{4.614811in}{1.315431in}}%
\pgfpathlineto{\pgfqpoint{4.611639in}{1.315514in}}%
\pgfpathlineto{\pgfqpoint{4.608467in}{1.315619in}}%
\pgfpathlineto{\pgfqpoint{4.605295in}{1.315269in}}%
\pgfpathlineto{\pgfqpoint{4.602123in}{1.315709in}}%
\pgfpathlineto{\pgfqpoint{4.598951in}{1.315425in}}%
\pgfpathlineto{\pgfqpoint{4.595779in}{1.315350in}}%
\pgfpathlineto{\pgfqpoint{4.592607in}{1.314490in}}%
\pgfpathlineto{\pgfqpoint{4.589435in}{1.313915in}}%
\pgfpathlineto{\pgfqpoint{4.586263in}{1.313390in}}%
\pgfpathlineto{\pgfqpoint{4.583091in}{1.313238in}}%
\pgfpathlineto{\pgfqpoint{4.579919in}{1.312535in}}%
\pgfpathlineto{\pgfqpoint{4.576747in}{1.312528in}}%
\pgfpathlineto{\pgfqpoint{4.573575in}{1.312821in}}%
\pgfpathlineto{\pgfqpoint{4.570403in}{1.312803in}}%
\pgfpathlineto{\pgfqpoint{4.567231in}{1.312533in}}%
\pgfpathlineto{\pgfqpoint{4.564059in}{1.312330in}}%
\pgfpathlineto{\pgfqpoint{4.560887in}{1.311944in}}%
\pgfpathlineto{\pgfqpoint{4.557715in}{1.312196in}}%
\pgfpathlineto{\pgfqpoint{4.554543in}{1.312338in}}%
\pgfpathlineto{\pgfqpoint{4.551371in}{1.311871in}}%
\pgfpathlineto{\pgfqpoint{4.548199in}{1.312022in}}%
\pgfpathlineto{\pgfqpoint{4.545027in}{1.312117in}}%
\pgfpathlineto{\pgfqpoint{4.541855in}{1.312092in}}%
\pgfpathlineto{\pgfqpoint{4.538683in}{1.311912in}}%
\pgfpathlineto{\pgfqpoint{4.535510in}{1.311545in}}%
\pgfpathlineto{\pgfqpoint{4.532338in}{1.311769in}}%
\pgfpathlineto{\pgfqpoint{4.529166in}{1.311575in}}%
\pgfpathlineto{\pgfqpoint{4.525994in}{1.311394in}}%
\pgfpathlineto{\pgfqpoint{4.522822in}{1.311523in}}%
\pgfpathlineto{\pgfqpoint{4.519650in}{1.311500in}}%
\pgfpathlineto{\pgfqpoint{4.516478in}{1.311318in}}%
\pgfpathlineto{\pgfqpoint{4.513306in}{1.311594in}}%
\pgfpathlineto{\pgfqpoint{4.510134in}{1.311683in}}%
\pgfpathlineto{\pgfqpoint{4.506962in}{1.311709in}}%
\pgfpathlineto{\pgfqpoint{4.503790in}{1.311566in}}%
\pgfpathlineto{\pgfqpoint{4.500618in}{1.311055in}}%
\pgfpathlineto{\pgfqpoint{4.497446in}{1.310680in}}%
\pgfpathlineto{\pgfqpoint{4.494274in}{1.310120in}}%
\pgfpathlineto{\pgfqpoint{4.491102in}{1.309940in}}%
\pgfpathlineto{\pgfqpoint{4.487930in}{1.309667in}}%
\pgfpathlineto{\pgfqpoint{4.484758in}{1.309227in}}%
\pgfpathlineto{\pgfqpoint{4.481586in}{1.309021in}}%
\pgfpathlineto{\pgfqpoint{4.478414in}{1.308861in}}%
\pgfpathlineto{\pgfqpoint{4.475242in}{1.308976in}}%
\pgfpathlineto{\pgfqpoint{4.472070in}{1.309192in}}%
\pgfpathlineto{\pgfqpoint{4.468898in}{1.308943in}}%
\pgfpathlineto{\pgfqpoint{4.465726in}{1.308740in}}%
\pgfpathlineto{\pgfqpoint{4.462554in}{1.308362in}}%
\pgfpathlineto{\pgfqpoint{4.459381in}{1.308473in}}%
\pgfpathlineto{\pgfqpoint{4.456209in}{1.308238in}}%
\pgfpathlineto{\pgfqpoint{4.453037in}{1.308503in}}%
\pgfpathlineto{\pgfqpoint{4.449865in}{1.308593in}}%
\pgfpathlineto{\pgfqpoint{4.446693in}{1.308516in}}%
\pgfpathlineto{\pgfqpoint{4.443521in}{1.308735in}}%
\pgfpathlineto{\pgfqpoint{4.440349in}{1.308583in}}%
\pgfpathlineto{\pgfqpoint{4.437177in}{1.308410in}}%
\pgfpathlineto{\pgfqpoint{4.434005in}{1.308519in}}%
\pgfpathlineto{\pgfqpoint{4.430833in}{1.308459in}}%
\pgfpathlineto{\pgfqpoint{4.427661in}{1.308137in}}%
\pgfpathlineto{\pgfqpoint{4.424489in}{1.307941in}}%
\pgfpathlineto{\pgfqpoint{4.421317in}{1.307707in}}%
\pgfpathlineto{\pgfqpoint{4.418145in}{1.307377in}}%
\pgfpathlineto{\pgfqpoint{4.414973in}{1.307376in}}%
\pgfpathlineto{\pgfqpoint{4.411801in}{1.307748in}}%
\pgfpathlineto{\pgfqpoint{4.408629in}{1.307548in}}%
\pgfpathlineto{\pgfqpoint{4.405457in}{1.306870in}}%
\pgfpathlineto{\pgfqpoint{4.402285in}{1.306942in}}%
\pgfpathlineto{\pgfqpoint{4.399113in}{1.306716in}}%
\pgfpathlineto{\pgfqpoint{4.395941in}{1.306472in}}%
\pgfpathlineto{\pgfqpoint{4.392769in}{1.306064in}}%
\pgfpathlineto{\pgfqpoint{4.389597in}{1.306279in}}%
\pgfpathlineto{\pgfqpoint{4.386425in}{1.306193in}}%
\pgfpathlineto{\pgfqpoint{4.383253in}{1.306328in}}%
\pgfpathlineto{\pgfqpoint{4.380080in}{1.305901in}}%
\pgfpathlineto{\pgfqpoint{4.376908in}{1.305620in}}%
\pgfpathlineto{\pgfqpoint{4.373736in}{1.305270in}}%
\pgfpathlineto{\pgfqpoint{4.370564in}{1.305059in}}%
\pgfpathlineto{\pgfqpoint{4.367392in}{1.305062in}}%
\pgfpathlineto{\pgfqpoint{4.364220in}{1.304891in}}%
\pgfpathlineto{\pgfqpoint{4.361048in}{1.304512in}}%
\pgfpathlineto{\pgfqpoint{4.357876in}{1.304731in}}%
\pgfpathlineto{\pgfqpoint{4.354704in}{1.304841in}}%
\pgfpathlineto{\pgfqpoint{4.351532in}{1.304344in}}%
\pgfpathlineto{\pgfqpoint{4.348360in}{1.303969in}}%
\pgfpathlineto{\pgfqpoint{4.345188in}{1.304100in}}%
\pgfpathlineto{\pgfqpoint{4.342016in}{1.303602in}}%
\pgfpathlineto{\pgfqpoint{4.338844in}{1.303527in}}%
\pgfpathlineto{\pgfqpoint{4.335672in}{1.303100in}}%
\pgfpathlineto{\pgfqpoint{4.332500in}{1.303215in}}%
\pgfpathlineto{\pgfqpoint{4.329328in}{1.302516in}}%
\pgfpathlineto{\pgfqpoint{4.326156in}{1.302048in}}%
\pgfpathlineto{\pgfqpoint{4.322984in}{1.301990in}}%
\pgfpathlineto{\pgfqpoint{4.319812in}{1.301954in}}%
\pgfpathlineto{\pgfqpoint{4.316640in}{1.301589in}}%
\pgfpathlineto{\pgfqpoint{4.313468in}{1.300855in}}%
\pgfpathlineto{\pgfqpoint{4.310296in}{1.300596in}}%
\pgfpathlineto{\pgfqpoint{4.307124in}{1.300451in}}%
\pgfpathlineto{\pgfqpoint{4.303952in}{1.299866in}}%
\pgfpathlineto{\pgfqpoint{4.300779in}{1.299717in}}%
\pgfpathlineto{\pgfqpoint{4.297607in}{1.299466in}}%
\pgfpathlineto{\pgfqpoint{4.294435in}{1.299316in}}%
\pgfpathlineto{\pgfqpoint{4.291263in}{1.299520in}}%
\pgfpathlineto{\pgfqpoint{4.288091in}{1.299791in}}%
\pgfpathlineto{\pgfqpoint{4.284919in}{1.299820in}}%
\pgfpathlineto{\pgfqpoint{4.281747in}{1.299803in}}%
\pgfpathlineto{\pgfqpoint{4.278575in}{1.299929in}}%
\pgfpathlineto{\pgfqpoint{4.275403in}{1.299468in}}%
\pgfpathlineto{\pgfqpoint{4.272231in}{1.299373in}}%
\pgfpathlineto{\pgfqpoint{4.269059in}{1.299120in}}%
\pgfpathlineto{\pgfqpoint{4.265887in}{1.298875in}}%
\pgfpathlineto{\pgfqpoint{4.262715in}{1.298422in}}%
\pgfpathlineto{\pgfqpoint{4.259543in}{1.298503in}}%
\pgfpathlineto{\pgfqpoint{4.256371in}{1.298272in}}%
\pgfpathlineto{\pgfqpoint{4.253199in}{1.297861in}}%
\pgfpathlineto{\pgfqpoint{4.250027in}{1.298002in}}%
\pgfpathlineto{\pgfqpoint{4.246855in}{1.298092in}}%
\pgfpathlineto{\pgfqpoint{4.243683in}{1.298310in}}%
\pgfpathlineto{\pgfqpoint{4.240511in}{1.298255in}}%
\pgfpathlineto{\pgfqpoint{4.237339in}{1.298198in}}%
\pgfpathlineto{\pgfqpoint{4.234167in}{1.297784in}}%
\pgfpathlineto{\pgfqpoint{4.230995in}{1.297630in}}%
\pgfpathlineto{\pgfqpoint{4.227823in}{1.296909in}}%
\pgfpathlineto{\pgfqpoint{4.224650in}{1.296851in}}%
\pgfpathlineto{\pgfqpoint{4.221478in}{1.296950in}}%
\pgfpathlineto{\pgfqpoint{4.218306in}{1.297195in}}%
\pgfpathlineto{\pgfqpoint{4.215134in}{1.297063in}}%
\pgfpathlineto{\pgfqpoint{4.211962in}{1.297025in}}%
\pgfpathlineto{\pgfqpoint{4.208790in}{1.297365in}}%
\pgfpathlineto{\pgfqpoint{4.205618in}{1.297324in}}%
\pgfpathlineto{\pgfqpoint{4.202446in}{1.297573in}}%
\pgfpathlineto{\pgfqpoint{4.199274in}{1.297156in}}%
\pgfpathlineto{\pgfqpoint{4.196102in}{1.296883in}}%
\pgfpathlineto{\pgfqpoint{4.192930in}{1.296622in}}%
\pgfpathlineto{\pgfqpoint{4.189758in}{1.295969in}}%
\pgfpathlineto{\pgfqpoint{4.186586in}{1.295973in}}%
\pgfpathlineto{\pgfqpoint{4.183414in}{1.295846in}}%
\pgfpathlineto{\pgfqpoint{4.180242in}{1.295689in}}%
\pgfpathlineto{\pgfqpoint{4.177070in}{1.295551in}}%
\pgfpathlineto{\pgfqpoint{4.173898in}{1.294678in}}%
\pgfpathlineto{\pgfqpoint{4.170726in}{1.294782in}}%
\pgfpathlineto{\pgfqpoint{4.167554in}{1.294618in}}%
\pgfpathlineto{\pgfqpoint{4.164382in}{1.294362in}}%
\pgfpathlineto{\pgfqpoint{4.161210in}{1.294309in}}%
\pgfpathlineto{\pgfqpoint{4.158038in}{1.294551in}}%
\pgfpathlineto{\pgfqpoint{4.154866in}{1.294370in}}%
\pgfpathlineto{\pgfqpoint{4.151694in}{1.294143in}}%
\pgfpathlineto{\pgfqpoint{4.148522in}{1.293923in}}%
\pgfpathlineto{\pgfqpoint{4.145349in}{1.293720in}}%
\pgfpathlineto{\pgfqpoint{4.142177in}{1.293596in}}%
\pgfpathlineto{\pgfqpoint{4.139005in}{1.293179in}}%
\pgfpathlineto{\pgfqpoint{4.135833in}{1.293143in}}%
\pgfpathlineto{\pgfqpoint{4.132661in}{1.293380in}}%
\pgfpathlineto{\pgfqpoint{4.129489in}{1.293415in}}%
\pgfpathlineto{\pgfqpoint{4.126317in}{1.293630in}}%
\pgfpathlineto{\pgfqpoint{4.123145in}{1.293734in}}%
\pgfpathlineto{\pgfqpoint{4.119973in}{1.293535in}}%
\pgfpathlineto{\pgfqpoint{4.116801in}{1.293568in}}%
\pgfpathlineto{\pgfqpoint{4.113629in}{1.293450in}}%
\pgfpathlineto{\pgfqpoint{4.110457in}{1.293434in}}%
\pgfpathlineto{\pgfqpoint{4.107285in}{1.293191in}}%
\pgfpathlineto{\pgfqpoint{4.104113in}{1.293041in}}%
\pgfpathlineto{\pgfqpoint{4.100941in}{1.293123in}}%
\pgfpathlineto{\pgfqpoint{4.097769in}{1.292963in}}%
\pgfpathlineto{\pgfqpoint{4.094597in}{1.293160in}}%
\pgfpathlineto{\pgfqpoint{4.091425in}{1.293433in}}%
\pgfpathlineto{\pgfqpoint{4.088253in}{1.293558in}}%
\pgfpathlineto{\pgfqpoint{4.085081in}{1.293760in}}%
\pgfpathlineto{\pgfqpoint{4.081909in}{1.293561in}}%
\pgfpathlineto{\pgfqpoint{4.078737in}{1.293360in}}%
\pgfpathlineto{\pgfqpoint{4.075565in}{1.293399in}}%
\pgfpathlineto{\pgfqpoint{4.072393in}{1.293409in}}%
\pgfpathlineto{\pgfqpoint{4.069221in}{1.293277in}}%
\pgfpathlineto{\pgfqpoint{4.066048in}{1.292692in}}%
\pgfpathlineto{\pgfqpoint{4.062876in}{1.292289in}}%
\pgfpathlineto{\pgfqpoint{4.059704in}{1.292282in}}%
\pgfpathlineto{\pgfqpoint{4.056532in}{1.292010in}}%
\pgfpathlineto{\pgfqpoint{4.053360in}{1.291918in}}%
\pgfpathlineto{\pgfqpoint{4.050188in}{1.291949in}}%
\pgfpathlineto{\pgfqpoint{4.047016in}{1.292017in}}%
\pgfpathlineto{\pgfqpoint{4.043844in}{1.291845in}}%
\pgfpathlineto{\pgfqpoint{4.040672in}{1.291606in}}%
\pgfpathlineto{\pgfqpoint{4.037500in}{1.291000in}}%
\pgfpathlineto{\pgfqpoint{4.034328in}{1.290725in}}%
\pgfpathlineto{\pgfqpoint{4.031156in}{1.290533in}}%
\pgfpathlineto{\pgfqpoint{4.027984in}{1.290015in}}%
\pgfpathlineto{\pgfqpoint{4.024812in}{1.289456in}}%
\pgfpathlineto{\pgfqpoint{4.021640in}{1.289285in}}%
\pgfpathlineto{\pgfqpoint{4.018468in}{1.289067in}}%
\pgfpathlineto{\pgfqpoint{4.015296in}{1.288996in}}%
\pgfpathlineto{\pgfqpoint{4.012124in}{1.288606in}}%
\pgfpathlineto{\pgfqpoint{4.008952in}{1.288483in}}%
\pgfpathlineto{\pgfqpoint{4.005780in}{1.288449in}}%
\pgfpathlineto{\pgfqpoint{4.002608in}{1.288614in}}%
\pgfpathlineto{\pgfqpoint{3.999436in}{1.288722in}}%
\pgfpathlineto{\pgfqpoint{3.996264in}{1.288505in}}%
\pgfpathlineto{\pgfqpoint{3.993092in}{1.288298in}}%
\pgfpathlineto{\pgfqpoint{3.989919in}{1.287534in}}%
\pgfpathlineto{\pgfqpoint{3.986747in}{1.287287in}}%
\pgfpathlineto{\pgfqpoint{3.983575in}{1.287085in}}%
\pgfpathlineto{\pgfqpoint{3.980403in}{1.286618in}}%
\pgfpathlineto{\pgfqpoint{3.977231in}{1.287020in}}%
\pgfpathlineto{\pgfqpoint{3.974059in}{1.286941in}}%
\pgfpathlineto{\pgfqpoint{3.970887in}{1.287186in}}%
\pgfpathlineto{\pgfqpoint{3.967715in}{1.287382in}}%
\pgfpathlineto{\pgfqpoint{3.964543in}{1.287528in}}%
\pgfpathlineto{\pgfqpoint{3.961371in}{1.287675in}}%
\pgfpathlineto{\pgfqpoint{3.958199in}{1.287501in}}%
\pgfpathlineto{\pgfqpoint{3.955027in}{1.287522in}}%
\pgfpathlineto{\pgfqpoint{3.951855in}{1.287377in}}%
\pgfpathlineto{\pgfqpoint{3.948683in}{1.287142in}}%
\pgfpathlineto{\pgfqpoint{3.945511in}{1.286569in}}%
\pgfpathlineto{\pgfqpoint{3.942339in}{1.286229in}}%
\pgfpathlineto{\pgfqpoint{3.939167in}{1.286342in}}%
\pgfpathlineto{\pgfqpoint{3.935995in}{1.286201in}}%
\pgfpathlineto{\pgfqpoint{3.932823in}{1.286042in}}%
\pgfpathlineto{\pgfqpoint{3.929651in}{1.286251in}}%
\pgfpathlineto{\pgfqpoint{3.926479in}{1.286126in}}%
\pgfpathlineto{\pgfqpoint{3.923307in}{1.286099in}}%
\pgfpathlineto{\pgfqpoint{3.920135in}{1.285573in}}%
\pgfpathlineto{\pgfqpoint{3.916963in}{1.285696in}}%
\pgfpathlineto{\pgfqpoint{3.913791in}{1.285259in}}%
\pgfpathlineto{\pgfqpoint{3.910618in}{1.285548in}}%
\pgfpathlineto{\pgfqpoint{3.907446in}{1.285507in}}%
\pgfpathlineto{\pgfqpoint{3.904274in}{1.285302in}}%
\pgfpathlineto{\pgfqpoint{3.901102in}{1.284655in}}%
\pgfpathlineto{\pgfqpoint{3.897930in}{1.284465in}}%
\pgfpathlineto{\pgfqpoint{3.894758in}{1.284152in}}%
\pgfpathlineto{\pgfqpoint{3.891586in}{1.283772in}}%
\pgfpathlineto{\pgfqpoint{3.888414in}{1.283894in}}%
\pgfpathlineto{\pgfqpoint{3.885242in}{1.283749in}}%
\pgfpathlineto{\pgfqpoint{3.882070in}{1.283405in}}%
\pgfpathlineto{\pgfqpoint{3.878898in}{1.283603in}}%
\pgfpathlineto{\pgfqpoint{3.875726in}{1.283478in}}%
\pgfpathlineto{\pgfqpoint{3.872554in}{1.283044in}}%
\pgfpathlineto{\pgfqpoint{3.869382in}{1.282673in}}%
\pgfpathlineto{\pgfqpoint{3.866210in}{1.282192in}}%
\pgfpathlineto{\pgfqpoint{3.863038in}{1.281638in}}%
\pgfpathlineto{\pgfqpoint{3.859866in}{1.281590in}}%
\pgfpathlineto{\pgfqpoint{3.856694in}{1.281865in}}%
\pgfpathlineto{\pgfqpoint{3.853522in}{1.281825in}}%
\pgfpathlineto{\pgfqpoint{3.850350in}{1.281822in}}%
\pgfpathlineto{\pgfqpoint{3.847178in}{1.281950in}}%
\pgfpathlineto{\pgfqpoint{3.844006in}{1.281541in}}%
\pgfpathlineto{\pgfqpoint{3.840834in}{1.281491in}}%
\pgfpathlineto{\pgfqpoint{3.837662in}{1.280907in}}%
\pgfpathlineto{\pgfqpoint{3.834490in}{1.280808in}}%
\pgfpathlineto{\pgfqpoint{3.831317in}{1.281073in}}%
\pgfpathlineto{\pgfqpoint{3.828145in}{1.280866in}}%
\pgfpathlineto{\pgfqpoint{3.824973in}{1.281023in}}%
\pgfpathlineto{\pgfqpoint{3.821801in}{1.281355in}}%
\pgfpathlineto{\pgfqpoint{3.818629in}{1.281029in}}%
\pgfpathlineto{\pgfqpoint{3.815457in}{1.281235in}}%
\pgfpathlineto{\pgfqpoint{3.812285in}{1.281466in}}%
\pgfpathlineto{\pgfqpoint{3.809113in}{1.281632in}}%
\pgfpathlineto{\pgfqpoint{3.805941in}{1.281377in}}%
\pgfpathlineto{\pgfqpoint{3.802769in}{1.280719in}}%
\pgfpathlineto{\pgfqpoint{3.799597in}{1.280488in}}%
\pgfpathlineto{\pgfqpoint{3.796425in}{1.280256in}}%
\pgfpathlineto{\pgfqpoint{3.793253in}{1.280128in}}%
\pgfpathlineto{\pgfqpoint{3.790081in}{1.279985in}}%
\pgfpathlineto{\pgfqpoint{3.786909in}{1.279877in}}%
\pgfpathlineto{\pgfqpoint{3.783737in}{1.279446in}}%
\pgfpathlineto{\pgfqpoint{3.780565in}{1.279651in}}%
\pgfpathlineto{\pgfqpoint{3.777393in}{1.279584in}}%
\pgfpathlineto{\pgfqpoint{3.774221in}{1.279683in}}%
\pgfpathlineto{\pgfqpoint{3.771049in}{1.278991in}}%
\pgfpathlineto{\pgfqpoint{3.767877in}{1.279388in}}%
\pgfpathlineto{\pgfqpoint{3.764705in}{1.279211in}}%
\pgfpathlineto{\pgfqpoint{3.761533in}{1.279472in}}%
\pgfpathlineto{\pgfqpoint{3.758361in}{1.279381in}}%
\pgfpathlineto{\pgfqpoint{3.755188in}{1.279097in}}%
\pgfpathlineto{\pgfqpoint{3.752016in}{1.278895in}}%
\pgfpathlineto{\pgfqpoint{3.748844in}{1.279121in}}%
\pgfpathlineto{\pgfqpoint{3.745672in}{1.279444in}}%
\pgfpathlineto{\pgfqpoint{3.742500in}{1.279700in}}%
\pgfpathlineto{\pgfqpoint{3.739328in}{1.279151in}}%
\pgfpathlineto{\pgfqpoint{3.736156in}{1.279150in}}%
\pgfpathlineto{\pgfqpoint{3.732984in}{1.278932in}}%
\pgfpathlineto{\pgfqpoint{3.729812in}{1.278282in}}%
\pgfpathlineto{\pgfqpoint{3.726640in}{1.278209in}}%
\pgfpathlineto{\pgfqpoint{3.723468in}{1.277882in}}%
\pgfpathlineto{\pgfqpoint{3.720296in}{1.277973in}}%
\pgfpathlineto{\pgfqpoint{3.717124in}{1.278158in}}%
\pgfpathlineto{\pgfqpoint{3.713952in}{1.278054in}}%
\pgfpathlineto{\pgfqpoint{3.710780in}{1.278496in}}%
\pgfpathlineto{\pgfqpoint{3.707608in}{1.278465in}}%
\pgfpathlineto{\pgfqpoint{3.704436in}{1.278138in}}%
\pgfpathlineto{\pgfqpoint{3.701264in}{1.277892in}}%
\pgfpathlineto{\pgfqpoint{3.698092in}{1.277311in}}%
\pgfpathlineto{\pgfqpoint{3.694920in}{1.277348in}}%
\pgfpathlineto{\pgfqpoint{3.691748in}{1.277274in}}%
\pgfpathlineto{\pgfqpoint{3.688576in}{1.277664in}}%
\pgfpathlineto{\pgfqpoint{3.685404in}{1.277449in}}%
\pgfpathlineto{\pgfqpoint{3.682232in}{1.277182in}}%
\pgfpathlineto{\pgfqpoint{3.679060in}{1.277379in}}%
\pgfpathlineto{\pgfqpoint{3.675887in}{1.277060in}}%
\pgfpathlineto{\pgfqpoint{3.672715in}{1.277078in}}%
\pgfpathlineto{\pgfqpoint{3.669543in}{1.277175in}}%
\pgfpathlineto{\pgfqpoint{3.666371in}{1.277322in}}%
\pgfpathlineto{\pgfqpoint{3.663199in}{1.277716in}}%
\pgfpathlineto{\pgfqpoint{3.660027in}{1.277700in}}%
\pgfpathlineto{\pgfqpoint{3.656855in}{1.277427in}}%
\pgfpathlineto{\pgfqpoint{3.653683in}{1.276982in}}%
\pgfpathlineto{\pgfqpoint{3.650511in}{1.276437in}}%
\pgfpathlineto{\pgfqpoint{3.647339in}{1.275871in}}%
\pgfpathlineto{\pgfqpoint{3.644167in}{1.275516in}}%
\pgfpathlineto{\pgfqpoint{3.640995in}{1.275346in}}%
\pgfpathlineto{\pgfqpoint{3.637823in}{1.275033in}}%
\pgfpathlineto{\pgfqpoint{3.634651in}{1.274944in}}%
\pgfpathlineto{\pgfqpoint{3.631479in}{1.275033in}}%
\pgfpathlineto{\pgfqpoint{3.628307in}{1.274895in}}%
\pgfpathlineto{\pgfqpoint{3.625135in}{1.274660in}}%
\pgfpathlineto{\pgfqpoint{3.621963in}{1.274497in}}%
\pgfpathlineto{\pgfqpoint{3.618791in}{1.274339in}}%
\pgfpathlineto{\pgfqpoint{3.615619in}{1.274424in}}%
\pgfpathlineto{\pgfqpoint{3.612447in}{1.274331in}}%
\pgfpathlineto{\pgfqpoint{3.609275in}{1.274116in}}%
\pgfpathlineto{\pgfqpoint{3.606103in}{1.274030in}}%
\pgfpathlineto{\pgfqpoint{3.602931in}{1.273663in}}%
\pgfpathlineto{\pgfqpoint{3.599759in}{1.273649in}}%
\pgfpathlineto{\pgfqpoint{3.596586in}{1.273149in}}%
\pgfpathlineto{\pgfqpoint{3.593414in}{1.273000in}}%
\pgfpathlineto{\pgfqpoint{3.590242in}{1.272871in}}%
\pgfpathlineto{\pgfqpoint{3.587070in}{1.272832in}}%
\pgfpathlineto{\pgfqpoint{3.583898in}{1.272843in}}%
\pgfpathlineto{\pgfqpoint{3.580726in}{1.272488in}}%
\pgfpathlineto{\pgfqpoint{3.577554in}{1.272603in}}%
\pgfpathlineto{\pgfqpoint{3.574382in}{1.272269in}}%
\pgfpathlineto{\pgfqpoint{3.571210in}{1.272083in}}%
\pgfpathlineto{\pgfqpoint{3.568038in}{1.271920in}}%
\pgfpathlineto{\pgfqpoint{3.564866in}{1.272048in}}%
\pgfpathlineto{\pgfqpoint{3.561694in}{1.272176in}}%
\pgfpathlineto{\pgfqpoint{3.558522in}{1.272135in}}%
\pgfpathlineto{\pgfqpoint{3.555350in}{1.271988in}}%
\pgfpathlineto{\pgfqpoint{3.552178in}{1.272080in}}%
\pgfpathlineto{\pgfqpoint{3.549006in}{1.272210in}}%
\pgfpathlineto{\pgfqpoint{3.545834in}{1.272454in}}%
\pgfpathlineto{\pgfqpoint{3.542662in}{1.272720in}}%
\pgfpathlineto{\pgfqpoint{3.539490in}{1.272979in}}%
\pgfpathlineto{\pgfqpoint{3.536318in}{1.272959in}}%
\pgfpathlineto{\pgfqpoint{3.533146in}{1.272902in}}%
\pgfpathlineto{\pgfqpoint{3.529974in}{1.272674in}}%
\pgfpathlineto{\pgfqpoint{3.526802in}{1.272787in}}%
\pgfpathlineto{\pgfqpoint{3.523630in}{1.272897in}}%
\pgfpathlineto{\pgfqpoint{3.520457in}{1.272908in}}%
\pgfpathlineto{\pgfqpoint{3.517285in}{1.273051in}}%
\pgfpathlineto{\pgfqpoint{3.514113in}{1.272868in}}%
\pgfpathlineto{\pgfqpoint{3.510941in}{1.273073in}}%
\pgfpathlineto{\pgfqpoint{3.507769in}{1.273280in}}%
\pgfpathlineto{\pgfqpoint{3.504597in}{1.273296in}}%
\pgfpathlineto{\pgfqpoint{3.501425in}{1.273366in}}%
\pgfpathlineto{\pgfqpoint{3.498253in}{1.273288in}}%
\pgfpathlineto{\pgfqpoint{3.495081in}{1.273310in}}%
\pgfpathlineto{\pgfqpoint{3.491909in}{1.273393in}}%
\pgfpathlineto{\pgfqpoint{3.488737in}{1.272620in}}%
\pgfpathlineto{\pgfqpoint{3.485565in}{1.272400in}}%
\pgfpathlineto{\pgfqpoint{3.482393in}{1.272110in}}%
\pgfpathlineto{\pgfqpoint{3.479221in}{1.272181in}}%
\pgfpathlineto{\pgfqpoint{3.476049in}{1.271773in}}%
\pgfpathlineto{\pgfqpoint{3.472877in}{1.271675in}}%
\pgfpathlineto{\pgfqpoint{3.469705in}{1.271382in}}%
\pgfpathlineto{\pgfqpoint{3.466533in}{1.271508in}}%
\pgfpathlineto{\pgfqpoint{3.463361in}{1.271517in}}%
\pgfpathlineto{\pgfqpoint{3.460189in}{1.271629in}}%
\pgfpathlineto{\pgfqpoint{3.457017in}{1.271693in}}%
\pgfpathlineto{\pgfqpoint{3.453845in}{1.271384in}}%
\pgfpathlineto{\pgfqpoint{3.450673in}{1.271352in}}%
\pgfpathlineto{\pgfqpoint{3.447501in}{1.270935in}}%
\pgfpathlineto{\pgfqpoint{3.444329in}{1.270652in}}%
\pgfpathlineto{\pgfqpoint{3.441156in}{1.270374in}}%
\pgfpathlineto{\pgfqpoint{3.437984in}{1.270668in}}%
\pgfpathlineto{\pgfqpoint{3.434812in}{1.270488in}}%
\pgfpathlineto{\pgfqpoint{3.431640in}{1.270478in}}%
\pgfpathlineto{\pgfqpoint{3.428468in}{1.270546in}}%
\pgfpathlineto{\pgfqpoint{3.425296in}{1.270263in}}%
\pgfpathlineto{\pgfqpoint{3.422124in}{1.270248in}}%
\pgfpathlineto{\pgfqpoint{3.418952in}{1.270155in}}%
\pgfpathlineto{\pgfqpoint{3.415780in}{1.269837in}}%
\pgfpathlineto{\pgfqpoint{3.412608in}{1.269438in}}%
\pgfpathlineto{\pgfqpoint{3.409436in}{1.268977in}}%
\pgfpathlineto{\pgfqpoint{3.406264in}{1.268827in}}%
\pgfpathlineto{\pgfqpoint{3.403092in}{1.268818in}}%
\pgfpathlineto{\pgfqpoint{3.399920in}{1.268231in}}%
\pgfpathlineto{\pgfqpoint{3.396748in}{1.267496in}}%
\pgfpathlineto{\pgfqpoint{3.393576in}{1.267623in}}%
\pgfpathlineto{\pgfqpoint{3.390404in}{1.267420in}}%
\pgfpathlineto{\pgfqpoint{3.387232in}{1.267333in}}%
\pgfpathlineto{\pgfqpoint{3.384060in}{1.267776in}}%
\pgfpathlineto{\pgfqpoint{3.380888in}{1.267970in}}%
\pgfpathlineto{\pgfqpoint{3.377716in}{1.268160in}}%
\pgfpathlineto{\pgfqpoint{3.374544in}{1.267922in}}%
\pgfpathlineto{\pgfqpoint{3.371372in}{1.267147in}}%
\pgfpathlineto{\pgfqpoint{3.368200in}{1.267298in}}%
\pgfpathlineto{\pgfqpoint{3.365028in}{1.266897in}}%
\pgfpathlineto{\pgfqpoint{3.361855in}{1.266627in}}%
\pgfpathlineto{\pgfqpoint{3.358683in}{1.266625in}}%
\pgfpathlineto{\pgfqpoint{3.355511in}{1.266157in}}%
\pgfpathlineto{\pgfqpoint{3.352339in}{1.266396in}}%
\pgfpathlineto{\pgfqpoint{3.349167in}{1.265876in}}%
\pgfpathlineto{\pgfqpoint{3.345995in}{1.266078in}}%
\pgfpathlineto{\pgfqpoint{3.342823in}{1.265931in}}%
\pgfpathlineto{\pgfqpoint{3.339651in}{1.265790in}}%
\pgfpathlineto{\pgfqpoint{3.336479in}{1.265528in}}%
\pgfpathlineto{\pgfqpoint{3.333307in}{1.265188in}}%
\pgfpathlineto{\pgfqpoint{3.330135in}{1.265101in}}%
\pgfpathlineto{\pgfqpoint{3.326963in}{1.265334in}}%
\pgfpathlineto{\pgfqpoint{3.323791in}{1.264475in}}%
\pgfpathlineto{\pgfqpoint{3.320619in}{1.264634in}}%
\pgfpathlineto{\pgfqpoint{3.317447in}{1.264402in}}%
\pgfpathlineto{\pgfqpoint{3.314275in}{1.264402in}}%
\pgfpathlineto{\pgfqpoint{3.311103in}{1.264228in}}%
\pgfpathlineto{\pgfqpoint{3.307931in}{1.264042in}}%
\pgfpathlineto{\pgfqpoint{3.304759in}{1.264064in}}%
\pgfpathlineto{\pgfqpoint{3.301587in}{1.264274in}}%
\pgfpathlineto{\pgfqpoint{3.298415in}{1.264416in}}%
\pgfpathlineto{\pgfqpoint{3.295243in}{1.264538in}}%
\pgfpathlineto{\pgfqpoint{3.292071in}{1.264792in}}%
\pgfpathlineto{\pgfqpoint{3.288899in}{1.264446in}}%
\pgfpathlineto{\pgfqpoint{3.285726in}{1.264345in}}%
\pgfpathlineto{\pgfqpoint{3.282554in}{1.264412in}}%
\pgfpathlineto{\pgfqpoint{3.279382in}{1.263917in}}%
\pgfpathlineto{\pgfqpoint{3.276210in}{1.264049in}}%
\pgfpathlineto{\pgfqpoint{3.273038in}{1.263710in}}%
\pgfpathlineto{\pgfqpoint{3.269866in}{1.263650in}}%
\pgfpathlineto{\pgfqpoint{3.266694in}{1.263358in}}%
\pgfpathlineto{\pgfqpoint{3.263522in}{1.263221in}}%
\pgfpathlineto{\pgfqpoint{3.260350in}{1.262650in}}%
\pgfpathlineto{\pgfqpoint{3.257178in}{1.262502in}}%
\pgfpathlineto{\pgfqpoint{3.254006in}{1.262140in}}%
\pgfpathlineto{\pgfqpoint{3.250834in}{1.262501in}}%
\pgfpathlineto{\pgfqpoint{3.247662in}{1.262422in}}%
\pgfpathlineto{\pgfqpoint{3.244490in}{1.262392in}}%
\pgfpathlineto{\pgfqpoint{3.241318in}{1.263892in}}%
\pgfpathlineto{\pgfqpoint{3.238146in}{1.263484in}}%
\pgfpathlineto{\pgfqpoint{3.234974in}{1.263003in}}%
\pgfpathlineto{\pgfqpoint{3.231802in}{1.263667in}}%
\pgfpathlineto{\pgfqpoint{3.228630in}{1.262438in}}%
\pgfpathlineto{\pgfqpoint{3.225458in}{1.262067in}}%
\pgfpathlineto{\pgfqpoint{3.222286in}{1.261687in}}%
\pgfpathlineto{\pgfqpoint{3.219114in}{1.261408in}}%
\pgfpathlineto{\pgfqpoint{3.215942in}{1.261149in}}%
\pgfpathlineto{\pgfqpoint{3.212770in}{1.260944in}}%
\pgfpathlineto{\pgfqpoint{3.209598in}{1.260005in}}%
\pgfpathlineto{\pgfqpoint{3.206425in}{1.260166in}}%
\pgfpathlineto{\pgfqpoint{3.203253in}{1.260376in}}%
\pgfpathlineto{\pgfqpoint{3.200081in}{1.259157in}}%
\pgfpathlineto{\pgfqpoint{3.196909in}{1.259120in}}%
\pgfpathlineto{\pgfqpoint{3.193737in}{1.258935in}}%
\pgfpathlineto{\pgfqpoint{3.190565in}{1.258587in}}%
\pgfpathlineto{\pgfqpoint{3.187393in}{1.258634in}}%
\pgfpathlineto{\pgfqpoint{3.184221in}{1.257309in}}%
\pgfpathlineto{\pgfqpoint{3.181049in}{1.257487in}}%
\pgfpathlineto{\pgfqpoint{3.177877in}{1.256590in}}%
\pgfpathlineto{\pgfqpoint{3.174705in}{1.256163in}}%
\pgfpathlineto{\pgfqpoint{3.171533in}{1.256497in}}%
\pgfpathlineto{\pgfqpoint{3.168361in}{1.256374in}}%
\pgfpathlineto{\pgfqpoint{3.165189in}{1.256207in}}%
\pgfpathlineto{\pgfqpoint{3.162017in}{1.256190in}}%
\pgfpathlineto{\pgfqpoint{3.158845in}{1.256296in}}%
\pgfpathlineto{\pgfqpoint{3.155673in}{1.256179in}}%
\pgfpathlineto{\pgfqpoint{3.152501in}{1.256221in}}%
\pgfpathlineto{\pgfqpoint{3.149329in}{1.254680in}}%
\pgfpathlineto{\pgfqpoint{3.146157in}{1.254657in}}%
\pgfpathlineto{\pgfqpoint{3.142985in}{1.254368in}}%
\pgfpathlineto{\pgfqpoint{3.139813in}{1.254718in}}%
\pgfpathlineto{\pgfqpoint{3.136641in}{1.254936in}}%
\pgfpathlineto{\pgfqpoint{3.133469in}{1.255180in}}%
\pgfpathlineto{\pgfqpoint{3.130297in}{1.255162in}}%
\pgfpathlineto{\pgfqpoint{3.127124in}{1.255210in}}%
\pgfpathlineto{\pgfqpoint{3.123952in}{1.255095in}}%
\pgfpathlineto{\pgfqpoint{3.120780in}{1.255033in}}%
\pgfpathlineto{\pgfqpoint{3.117608in}{1.255052in}}%
\pgfpathlineto{\pgfqpoint{3.114436in}{1.255114in}}%
\pgfpathlineto{\pgfqpoint{3.111264in}{1.255012in}}%
\pgfpathlineto{\pgfqpoint{3.108092in}{1.255144in}}%
\pgfpathlineto{\pgfqpoint{3.104920in}{1.255416in}}%
\pgfpathlineto{\pgfqpoint{3.101748in}{1.255408in}}%
\pgfpathlineto{\pgfqpoint{3.098576in}{1.255226in}}%
\pgfpathlineto{\pgfqpoint{3.095404in}{1.255171in}}%
\pgfpathlineto{\pgfqpoint{3.092232in}{1.255335in}}%
\pgfpathlineto{\pgfqpoint{3.089060in}{1.255198in}}%
\pgfpathlineto{\pgfqpoint{3.085888in}{1.254998in}}%
\pgfpathlineto{\pgfqpoint{3.082716in}{1.254846in}}%
\pgfpathlineto{\pgfqpoint{3.079544in}{1.255031in}}%
\pgfpathlineto{\pgfqpoint{3.076372in}{1.254842in}}%
\pgfpathlineto{\pgfqpoint{3.073200in}{1.254641in}}%
\pgfpathlineto{\pgfqpoint{3.070028in}{1.254741in}}%
\pgfpathlineto{\pgfqpoint{3.066856in}{1.254840in}}%
\pgfpathlineto{\pgfqpoint{3.063684in}{1.254738in}}%
\pgfpathlineto{\pgfqpoint{3.060512in}{1.254751in}}%
\pgfpathlineto{\pgfqpoint{3.057340in}{1.254838in}}%
\pgfpathlineto{\pgfqpoint{3.054168in}{1.254919in}}%
\pgfpathlineto{\pgfqpoint{3.050995in}{1.254847in}}%
\pgfpathlineto{\pgfqpoint{3.047823in}{1.254707in}}%
\pgfpathlineto{\pgfqpoint{3.044651in}{1.254620in}}%
\pgfpathlineto{\pgfqpoint{3.041479in}{1.254578in}}%
\pgfpathlineto{\pgfqpoint{3.038307in}{1.254446in}}%
\pgfpathlineto{\pgfqpoint{3.035135in}{1.254394in}}%
\pgfpathlineto{\pgfqpoint{3.031963in}{1.254452in}}%
\pgfpathlineto{\pgfqpoint{3.028791in}{1.254401in}}%
\pgfpathlineto{\pgfqpoint{3.025619in}{1.254218in}}%
\pgfpathlineto{\pgfqpoint{3.022447in}{1.254302in}}%
\pgfpathlineto{\pgfqpoint{3.019275in}{1.254004in}}%
\pgfpathlineto{\pgfqpoint{3.016103in}{1.254067in}}%
\pgfpathlineto{\pgfqpoint{3.012931in}{1.254111in}}%
\pgfpathlineto{\pgfqpoint{3.009759in}{1.254005in}}%
\pgfpathlineto{\pgfqpoint{3.006587in}{1.253895in}}%
\pgfpathlineto{\pgfqpoint{3.003415in}{1.253956in}}%
\pgfpathlineto{\pgfqpoint{3.000243in}{1.253955in}}%
\pgfpathlineto{\pgfqpoint{2.997071in}{1.253923in}}%
\pgfpathlineto{\pgfqpoint{2.993899in}{1.253786in}}%
\pgfpathlineto{\pgfqpoint{2.990727in}{1.253669in}}%
\pgfpathlineto{\pgfqpoint{2.987555in}{1.253608in}}%
\pgfpathlineto{\pgfqpoint{2.984383in}{1.253516in}}%
\pgfpathlineto{\pgfqpoint{2.981211in}{1.253503in}}%
\pgfpathlineto{\pgfqpoint{2.978039in}{1.253477in}}%
\pgfpathlineto{\pgfqpoint{2.974867in}{1.253587in}}%
\pgfpathlineto{\pgfqpoint{2.971694in}{1.253625in}}%
\pgfpathlineto{\pgfqpoint{2.968522in}{1.253800in}}%
\pgfpathlineto{\pgfqpoint{2.965350in}{1.253854in}}%
\pgfpathlineto{\pgfqpoint{2.962178in}{1.253737in}}%
\pgfpathlineto{\pgfqpoint{2.959006in}{1.253754in}}%
\pgfpathlineto{\pgfqpoint{2.955834in}{1.253207in}}%
\pgfpathlineto{\pgfqpoint{2.952662in}{1.253093in}}%
\pgfpathlineto{\pgfqpoint{2.949490in}{1.253012in}}%
\pgfpathlineto{\pgfqpoint{2.946318in}{1.253161in}}%
\pgfpathlineto{\pgfqpoint{2.943146in}{1.253115in}}%
\pgfpathlineto{\pgfqpoint{2.939974in}{1.252903in}}%
\pgfpathlineto{\pgfqpoint{2.936802in}{1.252778in}}%
\pgfpathlineto{\pgfqpoint{2.933630in}{1.253016in}}%
\pgfpathlineto{\pgfqpoint{2.930458in}{1.253177in}}%
\pgfpathlineto{\pgfqpoint{2.927286in}{1.253031in}}%
\pgfpathlineto{\pgfqpoint{2.924114in}{1.252793in}}%
\pgfpathlineto{\pgfqpoint{2.920942in}{1.252974in}}%
\pgfpathlineto{\pgfqpoint{2.917770in}{1.252881in}}%
\pgfpathlineto{\pgfqpoint{2.914598in}{1.252850in}}%
\pgfpathlineto{\pgfqpoint{2.911426in}{1.252739in}}%
\pgfpathlineto{\pgfqpoint{2.908254in}{1.252754in}}%
\pgfpathlineto{\pgfqpoint{2.905082in}{1.252690in}}%
\pgfpathlineto{\pgfqpoint{2.901910in}{1.252686in}}%
\pgfpathlineto{\pgfqpoint{2.898738in}{1.252662in}}%
\pgfpathlineto{\pgfqpoint{2.895565in}{1.252449in}}%
\pgfpathlineto{\pgfqpoint{2.892393in}{1.252630in}}%
\pgfpathlineto{\pgfqpoint{2.889221in}{1.252497in}}%
\pgfpathlineto{\pgfqpoint{2.886049in}{1.252538in}}%
\pgfpathlineto{\pgfqpoint{2.882877in}{1.252468in}}%
\pgfpathlineto{\pgfqpoint{2.879705in}{1.252487in}}%
\pgfpathlineto{\pgfqpoint{2.876533in}{1.252434in}}%
\pgfpathlineto{\pgfqpoint{2.873361in}{1.252473in}}%
\pgfpathlineto{\pgfqpoint{2.870189in}{1.252454in}}%
\pgfpathlineto{\pgfqpoint{2.867017in}{1.252397in}}%
\pgfpathlineto{\pgfqpoint{2.863845in}{1.252467in}}%
\pgfpathlineto{\pgfqpoint{2.860673in}{1.252283in}}%
\pgfpathlineto{\pgfqpoint{2.857501in}{1.252188in}}%
\pgfpathlineto{\pgfqpoint{2.854329in}{1.252227in}}%
\pgfpathlineto{\pgfqpoint{2.851157in}{1.252200in}}%
\pgfpathlineto{\pgfqpoint{2.847985in}{1.252157in}}%
\pgfpathlineto{\pgfqpoint{2.844813in}{1.252203in}}%
\pgfpathlineto{\pgfqpoint{2.841641in}{1.252198in}}%
\pgfpathlineto{\pgfqpoint{2.838469in}{1.252378in}}%
\pgfpathlineto{\pgfqpoint{2.835297in}{1.252378in}}%
\pgfpathlineto{\pgfqpoint{2.832125in}{1.252195in}}%
\pgfpathlineto{\pgfqpoint{2.828953in}{1.252048in}}%
\pgfpathlineto{\pgfqpoint{2.825781in}{1.252032in}}%
\pgfpathlineto{\pgfqpoint{2.822609in}{1.251932in}}%
\pgfpathlineto{\pgfqpoint{2.819437in}{1.251943in}}%
\pgfpathlineto{\pgfqpoint{2.816264in}{1.252145in}}%
\pgfpathlineto{\pgfqpoint{2.813092in}{1.252234in}}%
\pgfpathlineto{\pgfqpoint{2.809920in}{1.252009in}}%
\pgfpathlineto{\pgfqpoint{2.806748in}{1.251969in}}%
\pgfpathlineto{\pgfqpoint{2.803576in}{1.251989in}}%
\pgfpathlineto{\pgfqpoint{2.800404in}{1.251932in}}%
\pgfpathlineto{\pgfqpoint{2.797232in}{1.251868in}}%
\pgfpathlineto{\pgfqpoint{2.794060in}{1.251835in}}%
\pgfpathlineto{\pgfqpoint{2.790888in}{1.251768in}}%
\pgfpathlineto{\pgfqpoint{2.787716in}{1.251717in}}%
\pgfpathlineto{\pgfqpoint{2.784544in}{1.251549in}}%
\pgfpathlineto{\pgfqpoint{2.781372in}{1.251407in}}%
\pgfpathlineto{\pgfqpoint{2.778200in}{1.251324in}}%
\pgfpathlineto{\pgfqpoint{2.775028in}{1.251157in}}%
\pgfpathlineto{\pgfqpoint{2.771856in}{1.250993in}}%
\pgfpathlineto{\pgfqpoint{2.768684in}{1.250891in}}%
\pgfpathlineto{\pgfqpoint{2.765512in}{1.250855in}}%
\pgfpathlineto{\pgfqpoint{2.762340in}{1.251116in}}%
\pgfpathlineto{\pgfqpoint{2.759168in}{1.251287in}}%
\pgfpathlineto{\pgfqpoint{2.755996in}{1.251308in}}%
\pgfpathlineto{\pgfqpoint{2.752824in}{1.251197in}}%
\pgfpathlineto{\pgfqpoint{2.749652in}{1.251102in}}%
\pgfpathlineto{\pgfqpoint{2.746480in}{1.251205in}}%
\pgfpathlineto{\pgfqpoint{2.743308in}{1.251271in}}%
\pgfpathlineto{\pgfqpoint{2.740136in}{1.251338in}}%
\pgfpathlineto{\pgfqpoint{2.736963in}{1.251557in}}%
\pgfpathlineto{\pgfqpoint{2.733791in}{1.251669in}}%
\pgfpathlineto{\pgfqpoint{2.730619in}{1.251990in}}%
\pgfpathlineto{\pgfqpoint{2.727447in}{1.251751in}}%
\pgfpathlineto{\pgfqpoint{2.724275in}{1.251485in}}%
\pgfpathlineto{\pgfqpoint{2.721103in}{1.251440in}}%
\pgfpathlineto{\pgfqpoint{2.717931in}{1.251464in}}%
\pgfpathlineto{\pgfqpoint{2.714759in}{1.251350in}}%
\pgfpathlineto{\pgfqpoint{2.711587in}{1.251021in}}%
\pgfpathlineto{\pgfqpoint{2.708415in}{1.250957in}}%
\pgfpathlineto{\pgfqpoint{2.705243in}{1.250898in}}%
\pgfpathlineto{\pgfqpoint{2.702071in}{1.250590in}}%
\pgfpathlineto{\pgfqpoint{2.698899in}{1.250790in}}%
\pgfpathlineto{\pgfqpoint{2.695727in}{1.250707in}}%
\pgfpathlineto{\pgfqpoint{2.692555in}{1.250919in}}%
\pgfpathlineto{\pgfqpoint{2.689383in}{1.250671in}}%
\pgfpathlineto{\pgfqpoint{2.686211in}{1.250640in}}%
\pgfpathlineto{\pgfqpoint{2.683039in}{1.250648in}}%
\pgfpathlineto{\pgfqpoint{2.679867in}{1.250678in}}%
\pgfpathlineto{\pgfqpoint{2.676695in}{1.250621in}}%
\pgfpathlineto{\pgfqpoint{2.673523in}{1.250640in}}%
\pgfpathlineto{\pgfqpoint{2.670351in}{1.250545in}}%
\pgfpathlineto{\pgfqpoint{2.667179in}{1.250335in}}%
\pgfpathlineto{\pgfqpoint{2.664007in}{1.250149in}}%
\pgfpathlineto{\pgfqpoint{2.660834in}{1.250076in}}%
\pgfpathlineto{\pgfqpoint{2.657662in}{1.250132in}}%
\pgfpathlineto{\pgfqpoint{2.654490in}{1.250059in}}%
\pgfpathlineto{\pgfqpoint{2.651318in}{1.250156in}}%
\pgfpathlineto{\pgfqpoint{2.648146in}{1.250128in}}%
\pgfpathlineto{\pgfqpoint{2.644974in}{1.249941in}}%
\pgfpathlineto{\pgfqpoint{2.641802in}{1.249744in}}%
\pgfpathlineto{\pgfqpoint{2.638630in}{1.249664in}}%
\pgfpathlineto{\pgfqpoint{2.635458in}{1.249411in}}%
\pgfpathlineto{\pgfqpoint{2.632286in}{1.249181in}}%
\pgfpathlineto{\pgfqpoint{2.629114in}{1.249098in}}%
\pgfpathlineto{\pgfqpoint{2.625942in}{1.248948in}}%
\pgfpathlineto{\pgfqpoint{2.622770in}{1.248678in}}%
\pgfpathlineto{\pgfqpoint{2.619598in}{1.248837in}}%
\pgfpathlineto{\pgfqpoint{2.616426in}{1.249013in}}%
\pgfpathlineto{\pgfqpoint{2.613254in}{1.249040in}}%
\pgfpathlineto{\pgfqpoint{2.610082in}{1.248981in}}%
\pgfpathlineto{\pgfqpoint{2.606910in}{1.248966in}}%
\pgfpathlineto{\pgfqpoint{2.603738in}{1.248906in}}%
\pgfpathlineto{\pgfqpoint{2.600566in}{1.249023in}}%
\pgfpathlineto{\pgfqpoint{2.597394in}{1.249007in}}%
\pgfpathlineto{\pgfqpoint{2.594222in}{1.249063in}}%
\pgfpathlineto{\pgfqpoint{2.591050in}{1.249079in}}%
\pgfpathlineto{\pgfqpoint{2.587878in}{1.248912in}}%
\pgfpathlineto{\pgfqpoint{2.584706in}{1.248812in}}%
\pgfpathlineto{\pgfqpoint{2.581533in}{1.248931in}}%
\pgfpathlineto{\pgfqpoint{2.578361in}{1.248890in}}%
\pgfpathlineto{\pgfqpoint{2.575189in}{1.248571in}}%
\pgfpathlineto{\pgfqpoint{2.572017in}{1.248588in}}%
\pgfpathlineto{\pgfqpoint{2.568845in}{1.248489in}}%
\pgfpathlineto{\pgfqpoint{2.565673in}{1.248184in}}%
\pgfpathlineto{\pgfqpoint{2.562501in}{1.247957in}}%
\pgfpathlineto{\pgfqpoint{2.559329in}{1.247839in}}%
\pgfpathlineto{\pgfqpoint{2.556157in}{1.247873in}}%
\pgfpathlineto{\pgfqpoint{2.552985in}{1.247957in}}%
\pgfpathlineto{\pgfqpoint{2.549813in}{1.247840in}}%
\pgfpathlineto{\pgfqpoint{2.546641in}{1.247989in}}%
\pgfpathlineto{\pgfqpoint{2.543469in}{1.247905in}}%
\pgfpathlineto{\pgfqpoint{2.540297in}{1.247947in}}%
\pgfpathlineto{\pgfqpoint{2.537125in}{1.248006in}}%
\pgfpathlineto{\pgfqpoint{2.533953in}{1.248016in}}%
\pgfpathlineto{\pgfqpoint{2.530781in}{1.248084in}}%
\pgfpathlineto{\pgfqpoint{2.527609in}{1.247995in}}%
\pgfpathlineto{\pgfqpoint{2.524437in}{1.247813in}}%
\pgfpathlineto{\pgfqpoint{2.521265in}{1.247825in}}%
\pgfpathlineto{\pgfqpoint{2.518093in}{1.247886in}}%
\pgfpathlineto{\pgfqpoint{2.514921in}{1.247887in}}%
\pgfpathlineto{\pgfqpoint{2.511749in}{1.247671in}}%
\pgfpathlineto{\pgfqpoint{2.508577in}{1.247816in}}%
\pgfpathlineto{\pgfqpoint{2.505405in}{1.247796in}}%
\pgfpathlineto{\pgfqpoint{2.502232in}{1.247792in}}%
\pgfpathlineto{\pgfqpoint{2.499060in}{1.247589in}}%
\pgfpathlineto{\pgfqpoint{2.495888in}{1.247594in}}%
\pgfpathlineto{\pgfqpoint{2.492716in}{1.247728in}}%
\pgfpathlineto{\pgfqpoint{2.489544in}{1.247805in}}%
\pgfpathlineto{\pgfqpoint{2.486372in}{1.247550in}}%
\pgfpathlineto{\pgfqpoint{2.483200in}{1.247264in}}%
\pgfpathlineto{\pgfqpoint{2.480028in}{1.247210in}}%
\pgfpathlineto{\pgfqpoint{2.476856in}{1.247068in}}%
\pgfpathlineto{\pgfqpoint{2.473684in}{1.246926in}}%
\pgfpathlineto{\pgfqpoint{2.470512in}{1.246822in}}%
\pgfpathlineto{\pgfqpoint{2.467340in}{1.246719in}}%
\pgfpathlineto{\pgfqpoint{2.464168in}{1.246665in}}%
\pgfpathlineto{\pgfqpoint{2.460996in}{1.246615in}}%
\pgfpathlineto{\pgfqpoint{2.457824in}{1.246477in}}%
\pgfpathlineto{\pgfqpoint{2.454652in}{1.246489in}}%
\pgfpathlineto{\pgfqpoint{2.451480in}{1.246547in}}%
\pgfpathlineto{\pgfqpoint{2.448308in}{1.246462in}}%
\pgfpathlineto{\pgfqpoint{2.445136in}{1.246273in}}%
\pgfpathlineto{\pgfqpoint{2.441964in}{1.246306in}}%
\pgfpathlineto{\pgfqpoint{2.438792in}{1.246097in}}%
\pgfpathlineto{\pgfqpoint{2.435620in}{1.246184in}}%
\pgfpathlineto{\pgfqpoint{2.432448in}{1.246052in}}%
\pgfpathlineto{\pgfqpoint{2.429276in}{1.246024in}}%
\pgfpathlineto{\pgfqpoint{2.426103in}{1.246133in}}%
\pgfpathlineto{\pgfqpoint{2.422931in}{1.246162in}}%
\pgfpathlineto{\pgfqpoint{2.419759in}{1.246293in}}%
\pgfpathlineto{\pgfqpoint{2.416587in}{1.246157in}}%
\pgfpathlineto{\pgfqpoint{2.413415in}{1.246155in}}%
\pgfpathlineto{\pgfqpoint{2.410243in}{1.246249in}}%
\pgfpathlineto{\pgfqpoint{2.407071in}{1.246444in}}%
\pgfpathlineto{\pgfqpoint{2.403899in}{1.246402in}}%
\pgfpathlineto{\pgfqpoint{2.400727in}{1.246426in}}%
\pgfpathlineto{\pgfqpoint{2.397555in}{1.246291in}}%
\pgfpathlineto{\pgfqpoint{2.394383in}{1.246353in}}%
\pgfpathlineto{\pgfqpoint{2.391211in}{1.246493in}}%
\pgfpathlineto{\pgfqpoint{2.388039in}{1.246157in}}%
\pgfpathlineto{\pgfqpoint{2.384867in}{1.246142in}}%
\pgfpathlineto{\pgfqpoint{2.381695in}{1.246166in}}%
\pgfpathlineto{\pgfqpoint{2.378523in}{1.246225in}}%
\pgfpathlineto{\pgfqpoint{2.375351in}{1.245907in}}%
\pgfpathlineto{\pgfqpoint{2.372179in}{1.245622in}}%
\pgfpathlineto{\pgfqpoint{2.369007in}{1.245776in}}%
\pgfpathlineto{\pgfqpoint{2.365835in}{1.245892in}}%
\pgfpathlineto{\pgfqpoint{2.362663in}{1.245965in}}%
\pgfpathlineto{\pgfqpoint{2.359491in}{1.245781in}}%
\pgfpathlineto{\pgfqpoint{2.356319in}{1.245597in}}%
\pgfpathlineto{\pgfqpoint{2.353147in}{1.245674in}}%
\pgfpathlineto{\pgfqpoint{2.349975in}{1.245719in}}%
\pgfpathlineto{\pgfqpoint{2.346802in}{1.245689in}}%
\pgfpathlineto{\pgfqpoint{2.343630in}{1.245648in}}%
\pgfpathlineto{\pgfqpoint{2.340458in}{1.245941in}}%
\pgfpathlineto{\pgfqpoint{2.337286in}{1.245960in}}%
\pgfpathlineto{\pgfqpoint{2.334114in}{1.245871in}}%
\pgfpathlineto{\pgfqpoint{2.330942in}{1.245987in}}%
\pgfpathlineto{\pgfqpoint{2.327770in}{1.245896in}}%
\pgfpathlineto{\pgfqpoint{2.324598in}{1.246108in}}%
\pgfpathlineto{\pgfqpoint{2.321426in}{1.245780in}}%
\pgfpathlineto{\pgfqpoint{2.318254in}{1.245836in}}%
\pgfpathlineto{\pgfqpoint{2.315082in}{1.245936in}}%
\pgfpathlineto{\pgfqpoint{2.311910in}{1.245881in}}%
\pgfpathlineto{\pgfqpoint{2.308738in}{1.245871in}}%
\pgfpathlineto{\pgfqpoint{2.305566in}{1.245756in}}%
\pgfpathlineto{\pgfqpoint{2.302394in}{1.245994in}}%
\pgfpathlineto{\pgfqpoint{2.299222in}{1.245937in}}%
\pgfpathlineto{\pgfqpoint{2.296050in}{1.245929in}}%
\pgfpathlineto{\pgfqpoint{2.292878in}{1.245964in}}%
\pgfpathlineto{\pgfqpoint{2.289706in}{1.245923in}}%
\pgfpathlineto{\pgfqpoint{2.286534in}{1.245687in}}%
\pgfpathlineto{\pgfqpoint{2.283362in}{1.245683in}}%
\pgfpathlineto{\pgfqpoint{2.280190in}{1.245624in}}%
\pgfpathlineto{\pgfqpoint{2.277018in}{1.245767in}}%
\pgfpathlineto{\pgfqpoint{2.273846in}{1.245700in}}%
\pgfpathlineto{\pgfqpoint{2.270674in}{1.245799in}}%
\pgfpathlineto{\pgfqpoint{2.267501in}{1.245702in}}%
\pgfpathlineto{\pgfqpoint{2.264329in}{1.245813in}}%
\pgfpathlineto{\pgfqpoint{2.261157in}{1.245579in}}%
\pgfpathlineto{\pgfqpoint{2.257985in}{1.245612in}}%
\pgfpathlineto{\pgfqpoint{2.254813in}{1.245423in}}%
\pgfpathlineto{\pgfqpoint{2.251641in}{1.245516in}}%
\pgfpathlineto{\pgfqpoint{2.248469in}{1.245552in}}%
\pgfpathlineto{\pgfqpoint{2.245297in}{1.245673in}}%
\pgfpathlineto{\pgfqpoint{2.242125in}{1.245770in}}%
\pgfpathlineto{\pgfqpoint{2.238953in}{1.245670in}}%
\pgfpathlineto{\pgfqpoint{2.235781in}{1.245722in}}%
\pgfpathlineto{\pgfqpoint{2.232609in}{1.245853in}}%
\pgfpathlineto{\pgfqpoint{2.229437in}{1.245862in}}%
\pgfpathlineto{\pgfqpoint{2.226265in}{1.245847in}}%
\pgfpathlineto{\pgfqpoint{2.223093in}{1.245716in}}%
\pgfpathlineto{\pgfqpoint{2.219921in}{1.245746in}}%
\pgfpathlineto{\pgfqpoint{2.216749in}{1.245724in}}%
\pgfpathlineto{\pgfqpoint{2.213577in}{1.245674in}}%
\pgfpathlineto{\pgfqpoint{2.210405in}{1.245486in}}%
\pgfpathlineto{\pgfqpoint{2.207233in}{1.245297in}}%
\pgfpathlineto{\pgfqpoint{2.204061in}{1.245101in}}%
\pgfpathlineto{\pgfqpoint{2.200889in}{1.245178in}}%
\pgfpathlineto{\pgfqpoint{2.197717in}{1.245228in}}%
\pgfpathlineto{\pgfqpoint{2.194545in}{1.245381in}}%
\pgfpathlineto{\pgfqpoint{2.191372in}{1.245259in}}%
\pgfpathlineto{\pgfqpoint{2.188200in}{1.245125in}}%
\pgfpathlineto{\pgfqpoint{2.185028in}{1.245161in}}%
\pgfpathlineto{\pgfqpoint{2.181856in}{1.245258in}}%
\pgfpathlineto{\pgfqpoint{2.178684in}{1.245013in}}%
\pgfpathlineto{\pgfqpoint{2.175512in}{1.244967in}}%
\pgfpathlineto{\pgfqpoint{2.172340in}{1.244949in}}%
\pgfpathlineto{\pgfqpoint{2.169168in}{1.244985in}}%
\pgfpathlineto{\pgfqpoint{2.165996in}{1.244690in}}%
\pgfpathlineto{\pgfqpoint{2.162824in}{1.244595in}}%
\pgfpathlineto{\pgfqpoint{2.159652in}{1.244537in}}%
\pgfpathlineto{\pgfqpoint{2.156480in}{1.244439in}}%
\pgfpathlineto{\pgfqpoint{2.153308in}{1.244460in}}%
\pgfpathlineto{\pgfqpoint{2.150136in}{1.244533in}}%
\pgfpathlineto{\pgfqpoint{2.146964in}{1.244535in}}%
\pgfpathlineto{\pgfqpoint{2.143792in}{1.244552in}}%
\pgfpathlineto{\pgfqpoint{2.140620in}{1.244690in}}%
\pgfpathlineto{\pgfqpoint{2.137448in}{1.244650in}}%
\pgfpathlineto{\pgfqpoint{2.134276in}{1.244568in}}%
\pgfpathlineto{\pgfqpoint{2.131104in}{1.244411in}}%
\pgfpathlineto{\pgfqpoint{2.127932in}{1.244209in}}%
\pgfpathlineto{\pgfqpoint{2.124760in}{1.244152in}}%
\pgfpathlineto{\pgfqpoint{2.121588in}{1.244228in}}%
\pgfpathlineto{\pgfqpoint{2.118416in}{1.244138in}}%
\pgfpathlineto{\pgfqpoint{2.115244in}{1.244237in}}%
\pgfpathlineto{\pgfqpoint{2.112071in}{1.243988in}}%
\pgfpathlineto{\pgfqpoint{2.108899in}{1.244054in}}%
\pgfpathlineto{\pgfqpoint{2.105727in}{1.244153in}}%
\pgfpathlineto{\pgfqpoint{2.102555in}{1.244193in}}%
\pgfpathlineto{\pgfqpoint{2.099383in}{1.244060in}}%
\pgfpathlineto{\pgfqpoint{2.096211in}{1.244030in}}%
\pgfpathlineto{\pgfqpoint{2.093039in}{1.244170in}}%
\pgfpathlineto{\pgfqpoint{2.089867in}{1.244184in}}%
\pgfpathlineto{\pgfqpoint{2.086695in}{1.244151in}}%
\pgfpathlineto{\pgfqpoint{2.083523in}{1.244175in}}%
\pgfpathlineto{\pgfqpoint{2.080351in}{1.244190in}}%
\pgfpathlineto{\pgfqpoint{2.077179in}{1.244030in}}%
\pgfpathlineto{\pgfqpoint{2.074007in}{1.244128in}}%
\pgfpathlineto{\pgfqpoint{2.070835in}{1.244268in}}%
\pgfpathlineto{\pgfqpoint{2.067663in}{1.244294in}}%
\pgfpathlineto{\pgfqpoint{2.064491in}{1.244060in}}%
\pgfpathlineto{\pgfqpoint{2.061319in}{1.244011in}}%
\pgfpathlineto{\pgfqpoint{2.058147in}{1.243758in}}%
\pgfpathlineto{\pgfqpoint{2.054975in}{1.243678in}}%
\pgfpathlineto{\pgfqpoint{2.051803in}{1.243673in}}%
\pgfpathlineto{\pgfqpoint{2.048631in}{1.243595in}}%
\pgfpathlineto{\pgfqpoint{2.045459in}{1.243248in}}%
\pgfpathlineto{\pgfqpoint{2.042287in}{1.243211in}}%
\pgfpathlineto{\pgfqpoint{2.039115in}{1.242733in}}%
\pgfpathlineto{\pgfqpoint{2.035943in}{1.243285in}}%
\pgfpathlineto{\pgfqpoint{2.032770in}{1.243754in}}%
\pgfpathlineto{\pgfqpoint{2.029598in}{1.244238in}}%
\pgfpathlineto{\pgfqpoint{2.026426in}{1.244749in}}%
\pgfpathlineto{\pgfqpoint{2.023254in}{1.246332in}}%
\pgfpathlineto{\pgfqpoint{2.020082in}{1.246657in}}%
\pgfpathlineto{\pgfqpoint{2.016910in}{1.247273in}}%
\pgfpathlineto{\pgfqpoint{2.013738in}{1.247885in}}%
\pgfpathlineto{\pgfqpoint{2.010566in}{1.248505in}}%
\pgfpathlineto{\pgfqpoint{2.007394in}{1.249039in}}%
\pgfpathlineto{\pgfqpoint{2.004222in}{1.250572in}}%
\pgfpathlineto{\pgfqpoint{2.001050in}{1.250863in}}%
\pgfpathlineto{\pgfqpoint{1.997878in}{1.251015in}}%
\pgfpathlineto{\pgfqpoint{1.994706in}{1.250187in}}%
\pgfpathlineto{\pgfqpoint{1.991534in}{1.249461in}}%
\pgfpathlineto{\pgfqpoint{1.988362in}{1.249690in}}%
\pgfpathlineto{\pgfqpoint{1.985190in}{1.250075in}}%
\pgfpathlineto{\pgfqpoint{1.982018in}{1.229254in}}%
\pgfpathlineto{\pgfqpoint{1.978846in}{1.229701in}}%
\pgfpathlineto{\pgfqpoint{1.975674in}{1.230005in}}%
\pgfpathlineto{\pgfqpoint{1.972502in}{1.231427in}}%
\pgfpathlineto{\pgfqpoint{1.969330in}{1.231919in}}%
\pgfpathlineto{\pgfqpoint{1.966158in}{1.232712in}}%
\pgfpathlineto{\pgfqpoint{1.962986in}{1.232807in}}%
\pgfpathlineto{\pgfqpoint{1.959814in}{1.217158in}}%
\pgfpathlineto{\pgfqpoint{1.956641in}{1.192649in}}%
\pgfpathlineto{\pgfqpoint{1.953469in}{1.173001in}}%
\pgfpathlineto{\pgfqpoint{1.950297in}{1.147379in}}%
\pgfpathlineto{\pgfqpoint{1.947125in}{1.127749in}}%
\pgfpathlineto{\pgfqpoint{1.943953in}{1.127697in}}%
\pgfpathlineto{\pgfqpoint{1.940781in}{1.129362in}}%
\pgfpathclose%
\pgfusepath{stroke,fill}%
\end{pgfscope}%
\begin{pgfscope}%
\pgfpathrectangle{\pgfqpoint{1.623736in}{1.000625in}}{\pgfqpoint{6.975000in}{3.020000in}} %
\pgfusepath{clip}%
\pgfsetbuttcap%
\pgfsetroundjoin%
\definecolor{currentfill}{rgb}{0.576471,0.470588,0.376471}%
\pgfsetfillcolor{currentfill}%
\pgfsetfillopacity{0.200000}%
\pgfsetlinewidth{0.803000pt}%
\definecolor{currentstroke}{rgb}{0.576471,0.470588,0.376471}%
\pgfsetstrokecolor{currentstroke}%
\pgfsetstrokeopacity{0.200000}%
\pgfsetdash{}{0pt}%
\pgfpathmoveto{\pgfqpoint{1.940781in}{1.129362in}}%
\pgfpathlineto{\pgfqpoint{1.940781in}{1.127380in}}%
\pgfpathlineto{\pgfqpoint{1.943953in}{1.126824in}}%
\pgfpathlineto{\pgfqpoint{1.947125in}{1.126875in}}%
\pgfpathlineto{\pgfqpoint{1.950297in}{1.143521in}}%
\pgfpathlineto{\pgfqpoint{1.953469in}{1.168813in}}%
\pgfpathlineto{\pgfqpoint{1.956641in}{1.191671in}}%
\pgfpathlineto{\pgfqpoint{1.959814in}{1.193123in}}%
\pgfpathlineto{\pgfqpoint{1.962986in}{1.191640in}}%
\pgfpathlineto{\pgfqpoint{1.966158in}{1.192679in}}%
\pgfpathlineto{\pgfqpoint{1.969330in}{1.191443in}}%
\pgfpathlineto{\pgfqpoint{1.972502in}{1.189994in}}%
\pgfpathlineto{\pgfqpoint{1.975674in}{1.190782in}}%
\pgfpathlineto{\pgfqpoint{1.978846in}{1.193208in}}%
\pgfpathlineto{\pgfqpoint{1.982018in}{1.192658in}}%
\pgfpathlineto{\pgfqpoint{1.985190in}{1.213953in}}%
\pgfpathlineto{\pgfqpoint{1.988362in}{1.214355in}}%
\pgfpathlineto{\pgfqpoint{1.991534in}{1.216530in}}%
\pgfpathlineto{\pgfqpoint{1.994706in}{1.216134in}}%
\pgfpathlineto{\pgfqpoint{1.997878in}{1.215634in}}%
\pgfpathlineto{\pgfqpoint{2.001050in}{1.215451in}}%
\pgfpathlineto{\pgfqpoint{2.004222in}{1.215395in}}%
\pgfpathlineto{\pgfqpoint{2.007394in}{1.215059in}}%
\pgfpathlineto{\pgfqpoint{2.010566in}{1.214176in}}%
\pgfpathlineto{\pgfqpoint{2.013738in}{1.213627in}}%
\pgfpathlineto{\pgfqpoint{2.016910in}{1.213211in}}%
\pgfpathlineto{\pgfqpoint{2.020082in}{1.213534in}}%
\pgfpathlineto{\pgfqpoint{2.023254in}{1.213362in}}%
\pgfpathlineto{\pgfqpoint{2.026426in}{1.212782in}}%
\pgfpathlineto{\pgfqpoint{2.029598in}{1.211504in}}%
\pgfpathlineto{\pgfqpoint{2.032770in}{1.211843in}}%
\pgfpathlineto{\pgfqpoint{2.035943in}{1.209925in}}%
\pgfpathlineto{\pgfqpoint{2.039115in}{1.209271in}}%
\pgfpathlineto{\pgfqpoint{2.042287in}{1.207225in}}%
\pgfpathlineto{\pgfqpoint{2.045459in}{1.208312in}}%
\pgfpathlineto{\pgfqpoint{2.048631in}{1.208889in}}%
\pgfpathlineto{\pgfqpoint{2.051803in}{1.208811in}}%
\pgfpathlineto{\pgfqpoint{2.054975in}{1.208788in}}%
\pgfpathlineto{\pgfqpoint{2.058147in}{1.208666in}}%
\pgfpathlineto{\pgfqpoint{2.061319in}{1.208537in}}%
\pgfpathlineto{\pgfqpoint{2.064491in}{1.208491in}}%
\pgfpathlineto{\pgfqpoint{2.067663in}{1.208508in}}%
\pgfpathlineto{\pgfqpoint{2.070835in}{1.208600in}}%
\pgfpathlineto{\pgfqpoint{2.074007in}{1.208580in}}%
\pgfpathlineto{\pgfqpoint{2.077179in}{1.208533in}}%
\pgfpathlineto{\pgfqpoint{2.080351in}{1.208649in}}%
\pgfpathlineto{\pgfqpoint{2.083523in}{1.208694in}}%
\pgfpathlineto{\pgfqpoint{2.086695in}{1.208827in}}%
\pgfpathlineto{\pgfqpoint{2.089867in}{1.208742in}}%
\pgfpathlineto{\pgfqpoint{2.093039in}{1.208815in}}%
\pgfpathlineto{\pgfqpoint{2.096211in}{1.208884in}}%
\pgfpathlineto{\pgfqpoint{2.099383in}{1.208578in}}%
\pgfpathlineto{\pgfqpoint{2.102555in}{1.208619in}}%
\pgfpathlineto{\pgfqpoint{2.105727in}{1.208514in}}%
\pgfpathlineto{\pgfqpoint{2.108899in}{1.208721in}}%
\pgfpathlineto{\pgfqpoint{2.112071in}{1.208701in}}%
\pgfpathlineto{\pgfqpoint{2.115244in}{1.208703in}}%
\pgfpathlineto{\pgfqpoint{2.118416in}{1.208902in}}%
\pgfpathlineto{\pgfqpoint{2.121588in}{1.209003in}}%
\pgfpathlineto{\pgfqpoint{2.124760in}{1.209004in}}%
\pgfpathlineto{\pgfqpoint{2.127932in}{1.209000in}}%
\pgfpathlineto{\pgfqpoint{2.131104in}{1.208833in}}%
\pgfpathlineto{\pgfqpoint{2.134276in}{1.208986in}}%
\pgfpathlineto{\pgfqpoint{2.137448in}{1.208924in}}%
\pgfpathlineto{\pgfqpoint{2.140620in}{1.208929in}}%
\pgfpathlineto{\pgfqpoint{2.143792in}{1.209114in}}%
\pgfpathlineto{\pgfqpoint{2.146964in}{1.209335in}}%
\pgfpathlineto{\pgfqpoint{2.150136in}{1.209427in}}%
\pgfpathlineto{\pgfqpoint{2.153308in}{1.209569in}}%
\pgfpathlineto{\pgfqpoint{2.156480in}{1.209610in}}%
\pgfpathlineto{\pgfqpoint{2.159652in}{1.209573in}}%
\pgfpathlineto{\pgfqpoint{2.162824in}{1.209566in}}%
\pgfpathlineto{\pgfqpoint{2.165996in}{1.209625in}}%
\pgfpathlineto{\pgfqpoint{2.169168in}{1.209615in}}%
\pgfpathlineto{\pgfqpoint{2.172340in}{1.209722in}}%
\pgfpathlineto{\pgfqpoint{2.175512in}{1.209625in}}%
\pgfpathlineto{\pgfqpoint{2.178684in}{1.209457in}}%
\pgfpathlineto{\pgfqpoint{2.181856in}{1.209552in}}%
\pgfpathlineto{\pgfqpoint{2.185028in}{1.209615in}}%
\pgfpathlineto{\pgfqpoint{2.188200in}{1.209617in}}%
\pgfpathlineto{\pgfqpoint{2.191372in}{1.209634in}}%
\pgfpathlineto{\pgfqpoint{2.194545in}{1.209658in}}%
\pgfpathlineto{\pgfqpoint{2.197717in}{1.209798in}}%
\pgfpathlineto{\pgfqpoint{2.200889in}{1.209890in}}%
\pgfpathlineto{\pgfqpoint{2.204061in}{1.209790in}}%
\pgfpathlineto{\pgfqpoint{2.207233in}{1.209686in}}%
\pgfpathlineto{\pgfqpoint{2.210405in}{1.209777in}}%
\pgfpathlineto{\pgfqpoint{2.213577in}{1.209779in}}%
\pgfpathlineto{\pgfqpoint{2.216749in}{1.209852in}}%
\pgfpathlineto{\pgfqpoint{2.219921in}{1.209920in}}%
\pgfpathlineto{\pgfqpoint{2.223093in}{1.210198in}}%
\pgfpathlineto{\pgfqpoint{2.226265in}{1.210269in}}%
\pgfpathlineto{\pgfqpoint{2.229437in}{1.210434in}}%
\pgfpathlineto{\pgfqpoint{2.232609in}{1.210364in}}%
\pgfpathlineto{\pgfqpoint{2.235781in}{1.210422in}}%
\pgfpathlineto{\pgfqpoint{2.238953in}{1.210396in}}%
\pgfpathlineto{\pgfqpoint{2.242125in}{1.210311in}}%
\pgfpathlineto{\pgfqpoint{2.245297in}{1.210323in}}%
\pgfpathlineto{\pgfqpoint{2.248469in}{1.210400in}}%
\pgfpathlineto{\pgfqpoint{2.251641in}{1.210596in}}%
\pgfpathlineto{\pgfqpoint{2.254813in}{1.210615in}}%
\pgfpathlineto{\pgfqpoint{2.257985in}{1.210475in}}%
\pgfpathlineto{\pgfqpoint{2.261157in}{1.210377in}}%
\pgfpathlineto{\pgfqpoint{2.264329in}{1.210459in}}%
\pgfpathlineto{\pgfqpoint{2.267501in}{1.210321in}}%
\pgfpathlineto{\pgfqpoint{2.270674in}{1.210147in}}%
\pgfpathlineto{\pgfqpoint{2.273846in}{1.210147in}}%
\pgfpathlineto{\pgfqpoint{2.277018in}{1.210163in}}%
\pgfpathlineto{\pgfqpoint{2.280190in}{1.210206in}}%
\pgfpathlineto{\pgfqpoint{2.283362in}{1.210083in}}%
\pgfpathlineto{\pgfqpoint{2.286534in}{1.209957in}}%
\pgfpathlineto{\pgfqpoint{2.289706in}{1.210016in}}%
\pgfpathlineto{\pgfqpoint{2.292878in}{1.210081in}}%
\pgfpathlineto{\pgfqpoint{2.296050in}{1.210161in}}%
\pgfpathlineto{\pgfqpoint{2.299222in}{1.210276in}}%
\pgfpathlineto{\pgfqpoint{2.302394in}{1.209920in}}%
\pgfpathlineto{\pgfqpoint{2.305566in}{1.209953in}}%
\pgfpathlineto{\pgfqpoint{2.308738in}{1.209920in}}%
\pgfpathlineto{\pgfqpoint{2.311910in}{1.209835in}}%
\pgfpathlineto{\pgfqpoint{2.315082in}{1.209757in}}%
\pgfpathlineto{\pgfqpoint{2.318254in}{1.209997in}}%
\pgfpathlineto{\pgfqpoint{2.321426in}{1.209892in}}%
\pgfpathlineto{\pgfqpoint{2.324598in}{1.209810in}}%
\pgfpathlineto{\pgfqpoint{2.327770in}{1.210092in}}%
\pgfpathlineto{\pgfqpoint{2.330942in}{1.210110in}}%
\pgfpathlineto{\pgfqpoint{2.334114in}{1.210128in}}%
\pgfpathlineto{\pgfqpoint{2.337286in}{1.209997in}}%
\pgfpathlineto{\pgfqpoint{2.340458in}{1.210102in}}%
\pgfpathlineto{\pgfqpoint{2.343630in}{1.210155in}}%
\pgfpathlineto{\pgfqpoint{2.346802in}{1.210124in}}%
\pgfpathlineto{\pgfqpoint{2.349975in}{1.209988in}}%
\pgfpathlineto{\pgfqpoint{2.353147in}{1.209947in}}%
\pgfpathlineto{\pgfqpoint{2.356319in}{1.210084in}}%
\pgfpathlineto{\pgfqpoint{2.359491in}{1.210105in}}%
\pgfpathlineto{\pgfqpoint{2.362663in}{1.210044in}}%
\pgfpathlineto{\pgfqpoint{2.365835in}{1.210206in}}%
\pgfpathlineto{\pgfqpoint{2.369007in}{1.210126in}}%
\pgfpathlineto{\pgfqpoint{2.372179in}{1.210075in}}%
\pgfpathlineto{\pgfqpoint{2.375351in}{1.209961in}}%
\pgfpathlineto{\pgfqpoint{2.378523in}{1.209966in}}%
\pgfpathlineto{\pgfqpoint{2.381695in}{1.210008in}}%
\pgfpathlineto{\pgfqpoint{2.384867in}{1.210083in}}%
\pgfpathlineto{\pgfqpoint{2.388039in}{1.210100in}}%
\pgfpathlineto{\pgfqpoint{2.391211in}{1.209999in}}%
\pgfpathlineto{\pgfqpoint{2.394383in}{1.209848in}}%
\pgfpathlineto{\pgfqpoint{2.397555in}{1.209876in}}%
\pgfpathlineto{\pgfqpoint{2.400727in}{1.209816in}}%
\pgfpathlineto{\pgfqpoint{2.403899in}{1.209797in}}%
\pgfpathlineto{\pgfqpoint{2.407071in}{1.209754in}}%
\pgfpathlineto{\pgfqpoint{2.410243in}{1.209814in}}%
\pgfpathlineto{\pgfqpoint{2.413415in}{1.209705in}}%
\pgfpathlineto{\pgfqpoint{2.416587in}{1.209826in}}%
\pgfpathlineto{\pgfqpoint{2.419759in}{1.209814in}}%
\pgfpathlineto{\pgfqpoint{2.422931in}{1.210223in}}%
\pgfpathlineto{\pgfqpoint{2.426103in}{1.210225in}}%
\pgfpathlineto{\pgfqpoint{2.429276in}{1.210277in}}%
\pgfpathlineto{\pgfqpoint{2.432448in}{1.210345in}}%
\pgfpathlineto{\pgfqpoint{2.435620in}{1.210468in}}%
\pgfpathlineto{\pgfqpoint{2.438792in}{1.210753in}}%
\pgfpathlineto{\pgfqpoint{2.441964in}{1.210791in}}%
\pgfpathlineto{\pgfqpoint{2.445136in}{1.210778in}}%
\pgfpathlineto{\pgfqpoint{2.448308in}{1.210770in}}%
\pgfpathlineto{\pgfqpoint{2.451480in}{1.210847in}}%
\pgfpathlineto{\pgfqpoint{2.454652in}{1.210823in}}%
\pgfpathlineto{\pgfqpoint{2.457824in}{1.210807in}}%
\pgfpathlineto{\pgfqpoint{2.460996in}{1.210915in}}%
\pgfpathlineto{\pgfqpoint{2.464168in}{1.210892in}}%
\pgfpathlineto{\pgfqpoint{2.467340in}{1.211065in}}%
\pgfpathlineto{\pgfqpoint{2.470512in}{1.211104in}}%
\pgfpathlineto{\pgfqpoint{2.473684in}{1.211194in}}%
\pgfpathlineto{\pgfqpoint{2.476856in}{1.211346in}}%
\pgfpathlineto{\pgfqpoint{2.480028in}{1.211226in}}%
\pgfpathlineto{\pgfqpoint{2.483200in}{1.211236in}}%
\pgfpathlineto{\pgfqpoint{2.486372in}{1.211293in}}%
\pgfpathlineto{\pgfqpoint{2.489544in}{1.211240in}}%
\pgfpathlineto{\pgfqpoint{2.492716in}{1.211141in}}%
\pgfpathlineto{\pgfqpoint{2.495888in}{1.211168in}}%
\pgfpathlineto{\pgfqpoint{2.499060in}{1.211265in}}%
\pgfpathlineto{\pgfqpoint{2.502232in}{1.211125in}}%
\pgfpathlineto{\pgfqpoint{2.505405in}{1.211051in}}%
\pgfpathlineto{\pgfqpoint{2.508577in}{1.211032in}}%
\pgfpathlineto{\pgfqpoint{2.511749in}{1.211008in}}%
\pgfpathlineto{\pgfqpoint{2.514921in}{1.211095in}}%
\pgfpathlineto{\pgfqpoint{2.518093in}{1.211031in}}%
\pgfpathlineto{\pgfqpoint{2.521265in}{1.211185in}}%
\pgfpathlineto{\pgfqpoint{2.524437in}{1.211115in}}%
\pgfpathlineto{\pgfqpoint{2.527609in}{1.211052in}}%
\pgfpathlineto{\pgfqpoint{2.530781in}{1.210987in}}%
\pgfpathlineto{\pgfqpoint{2.533953in}{1.211093in}}%
\pgfpathlineto{\pgfqpoint{2.537125in}{1.211187in}}%
\pgfpathlineto{\pgfqpoint{2.540297in}{1.211597in}}%
\pgfpathlineto{\pgfqpoint{2.543469in}{1.211679in}}%
\pgfpathlineto{\pgfqpoint{2.546641in}{1.211636in}}%
\pgfpathlineto{\pgfqpoint{2.549813in}{1.211668in}}%
\pgfpathlineto{\pgfqpoint{2.552985in}{1.211519in}}%
\pgfpathlineto{\pgfqpoint{2.556157in}{1.211474in}}%
\pgfpathlineto{\pgfqpoint{2.559329in}{1.211539in}}%
\pgfpathlineto{\pgfqpoint{2.562501in}{1.211597in}}%
\pgfpathlineto{\pgfqpoint{2.565673in}{1.211418in}}%
\pgfpathlineto{\pgfqpoint{2.568845in}{1.211486in}}%
\pgfpathlineto{\pgfqpoint{2.572017in}{1.211530in}}%
\pgfpathlineto{\pgfqpoint{2.575189in}{1.211493in}}%
\pgfpathlineto{\pgfqpoint{2.578361in}{1.211441in}}%
\pgfpathlineto{\pgfqpoint{2.581533in}{1.211573in}}%
\pgfpathlineto{\pgfqpoint{2.584706in}{1.211620in}}%
\pgfpathlineto{\pgfqpoint{2.587878in}{1.211704in}}%
\pgfpathlineto{\pgfqpoint{2.591050in}{1.211559in}}%
\pgfpathlineto{\pgfqpoint{2.594222in}{1.211547in}}%
\pgfpathlineto{\pgfqpoint{2.597394in}{1.211531in}}%
\pgfpathlineto{\pgfqpoint{2.600566in}{1.211525in}}%
\pgfpathlineto{\pgfqpoint{2.603738in}{1.211483in}}%
\pgfpathlineto{\pgfqpoint{2.606910in}{1.211314in}}%
\pgfpathlineto{\pgfqpoint{2.610082in}{1.211615in}}%
\pgfpathlineto{\pgfqpoint{2.613254in}{1.211562in}}%
\pgfpathlineto{\pgfqpoint{2.616426in}{1.211475in}}%
\pgfpathlineto{\pgfqpoint{2.619598in}{1.211354in}}%
\pgfpathlineto{\pgfqpoint{2.622770in}{1.211370in}}%
\pgfpathlineto{\pgfqpoint{2.625942in}{1.211311in}}%
\pgfpathlineto{\pgfqpoint{2.629114in}{1.211236in}}%
\pgfpathlineto{\pgfqpoint{2.632286in}{1.210960in}}%
\pgfpathlineto{\pgfqpoint{2.635458in}{1.210954in}}%
\pgfpathlineto{\pgfqpoint{2.638630in}{1.210936in}}%
\pgfpathlineto{\pgfqpoint{2.641802in}{1.210858in}}%
\pgfpathlineto{\pgfqpoint{2.644974in}{1.210667in}}%
\pgfpathlineto{\pgfqpoint{2.648146in}{1.210465in}}%
\pgfpathlineto{\pgfqpoint{2.651318in}{1.210389in}}%
\pgfpathlineto{\pgfqpoint{2.654490in}{1.210704in}}%
\pgfpathlineto{\pgfqpoint{2.657662in}{1.210892in}}%
\pgfpathlineto{\pgfqpoint{2.660834in}{1.210863in}}%
\pgfpathlineto{\pgfqpoint{2.664007in}{1.210841in}}%
\pgfpathlineto{\pgfqpoint{2.667179in}{1.210876in}}%
\pgfpathlineto{\pgfqpoint{2.670351in}{1.210663in}}%
\pgfpathlineto{\pgfqpoint{2.673523in}{1.210583in}}%
\pgfpathlineto{\pgfqpoint{2.676695in}{1.210666in}}%
\pgfpathlineto{\pgfqpoint{2.679867in}{1.210698in}}%
\pgfpathlineto{\pgfqpoint{2.683039in}{1.210725in}}%
\pgfpathlineto{\pgfqpoint{2.686211in}{1.210712in}}%
\pgfpathlineto{\pgfqpoint{2.689383in}{1.210654in}}%
\pgfpathlineto{\pgfqpoint{2.692555in}{1.210624in}}%
\pgfpathlineto{\pgfqpoint{2.695727in}{1.210608in}}%
\pgfpathlineto{\pgfqpoint{2.698899in}{1.210593in}}%
\pgfpathlineto{\pgfqpoint{2.702071in}{1.210710in}}%
\pgfpathlineto{\pgfqpoint{2.705243in}{1.210470in}}%
\pgfpathlineto{\pgfqpoint{2.708415in}{1.210471in}}%
\pgfpathlineto{\pgfqpoint{2.711587in}{1.210415in}}%
\pgfpathlineto{\pgfqpoint{2.714759in}{1.210512in}}%
\pgfpathlineto{\pgfqpoint{2.717931in}{1.210326in}}%
\pgfpathlineto{\pgfqpoint{2.721103in}{1.210228in}}%
\pgfpathlineto{\pgfqpoint{2.724275in}{1.210227in}}%
\pgfpathlineto{\pgfqpoint{2.727447in}{1.210295in}}%
\pgfpathlineto{\pgfqpoint{2.730619in}{1.210473in}}%
\pgfpathlineto{\pgfqpoint{2.733791in}{1.210757in}}%
\pgfpathlineto{\pgfqpoint{2.736963in}{1.210790in}}%
\pgfpathlineto{\pgfqpoint{2.740136in}{1.210728in}}%
\pgfpathlineto{\pgfqpoint{2.743308in}{1.210699in}}%
\pgfpathlineto{\pgfqpoint{2.746480in}{1.210790in}}%
\pgfpathlineto{\pgfqpoint{2.749652in}{1.210808in}}%
\pgfpathlineto{\pgfqpoint{2.752824in}{1.210787in}}%
\pgfpathlineto{\pgfqpoint{2.755996in}{1.210659in}}%
\pgfpathlineto{\pgfqpoint{2.759168in}{1.210592in}}%
\pgfpathlineto{\pgfqpoint{2.762340in}{1.210863in}}%
\pgfpathlineto{\pgfqpoint{2.765512in}{1.211052in}}%
\pgfpathlineto{\pgfqpoint{2.768684in}{1.210876in}}%
\pgfpathlineto{\pgfqpoint{2.771856in}{1.210874in}}%
\pgfpathlineto{\pgfqpoint{2.775028in}{1.210931in}}%
\pgfpathlineto{\pgfqpoint{2.778200in}{1.210902in}}%
\pgfpathlineto{\pgfqpoint{2.781372in}{1.210615in}}%
\pgfpathlineto{\pgfqpoint{2.784544in}{1.210601in}}%
\pgfpathlineto{\pgfqpoint{2.787716in}{1.210584in}}%
\pgfpathlineto{\pgfqpoint{2.790888in}{1.210657in}}%
\pgfpathlineto{\pgfqpoint{2.794060in}{1.210778in}}%
\pgfpathlineto{\pgfqpoint{2.797232in}{1.210963in}}%
\pgfpathlineto{\pgfqpoint{2.800404in}{1.210992in}}%
\pgfpathlineto{\pgfqpoint{2.803576in}{1.211065in}}%
\pgfpathlineto{\pgfqpoint{2.806748in}{1.211160in}}%
\pgfpathlineto{\pgfqpoint{2.809920in}{1.211280in}}%
\pgfpathlineto{\pgfqpoint{2.813092in}{1.211200in}}%
\pgfpathlineto{\pgfqpoint{2.816264in}{1.211230in}}%
\pgfpathlineto{\pgfqpoint{2.819437in}{1.211460in}}%
\pgfpathlineto{\pgfqpoint{2.822609in}{1.211477in}}%
\pgfpathlineto{\pgfqpoint{2.825781in}{1.211557in}}%
\pgfpathlineto{\pgfqpoint{2.828953in}{1.211889in}}%
\pgfpathlineto{\pgfqpoint{2.832125in}{1.211842in}}%
\pgfpathlineto{\pgfqpoint{2.835297in}{1.211647in}}%
\pgfpathlineto{\pgfqpoint{2.838469in}{1.211632in}}%
\pgfpathlineto{\pgfqpoint{2.841641in}{1.211839in}}%
\pgfpathlineto{\pgfqpoint{2.844813in}{1.211838in}}%
\pgfpathlineto{\pgfqpoint{2.847985in}{1.211770in}}%
\pgfpathlineto{\pgfqpoint{2.851157in}{1.211516in}}%
\pgfpathlineto{\pgfqpoint{2.854329in}{1.211493in}}%
\pgfpathlineto{\pgfqpoint{2.857501in}{1.211597in}}%
\pgfpathlineto{\pgfqpoint{2.860673in}{1.211769in}}%
\pgfpathlineto{\pgfqpoint{2.863845in}{1.211645in}}%
\pgfpathlineto{\pgfqpoint{2.867017in}{1.211527in}}%
\pgfpathlineto{\pgfqpoint{2.870189in}{1.211831in}}%
\pgfpathlineto{\pgfqpoint{2.873361in}{1.211778in}}%
\pgfpathlineto{\pgfqpoint{2.876533in}{1.211918in}}%
\pgfpathlineto{\pgfqpoint{2.879705in}{1.212102in}}%
\pgfpathlineto{\pgfqpoint{2.882877in}{1.212118in}}%
\pgfpathlineto{\pgfqpoint{2.886049in}{1.211907in}}%
\pgfpathlineto{\pgfqpoint{2.889221in}{1.212017in}}%
\pgfpathlineto{\pgfqpoint{2.892393in}{1.211900in}}%
\pgfpathlineto{\pgfqpoint{2.895565in}{1.212122in}}%
\pgfpathlineto{\pgfqpoint{2.898738in}{1.212252in}}%
\pgfpathlineto{\pgfqpoint{2.901910in}{1.212454in}}%
\pgfpathlineto{\pgfqpoint{2.905082in}{1.212252in}}%
\pgfpathlineto{\pgfqpoint{2.908254in}{1.212164in}}%
\pgfpathlineto{\pgfqpoint{2.911426in}{1.212005in}}%
\pgfpathlineto{\pgfqpoint{2.914598in}{1.212022in}}%
\pgfpathlineto{\pgfqpoint{2.917770in}{1.212194in}}%
\pgfpathlineto{\pgfqpoint{2.920942in}{1.212160in}}%
\pgfpathlineto{\pgfqpoint{2.924114in}{1.212237in}}%
\pgfpathlineto{\pgfqpoint{2.927286in}{1.212308in}}%
\pgfpathlineto{\pgfqpoint{2.930458in}{1.212373in}}%
\pgfpathlineto{\pgfqpoint{2.933630in}{1.212502in}}%
\pgfpathlineto{\pgfqpoint{2.936802in}{1.212491in}}%
\pgfpathlineto{\pgfqpoint{2.939974in}{1.212496in}}%
\pgfpathlineto{\pgfqpoint{2.943146in}{1.212363in}}%
\pgfpathlineto{\pgfqpoint{2.946318in}{1.212550in}}%
\pgfpathlineto{\pgfqpoint{2.949490in}{1.212564in}}%
\pgfpathlineto{\pgfqpoint{2.952662in}{1.212467in}}%
\pgfpathlineto{\pgfqpoint{2.955834in}{1.212654in}}%
\pgfpathlineto{\pgfqpoint{2.959006in}{1.213302in}}%
\pgfpathlineto{\pgfqpoint{2.962178in}{1.213302in}}%
\pgfpathlineto{\pgfqpoint{2.965350in}{1.213126in}}%
\pgfpathlineto{\pgfqpoint{2.968522in}{1.213283in}}%
\pgfpathlineto{\pgfqpoint{2.971694in}{1.213525in}}%
\pgfpathlineto{\pgfqpoint{2.974867in}{1.213630in}}%
\pgfpathlineto{\pgfqpoint{2.978039in}{1.213686in}}%
\pgfpathlineto{\pgfqpoint{2.981211in}{1.213925in}}%
\pgfpathlineto{\pgfqpoint{2.984383in}{1.214035in}}%
\pgfpathlineto{\pgfqpoint{2.987555in}{1.214332in}}%
\pgfpathlineto{\pgfqpoint{2.990727in}{1.214398in}}%
\pgfpathlineto{\pgfqpoint{2.993899in}{1.214458in}}%
\pgfpathlineto{\pgfqpoint{2.997071in}{1.214634in}}%
\pgfpathlineto{\pgfqpoint{3.000243in}{1.214772in}}%
\pgfpathlineto{\pgfqpoint{3.003415in}{1.214801in}}%
\pgfpathlineto{\pgfqpoint{3.006587in}{1.215068in}}%
\pgfpathlineto{\pgfqpoint{3.009759in}{1.215177in}}%
\pgfpathlineto{\pgfqpoint{3.012931in}{1.215091in}}%
\pgfpathlineto{\pgfqpoint{3.016103in}{1.215019in}}%
\pgfpathlineto{\pgfqpoint{3.019275in}{1.214886in}}%
\pgfpathlineto{\pgfqpoint{3.022447in}{1.214787in}}%
\pgfpathlineto{\pgfqpoint{3.025619in}{1.214847in}}%
\pgfpathlineto{\pgfqpoint{3.028791in}{1.214930in}}%
\pgfpathlineto{\pgfqpoint{3.031963in}{1.214967in}}%
\pgfpathlineto{\pgfqpoint{3.035135in}{1.214930in}}%
\pgfpathlineto{\pgfqpoint{3.038307in}{1.214920in}}%
\pgfpathlineto{\pgfqpoint{3.041479in}{1.214781in}}%
\pgfpathlineto{\pgfqpoint{3.044651in}{1.214809in}}%
\pgfpathlineto{\pgfqpoint{3.047823in}{1.214785in}}%
\pgfpathlineto{\pgfqpoint{3.050995in}{1.214744in}}%
\pgfpathlineto{\pgfqpoint{3.054168in}{1.214708in}}%
\pgfpathlineto{\pgfqpoint{3.057340in}{1.214965in}}%
\pgfpathlineto{\pgfqpoint{3.060512in}{1.215153in}}%
\pgfpathlineto{\pgfqpoint{3.063684in}{1.215119in}}%
\pgfpathlineto{\pgfqpoint{3.066856in}{1.215014in}}%
\pgfpathlineto{\pgfqpoint{3.070028in}{1.214967in}}%
\pgfpathlineto{\pgfqpoint{3.073200in}{1.214852in}}%
\pgfpathlineto{\pgfqpoint{3.076372in}{1.214702in}}%
\pgfpathlineto{\pgfqpoint{3.079544in}{1.214796in}}%
\pgfpathlineto{\pgfqpoint{3.082716in}{1.214902in}}%
\pgfpathlineto{\pgfqpoint{3.085888in}{1.214949in}}%
\pgfpathlineto{\pgfqpoint{3.089060in}{1.214889in}}%
\pgfpathlineto{\pgfqpoint{3.092232in}{1.214972in}}%
\pgfpathlineto{\pgfqpoint{3.095404in}{1.215126in}}%
\pgfpathlineto{\pgfqpoint{3.098576in}{1.215077in}}%
\pgfpathlineto{\pgfqpoint{3.101748in}{1.215108in}}%
\pgfpathlineto{\pgfqpoint{3.104920in}{1.215145in}}%
\pgfpathlineto{\pgfqpoint{3.108092in}{1.215243in}}%
\pgfpathlineto{\pgfqpoint{3.111264in}{1.215521in}}%
\pgfpathlineto{\pgfqpoint{3.114436in}{1.215472in}}%
\pgfpathlineto{\pgfqpoint{3.117608in}{1.215590in}}%
\pgfpathlineto{\pgfqpoint{3.120780in}{1.215491in}}%
\pgfpathlineto{\pgfqpoint{3.123952in}{1.215661in}}%
\pgfpathlineto{\pgfqpoint{3.127124in}{1.215713in}}%
\pgfpathlineto{\pgfqpoint{3.130297in}{1.215598in}}%
\pgfpathlineto{\pgfqpoint{3.133469in}{1.215425in}}%
\pgfpathlineto{\pgfqpoint{3.136641in}{1.215616in}}%
\pgfpathlineto{\pgfqpoint{3.139813in}{1.215853in}}%
\pgfpathlineto{\pgfqpoint{3.142985in}{1.216196in}}%
\pgfpathlineto{\pgfqpoint{3.146157in}{1.215558in}}%
\pgfpathlineto{\pgfqpoint{3.149329in}{1.215419in}}%
\pgfpathlineto{\pgfqpoint{3.152501in}{1.215303in}}%
\pgfpathlineto{\pgfqpoint{3.155673in}{1.215486in}}%
\pgfpathlineto{\pgfqpoint{3.158845in}{1.215656in}}%
\pgfpathlineto{\pgfqpoint{3.162017in}{1.215688in}}%
\pgfpathlineto{\pgfqpoint{3.165189in}{1.215427in}}%
\pgfpathlineto{\pgfqpoint{3.168361in}{1.216033in}}%
\pgfpathlineto{\pgfqpoint{3.171533in}{1.216022in}}%
\pgfpathlineto{\pgfqpoint{3.174705in}{1.216041in}}%
\pgfpathlineto{\pgfqpoint{3.177877in}{1.215768in}}%
\pgfpathlineto{\pgfqpoint{3.181049in}{1.215481in}}%
\pgfpathlineto{\pgfqpoint{3.184221in}{1.215610in}}%
\pgfpathlineto{\pgfqpoint{3.187393in}{1.215719in}}%
\pgfpathlineto{\pgfqpoint{3.190565in}{1.215667in}}%
\pgfpathlineto{\pgfqpoint{3.193737in}{1.215628in}}%
\pgfpathlineto{\pgfqpoint{3.196909in}{1.215446in}}%
\pgfpathlineto{\pgfqpoint{3.200081in}{1.215598in}}%
\pgfpathlineto{\pgfqpoint{3.203253in}{1.215411in}}%
\pgfpathlineto{\pgfqpoint{3.206425in}{1.215316in}}%
\pgfpathlineto{\pgfqpoint{3.209598in}{1.214354in}}%
\pgfpathlineto{\pgfqpoint{3.212770in}{1.213994in}}%
\pgfpathlineto{\pgfqpoint{3.215942in}{1.214058in}}%
\pgfpathlineto{\pgfqpoint{3.219114in}{1.213773in}}%
\pgfpathlineto{\pgfqpoint{3.222286in}{1.213726in}}%
\pgfpathlineto{\pgfqpoint{3.225458in}{1.214258in}}%
\pgfpathlineto{\pgfqpoint{3.228630in}{1.213961in}}%
\pgfpathlineto{\pgfqpoint{3.231802in}{1.213358in}}%
\pgfpathlineto{\pgfqpoint{3.234974in}{1.213173in}}%
\pgfpathlineto{\pgfqpoint{3.238146in}{1.212631in}}%
\pgfpathlineto{\pgfqpoint{3.241318in}{1.212459in}}%
\pgfpathlineto{\pgfqpoint{3.244490in}{1.211859in}}%
\pgfpathlineto{\pgfqpoint{3.247662in}{1.211638in}}%
\pgfpathlineto{\pgfqpoint{3.250834in}{1.211592in}}%
\pgfpathlineto{\pgfqpoint{3.254006in}{1.211314in}}%
\pgfpathlineto{\pgfqpoint{3.257178in}{1.210865in}}%
\pgfpathlineto{\pgfqpoint{3.260350in}{1.210532in}}%
\pgfpathlineto{\pgfqpoint{3.263522in}{1.210018in}}%
\pgfpathlineto{\pgfqpoint{3.266694in}{1.210202in}}%
\pgfpathlineto{\pgfqpoint{3.269866in}{1.210118in}}%
\pgfpathlineto{\pgfqpoint{3.273038in}{1.209740in}}%
\pgfpathlineto{\pgfqpoint{3.276210in}{1.209959in}}%
\pgfpathlineto{\pgfqpoint{3.279382in}{1.210445in}}%
\pgfpathlineto{\pgfqpoint{3.282554in}{1.210834in}}%
\pgfpathlineto{\pgfqpoint{3.285726in}{1.210524in}}%
\pgfpathlineto{\pgfqpoint{3.288899in}{1.210517in}}%
\pgfpathlineto{\pgfqpoint{3.292071in}{1.210899in}}%
\pgfpathlineto{\pgfqpoint{3.295243in}{1.210914in}}%
\pgfpathlineto{\pgfqpoint{3.298415in}{1.210826in}}%
\pgfpathlineto{\pgfqpoint{3.301587in}{1.211373in}}%
\pgfpathlineto{\pgfqpoint{3.304759in}{1.211266in}}%
\pgfpathlineto{\pgfqpoint{3.307931in}{1.211677in}}%
\pgfpathlineto{\pgfqpoint{3.311103in}{1.211326in}}%
\pgfpathlineto{\pgfqpoint{3.314275in}{1.211387in}}%
\pgfpathlineto{\pgfqpoint{3.317447in}{1.211317in}}%
\pgfpathlineto{\pgfqpoint{3.320619in}{1.211414in}}%
\pgfpathlineto{\pgfqpoint{3.323791in}{1.211640in}}%
\pgfpathlineto{\pgfqpoint{3.326963in}{1.211318in}}%
\pgfpathlineto{\pgfqpoint{3.330135in}{1.211868in}}%
\pgfpathlineto{\pgfqpoint{3.333307in}{1.211766in}}%
\pgfpathlineto{\pgfqpoint{3.336479in}{1.211581in}}%
\pgfpathlineto{\pgfqpoint{3.339651in}{1.211779in}}%
\pgfpathlineto{\pgfqpoint{3.342823in}{1.211776in}}%
\pgfpathlineto{\pgfqpoint{3.345995in}{1.211731in}}%
\pgfpathlineto{\pgfqpoint{3.349167in}{1.211468in}}%
\pgfpathlineto{\pgfqpoint{3.352339in}{1.210868in}}%
\pgfpathlineto{\pgfqpoint{3.355511in}{1.210624in}}%
\pgfpathlineto{\pgfqpoint{3.358683in}{1.210835in}}%
\pgfpathlineto{\pgfqpoint{3.361855in}{1.210626in}}%
\pgfpathlineto{\pgfqpoint{3.365028in}{1.210332in}}%
\pgfpathlineto{\pgfqpoint{3.368200in}{1.210453in}}%
\pgfpathlineto{\pgfqpoint{3.371372in}{1.210240in}}%
\pgfpathlineto{\pgfqpoint{3.374544in}{1.210364in}}%
\pgfpathlineto{\pgfqpoint{3.377716in}{1.210381in}}%
\pgfpathlineto{\pgfqpoint{3.380888in}{1.210661in}}%
\pgfpathlineto{\pgfqpoint{3.384060in}{1.211037in}}%
\pgfpathlineto{\pgfqpoint{3.387232in}{1.211063in}}%
\pgfpathlineto{\pgfqpoint{3.390404in}{1.211004in}}%
\pgfpathlineto{\pgfqpoint{3.393576in}{1.210977in}}%
\pgfpathlineto{\pgfqpoint{3.396748in}{1.210545in}}%
\pgfpathlineto{\pgfqpoint{3.399920in}{1.210271in}}%
\pgfpathlineto{\pgfqpoint{3.403092in}{1.209655in}}%
\pgfpathlineto{\pgfqpoint{3.406264in}{1.209784in}}%
\pgfpathlineto{\pgfqpoint{3.409436in}{1.209857in}}%
\pgfpathlineto{\pgfqpoint{3.412608in}{1.209901in}}%
\pgfpathlineto{\pgfqpoint{3.415780in}{1.209575in}}%
\pgfpathlineto{\pgfqpoint{3.418952in}{1.209299in}}%
\pgfpathlineto{\pgfqpoint{3.422124in}{1.209595in}}%
\pgfpathlineto{\pgfqpoint{3.425296in}{1.209461in}}%
\pgfpathlineto{\pgfqpoint{3.428468in}{1.209168in}}%
\pgfpathlineto{\pgfqpoint{3.431640in}{1.209270in}}%
\pgfpathlineto{\pgfqpoint{3.434812in}{1.208978in}}%
\pgfpathlineto{\pgfqpoint{3.437984in}{1.208754in}}%
\pgfpathlineto{\pgfqpoint{3.441156in}{1.208925in}}%
\pgfpathlineto{\pgfqpoint{3.444329in}{1.209035in}}%
\pgfpathlineto{\pgfqpoint{3.447501in}{1.208794in}}%
\pgfpathlineto{\pgfqpoint{3.450673in}{1.208784in}}%
\pgfpathlineto{\pgfqpoint{3.453845in}{1.208625in}}%
\pgfpathlineto{\pgfqpoint{3.457017in}{1.208623in}}%
\pgfpathlineto{\pgfqpoint{3.460189in}{1.208741in}}%
\pgfpathlineto{\pgfqpoint{3.463361in}{1.208502in}}%
\pgfpathlineto{\pgfqpoint{3.466533in}{1.208068in}}%
\pgfpathlineto{\pgfqpoint{3.469705in}{1.208154in}}%
\pgfpathlineto{\pgfqpoint{3.472877in}{1.208164in}}%
\pgfpathlineto{\pgfqpoint{3.476049in}{1.208529in}}%
\pgfpathlineto{\pgfqpoint{3.479221in}{1.208449in}}%
\pgfpathlineto{\pgfqpoint{3.482393in}{1.208550in}}%
\pgfpathlineto{\pgfqpoint{3.485565in}{1.208637in}}%
\pgfpathlineto{\pgfqpoint{3.488737in}{1.208779in}}%
\pgfpathlineto{\pgfqpoint{3.491909in}{1.208646in}}%
\pgfpathlineto{\pgfqpoint{3.495081in}{1.208608in}}%
\pgfpathlineto{\pgfqpoint{3.498253in}{1.209019in}}%
\pgfpathlineto{\pgfqpoint{3.501425in}{1.208498in}}%
\pgfpathlineto{\pgfqpoint{3.504597in}{1.208697in}}%
\pgfpathlineto{\pgfqpoint{3.507769in}{1.208669in}}%
\pgfpathlineto{\pgfqpoint{3.510941in}{1.208718in}}%
\pgfpathlineto{\pgfqpoint{3.514113in}{1.208739in}}%
\pgfpathlineto{\pgfqpoint{3.517285in}{1.208607in}}%
\pgfpathlineto{\pgfqpoint{3.520457in}{1.207931in}}%
\pgfpathlineto{\pgfqpoint{3.523630in}{1.207635in}}%
\pgfpathlineto{\pgfqpoint{3.526802in}{1.207520in}}%
\pgfpathlineto{\pgfqpoint{3.529974in}{1.207803in}}%
\pgfpathlineto{\pgfqpoint{3.533146in}{1.207438in}}%
\pgfpathlineto{\pgfqpoint{3.536318in}{1.207192in}}%
\pgfpathlineto{\pgfqpoint{3.539490in}{1.207294in}}%
\pgfpathlineto{\pgfqpoint{3.542662in}{1.206753in}}%
\pgfpathlineto{\pgfqpoint{3.545834in}{1.207060in}}%
\pgfpathlineto{\pgfqpoint{3.549006in}{1.207287in}}%
\pgfpathlineto{\pgfqpoint{3.552178in}{1.207335in}}%
\pgfpathlineto{\pgfqpoint{3.555350in}{1.207253in}}%
\pgfpathlineto{\pgfqpoint{3.558522in}{1.207488in}}%
\pgfpathlineto{\pgfqpoint{3.561694in}{1.207526in}}%
\pgfpathlineto{\pgfqpoint{3.564866in}{1.207532in}}%
\pgfpathlineto{\pgfqpoint{3.568038in}{1.208036in}}%
\pgfpathlineto{\pgfqpoint{3.571210in}{1.207972in}}%
\pgfpathlineto{\pgfqpoint{3.574382in}{1.207445in}}%
\pgfpathlineto{\pgfqpoint{3.577554in}{1.207418in}}%
\pgfpathlineto{\pgfqpoint{3.580726in}{1.207335in}}%
\pgfpathlineto{\pgfqpoint{3.583898in}{1.206716in}}%
\pgfpathlineto{\pgfqpoint{3.587070in}{1.206670in}}%
\pgfpathlineto{\pgfqpoint{3.590242in}{1.206663in}}%
\pgfpathlineto{\pgfqpoint{3.593414in}{1.207087in}}%
\pgfpathlineto{\pgfqpoint{3.596586in}{1.206964in}}%
\pgfpathlineto{\pgfqpoint{3.599759in}{1.206294in}}%
\pgfpathlineto{\pgfqpoint{3.602931in}{1.206082in}}%
\pgfpathlineto{\pgfqpoint{3.606103in}{1.206092in}}%
\pgfpathlineto{\pgfqpoint{3.609275in}{1.206542in}}%
\pgfpathlineto{\pgfqpoint{3.612447in}{1.206608in}}%
\pgfpathlineto{\pgfqpoint{3.615619in}{1.206639in}}%
\pgfpathlineto{\pgfqpoint{3.618791in}{1.206444in}}%
\pgfpathlineto{\pgfqpoint{3.621963in}{1.206205in}}%
\pgfpathlineto{\pgfqpoint{3.625135in}{1.206360in}}%
\pgfpathlineto{\pgfqpoint{3.628307in}{1.206100in}}%
\pgfpathlineto{\pgfqpoint{3.631479in}{1.205753in}}%
\pgfpathlineto{\pgfqpoint{3.634651in}{1.205674in}}%
\pgfpathlineto{\pgfqpoint{3.637823in}{1.205321in}}%
\pgfpathlineto{\pgfqpoint{3.640995in}{1.204866in}}%
\pgfpathlineto{\pgfqpoint{3.644167in}{1.204589in}}%
\pgfpathlineto{\pgfqpoint{3.647339in}{1.204012in}}%
\pgfpathlineto{\pgfqpoint{3.650511in}{1.203833in}}%
\pgfpathlineto{\pgfqpoint{3.653683in}{1.203275in}}%
\pgfpathlineto{\pgfqpoint{3.656855in}{1.202588in}}%
\pgfpathlineto{\pgfqpoint{3.660027in}{1.202387in}}%
\pgfpathlineto{\pgfqpoint{3.663199in}{1.202106in}}%
\pgfpathlineto{\pgfqpoint{3.666371in}{1.202175in}}%
\pgfpathlineto{\pgfqpoint{3.669543in}{1.201161in}}%
\pgfpathlineto{\pgfqpoint{3.672715in}{1.201125in}}%
\pgfpathlineto{\pgfqpoint{3.675887in}{1.201293in}}%
\pgfpathlineto{\pgfqpoint{3.679060in}{1.201146in}}%
\pgfpathlineto{\pgfqpoint{3.682232in}{1.201131in}}%
\pgfpathlineto{\pgfqpoint{3.685404in}{1.201079in}}%
\pgfpathlineto{\pgfqpoint{3.688576in}{1.200725in}}%
\pgfpathlineto{\pgfqpoint{3.691748in}{1.200659in}}%
\pgfpathlineto{\pgfqpoint{3.694920in}{1.200540in}}%
\pgfpathlineto{\pgfqpoint{3.698092in}{1.200272in}}%
\pgfpathlineto{\pgfqpoint{3.701264in}{1.199845in}}%
\pgfpathlineto{\pgfqpoint{3.704436in}{1.199894in}}%
\pgfpathlineto{\pgfqpoint{3.707608in}{1.199438in}}%
\pgfpathlineto{\pgfqpoint{3.710780in}{1.199125in}}%
\pgfpathlineto{\pgfqpoint{3.713952in}{1.199258in}}%
\pgfpathlineto{\pgfqpoint{3.717124in}{1.199580in}}%
\pgfpathlineto{\pgfqpoint{3.720296in}{1.199445in}}%
\pgfpathlineto{\pgfqpoint{3.723468in}{1.199236in}}%
\pgfpathlineto{\pgfqpoint{3.726640in}{1.199085in}}%
\pgfpathlineto{\pgfqpoint{3.729812in}{1.199051in}}%
\pgfpathlineto{\pgfqpoint{3.732984in}{1.199103in}}%
\pgfpathlineto{\pgfqpoint{3.736156in}{1.198877in}}%
\pgfpathlineto{\pgfqpoint{3.739328in}{1.199122in}}%
\pgfpathlineto{\pgfqpoint{3.742500in}{1.198583in}}%
\pgfpathlineto{\pgfqpoint{3.745672in}{1.198633in}}%
\pgfpathlineto{\pgfqpoint{3.748844in}{1.198351in}}%
\pgfpathlineto{\pgfqpoint{3.752016in}{1.198080in}}%
\pgfpathlineto{\pgfqpoint{3.755188in}{1.198305in}}%
\pgfpathlineto{\pgfqpoint{3.758361in}{1.198142in}}%
\pgfpathlineto{\pgfqpoint{3.761533in}{1.198000in}}%
\pgfpathlineto{\pgfqpoint{3.764705in}{1.197999in}}%
\pgfpathlineto{\pgfqpoint{3.767877in}{1.197462in}}%
\pgfpathlineto{\pgfqpoint{3.771049in}{1.197192in}}%
\pgfpathlineto{\pgfqpoint{3.774221in}{1.196890in}}%
\pgfpathlineto{\pgfqpoint{3.777393in}{1.196809in}}%
\pgfpathlineto{\pgfqpoint{3.780565in}{1.196559in}}%
\pgfpathlineto{\pgfqpoint{3.783737in}{1.196453in}}%
\pgfpathlineto{\pgfqpoint{3.786909in}{1.196049in}}%
\pgfpathlineto{\pgfqpoint{3.790081in}{1.195926in}}%
\pgfpathlineto{\pgfqpoint{3.793253in}{1.195991in}}%
\pgfpathlineto{\pgfqpoint{3.796425in}{1.195938in}}%
\pgfpathlineto{\pgfqpoint{3.799597in}{1.196020in}}%
\pgfpathlineto{\pgfqpoint{3.802769in}{1.195807in}}%
\pgfpathlineto{\pgfqpoint{3.805941in}{1.195414in}}%
\pgfpathlineto{\pgfqpoint{3.809113in}{1.195030in}}%
\pgfpathlineto{\pgfqpoint{3.812285in}{1.195243in}}%
\pgfpathlineto{\pgfqpoint{3.815457in}{1.195592in}}%
\pgfpathlineto{\pgfqpoint{3.818629in}{1.195583in}}%
\pgfpathlineto{\pgfqpoint{3.821801in}{1.195431in}}%
\pgfpathlineto{\pgfqpoint{3.824973in}{1.195177in}}%
\pgfpathlineto{\pgfqpoint{3.828145in}{1.195268in}}%
\pgfpathlineto{\pgfqpoint{3.831317in}{1.195185in}}%
\pgfpathlineto{\pgfqpoint{3.834490in}{1.195335in}}%
\pgfpathlineto{\pgfqpoint{3.837662in}{1.195013in}}%
\pgfpathlineto{\pgfqpoint{3.840834in}{1.194790in}}%
\pgfpathlineto{\pgfqpoint{3.844006in}{1.194448in}}%
\pgfpathlineto{\pgfqpoint{3.847178in}{1.194389in}}%
\pgfpathlineto{\pgfqpoint{3.850350in}{1.194027in}}%
\pgfpathlineto{\pgfqpoint{3.853522in}{1.193788in}}%
\pgfpathlineto{\pgfqpoint{3.856694in}{1.193720in}}%
\pgfpathlineto{\pgfqpoint{3.859866in}{1.193639in}}%
\pgfpathlineto{\pgfqpoint{3.863038in}{1.193309in}}%
\pgfpathlineto{\pgfqpoint{3.866210in}{1.192795in}}%
\pgfpathlineto{\pgfqpoint{3.869382in}{1.192642in}}%
\pgfpathlineto{\pgfqpoint{3.872554in}{1.192574in}}%
\pgfpathlineto{\pgfqpoint{3.875726in}{1.192497in}}%
\pgfpathlineto{\pgfqpoint{3.878898in}{1.192510in}}%
\pgfpathlineto{\pgfqpoint{3.882070in}{1.192338in}}%
\pgfpathlineto{\pgfqpoint{3.885242in}{1.191786in}}%
\pgfpathlineto{\pgfqpoint{3.888414in}{1.191805in}}%
\pgfpathlineto{\pgfqpoint{3.891586in}{1.191970in}}%
\pgfpathlineto{\pgfqpoint{3.894758in}{1.191875in}}%
\pgfpathlineto{\pgfqpoint{3.897930in}{1.191815in}}%
\pgfpathlineto{\pgfqpoint{3.901102in}{1.192037in}}%
\pgfpathlineto{\pgfqpoint{3.904274in}{1.192427in}}%
\pgfpathlineto{\pgfqpoint{3.907446in}{1.192335in}}%
\pgfpathlineto{\pgfqpoint{3.910618in}{1.192576in}}%
\pgfpathlineto{\pgfqpoint{3.913791in}{1.192314in}}%
\pgfpathlineto{\pgfqpoint{3.916963in}{1.192103in}}%
\pgfpathlineto{\pgfqpoint{3.920135in}{1.191849in}}%
\pgfpathlineto{\pgfqpoint{3.923307in}{1.191867in}}%
\pgfpathlineto{\pgfqpoint{3.926479in}{1.191881in}}%
\pgfpathlineto{\pgfqpoint{3.929651in}{1.191497in}}%
\pgfpathlineto{\pgfqpoint{3.932823in}{1.191248in}}%
\pgfpathlineto{\pgfqpoint{3.935995in}{1.190740in}}%
\pgfpathlineto{\pgfqpoint{3.939167in}{1.190775in}}%
\pgfpathlineto{\pgfqpoint{3.942339in}{1.190658in}}%
\pgfpathlineto{\pgfqpoint{3.945511in}{1.190467in}}%
\pgfpathlineto{\pgfqpoint{3.948683in}{1.190484in}}%
\pgfpathlineto{\pgfqpoint{3.951855in}{1.190133in}}%
\pgfpathlineto{\pgfqpoint{3.955027in}{1.189244in}}%
\pgfpathlineto{\pgfqpoint{3.958199in}{1.189227in}}%
\pgfpathlineto{\pgfqpoint{3.961371in}{1.188825in}}%
\pgfpathlineto{\pgfqpoint{3.964543in}{1.188848in}}%
\pgfpathlineto{\pgfqpoint{3.967715in}{1.188728in}}%
\pgfpathlineto{\pgfqpoint{3.970887in}{1.188678in}}%
\pgfpathlineto{\pgfqpoint{3.974059in}{1.188767in}}%
\pgfpathlineto{\pgfqpoint{3.977231in}{1.188559in}}%
\pgfpathlineto{\pgfqpoint{3.980403in}{1.188798in}}%
\pgfpathlineto{\pgfqpoint{3.983575in}{1.188497in}}%
\pgfpathlineto{\pgfqpoint{3.986747in}{1.188339in}}%
\pgfpathlineto{\pgfqpoint{3.989919in}{1.188281in}}%
\pgfpathlineto{\pgfqpoint{3.993092in}{1.188326in}}%
\pgfpathlineto{\pgfqpoint{3.996264in}{1.188462in}}%
\pgfpathlineto{\pgfqpoint{3.999436in}{1.188322in}}%
\pgfpathlineto{\pgfqpoint{4.002608in}{1.188492in}}%
\pgfpathlineto{\pgfqpoint{4.005780in}{1.188845in}}%
\pgfpathlineto{\pgfqpoint{4.008952in}{1.188890in}}%
\pgfpathlineto{\pgfqpoint{4.012124in}{1.188789in}}%
\pgfpathlineto{\pgfqpoint{4.015296in}{1.188514in}}%
\pgfpathlineto{\pgfqpoint{4.018468in}{1.188648in}}%
\pgfpathlineto{\pgfqpoint{4.021640in}{1.188371in}}%
\pgfpathlineto{\pgfqpoint{4.024812in}{1.188357in}}%
\pgfpathlineto{\pgfqpoint{4.027984in}{1.188402in}}%
\pgfpathlineto{\pgfqpoint{4.031156in}{1.188297in}}%
\pgfpathlineto{\pgfqpoint{4.034328in}{1.188183in}}%
\pgfpathlineto{\pgfqpoint{4.037500in}{1.188009in}}%
\pgfpathlineto{\pgfqpoint{4.040672in}{1.187902in}}%
\pgfpathlineto{\pgfqpoint{4.043844in}{1.187851in}}%
\pgfpathlineto{\pgfqpoint{4.047016in}{1.187673in}}%
\pgfpathlineto{\pgfqpoint{4.050188in}{1.187738in}}%
\pgfpathlineto{\pgfqpoint{4.053360in}{1.187391in}}%
\pgfpathlineto{\pgfqpoint{4.056532in}{1.187169in}}%
\pgfpathlineto{\pgfqpoint{4.059704in}{1.187306in}}%
\pgfpathlineto{\pgfqpoint{4.062876in}{1.187309in}}%
\pgfpathlineto{\pgfqpoint{4.066048in}{1.187208in}}%
\pgfpathlineto{\pgfqpoint{4.069221in}{1.186998in}}%
\pgfpathlineto{\pgfqpoint{4.072393in}{1.186914in}}%
\pgfpathlineto{\pgfqpoint{4.075565in}{1.186556in}}%
\pgfpathlineto{\pgfqpoint{4.078737in}{1.186060in}}%
\pgfpathlineto{\pgfqpoint{4.081909in}{1.185855in}}%
\pgfpathlineto{\pgfqpoint{4.085081in}{1.185476in}}%
\pgfpathlineto{\pgfqpoint{4.088253in}{1.185437in}}%
\pgfpathlineto{\pgfqpoint{4.091425in}{1.185054in}}%
\pgfpathlineto{\pgfqpoint{4.094597in}{1.185220in}}%
\pgfpathlineto{\pgfqpoint{4.097769in}{1.185203in}}%
\pgfpathlineto{\pgfqpoint{4.100941in}{1.185240in}}%
\pgfpathlineto{\pgfqpoint{4.104113in}{1.185118in}}%
\pgfpathlineto{\pgfqpoint{4.107285in}{1.184947in}}%
\pgfpathlineto{\pgfqpoint{4.110457in}{1.184367in}}%
\pgfpathlineto{\pgfqpoint{4.113629in}{1.184404in}}%
\pgfpathlineto{\pgfqpoint{4.116801in}{1.184389in}}%
\pgfpathlineto{\pgfqpoint{4.119973in}{1.184623in}}%
\pgfpathlineto{\pgfqpoint{4.123145in}{1.184575in}}%
\pgfpathlineto{\pgfqpoint{4.126317in}{1.184489in}}%
\pgfpathlineto{\pgfqpoint{4.129489in}{1.184322in}}%
\pgfpathlineto{\pgfqpoint{4.132661in}{1.184271in}}%
\pgfpathlineto{\pgfqpoint{4.135833in}{1.184195in}}%
\pgfpathlineto{\pgfqpoint{4.139005in}{1.184028in}}%
\pgfpathlineto{\pgfqpoint{4.142177in}{1.184024in}}%
\pgfpathlineto{\pgfqpoint{4.145349in}{1.184194in}}%
\pgfpathlineto{\pgfqpoint{4.148522in}{1.184472in}}%
\pgfpathlineto{\pgfqpoint{4.151694in}{1.184350in}}%
\pgfpathlineto{\pgfqpoint{4.154866in}{1.184303in}}%
\pgfpathlineto{\pgfqpoint{4.158038in}{1.184315in}}%
\pgfpathlineto{\pgfqpoint{4.161210in}{1.184790in}}%
\pgfpathlineto{\pgfqpoint{4.164382in}{1.185088in}}%
\pgfpathlineto{\pgfqpoint{4.167554in}{1.185196in}}%
\pgfpathlineto{\pgfqpoint{4.170726in}{1.185203in}}%
\pgfpathlineto{\pgfqpoint{4.173898in}{1.184889in}}%
\pgfpathlineto{\pgfqpoint{4.177070in}{1.184848in}}%
\pgfpathlineto{\pgfqpoint{4.180242in}{1.184731in}}%
\pgfpathlineto{\pgfqpoint{4.183414in}{1.184653in}}%
\pgfpathlineto{\pgfqpoint{4.186586in}{1.184361in}}%
\pgfpathlineto{\pgfqpoint{4.189758in}{1.184412in}}%
\pgfpathlineto{\pgfqpoint{4.192930in}{1.184589in}}%
\pgfpathlineto{\pgfqpoint{4.196102in}{1.184661in}}%
\pgfpathlineto{\pgfqpoint{4.199274in}{1.184648in}}%
\pgfpathlineto{\pgfqpoint{4.202446in}{1.184615in}}%
\pgfpathlineto{\pgfqpoint{4.205618in}{1.185056in}}%
\pgfpathlineto{\pgfqpoint{4.208790in}{1.184997in}}%
\pgfpathlineto{\pgfqpoint{4.211962in}{1.184360in}}%
\pgfpathlineto{\pgfqpoint{4.215134in}{1.183888in}}%
\pgfpathlineto{\pgfqpoint{4.218306in}{1.183715in}}%
\pgfpathlineto{\pgfqpoint{4.221478in}{1.183736in}}%
\pgfpathlineto{\pgfqpoint{4.224650in}{1.183966in}}%
\pgfpathlineto{\pgfqpoint{4.227823in}{1.183821in}}%
\pgfpathlineto{\pgfqpoint{4.230995in}{1.184354in}}%
\pgfpathlineto{\pgfqpoint{4.234167in}{1.184558in}}%
\pgfpathlineto{\pgfqpoint{4.237339in}{1.184469in}}%
\pgfpathlineto{\pgfqpoint{4.240511in}{1.184265in}}%
\pgfpathlineto{\pgfqpoint{4.243683in}{1.184066in}}%
\pgfpathlineto{\pgfqpoint{4.246855in}{1.184081in}}%
\pgfpathlineto{\pgfqpoint{4.250027in}{1.184261in}}%
\pgfpathlineto{\pgfqpoint{4.253199in}{1.184293in}}%
\pgfpathlineto{\pgfqpoint{4.256371in}{1.184442in}}%
\pgfpathlineto{\pgfqpoint{4.259543in}{1.184008in}}%
\pgfpathlineto{\pgfqpoint{4.262715in}{1.183799in}}%
\pgfpathlineto{\pgfqpoint{4.265887in}{1.183779in}}%
\pgfpathlineto{\pgfqpoint{4.269059in}{1.184100in}}%
\pgfpathlineto{\pgfqpoint{4.272231in}{1.184303in}}%
\pgfpathlineto{\pgfqpoint{4.275403in}{1.183880in}}%
\pgfpathlineto{\pgfqpoint{4.278575in}{1.183766in}}%
\pgfpathlineto{\pgfqpoint{4.281747in}{1.183602in}}%
\pgfpathlineto{\pgfqpoint{4.284919in}{1.183657in}}%
\pgfpathlineto{\pgfqpoint{4.288091in}{1.183609in}}%
\pgfpathlineto{\pgfqpoint{4.291263in}{1.183500in}}%
\pgfpathlineto{\pgfqpoint{4.294435in}{1.183459in}}%
\pgfpathlineto{\pgfqpoint{4.297607in}{1.183445in}}%
\pgfpathlineto{\pgfqpoint{4.300779in}{1.183470in}}%
\pgfpathlineto{\pgfqpoint{4.303952in}{1.183238in}}%
\pgfpathlineto{\pgfqpoint{4.307124in}{1.183031in}}%
\pgfpathlineto{\pgfqpoint{4.310296in}{1.183178in}}%
\pgfpathlineto{\pgfqpoint{4.313468in}{1.183171in}}%
\pgfpathlineto{\pgfqpoint{4.316640in}{1.182961in}}%
\pgfpathlineto{\pgfqpoint{4.319812in}{1.183085in}}%
\pgfpathlineto{\pgfqpoint{4.322984in}{1.183020in}}%
\pgfpathlineto{\pgfqpoint{4.326156in}{1.182739in}}%
\pgfpathlineto{\pgfqpoint{4.329328in}{1.182459in}}%
\pgfpathlineto{\pgfqpoint{4.332500in}{1.182700in}}%
\pgfpathlineto{\pgfqpoint{4.335672in}{1.182583in}}%
\pgfpathlineto{\pgfqpoint{4.338844in}{1.182673in}}%
\pgfpathlineto{\pgfqpoint{4.342016in}{1.182545in}}%
\pgfpathlineto{\pgfqpoint{4.345188in}{1.181835in}}%
\pgfpathlineto{\pgfqpoint{4.348360in}{1.182067in}}%
\pgfpathlineto{\pgfqpoint{4.351532in}{1.181813in}}%
\pgfpathlineto{\pgfqpoint{4.354704in}{1.181773in}}%
\pgfpathlineto{\pgfqpoint{4.357876in}{1.181501in}}%
\pgfpathlineto{\pgfqpoint{4.361048in}{1.181622in}}%
\pgfpathlineto{\pgfqpoint{4.364220in}{1.181371in}}%
\pgfpathlineto{\pgfqpoint{4.367392in}{1.181135in}}%
\pgfpathlineto{\pgfqpoint{4.370564in}{1.180454in}}%
\pgfpathlineto{\pgfqpoint{4.373736in}{1.180528in}}%
\pgfpathlineto{\pgfqpoint{4.376908in}{1.180419in}}%
\pgfpathlineto{\pgfqpoint{4.380080in}{1.180532in}}%
\pgfpathlineto{\pgfqpoint{4.383253in}{1.180757in}}%
\pgfpathlineto{\pgfqpoint{4.386425in}{1.180956in}}%
\pgfpathlineto{\pgfqpoint{4.389597in}{1.180803in}}%
\pgfpathlineto{\pgfqpoint{4.392769in}{1.180812in}}%
\pgfpathlineto{\pgfqpoint{4.395941in}{1.180849in}}%
\pgfpathlineto{\pgfqpoint{4.399113in}{1.181023in}}%
\pgfpathlineto{\pgfqpoint{4.402285in}{1.180935in}}%
\pgfpathlineto{\pgfqpoint{4.405457in}{1.180926in}}%
\pgfpathlineto{\pgfqpoint{4.408629in}{1.180682in}}%
\pgfpathlineto{\pgfqpoint{4.411801in}{1.180618in}}%
\pgfpathlineto{\pgfqpoint{4.414973in}{1.180508in}}%
\pgfpathlineto{\pgfqpoint{4.418145in}{1.180368in}}%
\pgfpathlineto{\pgfqpoint{4.421317in}{1.179997in}}%
\pgfpathlineto{\pgfqpoint{4.424489in}{1.179486in}}%
\pgfpathlineto{\pgfqpoint{4.427661in}{1.179271in}}%
\pgfpathlineto{\pgfqpoint{4.430833in}{1.178329in}}%
\pgfpathlineto{\pgfqpoint{4.434005in}{1.178979in}}%
\pgfpathlineto{\pgfqpoint{4.437177in}{1.178844in}}%
\pgfpathlineto{\pgfqpoint{4.440349in}{1.178963in}}%
\pgfpathlineto{\pgfqpoint{4.443521in}{1.178587in}}%
\pgfpathlineto{\pgfqpoint{4.446693in}{1.178331in}}%
\pgfpathlineto{\pgfqpoint{4.449865in}{1.177760in}}%
\pgfpathlineto{\pgfqpoint{4.453037in}{1.177756in}}%
\pgfpathlineto{\pgfqpoint{4.456209in}{1.177582in}}%
\pgfpathlineto{\pgfqpoint{4.459381in}{1.177437in}}%
\pgfpathlineto{\pgfqpoint{4.462554in}{1.177294in}}%
\pgfpathlineto{\pgfqpoint{4.465726in}{1.177356in}}%
\pgfpathlineto{\pgfqpoint{4.468898in}{1.177459in}}%
\pgfpathlineto{\pgfqpoint{4.472070in}{1.177423in}}%
\pgfpathlineto{\pgfqpoint{4.475242in}{1.177217in}}%
\pgfpathlineto{\pgfqpoint{4.478414in}{1.176933in}}%
\pgfpathlineto{\pgfqpoint{4.481586in}{1.176432in}}%
\pgfpathlineto{\pgfqpoint{4.484758in}{1.176422in}}%
\pgfpathlineto{\pgfqpoint{4.487930in}{1.176317in}}%
\pgfpathlineto{\pgfqpoint{4.491102in}{1.176170in}}%
\pgfpathlineto{\pgfqpoint{4.494274in}{1.175748in}}%
\pgfpathlineto{\pgfqpoint{4.497446in}{1.175380in}}%
\pgfpathlineto{\pgfqpoint{4.500618in}{1.175238in}}%
\pgfpathlineto{\pgfqpoint{4.503790in}{1.175185in}}%
\pgfpathlineto{\pgfqpoint{4.506962in}{1.175199in}}%
\pgfpathlineto{\pgfqpoint{4.510134in}{1.175038in}}%
\pgfpathlineto{\pgfqpoint{4.513306in}{1.175026in}}%
\pgfpathlineto{\pgfqpoint{4.516478in}{1.174961in}}%
\pgfpathlineto{\pgfqpoint{4.519650in}{1.175041in}}%
\pgfpathlineto{\pgfqpoint{4.522822in}{1.175106in}}%
\pgfpathlineto{\pgfqpoint{4.525994in}{1.175086in}}%
\pgfpathlineto{\pgfqpoint{4.529166in}{1.174879in}}%
\pgfpathlineto{\pgfqpoint{4.532338in}{1.175066in}}%
\pgfpathlineto{\pgfqpoint{4.535510in}{1.175056in}}%
\pgfpathlineto{\pgfqpoint{4.538683in}{1.174773in}}%
\pgfpathlineto{\pgfqpoint{4.541855in}{1.174413in}}%
\pgfpathlineto{\pgfqpoint{4.545027in}{1.174627in}}%
\pgfpathlineto{\pgfqpoint{4.548199in}{1.174399in}}%
\pgfpathlineto{\pgfqpoint{4.551371in}{1.174482in}}%
\pgfpathlineto{\pgfqpoint{4.554543in}{1.174384in}}%
\pgfpathlineto{\pgfqpoint{4.557715in}{1.174544in}}%
\pgfpathlineto{\pgfqpoint{4.560887in}{1.174754in}}%
\pgfpathlineto{\pgfqpoint{4.564059in}{1.174528in}}%
\pgfpathlineto{\pgfqpoint{4.567231in}{1.174031in}}%
\pgfpathlineto{\pgfqpoint{4.570403in}{1.173844in}}%
\pgfpathlineto{\pgfqpoint{4.573575in}{1.173667in}}%
\pgfpathlineto{\pgfqpoint{4.576747in}{1.173696in}}%
\pgfpathlineto{\pgfqpoint{4.579919in}{1.173378in}}%
\pgfpathlineto{\pgfqpoint{4.583091in}{1.173238in}}%
\pgfpathlineto{\pgfqpoint{4.586263in}{1.173076in}}%
\pgfpathlineto{\pgfqpoint{4.589435in}{1.173183in}}%
\pgfpathlineto{\pgfqpoint{4.592607in}{1.172954in}}%
\pgfpathlineto{\pgfqpoint{4.595779in}{1.172943in}}%
\pgfpathlineto{\pgfqpoint{4.598951in}{1.173016in}}%
\pgfpathlineto{\pgfqpoint{4.602123in}{1.173001in}}%
\pgfpathlineto{\pgfqpoint{4.605295in}{1.172841in}}%
\pgfpathlineto{\pgfqpoint{4.608467in}{1.173007in}}%
\pgfpathlineto{\pgfqpoint{4.611639in}{1.173262in}}%
\pgfpathlineto{\pgfqpoint{4.614811in}{1.173378in}}%
\pgfpathlineto{\pgfqpoint{4.617984in}{1.173367in}}%
\pgfpathlineto{\pgfqpoint{4.621156in}{1.173345in}}%
\pgfpathlineto{\pgfqpoint{4.624328in}{1.173274in}}%
\pgfpathlineto{\pgfqpoint{4.627500in}{1.172974in}}%
\pgfpathlineto{\pgfqpoint{4.630672in}{1.172652in}}%
\pgfpathlineto{\pgfqpoint{4.633844in}{1.172260in}}%
\pgfpathlineto{\pgfqpoint{4.637016in}{1.172135in}}%
\pgfpathlineto{\pgfqpoint{4.640188in}{1.171985in}}%
\pgfpathlineto{\pgfqpoint{4.643360in}{1.171695in}}%
\pgfpathlineto{\pgfqpoint{4.646532in}{1.171795in}}%
\pgfpathlineto{\pgfqpoint{4.649704in}{1.171535in}}%
\pgfpathlineto{\pgfqpoint{4.652876in}{1.171182in}}%
\pgfpathlineto{\pgfqpoint{4.656048in}{1.170720in}}%
\pgfpathlineto{\pgfqpoint{4.659220in}{1.171113in}}%
\pgfpathlineto{\pgfqpoint{4.662392in}{1.171014in}}%
\pgfpathlineto{\pgfqpoint{4.665564in}{1.170503in}}%
\pgfpathlineto{\pgfqpoint{4.668736in}{1.170139in}}%
\pgfpathlineto{\pgfqpoint{4.671908in}{1.170128in}}%
\pgfpathlineto{\pgfqpoint{4.675080in}{1.170080in}}%
\pgfpathlineto{\pgfqpoint{4.678252in}{1.169484in}}%
\pgfpathlineto{\pgfqpoint{4.681424in}{1.169654in}}%
\pgfpathlineto{\pgfqpoint{4.684596in}{1.169201in}}%
\pgfpathlineto{\pgfqpoint{4.687768in}{1.169457in}}%
\pgfpathlineto{\pgfqpoint{4.690940in}{1.169376in}}%
\pgfpathlineto{\pgfqpoint{4.694112in}{1.168650in}}%
\pgfpathlineto{\pgfqpoint{4.697285in}{1.167946in}}%
\pgfpathlineto{\pgfqpoint{4.700457in}{1.167790in}}%
\pgfpathlineto{\pgfqpoint{4.703629in}{1.167563in}}%
\pgfpathlineto{\pgfqpoint{4.706801in}{1.167018in}}%
\pgfpathlineto{\pgfqpoint{4.709973in}{1.167007in}}%
\pgfpathlineto{\pgfqpoint{4.713145in}{1.167000in}}%
\pgfpathlineto{\pgfqpoint{4.716317in}{1.166832in}}%
\pgfpathlineto{\pgfqpoint{4.719489in}{1.166266in}}%
\pgfpathlineto{\pgfqpoint{4.722661in}{1.166395in}}%
\pgfpathlineto{\pgfqpoint{4.725833in}{1.166183in}}%
\pgfpathlineto{\pgfqpoint{4.729005in}{1.165597in}}%
\pgfpathlineto{\pgfqpoint{4.732177in}{1.165440in}}%
\pgfpathlineto{\pgfqpoint{4.735349in}{1.165519in}}%
\pgfpathlineto{\pgfqpoint{4.738521in}{1.165110in}}%
\pgfpathlineto{\pgfqpoint{4.741693in}{1.164643in}}%
\pgfpathlineto{\pgfqpoint{4.744865in}{1.164775in}}%
\pgfpathlineto{\pgfqpoint{4.748037in}{1.164843in}}%
\pgfpathlineto{\pgfqpoint{4.751209in}{1.164771in}}%
\pgfpathlineto{\pgfqpoint{4.754381in}{1.164648in}}%
\pgfpathlineto{\pgfqpoint{4.757553in}{1.164638in}}%
\pgfpathlineto{\pgfqpoint{4.760725in}{1.164677in}}%
\pgfpathlineto{\pgfqpoint{4.763897in}{1.165006in}}%
\pgfpathlineto{\pgfqpoint{4.767069in}{1.164796in}}%
\pgfpathlineto{\pgfqpoint{4.770241in}{1.164340in}}%
\pgfpathlineto{\pgfqpoint{4.773414in}{1.164110in}}%
\pgfpathlineto{\pgfqpoint{4.776586in}{1.163792in}}%
\pgfpathlineto{\pgfqpoint{4.779758in}{1.163763in}}%
\pgfpathlineto{\pgfqpoint{4.782930in}{1.163544in}}%
\pgfpathlineto{\pgfqpoint{4.786102in}{1.163647in}}%
\pgfpathlineto{\pgfqpoint{4.789274in}{1.163583in}}%
\pgfpathlineto{\pgfqpoint{4.792446in}{1.163049in}}%
\pgfpathlineto{\pgfqpoint{4.795618in}{1.162330in}}%
\pgfpathlineto{\pgfqpoint{4.798790in}{1.162294in}}%
\pgfpathlineto{\pgfqpoint{4.801962in}{1.162211in}}%
\pgfpathlineto{\pgfqpoint{4.805134in}{1.162224in}}%
\pgfpathlineto{\pgfqpoint{4.808306in}{1.162332in}}%
\pgfpathlineto{\pgfqpoint{4.811478in}{1.162331in}}%
\pgfpathlineto{\pgfqpoint{4.814650in}{1.162442in}}%
\pgfpathlineto{\pgfqpoint{4.817822in}{1.162303in}}%
\pgfpathlineto{\pgfqpoint{4.820994in}{1.161880in}}%
\pgfpathlineto{\pgfqpoint{4.824166in}{1.161473in}}%
\pgfpathlineto{\pgfqpoint{4.827338in}{1.161238in}}%
\pgfpathlineto{\pgfqpoint{4.830510in}{1.161190in}}%
\pgfpathlineto{\pgfqpoint{4.833682in}{1.160854in}}%
\pgfpathlineto{\pgfqpoint{4.836854in}{1.160858in}}%
\pgfpathlineto{\pgfqpoint{4.840026in}{1.160644in}}%
\pgfpathlineto{\pgfqpoint{4.843198in}{1.160433in}}%
\pgfpathlineto{\pgfqpoint{4.846370in}{1.160352in}}%
\pgfpathlineto{\pgfqpoint{4.849542in}{1.160380in}}%
\pgfpathlineto{\pgfqpoint{4.852715in}{1.160220in}}%
\pgfpathlineto{\pgfqpoint{4.855887in}{1.160015in}}%
\pgfpathlineto{\pgfqpoint{4.859059in}{1.159995in}}%
\pgfpathlineto{\pgfqpoint{4.862231in}{1.160148in}}%
\pgfpathlineto{\pgfqpoint{4.865403in}{1.160025in}}%
\pgfpathlineto{\pgfqpoint{4.868575in}{1.159822in}}%
\pgfpathlineto{\pgfqpoint{4.871747in}{1.159825in}}%
\pgfpathlineto{\pgfqpoint{4.874919in}{1.159582in}}%
\pgfpathlineto{\pgfqpoint{4.878091in}{1.159796in}}%
\pgfpathlineto{\pgfqpoint{4.881263in}{1.159660in}}%
\pgfpathlineto{\pgfqpoint{4.884435in}{1.159236in}}%
\pgfpathlineto{\pgfqpoint{4.887607in}{1.159195in}}%
\pgfpathlineto{\pgfqpoint{4.890779in}{1.159274in}}%
\pgfpathlineto{\pgfqpoint{4.893951in}{1.158646in}}%
\pgfpathlineto{\pgfqpoint{4.897123in}{1.158331in}}%
\pgfpathlineto{\pgfqpoint{4.900295in}{1.158167in}}%
\pgfpathlineto{\pgfqpoint{4.903467in}{1.157972in}}%
\pgfpathlineto{\pgfqpoint{4.906639in}{1.157854in}}%
\pgfpathlineto{\pgfqpoint{4.909811in}{1.157644in}}%
\pgfpathlineto{\pgfqpoint{4.912983in}{1.157620in}}%
\pgfpathlineto{\pgfqpoint{4.916155in}{1.157831in}}%
\pgfpathlineto{\pgfqpoint{4.919327in}{1.157779in}}%
\pgfpathlineto{\pgfqpoint{4.922499in}{1.157609in}}%
\pgfpathlineto{\pgfqpoint{4.925671in}{1.157387in}}%
\pgfpathlineto{\pgfqpoint{4.928844in}{1.157524in}}%
\pgfpathlineto{\pgfqpoint{4.932016in}{1.157200in}}%
\pgfpathlineto{\pgfqpoint{4.935188in}{1.157090in}}%
\pgfpathlineto{\pgfqpoint{4.938360in}{1.157188in}}%
\pgfpathlineto{\pgfqpoint{4.941532in}{1.157183in}}%
\pgfpathlineto{\pgfqpoint{4.944704in}{1.157311in}}%
\pgfpathlineto{\pgfqpoint{4.947876in}{1.157087in}}%
\pgfpathlineto{\pgfqpoint{4.951048in}{1.156928in}}%
\pgfpathlineto{\pgfqpoint{4.954220in}{1.156901in}}%
\pgfpathlineto{\pgfqpoint{4.957392in}{1.156518in}}%
\pgfpathlineto{\pgfqpoint{4.960564in}{1.156572in}}%
\pgfpathlineto{\pgfqpoint{4.963736in}{1.156316in}}%
\pgfpathlineto{\pgfqpoint{4.966908in}{1.156409in}}%
\pgfpathlineto{\pgfqpoint{4.970080in}{1.156217in}}%
\pgfpathlineto{\pgfqpoint{4.973252in}{1.155635in}}%
\pgfpathlineto{\pgfqpoint{4.976424in}{1.155433in}}%
\pgfpathlineto{\pgfqpoint{4.979596in}{1.155855in}}%
\pgfpathlineto{\pgfqpoint{4.982768in}{1.155881in}}%
\pgfpathlineto{\pgfqpoint{4.985940in}{1.155586in}}%
\pgfpathlineto{\pgfqpoint{4.989112in}{1.155556in}}%
\pgfpathlineto{\pgfqpoint{4.992284in}{1.155320in}}%
\pgfpathlineto{\pgfqpoint{4.995456in}{1.155605in}}%
\pgfpathlineto{\pgfqpoint{4.998628in}{1.155839in}}%
\pgfpathlineto{\pgfqpoint{5.001800in}{1.155722in}}%
\pgfpathlineto{\pgfqpoint{5.004972in}{1.155388in}}%
\pgfpathlineto{\pgfqpoint{5.008145in}{1.154658in}}%
\pgfpathlineto{\pgfqpoint{5.011317in}{1.154394in}}%
\pgfpathlineto{\pgfqpoint{5.014489in}{1.154088in}}%
\pgfpathlineto{\pgfqpoint{5.017661in}{1.153573in}}%
\pgfpathlineto{\pgfqpoint{5.020833in}{1.153055in}}%
\pgfpathlineto{\pgfqpoint{5.024005in}{1.152657in}}%
\pgfpathlineto{\pgfqpoint{5.027177in}{1.152274in}}%
\pgfpathlineto{\pgfqpoint{5.030349in}{1.152495in}}%
\pgfpathlineto{\pgfqpoint{5.033521in}{1.152178in}}%
\pgfpathlineto{\pgfqpoint{5.036693in}{1.151911in}}%
\pgfpathlineto{\pgfqpoint{5.039865in}{1.151900in}}%
\pgfpathlineto{\pgfqpoint{5.043037in}{1.151441in}}%
\pgfpathlineto{\pgfqpoint{5.046209in}{1.151748in}}%
\pgfpathlineto{\pgfqpoint{5.049381in}{1.151658in}}%
\pgfpathlineto{\pgfqpoint{5.052553in}{1.151546in}}%
\pgfpathlineto{\pgfqpoint{5.055725in}{1.151777in}}%
\pgfpathlineto{\pgfqpoint{5.058897in}{1.151868in}}%
\pgfpathlineto{\pgfqpoint{5.062069in}{1.151954in}}%
\pgfpathlineto{\pgfqpoint{5.065241in}{1.152056in}}%
\pgfpathlineto{\pgfqpoint{5.068413in}{1.152388in}}%
\pgfpathlineto{\pgfqpoint{5.071585in}{1.152321in}}%
\pgfpathlineto{\pgfqpoint{5.074757in}{1.152195in}}%
\pgfpathlineto{\pgfqpoint{5.077929in}{1.152309in}}%
\pgfpathlineto{\pgfqpoint{5.081101in}{1.152383in}}%
\pgfpathlineto{\pgfqpoint{5.084273in}{1.151947in}}%
\pgfpathlineto{\pgfqpoint{5.087446in}{1.152227in}}%
\pgfpathlineto{\pgfqpoint{5.090618in}{1.151771in}}%
\pgfpathlineto{\pgfqpoint{5.093790in}{1.151781in}}%
\pgfpathlineto{\pgfqpoint{5.096962in}{1.152118in}}%
\pgfpathlineto{\pgfqpoint{5.100134in}{1.152040in}}%
\pgfpathlineto{\pgfqpoint{5.103306in}{1.152215in}}%
\pgfpathlineto{\pgfqpoint{5.106478in}{1.152629in}}%
\pgfpathlineto{\pgfqpoint{5.109650in}{1.152675in}}%
\pgfpathlineto{\pgfqpoint{5.112822in}{1.152636in}}%
\pgfpathlineto{\pgfqpoint{5.115994in}{1.152554in}}%
\pgfpathlineto{\pgfqpoint{5.119166in}{1.152374in}}%
\pgfpathlineto{\pgfqpoint{5.122338in}{1.152343in}}%
\pgfpathlineto{\pgfqpoint{5.125510in}{1.152111in}}%
\pgfpathlineto{\pgfqpoint{5.128682in}{1.152343in}}%
\pgfpathlineto{\pgfqpoint{5.131854in}{1.152708in}}%
\pgfpathlineto{\pgfqpoint{5.135026in}{1.152432in}}%
\pgfpathlineto{\pgfqpoint{5.138198in}{1.152642in}}%
\pgfpathlineto{\pgfqpoint{5.141370in}{1.152598in}}%
\pgfpathlineto{\pgfqpoint{5.144542in}{1.152561in}}%
\pgfpathlineto{\pgfqpoint{5.147714in}{1.152586in}}%
\pgfpathlineto{\pgfqpoint{5.150886in}{1.152634in}}%
\pgfpathlineto{\pgfqpoint{5.154058in}{1.152581in}}%
\pgfpathlineto{\pgfqpoint{5.157230in}{1.152504in}}%
\pgfpathlineto{\pgfqpoint{5.160402in}{1.152422in}}%
\pgfpathlineto{\pgfqpoint{5.163575in}{1.152196in}}%
\pgfpathlineto{\pgfqpoint{5.166747in}{1.151977in}}%
\pgfpathlineto{\pgfqpoint{5.169919in}{1.151487in}}%
\pgfpathlineto{\pgfqpoint{5.173091in}{1.151320in}}%
\pgfpathlineto{\pgfqpoint{5.176263in}{1.151496in}}%
\pgfpathlineto{\pgfqpoint{5.179435in}{1.151429in}}%
\pgfpathlineto{\pgfqpoint{5.182607in}{1.151628in}}%
\pgfpathlineto{\pgfqpoint{5.185779in}{1.151528in}}%
\pgfpathlineto{\pgfqpoint{5.188951in}{1.151391in}}%
\pgfpathlineto{\pgfqpoint{5.192123in}{1.151412in}}%
\pgfpathlineto{\pgfqpoint{5.195295in}{1.151132in}}%
\pgfpathlineto{\pgfqpoint{5.198467in}{1.150651in}}%
\pgfpathlineto{\pgfqpoint{5.201639in}{1.150764in}}%
\pgfpathlineto{\pgfqpoint{5.204811in}{1.150525in}}%
\pgfpathlineto{\pgfqpoint{5.207983in}{1.150251in}}%
\pgfpathlineto{\pgfqpoint{5.211155in}{1.149778in}}%
\pgfpathlineto{\pgfqpoint{5.214327in}{1.149685in}}%
\pgfpathlineto{\pgfqpoint{5.217499in}{1.149567in}}%
\pgfpathlineto{\pgfqpoint{5.220671in}{1.149821in}}%
\pgfpathlineto{\pgfqpoint{5.223843in}{1.149699in}}%
\pgfpathlineto{\pgfqpoint{5.227015in}{1.149525in}}%
\pgfpathlineto{\pgfqpoint{5.230187in}{1.149523in}}%
\pgfpathlineto{\pgfqpoint{5.233359in}{1.149263in}}%
\pgfpathlineto{\pgfqpoint{5.236531in}{1.149348in}}%
\pgfpathlineto{\pgfqpoint{5.239703in}{1.149024in}}%
\pgfpathlineto{\pgfqpoint{5.242876in}{1.148599in}}%
\pgfpathlineto{\pgfqpoint{5.246048in}{1.148279in}}%
\pgfpathlineto{\pgfqpoint{5.249220in}{1.147945in}}%
\pgfpathlineto{\pgfqpoint{5.252392in}{1.148009in}}%
\pgfpathlineto{\pgfqpoint{5.255564in}{1.147774in}}%
\pgfpathlineto{\pgfqpoint{5.258736in}{1.147556in}}%
\pgfpathlineto{\pgfqpoint{5.261908in}{1.147400in}}%
\pgfpathlineto{\pgfqpoint{5.265080in}{1.147600in}}%
\pgfpathlineto{\pgfqpoint{5.268252in}{1.147393in}}%
\pgfpathlineto{\pgfqpoint{5.271424in}{1.146690in}}%
\pgfpathlineto{\pgfqpoint{5.274596in}{1.146315in}}%
\pgfpathlineto{\pgfqpoint{5.277768in}{1.145452in}}%
\pgfpathlineto{\pgfqpoint{5.280940in}{1.145132in}}%
\pgfpathlineto{\pgfqpoint{5.284112in}{1.145109in}}%
\pgfpathlineto{\pgfqpoint{5.287284in}{1.145023in}}%
\pgfpathlineto{\pgfqpoint{5.290456in}{1.145263in}}%
\pgfpathlineto{\pgfqpoint{5.293628in}{1.145030in}}%
\pgfpathlineto{\pgfqpoint{5.296800in}{1.145028in}}%
\pgfpathlineto{\pgfqpoint{5.299972in}{1.145052in}}%
\pgfpathlineto{\pgfqpoint{5.303144in}{1.144682in}}%
\pgfpathlineto{\pgfqpoint{5.306316in}{1.144844in}}%
\pgfpathlineto{\pgfqpoint{5.309488in}{1.145096in}}%
\pgfpathlineto{\pgfqpoint{5.312660in}{1.145124in}}%
\pgfpathlineto{\pgfqpoint{5.315832in}{1.145065in}}%
\pgfpathlineto{\pgfqpoint{5.319004in}{1.144919in}}%
\pgfpathlineto{\pgfqpoint{5.322177in}{1.144754in}}%
\pgfpathlineto{\pgfqpoint{5.325349in}{1.144717in}}%
\pgfpathlineto{\pgfqpoint{5.328521in}{1.144759in}}%
\pgfpathlineto{\pgfqpoint{5.331693in}{1.144031in}}%
\pgfpathlineto{\pgfqpoint{5.334865in}{1.143843in}}%
\pgfpathlineto{\pgfqpoint{5.338037in}{1.143806in}}%
\pgfpathlineto{\pgfqpoint{5.341209in}{1.143830in}}%
\pgfpathlineto{\pgfqpoint{5.344381in}{1.143638in}}%
\pgfpathlineto{\pgfqpoint{5.347553in}{1.143740in}}%
\pgfpathlineto{\pgfqpoint{5.350725in}{1.143713in}}%
\pgfpathlineto{\pgfqpoint{5.353897in}{1.143533in}}%
\pgfpathlineto{\pgfqpoint{5.357069in}{1.143750in}}%
\pgfpathlineto{\pgfqpoint{5.360241in}{1.143000in}}%
\pgfpathlineto{\pgfqpoint{5.363413in}{1.143347in}}%
\pgfpathlineto{\pgfqpoint{5.366585in}{1.143613in}}%
\pgfpathlineto{\pgfqpoint{5.369757in}{1.143863in}}%
\pgfpathlineto{\pgfqpoint{5.372929in}{1.143869in}}%
\pgfpathlineto{\pgfqpoint{5.376101in}{1.144215in}}%
\pgfpathlineto{\pgfqpoint{5.379273in}{1.144025in}}%
\pgfpathlineto{\pgfqpoint{5.382445in}{1.143833in}}%
\pgfpathlineto{\pgfqpoint{5.385617in}{1.143833in}}%
\pgfpathlineto{\pgfqpoint{5.388789in}{1.143387in}}%
\pgfpathlineto{\pgfqpoint{5.391961in}{1.143397in}}%
\pgfpathlineto{\pgfqpoint{5.395133in}{1.143422in}}%
\pgfpathlineto{\pgfqpoint{5.398306in}{1.143524in}}%
\pgfpathlineto{\pgfqpoint{5.401478in}{1.143271in}}%
\pgfpathlineto{\pgfqpoint{5.404650in}{1.143136in}}%
\pgfpathlineto{\pgfqpoint{5.407822in}{1.142668in}}%
\pgfpathlineto{\pgfqpoint{5.410994in}{1.142173in}}%
\pgfpathlineto{\pgfqpoint{5.414166in}{1.142151in}}%
\pgfpathlineto{\pgfqpoint{5.417338in}{1.142153in}}%
\pgfpathlineto{\pgfqpoint{5.420510in}{1.142201in}}%
\pgfpathlineto{\pgfqpoint{5.423682in}{1.142208in}}%
\pgfpathlineto{\pgfqpoint{5.426854in}{1.142215in}}%
\pgfpathlineto{\pgfqpoint{5.430026in}{1.142223in}}%
\pgfpathlineto{\pgfqpoint{5.433198in}{1.142245in}}%
\pgfpathlineto{\pgfqpoint{5.436370in}{1.142242in}}%
\pgfpathlineto{\pgfqpoint{5.439542in}{1.142280in}}%
\pgfpathlineto{\pgfqpoint{5.442714in}{1.142317in}}%
\pgfpathlineto{\pgfqpoint{5.445886in}{1.142317in}}%
\pgfpathlineto{\pgfqpoint{5.449058in}{1.142328in}}%
\pgfpathlineto{\pgfqpoint{5.452230in}{1.142328in}}%
\pgfpathlineto{\pgfqpoint{5.455402in}{1.142334in}}%
\pgfpathlineto{\pgfqpoint{5.458574in}{1.142337in}}%
\pgfpathlineto{\pgfqpoint{5.461746in}{1.142381in}}%
\pgfpathlineto{\pgfqpoint{5.464918in}{1.142423in}}%
\pgfpathlineto{\pgfqpoint{5.468090in}{1.142441in}}%
\pgfpathlineto{\pgfqpoint{5.471262in}{1.142443in}}%
\pgfpathlineto{\pgfqpoint{5.474434in}{1.142471in}}%
\pgfpathlineto{\pgfqpoint{5.477607in}{1.142498in}}%
\pgfpathlineto{\pgfqpoint{5.480779in}{1.142503in}}%
\pgfpathlineto{\pgfqpoint{5.483951in}{1.142504in}}%
\pgfpathlineto{\pgfqpoint{5.487123in}{1.142513in}}%
\pgfpathlineto{\pgfqpoint{5.490295in}{1.142514in}}%
\pgfpathlineto{\pgfqpoint{5.493467in}{1.142511in}}%
\pgfpathlineto{\pgfqpoint{5.496639in}{1.142521in}}%
\pgfpathlineto{\pgfqpoint{5.499811in}{1.142520in}}%
\pgfpathlineto{\pgfqpoint{5.502983in}{1.142535in}}%
\pgfpathlineto{\pgfqpoint{5.506155in}{1.142576in}}%
\pgfpathlineto{\pgfqpoint{5.509327in}{1.142601in}}%
\pgfpathlineto{\pgfqpoint{5.512499in}{1.142594in}}%
\pgfpathlineto{\pgfqpoint{5.515671in}{1.142574in}}%
\pgfpathlineto{\pgfqpoint{5.518843in}{1.142647in}}%
\pgfpathlineto{\pgfqpoint{5.522015in}{1.142624in}}%
\pgfpathlineto{\pgfqpoint{5.525187in}{1.142629in}}%
\pgfpathlineto{\pgfqpoint{5.528359in}{1.142617in}}%
\pgfpathlineto{\pgfqpoint{5.531531in}{1.142631in}}%
\pgfpathlineto{\pgfqpoint{5.534703in}{1.142615in}}%
\pgfpathlineto{\pgfqpoint{5.537875in}{1.142630in}}%
\pgfpathlineto{\pgfqpoint{5.541047in}{1.142616in}}%
\pgfpathlineto{\pgfqpoint{5.544219in}{1.142632in}}%
\pgfpathlineto{\pgfqpoint{5.547391in}{1.142617in}}%
\pgfpathlineto{\pgfqpoint{5.550563in}{1.142633in}}%
\pgfpathlineto{\pgfqpoint{5.553735in}{1.142656in}}%
\pgfpathlineto{\pgfqpoint{5.556908in}{1.142668in}}%
\pgfpathlineto{\pgfqpoint{5.560080in}{1.142662in}}%
\pgfpathlineto{\pgfqpoint{5.563252in}{1.142651in}}%
\pgfpathlineto{\pgfqpoint{5.566424in}{1.142656in}}%
\pgfpathlineto{\pgfqpoint{5.569596in}{1.142670in}}%
\pgfpathlineto{\pgfqpoint{5.572768in}{1.142685in}}%
\pgfpathlineto{\pgfqpoint{5.575940in}{1.142733in}}%
\pgfpathlineto{\pgfqpoint{5.579112in}{1.142762in}}%
\pgfpathlineto{\pgfqpoint{5.582284in}{1.142744in}}%
\pgfpathlineto{\pgfqpoint{5.585456in}{1.142738in}}%
\pgfpathlineto{\pgfqpoint{5.588628in}{1.142751in}}%
\pgfpathlineto{\pgfqpoint{5.591800in}{1.142767in}}%
\pgfpathlineto{\pgfqpoint{5.594972in}{1.142760in}}%
\pgfpathlineto{\pgfqpoint{5.598144in}{1.142758in}}%
\pgfpathlineto{\pgfqpoint{5.601316in}{1.142737in}}%
\pgfpathlineto{\pgfqpoint{5.604488in}{1.142776in}}%
\pgfpathlineto{\pgfqpoint{5.607660in}{1.142829in}}%
\pgfpathlineto{\pgfqpoint{5.610832in}{1.142831in}}%
\pgfpathlineto{\pgfqpoint{5.614004in}{1.142861in}}%
\pgfpathlineto{\pgfqpoint{5.617176in}{1.142859in}}%
\pgfpathlineto{\pgfqpoint{5.620348in}{1.142858in}}%
\pgfpathlineto{\pgfqpoint{5.623520in}{1.142878in}}%
\pgfpathlineto{\pgfqpoint{5.626692in}{1.142892in}}%
\pgfpathlineto{\pgfqpoint{5.629864in}{1.142911in}}%
\pgfpathlineto{\pgfqpoint{5.633037in}{1.142911in}}%
\pgfpathlineto{\pgfqpoint{5.636209in}{1.142931in}}%
\pgfpathlineto{\pgfqpoint{5.639381in}{1.142919in}}%
\pgfpathlineto{\pgfqpoint{5.642553in}{1.142882in}}%
\pgfpathlineto{\pgfqpoint{5.645725in}{1.142921in}}%
\pgfpathlineto{\pgfqpoint{5.648897in}{1.142938in}}%
\pgfpathlineto{\pgfqpoint{5.652069in}{1.142928in}}%
\pgfpathlineto{\pgfqpoint{5.655241in}{1.142948in}}%
\pgfpathlineto{\pgfqpoint{5.658413in}{1.142974in}}%
\pgfpathlineto{\pgfqpoint{5.661585in}{1.142962in}}%
\pgfpathlineto{\pgfqpoint{5.664757in}{1.142950in}}%
\pgfpathlineto{\pgfqpoint{5.667929in}{1.142934in}}%
\pgfpathlineto{\pgfqpoint{5.671101in}{1.142972in}}%
\pgfpathlineto{\pgfqpoint{5.674273in}{1.142975in}}%
\pgfpathlineto{\pgfqpoint{5.677445in}{1.142990in}}%
\pgfpathlineto{\pgfqpoint{5.680617in}{1.143028in}}%
\pgfpathlineto{\pgfqpoint{5.683789in}{1.143042in}}%
\pgfpathlineto{\pgfqpoint{5.686961in}{1.143037in}}%
\pgfpathlineto{\pgfqpoint{5.690133in}{1.143042in}}%
\pgfpathlineto{\pgfqpoint{5.693305in}{1.143067in}}%
\pgfpathlineto{\pgfqpoint{5.696477in}{1.143066in}}%
\pgfpathlineto{\pgfqpoint{5.699649in}{1.143074in}}%
\pgfpathlineto{\pgfqpoint{5.702821in}{1.143084in}}%
\pgfpathlineto{\pgfqpoint{5.705993in}{1.143088in}}%
\pgfpathlineto{\pgfqpoint{5.709165in}{1.143092in}}%
\pgfpathlineto{\pgfqpoint{5.712338in}{1.143089in}}%
\pgfpathlineto{\pgfqpoint{5.715510in}{1.143101in}}%
\pgfpathlineto{\pgfqpoint{5.718682in}{1.143060in}}%
\pgfpathlineto{\pgfqpoint{5.721854in}{1.143076in}}%
\pgfpathlineto{\pgfqpoint{5.725026in}{1.143103in}}%
\pgfpathlineto{\pgfqpoint{5.728198in}{1.143158in}}%
\pgfpathlineto{\pgfqpoint{5.731370in}{1.143174in}}%
\pgfpathlineto{\pgfqpoint{5.734542in}{1.143159in}}%
\pgfpathlineto{\pgfqpoint{5.737714in}{1.143163in}}%
\pgfpathlineto{\pgfqpoint{5.740886in}{1.143193in}}%
\pgfpathlineto{\pgfqpoint{5.744058in}{1.143229in}}%
\pgfpathlineto{\pgfqpoint{5.747230in}{1.143214in}}%
\pgfpathlineto{\pgfqpoint{5.750402in}{1.143233in}}%
\pgfpathlineto{\pgfqpoint{5.753574in}{1.143247in}}%
\pgfpathlineto{\pgfqpoint{5.756746in}{1.143271in}}%
\pgfpathlineto{\pgfqpoint{5.759918in}{1.143264in}}%
\pgfpathlineto{\pgfqpoint{5.763090in}{1.143277in}}%
\pgfpathlineto{\pgfqpoint{5.766262in}{1.143311in}}%
\pgfpathlineto{\pgfqpoint{5.769434in}{1.143309in}}%
\pgfpathlineto{\pgfqpoint{5.772606in}{1.143304in}}%
\pgfpathlineto{\pgfqpoint{5.775778in}{1.143317in}}%
\pgfpathlineto{\pgfqpoint{5.778950in}{1.143353in}}%
\pgfpathlineto{\pgfqpoint{5.782122in}{1.143350in}}%
\pgfpathlineto{\pgfqpoint{5.785294in}{1.143354in}}%
\pgfpathlineto{\pgfqpoint{5.788466in}{1.143335in}}%
\pgfpathlineto{\pgfqpoint{5.791639in}{1.143336in}}%
\pgfpathlineto{\pgfqpoint{5.794811in}{1.143348in}}%
\pgfpathlineto{\pgfqpoint{5.797983in}{1.143344in}}%
\pgfpathlineto{\pgfqpoint{5.801155in}{1.143399in}}%
\pgfpathlineto{\pgfqpoint{5.804327in}{1.143391in}}%
\pgfpathlineto{\pgfqpoint{5.807499in}{1.143406in}}%
\pgfpathlineto{\pgfqpoint{5.810671in}{1.143375in}}%
\pgfpathlineto{\pgfqpoint{5.813843in}{1.143409in}}%
\pgfpathlineto{\pgfqpoint{5.817015in}{1.143429in}}%
\pgfpathlineto{\pgfqpoint{5.820187in}{1.143431in}}%
\pgfpathlineto{\pgfqpoint{5.823359in}{1.143428in}}%
\pgfpathlineto{\pgfqpoint{5.826531in}{1.143435in}}%
\pgfpathlineto{\pgfqpoint{5.829703in}{1.143442in}}%
\pgfpathlineto{\pgfqpoint{5.832875in}{1.143465in}}%
\pgfpathlineto{\pgfqpoint{5.836047in}{1.143445in}}%
\pgfpathlineto{\pgfqpoint{5.839219in}{1.143442in}}%
\pgfpathlineto{\pgfqpoint{5.842391in}{1.143427in}}%
\pgfpathlineto{\pgfqpoint{5.845563in}{1.143417in}}%
\pgfpathlineto{\pgfqpoint{5.848735in}{1.143395in}}%
\pgfpathlineto{\pgfqpoint{5.851907in}{1.143382in}}%
\pgfpathlineto{\pgfqpoint{5.855079in}{1.143377in}}%
\pgfpathlineto{\pgfqpoint{5.858251in}{1.143393in}}%
\pgfpathlineto{\pgfqpoint{5.861423in}{1.143437in}}%
\pgfpathlineto{\pgfqpoint{5.864595in}{1.143459in}}%
\pgfpathlineto{\pgfqpoint{5.867768in}{1.143460in}}%
\pgfpathlineto{\pgfqpoint{5.870940in}{1.143499in}}%
\pgfpathlineto{\pgfqpoint{5.874112in}{1.143533in}}%
\pgfpathlineto{\pgfqpoint{5.877284in}{1.143534in}}%
\pgfpathlineto{\pgfqpoint{5.880456in}{1.143496in}}%
\pgfpathlineto{\pgfqpoint{5.883628in}{1.143497in}}%
\pgfpathlineto{\pgfqpoint{5.886800in}{1.143455in}}%
\pgfpathlineto{\pgfqpoint{5.889972in}{1.143470in}}%
\pgfpathlineto{\pgfqpoint{5.893144in}{1.143479in}}%
\pgfpathlineto{\pgfqpoint{5.896316in}{1.143483in}}%
\pgfpathlineto{\pgfqpoint{5.899488in}{1.143520in}}%
\pgfpathlineto{\pgfqpoint{5.902660in}{1.143526in}}%
\pgfpathlineto{\pgfqpoint{5.905832in}{1.143540in}}%
\pgfpathlineto{\pgfqpoint{5.909004in}{1.143571in}}%
\pgfpathlineto{\pgfqpoint{5.912176in}{1.143582in}}%
\pgfpathlineto{\pgfqpoint{5.915348in}{1.143589in}}%
\pgfpathlineto{\pgfqpoint{5.918520in}{1.143622in}}%
\pgfpathlineto{\pgfqpoint{5.921692in}{1.143636in}}%
\pgfpathlineto{\pgfqpoint{5.924864in}{1.143625in}}%
\pgfpathlineto{\pgfqpoint{5.928036in}{1.143628in}}%
\pgfpathlineto{\pgfqpoint{5.931208in}{1.143631in}}%
\pgfpathlineto{\pgfqpoint{5.934380in}{1.143660in}}%
\pgfpathlineto{\pgfqpoint{5.937552in}{1.143677in}}%
\pgfpathlineto{\pgfqpoint{5.940724in}{1.143669in}}%
\pgfpathlineto{\pgfqpoint{5.943896in}{1.143703in}}%
\pgfpathlineto{\pgfqpoint{5.947069in}{1.143719in}}%
\pgfpathlineto{\pgfqpoint{5.950241in}{1.143720in}}%
\pgfpathlineto{\pgfqpoint{5.953413in}{1.143718in}}%
\pgfpathlineto{\pgfqpoint{5.956585in}{1.143722in}}%
\pgfpathlineto{\pgfqpoint{5.959757in}{1.143714in}}%
\pgfpathlineto{\pgfqpoint{5.962929in}{1.143709in}}%
\pgfpathlineto{\pgfqpoint{5.966101in}{1.143694in}}%
\pgfpathlineto{\pgfqpoint{5.969273in}{1.143719in}}%
\pgfpathlineto{\pgfqpoint{5.972445in}{1.143734in}}%
\pgfpathlineto{\pgfqpoint{5.975617in}{1.143719in}}%
\pgfpathlineto{\pgfqpoint{5.978789in}{1.143734in}}%
\pgfpathlineto{\pgfqpoint{5.981961in}{1.143734in}}%
\pgfpathlineto{\pgfqpoint{5.985133in}{1.143758in}}%
\pgfpathlineto{\pgfqpoint{5.988305in}{1.143730in}}%
\pgfpathlineto{\pgfqpoint{5.991477in}{1.143691in}}%
\pgfpathlineto{\pgfqpoint{5.994649in}{1.143722in}}%
\pgfpathlineto{\pgfqpoint{5.997821in}{1.143649in}}%
\pgfpathlineto{\pgfqpoint{6.000993in}{1.143642in}}%
\pgfpathlineto{\pgfqpoint{6.004165in}{1.143649in}}%
\pgfpathlineto{\pgfqpoint{6.007337in}{1.143692in}}%
\pgfpathlineto{\pgfqpoint{6.010509in}{1.143700in}}%
\pgfpathlineto{\pgfqpoint{6.013681in}{1.143744in}}%
\pgfpathlineto{\pgfqpoint{6.016853in}{1.143767in}}%
\pgfpathlineto{\pgfqpoint{6.020025in}{1.143759in}}%
\pgfpathlineto{\pgfqpoint{6.023197in}{1.143756in}}%
\pgfpathlineto{\pgfqpoint{6.026370in}{1.143779in}}%
\pgfpathlineto{\pgfqpoint{6.029542in}{1.143817in}}%
\pgfpathlineto{\pgfqpoint{6.032714in}{1.143840in}}%
\pgfpathlineto{\pgfqpoint{6.035886in}{1.143853in}}%
\pgfpathlineto{\pgfqpoint{6.039058in}{1.143872in}}%
\pgfpathlineto{\pgfqpoint{6.042230in}{1.143886in}}%
\pgfpathlineto{\pgfqpoint{6.045402in}{1.143886in}}%
\pgfpathlineto{\pgfqpoint{6.048574in}{1.143889in}}%
\pgfpathlineto{\pgfqpoint{6.051746in}{1.143898in}}%
\pgfpathlineto{\pgfqpoint{6.054918in}{1.143917in}}%
\pgfpathlineto{\pgfqpoint{6.058090in}{1.143932in}}%
\pgfpathlineto{\pgfqpoint{6.061262in}{1.143955in}}%
\pgfpathlineto{\pgfqpoint{6.064434in}{1.143934in}}%
\pgfpathlineto{\pgfqpoint{6.067606in}{1.143944in}}%
\pgfpathlineto{\pgfqpoint{6.070778in}{1.143963in}}%
\pgfpathlineto{\pgfqpoint{6.073950in}{1.143990in}}%
\pgfpathlineto{\pgfqpoint{6.077122in}{1.144014in}}%
\pgfpathlineto{\pgfqpoint{6.080294in}{1.144027in}}%
\pgfpathlineto{\pgfqpoint{6.083466in}{1.144050in}}%
\pgfpathlineto{\pgfqpoint{6.086638in}{1.144024in}}%
\pgfpathlineto{\pgfqpoint{6.089810in}{1.144008in}}%
\pgfpathlineto{\pgfqpoint{6.092982in}{1.143971in}}%
\pgfpathlineto{\pgfqpoint{6.096154in}{1.143988in}}%
\pgfpathlineto{\pgfqpoint{6.099326in}{1.143996in}}%
\pgfpathlineto{\pgfqpoint{6.102499in}{1.143975in}}%
\pgfpathlineto{\pgfqpoint{6.105671in}{1.143960in}}%
\pgfpathlineto{\pgfqpoint{6.108843in}{1.143940in}}%
\pgfpathlineto{\pgfqpoint{6.112015in}{1.143949in}}%
\pgfpathlineto{\pgfqpoint{6.115187in}{1.143977in}}%
\pgfpathlineto{\pgfqpoint{6.118359in}{1.144018in}}%
\pgfpathlineto{\pgfqpoint{6.121531in}{1.144011in}}%
\pgfpathlineto{\pgfqpoint{6.124703in}{1.144035in}}%
\pgfpathlineto{\pgfqpoint{6.127875in}{1.144054in}}%
\pgfpathlineto{\pgfqpoint{6.131047in}{1.144053in}}%
\pgfpathlineto{\pgfqpoint{6.134219in}{1.144055in}}%
\pgfpathlineto{\pgfqpoint{6.137391in}{1.144055in}}%
\pgfpathlineto{\pgfqpoint{6.140563in}{1.144058in}}%
\pgfpathlineto{\pgfqpoint{6.143735in}{1.144057in}}%
\pgfpathlineto{\pgfqpoint{6.146907in}{1.144034in}}%
\pgfpathlineto{\pgfqpoint{6.150079in}{1.144020in}}%
\pgfpathlineto{\pgfqpoint{6.153251in}{1.144010in}}%
\pgfpathlineto{\pgfqpoint{6.156423in}{1.144012in}}%
\pgfpathlineto{\pgfqpoint{6.159595in}{1.144012in}}%
\pgfpathlineto{\pgfqpoint{6.162767in}{1.144005in}}%
\pgfpathlineto{\pgfqpoint{6.165939in}{1.144001in}}%
\pgfpathlineto{\pgfqpoint{6.169111in}{1.144009in}}%
\pgfpathlineto{\pgfqpoint{6.172283in}{1.144052in}}%
\pgfpathlineto{\pgfqpoint{6.175455in}{1.144073in}}%
\pgfpathlineto{\pgfqpoint{6.178627in}{1.144046in}}%
\pgfpathlineto{\pgfqpoint{6.181800in}{1.144025in}}%
\pgfpathlineto{\pgfqpoint{6.184972in}{1.144009in}}%
\pgfpathlineto{\pgfqpoint{6.188144in}{1.144027in}}%
\pgfpathlineto{\pgfqpoint{6.191316in}{1.144031in}}%
\pgfpathlineto{\pgfqpoint{6.194488in}{1.144004in}}%
\pgfpathlineto{\pgfqpoint{6.197660in}{1.144048in}}%
\pgfpathlineto{\pgfqpoint{6.200832in}{1.144115in}}%
\pgfpathlineto{\pgfqpoint{6.204004in}{1.144104in}}%
\pgfpathlineto{\pgfqpoint{6.207176in}{1.144103in}}%
\pgfpathlineto{\pgfqpoint{6.210348in}{1.144165in}}%
\pgfpathlineto{\pgfqpoint{6.213520in}{1.144159in}}%
\pgfpathlineto{\pgfqpoint{6.216692in}{1.144164in}}%
\pgfpathlineto{\pgfqpoint{6.219864in}{1.144212in}}%
\pgfpathlineto{\pgfqpoint{6.223036in}{1.144206in}}%
\pgfpathlineto{\pgfqpoint{6.226208in}{1.144206in}}%
\pgfpathlineto{\pgfqpoint{6.229380in}{1.144234in}}%
\pgfpathlineto{\pgfqpoint{6.232552in}{1.144285in}}%
\pgfpathlineto{\pgfqpoint{6.235724in}{1.144312in}}%
\pgfpathlineto{\pgfqpoint{6.238896in}{1.144323in}}%
\pgfpathlineto{\pgfqpoint{6.242068in}{1.144295in}}%
\pgfpathlineto{\pgfqpoint{6.245240in}{1.144303in}}%
\pgfpathlineto{\pgfqpoint{6.248412in}{1.144291in}}%
\pgfpathlineto{\pgfqpoint{6.251584in}{1.144326in}}%
\pgfpathlineto{\pgfqpoint{6.254756in}{1.144343in}}%
\pgfpathlineto{\pgfqpoint{6.257928in}{1.144365in}}%
\pgfpathlineto{\pgfqpoint{6.261101in}{1.144392in}}%
\pgfpathlineto{\pgfqpoint{6.264273in}{1.144393in}}%
\pgfpathlineto{\pgfqpoint{6.267445in}{1.144401in}}%
\pgfpathlineto{\pgfqpoint{6.270617in}{1.144372in}}%
\pgfpathlineto{\pgfqpoint{6.273789in}{1.144351in}}%
\pgfpathlineto{\pgfqpoint{6.276961in}{1.144380in}}%
\pgfpathlineto{\pgfqpoint{6.280133in}{1.144377in}}%
\pgfpathlineto{\pgfqpoint{6.283305in}{1.144375in}}%
\pgfpathlineto{\pgfqpoint{6.286477in}{1.144349in}}%
\pgfpathlineto{\pgfqpoint{6.289649in}{1.144369in}}%
\pgfpathlineto{\pgfqpoint{6.292821in}{1.144367in}}%
\pgfpathlineto{\pgfqpoint{6.295993in}{1.144377in}}%
\pgfpathlineto{\pgfqpoint{6.299165in}{1.144394in}}%
\pgfpathlineto{\pgfqpoint{6.302337in}{1.144378in}}%
\pgfpathlineto{\pgfqpoint{6.305509in}{1.144370in}}%
\pgfpathlineto{\pgfqpoint{6.308681in}{1.144352in}}%
\pgfpathlineto{\pgfqpoint{6.311853in}{1.144381in}}%
\pgfpathlineto{\pgfqpoint{6.315025in}{1.144367in}}%
\pgfpathlineto{\pgfqpoint{6.318197in}{1.144428in}}%
\pgfpathlineto{\pgfqpoint{6.321369in}{1.144429in}}%
\pgfpathlineto{\pgfqpoint{6.324541in}{1.144455in}}%
\pgfpathlineto{\pgfqpoint{6.327713in}{1.144464in}}%
\pgfpathlineto{\pgfqpoint{6.330885in}{1.144498in}}%
\pgfpathlineto{\pgfqpoint{6.334057in}{1.144494in}}%
\pgfpathlineto{\pgfqpoint{6.337230in}{1.144552in}}%
\pgfpathlineto{\pgfqpoint{6.340402in}{1.144565in}}%
\pgfpathlineto{\pgfqpoint{6.343574in}{1.144593in}}%
\pgfpathlineto{\pgfqpoint{6.346746in}{1.144607in}}%
\pgfpathlineto{\pgfqpoint{6.349918in}{1.144603in}}%
\pgfpathlineto{\pgfqpoint{6.353090in}{1.144605in}}%
\pgfpathlineto{\pgfqpoint{6.356262in}{1.144617in}}%
\pgfpathlineto{\pgfqpoint{6.359434in}{1.144607in}}%
\pgfpathlineto{\pgfqpoint{6.362606in}{1.144603in}}%
\pgfpathlineto{\pgfqpoint{6.365778in}{1.144626in}}%
\pgfpathlineto{\pgfqpoint{6.368950in}{1.144643in}}%
\pgfpathlineto{\pgfqpoint{6.372122in}{1.144639in}}%
\pgfpathlineto{\pgfqpoint{6.375294in}{1.144638in}}%
\pgfpathlineto{\pgfqpoint{6.378466in}{1.144659in}}%
\pgfpathlineto{\pgfqpoint{6.381638in}{1.144709in}}%
\pgfpathlineto{\pgfqpoint{6.384810in}{1.144698in}}%
\pgfpathlineto{\pgfqpoint{6.387982in}{1.144704in}}%
\pgfpathlineto{\pgfqpoint{6.391154in}{1.144681in}}%
\pgfpathlineto{\pgfqpoint{6.394326in}{1.144688in}}%
\pgfpathlineto{\pgfqpoint{6.397498in}{1.144722in}}%
\pgfpathlineto{\pgfqpoint{6.400670in}{1.144712in}}%
\pgfpathlineto{\pgfqpoint{6.403842in}{1.144739in}}%
\pgfpathlineto{\pgfqpoint{6.407014in}{1.144784in}}%
\pgfpathlineto{\pgfqpoint{6.410186in}{1.144763in}}%
\pgfpathlineto{\pgfqpoint{6.413358in}{1.144730in}}%
\pgfpathlineto{\pgfqpoint{6.416531in}{1.144716in}}%
\pgfpathlineto{\pgfqpoint{6.419703in}{1.144708in}}%
\pgfpathlineto{\pgfqpoint{6.422875in}{1.144723in}}%
\pgfpathlineto{\pgfqpoint{6.426047in}{1.144719in}}%
\pgfpathlineto{\pgfqpoint{6.429219in}{1.144676in}}%
\pgfpathlineto{\pgfqpoint{6.432391in}{1.144670in}}%
\pgfpathlineto{\pgfqpoint{6.435563in}{1.144698in}}%
\pgfpathlineto{\pgfqpoint{6.438735in}{1.144734in}}%
\pgfpathlineto{\pgfqpoint{6.441907in}{1.144721in}}%
\pgfpathlineto{\pgfqpoint{6.445079in}{1.144729in}}%
\pgfpathlineto{\pgfqpoint{6.448251in}{1.144763in}}%
\pgfpathlineto{\pgfqpoint{6.451423in}{1.144710in}}%
\pgfpathlineto{\pgfqpoint{6.454595in}{1.144707in}}%
\pgfpathlineto{\pgfqpoint{6.457767in}{1.144742in}}%
\pgfpathlineto{\pgfqpoint{6.460939in}{1.144753in}}%
\pgfpathlineto{\pgfqpoint{6.464111in}{1.144762in}}%
\pgfpathlineto{\pgfqpoint{6.467283in}{1.144766in}}%
\pgfpathlineto{\pgfqpoint{6.470455in}{1.144764in}}%
\pgfpathlineto{\pgfqpoint{6.473627in}{1.144781in}}%
\pgfpathlineto{\pgfqpoint{6.476799in}{1.144791in}}%
\pgfpathlineto{\pgfqpoint{6.479971in}{1.144782in}}%
\pgfpathlineto{\pgfqpoint{6.483143in}{1.144827in}}%
\pgfpathlineto{\pgfqpoint{6.486315in}{1.144828in}}%
\pgfpathlineto{\pgfqpoint{6.489487in}{1.144896in}}%
\pgfpathlineto{\pgfqpoint{6.492659in}{1.144878in}}%
\pgfpathlineto{\pgfqpoint{6.495832in}{1.144868in}}%
\pgfpathlineto{\pgfqpoint{6.499004in}{1.144888in}}%
\pgfpathlineto{\pgfqpoint{6.502176in}{1.144871in}}%
\pgfpathlineto{\pgfqpoint{6.505348in}{1.144887in}}%
\pgfpathlineto{\pgfqpoint{6.508520in}{1.144858in}}%
\pgfpathlineto{\pgfqpoint{6.511692in}{1.144860in}}%
\pgfpathlineto{\pgfqpoint{6.514864in}{1.144851in}}%
\pgfpathlineto{\pgfqpoint{6.518036in}{1.144821in}}%
\pgfpathlineto{\pgfqpoint{6.521208in}{1.144840in}}%
\pgfpathlineto{\pgfqpoint{6.524380in}{1.144841in}}%
\pgfpathlineto{\pgfqpoint{6.527552in}{1.144827in}}%
\pgfpathlineto{\pgfqpoint{6.530724in}{1.144806in}}%
\pgfpathlineto{\pgfqpoint{6.533896in}{1.144816in}}%
\pgfpathlineto{\pgfqpoint{6.537068in}{1.144800in}}%
\pgfpathlineto{\pgfqpoint{6.540240in}{1.144782in}}%
\pgfpathlineto{\pgfqpoint{6.543412in}{1.144771in}}%
\pgfpathlineto{\pgfqpoint{6.546584in}{1.144736in}}%
\pgfpathlineto{\pgfqpoint{6.549756in}{1.144759in}}%
\pgfpathlineto{\pgfqpoint{6.552928in}{1.144753in}}%
\pgfpathlineto{\pgfqpoint{6.556100in}{1.144764in}}%
\pgfpathlineto{\pgfqpoint{6.559272in}{1.144759in}}%
\pgfpathlineto{\pgfqpoint{6.562444in}{1.144758in}}%
\pgfpathlineto{\pgfqpoint{6.565616in}{1.144787in}}%
\pgfpathlineto{\pgfqpoint{6.568788in}{1.144785in}}%
\pgfpathlineto{\pgfqpoint{6.571961in}{1.144797in}}%
\pgfpathlineto{\pgfqpoint{6.575133in}{1.144806in}}%
\pgfpathlineto{\pgfqpoint{6.578305in}{1.144786in}}%
\pgfpathlineto{\pgfqpoint{6.581477in}{1.144781in}}%
\pgfpathlineto{\pgfqpoint{6.584649in}{1.144817in}}%
\pgfpathlineto{\pgfqpoint{6.587821in}{1.144850in}}%
\pgfpathlineto{\pgfqpoint{6.590993in}{1.144850in}}%
\pgfpathlineto{\pgfqpoint{6.594165in}{1.144862in}}%
\pgfpathlineto{\pgfqpoint{6.597337in}{1.144851in}}%
\pgfpathlineto{\pgfqpoint{6.600509in}{1.144793in}}%
\pgfpathlineto{\pgfqpoint{6.603681in}{1.144762in}}%
\pgfpathlineto{\pgfqpoint{6.606853in}{1.144778in}}%
\pgfpathlineto{\pgfqpoint{6.610025in}{1.144788in}}%
\pgfpathlineto{\pgfqpoint{6.613197in}{1.144711in}}%
\pgfpathlineto{\pgfqpoint{6.616369in}{1.144680in}}%
\pgfpathlineto{\pgfqpoint{6.619541in}{1.144700in}}%
\pgfpathlineto{\pgfqpoint{6.622713in}{1.144747in}}%
\pgfpathlineto{\pgfqpoint{6.625885in}{1.144836in}}%
\pgfpathlineto{\pgfqpoint{6.629057in}{1.144823in}}%
\pgfpathlineto{\pgfqpoint{6.632229in}{1.144861in}}%
\pgfpathlineto{\pgfqpoint{6.635401in}{1.144957in}}%
\pgfpathlineto{\pgfqpoint{6.638573in}{1.144978in}}%
\pgfpathlineto{\pgfqpoint{6.641745in}{1.144972in}}%
\pgfpathlineto{\pgfqpoint{6.644917in}{1.144993in}}%
\pgfpathlineto{\pgfqpoint{6.648089in}{1.144959in}}%
\pgfpathlineto{\pgfqpoint{6.651262in}{1.145005in}}%
\pgfpathlineto{\pgfqpoint{6.654434in}{1.145009in}}%
\pgfpathlineto{\pgfqpoint{6.657606in}{1.145001in}}%
\pgfpathlineto{\pgfqpoint{6.660778in}{1.144964in}}%
\pgfpathlineto{\pgfqpoint{6.663950in}{1.144936in}}%
\pgfpathlineto{\pgfqpoint{6.667122in}{1.144947in}}%
\pgfpathlineto{\pgfqpoint{6.670294in}{1.144994in}}%
\pgfpathlineto{\pgfqpoint{6.673466in}{1.145025in}}%
\pgfpathlineto{\pgfqpoint{6.676638in}{1.145020in}}%
\pgfpathlineto{\pgfqpoint{6.679810in}{1.145034in}}%
\pgfpathlineto{\pgfqpoint{6.682982in}{1.145029in}}%
\pgfpathlineto{\pgfqpoint{6.686154in}{1.145045in}}%
\pgfpathlineto{\pgfqpoint{6.689326in}{1.145052in}}%
\pgfpathlineto{\pgfqpoint{6.692498in}{1.145145in}}%
\pgfpathlineto{\pgfqpoint{6.695670in}{1.145223in}}%
\pgfpathlineto{\pgfqpoint{6.698842in}{1.145220in}}%
\pgfpathlineto{\pgfqpoint{6.702014in}{1.145209in}}%
\pgfpathlineto{\pgfqpoint{6.705186in}{1.145198in}}%
\pgfpathlineto{\pgfqpoint{6.708358in}{1.145192in}}%
\pgfpathlineto{\pgfqpoint{6.711530in}{1.145194in}}%
\pgfpathlineto{\pgfqpoint{6.714702in}{1.145291in}}%
\pgfpathlineto{\pgfqpoint{6.717874in}{1.145294in}}%
\pgfpathlineto{\pgfqpoint{6.721046in}{1.145324in}}%
\pgfpathlineto{\pgfqpoint{6.724218in}{1.145318in}}%
\pgfpathlineto{\pgfqpoint{6.727391in}{1.145289in}}%
\pgfpathlineto{\pgfqpoint{6.730563in}{1.145307in}}%
\pgfpathlineto{\pgfqpoint{6.733735in}{1.145284in}}%
\pgfpathlineto{\pgfqpoint{6.736907in}{1.145319in}}%
\pgfpathlineto{\pgfqpoint{6.740079in}{1.145273in}}%
\pgfpathlineto{\pgfqpoint{6.743251in}{1.145246in}}%
\pgfpathlineto{\pgfqpoint{6.746423in}{1.145199in}}%
\pgfpathlineto{\pgfqpoint{6.749595in}{1.145175in}}%
\pgfpathlineto{\pgfqpoint{6.752767in}{1.145181in}}%
\pgfpathlineto{\pgfqpoint{6.755939in}{1.145207in}}%
\pgfpathlineto{\pgfqpoint{6.759111in}{1.145185in}}%
\pgfpathlineto{\pgfqpoint{6.762283in}{1.145205in}}%
\pgfpathlineto{\pgfqpoint{6.765455in}{1.145243in}}%
\pgfpathlineto{\pgfqpoint{6.768627in}{1.145250in}}%
\pgfpathlineto{\pgfqpoint{6.771799in}{1.145255in}}%
\pgfpathlineto{\pgfqpoint{6.774971in}{1.145253in}}%
\pgfpathlineto{\pgfqpoint{6.778143in}{1.145187in}}%
\pgfpathlineto{\pgfqpoint{6.781315in}{1.145207in}}%
\pgfpathlineto{\pgfqpoint{6.784487in}{1.145199in}}%
\pgfpathlineto{\pgfqpoint{6.787659in}{1.145220in}}%
\pgfpathlineto{\pgfqpoint{6.790831in}{1.145247in}}%
\pgfpathlineto{\pgfqpoint{6.794003in}{1.145255in}}%
\pgfpathlineto{\pgfqpoint{6.797175in}{1.145251in}}%
\pgfpathlineto{\pgfqpoint{6.800347in}{1.145246in}}%
\pgfpathlineto{\pgfqpoint{6.803519in}{1.145234in}}%
\pgfpathlineto{\pgfqpoint{6.806692in}{1.145224in}}%
\pgfpathlineto{\pgfqpoint{6.809864in}{1.145266in}}%
\pgfpathlineto{\pgfqpoint{6.813036in}{1.145295in}}%
\pgfpathlineto{\pgfqpoint{6.816208in}{1.145296in}}%
\pgfpathlineto{\pgfqpoint{6.819380in}{1.145263in}}%
\pgfpathlineto{\pgfqpoint{6.822552in}{1.145277in}}%
\pgfpathlineto{\pgfqpoint{6.825724in}{1.145277in}}%
\pgfpathlineto{\pgfqpoint{6.828896in}{1.145315in}}%
\pgfpathlineto{\pgfqpoint{6.832068in}{1.145326in}}%
\pgfpathlineto{\pgfqpoint{6.835240in}{1.145312in}}%
\pgfpathlineto{\pgfqpoint{6.838412in}{1.145348in}}%
\pgfpathlineto{\pgfqpoint{6.841584in}{1.145318in}}%
\pgfpathlineto{\pgfqpoint{6.844756in}{1.145375in}}%
\pgfpathlineto{\pgfqpoint{6.847928in}{1.145394in}}%
\pgfpathlineto{\pgfqpoint{6.851100in}{1.145401in}}%
\pgfpathlineto{\pgfqpoint{6.854272in}{1.145433in}}%
\pgfpathlineto{\pgfqpoint{6.857444in}{1.145430in}}%
\pgfpathlineto{\pgfqpoint{6.860616in}{1.145393in}}%
\pgfpathlineto{\pgfqpoint{6.863788in}{1.145415in}}%
\pgfpathlineto{\pgfqpoint{6.866960in}{1.145450in}}%
\pgfpathlineto{\pgfqpoint{6.870132in}{1.145504in}}%
\pgfpathlineto{\pgfqpoint{6.873304in}{1.145513in}}%
\pgfpathlineto{\pgfqpoint{6.876476in}{1.145523in}}%
\pgfpathlineto{\pgfqpoint{6.879648in}{1.145581in}}%
\pgfpathlineto{\pgfqpoint{6.882820in}{1.145569in}}%
\pgfpathlineto{\pgfqpoint{6.885993in}{1.145569in}}%
\pgfpathlineto{\pgfqpoint{6.889165in}{1.145580in}}%
\pgfpathlineto{\pgfqpoint{6.892337in}{1.145624in}}%
\pgfpathlineto{\pgfqpoint{6.895509in}{1.145590in}}%
\pgfpathlineto{\pgfqpoint{6.898681in}{1.145633in}}%
\pgfpathlineto{\pgfqpoint{6.901853in}{1.145668in}}%
\pgfpathlineto{\pgfqpoint{6.905025in}{1.145655in}}%
\pgfpathlineto{\pgfqpoint{6.908197in}{1.145722in}}%
\pgfpathlineto{\pgfqpoint{6.911369in}{1.145713in}}%
\pgfpathlineto{\pgfqpoint{6.914541in}{1.145723in}}%
\pgfpathlineto{\pgfqpoint{6.917713in}{1.145724in}}%
\pgfpathlineto{\pgfqpoint{6.920885in}{1.145755in}}%
\pgfpathlineto{\pgfqpoint{6.924057in}{1.145744in}}%
\pgfpathlineto{\pgfqpoint{6.927229in}{1.145761in}}%
\pgfpathlineto{\pgfqpoint{6.930401in}{1.145779in}}%
\pgfpathlineto{\pgfqpoint{6.933573in}{1.145792in}}%
\pgfpathlineto{\pgfqpoint{6.936745in}{1.145785in}}%
\pgfpathlineto{\pgfqpoint{6.939917in}{1.145801in}}%
\pgfpathlineto{\pgfqpoint{6.943089in}{1.145781in}}%
\pgfpathlineto{\pgfqpoint{6.946261in}{1.145794in}}%
\pgfpathlineto{\pgfqpoint{6.949433in}{1.145818in}}%
\pgfpathlineto{\pgfqpoint{6.952605in}{1.145755in}}%
\pgfpathlineto{\pgfqpoint{6.955777in}{1.145764in}}%
\pgfpathlineto{\pgfqpoint{6.958949in}{1.145805in}}%
\pgfpathlineto{\pgfqpoint{6.962122in}{1.145840in}}%
\pgfpathlineto{\pgfqpoint{6.965294in}{1.145860in}}%
\pgfpathlineto{\pgfqpoint{6.968466in}{1.145837in}}%
\pgfpathlineto{\pgfqpoint{6.971638in}{1.145854in}}%
\pgfpathlineto{\pgfqpoint{6.974810in}{1.145821in}}%
\pgfpathlineto{\pgfqpoint{6.977982in}{1.145845in}}%
\pgfpathlineto{\pgfqpoint{6.981154in}{1.145844in}}%
\pgfpathlineto{\pgfqpoint{6.984326in}{1.145872in}}%
\pgfpathlineto{\pgfqpoint{6.987498in}{1.145873in}}%
\pgfpathlineto{\pgfqpoint{6.990670in}{1.145857in}}%
\pgfpathlineto{\pgfqpoint{6.993842in}{1.145795in}}%
\pgfpathlineto{\pgfqpoint{6.997014in}{1.145765in}}%
\pgfpathlineto{\pgfqpoint{7.000186in}{1.145747in}}%
\pgfpathlineto{\pgfqpoint{7.003358in}{1.145698in}}%
\pgfpathlineto{\pgfqpoint{7.006530in}{1.145721in}}%
\pgfpathlineto{\pgfqpoint{7.009702in}{1.145696in}}%
\pgfpathlineto{\pgfqpoint{7.012874in}{1.145756in}}%
\pgfpathlineto{\pgfqpoint{7.016046in}{1.145707in}}%
\pgfpathlineto{\pgfqpoint{7.019218in}{1.145667in}}%
\pgfpathlineto{\pgfqpoint{7.022390in}{1.145685in}}%
\pgfpathlineto{\pgfqpoint{7.025562in}{1.145682in}}%
\pgfpathlineto{\pgfqpoint{7.028734in}{1.145705in}}%
\pgfpathlineto{\pgfqpoint{7.031906in}{1.145731in}}%
\pgfpathlineto{\pgfqpoint{7.035078in}{1.145743in}}%
\pgfpathlineto{\pgfqpoint{7.038250in}{1.145738in}}%
\pgfpathlineto{\pgfqpoint{7.041423in}{1.145735in}}%
\pgfpathlineto{\pgfqpoint{7.044595in}{1.145726in}}%
\pgfpathlineto{\pgfqpoint{7.047767in}{1.145735in}}%
\pgfpathlineto{\pgfqpoint{7.050939in}{1.145752in}}%
\pgfpathlineto{\pgfqpoint{7.054111in}{1.145724in}}%
\pgfpathlineto{\pgfqpoint{7.057283in}{1.145732in}}%
\pgfpathlineto{\pgfqpoint{7.060455in}{1.145721in}}%
\pgfpathlineto{\pgfqpoint{7.063627in}{1.145735in}}%
\pgfpathlineto{\pgfqpoint{7.066799in}{1.145741in}}%
\pgfpathlineto{\pgfqpoint{7.069971in}{1.145729in}}%
\pgfpathlineto{\pgfqpoint{7.073143in}{1.145743in}}%
\pgfpathlineto{\pgfqpoint{7.076315in}{1.145752in}}%
\pgfpathlineto{\pgfqpoint{7.079487in}{1.145762in}}%
\pgfpathlineto{\pgfqpoint{7.082659in}{1.145771in}}%
\pgfpathlineto{\pgfqpoint{7.085831in}{1.145778in}}%
\pgfpathlineto{\pgfqpoint{7.089003in}{1.145740in}}%
\pgfpathlineto{\pgfqpoint{7.092175in}{1.145742in}}%
\pgfpathlineto{\pgfqpoint{7.095347in}{1.145722in}}%
\pgfpathlineto{\pgfqpoint{7.098519in}{1.145663in}}%
\pgfpathlineto{\pgfqpoint{7.101691in}{1.145687in}}%
\pgfpathlineto{\pgfqpoint{7.104863in}{1.145704in}}%
\pgfpathlineto{\pgfqpoint{7.108035in}{1.145712in}}%
\pgfpathlineto{\pgfqpoint{7.111207in}{1.145698in}}%
\pgfpathlineto{\pgfqpoint{7.114379in}{1.145742in}}%
\pgfpathlineto{\pgfqpoint{7.117551in}{1.145779in}}%
\pgfpathlineto{\pgfqpoint{7.120724in}{1.145780in}}%
\pgfpathlineto{\pgfqpoint{7.123896in}{1.145849in}}%
\pgfpathlineto{\pgfqpoint{7.127068in}{1.145885in}}%
\pgfpathlineto{\pgfqpoint{7.130240in}{1.145903in}}%
\pgfpathlineto{\pgfqpoint{7.133412in}{1.145911in}}%
\pgfpathlineto{\pgfqpoint{7.136584in}{1.145873in}}%
\pgfpathlineto{\pgfqpoint{7.139756in}{1.145889in}}%
\pgfpathlineto{\pgfqpoint{7.142928in}{1.145865in}}%
\pgfpathlineto{\pgfqpoint{7.146100in}{1.145822in}}%
\pgfpathlineto{\pgfqpoint{7.149272in}{1.145831in}}%
\pgfpathlineto{\pgfqpoint{7.152444in}{1.145868in}}%
\pgfpathlineto{\pgfqpoint{7.155616in}{1.145816in}}%
\pgfpathlineto{\pgfqpoint{7.158788in}{1.145822in}}%
\pgfpathlineto{\pgfqpoint{7.161960in}{1.145781in}}%
\pgfpathlineto{\pgfqpoint{7.165132in}{1.145724in}}%
\pgfpathlineto{\pgfqpoint{7.168304in}{1.145729in}}%
\pgfpathlineto{\pgfqpoint{7.171476in}{1.145736in}}%
\pgfpathlineto{\pgfqpoint{7.174648in}{1.145799in}}%
\pgfpathlineto{\pgfqpoint{7.177820in}{1.145774in}}%
\pgfpathlineto{\pgfqpoint{7.180992in}{1.145765in}}%
\pgfpathlineto{\pgfqpoint{7.184164in}{1.145779in}}%
\pgfpathlineto{\pgfqpoint{7.187336in}{1.145801in}}%
\pgfpathlineto{\pgfqpoint{7.190508in}{1.145807in}}%
\pgfpathlineto{\pgfqpoint{7.193680in}{1.145860in}}%
\pgfpathlineto{\pgfqpoint{7.196853in}{1.145823in}}%
\pgfpathlineto{\pgfqpoint{7.200025in}{1.145855in}}%
\pgfpathlineto{\pgfqpoint{7.203197in}{1.145885in}}%
\pgfpathlineto{\pgfqpoint{7.206369in}{1.145947in}}%
\pgfpathlineto{\pgfqpoint{7.209541in}{1.145981in}}%
\pgfpathlineto{\pgfqpoint{7.212713in}{1.145955in}}%
\pgfpathlineto{\pgfqpoint{7.215885in}{1.145882in}}%
\pgfpathlineto{\pgfqpoint{7.219057in}{1.145860in}}%
\pgfpathlineto{\pgfqpoint{7.222229in}{1.145798in}}%
\pgfpathlineto{\pgfqpoint{7.225401in}{1.145805in}}%
\pgfpathlineto{\pgfqpoint{7.228573in}{1.145754in}}%
\pgfpathlineto{\pgfqpoint{7.231745in}{1.145760in}}%
\pgfpathlineto{\pgfqpoint{7.234917in}{1.145735in}}%
\pgfpathlineto{\pgfqpoint{7.238089in}{1.145769in}}%
\pgfpathlineto{\pgfqpoint{7.241261in}{1.145818in}}%
\pgfpathlineto{\pgfqpoint{7.244433in}{1.145787in}}%
\pgfpathlineto{\pgfqpoint{7.247605in}{1.145741in}}%
\pgfpathlineto{\pgfqpoint{7.250777in}{1.145749in}}%
\pgfpathlineto{\pgfqpoint{7.253949in}{1.145770in}}%
\pgfpathlineto{\pgfqpoint{7.257121in}{1.145770in}}%
\pgfpathlineto{\pgfqpoint{7.260293in}{1.145778in}}%
\pgfpathlineto{\pgfqpoint{7.263465in}{1.145807in}}%
\pgfpathlineto{\pgfqpoint{7.266637in}{1.145876in}}%
\pgfpathlineto{\pgfqpoint{7.269809in}{1.145915in}}%
\pgfpathlineto{\pgfqpoint{7.272981in}{1.145965in}}%
\pgfpathlineto{\pgfqpoint{7.276154in}{1.145956in}}%
\pgfpathlineto{\pgfqpoint{7.279326in}{1.145947in}}%
\pgfpathlineto{\pgfqpoint{7.282498in}{1.145966in}}%
\pgfpathlineto{\pgfqpoint{7.285670in}{1.145975in}}%
\pgfpathlineto{\pgfqpoint{7.288842in}{1.146068in}}%
\pgfpathlineto{\pgfqpoint{7.292014in}{1.146075in}}%
\pgfpathlineto{\pgfqpoint{7.295186in}{1.146099in}}%
\pgfpathlineto{\pgfqpoint{7.298358in}{1.146112in}}%
\pgfpathlineto{\pgfqpoint{7.301530in}{1.146112in}}%
\pgfpathlineto{\pgfqpoint{7.304702in}{1.146105in}}%
\pgfpathlineto{\pgfqpoint{7.307874in}{1.146058in}}%
\pgfpathlineto{\pgfqpoint{7.311046in}{1.146113in}}%
\pgfpathlineto{\pgfqpoint{7.314218in}{1.146084in}}%
\pgfpathlineto{\pgfqpoint{7.317390in}{1.146134in}}%
\pgfpathlineto{\pgfqpoint{7.320562in}{1.146153in}}%
\pgfpathlineto{\pgfqpoint{7.323734in}{1.146156in}}%
\pgfpathlineto{\pgfqpoint{7.326906in}{1.146150in}}%
\pgfpathlineto{\pgfqpoint{7.330078in}{1.146107in}}%
\pgfpathlineto{\pgfqpoint{7.333250in}{1.146057in}}%
\pgfpathlineto{\pgfqpoint{7.336422in}{1.146056in}}%
\pgfpathlineto{\pgfqpoint{7.339594in}{1.146083in}}%
\pgfpathlineto{\pgfqpoint{7.342766in}{1.146052in}}%
\pgfpathlineto{\pgfqpoint{7.345938in}{1.146142in}}%
\pgfpathlineto{\pgfqpoint{7.349110in}{1.146155in}}%
\pgfpathlineto{\pgfqpoint{7.352282in}{1.146147in}}%
\pgfpathlineto{\pgfqpoint{7.355455in}{1.146174in}}%
\pgfpathlineto{\pgfqpoint{7.358627in}{1.146152in}}%
\pgfpathlineto{\pgfqpoint{7.361799in}{1.146176in}}%
\pgfpathlineto{\pgfqpoint{7.364971in}{1.146258in}}%
\pgfpathlineto{\pgfqpoint{7.368143in}{1.146305in}}%
\pgfpathlineto{\pgfqpoint{7.371315in}{1.146334in}}%
\pgfpathlineto{\pgfqpoint{7.374487in}{1.146399in}}%
\pgfpathlineto{\pgfqpoint{7.377659in}{1.146419in}}%
\pgfpathlineto{\pgfqpoint{7.380831in}{1.146424in}}%
\pgfpathlineto{\pgfqpoint{7.384003in}{1.146365in}}%
\pgfpathlineto{\pgfqpoint{7.387175in}{1.146321in}}%
\pgfpathlineto{\pgfqpoint{7.390347in}{1.146319in}}%
\pgfpathlineto{\pgfqpoint{7.393519in}{1.146292in}}%
\pgfpathlineto{\pgfqpoint{7.396691in}{1.146313in}}%
\pgfpathlineto{\pgfqpoint{7.399863in}{1.146327in}}%
\pgfpathlineto{\pgfqpoint{7.403035in}{1.146286in}}%
\pgfpathlineto{\pgfqpoint{7.406207in}{1.146307in}}%
\pgfpathlineto{\pgfqpoint{7.409379in}{1.146302in}}%
\pgfpathlineto{\pgfqpoint{7.412551in}{1.146255in}}%
\pgfpathlineto{\pgfqpoint{7.415723in}{1.146264in}}%
\pgfpathlineto{\pgfqpoint{7.418895in}{1.146289in}}%
\pgfpathlineto{\pgfqpoint{7.422067in}{1.146303in}}%
\pgfpathlineto{\pgfqpoint{7.425239in}{1.146348in}}%
\pgfpathlineto{\pgfqpoint{7.428411in}{1.146329in}}%
\pgfpathlineto{\pgfqpoint{7.431584in}{1.146310in}}%
\pgfpathlineto{\pgfqpoint{7.434756in}{1.146332in}}%
\pgfpathlineto{\pgfqpoint{7.437928in}{1.146307in}}%
\pgfpathlineto{\pgfqpoint{7.441100in}{1.146303in}}%
\pgfpathlineto{\pgfqpoint{7.444272in}{1.146307in}}%
\pgfpathlineto{\pgfqpoint{7.447444in}{1.146273in}}%
\pgfpathlineto{\pgfqpoint{7.450616in}{1.146297in}}%
\pgfpathlineto{\pgfqpoint{7.453788in}{1.146309in}}%
\pgfpathlineto{\pgfqpoint{7.456960in}{1.146286in}}%
\pgfpathlineto{\pgfqpoint{7.460132in}{1.146322in}}%
\pgfpathlineto{\pgfqpoint{7.463304in}{1.146369in}}%
\pgfpathlineto{\pgfqpoint{7.466476in}{1.146389in}}%
\pgfpathlineto{\pgfqpoint{7.469648in}{1.146353in}}%
\pgfpathlineto{\pgfqpoint{7.472820in}{1.146346in}}%
\pgfpathlineto{\pgfqpoint{7.475992in}{1.146356in}}%
\pgfpathlineto{\pgfqpoint{7.479164in}{1.146366in}}%
\pgfpathlineto{\pgfqpoint{7.482336in}{1.146360in}}%
\pgfpathlineto{\pgfqpoint{7.485508in}{1.146343in}}%
\pgfpathlineto{\pgfqpoint{7.488680in}{1.146330in}}%
\pgfpathlineto{\pgfqpoint{7.491852in}{1.146326in}}%
\pgfpathlineto{\pgfqpoint{7.495024in}{1.146347in}}%
\pgfpathlineto{\pgfqpoint{7.498196in}{1.146348in}}%
\pgfpathlineto{\pgfqpoint{7.501368in}{1.146349in}}%
\pgfpathlineto{\pgfqpoint{7.504540in}{1.146425in}}%
\pgfpathlineto{\pgfqpoint{7.507712in}{1.146434in}}%
\pgfpathlineto{\pgfqpoint{7.510885in}{1.146391in}}%
\pgfpathlineto{\pgfqpoint{7.514057in}{1.146441in}}%
\pgfpathlineto{\pgfqpoint{7.517229in}{1.146434in}}%
\pgfpathlineto{\pgfqpoint{7.520401in}{1.146350in}}%
\pgfpathlineto{\pgfqpoint{7.523573in}{1.146324in}}%
\pgfpathlineto{\pgfqpoint{7.526745in}{1.146272in}}%
\pgfpathlineto{\pgfqpoint{7.529917in}{1.146283in}}%
\pgfpathlineto{\pgfqpoint{7.533089in}{1.146264in}}%
\pgfpathlineto{\pgfqpoint{7.536261in}{1.146304in}}%
\pgfpathlineto{\pgfqpoint{7.539433in}{1.146349in}}%
\pgfpathlineto{\pgfqpoint{7.542605in}{1.146362in}}%
\pgfpathlineto{\pgfqpoint{7.545777in}{1.146400in}}%
\pgfpathlineto{\pgfqpoint{7.548949in}{1.146403in}}%
\pgfpathlineto{\pgfqpoint{7.552121in}{1.146411in}}%
\pgfpathlineto{\pgfqpoint{7.555293in}{1.146457in}}%
\pgfpathlineto{\pgfqpoint{7.558465in}{1.146500in}}%
\pgfpathlineto{\pgfqpoint{7.561637in}{1.146422in}}%
\pgfpathlineto{\pgfqpoint{7.564809in}{1.146426in}}%
\pgfpathlineto{\pgfqpoint{7.567981in}{1.146447in}}%
\pgfpathlineto{\pgfqpoint{7.571153in}{1.146476in}}%
\pgfpathlineto{\pgfqpoint{7.574325in}{1.146518in}}%
\pgfpathlineto{\pgfqpoint{7.577497in}{1.146580in}}%
\pgfpathlineto{\pgfqpoint{7.580669in}{1.146632in}}%
\pgfpathlineto{\pgfqpoint{7.583841in}{1.146643in}}%
\pgfpathlineto{\pgfqpoint{7.587013in}{1.146612in}}%
\pgfpathlineto{\pgfqpoint{7.590186in}{1.146681in}}%
\pgfpathlineto{\pgfqpoint{7.593358in}{1.146675in}}%
\pgfpathlineto{\pgfqpoint{7.596530in}{1.146657in}}%
\pgfpathlineto{\pgfqpoint{7.599702in}{1.146618in}}%
\pgfpathlineto{\pgfqpoint{7.602874in}{1.146626in}}%
\pgfpathlineto{\pgfqpoint{7.606046in}{1.146679in}}%
\pgfpathlineto{\pgfqpoint{7.609218in}{1.146731in}}%
\pgfpathlineto{\pgfqpoint{7.612390in}{1.146763in}}%
\pgfpathlineto{\pgfqpoint{7.615562in}{1.146709in}}%
\pgfpathlineto{\pgfqpoint{7.618734in}{1.146645in}}%
\pgfpathlineto{\pgfqpoint{7.621906in}{1.146634in}}%
\pgfpathlineto{\pgfqpoint{7.625078in}{1.146687in}}%
\pgfpathlineto{\pgfqpoint{7.628250in}{1.146682in}}%
\pgfpathlineto{\pgfqpoint{7.631422in}{1.146667in}}%
\pgfpathlineto{\pgfqpoint{7.634594in}{1.146625in}}%
\pgfpathlineto{\pgfqpoint{7.637766in}{1.146593in}}%
\pgfpathlineto{\pgfqpoint{7.640938in}{1.146636in}}%
\pgfpathlineto{\pgfqpoint{7.644110in}{1.146628in}}%
\pgfpathlineto{\pgfqpoint{7.647282in}{1.146589in}}%
\pgfpathlineto{\pgfqpoint{7.650454in}{1.146573in}}%
\pgfpathlineto{\pgfqpoint{7.653626in}{1.146536in}}%
\pgfpathlineto{\pgfqpoint{7.656798in}{1.146630in}}%
\pgfpathlineto{\pgfqpoint{7.659970in}{1.146642in}}%
\pgfpathlineto{\pgfqpoint{7.663142in}{1.146640in}}%
\pgfpathlineto{\pgfqpoint{7.666315in}{1.146597in}}%
\pgfpathlineto{\pgfqpoint{7.669487in}{1.146657in}}%
\pgfpathlineto{\pgfqpoint{7.672659in}{1.146705in}}%
\pgfpathlineto{\pgfqpoint{7.675831in}{1.146737in}}%
\pgfpathlineto{\pgfqpoint{7.679003in}{1.146776in}}%
\pgfpathlineto{\pgfqpoint{7.682175in}{1.146794in}}%
\pgfpathlineto{\pgfqpoint{7.685347in}{1.146810in}}%
\pgfpathlineto{\pgfqpoint{7.688519in}{1.146826in}}%
\pgfpathlineto{\pgfqpoint{7.691691in}{1.146823in}}%
\pgfpathlineto{\pgfqpoint{7.694863in}{1.146836in}}%
\pgfpathlineto{\pgfqpoint{7.698035in}{1.146895in}}%
\pgfpathlineto{\pgfqpoint{7.701207in}{1.146965in}}%
\pgfpathlineto{\pgfqpoint{7.704379in}{1.146972in}}%
\pgfpathlineto{\pgfqpoint{7.707551in}{1.146962in}}%
\pgfpathlineto{\pgfqpoint{7.710723in}{1.146899in}}%
\pgfpathlineto{\pgfqpoint{7.713895in}{1.146957in}}%
\pgfpathlineto{\pgfqpoint{7.717067in}{1.146956in}}%
\pgfpathlineto{\pgfqpoint{7.720239in}{1.146909in}}%
\pgfpathlineto{\pgfqpoint{7.723411in}{1.146974in}}%
\pgfpathlineto{\pgfqpoint{7.726583in}{1.146969in}}%
\pgfpathlineto{\pgfqpoint{7.729755in}{1.146953in}}%
\pgfpathlineto{\pgfqpoint{7.732927in}{1.146994in}}%
\pgfpathlineto{\pgfqpoint{7.736099in}{1.146941in}}%
\pgfpathlineto{\pgfqpoint{7.739271in}{1.146901in}}%
\pgfpathlineto{\pgfqpoint{7.742443in}{1.146877in}}%
\pgfpathlineto{\pgfqpoint{7.745616in}{1.146942in}}%
\pgfpathlineto{\pgfqpoint{7.748788in}{1.146926in}}%
\pgfpathlineto{\pgfqpoint{7.751960in}{1.146804in}}%
\pgfpathlineto{\pgfqpoint{7.755132in}{1.146781in}}%
\pgfpathlineto{\pgfqpoint{7.758304in}{1.146801in}}%
\pgfpathlineto{\pgfqpoint{7.761476in}{1.146826in}}%
\pgfpathlineto{\pgfqpoint{7.764648in}{1.146822in}}%
\pgfpathlineto{\pgfqpoint{7.767820in}{1.146834in}}%
\pgfpathlineto{\pgfqpoint{7.770992in}{1.146834in}}%
\pgfpathlineto{\pgfqpoint{7.774164in}{1.146836in}}%
\pgfpathlineto{\pgfqpoint{7.777336in}{1.146850in}}%
\pgfpathlineto{\pgfqpoint{7.780508in}{1.146885in}}%
\pgfpathlineto{\pgfqpoint{7.783680in}{1.146933in}}%
\pgfpathlineto{\pgfqpoint{7.786852in}{1.146921in}}%
\pgfpathlineto{\pgfqpoint{7.790024in}{1.147236in}}%
\pgfpathlineto{\pgfqpoint{7.793196in}{1.147532in}}%
\pgfpathlineto{\pgfqpoint{7.796368in}{1.147833in}}%
\pgfpathlineto{\pgfqpoint{7.799540in}{1.148136in}}%
\pgfpathlineto{\pgfqpoint{7.802712in}{1.148441in}}%
\pgfpathlineto{\pgfqpoint{7.805884in}{1.148747in}}%
\pgfpathlineto{\pgfqpoint{7.809056in}{1.149045in}}%
\pgfpathlineto{\pgfqpoint{7.812228in}{1.149342in}}%
\pgfpathlineto{\pgfqpoint{7.815400in}{1.149641in}}%
\pgfpathlineto{\pgfqpoint{7.818572in}{1.149939in}}%
\pgfpathlineto{\pgfqpoint{7.821744in}{1.150232in}}%
\pgfpathlineto{\pgfqpoint{7.824917in}{1.150528in}}%
\pgfpathlineto{\pgfqpoint{7.828089in}{1.150831in}}%
\pgfpathlineto{\pgfqpoint{7.831261in}{1.151122in}}%
\pgfpathlineto{\pgfqpoint{7.834433in}{1.151411in}}%
\pgfpathlineto{\pgfqpoint{7.837605in}{1.151710in}}%
\pgfpathlineto{\pgfqpoint{7.840777in}{1.152006in}}%
\pgfpathlineto{\pgfqpoint{7.843949in}{1.152303in}}%
\pgfpathlineto{\pgfqpoint{7.847121in}{1.152596in}}%
\pgfpathlineto{\pgfqpoint{7.850293in}{1.152891in}}%
\pgfpathlineto{\pgfqpoint{7.853465in}{1.153185in}}%
\pgfpathlineto{\pgfqpoint{7.856637in}{1.153480in}}%
\pgfpathlineto{\pgfqpoint{7.859809in}{1.153775in}}%
\pgfpathlineto{\pgfqpoint{7.862981in}{1.154070in}}%
\pgfpathlineto{\pgfqpoint{7.866153in}{1.154369in}}%
\pgfpathlineto{\pgfqpoint{7.869325in}{1.154659in}}%
\pgfpathlineto{\pgfqpoint{7.872497in}{1.154955in}}%
\pgfpathlineto{\pgfqpoint{7.875669in}{1.155245in}}%
\pgfpathlineto{\pgfqpoint{7.878841in}{1.155538in}}%
\pgfpathlineto{\pgfqpoint{7.882013in}{1.155829in}}%
\pgfpathlineto{\pgfqpoint{7.885185in}{1.156111in}}%
\pgfpathlineto{\pgfqpoint{7.888357in}{1.157085in}}%
\pgfpathlineto{\pgfqpoint{7.891529in}{1.159938in}}%
\pgfpathlineto{\pgfqpoint{7.894701in}{1.162936in}}%
\pgfpathlineto{\pgfqpoint{7.897873in}{1.167541in}}%
\pgfpathlineto{\pgfqpoint{7.901046in}{1.170472in}}%
\pgfpathlineto{\pgfqpoint{7.904218in}{1.173441in}}%
\pgfpathlineto{\pgfqpoint{7.907390in}{1.176342in}}%
\pgfpathlineto{\pgfqpoint{7.910562in}{1.179360in}}%
\pgfpathlineto{\pgfqpoint{7.913734in}{1.182326in}}%
\pgfpathlineto{\pgfqpoint{7.916906in}{1.185341in}}%
\pgfpathlineto{\pgfqpoint{7.920078in}{1.188346in}}%
\pgfpathlineto{\pgfqpoint{7.923250in}{1.191375in}}%
\pgfpathlineto{\pgfqpoint{7.926422in}{1.194402in}}%
\pgfpathlineto{\pgfqpoint{7.929594in}{1.197363in}}%
\pgfpathlineto{\pgfqpoint{7.932766in}{1.200369in}}%
\pgfpathlineto{\pgfqpoint{7.935938in}{1.203374in}}%
\pgfpathlineto{\pgfqpoint{7.939110in}{1.206374in}}%
\pgfpathlineto{\pgfqpoint{7.942282in}{1.209388in}}%
\pgfpathlineto{\pgfqpoint{7.945454in}{1.212397in}}%
\pgfpathlineto{\pgfqpoint{7.948626in}{1.215382in}}%
\pgfpathlineto{\pgfqpoint{7.951798in}{1.218411in}}%
\pgfpathlineto{\pgfqpoint{7.954970in}{1.221356in}}%
\pgfpathlineto{\pgfqpoint{7.958142in}{1.224260in}}%
\pgfpathlineto{\pgfqpoint{7.961314in}{1.227232in}}%
\pgfpathlineto{\pgfqpoint{7.964486in}{1.230328in}}%
\pgfpathlineto{\pgfqpoint{7.967658in}{1.233304in}}%
\pgfpathlineto{\pgfqpoint{7.970830in}{1.236287in}}%
\pgfpathlineto{\pgfqpoint{7.974002in}{1.239299in}}%
\pgfpathlineto{\pgfqpoint{7.977174in}{1.242321in}}%
\pgfpathlineto{\pgfqpoint{7.980347in}{1.245332in}}%
\pgfpathlineto{\pgfqpoint{7.983519in}{1.248324in}}%
\pgfpathlineto{\pgfqpoint{7.986691in}{1.251368in}}%
\pgfpathlineto{\pgfqpoint{7.989863in}{1.254409in}}%
\pgfpathlineto{\pgfqpoint{7.993035in}{1.257412in}}%
\pgfpathlineto{\pgfqpoint{7.996207in}{1.260451in}}%
\pgfpathlineto{\pgfqpoint{7.999379in}{1.263109in}}%
\pgfpathlineto{\pgfqpoint{8.002551in}{1.265847in}}%
\pgfpathlineto{\pgfqpoint{8.005723in}{1.268851in}}%
\pgfpathlineto{\pgfqpoint{8.008895in}{1.271825in}}%
\pgfpathlineto{\pgfqpoint{8.012067in}{1.274782in}}%
\pgfpathlineto{\pgfqpoint{8.015239in}{1.277736in}}%
\pgfpathlineto{\pgfqpoint{8.018411in}{1.280719in}}%
\pgfpathlineto{\pgfqpoint{8.021583in}{1.283686in}}%
\pgfpathlineto{\pgfqpoint{8.024755in}{1.286683in}}%
\pgfpathlineto{\pgfqpoint{8.027927in}{1.289616in}}%
\pgfpathlineto{\pgfqpoint{8.031099in}{1.292602in}}%
\pgfpathlineto{\pgfqpoint{8.034271in}{1.295617in}}%
\pgfpathlineto{\pgfqpoint{8.037443in}{1.298627in}}%
\pgfpathlineto{\pgfqpoint{8.040615in}{1.301654in}}%
\pgfpathlineto{\pgfqpoint{8.043787in}{1.304673in}}%
\pgfpathlineto{\pgfqpoint{8.046959in}{1.307673in}}%
\pgfpathlineto{\pgfqpoint{8.050131in}{1.310691in}}%
\pgfpathlineto{\pgfqpoint{8.053303in}{1.313658in}}%
\pgfpathlineto{\pgfqpoint{8.056475in}{1.316653in}}%
\pgfpathlineto{\pgfqpoint{8.059648in}{1.319588in}}%
\pgfpathlineto{\pgfqpoint{8.062820in}{1.322555in}}%
\pgfpathlineto{\pgfqpoint{8.065992in}{1.325557in}}%
\pgfpathlineto{\pgfqpoint{8.069164in}{1.328602in}}%
\pgfpathlineto{\pgfqpoint{8.072336in}{1.331609in}}%
\pgfpathlineto{\pgfqpoint{8.075508in}{1.334567in}}%
\pgfpathlineto{\pgfqpoint{8.078680in}{1.337609in}}%
\pgfpathlineto{\pgfqpoint{8.081852in}{1.340668in}}%
\pgfpathlineto{\pgfqpoint{8.085024in}{1.343716in}}%
\pgfpathlineto{\pgfqpoint{8.088196in}{1.346726in}}%
\pgfpathlineto{\pgfqpoint{8.091368in}{1.349685in}}%
\pgfpathlineto{\pgfqpoint{8.094540in}{1.352762in}}%
\pgfpathlineto{\pgfqpoint{8.097712in}{1.355844in}}%
\pgfpathlineto{\pgfqpoint{8.100884in}{1.358905in}}%
\pgfpathlineto{\pgfqpoint{8.104056in}{1.361869in}}%
\pgfpathlineto{\pgfqpoint{8.107228in}{1.364821in}}%
\pgfpathlineto{\pgfqpoint{8.110400in}{1.367815in}}%
\pgfpathlineto{\pgfqpoint{8.113572in}{1.370839in}}%
\pgfpathlineto{\pgfqpoint{8.116744in}{1.373860in}}%
\pgfpathlineto{\pgfqpoint{8.119916in}{1.376816in}}%
\pgfpathlineto{\pgfqpoint{8.123088in}{1.379848in}}%
\pgfpathlineto{\pgfqpoint{8.126260in}{1.382848in}}%
\pgfpathlineto{\pgfqpoint{8.129432in}{1.385832in}}%
\pgfpathlineto{\pgfqpoint{8.132604in}{1.388754in}}%
\pgfpathlineto{\pgfqpoint{8.135777in}{1.391752in}}%
\pgfpathlineto{\pgfqpoint{8.138949in}{1.394754in}}%
\pgfpathlineto{\pgfqpoint{8.142121in}{1.397734in}}%
\pgfpathlineto{\pgfqpoint{8.145293in}{1.400854in}}%
\pgfpathlineto{\pgfqpoint{8.148465in}{1.403888in}}%
\pgfpathlineto{\pgfqpoint{8.151637in}{1.406905in}}%
\pgfpathlineto{\pgfqpoint{8.154809in}{1.409913in}}%
\pgfpathlineto{\pgfqpoint{8.157981in}{1.412953in}}%
\pgfpathlineto{\pgfqpoint{8.161153in}{1.415932in}}%
\pgfpathlineto{\pgfqpoint{8.164325in}{1.418913in}}%
\pgfpathlineto{\pgfqpoint{8.167497in}{1.421961in}}%
\pgfpathlineto{\pgfqpoint{8.170669in}{1.424955in}}%
\pgfpathlineto{\pgfqpoint{8.173841in}{1.427996in}}%
\pgfpathlineto{\pgfqpoint{8.177013in}{1.431009in}}%
\pgfpathlineto{\pgfqpoint{8.180185in}{1.433875in}}%
\pgfpathlineto{\pgfqpoint{8.183357in}{1.436915in}}%
\pgfpathlineto{\pgfqpoint{8.186529in}{1.439950in}}%
\pgfpathlineto{\pgfqpoint{8.189701in}{1.443009in}}%
\pgfpathlineto{\pgfqpoint{8.192873in}{1.446019in}}%
\pgfpathlineto{\pgfqpoint{8.196045in}{1.448958in}}%
\pgfpathlineto{\pgfqpoint{8.199217in}{1.452002in}}%
\pgfpathlineto{\pgfqpoint{8.202389in}{1.455042in}}%
\pgfpathlineto{\pgfqpoint{8.205561in}{1.458015in}}%
\pgfpathlineto{\pgfqpoint{8.208733in}{1.461036in}}%
\pgfpathlineto{\pgfqpoint{8.211905in}{1.464062in}}%
\pgfpathlineto{\pgfqpoint{8.215078in}{1.467154in}}%
\pgfpathlineto{\pgfqpoint{8.218250in}{1.470128in}}%
\pgfpathlineto{\pgfqpoint{8.221422in}{1.473105in}}%
\pgfpathlineto{\pgfqpoint{8.224594in}{1.476108in}}%
\pgfpathlineto{\pgfqpoint{8.227766in}{1.479119in}}%
\pgfpathlineto{\pgfqpoint{8.230938in}{1.481957in}}%
\pgfpathlineto{\pgfqpoint{8.234110in}{1.484955in}}%
\pgfpathlineto{\pgfqpoint{8.237282in}{1.487881in}}%
\pgfpathlineto{\pgfqpoint{8.240454in}{1.490908in}}%
\pgfpathlineto{\pgfqpoint{8.243626in}{1.493882in}}%
\pgfpathlineto{\pgfqpoint{8.246798in}{1.496882in}}%
\pgfpathlineto{\pgfqpoint{8.249970in}{1.499847in}}%
\pgfpathlineto{\pgfqpoint{8.253142in}{1.502785in}}%
\pgfpathlineto{\pgfqpoint{8.256314in}{1.505670in}}%
\pgfpathlineto{\pgfqpoint{8.259486in}{1.508672in}}%
\pgfpathlineto{\pgfqpoint{8.262658in}{1.511730in}}%
\pgfpathlineto{\pgfqpoint{8.265830in}{1.514781in}}%
\pgfpathlineto{\pgfqpoint{8.269002in}{1.517764in}}%
\pgfpathlineto{\pgfqpoint{8.272174in}{1.520725in}}%
\pgfpathlineto{\pgfqpoint{8.275346in}{1.523666in}}%
\pgfpathlineto{\pgfqpoint{8.278518in}{1.526708in}}%
\pgfpathlineto{\pgfqpoint{8.281690in}{1.529732in}}%
\pgfpathlineto{\pgfqpoint{8.281690in}{1.938557in}}%
\pgfpathlineto{\pgfqpoint{8.281690in}{1.938557in}}%
\pgfpathlineto{\pgfqpoint{8.278518in}{1.935414in}}%
\pgfpathlineto{\pgfqpoint{8.275346in}{1.932257in}}%
\pgfpathlineto{\pgfqpoint{8.272174in}{1.929103in}}%
\pgfpathlineto{\pgfqpoint{8.269002in}{1.925931in}}%
\pgfpathlineto{\pgfqpoint{8.265830in}{1.922794in}}%
\pgfpathlineto{\pgfqpoint{8.262658in}{1.919638in}}%
\pgfpathlineto{\pgfqpoint{8.259486in}{1.916492in}}%
\pgfpathlineto{\pgfqpoint{8.256314in}{1.913328in}}%
\pgfpathlineto{\pgfqpoint{8.253142in}{1.910130in}}%
\pgfpathlineto{\pgfqpoint{8.249970in}{1.906954in}}%
\pgfpathlineto{\pgfqpoint{8.246798in}{1.903809in}}%
\pgfpathlineto{\pgfqpoint{8.243626in}{1.900629in}}%
\pgfpathlineto{\pgfqpoint{8.240454in}{1.897473in}}%
\pgfpathlineto{\pgfqpoint{8.237282in}{1.894292in}}%
\pgfpathlineto{\pgfqpoint{8.234110in}{1.891063in}}%
\pgfpathlineto{\pgfqpoint{8.230938in}{1.887877in}}%
\pgfpathlineto{\pgfqpoint{8.227766in}{1.884635in}}%
\pgfpathlineto{\pgfqpoint{8.224594in}{1.881501in}}%
\pgfpathlineto{\pgfqpoint{8.221422in}{1.878339in}}%
\pgfpathlineto{\pgfqpoint{8.218250in}{1.875163in}}%
\pgfpathlineto{\pgfqpoint{8.215078in}{1.871973in}}%
\pgfpathlineto{\pgfqpoint{8.211905in}{1.868799in}}%
\pgfpathlineto{\pgfqpoint{8.208733in}{1.865672in}}%
\pgfpathlineto{\pgfqpoint{8.205561in}{1.862520in}}%
\pgfpathlineto{\pgfqpoint{8.202389in}{1.859345in}}%
\pgfpathlineto{\pgfqpoint{8.199217in}{1.856181in}}%
\pgfpathlineto{\pgfqpoint{8.196045in}{1.853006in}}%
\pgfpathlineto{\pgfqpoint{8.192873in}{1.849784in}}%
\pgfpathlineto{\pgfqpoint{8.189701in}{1.846609in}}%
\pgfpathlineto{\pgfqpoint{8.186529in}{1.843421in}}%
\pgfpathlineto{\pgfqpoint{8.183357in}{1.840252in}}%
\pgfpathlineto{\pgfqpoint{8.180185in}{1.837097in}}%
\pgfpathlineto{\pgfqpoint{8.177013in}{1.833881in}}%
\pgfpathlineto{\pgfqpoint{8.173841in}{1.830713in}}%
\pgfpathlineto{\pgfqpoint{8.170669in}{1.827540in}}%
\pgfpathlineto{\pgfqpoint{8.167497in}{1.824329in}}%
\pgfpathlineto{\pgfqpoint{8.164325in}{1.821170in}}%
\pgfpathlineto{\pgfqpoint{8.161153in}{1.817992in}}%
\pgfpathlineto{\pgfqpoint{8.157981in}{1.814806in}}%
\pgfpathlineto{\pgfqpoint{8.154809in}{1.811685in}}%
\pgfpathlineto{\pgfqpoint{8.151637in}{1.808506in}}%
\pgfpathlineto{\pgfqpoint{8.148465in}{1.805290in}}%
\pgfpathlineto{\pgfqpoint{8.145293in}{1.802102in}}%
\pgfpathlineto{\pgfqpoint{8.142121in}{1.798967in}}%
\pgfpathlineto{\pgfqpoint{8.138949in}{1.795792in}}%
\pgfpathlineto{\pgfqpoint{8.135777in}{1.792616in}}%
\pgfpathlineto{\pgfqpoint{8.132604in}{1.789455in}}%
\pgfpathlineto{\pgfqpoint{8.129432in}{1.786253in}}%
\pgfpathlineto{\pgfqpoint{8.126260in}{1.783286in}}%
\pgfpathlineto{\pgfqpoint{8.123088in}{1.780080in}}%
\pgfpathlineto{\pgfqpoint{8.119916in}{1.776910in}}%
\pgfpathlineto{\pgfqpoint{8.116744in}{1.773712in}}%
\pgfpathlineto{\pgfqpoint{8.113572in}{1.770543in}}%
\pgfpathlineto{\pgfqpoint{8.110400in}{1.767392in}}%
\pgfpathlineto{\pgfqpoint{8.107228in}{1.764207in}}%
\pgfpathlineto{\pgfqpoint{8.104056in}{1.761025in}}%
\pgfpathlineto{\pgfqpoint{8.100884in}{1.757878in}}%
\pgfpathlineto{\pgfqpoint{8.097712in}{1.754642in}}%
\pgfpathlineto{\pgfqpoint{8.094540in}{1.751534in}}%
\pgfpathlineto{\pgfqpoint{8.091368in}{1.748356in}}%
\pgfpathlineto{\pgfqpoint{8.088196in}{1.745167in}}%
\pgfpathlineto{\pgfqpoint{8.085024in}{1.741973in}}%
\pgfpathlineto{\pgfqpoint{8.081852in}{1.738793in}}%
\pgfpathlineto{\pgfqpoint{8.078680in}{1.735618in}}%
\pgfpathlineto{\pgfqpoint{8.075508in}{1.732429in}}%
\pgfpathlineto{\pgfqpoint{8.072336in}{1.729253in}}%
\pgfpathlineto{\pgfqpoint{8.069164in}{1.726095in}}%
\pgfpathlineto{\pgfqpoint{8.065992in}{1.722930in}}%
\pgfpathlineto{\pgfqpoint{8.062820in}{1.719741in}}%
\pgfpathlineto{\pgfqpoint{8.059648in}{1.716555in}}%
\pgfpathlineto{\pgfqpoint{8.056475in}{1.713323in}}%
\pgfpathlineto{\pgfqpoint{8.053303in}{1.710134in}}%
\pgfpathlineto{\pgfqpoint{8.050131in}{1.706952in}}%
\pgfpathlineto{\pgfqpoint{8.046959in}{1.703783in}}%
\pgfpathlineto{\pgfqpoint{8.043787in}{1.700654in}}%
\pgfpathlineto{\pgfqpoint{8.040615in}{1.697509in}}%
\pgfpathlineto{\pgfqpoint{8.037443in}{1.694343in}}%
\pgfpathlineto{\pgfqpoint{8.034271in}{1.691042in}}%
\pgfpathlineto{\pgfqpoint{8.031099in}{1.687861in}}%
\pgfpathlineto{\pgfqpoint{8.027927in}{1.684709in}}%
\pgfpathlineto{\pgfqpoint{8.024755in}{1.681510in}}%
\pgfpathlineto{\pgfqpoint{8.021583in}{1.678294in}}%
\pgfpathlineto{\pgfqpoint{8.018411in}{1.675135in}}%
\pgfpathlineto{\pgfqpoint{8.015239in}{1.671639in}}%
\pgfpathlineto{\pgfqpoint{8.012067in}{1.668298in}}%
\pgfpathlineto{\pgfqpoint{8.008895in}{1.665120in}}%
\pgfpathlineto{\pgfqpoint{8.005723in}{1.661890in}}%
\pgfpathlineto{\pgfqpoint{8.002551in}{1.658733in}}%
\pgfpathlineto{\pgfqpoint{7.999379in}{1.655445in}}%
\pgfpathlineto{\pgfqpoint{7.996207in}{1.652327in}}%
\pgfpathlineto{\pgfqpoint{7.993035in}{1.649212in}}%
\pgfpathlineto{\pgfqpoint{7.989863in}{1.646060in}}%
\pgfpathlineto{\pgfqpoint{7.986691in}{1.642966in}}%
\pgfpathlineto{\pgfqpoint{7.983519in}{1.639783in}}%
\pgfpathlineto{\pgfqpoint{7.980347in}{1.636618in}}%
\pgfpathlineto{\pgfqpoint{7.977174in}{1.633445in}}%
\pgfpathlineto{\pgfqpoint{7.974002in}{1.630267in}}%
\pgfpathlineto{\pgfqpoint{7.970830in}{1.627101in}}%
\pgfpathlineto{\pgfqpoint{7.967658in}{1.623871in}}%
\pgfpathlineto{\pgfqpoint{7.964486in}{1.620683in}}%
\pgfpathlineto{\pgfqpoint{7.961314in}{1.617486in}}%
\pgfpathlineto{\pgfqpoint{7.958142in}{1.614324in}}%
\pgfpathlineto{\pgfqpoint{7.954970in}{1.611128in}}%
\pgfpathlineto{\pgfqpoint{7.951798in}{1.607980in}}%
\pgfpathlineto{\pgfqpoint{7.948626in}{1.604817in}}%
\pgfpathlineto{\pgfqpoint{7.945454in}{1.601616in}}%
\pgfpathlineto{\pgfqpoint{7.942282in}{1.598447in}}%
\pgfpathlineto{\pgfqpoint{7.939110in}{1.595265in}}%
\pgfpathlineto{\pgfqpoint{7.935938in}{1.592118in}}%
\pgfpathlineto{\pgfqpoint{7.932766in}{1.588948in}}%
\pgfpathlineto{\pgfqpoint{7.929594in}{1.585806in}}%
\pgfpathlineto{\pgfqpoint{7.926422in}{1.582620in}}%
\pgfpathlineto{\pgfqpoint{7.923250in}{1.579489in}}%
\pgfpathlineto{\pgfqpoint{7.920078in}{1.576349in}}%
\pgfpathlineto{\pgfqpoint{7.916906in}{1.573173in}}%
\pgfpathlineto{\pgfqpoint{7.913734in}{1.570015in}}%
\pgfpathlineto{\pgfqpoint{7.910562in}{1.566863in}}%
\pgfpathlineto{\pgfqpoint{7.907390in}{1.563709in}}%
\pgfpathlineto{\pgfqpoint{7.904218in}{1.560529in}}%
\pgfpathlineto{\pgfqpoint{7.901046in}{1.557351in}}%
\pgfpathlineto{\pgfqpoint{7.897873in}{1.554155in}}%
\pgfpathlineto{\pgfqpoint{7.894701in}{1.551511in}}%
\pgfpathlineto{\pgfqpoint{7.891529in}{1.548304in}}%
\pgfpathlineto{\pgfqpoint{7.888357in}{1.545080in}}%
\pgfpathlineto{\pgfqpoint{7.885185in}{1.541451in}}%
\pgfpathlineto{\pgfqpoint{7.882013in}{1.537624in}}%
\pgfpathlineto{\pgfqpoint{7.878841in}{1.533760in}}%
\pgfpathlineto{\pgfqpoint{7.875669in}{1.529867in}}%
\pgfpathlineto{\pgfqpoint{7.872497in}{1.526009in}}%
\pgfpathlineto{\pgfqpoint{7.869325in}{1.522104in}}%
\pgfpathlineto{\pgfqpoint{7.866153in}{1.518244in}}%
\pgfpathlineto{\pgfqpoint{7.862981in}{1.514288in}}%
\pgfpathlineto{\pgfqpoint{7.859809in}{1.510380in}}%
\pgfpathlineto{\pgfqpoint{7.856637in}{1.506422in}}%
\pgfpathlineto{\pgfqpoint{7.853465in}{1.502537in}}%
\pgfpathlineto{\pgfqpoint{7.850293in}{1.498632in}}%
\pgfpathlineto{\pgfqpoint{7.847121in}{1.494752in}}%
\pgfpathlineto{\pgfqpoint{7.843949in}{1.490861in}}%
\pgfpathlineto{\pgfqpoint{7.840777in}{1.486959in}}%
\pgfpathlineto{\pgfqpoint{7.837605in}{1.483090in}}%
\pgfpathlineto{\pgfqpoint{7.834433in}{1.479179in}}%
\pgfpathlineto{\pgfqpoint{7.831261in}{1.475315in}}%
\pgfpathlineto{\pgfqpoint{7.828089in}{1.471508in}}%
\pgfpathlineto{\pgfqpoint{7.824917in}{1.467654in}}%
\pgfpathlineto{\pgfqpoint{7.821744in}{1.463728in}}%
\pgfpathlineto{\pgfqpoint{7.818572in}{1.459821in}}%
\pgfpathlineto{\pgfqpoint{7.815400in}{1.455928in}}%
\pgfpathlineto{\pgfqpoint{7.812228in}{1.452040in}}%
\pgfpathlineto{\pgfqpoint{7.809056in}{1.448185in}}%
\pgfpathlineto{\pgfqpoint{7.805884in}{1.444294in}}%
\pgfpathlineto{\pgfqpoint{7.802712in}{1.440339in}}%
\pgfpathlineto{\pgfqpoint{7.799540in}{1.436473in}}%
\pgfpathlineto{\pgfqpoint{7.796368in}{1.432592in}}%
\pgfpathlineto{\pgfqpoint{7.793196in}{1.428688in}}%
\pgfpathlineto{\pgfqpoint{7.790024in}{1.424823in}}%
\pgfpathlineto{\pgfqpoint{7.786852in}{1.420866in}}%
\pgfpathlineto{\pgfqpoint{7.783680in}{1.420716in}}%
\pgfpathlineto{\pgfqpoint{7.780508in}{1.421013in}}%
\pgfpathlineto{\pgfqpoint{7.777336in}{1.420577in}}%
\pgfpathlineto{\pgfqpoint{7.774164in}{1.420614in}}%
\pgfpathlineto{\pgfqpoint{7.770992in}{1.420488in}}%
\pgfpathlineto{\pgfqpoint{7.767820in}{1.420622in}}%
\pgfpathlineto{\pgfqpoint{7.764648in}{1.420780in}}%
\pgfpathlineto{\pgfqpoint{7.761476in}{1.420648in}}%
\pgfpathlineto{\pgfqpoint{7.758304in}{1.420300in}}%
\pgfpathlineto{\pgfqpoint{7.755132in}{1.420012in}}%
\pgfpathlineto{\pgfqpoint{7.751960in}{1.420117in}}%
\pgfpathlineto{\pgfqpoint{7.748788in}{1.420874in}}%
\pgfpathlineto{\pgfqpoint{7.745616in}{1.421140in}}%
\pgfpathlineto{\pgfqpoint{7.742443in}{1.420891in}}%
\pgfpathlineto{\pgfqpoint{7.739271in}{1.421025in}}%
\pgfpathlineto{\pgfqpoint{7.736099in}{1.421413in}}%
\pgfpathlineto{\pgfqpoint{7.732927in}{1.421563in}}%
\pgfpathlineto{\pgfqpoint{7.729755in}{1.421482in}}%
\pgfpathlineto{\pgfqpoint{7.726583in}{1.421726in}}%
\pgfpathlineto{\pgfqpoint{7.723411in}{1.421556in}}%
\pgfpathlineto{\pgfqpoint{7.720239in}{1.421442in}}%
\pgfpathlineto{\pgfqpoint{7.717067in}{1.421580in}}%
\pgfpathlineto{\pgfqpoint{7.713895in}{1.421604in}}%
\pgfpathlineto{\pgfqpoint{7.710723in}{1.421207in}}%
\pgfpathlineto{\pgfqpoint{7.707551in}{1.421178in}}%
\pgfpathlineto{\pgfqpoint{7.704379in}{1.421199in}}%
\pgfpathlineto{\pgfqpoint{7.701207in}{1.420933in}}%
\pgfpathlineto{\pgfqpoint{7.698035in}{1.420785in}}%
\pgfpathlineto{\pgfqpoint{7.694863in}{1.420581in}}%
\pgfpathlineto{\pgfqpoint{7.691691in}{1.420458in}}%
\pgfpathlineto{\pgfqpoint{7.688519in}{1.420164in}}%
\pgfpathlineto{\pgfqpoint{7.685347in}{1.419697in}}%
\pgfpathlineto{\pgfqpoint{7.682175in}{1.419581in}}%
\pgfpathlineto{\pgfqpoint{7.679003in}{1.419269in}}%
\pgfpathlineto{\pgfqpoint{7.675831in}{1.419040in}}%
\pgfpathlineto{\pgfqpoint{7.672659in}{1.418757in}}%
\pgfpathlineto{\pgfqpoint{7.669487in}{1.418443in}}%
\pgfpathlineto{\pgfqpoint{7.666315in}{1.418161in}}%
\pgfpathlineto{\pgfqpoint{7.663142in}{1.418300in}}%
\pgfpathlineto{\pgfqpoint{7.659970in}{1.418246in}}%
\pgfpathlineto{\pgfqpoint{7.656798in}{1.418330in}}%
\pgfpathlineto{\pgfqpoint{7.653626in}{1.417382in}}%
\pgfpathlineto{\pgfqpoint{7.650454in}{1.417420in}}%
\pgfpathlineto{\pgfqpoint{7.647282in}{1.417275in}}%
\pgfpathlineto{\pgfqpoint{7.644110in}{1.417176in}}%
\pgfpathlineto{\pgfqpoint{7.640938in}{1.417263in}}%
\pgfpathlineto{\pgfqpoint{7.637766in}{1.416990in}}%
\pgfpathlineto{\pgfqpoint{7.634594in}{1.417075in}}%
\pgfpathlineto{\pgfqpoint{7.631422in}{1.417146in}}%
\pgfpathlineto{\pgfqpoint{7.628250in}{1.416831in}}%
\pgfpathlineto{\pgfqpoint{7.625078in}{1.416545in}}%
\pgfpathlineto{\pgfqpoint{7.621906in}{1.415802in}}%
\pgfpathlineto{\pgfqpoint{7.618734in}{1.415752in}}%
\pgfpathlineto{\pgfqpoint{7.615562in}{1.416038in}}%
\pgfpathlineto{\pgfqpoint{7.612390in}{1.416428in}}%
\pgfpathlineto{\pgfqpoint{7.609218in}{1.416293in}}%
\pgfpathlineto{\pgfqpoint{7.606046in}{1.415845in}}%
\pgfpathlineto{\pgfqpoint{7.602874in}{1.415749in}}%
\pgfpathlineto{\pgfqpoint{7.599702in}{1.415652in}}%
\pgfpathlineto{\pgfqpoint{7.596530in}{1.415849in}}%
\pgfpathlineto{\pgfqpoint{7.593358in}{1.415677in}}%
\pgfpathlineto{\pgfqpoint{7.590186in}{1.415662in}}%
\pgfpathlineto{\pgfqpoint{7.587013in}{1.415376in}}%
\pgfpathlineto{\pgfqpoint{7.583841in}{1.415642in}}%
\pgfpathlineto{\pgfqpoint{7.580669in}{1.415458in}}%
\pgfpathlineto{\pgfqpoint{7.577497in}{1.415152in}}%
\pgfpathlineto{\pgfqpoint{7.574325in}{1.414913in}}%
\pgfpathlineto{\pgfqpoint{7.571153in}{1.414604in}}%
\pgfpathlineto{\pgfqpoint{7.567981in}{1.414442in}}%
\pgfpathlineto{\pgfqpoint{7.564809in}{1.414222in}}%
\pgfpathlineto{\pgfqpoint{7.561637in}{1.414163in}}%
\pgfpathlineto{\pgfqpoint{7.558465in}{1.414554in}}%
\pgfpathlineto{\pgfqpoint{7.555293in}{1.414443in}}%
\pgfpathlineto{\pgfqpoint{7.552121in}{1.414192in}}%
\pgfpathlineto{\pgfqpoint{7.548949in}{1.414082in}}%
\pgfpathlineto{\pgfqpoint{7.545777in}{1.414084in}}%
\pgfpathlineto{\pgfqpoint{7.542605in}{1.413550in}}%
\pgfpathlineto{\pgfqpoint{7.539433in}{1.413400in}}%
\pgfpathlineto{\pgfqpoint{7.536261in}{1.413697in}}%
\pgfpathlineto{\pgfqpoint{7.533089in}{1.413079in}}%
\pgfpathlineto{\pgfqpoint{7.529917in}{1.413343in}}%
\pgfpathlineto{\pgfqpoint{7.526745in}{1.413057in}}%
\pgfpathlineto{\pgfqpoint{7.523573in}{1.413137in}}%
\pgfpathlineto{\pgfqpoint{7.520401in}{1.413208in}}%
\pgfpathlineto{\pgfqpoint{7.517229in}{1.413542in}}%
\pgfpathlineto{\pgfqpoint{7.514057in}{1.413517in}}%
\pgfpathlineto{\pgfqpoint{7.510885in}{1.413361in}}%
\pgfpathlineto{\pgfqpoint{7.507712in}{1.413525in}}%
\pgfpathlineto{\pgfqpoint{7.504540in}{1.413426in}}%
\pgfpathlineto{\pgfqpoint{7.501368in}{1.413102in}}%
\pgfpathlineto{\pgfqpoint{7.498196in}{1.413293in}}%
\pgfpathlineto{\pgfqpoint{7.495024in}{1.413325in}}%
\pgfpathlineto{\pgfqpoint{7.491852in}{1.413076in}}%
\pgfpathlineto{\pgfqpoint{7.488680in}{1.412514in}}%
\pgfpathlineto{\pgfqpoint{7.485508in}{1.412086in}}%
\pgfpathlineto{\pgfqpoint{7.482336in}{1.412404in}}%
\pgfpathlineto{\pgfqpoint{7.479164in}{1.412010in}}%
\pgfpathlineto{\pgfqpoint{7.475992in}{1.411443in}}%
\pgfpathlineto{\pgfqpoint{7.472820in}{1.410985in}}%
\pgfpathlineto{\pgfqpoint{7.469648in}{1.410971in}}%
\pgfpathlineto{\pgfqpoint{7.466476in}{1.410895in}}%
\pgfpathlineto{\pgfqpoint{7.463304in}{1.410428in}}%
\pgfpathlineto{\pgfqpoint{7.460132in}{1.410413in}}%
\pgfpathlineto{\pgfqpoint{7.456960in}{1.410010in}}%
\pgfpathlineto{\pgfqpoint{7.453788in}{1.410262in}}%
\pgfpathlineto{\pgfqpoint{7.450616in}{1.410032in}}%
\pgfpathlineto{\pgfqpoint{7.447444in}{1.409620in}}%
\pgfpathlineto{\pgfqpoint{7.444272in}{1.409527in}}%
\pgfpathlineto{\pgfqpoint{7.441100in}{1.409028in}}%
\pgfpathlineto{\pgfqpoint{7.437928in}{1.408989in}}%
\pgfpathlineto{\pgfqpoint{7.434756in}{1.408956in}}%
\pgfpathlineto{\pgfqpoint{7.431584in}{1.408758in}}%
\pgfpathlineto{\pgfqpoint{7.428411in}{1.408537in}}%
\pgfpathlineto{\pgfqpoint{7.425239in}{1.408608in}}%
\pgfpathlineto{\pgfqpoint{7.422067in}{1.408558in}}%
\pgfpathlineto{\pgfqpoint{7.418895in}{1.408479in}}%
\pgfpathlineto{\pgfqpoint{7.415723in}{1.408309in}}%
\pgfpathlineto{\pgfqpoint{7.412551in}{1.407798in}}%
\pgfpathlineto{\pgfqpoint{7.409379in}{1.407641in}}%
\pgfpathlineto{\pgfqpoint{7.406207in}{1.407889in}}%
\pgfpathlineto{\pgfqpoint{7.403035in}{1.408084in}}%
\pgfpathlineto{\pgfqpoint{7.399863in}{1.408229in}}%
\pgfpathlineto{\pgfqpoint{7.396691in}{1.408062in}}%
\pgfpathlineto{\pgfqpoint{7.393519in}{1.408008in}}%
\pgfpathlineto{\pgfqpoint{7.390347in}{1.408168in}}%
\pgfpathlineto{\pgfqpoint{7.387175in}{1.408346in}}%
\pgfpathlineto{\pgfqpoint{7.384003in}{1.408106in}}%
\pgfpathlineto{\pgfqpoint{7.380831in}{1.408448in}}%
\pgfpathlineto{\pgfqpoint{7.377659in}{1.408396in}}%
\pgfpathlineto{\pgfqpoint{7.374487in}{1.408247in}}%
\pgfpathlineto{\pgfqpoint{7.371315in}{1.407739in}}%
\pgfpathlineto{\pgfqpoint{7.368143in}{1.407496in}}%
\pgfpathlineto{\pgfqpoint{7.364971in}{1.407284in}}%
\pgfpathlineto{\pgfqpoint{7.361799in}{1.406475in}}%
\pgfpathlineto{\pgfqpoint{7.358627in}{1.406272in}}%
\pgfpathlineto{\pgfqpoint{7.355455in}{1.406513in}}%
\pgfpathlineto{\pgfqpoint{7.352282in}{1.406303in}}%
\pgfpathlineto{\pgfqpoint{7.349110in}{1.406454in}}%
\pgfpathlineto{\pgfqpoint{7.345938in}{1.406278in}}%
\pgfpathlineto{\pgfqpoint{7.342766in}{1.405995in}}%
\pgfpathlineto{\pgfqpoint{7.339594in}{1.406191in}}%
\pgfpathlineto{\pgfqpoint{7.336422in}{1.406110in}}%
\pgfpathlineto{\pgfqpoint{7.333250in}{1.406182in}}%
\pgfpathlineto{\pgfqpoint{7.330078in}{1.406354in}}%
\pgfpathlineto{\pgfqpoint{7.326906in}{1.406576in}}%
\pgfpathlineto{\pgfqpoint{7.323734in}{1.406422in}}%
\pgfpathlineto{\pgfqpoint{7.320562in}{1.406158in}}%
\pgfpathlineto{\pgfqpoint{7.317390in}{1.405546in}}%
\pgfpathlineto{\pgfqpoint{7.314218in}{1.404866in}}%
\pgfpathlineto{\pgfqpoint{7.311046in}{1.404951in}}%
\pgfpathlineto{\pgfqpoint{7.307874in}{1.404201in}}%
\pgfpathlineto{\pgfqpoint{7.304702in}{1.404523in}}%
\pgfpathlineto{\pgfqpoint{7.301530in}{1.404260in}}%
\pgfpathlineto{\pgfqpoint{7.298358in}{1.404153in}}%
\pgfpathlineto{\pgfqpoint{7.295186in}{1.403949in}}%
\pgfpathlineto{\pgfqpoint{7.292014in}{1.403843in}}%
\pgfpathlineto{\pgfqpoint{7.288842in}{1.403385in}}%
\pgfpathlineto{\pgfqpoint{7.285670in}{1.402783in}}%
\pgfpathlineto{\pgfqpoint{7.282498in}{1.402900in}}%
\pgfpathlineto{\pgfqpoint{7.279326in}{1.402697in}}%
\pgfpathlineto{\pgfqpoint{7.276154in}{1.402929in}}%
\pgfpathlineto{\pgfqpoint{7.272981in}{1.402939in}}%
\pgfpathlineto{\pgfqpoint{7.269809in}{1.402728in}}%
\pgfpathlineto{\pgfqpoint{7.266637in}{1.402409in}}%
\pgfpathlineto{\pgfqpoint{7.263465in}{1.401956in}}%
\pgfpathlineto{\pgfqpoint{7.260293in}{1.401726in}}%
\pgfpathlineto{\pgfqpoint{7.257121in}{1.401687in}}%
\pgfpathlineto{\pgfqpoint{7.253949in}{1.401628in}}%
\pgfpathlineto{\pgfqpoint{7.250777in}{1.401443in}}%
\pgfpathlineto{\pgfqpoint{7.247605in}{1.401267in}}%
\pgfpathlineto{\pgfqpoint{7.244433in}{1.401656in}}%
\pgfpathlineto{\pgfqpoint{7.241261in}{1.401744in}}%
\pgfpathlineto{\pgfqpoint{7.238089in}{1.401441in}}%
\pgfpathlineto{\pgfqpoint{7.234917in}{1.401396in}}%
\pgfpathlineto{\pgfqpoint{7.231745in}{1.401603in}}%
\pgfpathlineto{\pgfqpoint{7.228573in}{1.401484in}}%
\pgfpathlineto{\pgfqpoint{7.225401in}{1.401853in}}%
\pgfpathlineto{\pgfqpoint{7.222229in}{1.401568in}}%
\pgfpathlineto{\pgfqpoint{7.219057in}{1.401923in}}%
\pgfpathlineto{\pgfqpoint{7.215885in}{1.401946in}}%
\pgfpathlineto{\pgfqpoint{7.212713in}{1.402281in}}%
\pgfpathlineto{\pgfqpoint{7.209541in}{1.402398in}}%
\pgfpathlineto{\pgfqpoint{7.206369in}{1.402212in}}%
\pgfpathlineto{\pgfqpoint{7.203197in}{1.401950in}}%
\pgfpathlineto{\pgfqpoint{7.200025in}{1.401588in}}%
\pgfpathlineto{\pgfqpoint{7.196853in}{1.401113in}}%
\pgfpathlineto{\pgfqpoint{7.193680in}{1.401333in}}%
\pgfpathlineto{\pgfqpoint{7.190508in}{1.401014in}}%
\pgfpathlineto{\pgfqpoint{7.187336in}{1.400576in}}%
\pgfpathlineto{\pgfqpoint{7.184164in}{1.400282in}}%
\pgfpathlineto{\pgfqpoint{7.180992in}{1.400016in}}%
\pgfpathlineto{\pgfqpoint{7.177820in}{1.399981in}}%
\pgfpathlineto{\pgfqpoint{7.174648in}{1.399937in}}%
\pgfpathlineto{\pgfqpoint{7.171476in}{1.399490in}}%
\pgfpathlineto{\pgfqpoint{7.168304in}{1.399411in}}%
\pgfpathlineto{\pgfqpoint{7.165132in}{1.399486in}}%
\pgfpathlineto{\pgfqpoint{7.161960in}{1.399568in}}%
\pgfpathlineto{\pgfqpoint{7.158788in}{1.399885in}}%
\pgfpathlineto{\pgfqpoint{7.155616in}{1.399637in}}%
\pgfpathlineto{\pgfqpoint{7.152444in}{1.399889in}}%
\pgfpathlineto{\pgfqpoint{7.149272in}{1.399564in}}%
\pgfpathlineto{\pgfqpoint{7.146100in}{1.399784in}}%
\pgfpathlineto{\pgfqpoint{7.142928in}{1.399971in}}%
\pgfpathlineto{\pgfqpoint{7.139756in}{1.400049in}}%
\pgfpathlineto{\pgfqpoint{7.136584in}{1.400124in}}%
\pgfpathlineto{\pgfqpoint{7.133412in}{1.400109in}}%
\pgfpathlineto{\pgfqpoint{7.130240in}{1.400113in}}%
\pgfpathlineto{\pgfqpoint{7.127068in}{1.399662in}}%
\pgfpathlineto{\pgfqpoint{7.123896in}{1.399360in}}%
\pgfpathlineto{\pgfqpoint{7.120724in}{1.398838in}}%
\pgfpathlineto{\pgfqpoint{7.117551in}{1.399083in}}%
\pgfpathlineto{\pgfqpoint{7.114379in}{1.398937in}}%
\pgfpathlineto{\pgfqpoint{7.111207in}{1.398659in}}%
\pgfpathlineto{\pgfqpoint{7.108035in}{1.398839in}}%
\pgfpathlineto{\pgfqpoint{7.104863in}{1.398771in}}%
\pgfpathlineto{\pgfqpoint{7.101691in}{1.398720in}}%
\pgfpathlineto{\pgfqpoint{7.098519in}{1.398407in}}%
\pgfpathlineto{\pgfqpoint{7.095347in}{1.398474in}}%
\pgfpathlineto{\pgfqpoint{7.092175in}{1.398599in}}%
\pgfpathlineto{\pgfqpoint{7.089003in}{1.398576in}}%
\pgfpathlineto{\pgfqpoint{7.085831in}{1.398847in}}%
\pgfpathlineto{\pgfqpoint{7.082659in}{1.398735in}}%
\pgfpathlineto{\pgfqpoint{7.079487in}{1.398756in}}%
\pgfpathlineto{\pgfqpoint{7.076315in}{1.398462in}}%
\pgfpathlineto{\pgfqpoint{7.073143in}{1.397965in}}%
\pgfpathlineto{\pgfqpoint{7.069971in}{1.397645in}}%
\pgfpathlineto{\pgfqpoint{7.066799in}{1.397781in}}%
\pgfpathlineto{\pgfqpoint{7.063627in}{1.397509in}}%
\pgfpathlineto{\pgfqpoint{7.060455in}{1.397925in}}%
\pgfpathlineto{\pgfqpoint{7.057283in}{1.398202in}}%
\pgfpathlineto{\pgfqpoint{7.054111in}{1.397984in}}%
\pgfpathlineto{\pgfqpoint{7.050939in}{1.398394in}}%
\pgfpathlineto{\pgfqpoint{7.047767in}{1.398488in}}%
\pgfpathlineto{\pgfqpoint{7.044595in}{1.398402in}}%
\pgfpathlineto{\pgfqpoint{7.041423in}{1.397882in}}%
\pgfpathlineto{\pgfqpoint{7.038250in}{1.398217in}}%
\pgfpathlineto{\pgfqpoint{7.035078in}{1.398203in}}%
\pgfpathlineto{\pgfqpoint{7.031906in}{1.398004in}}%
\pgfpathlineto{\pgfqpoint{7.028734in}{1.397754in}}%
\pgfpathlineto{\pgfqpoint{7.025562in}{1.397767in}}%
\pgfpathlineto{\pgfqpoint{7.022390in}{1.397900in}}%
\pgfpathlineto{\pgfqpoint{7.019218in}{1.397990in}}%
\pgfpathlineto{\pgfqpoint{7.016046in}{1.397969in}}%
\pgfpathlineto{\pgfqpoint{7.012874in}{1.398249in}}%
\pgfpathlineto{\pgfqpoint{7.009702in}{1.397996in}}%
\pgfpathlineto{\pgfqpoint{7.006530in}{1.397957in}}%
\pgfpathlineto{\pgfqpoint{7.003358in}{1.397695in}}%
\pgfpathlineto{\pgfqpoint{7.000186in}{1.397799in}}%
\pgfpathlineto{\pgfqpoint{6.997014in}{1.397698in}}%
\pgfpathlineto{\pgfqpoint{6.993842in}{1.398037in}}%
\pgfpathlineto{\pgfqpoint{6.990670in}{1.398330in}}%
\pgfpathlineto{\pgfqpoint{6.987498in}{1.398378in}}%
\pgfpathlineto{\pgfqpoint{6.984326in}{1.398111in}}%
\pgfpathlineto{\pgfqpoint{6.981154in}{1.397737in}}%
\pgfpathlineto{\pgfqpoint{6.977982in}{1.397704in}}%
\pgfpathlineto{\pgfqpoint{6.974810in}{1.397612in}}%
\pgfpathlineto{\pgfqpoint{6.971638in}{1.397368in}}%
\pgfpathlineto{\pgfqpoint{6.968466in}{1.397593in}}%
\pgfpathlineto{\pgfqpoint{6.965294in}{1.397776in}}%
\pgfpathlineto{\pgfqpoint{6.962122in}{1.397220in}}%
\pgfpathlineto{\pgfqpoint{6.958949in}{1.396556in}}%
\pgfpathlineto{\pgfqpoint{6.955777in}{1.396203in}}%
\pgfpathlineto{\pgfqpoint{6.952605in}{1.396122in}}%
\pgfpathlineto{\pgfqpoint{6.949433in}{1.396491in}}%
\pgfpathlineto{\pgfqpoint{6.946261in}{1.396387in}}%
\pgfpathlineto{\pgfqpoint{6.943089in}{1.396216in}}%
\pgfpathlineto{\pgfqpoint{6.939917in}{1.395988in}}%
\pgfpathlineto{\pgfqpoint{6.936745in}{1.396128in}}%
\pgfpathlineto{\pgfqpoint{6.933573in}{1.396128in}}%
\pgfpathlineto{\pgfqpoint{6.930401in}{1.395981in}}%
\pgfpathlineto{\pgfqpoint{6.927229in}{1.395886in}}%
\pgfpathlineto{\pgfqpoint{6.924057in}{1.395667in}}%
\pgfpathlineto{\pgfqpoint{6.920885in}{1.395755in}}%
\pgfpathlineto{\pgfqpoint{6.917713in}{1.395174in}}%
\pgfpathlineto{\pgfqpoint{6.914541in}{1.395310in}}%
\pgfpathlineto{\pgfqpoint{6.911369in}{1.395077in}}%
\pgfpathlineto{\pgfqpoint{6.908197in}{1.395122in}}%
\pgfpathlineto{\pgfqpoint{6.905025in}{1.394451in}}%
\pgfpathlineto{\pgfqpoint{6.901853in}{1.394244in}}%
\pgfpathlineto{\pgfqpoint{6.898681in}{1.393914in}}%
\pgfpathlineto{\pgfqpoint{6.895509in}{1.393550in}}%
\pgfpathlineto{\pgfqpoint{6.892337in}{1.393411in}}%
\pgfpathlineto{\pgfqpoint{6.889165in}{1.393020in}}%
\pgfpathlineto{\pgfqpoint{6.885993in}{1.392974in}}%
\pgfpathlineto{\pgfqpoint{6.882820in}{1.392719in}}%
\pgfpathlineto{\pgfqpoint{6.879648in}{1.392907in}}%
\pgfpathlineto{\pgfqpoint{6.876476in}{1.392311in}}%
\pgfpathlineto{\pgfqpoint{6.873304in}{1.392050in}}%
\pgfpathlineto{\pgfqpoint{6.870132in}{1.391880in}}%
\pgfpathlineto{\pgfqpoint{6.866960in}{1.391456in}}%
\pgfpathlineto{\pgfqpoint{6.863788in}{1.391508in}}%
\pgfpathlineto{\pgfqpoint{6.860616in}{1.391318in}}%
\pgfpathlineto{\pgfqpoint{6.857444in}{1.391261in}}%
\pgfpathlineto{\pgfqpoint{6.854272in}{1.391236in}}%
\pgfpathlineto{\pgfqpoint{6.851100in}{1.390879in}}%
\pgfpathlineto{\pgfqpoint{6.847928in}{1.390501in}}%
\pgfpathlineto{\pgfqpoint{6.844756in}{1.390377in}}%
\pgfpathlineto{\pgfqpoint{6.841584in}{1.389952in}}%
\pgfpathlineto{\pgfqpoint{6.838412in}{1.390254in}}%
\pgfpathlineto{\pgfqpoint{6.835240in}{1.389525in}}%
\pgfpathlineto{\pgfqpoint{6.832068in}{1.389836in}}%
\pgfpathlineto{\pgfqpoint{6.828896in}{1.389677in}}%
\pgfpathlineto{\pgfqpoint{6.825724in}{1.389256in}}%
\pgfpathlineto{\pgfqpoint{6.822552in}{1.389106in}}%
\pgfpathlineto{\pgfqpoint{6.819380in}{1.389118in}}%
\pgfpathlineto{\pgfqpoint{6.816208in}{1.389455in}}%
\pgfpathlineto{\pgfqpoint{6.813036in}{1.389189in}}%
\pgfpathlineto{\pgfqpoint{6.809864in}{1.389166in}}%
\pgfpathlineto{\pgfqpoint{6.806692in}{1.388826in}}%
\pgfpathlineto{\pgfqpoint{6.803519in}{1.388796in}}%
\pgfpathlineto{\pgfqpoint{6.800347in}{1.388765in}}%
\pgfpathlineto{\pgfqpoint{6.797175in}{1.388652in}}%
\pgfpathlineto{\pgfqpoint{6.794003in}{1.388316in}}%
\pgfpathlineto{\pgfqpoint{6.790831in}{1.388603in}}%
\pgfpathlineto{\pgfqpoint{6.787659in}{1.388452in}}%
\pgfpathlineto{\pgfqpoint{6.784487in}{1.387821in}}%
\pgfpathlineto{\pgfqpoint{6.781315in}{1.387934in}}%
\pgfpathlineto{\pgfqpoint{6.778143in}{1.387712in}}%
\pgfpathlineto{\pgfqpoint{6.774971in}{1.388105in}}%
\pgfpathlineto{\pgfqpoint{6.771799in}{1.388144in}}%
\pgfpathlineto{\pgfqpoint{6.768627in}{1.387968in}}%
\pgfpathlineto{\pgfqpoint{6.765455in}{1.388040in}}%
\pgfpathlineto{\pgfqpoint{6.762283in}{1.387521in}}%
\pgfpathlineto{\pgfqpoint{6.759111in}{1.387211in}}%
\pgfpathlineto{\pgfqpoint{6.755939in}{1.387362in}}%
\pgfpathlineto{\pgfqpoint{6.752767in}{1.387254in}}%
\pgfpathlineto{\pgfqpoint{6.749595in}{1.387127in}}%
\pgfpathlineto{\pgfqpoint{6.746423in}{1.387317in}}%
\pgfpathlineto{\pgfqpoint{6.743251in}{1.387586in}}%
\pgfpathlineto{\pgfqpoint{6.740079in}{1.387796in}}%
\pgfpathlineto{\pgfqpoint{6.736907in}{1.388152in}}%
\pgfpathlineto{\pgfqpoint{6.733735in}{1.387863in}}%
\pgfpathlineto{\pgfqpoint{6.730563in}{1.388137in}}%
\pgfpathlineto{\pgfqpoint{6.727391in}{1.387888in}}%
\pgfpathlineto{\pgfqpoint{6.724218in}{1.388167in}}%
\pgfpathlineto{\pgfqpoint{6.721046in}{1.388197in}}%
\pgfpathlineto{\pgfqpoint{6.717874in}{1.387945in}}%
\pgfpathlineto{\pgfqpoint{6.714702in}{1.387757in}}%
\pgfpathlineto{\pgfqpoint{6.711530in}{1.386554in}}%
\pgfpathlineto{\pgfqpoint{6.708358in}{1.386510in}}%
\pgfpathlineto{\pgfqpoint{6.705186in}{1.386542in}}%
\pgfpathlineto{\pgfqpoint{6.702014in}{1.386606in}}%
\pgfpathlineto{\pgfqpoint{6.698842in}{1.386717in}}%
\pgfpathlineto{\pgfqpoint{6.695670in}{1.386776in}}%
\pgfpathlineto{\pgfqpoint{6.692498in}{1.385853in}}%
\pgfpathlineto{\pgfqpoint{6.689326in}{1.384798in}}%
\pgfpathlineto{\pgfqpoint{6.686154in}{1.384727in}}%
\pgfpathlineto{\pgfqpoint{6.682982in}{1.384622in}}%
\pgfpathlineto{\pgfqpoint{6.679810in}{1.384748in}}%
\pgfpathlineto{\pgfqpoint{6.676638in}{1.384744in}}%
\pgfpathlineto{\pgfqpoint{6.673466in}{1.384896in}}%
\pgfpathlineto{\pgfqpoint{6.670294in}{1.384376in}}%
\pgfpathlineto{\pgfqpoint{6.667122in}{1.383952in}}%
\pgfpathlineto{\pgfqpoint{6.663950in}{1.383964in}}%
\pgfpathlineto{\pgfqpoint{6.660778in}{1.384357in}}%
\pgfpathlineto{\pgfqpoint{6.657606in}{1.384789in}}%
\pgfpathlineto{\pgfqpoint{6.654434in}{1.384720in}}%
\pgfpathlineto{\pgfqpoint{6.651262in}{1.384843in}}%
\pgfpathlineto{\pgfqpoint{6.648089in}{1.384219in}}%
\pgfpathlineto{\pgfqpoint{6.644917in}{1.384529in}}%
\pgfpathlineto{\pgfqpoint{6.641745in}{1.384543in}}%
\pgfpathlineto{\pgfqpoint{6.638573in}{1.384760in}}%
\pgfpathlineto{\pgfqpoint{6.635401in}{1.384270in}}%
\pgfpathlineto{\pgfqpoint{6.632229in}{1.383516in}}%
\pgfpathlineto{\pgfqpoint{6.629057in}{1.383125in}}%
\pgfpathlineto{\pgfqpoint{6.625885in}{1.383418in}}%
\pgfpathlineto{\pgfqpoint{6.622713in}{1.382525in}}%
\pgfpathlineto{\pgfqpoint{6.619541in}{1.382214in}}%
\pgfpathlineto{\pgfqpoint{6.616369in}{1.382125in}}%
\pgfpathlineto{\pgfqpoint{6.613197in}{1.382158in}}%
\pgfpathlineto{\pgfqpoint{6.610025in}{1.382677in}}%
\pgfpathlineto{\pgfqpoint{6.606853in}{1.382552in}}%
\pgfpathlineto{\pgfqpoint{6.603681in}{1.382250in}}%
\pgfpathlineto{\pgfqpoint{6.600509in}{1.382294in}}%
\pgfpathlineto{\pgfqpoint{6.597337in}{1.382819in}}%
\pgfpathlineto{\pgfqpoint{6.594165in}{1.382853in}}%
\pgfpathlineto{\pgfqpoint{6.590993in}{1.382543in}}%
\pgfpathlineto{\pgfqpoint{6.587821in}{1.382535in}}%
\pgfpathlineto{\pgfqpoint{6.584649in}{1.382363in}}%
\pgfpathlineto{\pgfqpoint{6.581477in}{1.382016in}}%
\pgfpathlineto{\pgfqpoint{6.578305in}{1.382012in}}%
\pgfpathlineto{\pgfqpoint{6.575133in}{1.382048in}}%
\pgfpathlineto{\pgfqpoint{6.571961in}{1.381912in}}%
\pgfpathlineto{\pgfqpoint{6.568788in}{1.381900in}}%
\pgfpathlineto{\pgfqpoint{6.565616in}{1.381775in}}%
\pgfpathlineto{\pgfqpoint{6.562444in}{1.381658in}}%
\pgfpathlineto{\pgfqpoint{6.559272in}{1.381835in}}%
\pgfpathlineto{\pgfqpoint{6.556100in}{1.382012in}}%
\pgfpathlineto{\pgfqpoint{6.552928in}{1.382020in}}%
\pgfpathlineto{\pgfqpoint{6.549756in}{1.381865in}}%
\pgfpathlineto{\pgfqpoint{6.546584in}{1.381585in}}%
\pgfpathlineto{\pgfqpoint{6.543412in}{1.381876in}}%
\pgfpathlineto{\pgfqpoint{6.540240in}{1.381656in}}%
\pgfpathlineto{\pgfqpoint{6.537068in}{1.381960in}}%
\pgfpathlineto{\pgfqpoint{6.533896in}{1.382293in}}%
\pgfpathlineto{\pgfqpoint{6.530724in}{1.382445in}}%
\pgfpathlineto{\pgfqpoint{6.527552in}{1.382650in}}%
\pgfpathlineto{\pgfqpoint{6.524380in}{1.382783in}}%
\pgfpathlineto{\pgfqpoint{6.521208in}{1.382776in}}%
\pgfpathlineto{\pgfqpoint{6.518036in}{1.382468in}}%
\pgfpathlineto{\pgfqpoint{6.514864in}{1.382874in}}%
\pgfpathlineto{\pgfqpoint{6.511692in}{1.382815in}}%
\pgfpathlineto{\pgfqpoint{6.508520in}{1.382375in}}%
\pgfpathlineto{\pgfqpoint{6.505348in}{1.382460in}}%
\pgfpathlineto{\pgfqpoint{6.502176in}{1.382295in}}%
\pgfpathlineto{\pgfqpoint{6.499004in}{1.382494in}}%
\pgfpathlineto{\pgfqpoint{6.495832in}{1.382476in}}%
\pgfpathlineto{\pgfqpoint{6.492659in}{1.382548in}}%
\pgfpathlineto{\pgfqpoint{6.489487in}{1.382586in}}%
\pgfpathlineto{\pgfqpoint{6.486315in}{1.381970in}}%
\pgfpathlineto{\pgfqpoint{6.483143in}{1.381794in}}%
\pgfpathlineto{\pgfqpoint{6.479971in}{1.381412in}}%
\pgfpathlineto{\pgfqpoint{6.476799in}{1.381303in}}%
\pgfpathlineto{\pgfqpoint{6.473627in}{1.380916in}}%
\pgfpathlineto{\pgfqpoint{6.470455in}{1.380651in}}%
\pgfpathlineto{\pgfqpoint{6.467283in}{1.380705in}}%
\pgfpathlineto{\pgfqpoint{6.464111in}{1.380460in}}%
\pgfpathlineto{\pgfqpoint{6.460939in}{1.380468in}}%
\pgfpathlineto{\pgfqpoint{6.457767in}{1.380317in}}%
\pgfpathlineto{\pgfqpoint{6.454595in}{1.380056in}}%
\pgfpathlineto{\pgfqpoint{6.451423in}{1.380274in}}%
\pgfpathlineto{\pgfqpoint{6.448251in}{1.380588in}}%
\pgfpathlineto{\pgfqpoint{6.445079in}{1.380271in}}%
\pgfpathlineto{\pgfqpoint{6.441907in}{1.379950in}}%
\pgfpathlineto{\pgfqpoint{6.438735in}{1.380091in}}%
\pgfpathlineto{\pgfqpoint{6.435563in}{1.379441in}}%
\pgfpathlineto{\pgfqpoint{6.432391in}{1.379331in}}%
\pgfpathlineto{\pgfqpoint{6.429219in}{1.379226in}}%
\pgfpathlineto{\pgfqpoint{6.426047in}{1.379230in}}%
\pgfpathlineto{\pgfqpoint{6.422875in}{1.379045in}}%
\pgfpathlineto{\pgfqpoint{6.419703in}{1.379020in}}%
\pgfpathlineto{\pgfqpoint{6.416531in}{1.378767in}}%
\pgfpathlineto{\pgfqpoint{6.413358in}{1.379035in}}%
\pgfpathlineto{\pgfqpoint{6.410186in}{1.379252in}}%
\pgfpathlineto{\pgfqpoint{6.407014in}{1.379537in}}%
\pgfpathlineto{\pgfqpoint{6.403842in}{1.379032in}}%
\pgfpathlineto{\pgfqpoint{6.400670in}{1.378922in}}%
\pgfpathlineto{\pgfqpoint{6.397498in}{1.378914in}}%
\pgfpathlineto{\pgfqpoint{6.394326in}{1.378589in}}%
\pgfpathlineto{\pgfqpoint{6.391154in}{1.378407in}}%
\pgfpathlineto{\pgfqpoint{6.387982in}{1.378586in}}%
\pgfpathlineto{\pgfqpoint{6.384810in}{1.378502in}}%
\pgfpathlineto{\pgfqpoint{6.381638in}{1.378206in}}%
\pgfpathlineto{\pgfqpoint{6.378466in}{1.377411in}}%
\pgfpathlineto{\pgfqpoint{6.375294in}{1.377175in}}%
\pgfpathlineto{\pgfqpoint{6.372122in}{1.377213in}}%
\pgfpathlineto{\pgfqpoint{6.368950in}{1.377215in}}%
\pgfpathlineto{\pgfqpoint{6.365778in}{1.377020in}}%
\pgfpathlineto{\pgfqpoint{6.362606in}{1.376737in}}%
\pgfpathlineto{\pgfqpoint{6.359434in}{1.376833in}}%
\pgfpathlineto{\pgfqpoint{6.356262in}{1.377012in}}%
\pgfpathlineto{\pgfqpoint{6.353090in}{1.376839in}}%
\pgfpathlineto{\pgfqpoint{6.349918in}{1.376863in}}%
\pgfpathlineto{\pgfqpoint{6.346746in}{1.376956in}}%
\pgfpathlineto{\pgfqpoint{6.343574in}{1.376739in}}%
\pgfpathlineto{\pgfqpoint{6.340402in}{1.376463in}}%
\pgfpathlineto{\pgfqpoint{6.337230in}{1.376278in}}%
\pgfpathlineto{\pgfqpoint{6.334057in}{1.375541in}}%
\pgfpathlineto{\pgfqpoint{6.330885in}{1.375487in}}%
\pgfpathlineto{\pgfqpoint{6.327713in}{1.375217in}}%
\pgfpathlineto{\pgfqpoint{6.324541in}{1.375078in}}%
\pgfpathlineto{\pgfqpoint{6.321369in}{1.374583in}}%
\pgfpathlineto{\pgfqpoint{6.318197in}{1.374594in}}%
\pgfpathlineto{\pgfqpoint{6.315025in}{1.374091in}}%
\pgfpathlineto{\pgfqpoint{6.311853in}{1.373800in}}%
\pgfpathlineto{\pgfqpoint{6.308681in}{1.373540in}}%
\pgfpathlineto{\pgfqpoint{6.305509in}{1.373564in}}%
\pgfpathlineto{\pgfqpoint{6.302337in}{1.373655in}}%
\pgfpathlineto{\pgfqpoint{6.299165in}{1.373732in}}%
\pgfpathlineto{\pgfqpoint{6.295993in}{1.373621in}}%
\pgfpathlineto{\pgfqpoint{6.292821in}{1.373481in}}%
\pgfpathlineto{\pgfqpoint{6.289649in}{1.373497in}}%
\pgfpathlineto{\pgfqpoint{6.286477in}{1.373095in}}%
\pgfpathlineto{\pgfqpoint{6.283305in}{1.373116in}}%
\pgfpathlineto{\pgfqpoint{6.280133in}{1.373137in}}%
\pgfpathlineto{\pgfqpoint{6.276961in}{1.373046in}}%
\pgfpathlineto{\pgfqpoint{6.273789in}{1.372751in}}%
\pgfpathlineto{\pgfqpoint{6.270617in}{1.372850in}}%
\pgfpathlineto{\pgfqpoint{6.267445in}{1.373018in}}%
\pgfpathlineto{\pgfqpoint{6.264273in}{1.372884in}}%
\pgfpathlineto{\pgfqpoint{6.261101in}{1.372649in}}%
\pgfpathlineto{\pgfqpoint{6.257928in}{1.372263in}}%
\pgfpathlineto{\pgfqpoint{6.254756in}{1.371928in}}%
\pgfpathlineto{\pgfqpoint{6.251584in}{1.371603in}}%
\pgfpathlineto{\pgfqpoint{6.248412in}{1.371243in}}%
\pgfpathlineto{\pgfqpoint{6.245240in}{1.371353in}}%
\pgfpathlineto{\pgfqpoint{6.242068in}{1.371261in}}%
\pgfpathlineto{\pgfqpoint{6.238896in}{1.371719in}}%
\pgfpathlineto{\pgfqpoint{6.235724in}{1.371581in}}%
\pgfpathlineto{\pgfqpoint{6.232552in}{1.371329in}}%
\pgfpathlineto{\pgfqpoint{6.229380in}{1.370516in}}%
\pgfpathlineto{\pgfqpoint{6.226208in}{1.370097in}}%
\pgfpathlineto{\pgfqpoint{6.223036in}{1.370113in}}%
\pgfpathlineto{\pgfqpoint{6.219864in}{1.370156in}}%
\pgfpathlineto{\pgfqpoint{6.216692in}{1.369438in}}%
\pgfpathlineto{\pgfqpoint{6.213520in}{1.369502in}}%
\pgfpathlineto{\pgfqpoint{6.210348in}{1.369654in}}%
\pgfpathlineto{\pgfqpoint{6.207176in}{1.368858in}}%
\pgfpathlineto{\pgfqpoint{6.204004in}{1.368789in}}%
\pgfpathlineto{\pgfqpoint{6.200832in}{1.369022in}}%
\pgfpathlineto{\pgfqpoint{6.197660in}{1.368357in}}%
\pgfpathlineto{\pgfqpoint{6.194488in}{1.368017in}}%
\pgfpathlineto{\pgfqpoint{6.191316in}{1.368316in}}%
\pgfpathlineto{\pgfqpoint{6.188144in}{1.368195in}}%
\pgfpathlineto{\pgfqpoint{6.184972in}{1.367987in}}%
\pgfpathlineto{\pgfqpoint{6.181800in}{1.368153in}}%
\pgfpathlineto{\pgfqpoint{6.178627in}{1.368479in}}%
\pgfpathlineto{\pgfqpoint{6.175455in}{1.368712in}}%
\pgfpathlineto{\pgfqpoint{6.172283in}{1.368319in}}%
\pgfpathlineto{\pgfqpoint{6.169111in}{1.367847in}}%
\pgfpathlineto{\pgfqpoint{6.165939in}{1.367683in}}%
\pgfpathlineto{\pgfqpoint{6.162767in}{1.367854in}}%
\pgfpathlineto{\pgfqpoint{6.159595in}{1.367869in}}%
\pgfpathlineto{\pgfqpoint{6.156423in}{1.367724in}}%
\pgfpathlineto{\pgfqpoint{6.153251in}{1.367896in}}%
\pgfpathlineto{\pgfqpoint{6.150079in}{1.368007in}}%
\pgfpathlineto{\pgfqpoint{6.146907in}{1.368072in}}%
\pgfpathlineto{\pgfqpoint{6.143735in}{1.368054in}}%
\pgfpathlineto{\pgfqpoint{6.140563in}{1.368004in}}%
\pgfpathlineto{\pgfqpoint{6.137391in}{1.367898in}}%
\pgfpathlineto{\pgfqpoint{6.134219in}{1.367986in}}%
\pgfpathlineto{\pgfqpoint{6.131047in}{1.367956in}}%
\pgfpathlineto{\pgfqpoint{6.127875in}{1.367884in}}%
\pgfpathlineto{\pgfqpoint{6.124703in}{1.367582in}}%
\pgfpathlineto{\pgfqpoint{6.121531in}{1.367381in}}%
\pgfpathlineto{\pgfqpoint{6.118359in}{1.367379in}}%
\pgfpathlineto{\pgfqpoint{6.115187in}{1.367002in}}%
\pgfpathlineto{\pgfqpoint{6.112015in}{1.366677in}}%
\pgfpathlineto{\pgfqpoint{6.108843in}{1.366590in}}%
\pgfpathlineto{\pgfqpoint{6.105671in}{1.366665in}}%
\pgfpathlineto{\pgfqpoint{6.102499in}{1.366783in}}%
\pgfpathlineto{\pgfqpoint{6.099326in}{1.367112in}}%
\pgfpathlineto{\pgfqpoint{6.096154in}{1.367007in}}%
\pgfpathlineto{\pgfqpoint{6.092982in}{1.366784in}}%
\pgfpathlineto{\pgfqpoint{6.089810in}{1.367292in}}%
\pgfpathlineto{\pgfqpoint{6.086638in}{1.367530in}}%
\pgfpathlineto{\pgfqpoint{6.083466in}{1.367558in}}%
\pgfpathlineto{\pgfqpoint{6.080294in}{1.367195in}}%
\pgfpathlineto{\pgfqpoint{6.077122in}{1.367105in}}%
\pgfpathlineto{\pgfqpoint{6.073950in}{1.366745in}}%
\pgfpathlineto{\pgfqpoint{6.070778in}{1.366201in}}%
\pgfpathlineto{\pgfqpoint{6.067606in}{1.365952in}}%
\pgfpathlineto{\pgfqpoint{6.064434in}{1.365895in}}%
\pgfpathlineto{\pgfqpoint{6.061262in}{1.366052in}}%
\pgfpathlineto{\pgfqpoint{6.058090in}{1.365791in}}%
\pgfpathlineto{\pgfqpoint{6.054918in}{1.365513in}}%
\pgfpathlineto{\pgfqpoint{6.051746in}{1.365203in}}%
\pgfpathlineto{\pgfqpoint{6.048574in}{1.364992in}}%
\pgfpathlineto{\pgfqpoint{6.045402in}{1.364992in}}%
\pgfpathlineto{\pgfqpoint{6.042230in}{1.365037in}}%
\pgfpathlineto{\pgfqpoint{6.039058in}{1.364887in}}%
\pgfpathlineto{\pgfqpoint{6.035886in}{1.364545in}}%
\pgfpathlineto{\pgfqpoint{6.032714in}{1.364314in}}%
\pgfpathlineto{\pgfqpoint{6.029542in}{1.364034in}}%
\pgfpathlineto{\pgfqpoint{6.026370in}{1.363432in}}%
\pgfpathlineto{\pgfqpoint{6.023197in}{1.363043in}}%
\pgfpathlineto{\pgfqpoint{6.020025in}{1.363033in}}%
\pgfpathlineto{\pgfqpoint{6.016853in}{1.363193in}}%
\pgfpathlineto{\pgfqpoint{6.013681in}{1.362817in}}%
\pgfpathlineto{\pgfqpoint{6.010509in}{1.362095in}}%
\pgfpathlineto{\pgfqpoint{6.007337in}{1.361998in}}%
\pgfpathlineto{\pgfqpoint{6.004165in}{1.361445in}}%
\pgfpathlineto{\pgfqpoint{6.000993in}{1.361302in}}%
\pgfpathlineto{\pgfqpoint{5.997821in}{1.361458in}}%
\pgfpathlineto{\pgfqpoint{5.994649in}{1.362412in}}%
\pgfpathlineto{\pgfqpoint{5.991477in}{1.361990in}}%
\pgfpathlineto{\pgfqpoint{5.988305in}{1.362465in}}%
\pgfpathlineto{\pgfqpoint{5.985133in}{1.362807in}}%
\pgfpathlineto{\pgfqpoint{5.981961in}{1.362454in}}%
\pgfpathlineto{\pgfqpoint{5.978789in}{1.362445in}}%
\pgfpathlineto{\pgfqpoint{5.975617in}{1.362263in}}%
\pgfpathlineto{\pgfqpoint{5.972445in}{1.362434in}}%
\pgfpathlineto{\pgfqpoint{5.969273in}{1.362235in}}%
\pgfpathlineto{\pgfqpoint{5.966101in}{1.361875in}}%
\pgfpathlineto{\pgfqpoint{5.962929in}{1.362082in}}%
\pgfpathlineto{\pgfqpoint{5.959757in}{1.362142in}}%
\pgfpathlineto{\pgfqpoint{5.956585in}{1.362263in}}%
\pgfpathlineto{\pgfqpoint{5.953413in}{1.362201in}}%
\pgfpathlineto{\pgfqpoint{5.950241in}{1.362235in}}%
\pgfpathlineto{\pgfqpoint{5.947069in}{1.362223in}}%
\pgfpathlineto{\pgfqpoint{5.943896in}{1.361991in}}%
\pgfpathlineto{\pgfqpoint{5.940724in}{1.361526in}}%
\pgfpathlineto{\pgfqpoint{5.937552in}{1.361653in}}%
\pgfpathlineto{\pgfqpoint{5.934380in}{1.361415in}}%
\pgfpathlineto{\pgfqpoint{5.931208in}{1.361017in}}%
\pgfpathlineto{\pgfqpoint{5.928036in}{1.360984in}}%
\pgfpathlineto{\pgfqpoint{5.924864in}{1.360926in}}%
\pgfpathlineto{\pgfqpoint{5.921692in}{1.361063in}}%
\pgfpathlineto{\pgfqpoint{5.918520in}{1.360863in}}%
\pgfpathlineto{\pgfqpoint{5.915348in}{1.360410in}}%
\pgfpathlineto{\pgfqpoint{5.912176in}{1.360314in}}%
\pgfpathlineto{\pgfqpoint{5.909004in}{1.360157in}}%
\pgfpathlineto{\pgfqpoint{5.905832in}{1.359732in}}%
\pgfpathlineto{\pgfqpoint{5.902660in}{1.359531in}}%
\pgfpathlineto{\pgfqpoint{5.899488in}{1.359449in}}%
\pgfpathlineto{\pgfqpoint{5.896316in}{1.358935in}}%
\pgfpathlineto{\pgfqpoint{5.893144in}{1.358872in}}%
\pgfpathlineto{\pgfqpoint{5.889972in}{1.358759in}}%
\pgfpathlineto{\pgfqpoint{5.886800in}{1.358570in}}%
\pgfpathlineto{\pgfqpoint{5.883628in}{1.359143in}}%
\pgfpathlineto{\pgfqpoint{5.880456in}{1.359120in}}%
\pgfpathlineto{\pgfqpoint{5.877284in}{1.359652in}}%
\pgfpathlineto{\pgfqpoint{5.874112in}{1.359640in}}%
\pgfpathlineto{\pgfqpoint{5.870940in}{1.359165in}}%
\pgfpathlineto{\pgfqpoint{5.867768in}{1.358642in}}%
\pgfpathlineto{\pgfqpoint{5.864595in}{1.358625in}}%
\pgfpathlineto{\pgfqpoint{5.861423in}{1.358340in}}%
\pgfpathlineto{\pgfqpoint{5.858251in}{1.357741in}}%
\pgfpathlineto{\pgfqpoint{5.855079in}{1.357524in}}%
\pgfpathlineto{\pgfqpoint{5.851907in}{1.357619in}}%
\pgfpathlineto{\pgfqpoint{5.848735in}{1.357754in}}%
\pgfpathlineto{\pgfqpoint{5.845563in}{1.358027in}}%
\pgfpathlineto{\pgfqpoint{5.842391in}{1.358159in}}%
\pgfpathlineto{\pgfqpoint{5.839219in}{1.358382in}}%
\pgfpathlineto{\pgfqpoint{5.836047in}{1.358408in}}%
\pgfpathlineto{\pgfqpoint{5.832875in}{1.358691in}}%
\pgfpathlineto{\pgfqpoint{5.829703in}{1.358410in}}%
\pgfpathlineto{\pgfqpoint{5.826531in}{1.358360in}}%
\pgfpathlineto{\pgfqpoint{5.823359in}{1.358254in}}%
\pgfpathlineto{\pgfqpoint{5.820187in}{1.358318in}}%
\pgfpathlineto{\pgfqpoint{5.817015in}{1.358320in}}%
\pgfpathlineto{\pgfqpoint{5.813843in}{1.358049in}}%
\pgfpathlineto{\pgfqpoint{5.810671in}{1.357532in}}%
\pgfpathlineto{\pgfqpoint{5.807499in}{1.357990in}}%
\pgfpathlineto{\pgfqpoint{5.804327in}{1.357823in}}%
\pgfpathlineto{\pgfqpoint{5.801155in}{1.357920in}}%
\pgfpathlineto{\pgfqpoint{5.797983in}{1.357207in}}%
\pgfpathlineto{\pgfqpoint{5.794811in}{1.357323in}}%
\pgfpathlineto{\pgfqpoint{5.791639in}{1.357174in}}%
\pgfpathlineto{\pgfqpoint{5.788466in}{1.357224in}}%
\pgfpathlineto{\pgfqpoint{5.785294in}{1.357508in}}%
\pgfpathlineto{\pgfqpoint{5.782122in}{1.357499in}}%
\pgfpathlineto{\pgfqpoint{5.778950in}{1.357635in}}%
\pgfpathlineto{\pgfqpoint{5.775778in}{1.357348in}}%
\pgfpathlineto{\pgfqpoint{5.772606in}{1.357380in}}%
\pgfpathlineto{\pgfqpoint{5.769434in}{1.357452in}}%
\pgfpathlineto{\pgfqpoint{5.766262in}{1.357448in}}%
\pgfpathlineto{\pgfqpoint{5.763090in}{1.357110in}}%
\pgfpathlineto{\pgfqpoint{5.759918in}{1.357020in}}%
\pgfpathlineto{\pgfqpoint{5.756746in}{1.356901in}}%
\pgfpathlineto{\pgfqpoint{5.753574in}{1.356595in}}%
\pgfpathlineto{\pgfqpoint{5.750402in}{1.356307in}}%
\pgfpathlineto{\pgfqpoint{5.747230in}{1.355897in}}%
\pgfpathlineto{\pgfqpoint{5.744058in}{1.356120in}}%
\pgfpathlineto{\pgfqpoint{5.740886in}{1.355790in}}%
\pgfpathlineto{\pgfqpoint{5.737714in}{1.355561in}}%
\pgfpathlineto{\pgfqpoint{5.734542in}{1.355402in}}%
\pgfpathlineto{\pgfqpoint{5.731370in}{1.355531in}}%
\pgfpathlineto{\pgfqpoint{5.728198in}{1.355374in}}%
\pgfpathlineto{\pgfqpoint{5.725026in}{1.354694in}}%
\pgfpathlineto{\pgfqpoint{5.721854in}{1.354802in}}%
\pgfpathlineto{\pgfqpoint{5.718682in}{1.354450in}}%
\pgfpathlineto{\pgfqpoint{5.715510in}{1.354931in}}%
\pgfpathlineto{\pgfqpoint{5.712338in}{1.354922in}}%
\pgfpathlineto{\pgfqpoint{5.709165in}{1.354948in}}%
\pgfpathlineto{\pgfqpoint{5.705993in}{1.354940in}}%
\pgfpathlineto{\pgfqpoint{5.702821in}{1.354888in}}%
\pgfpathlineto{\pgfqpoint{5.699649in}{1.354699in}}%
\pgfpathlineto{\pgfqpoint{5.696477in}{1.354738in}}%
\pgfpathlineto{\pgfqpoint{5.693305in}{1.354875in}}%
\pgfpathlineto{\pgfqpoint{5.690133in}{1.354477in}}%
\pgfpathlineto{\pgfqpoint{5.686961in}{1.354366in}}%
\pgfpathlineto{\pgfqpoint{5.683789in}{1.354526in}}%
\pgfpathlineto{\pgfqpoint{5.680617in}{1.354320in}}%
\pgfpathlineto{\pgfqpoint{5.677445in}{1.354113in}}%
\pgfpathlineto{\pgfqpoint{5.674273in}{1.353827in}}%
\pgfpathlineto{\pgfqpoint{5.671101in}{1.353800in}}%
\pgfpathlineto{\pgfqpoint{5.667929in}{1.353684in}}%
\pgfpathlineto{\pgfqpoint{5.664757in}{1.354062in}}%
\pgfpathlineto{\pgfqpoint{5.661585in}{1.354432in}}%
\pgfpathlineto{\pgfqpoint{5.658413in}{1.354502in}}%
\pgfpathlineto{\pgfqpoint{5.655241in}{1.354682in}}%
\pgfpathlineto{\pgfqpoint{5.652069in}{1.354044in}}%
\pgfpathlineto{\pgfqpoint{5.648897in}{1.354112in}}%
\pgfpathlineto{\pgfqpoint{5.645725in}{1.353906in}}%
\pgfpathlineto{\pgfqpoint{5.642553in}{1.353331in}}%
\pgfpathlineto{\pgfqpoint{5.639381in}{1.353632in}}%
\pgfpathlineto{\pgfqpoint{5.636209in}{1.353701in}}%
\pgfpathlineto{\pgfqpoint{5.633037in}{1.353745in}}%
\pgfpathlineto{\pgfqpoint{5.629864in}{1.353975in}}%
\pgfpathlineto{\pgfqpoint{5.626692in}{1.353880in}}%
\pgfpathlineto{\pgfqpoint{5.623520in}{1.353313in}}%
\pgfpathlineto{\pgfqpoint{5.620348in}{1.352987in}}%
\pgfpathlineto{\pgfqpoint{5.617176in}{1.353221in}}%
\pgfpathlineto{\pgfqpoint{5.614004in}{1.353446in}}%
\pgfpathlineto{\pgfqpoint{5.610832in}{1.353360in}}%
\pgfpathlineto{\pgfqpoint{5.607660in}{1.353368in}}%
\pgfpathlineto{\pgfqpoint{5.604488in}{1.352525in}}%
\pgfpathlineto{\pgfqpoint{5.601316in}{1.352239in}}%
\pgfpathlineto{\pgfqpoint{5.598144in}{1.352879in}}%
\pgfpathlineto{\pgfqpoint{5.594972in}{1.353221in}}%
\pgfpathlineto{\pgfqpoint{5.591800in}{1.353512in}}%
\pgfpathlineto{\pgfqpoint{5.588628in}{1.353331in}}%
\pgfpathlineto{\pgfqpoint{5.585456in}{1.352817in}}%
\pgfpathlineto{\pgfqpoint{5.582284in}{1.352715in}}%
\pgfpathlineto{\pgfqpoint{5.579112in}{1.352847in}}%
\pgfpathlineto{\pgfqpoint{5.575940in}{1.352663in}}%
\pgfpathlineto{\pgfqpoint{5.572768in}{1.352276in}}%
\pgfpathlineto{\pgfqpoint{5.569596in}{1.352304in}}%
\pgfpathlineto{\pgfqpoint{5.566424in}{1.352044in}}%
\pgfpathlineto{\pgfqpoint{5.563252in}{1.352259in}}%
\pgfpathlineto{\pgfqpoint{5.560080in}{1.352288in}}%
\pgfpathlineto{\pgfqpoint{5.556908in}{1.352315in}}%
\pgfpathlineto{\pgfqpoint{5.553735in}{1.352289in}}%
\pgfpathlineto{\pgfqpoint{5.550563in}{1.352214in}}%
\pgfpathlineto{\pgfqpoint{5.547391in}{1.351988in}}%
\pgfpathlineto{\pgfqpoint{5.544219in}{1.352203in}}%
\pgfpathlineto{\pgfqpoint{5.541047in}{1.352094in}}%
\pgfpathlineto{\pgfqpoint{5.537875in}{1.352781in}}%
\pgfpathlineto{\pgfqpoint{5.534703in}{1.352633in}}%
\pgfpathlineto{\pgfqpoint{5.531531in}{1.352736in}}%
\pgfpathlineto{\pgfqpoint{5.528359in}{1.352557in}}%
\pgfpathlineto{\pgfqpoint{5.525187in}{1.352432in}}%
\pgfpathlineto{\pgfqpoint{5.522015in}{1.352143in}}%
\pgfpathlineto{\pgfqpoint{5.518843in}{1.352355in}}%
\pgfpathlineto{\pgfqpoint{5.515671in}{1.351878in}}%
\pgfpathlineto{\pgfqpoint{5.512499in}{1.352109in}}%
\pgfpathlineto{\pgfqpoint{5.509327in}{1.352333in}}%
\pgfpathlineto{\pgfqpoint{5.506155in}{1.352529in}}%
\pgfpathlineto{\pgfqpoint{5.502983in}{1.352287in}}%
\pgfpathlineto{\pgfqpoint{5.499811in}{1.351744in}}%
\pgfpathlineto{\pgfqpoint{5.496639in}{1.351806in}}%
\pgfpathlineto{\pgfqpoint{5.493467in}{1.351641in}}%
\pgfpathlineto{\pgfqpoint{5.490295in}{1.351681in}}%
\pgfpathlineto{\pgfqpoint{5.487123in}{1.351926in}}%
\pgfpathlineto{\pgfqpoint{5.483951in}{1.352139in}}%
\pgfpathlineto{\pgfqpoint{5.480779in}{1.352152in}}%
\pgfpathlineto{\pgfqpoint{5.477607in}{1.352457in}}%
\pgfpathlineto{\pgfqpoint{5.474434in}{1.352635in}}%
\pgfpathlineto{\pgfqpoint{5.471262in}{1.352509in}}%
\pgfpathlineto{\pgfqpoint{5.468090in}{1.352334in}}%
\pgfpathlineto{\pgfqpoint{5.464918in}{1.351932in}}%
\pgfpathlineto{\pgfqpoint{5.461746in}{1.351471in}}%
\pgfpathlineto{\pgfqpoint{5.458574in}{1.351177in}}%
\pgfpathlineto{\pgfqpoint{5.455402in}{1.350909in}}%
\pgfpathlineto{\pgfqpoint{5.452230in}{1.350765in}}%
\pgfpathlineto{\pgfqpoint{5.449058in}{1.350651in}}%
\pgfpathlineto{\pgfqpoint{5.445886in}{1.350148in}}%
\pgfpathlineto{\pgfqpoint{5.442714in}{1.350012in}}%
\pgfpathlineto{\pgfqpoint{5.439542in}{1.349626in}}%
\pgfpathlineto{\pgfqpoint{5.436370in}{1.349136in}}%
\pgfpathlineto{\pgfqpoint{5.433198in}{1.348995in}}%
\pgfpathlineto{\pgfqpoint{5.430026in}{1.349160in}}%
\pgfpathlineto{\pgfqpoint{5.426854in}{1.349124in}}%
\pgfpathlineto{\pgfqpoint{5.423682in}{1.349029in}}%
\pgfpathlineto{\pgfqpoint{5.420510in}{1.348819in}}%
\pgfpathlineto{\pgfqpoint{5.417338in}{1.348462in}}%
\pgfpathlineto{\pgfqpoint{5.414166in}{1.348412in}}%
\pgfpathlineto{\pgfqpoint{5.410994in}{1.348634in}}%
\pgfpathlineto{\pgfqpoint{5.407822in}{1.348513in}}%
\pgfpathlineto{\pgfqpoint{5.404650in}{1.348069in}}%
\pgfpathlineto{\pgfqpoint{5.401478in}{1.348062in}}%
\pgfpathlineto{\pgfqpoint{5.398306in}{1.348121in}}%
\pgfpathlineto{\pgfqpoint{5.395133in}{1.347935in}}%
\pgfpathlineto{\pgfqpoint{5.391961in}{1.347707in}}%
\pgfpathlineto{\pgfqpoint{5.388789in}{1.347747in}}%
\pgfpathlineto{\pgfqpoint{5.385617in}{1.347350in}}%
\pgfpathlineto{\pgfqpoint{5.382445in}{1.347287in}}%
\pgfpathlineto{\pgfqpoint{5.379273in}{1.347339in}}%
\pgfpathlineto{\pgfqpoint{5.376101in}{1.347038in}}%
\pgfpathlineto{\pgfqpoint{5.372929in}{1.347170in}}%
\pgfpathlineto{\pgfqpoint{5.369757in}{1.346580in}}%
\pgfpathlineto{\pgfqpoint{5.366585in}{1.346316in}}%
\pgfpathlineto{\pgfqpoint{5.363413in}{1.346797in}}%
\pgfpathlineto{\pgfqpoint{5.360241in}{1.347134in}}%
\pgfpathlineto{\pgfqpoint{5.357069in}{1.346056in}}%
\pgfpathlineto{\pgfqpoint{5.353897in}{1.345680in}}%
\pgfpathlineto{\pgfqpoint{5.350725in}{1.345439in}}%
\pgfpathlineto{\pgfqpoint{5.347553in}{1.345265in}}%
\pgfpathlineto{\pgfqpoint{5.344381in}{1.345391in}}%
\pgfpathlineto{\pgfqpoint{5.341209in}{1.345326in}}%
\pgfpathlineto{\pgfqpoint{5.338037in}{1.344977in}}%
\pgfpathlineto{\pgfqpoint{5.334865in}{1.344359in}}%
\pgfpathlineto{\pgfqpoint{5.331693in}{1.343888in}}%
\pgfpathlineto{\pgfqpoint{5.328521in}{1.343251in}}%
\pgfpathlineto{\pgfqpoint{5.325349in}{1.343198in}}%
\pgfpathlineto{\pgfqpoint{5.322177in}{1.343318in}}%
\pgfpathlineto{\pgfqpoint{5.319004in}{1.343337in}}%
\pgfpathlineto{\pgfqpoint{5.315832in}{1.342940in}}%
\pgfpathlineto{\pgfqpoint{5.312660in}{1.343036in}}%
\pgfpathlineto{\pgfqpoint{5.309488in}{1.341983in}}%
\pgfpathlineto{\pgfqpoint{5.306316in}{1.341669in}}%
\pgfpathlineto{\pgfqpoint{5.303144in}{1.341212in}}%
\pgfpathlineto{\pgfqpoint{5.299972in}{1.340897in}}%
\pgfpathlineto{\pgfqpoint{5.296800in}{1.340728in}}%
\pgfpathlineto{\pgfqpoint{5.293628in}{1.340680in}}%
\pgfpathlineto{\pgfqpoint{5.290456in}{1.340347in}}%
\pgfpathlineto{\pgfqpoint{5.287284in}{1.340237in}}%
\pgfpathlineto{\pgfqpoint{5.284112in}{1.340406in}}%
\pgfpathlineto{\pgfqpoint{5.280940in}{1.340418in}}%
\pgfpathlineto{\pgfqpoint{5.277768in}{1.340344in}}%
\pgfpathlineto{\pgfqpoint{5.274596in}{1.340215in}}%
\pgfpathlineto{\pgfqpoint{5.271424in}{1.340070in}}%
\pgfpathlineto{\pgfqpoint{5.268252in}{1.339972in}}%
\pgfpathlineto{\pgfqpoint{5.265080in}{1.339854in}}%
\pgfpathlineto{\pgfqpoint{5.261908in}{1.339617in}}%
\pgfpathlineto{\pgfqpoint{5.258736in}{1.339461in}}%
\pgfpathlineto{\pgfqpoint{5.255564in}{1.338934in}}%
\pgfpathlineto{\pgfqpoint{5.252392in}{1.338599in}}%
\pgfpathlineto{\pgfqpoint{5.249220in}{1.338184in}}%
\pgfpathlineto{\pgfqpoint{5.246048in}{1.338077in}}%
\pgfpathlineto{\pgfqpoint{5.242876in}{1.337820in}}%
\pgfpathlineto{\pgfqpoint{5.239703in}{1.337641in}}%
\pgfpathlineto{\pgfqpoint{5.236531in}{1.337023in}}%
\pgfpathlineto{\pgfqpoint{5.233359in}{1.337222in}}%
\pgfpathlineto{\pgfqpoint{5.230187in}{1.337326in}}%
\pgfpathlineto{\pgfqpoint{5.227015in}{1.337355in}}%
\pgfpathlineto{\pgfqpoint{5.223843in}{1.337353in}}%
\pgfpathlineto{\pgfqpoint{5.220671in}{1.337764in}}%
\pgfpathlineto{\pgfqpoint{5.217499in}{1.337953in}}%
\pgfpathlineto{\pgfqpoint{5.214327in}{1.338122in}}%
\pgfpathlineto{\pgfqpoint{5.211155in}{1.337979in}}%
\pgfpathlineto{\pgfqpoint{5.207983in}{1.338035in}}%
\pgfpathlineto{\pgfqpoint{5.204811in}{1.337574in}}%
\pgfpathlineto{\pgfqpoint{5.201639in}{1.337228in}}%
\pgfpathlineto{\pgfqpoint{5.198467in}{1.337647in}}%
\pgfpathlineto{\pgfqpoint{5.195295in}{1.337283in}}%
\pgfpathlineto{\pgfqpoint{5.192123in}{1.337563in}}%
\pgfpathlineto{\pgfqpoint{5.188951in}{1.337870in}}%
\pgfpathlineto{\pgfqpoint{5.185779in}{1.337833in}}%
\pgfpathlineto{\pgfqpoint{5.182607in}{1.337717in}}%
\pgfpathlineto{\pgfqpoint{5.179435in}{1.337894in}}%
\pgfpathlineto{\pgfqpoint{5.176263in}{1.337795in}}%
\pgfpathlineto{\pgfqpoint{5.173091in}{1.337859in}}%
\pgfpathlineto{\pgfqpoint{5.169919in}{1.338306in}}%
\pgfpathlineto{\pgfqpoint{5.166747in}{1.337974in}}%
\pgfpathlineto{\pgfqpoint{5.163575in}{1.337762in}}%
\pgfpathlineto{\pgfqpoint{5.160402in}{1.337829in}}%
\pgfpathlineto{\pgfqpoint{5.157230in}{1.337418in}}%
\pgfpathlineto{\pgfqpoint{5.154058in}{1.337247in}}%
\pgfpathlineto{\pgfqpoint{5.150886in}{1.337192in}}%
\pgfpathlineto{\pgfqpoint{5.147714in}{1.337072in}}%
\pgfpathlineto{\pgfqpoint{5.144542in}{1.337287in}}%
\pgfpathlineto{\pgfqpoint{5.141370in}{1.337230in}}%
\pgfpathlineto{\pgfqpoint{5.138198in}{1.337313in}}%
\pgfpathlineto{\pgfqpoint{5.135026in}{1.337529in}}%
\pgfpathlineto{\pgfqpoint{5.131854in}{1.337416in}}%
\pgfpathlineto{\pgfqpoint{5.128682in}{1.337199in}}%
\pgfpathlineto{\pgfqpoint{5.125510in}{1.336940in}}%
\pgfpathlineto{\pgfqpoint{5.122338in}{1.336484in}}%
\pgfpathlineto{\pgfqpoint{5.119166in}{1.336517in}}%
\pgfpathlineto{\pgfqpoint{5.115994in}{1.336572in}}%
\pgfpathlineto{\pgfqpoint{5.112822in}{1.336541in}}%
\pgfpathlineto{\pgfqpoint{5.109650in}{1.336877in}}%
\pgfpathlineto{\pgfqpoint{5.106478in}{1.336744in}}%
\pgfpathlineto{\pgfqpoint{5.103306in}{1.336399in}}%
\pgfpathlineto{\pgfqpoint{5.100134in}{1.336164in}}%
\pgfpathlineto{\pgfqpoint{5.096962in}{1.336153in}}%
\pgfpathlineto{\pgfqpoint{5.093790in}{1.336295in}}%
\pgfpathlineto{\pgfqpoint{5.090618in}{1.336335in}}%
\pgfpathlineto{\pgfqpoint{5.087446in}{1.336251in}}%
\pgfpathlineto{\pgfqpoint{5.084273in}{1.336352in}}%
\pgfpathlineto{\pgfqpoint{5.081101in}{1.336243in}}%
\pgfpathlineto{\pgfqpoint{5.077929in}{1.335936in}}%
\pgfpathlineto{\pgfqpoint{5.074757in}{1.336093in}}%
\pgfpathlineto{\pgfqpoint{5.071585in}{1.335733in}}%
\pgfpathlineto{\pgfqpoint{5.068413in}{1.335635in}}%
\pgfpathlineto{\pgfqpoint{5.065241in}{1.335417in}}%
\pgfpathlineto{\pgfqpoint{5.062069in}{1.335266in}}%
\pgfpathlineto{\pgfqpoint{5.058897in}{1.334983in}}%
\pgfpathlineto{\pgfqpoint{5.055725in}{1.334863in}}%
\pgfpathlineto{\pgfqpoint{5.052553in}{1.335006in}}%
\pgfpathlineto{\pgfqpoint{5.049381in}{1.335312in}}%
\pgfpathlineto{\pgfqpoint{5.046209in}{1.335850in}}%
\pgfpathlineto{\pgfqpoint{5.043037in}{1.335684in}}%
\pgfpathlineto{\pgfqpoint{5.039865in}{1.335496in}}%
\pgfpathlineto{\pgfqpoint{5.036693in}{1.335234in}}%
\pgfpathlineto{\pgfqpoint{5.033521in}{1.334889in}}%
\pgfpathlineto{\pgfqpoint{5.030349in}{1.334764in}}%
\pgfpathlineto{\pgfqpoint{5.027177in}{1.334393in}}%
\pgfpathlineto{\pgfqpoint{5.024005in}{1.334039in}}%
\pgfpathlineto{\pgfqpoint{5.020833in}{1.334118in}}%
\pgfpathlineto{\pgfqpoint{5.017661in}{1.334168in}}%
\pgfpathlineto{\pgfqpoint{5.014489in}{1.333970in}}%
\pgfpathlineto{\pgfqpoint{5.011317in}{1.333908in}}%
\pgfpathlineto{\pgfqpoint{5.008145in}{1.333764in}}%
\pgfpathlineto{\pgfqpoint{5.004972in}{1.333946in}}%
\pgfpathlineto{\pgfqpoint{5.001800in}{1.333797in}}%
\pgfpathlineto{\pgfqpoint{4.998628in}{1.333630in}}%
\pgfpathlineto{\pgfqpoint{4.995456in}{1.333770in}}%
\pgfpathlineto{\pgfqpoint{4.992284in}{1.333845in}}%
\pgfpathlineto{\pgfqpoint{4.989112in}{1.333022in}}%
\pgfpathlineto{\pgfqpoint{4.985940in}{1.333367in}}%
\pgfpathlineto{\pgfqpoint{4.982768in}{1.332923in}}%
\pgfpathlineto{\pgfqpoint{4.979596in}{1.332713in}}%
\pgfpathlineto{\pgfqpoint{4.976424in}{1.332694in}}%
\pgfpathlineto{\pgfqpoint{4.973252in}{1.332508in}}%
\pgfpathlineto{\pgfqpoint{4.970080in}{1.332385in}}%
\pgfpathlineto{\pgfqpoint{4.966908in}{1.332125in}}%
\pgfpathlineto{\pgfqpoint{4.963736in}{1.331259in}}%
\pgfpathlineto{\pgfqpoint{4.960564in}{1.330934in}}%
\pgfpathlineto{\pgfqpoint{4.957392in}{1.330955in}}%
\pgfpathlineto{\pgfqpoint{4.954220in}{1.330291in}}%
\pgfpathlineto{\pgfqpoint{4.951048in}{1.330206in}}%
\pgfpathlineto{\pgfqpoint{4.947876in}{1.330280in}}%
\pgfpathlineto{\pgfqpoint{4.944704in}{1.330119in}}%
\pgfpathlineto{\pgfqpoint{4.941532in}{1.330048in}}%
\pgfpathlineto{\pgfqpoint{4.938360in}{1.330021in}}%
\pgfpathlineto{\pgfqpoint{4.935188in}{1.329830in}}%
\pgfpathlineto{\pgfqpoint{4.932016in}{1.329713in}}%
\pgfpathlineto{\pgfqpoint{4.928844in}{1.329248in}}%
\pgfpathlineto{\pgfqpoint{4.925671in}{1.329358in}}%
\pgfpathlineto{\pgfqpoint{4.922499in}{1.329206in}}%
\pgfpathlineto{\pgfqpoint{4.919327in}{1.329333in}}%
\pgfpathlineto{\pgfqpoint{4.916155in}{1.329541in}}%
\pgfpathlineto{\pgfqpoint{4.912983in}{1.329531in}}%
\pgfpathlineto{\pgfqpoint{4.909811in}{1.329486in}}%
\pgfpathlineto{\pgfqpoint{4.906639in}{1.329298in}}%
\pgfpathlineto{\pgfqpoint{4.903467in}{1.329033in}}%
\pgfpathlineto{\pgfqpoint{4.900295in}{1.328665in}}%
\pgfpathlineto{\pgfqpoint{4.897123in}{1.328360in}}%
\pgfpathlineto{\pgfqpoint{4.893951in}{1.327691in}}%
\pgfpathlineto{\pgfqpoint{4.890779in}{1.327300in}}%
\pgfpathlineto{\pgfqpoint{4.887607in}{1.327420in}}%
\pgfpathlineto{\pgfqpoint{4.884435in}{1.326976in}}%
\pgfpathlineto{\pgfqpoint{4.881263in}{1.326965in}}%
\pgfpathlineto{\pgfqpoint{4.878091in}{1.326813in}}%
\pgfpathlineto{\pgfqpoint{4.874919in}{1.326917in}}%
\pgfpathlineto{\pgfqpoint{4.871747in}{1.327010in}}%
\pgfpathlineto{\pgfqpoint{4.868575in}{1.326935in}}%
\pgfpathlineto{\pgfqpoint{4.865403in}{1.326886in}}%
\pgfpathlineto{\pgfqpoint{4.862231in}{1.326363in}}%
\pgfpathlineto{\pgfqpoint{4.859059in}{1.326512in}}%
\pgfpathlineto{\pgfqpoint{4.855887in}{1.326416in}}%
\pgfpathlineto{\pgfqpoint{4.852715in}{1.325874in}}%
\pgfpathlineto{\pgfqpoint{4.849542in}{1.325853in}}%
\pgfpathlineto{\pgfqpoint{4.846370in}{1.325485in}}%
\pgfpathlineto{\pgfqpoint{4.843198in}{1.325155in}}%
\pgfpathlineto{\pgfqpoint{4.840026in}{1.324744in}}%
\pgfpathlineto{\pgfqpoint{4.836854in}{1.324782in}}%
\pgfpathlineto{\pgfqpoint{4.833682in}{1.325055in}}%
\pgfpathlineto{\pgfqpoint{4.830510in}{1.324751in}}%
\pgfpathlineto{\pgfqpoint{4.827338in}{1.324377in}}%
\pgfpathlineto{\pgfqpoint{4.824166in}{1.324139in}}%
\pgfpathlineto{\pgfqpoint{4.820994in}{1.323838in}}%
\pgfpathlineto{\pgfqpoint{4.817822in}{1.323580in}}%
\pgfpathlineto{\pgfqpoint{4.814650in}{1.323112in}}%
\pgfpathlineto{\pgfqpoint{4.811478in}{1.323216in}}%
\pgfpathlineto{\pgfqpoint{4.808306in}{1.322698in}}%
\pgfpathlineto{\pgfqpoint{4.805134in}{1.322693in}}%
\pgfpathlineto{\pgfqpoint{4.801962in}{1.322604in}}%
\pgfpathlineto{\pgfqpoint{4.798790in}{1.322542in}}%
\pgfpathlineto{\pgfqpoint{4.795618in}{1.322592in}}%
\pgfpathlineto{\pgfqpoint{4.792446in}{1.322480in}}%
\pgfpathlineto{\pgfqpoint{4.789274in}{1.321770in}}%
\pgfpathlineto{\pgfqpoint{4.786102in}{1.321766in}}%
\pgfpathlineto{\pgfqpoint{4.782930in}{1.322069in}}%
\pgfpathlineto{\pgfqpoint{4.779758in}{1.322615in}}%
\pgfpathlineto{\pgfqpoint{4.776586in}{1.322658in}}%
\pgfpathlineto{\pgfqpoint{4.773414in}{1.322576in}}%
\pgfpathlineto{\pgfqpoint{4.770241in}{1.322089in}}%
\pgfpathlineto{\pgfqpoint{4.767069in}{1.321741in}}%
\pgfpathlineto{\pgfqpoint{4.763897in}{1.321573in}}%
\pgfpathlineto{\pgfqpoint{4.760725in}{1.321489in}}%
\pgfpathlineto{\pgfqpoint{4.757553in}{1.321653in}}%
\pgfpathlineto{\pgfqpoint{4.754381in}{1.321422in}}%
\pgfpathlineto{\pgfqpoint{4.751209in}{1.321166in}}%
\pgfpathlineto{\pgfqpoint{4.748037in}{1.320472in}}%
\pgfpathlineto{\pgfqpoint{4.744865in}{1.320276in}}%
\pgfpathlineto{\pgfqpoint{4.741693in}{1.320070in}}%
\pgfpathlineto{\pgfqpoint{4.738521in}{1.319363in}}%
\pgfpathlineto{\pgfqpoint{4.735349in}{1.318945in}}%
\pgfpathlineto{\pgfqpoint{4.732177in}{1.318778in}}%
\pgfpathlineto{\pgfqpoint{4.729005in}{1.319143in}}%
\pgfpathlineto{\pgfqpoint{4.725833in}{1.318786in}}%
\pgfpathlineto{\pgfqpoint{4.722661in}{1.318644in}}%
\pgfpathlineto{\pgfqpoint{4.719489in}{1.318858in}}%
\pgfpathlineto{\pgfqpoint{4.716317in}{1.318842in}}%
\pgfpathlineto{\pgfqpoint{4.713145in}{1.319051in}}%
\pgfpathlineto{\pgfqpoint{4.709973in}{1.319143in}}%
\pgfpathlineto{\pgfqpoint{4.706801in}{1.319077in}}%
\pgfpathlineto{\pgfqpoint{4.703629in}{1.318900in}}%
\pgfpathlineto{\pgfqpoint{4.700457in}{1.318858in}}%
\pgfpathlineto{\pgfqpoint{4.697285in}{1.318750in}}%
\pgfpathlineto{\pgfqpoint{4.694112in}{1.318242in}}%
\pgfpathlineto{\pgfqpoint{4.690940in}{1.317927in}}%
\pgfpathlineto{\pgfqpoint{4.687768in}{1.317912in}}%
\pgfpathlineto{\pgfqpoint{4.684596in}{1.317503in}}%
\pgfpathlineto{\pgfqpoint{4.681424in}{1.317055in}}%
\pgfpathlineto{\pgfqpoint{4.678252in}{1.317131in}}%
\pgfpathlineto{\pgfqpoint{4.675080in}{1.317071in}}%
\pgfpathlineto{\pgfqpoint{4.671908in}{1.316481in}}%
\pgfpathlineto{\pgfqpoint{4.668736in}{1.316499in}}%
\pgfpathlineto{\pgfqpoint{4.665564in}{1.316081in}}%
\pgfpathlineto{\pgfqpoint{4.662392in}{1.315775in}}%
\pgfpathlineto{\pgfqpoint{4.659220in}{1.315241in}}%
\pgfpathlineto{\pgfqpoint{4.656048in}{1.315140in}}%
\pgfpathlineto{\pgfqpoint{4.652876in}{1.315404in}}%
\pgfpathlineto{\pgfqpoint{4.649704in}{1.315775in}}%
\pgfpathlineto{\pgfqpoint{4.646532in}{1.315262in}}%
\pgfpathlineto{\pgfqpoint{4.643360in}{1.315265in}}%
\pgfpathlineto{\pgfqpoint{4.640188in}{1.315296in}}%
\pgfpathlineto{\pgfqpoint{4.637016in}{1.315310in}}%
\pgfpathlineto{\pgfqpoint{4.633844in}{1.315268in}}%
\pgfpathlineto{\pgfqpoint{4.630672in}{1.315255in}}%
\pgfpathlineto{\pgfqpoint{4.627500in}{1.315269in}}%
\pgfpathlineto{\pgfqpoint{4.624328in}{1.315128in}}%
\pgfpathlineto{\pgfqpoint{4.621156in}{1.315147in}}%
\pgfpathlineto{\pgfqpoint{4.617984in}{1.315394in}}%
\pgfpathlineto{\pgfqpoint{4.614811in}{1.315431in}}%
\pgfpathlineto{\pgfqpoint{4.611639in}{1.315514in}}%
\pgfpathlineto{\pgfqpoint{4.608467in}{1.315619in}}%
\pgfpathlineto{\pgfqpoint{4.605295in}{1.315269in}}%
\pgfpathlineto{\pgfqpoint{4.602123in}{1.315709in}}%
\pgfpathlineto{\pgfqpoint{4.598951in}{1.315425in}}%
\pgfpathlineto{\pgfqpoint{4.595779in}{1.315350in}}%
\pgfpathlineto{\pgfqpoint{4.592607in}{1.314490in}}%
\pgfpathlineto{\pgfqpoint{4.589435in}{1.313915in}}%
\pgfpathlineto{\pgfqpoint{4.586263in}{1.313390in}}%
\pgfpathlineto{\pgfqpoint{4.583091in}{1.313238in}}%
\pgfpathlineto{\pgfqpoint{4.579919in}{1.312535in}}%
\pgfpathlineto{\pgfqpoint{4.576747in}{1.312528in}}%
\pgfpathlineto{\pgfqpoint{4.573575in}{1.312821in}}%
\pgfpathlineto{\pgfqpoint{4.570403in}{1.312803in}}%
\pgfpathlineto{\pgfqpoint{4.567231in}{1.312533in}}%
\pgfpathlineto{\pgfqpoint{4.564059in}{1.312330in}}%
\pgfpathlineto{\pgfqpoint{4.560887in}{1.311944in}}%
\pgfpathlineto{\pgfqpoint{4.557715in}{1.312196in}}%
\pgfpathlineto{\pgfqpoint{4.554543in}{1.312338in}}%
\pgfpathlineto{\pgfqpoint{4.551371in}{1.311871in}}%
\pgfpathlineto{\pgfqpoint{4.548199in}{1.312022in}}%
\pgfpathlineto{\pgfqpoint{4.545027in}{1.312117in}}%
\pgfpathlineto{\pgfqpoint{4.541855in}{1.312092in}}%
\pgfpathlineto{\pgfqpoint{4.538683in}{1.311912in}}%
\pgfpathlineto{\pgfqpoint{4.535510in}{1.311545in}}%
\pgfpathlineto{\pgfqpoint{4.532338in}{1.311769in}}%
\pgfpathlineto{\pgfqpoint{4.529166in}{1.311575in}}%
\pgfpathlineto{\pgfqpoint{4.525994in}{1.311394in}}%
\pgfpathlineto{\pgfqpoint{4.522822in}{1.311523in}}%
\pgfpathlineto{\pgfqpoint{4.519650in}{1.311500in}}%
\pgfpathlineto{\pgfqpoint{4.516478in}{1.311318in}}%
\pgfpathlineto{\pgfqpoint{4.513306in}{1.311594in}}%
\pgfpathlineto{\pgfqpoint{4.510134in}{1.311683in}}%
\pgfpathlineto{\pgfqpoint{4.506962in}{1.311709in}}%
\pgfpathlineto{\pgfqpoint{4.503790in}{1.311566in}}%
\pgfpathlineto{\pgfqpoint{4.500618in}{1.311055in}}%
\pgfpathlineto{\pgfqpoint{4.497446in}{1.310680in}}%
\pgfpathlineto{\pgfqpoint{4.494274in}{1.310120in}}%
\pgfpathlineto{\pgfqpoint{4.491102in}{1.309940in}}%
\pgfpathlineto{\pgfqpoint{4.487930in}{1.309667in}}%
\pgfpathlineto{\pgfqpoint{4.484758in}{1.309227in}}%
\pgfpathlineto{\pgfqpoint{4.481586in}{1.309021in}}%
\pgfpathlineto{\pgfqpoint{4.478414in}{1.308861in}}%
\pgfpathlineto{\pgfqpoint{4.475242in}{1.308976in}}%
\pgfpathlineto{\pgfqpoint{4.472070in}{1.309192in}}%
\pgfpathlineto{\pgfqpoint{4.468898in}{1.308943in}}%
\pgfpathlineto{\pgfqpoint{4.465726in}{1.308740in}}%
\pgfpathlineto{\pgfqpoint{4.462554in}{1.308362in}}%
\pgfpathlineto{\pgfqpoint{4.459381in}{1.308473in}}%
\pgfpathlineto{\pgfqpoint{4.456209in}{1.308238in}}%
\pgfpathlineto{\pgfqpoint{4.453037in}{1.308503in}}%
\pgfpathlineto{\pgfqpoint{4.449865in}{1.308593in}}%
\pgfpathlineto{\pgfqpoint{4.446693in}{1.308516in}}%
\pgfpathlineto{\pgfqpoint{4.443521in}{1.308735in}}%
\pgfpathlineto{\pgfqpoint{4.440349in}{1.308583in}}%
\pgfpathlineto{\pgfqpoint{4.437177in}{1.308410in}}%
\pgfpathlineto{\pgfqpoint{4.434005in}{1.308519in}}%
\pgfpathlineto{\pgfqpoint{4.430833in}{1.308459in}}%
\pgfpathlineto{\pgfqpoint{4.427661in}{1.308137in}}%
\pgfpathlineto{\pgfqpoint{4.424489in}{1.307941in}}%
\pgfpathlineto{\pgfqpoint{4.421317in}{1.307707in}}%
\pgfpathlineto{\pgfqpoint{4.418145in}{1.307377in}}%
\pgfpathlineto{\pgfqpoint{4.414973in}{1.307376in}}%
\pgfpathlineto{\pgfqpoint{4.411801in}{1.307748in}}%
\pgfpathlineto{\pgfqpoint{4.408629in}{1.307548in}}%
\pgfpathlineto{\pgfqpoint{4.405457in}{1.306870in}}%
\pgfpathlineto{\pgfqpoint{4.402285in}{1.306942in}}%
\pgfpathlineto{\pgfqpoint{4.399113in}{1.306716in}}%
\pgfpathlineto{\pgfqpoint{4.395941in}{1.306472in}}%
\pgfpathlineto{\pgfqpoint{4.392769in}{1.306064in}}%
\pgfpathlineto{\pgfqpoint{4.389597in}{1.306279in}}%
\pgfpathlineto{\pgfqpoint{4.386425in}{1.306193in}}%
\pgfpathlineto{\pgfqpoint{4.383253in}{1.306328in}}%
\pgfpathlineto{\pgfqpoint{4.380080in}{1.305901in}}%
\pgfpathlineto{\pgfqpoint{4.376908in}{1.305620in}}%
\pgfpathlineto{\pgfqpoint{4.373736in}{1.305270in}}%
\pgfpathlineto{\pgfqpoint{4.370564in}{1.305059in}}%
\pgfpathlineto{\pgfqpoint{4.367392in}{1.305062in}}%
\pgfpathlineto{\pgfqpoint{4.364220in}{1.304891in}}%
\pgfpathlineto{\pgfqpoint{4.361048in}{1.304512in}}%
\pgfpathlineto{\pgfqpoint{4.357876in}{1.304731in}}%
\pgfpathlineto{\pgfqpoint{4.354704in}{1.304841in}}%
\pgfpathlineto{\pgfqpoint{4.351532in}{1.304344in}}%
\pgfpathlineto{\pgfqpoint{4.348360in}{1.303969in}}%
\pgfpathlineto{\pgfqpoint{4.345188in}{1.304100in}}%
\pgfpathlineto{\pgfqpoint{4.342016in}{1.303602in}}%
\pgfpathlineto{\pgfqpoint{4.338844in}{1.303527in}}%
\pgfpathlineto{\pgfqpoint{4.335672in}{1.303100in}}%
\pgfpathlineto{\pgfqpoint{4.332500in}{1.303215in}}%
\pgfpathlineto{\pgfqpoint{4.329328in}{1.302516in}}%
\pgfpathlineto{\pgfqpoint{4.326156in}{1.302048in}}%
\pgfpathlineto{\pgfqpoint{4.322984in}{1.301990in}}%
\pgfpathlineto{\pgfqpoint{4.319812in}{1.301954in}}%
\pgfpathlineto{\pgfqpoint{4.316640in}{1.301589in}}%
\pgfpathlineto{\pgfqpoint{4.313468in}{1.300855in}}%
\pgfpathlineto{\pgfqpoint{4.310296in}{1.300596in}}%
\pgfpathlineto{\pgfqpoint{4.307124in}{1.300451in}}%
\pgfpathlineto{\pgfqpoint{4.303952in}{1.299866in}}%
\pgfpathlineto{\pgfqpoint{4.300779in}{1.299717in}}%
\pgfpathlineto{\pgfqpoint{4.297607in}{1.299466in}}%
\pgfpathlineto{\pgfqpoint{4.294435in}{1.299316in}}%
\pgfpathlineto{\pgfqpoint{4.291263in}{1.299520in}}%
\pgfpathlineto{\pgfqpoint{4.288091in}{1.299791in}}%
\pgfpathlineto{\pgfqpoint{4.284919in}{1.299820in}}%
\pgfpathlineto{\pgfqpoint{4.281747in}{1.299803in}}%
\pgfpathlineto{\pgfqpoint{4.278575in}{1.299929in}}%
\pgfpathlineto{\pgfqpoint{4.275403in}{1.299468in}}%
\pgfpathlineto{\pgfqpoint{4.272231in}{1.299373in}}%
\pgfpathlineto{\pgfqpoint{4.269059in}{1.299120in}}%
\pgfpathlineto{\pgfqpoint{4.265887in}{1.298875in}}%
\pgfpathlineto{\pgfqpoint{4.262715in}{1.298422in}}%
\pgfpathlineto{\pgfqpoint{4.259543in}{1.298503in}}%
\pgfpathlineto{\pgfqpoint{4.256371in}{1.298272in}}%
\pgfpathlineto{\pgfqpoint{4.253199in}{1.297861in}}%
\pgfpathlineto{\pgfqpoint{4.250027in}{1.298002in}}%
\pgfpathlineto{\pgfqpoint{4.246855in}{1.298092in}}%
\pgfpathlineto{\pgfqpoint{4.243683in}{1.298310in}}%
\pgfpathlineto{\pgfqpoint{4.240511in}{1.298255in}}%
\pgfpathlineto{\pgfqpoint{4.237339in}{1.298198in}}%
\pgfpathlineto{\pgfqpoint{4.234167in}{1.297784in}}%
\pgfpathlineto{\pgfqpoint{4.230995in}{1.297630in}}%
\pgfpathlineto{\pgfqpoint{4.227823in}{1.296909in}}%
\pgfpathlineto{\pgfqpoint{4.224650in}{1.296851in}}%
\pgfpathlineto{\pgfqpoint{4.221478in}{1.296950in}}%
\pgfpathlineto{\pgfqpoint{4.218306in}{1.297195in}}%
\pgfpathlineto{\pgfqpoint{4.215134in}{1.297063in}}%
\pgfpathlineto{\pgfqpoint{4.211962in}{1.297025in}}%
\pgfpathlineto{\pgfqpoint{4.208790in}{1.297365in}}%
\pgfpathlineto{\pgfqpoint{4.205618in}{1.297324in}}%
\pgfpathlineto{\pgfqpoint{4.202446in}{1.297573in}}%
\pgfpathlineto{\pgfqpoint{4.199274in}{1.297156in}}%
\pgfpathlineto{\pgfqpoint{4.196102in}{1.296883in}}%
\pgfpathlineto{\pgfqpoint{4.192930in}{1.296622in}}%
\pgfpathlineto{\pgfqpoint{4.189758in}{1.295969in}}%
\pgfpathlineto{\pgfqpoint{4.186586in}{1.295973in}}%
\pgfpathlineto{\pgfqpoint{4.183414in}{1.295846in}}%
\pgfpathlineto{\pgfqpoint{4.180242in}{1.295689in}}%
\pgfpathlineto{\pgfqpoint{4.177070in}{1.295551in}}%
\pgfpathlineto{\pgfqpoint{4.173898in}{1.294678in}}%
\pgfpathlineto{\pgfqpoint{4.170726in}{1.294782in}}%
\pgfpathlineto{\pgfqpoint{4.167554in}{1.294618in}}%
\pgfpathlineto{\pgfqpoint{4.164382in}{1.294362in}}%
\pgfpathlineto{\pgfqpoint{4.161210in}{1.294309in}}%
\pgfpathlineto{\pgfqpoint{4.158038in}{1.294551in}}%
\pgfpathlineto{\pgfqpoint{4.154866in}{1.294370in}}%
\pgfpathlineto{\pgfqpoint{4.151694in}{1.294143in}}%
\pgfpathlineto{\pgfqpoint{4.148522in}{1.293923in}}%
\pgfpathlineto{\pgfqpoint{4.145349in}{1.293720in}}%
\pgfpathlineto{\pgfqpoint{4.142177in}{1.293596in}}%
\pgfpathlineto{\pgfqpoint{4.139005in}{1.293179in}}%
\pgfpathlineto{\pgfqpoint{4.135833in}{1.293143in}}%
\pgfpathlineto{\pgfqpoint{4.132661in}{1.293380in}}%
\pgfpathlineto{\pgfqpoint{4.129489in}{1.293415in}}%
\pgfpathlineto{\pgfqpoint{4.126317in}{1.293630in}}%
\pgfpathlineto{\pgfqpoint{4.123145in}{1.293734in}}%
\pgfpathlineto{\pgfqpoint{4.119973in}{1.293535in}}%
\pgfpathlineto{\pgfqpoint{4.116801in}{1.293568in}}%
\pgfpathlineto{\pgfqpoint{4.113629in}{1.293450in}}%
\pgfpathlineto{\pgfqpoint{4.110457in}{1.293434in}}%
\pgfpathlineto{\pgfqpoint{4.107285in}{1.293191in}}%
\pgfpathlineto{\pgfqpoint{4.104113in}{1.293041in}}%
\pgfpathlineto{\pgfqpoint{4.100941in}{1.293123in}}%
\pgfpathlineto{\pgfqpoint{4.097769in}{1.292963in}}%
\pgfpathlineto{\pgfqpoint{4.094597in}{1.293160in}}%
\pgfpathlineto{\pgfqpoint{4.091425in}{1.293433in}}%
\pgfpathlineto{\pgfqpoint{4.088253in}{1.293558in}}%
\pgfpathlineto{\pgfqpoint{4.085081in}{1.293760in}}%
\pgfpathlineto{\pgfqpoint{4.081909in}{1.293561in}}%
\pgfpathlineto{\pgfqpoint{4.078737in}{1.293360in}}%
\pgfpathlineto{\pgfqpoint{4.075565in}{1.293399in}}%
\pgfpathlineto{\pgfqpoint{4.072393in}{1.293409in}}%
\pgfpathlineto{\pgfqpoint{4.069221in}{1.293277in}}%
\pgfpathlineto{\pgfqpoint{4.066048in}{1.292692in}}%
\pgfpathlineto{\pgfqpoint{4.062876in}{1.292289in}}%
\pgfpathlineto{\pgfqpoint{4.059704in}{1.292282in}}%
\pgfpathlineto{\pgfqpoint{4.056532in}{1.292010in}}%
\pgfpathlineto{\pgfqpoint{4.053360in}{1.291918in}}%
\pgfpathlineto{\pgfqpoint{4.050188in}{1.291949in}}%
\pgfpathlineto{\pgfqpoint{4.047016in}{1.292017in}}%
\pgfpathlineto{\pgfqpoint{4.043844in}{1.291845in}}%
\pgfpathlineto{\pgfqpoint{4.040672in}{1.291606in}}%
\pgfpathlineto{\pgfqpoint{4.037500in}{1.291000in}}%
\pgfpathlineto{\pgfqpoint{4.034328in}{1.290725in}}%
\pgfpathlineto{\pgfqpoint{4.031156in}{1.290533in}}%
\pgfpathlineto{\pgfqpoint{4.027984in}{1.290015in}}%
\pgfpathlineto{\pgfqpoint{4.024812in}{1.289456in}}%
\pgfpathlineto{\pgfqpoint{4.021640in}{1.289285in}}%
\pgfpathlineto{\pgfqpoint{4.018468in}{1.289067in}}%
\pgfpathlineto{\pgfqpoint{4.015296in}{1.288996in}}%
\pgfpathlineto{\pgfqpoint{4.012124in}{1.288606in}}%
\pgfpathlineto{\pgfqpoint{4.008952in}{1.288483in}}%
\pgfpathlineto{\pgfqpoint{4.005780in}{1.288449in}}%
\pgfpathlineto{\pgfqpoint{4.002608in}{1.288614in}}%
\pgfpathlineto{\pgfqpoint{3.999436in}{1.288722in}}%
\pgfpathlineto{\pgfqpoint{3.996264in}{1.288505in}}%
\pgfpathlineto{\pgfqpoint{3.993092in}{1.288298in}}%
\pgfpathlineto{\pgfqpoint{3.989919in}{1.287534in}}%
\pgfpathlineto{\pgfqpoint{3.986747in}{1.287287in}}%
\pgfpathlineto{\pgfqpoint{3.983575in}{1.287085in}}%
\pgfpathlineto{\pgfqpoint{3.980403in}{1.286618in}}%
\pgfpathlineto{\pgfqpoint{3.977231in}{1.287020in}}%
\pgfpathlineto{\pgfqpoint{3.974059in}{1.286941in}}%
\pgfpathlineto{\pgfqpoint{3.970887in}{1.287186in}}%
\pgfpathlineto{\pgfqpoint{3.967715in}{1.287382in}}%
\pgfpathlineto{\pgfqpoint{3.964543in}{1.287528in}}%
\pgfpathlineto{\pgfqpoint{3.961371in}{1.287675in}}%
\pgfpathlineto{\pgfqpoint{3.958199in}{1.287501in}}%
\pgfpathlineto{\pgfqpoint{3.955027in}{1.287522in}}%
\pgfpathlineto{\pgfqpoint{3.951855in}{1.287377in}}%
\pgfpathlineto{\pgfqpoint{3.948683in}{1.287142in}}%
\pgfpathlineto{\pgfqpoint{3.945511in}{1.286569in}}%
\pgfpathlineto{\pgfqpoint{3.942339in}{1.286229in}}%
\pgfpathlineto{\pgfqpoint{3.939167in}{1.286342in}}%
\pgfpathlineto{\pgfqpoint{3.935995in}{1.286201in}}%
\pgfpathlineto{\pgfqpoint{3.932823in}{1.286042in}}%
\pgfpathlineto{\pgfqpoint{3.929651in}{1.286251in}}%
\pgfpathlineto{\pgfqpoint{3.926479in}{1.286126in}}%
\pgfpathlineto{\pgfqpoint{3.923307in}{1.286099in}}%
\pgfpathlineto{\pgfqpoint{3.920135in}{1.285573in}}%
\pgfpathlineto{\pgfqpoint{3.916963in}{1.285696in}}%
\pgfpathlineto{\pgfqpoint{3.913791in}{1.285259in}}%
\pgfpathlineto{\pgfqpoint{3.910618in}{1.285548in}}%
\pgfpathlineto{\pgfqpoint{3.907446in}{1.285507in}}%
\pgfpathlineto{\pgfqpoint{3.904274in}{1.285302in}}%
\pgfpathlineto{\pgfqpoint{3.901102in}{1.284655in}}%
\pgfpathlineto{\pgfqpoint{3.897930in}{1.284465in}}%
\pgfpathlineto{\pgfqpoint{3.894758in}{1.284152in}}%
\pgfpathlineto{\pgfqpoint{3.891586in}{1.283772in}}%
\pgfpathlineto{\pgfqpoint{3.888414in}{1.283894in}}%
\pgfpathlineto{\pgfqpoint{3.885242in}{1.283749in}}%
\pgfpathlineto{\pgfqpoint{3.882070in}{1.283405in}}%
\pgfpathlineto{\pgfqpoint{3.878898in}{1.283603in}}%
\pgfpathlineto{\pgfqpoint{3.875726in}{1.283478in}}%
\pgfpathlineto{\pgfqpoint{3.872554in}{1.283044in}}%
\pgfpathlineto{\pgfqpoint{3.869382in}{1.282673in}}%
\pgfpathlineto{\pgfqpoint{3.866210in}{1.282192in}}%
\pgfpathlineto{\pgfqpoint{3.863038in}{1.281638in}}%
\pgfpathlineto{\pgfqpoint{3.859866in}{1.281590in}}%
\pgfpathlineto{\pgfqpoint{3.856694in}{1.281865in}}%
\pgfpathlineto{\pgfqpoint{3.853522in}{1.281825in}}%
\pgfpathlineto{\pgfqpoint{3.850350in}{1.281822in}}%
\pgfpathlineto{\pgfqpoint{3.847178in}{1.281950in}}%
\pgfpathlineto{\pgfqpoint{3.844006in}{1.281541in}}%
\pgfpathlineto{\pgfqpoint{3.840834in}{1.281491in}}%
\pgfpathlineto{\pgfqpoint{3.837662in}{1.280907in}}%
\pgfpathlineto{\pgfqpoint{3.834490in}{1.280808in}}%
\pgfpathlineto{\pgfqpoint{3.831317in}{1.281073in}}%
\pgfpathlineto{\pgfqpoint{3.828145in}{1.280866in}}%
\pgfpathlineto{\pgfqpoint{3.824973in}{1.281023in}}%
\pgfpathlineto{\pgfqpoint{3.821801in}{1.281355in}}%
\pgfpathlineto{\pgfqpoint{3.818629in}{1.281029in}}%
\pgfpathlineto{\pgfqpoint{3.815457in}{1.281235in}}%
\pgfpathlineto{\pgfqpoint{3.812285in}{1.281466in}}%
\pgfpathlineto{\pgfqpoint{3.809113in}{1.281632in}}%
\pgfpathlineto{\pgfqpoint{3.805941in}{1.281377in}}%
\pgfpathlineto{\pgfqpoint{3.802769in}{1.280719in}}%
\pgfpathlineto{\pgfqpoint{3.799597in}{1.280488in}}%
\pgfpathlineto{\pgfqpoint{3.796425in}{1.280256in}}%
\pgfpathlineto{\pgfqpoint{3.793253in}{1.280128in}}%
\pgfpathlineto{\pgfqpoint{3.790081in}{1.279985in}}%
\pgfpathlineto{\pgfqpoint{3.786909in}{1.279877in}}%
\pgfpathlineto{\pgfqpoint{3.783737in}{1.279446in}}%
\pgfpathlineto{\pgfqpoint{3.780565in}{1.279651in}}%
\pgfpathlineto{\pgfqpoint{3.777393in}{1.279584in}}%
\pgfpathlineto{\pgfqpoint{3.774221in}{1.279683in}}%
\pgfpathlineto{\pgfqpoint{3.771049in}{1.278991in}}%
\pgfpathlineto{\pgfqpoint{3.767877in}{1.279388in}}%
\pgfpathlineto{\pgfqpoint{3.764705in}{1.279211in}}%
\pgfpathlineto{\pgfqpoint{3.761533in}{1.279472in}}%
\pgfpathlineto{\pgfqpoint{3.758361in}{1.279381in}}%
\pgfpathlineto{\pgfqpoint{3.755188in}{1.279097in}}%
\pgfpathlineto{\pgfqpoint{3.752016in}{1.278895in}}%
\pgfpathlineto{\pgfqpoint{3.748844in}{1.279121in}}%
\pgfpathlineto{\pgfqpoint{3.745672in}{1.279444in}}%
\pgfpathlineto{\pgfqpoint{3.742500in}{1.279700in}}%
\pgfpathlineto{\pgfqpoint{3.739328in}{1.279151in}}%
\pgfpathlineto{\pgfqpoint{3.736156in}{1.279150in}}%
\pgfpathlineto{\pgfqpoint{3.732984in}{1.278932in}}%
\pgfpathlineto{\pgfqpoint{3.729812in}{1.278282in}}%
\pgfpathlineto{\pgfqpoint{3.726640in}{1.278209in}}%
\pgfpathlineto{\pgfqpoint{3.723468in}{1.277882in}}%
\pgfpathlineto{\pgfqpoint{3.720296in}{1.277973in}}%
\pgfpathlineto{\pgfqpoint{3.717124in}{1.278158in}}%
\pgfpathlineto{\pgfqpoint{3.713952in}{1.278054in}}%
\pgfpathlineto{\pgfqpoint{3.710780in}{1.278496in}}%
\pgfpathlineto{\pgfqpoint{3.707608in}{1.278465in}}%
\pgfpathlineto{\pgfqpoint{3.704436in}{1.278138in}}%
\pgfpathlineto{\pgfqpoint{3.701264in}{1.277892in}}%
\pgfpathlineto{\pgfqpoint{3.698092in}{1.277311in}}%
\pgfpathlineto{\pgfqpoint{3.694920in}{1.277348in}}%
\pgfpathlineto{\pgfqpoint{3.691748in}{1.277274in}}%
\pgfpathlineto{\pgfqpoint{3.688576in}{1.277664in}}%
\pgfpathlineto{\pgfqpoint{3.685404in}{1.277449in}}%
\pgfpathlineto{\pgfqpoint{3.682232in}{1.277182in}}%
\pgfpathlineto{\pgfqpoint{3.679060in}{1.277379in}}%
\pgfpathlineto{\pgfqpoint{3.675887in}{1.277060in}}%
\pgfpathlineto{\pgfqpoint{3.672715in}{1.277078in}}%
\pgfpathlineto{\pgfqpoint{3.669543in}{1.277175in}}%
\pgfpathlineto{\pgfqpoint{3.666371in}{1.277322in}}%
\pgfpathlineto{\pgfqpoint{3.663199in}{1.277716in}}%
\pgfpathlineto{\pgfqpoint{3.660027in}{1.277700in}}%
\pgfpathlineto{\pgfqpoint{3.656855in}{1.277427in}}%
\pgfpathlineto{\pgfqpoint{3.653683in}{1.276982in}}%
\pgfpathlineto{\pgfqpoint{3.650511in}{1.276437in}}%
\pgfpathlineto{\pgfqpoint{3.647339in}{1.275871in}}%
\pgfpathlineto{\pgfqpoint{3.644167in}{1.275516in}}%
\pgfpathlineto{\pgfqpoint{3.640995in}{1.275346in}}%
\pgfpathlineto{\pgfqpoint{3.637823in}{1.275033in}}%
\pgfpathlineto{\pgfqpoint{3.634651in}{1.274944in}}%
\pgfpathlineto{\pgfqpoint{3.631479in}{1.275033in}}%
\pgfpathlineto{\pgfqpoint{3.628307in}{1.274895in}}%
\pgfpathlineto{\pgfqpoint{3.625135in}{1.274660in}}%
\pgfpathlineto{\pgfqpoint{3.621963in}{1.274497in}}%
\pgfpathlineto{\pgfqpoint{3.618791in}{1.274339in}}%
\pgfpathlineto{\pgfqpoint{3.615619in}{1.274424in}}%
\pgfpathlineto{\pgfqpoint{3.612447in}{1.274331in}}%
\pgfpathlineto{\pgfqpoint{3.609275in}{1.274116in}}%
\pgfpathlineto{\pgfqpoint{3.606103in}{1.274030in}}%
\pgfpathlineto{\pgfqpoint{3.602931in}{1.273663in}}%
\pgfpathlineto{\pgfqpoint{3.599759in}{1.273649in}}%
\pgfpathlineto{\pgfqpoint{3.596586in}{1.273149in}}%
\pgfpathlineto{\pgfqpoint{3.593414in}{1.273000in}}%
\pgfpathlineto{\pgfqpoint{3.590242in}{1.272871in}}%
\pgfpathlineto{\pgfqpoint{3.587070in}{1.272832in}}%
\pgfpathlineto{\pgfqpoint{3.583898in}{1.272843in}}%
\pgfpathlineto{\pgfqpoint{3.580726in}{1.272488in}}%
\pgfpathlineto{\pgfqpoint{3.577554in}{1.272603in}}%
\pgfpathlineto{\pgfqpoint{3.574382in}{1.272269in}}%
\pgfpathlineto{\pgfqpoint{3.571210in}{1.272083in}}%
\pgfpathlineto{\pgfqpoint{3.568038in}{1.271920in}}%
\pgfpathlineto{\pgfqpoint{3.564866in}{1.272048in}}%
\pgfpathlineto{\pgfqpoint{3.561694in}{1.272176in}}%
\pgfpathlineto{\pgfqpoint{3.558522in}{1.272135in}}%
\pgfpathlineto{\pgfqpoint{3.555350in}{1.271988in}}%
\pgfpathlineto{\pgfqpoint{3.552178in}{1.272080in}}%
\pgfpathlineto{\pgfqpoint{3.549006in}{1.272210in}}%
\pgfpathlineto{\pgfqpoint{3.545834in}{1.272454in}}%
\pgfpathlineto{\pgfqpoint{3.542662in}{1.272720in}}%
\pgfpathlineto{\pgfqpoint{3.539490in}{1.272979in}}%
\pgfpathlineto{\pgfqpoint{3.536318in}{1.272959in}}%
\pgfpathlineto{\pgfqpoint{3.533146in}{1.272902in}}%
\pgfpathlineto{\pgfqpoint{3.529974in}{1.272674in}}%
\pgfpathlineto{\pgfqpoint{3.526802in}{1.272787in}}%
\pgfpathlineto{\pgfqpoint{3.523630in}{1.272897in}}%
\pgfpathlineto{\pgfqpoint{3.520457in}{1.272908in}}%
\pgfpathlineto{\pgfqpoint{3.517285in}{1.273051in}}%
\pgfpathlineto{\pgfqpoint{3.514113in}{1.272868in}}%
\pgfpathlineto{\pgfqpoint{3.510941in}{1.273073in}}%
\pgfpathlineto{\pgfqpoint{3.507769in}{1.273280in}}%
\pgfpathlineto{\pgfqpoint{3.504597in}{1.273296in}}%
\pgfpathlineto{\pgfqpoint{3.501425in}{1.273366in}}%
\pgfpathlineto{\pgfqpoint{3.498253in}{1.273288in}}%
\pgfpathlineto{\pgfqpoint{3.495081in}{1.273310in}}%
\pgfpathlineto{\pgfqpoint{3.491909in}{1.273393in}}%
\pgfpathlineto{\pgfqpoint{3.488737in}{1.272620in}}%
\pgfpathlineto{\pgfqpoint{3.485565in}{1.272400in}}%
\pgfpathlineto{\pgfqpoint{3.482393in}{1.272110in}}%
\pgfpathlineto{\pgfqpoint{3.479221in}{1.272181in}}%
\pgfpathlineto{\pgfqpoint{3.476049in}{1.271773in}}%
\pgfpathlineto{\pgfqpoint{3.472877in}{1.271675in}}%
\pgfpathlineto{\pgfqpoint{3.469705in}{1.271382in}}%
\pgfpathlineto{\pgfqpoint{3.466533in}{1.271508in}}%
\pgfpathlineto{\pgfqpoint{3.463361in}{1.271517in}}%
\pgfpathlineto{\pgfqpoint{3.460189in}{1.271629in}}%
\pgfpathlineto{\pgfqpoint{3.457017in}{1.271693in}}%
\pgfpathlineto{\pgfqpoint{3.453845in}{1.271384in}}%
\pgfpathlineto{\pgfqpoint{3.450673in}{1.271352in}}%
\pgfpathlineto{\pgfqpoint{3.447501in}{1.270935in}}%
\pgfpathlineto{\pgfqpoint{3.444329in}{1.270652in}}%
\pgfpathlineto{\pgfqpoint{3.441156in}{1.270374in}}%
\pgfpathlineto{\pgfqpoint{3.437984in}{1.270668in}}%
\pgfpathlineto{\pgfqpoint{3.434812in}{1.270488in}}%
\pgfpathlineto{\pgfqpoint{3.431640in}{1.270478in}}%
\pgfpathlineto{\pgfqpoint{3.428468in}{1.270546in}}%
\pgfpathlineto{\pgfqpoint{3.425296in}{1.270263in}}%
\pgfpathlineto{\pgfqpoint{3.422124in}{1.270248in}}%
\pgfpathlineto{\pgfqpoint{3.418952in}{1.270155in}}%
\pgfpathlineto{\pgfqpoint{3.415780in}{1.269837in}}%
\pgfpathlineto{\pgfqpoint{3.412608in}{1.269438in}}%
\pgfpathlineto{\pgfqpoint{3.409436in}{1.268977in}}%
\pgfpathlineto{\pgfqpoint{3.406264in}{1.268827in}}%
\pgfpathlineto{\pgfqpoint{3.403092in}{1.268818in}}%
\pgfpathlineto{\pgfqpoint{3.399920in}{1.268231in}}%
\pgfpathlineto{\pgfqpoint{3.396748in}{1.267496in}}%
\pgfpathlineto{\pgfqpoint{3.393576in}{1.267623in}}%
\pgfpathlineto{\pgfqpoint{3.390404in}{1.267420in}}%
\pgfpathlineto{\pgfqpoint{3.387232in}{1.267333in}}%
\pgfpathlineto{\pgfqpoint{3.384060in}{1.267776in}}%
\pgfpathlineto{\pgfqpoint{3.380888in}{1.267970in}}%
\pgfpathlineto{\pgfqpoint{3.377716in}{1.268160in}}%
\pgfpathlineto{\pgfqpoint{3.374544in}{1.267922in}}%
\pgfpathlineto{\pgfqpoint{3.371372in}{1.267147in}}%
\pgfpathlineto{\pgfqpoint{3.368200in}{1.267298in}}%
\pgfpathlineto{\pgfqpoint{3.365028in}{1.266897in}}%
\pgfpathlineto{\pgfqpoint{3.361855in}{1.266627in}}%
\pgfpathlineto{\pgfqpoint{3.358683in}{1.266625in}}%
\pgfpathlineto{\pgfqpoint{3.355511in}{1.266157in}}%
\pgfpathlineto{\pgfqpoint{3.352339in}{1.266396in}}%
\pgfpathlineto{\pgfqpoint{3.349167in}{1.265876in}}%
\pgfpathlineto{\pgfqpoint{3.345995in}{1.266078in}}%
\pgfpathlineto{\pgfqpoint{3.342823in}{1.265931in}}%
\pgfpathlineto{\pgfqpoint{3.339651in}{1.265790in}}%
\pgfpathlineto{\pgfqpoint{3.336479in}{1.265528in}}%
\pgfpathlineto{\pgfqpoint{3.333307in}{1.265188in}}%
\pgfpathlineto{\pgfqpoint{3.330135in}{1.265101in}}%
\pgfpathlineto{\pgfqpoint{3.326963in}{1.265334in}}%
\pgfpathlineto{\pgfqpoint{3.323791in}{1.264475in}}%
\pgfpathlineto{\pgfqpoint{3.320619in}{1.264634in}}%
\pgfpathlineto{\pgfqpoint{3.317447in}{1.264402in}}%
\pgfpathlineto{\pgfqpoint{3.314275in}{1.264402in}}%
\pgfpathlineto{\pgfqpoint{3.311103in}{1.264228in}}%
\pgfpathlineto{\pgfqpoint{3.307931in}{1.264042in}}%
\pgfpathlineto{\pgfqpoint{3.304759in}{1.264064in}}%
\pgfpathlineto{\pgfqpoint{3.301587in}{1.264274in}}%
\pgfpathlineto{\pgfqpoint{3.298415in}{1.264416in}}%
\pgfpathlineto{\pgfqpoint{3.295243in}{1.264538in}}%
\pgfpathlineto{\pgfqpoint{3.292071in}{1.264792in}}%
\pgfpathlineto{\pgfqpoint{3.288899in}{1.264446in}}%
\pgfpathlineto{\pgfqpoint{3.285726in}{1.264345in}}%
\pgfpathlineto{\pgfqpoint{3.282554in}{1.264412in}}%
\pgfpathlineto{\pgfqpoint{3.279382in}{1.263917in}}%
\pgfpathlineto{\pgfqpoint{3.276210in}{1.264049in}}%
\pgfpathlineto{\pgfqpoint{3.273038in}{1.263710in}}%
\pgfpathlineto{\pgfqpoint{3.269866in}{1.263650in}}%
\pgfpathlineto{\pgfqpoint{3.266694in}{1.263358in}}%
\pgfpathlineto{\pgfqpoint{3.263522in}{1.263221in}}%
\pgfpathlineto{\pgfqpoint{3.260350in}{1.262650in}}%
\pgfpathlineto{\pgfqpoint{3.257178in}{1.262502in}}%
\pgfpathlineto{\pgfqpoint{3.254006in}{1.262140in}}%
\pgfpathlineto{\pgfqpoint{3.250834in}{1.262501in}}%
\pgfpathlineto{\pgfqpoint{3.247662in}{1.262422in}}%
\pgfpathlineto{\pgfqpoint{3.244490in}{1.262392in}}%
\pgfpathlineto{\pgfqpoint{3.241318in}{1.263892in}}%
\pgfpathlineto{\pgfqpoint{3.238146in}{1.263484in}}%
\pgfpathlineto{\pgfqpoint{3.234974in}{1.263003in}}%
\pgfpathlineto{\pgfqpoint{3.231802in}{1.263667in}}%
\pgfpathlineto{\pgfqpoint{3.228630in}{1.262438in}}%
\pgfpathlineto{\pgfqpoint{3.225458in}{1.262067in}}%
\pgfpathlineto{\pgfqpoint{3.222286in}{1.261687in}}%
\pgfpathlineto{\pgfqpoint{3.219114in}{1.261408in}}%
\pgfpathlineto{\pgfqpoint{3.215942in}{1.261149in}}%
\pgfpathlineto{\pgfqpoint{3.212770in}{1.260944in}}%
\pgfpathlineto{\pgfqpoint{3.209598in}{1.260005in}}%
\pgfpathlineto{\pgfqpoint{3.206425in}{1.260166in}}%
\pgfpathlineto{\pgfqpoint{3.203253in}{1.260376in}}%
\pgfpathlineto{\pgfqpoint{3.200081in}{1.259157in}}%
\pgfpathlineto{\pgfqpoint{3.196909in}{1.259120in}}%
\pgfpathlineto{\pgfqpoint{3.193737in}{1.258935in}}%
\pgfpathlineto{\pgfqpoint{3.190565in}{1.258587in}}%
\pgfpathlineto{\pgfqpoint{3.187393in}{1.258634in}}%
\pgfpathlineto{\pgfqpoint{3.184221in}{1.257309in}}%
\pgfpathlineto{\pgfqpoint{3.181049in}{1.257487in}}%
\pgfpathlineto{\pgfqpoint{3.177877in}{1.256590in}}%
\pgfpathlineto{\pgfqpoint{3.174705in}{1.256163in}}%
\pgfpathlineto{\pgfqpoint{3.171533in}{1.256497in}}%
\pgfpathlineto{\pgfqpoint{3.168361in}{1.256374in}}%
\pgfpathlineto{\pgfqpoint{3.165189in}{1.256207in}}%
\pgfpathlineto{\pgfqpoint{3.162017in}{1.256190in}}%
\pgfpathlineto{\pgfqpoint{3.158845in}{1.256296in}}%
\pgfpathlineto{\pgfqpoint{3.155673in}{1.256179in}}%
\pgfpathlineto{\pgfqpoint{3.152501in}{1.256221in}}%
\pgfpathlineto{\pgfqpoint{3.149329in}{1.254680in}}%
\pgfpathlineto{\pgfqpoint{3.146157in}{1.254657in}}%
\pgfpathlineto{\pgfqpoint{3.142985in}{1.254368in}}%
\pgfpathlineto{\pgfqpoint{3.139813in}{1.254718in}}%
\pgfpathlineto{\pgfqpoint{3.136641in}{1.254936in}}%
\pgfpathlineto{\pgfqpoint{3.133469in}{1.255180in}}%
\pgfpathlineto{\pgfqpoint{3.130297in}{1.255162in}}%
\pgfpathlineto{\pgfqpoint{3.127124in}{1.255210in}}%
\pgfpathlineto{\pgfqpoint{3.123952in}{1.255095in}}%
\pgfpathlineto{\pgfqpoint{3.120780in}{1.255033in}}%
\pgfpathlineto{\pgfqpoint{3.117608in}{1.255052in}}%
\pgfpathlineto{\pgfqpoint{3.114436in}{1.255114in}}%
\pgfpathlineto{\pgfqpoint{3.111264in}{1.255012in}}%
\pgfpathlineto{\pgfqpoint{3.108092in}{1.255144in}}%
\pgfpathlineto{\pgfqpoint{3.104920in}{1.255416in}}%
\pgfpathlineto{\pgfqpoint{3.101748in}{1.255408in}}%
\pgfpathlineto{\pgfqpoint{3.098576in}{1.255226in}}%
\pgfpathlineto{\pgfqpoint{3.095404in}{1.255171in}}%
\pgfpathlineto{\pgfqpoint{3.092232in}{1.255335in}}%
\pgfpathlineto{\pgfqpoint{3.089060in}{1.255198in}}%
\pgfpathlineto{\pgfqpoint{3.085888in}{1.254998in}}%
\pgfpathlineto{\pgfqpoint{3.082716in}{1.254846in}}%
\pgfpathlineto{\pgfqpoint{3.079544in}{1.255031in}}%
\pgfpathlineto{\pgfqpoint{3.076372in}{1.254842in}}%
\pgfpathlineto{\pgfqpoint{3.073200in}{1.254641in}}%
\pgfpathlineto{\pgfqpoint{3.070028in}{1.254741in}}%
\pgfpathlineto{\pgfqpoint{3.066856in}{1.254840in}}%
\pgfpathlineto{\pgfqpoint{3.063684in}{1.254738in}}%
\pgfpathlineto{\pgfqpoint{3.060512in}{1.254751in}}%
\pgfpathlineto{\pgfqpoint{3.057340in}{1.254838in}}%
\pgfpathlineto{\pgfqpoint{3.054168in}{1.254919in}}%
\pgfpathlineto{\pgfqpoint{3.050995in}{1.254847in}}%
\pgfpathlineto{\pgfqpoint{3.047823in}{1.254707in}}%
\pgfpathlineto{\pgfqpoint{3.044651in}{1.254620in}}%
\pgfpathlineto{\pgfqpoint{3.041479in}{1.254578in}}%
\pgfpathlineto{\pgfqpoint{3.038307in}{1.254446in}}%
\pgfpathlineto{\pgfqpoint{3.035135in}{1.254394in}}%
\pgfpathlineto{\pgfqpoint{3.031963in}{1.254452in}}%
\pgfpathlineto{\pgfqpoint{3.028791in}{1.254401in}}%
\pgfpathlineto{\pgfqpoint{3.025619in}{1.254218in}}%
\pgfpathlineto{\pgfqpoint{3.022447in}{1.254302in}}%
\pgfpathlineto{\pgfqpoint{3.019275in}{1.254004in}}%
\pgfpathlineto{\pgfqpoint{3.016103in}{1.254067in}}%
\pgfpathlineto{\pgfqpoint{3.012931in}{1.254111in}}%
\pgfpathlineto{\pgfqpoint{3.009759in}{1.254005in}}%
\pgfpathlineto{\pgfqpoint{3.006587in}{1.253895in}}%
\pgfpathlineto{\pgfqpoint{3.003415in}{1.253956in}}%
\pgfpathlineto{\pgfqpoint{3.000243in}{1.253955in}}%
\pgfpathlineto{\pgfqpoint{2.997071in}{1.253923in}}%
\pgfpathlineto{\pgfqpoint{2.993899in}{1.253786in}}%
\pgfpathlineto{\pgfqpoint{2.990727in}{1.253669in}}%
\pgfpathlineto{\pgfqpoint{2.987555in}{1.253608in}}%
\pgfpathlineto{\pgfqpoint{2.984383in}{1.253516in}}%
\pgfpathlineto{\pgfqpoint{2.981211in}{1.253503in}}%
\pgfpathlineto{\pgfqpoint{2.978039in}{1.253477in}}%
\pgfpathlineto{\pgfqpoint{2.974867in}{1.253587in}}%
\pgfpathlineto{\pgfqpoint{2.971694in}{1.253625in}}%
\pgfpathlineto{\pgfqpoint{2.968522in}{1.253800in}}%
\pgfpathlineto{\pgfqpoint{2.965350in}{1.253854in}}%
\pgfpathlineto{\pgfqpoint{2.962178in}{1.253737in}}%
\pgfpathlineto{\pgfqpoint{2.959006in}{1.253754in}}%
\pgfpathlineto{\pgfqpoint{2.955834in}{1.253207in}}%
\pgfpathlineto{\pgfqpoint{2.952662in}{1.253093in}}%
\pgfpathlineto{\pgfqpoint{2.949490in}{1.253012in}}%
\pgfpathlineto{\pgfqpoint{2.946318in}{1.253161in}}%
\pgfpathlineto{\pgfqpoint{2.943146in}{1.253115in}}%
\pgfpathlineto{\pgfqpoint{2.939974in}{1.252903in}}%
\pgfpathlineto{\pgfqpoint{2.936802in}{1.252778in}}%
\pgfpathlineto{\pgfqpoint{2.933630in}{1.253016in}}%
\pgfpathlineto{\pgfqpoint{2.930458in}{1.253177in}}%
\pgfpathlineto{\pgfqpoint{2.927286in}{1.253031in}}%
\pgfpathlineto{\pgfqpoint{2.924114in}{1.252793in}}%
\pgfpathlineto{\pgfqpoint{2.920942in}{1.252974in}}%
\pgfpathlineto{\pgfqpoint{2.917770in}{1.252881in}}%
\pgfpathlineto{\pgfqpoint{2.914598in}{1.252850in}}%
\pgfpathlineto{\pgfqpoint{2.911426in}{1.252739in}}%
\pgfpathlineto{\pgfqpoint{2.908254in}{1.252754in}}%
\pgfpathlineto{\pgfqpoint{2.905082in}{1.252690in}}%
\pgfpathlineto{\pgfqpoint{2.901910in}{1.252686in}}%
\pgfpathlineto{\pgfqpoint{2.898738in}{1.252662in}}%
\pgfpathlineto{\pgfqpoint{2.895565in}{1.252449in}}%
\pgfpathlineto{\pgfqpoint{2.892393in}{1.252630in}}%
\pgfpathlineto{\pgfqpoint{2.889221in}{1.252497in}}%
\pgfpathlineto{\pgfqpoint{2.886049in}{1.252538in}}%
\pgfpathlineto{\pgfqpoint{2.882877in}{1.252468in}}%
\pgfpathlineto{\pgfqpoint{2.879705in}{1.252487in}}%
\pgfpathlineto{\pgfqpoint{2.876533in}{1.252434in}}%
\pgfpathlineto{\pgfqpoint{2.873361in}{1.252473in}}%
\pgfpathlineto{\pgfqpoint{2.870189in}{1.252454in}}%
\pgfpathlineto{\pgfqpoint{2.867017in}{1.252397in}}%
\pgfpathlineto{\pgfqpoint{2.863845in}{1.252467in}}%
\pgfpathlineto{\pgfqpoint{2.860673in}{1.252283in}}%
\pgfpathlineto{\pgfqpoint{2.857501in}{1.252188in}}%
\pgfpathlineto{\pgfqpoint{2.854329in}{1.252227in}}%
\pgfpathlineto{\pgfqpoint{2.851157in}{1.252200in}}%
\pgfpathlineto{\pgfqpoint{2.847985in}{1.252157in}}%
\pgfpathlineto{\pgfqpoint{2.844813in}{1.252203in}}%
\pgfpathlineto{\pgfqpoint{2.841641in}{1.252198in}}%
\pgfpathlineto{\pgfqpoint{2.838469in}{1.252378in}}%
\pgfpathlineto{\pgfqpoint{2.835297in}{1.252378in}}%
\pgfpathlineto{\pgfqpoint{2.832125in}{1.252195in}}%
\pgfpathlineto{\pgfqpoint{2.828953in}{1.252048in}}%
\pgfpathlineto{\pgfqpoint{2.825781in}{1.252032in}}%
\pgfpathlineto{\pgfqpoint{2.822609in}{1.251932in}}%
\pgfpathlineto{\pgfqpoint{2.819437in}{1.251943in}}%
\pgfpathlineto{\pgfqpoint{2.816264in}{1.252145in}}%
\pgfpathlineto{\pgfqpoint{2.813092in}{1.252234in}}%
\pgfpathlineto{\pgfqpoint{2.809920in}{1.252009in}}%
\pgfpathlineto{\pgfqpoint{2.806748in}{1.251969in}}%
\pgfpathlineto{\pgfqpoint{2.803576in}{1.251989in}}%
\pgfpathlineto{\pgfqpoint{2.800404in}{1.251932in}}%
\pgfpathlineto{\pgfqpoint{2.797232in}{1.251868in}}%
\pgfpathlineto{\pgfqpoint{2.794060in}{1.251835in}}%
\pgfpathlineto{\pgfqpoint{2.790888in}{1.251768in}}%
\pgfpathlineto{\pgfqpoint{2.787716in}{1.251717in}}%
\pgfpathlineto{\pgfqpoint{2.784544in}{1.251549in}}%
\pgfpathlineto{\pgfqpoint{2.781372in}{1.251407in}}%
\pgfpathlineto{\pgfqpoint{2.778200in}{1.251324in}}%
\pgfpathlineto{\pgfqpoint{2.775028in}{1.251157in}}%
\pgfpathlineto{\pgfqpoint{2.771856in}{1.250993in}}%
\pgfpathlineto{\pgfqpoint{2.768684in}{1.250891in}}%
\pgfpathlineto{\pgfqpoint{2.765512in}{1.250855in}}%
\pgfpathlineto{\pgfqpoint{2.762340in}{1.251116in}}%
\pgfpathlineto{\pgfqpoint{2.759168in}{1.251287in}}%
\pgfpathlineto{\pgfqpoint{2.755996in}{1.251308in}}%
\pgfpathlineto{\pgfqpoint{2.752824in}{1.251197in}}%
\pgfpathlineto{\pgfqpoint{2.749652in}{1.251102in}}%
\pgfpathlineto{\pgfqpoint{2.746480in}{1.251205in}}%
\pgfpathlineto{\pgfqpoint{2.743308in}{1.251271in}}%
\pgfpathlineto{\pgfqpoint{2.740136in}{1.251338in}}%
\pgfpathlineto{\pgfqpoint{2.736963in}{1.251557in}}%
\pgfpathlineto{\pgfqpoint{2.733791in}{1.251669in}}%
\pgfpathlineto{\pgfqpoint{2.730619in}{1.251990in}}%
\pgfpathlineto{\pgfqpoint{2.727447in}{1.251751in}}%
\pgfpathlineto{\pgfqpoint{2.724275in}{1.251485in}}%
\pgfpathlineto{\pgfqpoint{2.721103in}{1.251440in}}%
\pgfpathlineto{\pgfqpoint{2.717931in}{1.251464in}}%
\pgfpathlineto{\pgfqpoint{2.714759in}{1.251350in}}%
\pgfpathlineto{\pgfqpoint{2.711587in}{1.251021in}}%
\pgfpathlineto{\pgfqpoint{2.708415in}{1.250957in}}%
\pgfpathlineto{\pgfqpoint{2.705243in}{1.250898in}}%
\pgfpathlineto{\pgfqpoint{2.702071in}{1.250590in}}%
\pgfpathlineto{\pgfqpoint{2.698899in}{1.250790in}}%
\pgfpathlineto{\pgfqpoint{2.695727in}{1.250707in}}%
\pgfpathlineto{\pgfqpoint{2.692555in}{1.250919in}}%
\pgfpathlineto{\pgfqpoint{2.689383in}{1.250671in}}%
\pgfpathlineto{\pgfqpoint{2.686211in}{1.250640in}}%
\pgfpathlineto{\pgfqpoint{2.683039in}{1.250648in}}%
\pgfpathlineto{\pgfqpoint{2.679867in}{1.250678in}}%
\pgfpathlineto{\pgfqpoint{2.676695in}{1.250621in}}%
\pgfpathlineto{\pgfqpoint{2.673523in}{1.250640in}}%
\pgfpathlineto{\pgfqpoint{2.670351in}{1.250545in}}%
\pgfpathlineto{\pgfqpoint{2.667179in}{1.250335in}}%
\pgfpathlineto{\pgfqpoint{2.664007in}{1.250149in}}%
\pgfpathlineto{\pgfqpoint{2.660834in}{1.250076in}}%
\pgfpathlineto{\pgfqpoint{2.657662in}{1.250132in}}%
\pgfpathlineto{\pgfqpoint{2.654490in}{1.250059in}}%
\pgfpathlineto{\pgfqpoint{2.651318in}{1.250156in}}%
\pgfpathlineto{\pgfqpoint{2.648146in}{1.250128in}}%
\pgfpathlineto{\pgfqpoint{2.644974in}{1.249941in}}%
\pgfpathlineto{\pgfqpoint{2.641802in}{1.249744in}}%
\pgfpathlineto{\pgfqpoint{2.638630in}{1.249664in}}%
\pgfpathlineto{\pgfqpoint{2.635458in}{1.249411in}}%
\pgfpathlineto{\pgfqpoint{2.632286in}{1.249181in}}%
\pgfpathlineto{\pgfqpoint{2.629114in}{1.249098in}}%
\pgfpathlineto{\pgfqpoint{2.625942in}{1.248948in}}%
\pgfpathlineto{\pgfqpoint{2.622770in}{1.248678in}}%
\pgfpathlineto{\pgfqpoint{2.619598in}{1.248837in}}%
\pgfpathlineto{\pgfqpoint{2.616426in}{1.249013in}}%
\pgfpathlineto{\pgfqpoint{2.613254in}{1.249040in}}%
\pgfpathlineto{\pgfqpoint{2.610082in}{1.248981in}}%
\pgfpathlineto{\pgfqpoint{2.606910in}{1.248966in}}%
\pgfpathlineto{\pgfqpoint{2.603738in}{1.248906in}}%
\pgfpathlineto{\pgfqpoint{2.600566in}{1.249023in}}%
\pgfpathlineto{\pgfqpoint{2.597394in}{1.249007in}}%
\pgfpathlineto{\pgfqpoint{2.594222in}{1.249063in}}%
\pgfpathlineto{\pgfqpoint{2.591050in}{1.249079in}}%
\pgfpathlineto{\pgfqpoint{2.587878in}{1.248912in}}%
\pgfpathlineto{\pgfqpoint{2.584706in}{1.248812in}}%
\pgfpathlineto{\pgfqpoint{2.581533in}{1.248931in}}%
\pgfpathlineto{\pgfqpoint{2.578361in}{1.248890in}}%
\pgfpathlineto{\pgfqpoint{2.575189in}{1.248571in}}%
\pgfpathlineto{\pgfqpoint{2.572017in}{1.248588in}}%
\pgfpathlineto{\pgfqpoint{2.568845in}{1.248489in}}%
\pgfpathlineto{\pgfqpoint{2.565673in}{1.248184in}}%
\pgfpathlineto{\pgfqpoint{2.562501in}{1.247957in}}%
\pgfpathlineto{\pgfqpoint{2.559329in}{1.247839in}}%
\pgfpathlineto{\pgfqpoint{2.556157in}{1.247873in}}%
\pgfpathlineto{\pgfqpoint{2.552985in}{1.247957in}}%
\pgfpathlineto{\pgfqpoint{2.549813in}{1.247840in}}%
\pgfpathlineto{\pgfqpoint{2.546641in}{1.247989in}}%
\pgfpathlineto{\pgfqpoint{2.543469in}{1.247905in}}%
\pgfpathlineto{\pgfqpoint{2.540297in}{1.247947in}}%
\pgfpathlineto{\pgfqpoint{2.537125in}{1.248006in}}%
\pgfpathlineto{\pgfqpoint{2.533953in}{1.248016in}}%
\pgfpathlineto{\pgfqpoint{2.530781in}{1.248084in}}%
\pgfpathlineto{\pgfqpoint{2.527609in}{1.247995in}}%
\pgfpathlineto{\pgfqpoint{2.524437in}{1.247813in}}%
\pgfpathlineto{\pgfqpoint{2.521265in}{1.247825in}}%
\pgfpathlineto{\pgfqpoint{2.518093in}{1.247886in}}%
\pgfpathlineto{\pgfqpoint{2.514921in}{1.247887in}}%
\pgfpathlineto{\pgfqpoint{2.511749in}{1.247671in}}%
\pgfpathlineto{\pgfqpoint{2.508577in}{1.247816in}}%
\pgfpathlineto{\pgfqpoint{2.505405in}{1.247796in}}%
\pgfpathlineto{\pgfqpoint{2.502232in}{1.247792in}}%
\pgfpathlineto{\pgfqpoint{2.499060in}{1.247589in}}%
\pgfpathlineto{\pgfqpoint{2.495888in}{1.247594in}}%
\pgfpathlineto{\pgfqpoint{2.492716in}{1.247728in}}%
\pgfpathlineto{\pgfqpoint{2.489544in}{1.247805in}}%
\pgfpathlineto{\pgfqpoint{2.486372in}{1.247550in}}%
\pgfpathlineto{\pgfqpoint{2.483200in}{1.247264in}}%
\pgfpathlineto{\pgfqpoint{2.480028in}{1.247210in}}%
\pgfpathlineto{\pgfqpoint{2.476856in}{1.247068in}}%
\pgfpathlineto{\pgfqpoint{2.473684in}{1.246926in}}%
\pgfpathlineto{\pgfqpoint{2.470512in}{1.246822in}}%
\pgfpathlineto{\pgfqpoint{2.467340in}{1.246719in}}%
\pgfpathlineto{\pgfqpoint{2.464168in}{1.246665in}}%
\pgfpathlineto{\pgfqpoint{2.460996in}{1.246615in}}%
\pgfpathlineto{\pgfqpoint{2.457824in}{1.246477in}}%
\pgfpathlineto{\pgfqpoint{2.454652in}{1.246489in}}%
\pgfpathlineto{\pgfqpoint{2.451480in}{1.246547in}}%
\pgfpathlineto{\pgfqpoint{2.448308in}{1.246462in}}%
\pgfpathlineto{\pgfqpoint{2.445136in}{1.246273in}}%
\pgfpathlineto{\pgfqpoint{2.441964in}{1.246306in}}%
\pgfpathlineto{\pgfqpoint{2.438792in}{1.246097in}}%
\pgfpathlineto{\pgfqpoint{2.435620in}{1.246184in}}%
\pgfpathlineto{\pgfqpoint{2.432448in}{1.246052in}}%
\pgfpathlineto{\pgfqpoint{2.429276in}{1.246024in}}%
\pgfpathlineto{\pgfqpoint{2.426103in}{1.246133in}}%
\pgfpathlineto{\pgfqpoint{2.422931in}{1.246162in}}%
\pgfpathlineto{\pgfqpoint{2.419759in}{1.246293in}}%
\pgfpathlineto{\pgfqpoint{2.416587in}{1.246157in}}%
\pgfpathlineto{\pgfqpoint{2.413415in}{1.246155in}}%
\pgfpathlineto{\pgfqpoint{2.410243in}{1.246249in}}%
\pgfpathlineto{\pgfqpoint{2.407071in}{1.246444in}}%
\pgfpathlineto{\pgfqpoint{2.403899in}{1.246402in}}%
\pgfpathlineto{\pgfqpoint{2.400727in}{1.246426in}}%
\pgfpathlineto{\pgfqpoint{2.397555in}{1.246291in}}%
\pgfpathlineto{\pgfqpoint{2.394383in}{1.246353in}}%
\pgfpathlineto{\pgfqpoint{2.391211in}{1.246493in}}%
\pgfpathlineto{\pgfqpoint{2.388039in}{1.246157in}}%
\pgfpathlineto{\pgfqpoint{2.384867in}{1.246142in}}%
\pgfpathlineto{\pgfqpoint{2.381695in}{1.246166in}}%
\pgfpathlineto{\pgfqpoint{2.378523in}{1.246225in}}%
\pgfpathlineto{\pgfqpoint{2.375351in}{1.245907in}}%
\pgfpathlineto{\pgfqpoint{2.372179in}{1.245622in}}%
\pgfpathlineto{\pgfqpoint{2.369007in}{1.245776in}}%
\pgfpathlineto{\pgfqpoint{2.365835in}{1.245892in}}%
\pgfpathlineto{\pgfqpoint{2.362663in}{1.245965in}}%
\pgfpathlineto{\pgfqpoint{2.359491in}{1.245781in}}%
\pgfpathlineto{\pgfqpoint{2.356319in}{1.245597in}}%
\pgfpathlineto{\pgfqpoint{2.353147in}{1.245674in}}%
\pgfpathlineto{\pgfqpoint{2.349975in}{1.245719in}}%
\pgfpathlineto{\pgfqpoint{2.346802in}{1.245689in}}%
\pgfpathlineto{\pgfqpoint{2.343630in}{1.245648in}}%
\pgfpathlineto{\pgfqpoint{2.340458in}{1.245941in}}%
\pgfpathlineto{\pgfqpoint{2.337286in}{1.245960in}}%
\pgfpathlineto{\pgfqpoint{2.334114in}{1.245871in}}%
\pgfpathlineto{\pgfqpoint{2.330942in}{1.245987in}}%
\pgfpathlineto{\pgfqpoint{2.327770in}{1.245896in}}%
\pgfpathlineto{\pgfqpoint{2.324598in}{1.246108in}}%
\pgfpathlineto{\pgfqpoint{2.321426in}{1.245780in}}%
\pgfpathlineto{\pgfqpoint{2.318254in}{1.245836in}}%
\pgfpathlineto{\pgfqpoint{2.315082in}{1.245936in}}%
\pgfpathlineto{\pgfqpoint{2.311910in}{1.245881in}}%
\pgfpathlineto{\pgfqpoint{2.308738in}{1.245871in}}%
\pgfpathlineto{\pgfqpoint{2.305566in}{1.245756in}}%
\pgfpathlineto{\pgfqpoint{2.302394in}{1.245994in}}%
\pgfpathlineto{\pgfqpoint{2.299222in}{1.245937in}}%
\pgfpathlineto{\pgfqpoint{2.296050in}{1.245929in}}%
\pgfpathlineto{\pgfqpoint{2.292878in}{1.245964in}}%
\pgfpathlineto{\pgfqpoint{2.289706in}{1.245923in}}%
\pgfpathlineto{\pgfqpoint{2.286534in}{1.245687in}}%
\pgfpathlineto{\pgfqpoint{2.283362in}{1.245683in}}%
\pgfpathlineto{\pgfqpoint{2.280190in}{1.245624in}}%
\pgfpathlineto{\pgfqpoint{2.277018in}{1.245767in}}%
\pgfpathlineto{\pgfqpoint{2.273846in}{1.245700in}}%
\pgfpathlineto{\pgfqpoint{2.270674in}{1.245799in}}%
\pgfpathlineto{\pgfqpoint{2.267501in}{1.245702in}}%
\pgfpathlineto{\pgfqpoint{2.264329in}{1.245813in}}%
\pgfpathlineto{\pgfqpoint{2.261157in}{1.245579in}}%
\pgfpathlineto{\pgfqpoint{2.257985in}{1.245612in}}%
\pgfpathlineto{\pgfqpoint{2.254813in}{1.245423in}}%
\pgfpathlineto{\pgfqpoint{2.251641in}{1.245516in}}%
\pgfpathlineto{\pgfqpoint{2.248469in}{1.245552in}}%
\pgfpathlineto{\pgfqpoint{2.245297in}{1.245673in}}%
\pgfpathlineto{\pgfqpoint{2.242125in}{1.245770in}}%
\pgfpathlineto{\pgfqpoint{2.238953in}{1.245670in}}%
\pgfpathlineto{\pgfqpoint{2.235781in}{1.245722in}}%
\pgfpathlineto{\pgfqpoint{2.232609in}{1.245853in}}%
\pgfpathlineto{\pgfqpoint{2.229437in}{1.245862in}}%
\pgfpathlineto{\pgfqpoint{2.226265in}{1.245847in}}%
\pgfpathlineto{\pgfqpoint{2.223093in}{1.245716in}}%
\pgfpathlineto{\pgfqpoint{2.219921in}{1.245746in}}%
\pgfpathlineto{\pgfqpoint{2.216749in}{1.245724in}}%
\pgfpathlineto{\pgfqpoint{2.213577in}{1.245674in}}%
\pgfpathlineto{\pgfqpoint{2.210405in}{1.245486in}}%
\pgfpathlineto{\pgfqpoint{2.207233in}{1.245297in}}%
\pgfpathlineto{\pgfqpoint{2.204061in}{1.245101in}}%
\pgfpathlineto{\pgfqpoint{2.200889in}{1.245178in}}%
\pgfpathlineto{\pgfqpoint{2.197717in}{1.245228in}}%
\pgfpathlineto{\pgfqpoint{2.194545in}{1.245381in}}%
\pgfpathlineto{\pgfqpoint{2.191372in}{1.245259in}}%
\pgfpathlineto{\pgfqpoint{2.188200in}{1.245125in}}%
\pgfpathlineto{\pgfqpoint{2.185028in}{1.245161in}}%
\pgfpathlineto{\pgfqpoint{2.181856in}{1.245258in}}%
\pgfpathlineto{\pgfqpoint{2.178684in}{1.245013in}}%
\pgfpathlineto{\pgfqpoint{2.175512in}{1.244967in}}%
\pgfpathlineto{\pgfqpoint{2.172340in}{1.244949in}}%
\pgfpathlineto{\pgfqpoint{2.169168in}{1.244985in}}%
\pgfpathlineto{\pgfqpoint{2.165996in}{1.244690in}}%
\pgfpathlineto{\pgfqpoint{2.162824in}{1.244595in}}%
\pgfpathlineto{\pgfqpoint{2.159652in}{1.244537in}}%
\pgfpathlineto{\pgfqpoint{2.156480in}{1.244439in}}%
\pgfpathlineto{\pgfqpoint{2.153308in}{1.244460in}}%
\pgfpathlineto{\pgfqpoint{2.150136in}{1.244533in}}%
\pgfpathlineto{\pgfqpoint{2.146964in}{1.244535in}}%
\pgfpathlineto{\pgfqpoint{2.143792in}{1.244552in}}%
\pgfpathlineto{\pgfqpoint{2.140620in}{1.244690in}}%
\pgfpathlineto{\pgfqpoint{2.137448in}{1.244650in}}%
\pgfpathlineto{\pgfqpoint{2.134276in}{1.244568in}}%
\pgfpathlineto{\pgfqpoint{2.131104in}{1.244411in}}%
\pgfpathlineto{\pgfqpoint{2.127932in}{1.244209in}}%
\pgfpathlineto{\pgfqpoint{2.124760in}{1.244152in}}%
\pgfpathlineto{\pgfqpoint{2.121588in}{1.244228in}}%
\pgfpathlineto{\pgfqpoint{2.118416in}{1.244138in}}%
\pgfpathlineto{\pgfqpoint{2.115244in}{1.244237in}}%
\pgfpathlineto{\pgfqpoint{2.112071in}{1.243988in}}%
\pgfpathlineto{\pgfqpoint{2.108899in}{1.244054in}}%
\pgfpathlineto{\pgfqpoint{2.105727in}{1.244153in}}%
\pgfpathlineto{\pgfqpoint{2.102555in}{1.244193in}}%
\pgfpathlineto{\pgfqpoint{2.099383in}{1.244060in}}%
\pgfpathlineto{\pgfqpoint{2.096211in}{1.244030in}}%
\pgfpathlineto{\pgfqpoint{2.093039in}{1.244170in}}%
\pgfpathlineto{\pgfqpoint{2.089867in}{1.244184in}}%
\pgfpathlineto{\pgfqpoint{2.086695in}{1.244151in}}%
\pgfpathlineto{\pgfqpoint{2.083523in}{1.244175in}}%
\pgfpathlineto{\pgfqpoint{2.080351in}{1.244190in}}%
\pgfpathlineto{\pgfqpoint{2.077179in}{1.244030in}}%
\pgfpathlineto{\pgfqpoint{2.074007in}{1.244128in}}%
\pgfpathlineto{\pgfqpoint{2.070835in}{1.244268in}}%
\pgfpathlineto{\pgfqpoint{2.067663in}{1.244294in}}%
\pgfpathlineto{\pgfqpoint{2.064491in}{1.244060in}}%
\pgfpathlineto{\pgfqpoint{2.061319in}{1.244011in}}%
\pgfpathlineto{\pgfqpoint{2.058147in}{1.243758in}}%
\pgfpathlineto{\pgfqpoint{2.054975in}{1.243678in}}%
\pgfpathlineto{\pgfqpoint{2.051803in}{1.243673in}}%
\pgfpathlineto{\pgfqpoint{2.048631in}{1.243595in}}%
\pgfpathlineto{\pgfqpoint{2.045459in}{1.243248in}}%
\pgfpathlineto{\pgfqpoint{2.042287in}{1.243211in}}%
\pgfpathlineto{\pgfqpoint{2.039115in}{1.242733in}}%
\pgfpathlineto{\pgfqpoint{2.035943in}{1.243285in}}%
\pgfpathlineto{\pgfqpoint{2.032770in}{1.243754in}}%
\pgfpathlineto{\pgfqpoint{2.029598in}{1.244238in}}%
\pgfpathlineto{\pgfqpoint{2.026426in}{1.244749in}}%
\pgfpathlineto{\pgfqpoint{2.023254in}{1.246332in}}%
\pgfpathlineto{\pgfqpoint{2.020082in}{1.246657in}}%
\pgfpathlineto{\pgfqpoint{2.016910in}{1.247273in}}%
\pgfpathlineto{\pgfqpoint{2.013738in}{1.247885in}}%
\pgfpathlineto{\pgfqpoint{2.010566in}{1.248505in}}%
\pgfpathlineto{\pgfqpoint{2.007394in}{1.249039in}}%
\pgfpathlineto{\pgfqpoint{2.004222in}{1.250572in}}%
\pgfpathlineto{\pgfqpoint{2.001050in}{1.250863in}}%
\pgfpathlineto{\pgfqpoint{1.997878in}{1.251015in}}%
\pgfpathlineto{\pgfqpoint{1.994706in}{1.250187in}}%
\pgfpathlineto{\pgfqpoint{1.991534in}{1.249461in}}%
\pgfpathlineto{\pgfqpoint{1.988362in}{1.249690in}}%
\pgfpathlineto{\pgfqpoint{1.985190in}{1.250075in}}%
\pgfpathlineto{\pgfqpoint{1.982018in}{1.229254in}}%
\pgfpathlineto{\pgfqpoint{1.978846in}{1.229701in}}%
\pgfpathlineto{\pgfqpoint{1.975674in}{1.230005in}}%
\pgfpathlineto{\pgfqpoint{1.972502in}{1.231427in}}%
\pgfpathlineto{\pgfqpoint{1.969330in}{1.231919in}}%
\pgfpathlineto{\pgfqpoint{1.966158in}{1.232712in}}%
\pgfpathlineto{\pgfqpoint{1.962986in}{1.232807in}}%
\pgfpathlineto{\pgfqpoint{1.959814in}{1.217158in}}%
\pgfpathlineto{\pgfqpoint{1.956641in}{1.192649in}}%
\pgfpathlineto{\pgfqpoint{1.953469in}{1.173001in}}%
\pgfpathlineto{\pgfqpoint{1.950297in}{1.147379in}}%
\pgfpathlineto{\pgfqpoint{1.947125in}{1.127749in}}%
\pgfpathlineto{\pgfqpoint{1.943953in}{1.127697in}}%
\pgfpathlineto{\pgfqpoint{1.940781in}{1.129362in}}%
\pgfpathclose%
\pgfusepath{stroke,fill}%
\end{pgfscope}%
\begin{pgfscope}%
\pgfpathrectangle{\pgfqpoint{1.623736in}{1.000625in}}{\pgfqpoint{6.975000in}{3.020000in}} %
\pgfusepath{clip}%
\pgfsetbuttcap%
\pgfsetroundjoin%
\pgfsetlinewidth{3.011250pt}%
\definecolor{currentstroke}{rgb}{0.866667,0.517647,0.321569}%
\pgfsetstrokecolor{currentstroke}%
\pgfsetdash{{6.000000pt}{12.000000pt}}{0.000000pt}%
\pgfpathmoveto{\pgfqpoint{1.940781in}{1.127790in}}%
\pgfpathlineto{\pgfqpoint{1.947125in}{1.127828in}}%
\pgfpathlineto{\pgfqpoint{1.950297in}{1.144687in}}%
\pgfpathlineto{\pgfqpoint{1.959814in}{1.212151in}}%
\pgfpathlineto{\pgfqpoint{1.985190in}{1.372179in}}%
\pgfpathlineto{\pgfqpoint{2.020082in}{1.498642in}}%
\pgfpathlineto{\pgfqpoint{2.099383in}{1.784341in}}%
\pgfpathlineto{\pgfqpoint{2.162824in}{2.019247in}}%
\pgfpathlineto{\pgfqpoint{2.327770in}{2.628928in}}%
\pgfpathlineto{\pgfqpoint{2.429276in}{3.005410in}}%
\pgfpathlineto{\pgfqpoint{2.451480in}{3.083419in}}%
\pgfpathlineto{\pgfqpoint{2.460996in}{3.095650in}}%
\pgfpathlineto{\pgfqpoint{2.486372in}{3.096171in}}%
\pgfpathlineto{\pgfqpoint{2.527609in}{3.096489in}}%
\pgfpathlineto{\pgfqpoint{2.597394in}{3.097314in}}%
\pgfpathlineto{\pgfqpoint{2.629114in}{3.097233in}}%
\pgfpathlineto{\pgfqpoint{2.657662in}{3.097413in}}%
\pgfpathlineto{\pgfqpoint{3.149329in}{3.098801in}}%
\pgfpathlineto{\pgfqpoint{3.155673in}{3.100088in}}%
\pgfpathlineto{\pgfqpoint{3.241318in}{3.099386in}}%
\pgfpathlineto{\pgfqpoint{3.247662in}{3.098382in}}%
\pgfpathlineto{\pgfqpoint{3.295243in}{3.097743in}}%
\pgfpathlineto{\pgfqpoint{3.326963in}{3.097413in}}%
\pgfpathlineto{\pgfqpoint{3.437984in}{3.098961in}}%
\pgfpathlineto{\pgfqpoint{3.460189in}{3.099195in}}%
\pgfpathlineto{\pgfqpoint{3.482393in}{3.098672in}}%
\pgfpathlineto{\pgfqpoint{3.514113in}{3.099125in}}%
\pgfpathlineto{\pgfqpoint{3.542662in}{3.099154in}}%
\pgfpathlineto{\pgfqpoint{3.663199in}{3.097454in}}%
\pgfpathlineto{\pgfqpoint{3.704436in}{3.095945in}}%
\pgfpathlineto{\pgfqpoint{3.748844in}{3.094730in}}%
\pgfpathlineto{\pgfqpoint{3.783737in}{3.093003in}}%
\pgfpathlineto{\pgfqpoint{3.866210in}{3.092695in}}%
\pgfpathlineto{\pgfqpoint{3.913791in}{3.091157in}}%
\pgfpathlineto{\pgfqpoint{3.948683in}{3.090950in}}%
\pgfpathlineto{\pgfqpoint{3.989919in}{3.089644in}}%
\pgfpathlineto{\pgfqpoint{4.053360in}{3.090122in}}%
\pgfpathlineto{\pgfqpoint{4.081909in}{3.090249in}}%
\pgfpathlineto{\pgfqpoint{4.167554in}{3.091058in}}%
\pgfpathlineto{\pgfqpoint{4.250027in}{3.089086in}}%
\pgfpathlineto{\pgfqpoint{4.272231in}{3.088531in}}%
\pgfpathlineto{\pgfqpoint{4.303952in}{3.088816in}}%
\pgfpathlineto{\pgfqpoint{4.392769in}{3.088860in}}%
\pgfpathlineto{\pgfqpoint{4.427661in}{3.089433in}}%
\pgfpathlineto{\pgfqpoint{4.449865in}{3.088110in}}%
\pgfpathlineto{\pgfqpoint{4.481586in}{3.087779in}}%
\pgfpathlineto{\pgfqpoint{4.500618in}{3.087845in}}%
\pgfpathlineto{\pgfqpoint{4.535510in}{3.088175in}}%
\pgfpathlineto{\pgfqpoint{4.662392in}{3.086516in}}%
\pgfpathlineto{\pgfqpoint{4.709973in}{3.086145in}}%
\pgfpathlineto{\pgfqpoint{4.849542in}{3.085574in}}%
\pgfpathlineto{\pgfqpoint{5.207983in}{3.079992in}}%
\pgfpathlineto{\pgfqpoint{5.223843in}{3.079123in}}%
\pgfpathlineto{\pgfqpoint{5.290456in}{3.079983in}}%
\pgfpathlineto{\pgfqpoint{5.309488in}{3.079839in}}%
\pgfpathlineto{\pgfqpoint{5.379273in}{3.080242in}}%
\pgfpathlineto{\pgfqpoint{5.395133in}{3.080043in}}%
\pgfpathlineto{\pgfqpoint{5.452230in}{3.079705in}}%
\pgfpathlineto{\pgfqpoint{5.506155in}{3.079174in}}%
\pgfpathlineto{\pgfqpoint{5.709165in}{3.074718in}}%
\pgfpathlineto{\pgfqpoint{6.118359in}{3.069883in}}%
\pgfpathlineto{\pgfqpoint{6.134219in}{3.069500in}}%
\pgfpathlineto{\pgfqpoint{6.150079in}{3.069578in}}%
\pgfpathlineto{\pgfqpoint{6.165939in}{3.069735in}}%
\pgfpathlineto{\pgfqpoint{6.235724in}{3.068510in}}%
\pgfpathlineto{\pgfqpoint{6.270617in}{3.068406in}}%
\pgfpathlineto{\pgfqpoint{6.295993in}{3.068233in}}%
\pgfpathlineto{\pgfqpoint{6.413358in}{3.066846in}}%
\pgfpathlineto{\pgfqpoint{6.546584in}{3.063002in}}%
\pgfpathlineto{\pgfqpoint{6.590993in}{3.062641in}}%
\pgfpathlineto{\pgfqpoint{6.657606in}{3.061079in}}%
\pgfpathlineto{\pgfqpoint{6.705186in}{3.059723in}}%
\pgfpathlineto{\pgfqpoint{6.714702in}{3.059872in}}%
\pgfpathlineto{\pgfqpoint{6.787659in}{3.060077in}}%
\pgfpathlineto{\pgfqpoint{6.809864in}{3.060767in}}%
\pgfpathlineto{\pgfqpoint{6.832068in}{3.060099in}}%
\pgfpathlineto{\pgfqpoint{6.854272in}{3.060137in}}%
\pgfpathlineto{\pgfqpoint{6.879648in}{3.059244in}}%
\pgfpathlineto{\pgfqpoint{6.908197in}{3.058938in}}%
\pgfpathlineto{\pgfqpoint{7.685347in}{3.050006in}}%
\pgfpathlineto{\pgfqpoint{7.720239in}{3.050073in}}%
\pgfpathlineto{\pgfqpoint{7.786852in}{3.047893in}}%
\pgfpathlineto{\pgfqpoint{7.875669in}{3.134315in}}%
\pgfpathlineto{\pgfqpoint{8.170669in}{3.423493in}}%
\pgfpathlineto{\pgfqpoint{8.230938in}{3.483189in}}%
\pgfpathlineto{\pgfqpoint{8.281690in}{3.532539in}}%
\pgfpathlineto{\pgfqpoint{8.281690in}{3.532539in}}%
\pgfusepath{stroke}%
\end{pgfscope}%
\begin{pgfscope}%
\pgfpathrectangle{\pgfqpoint{1.623736in}{1.000625in}}{\pgfqpoint{6.975000in}{3.020000in}} %
\pgfusepath{clip}%
\pgfsetbuttcap%
\pgfsetroundjoin%
\pgfsetlinewidth{3.011250pt}%
\definecolor{currentstroke}{rgb}{0.333333,0.658824,0.407843}%
\pgfsetstrokecolor{currentstroke}%
\pgfsetdash{{12.000000pt}{12.000000pt}}{0.000000pt}%
\pgfpathmoveto{\pgfqpoint{1.940781in}{1.126915in}}%
\pgfpathlineto{\pgfqpoint{1.947125in}{1.126818in}}%
\pgfpathlineto{\pgfqpoint{1.950297in}{1.142920in}}%
\pgfpathlineto{\pgfqpoint{1.959814in}{1.211342in}}%
\pgfpathlineto{\pgfqpoint{1.988362in}{1.389206in}}%
\pgfpathlineto{\pgfqpoint{2.010566in}{1.471443in}}%
\pgfpathlineto{\pgfqpoint{2.080351in}{1.722130in}}%
\pgfpathlineto{\pgfqpoint{2.400727in}{2.896352in}}%
\pgfpathlineto{\pgfqpoint{2.419759in}{2.942988in}}%
\pgfpathlineto{\pgfqpoint{2.426103in}{2.950775in}}%
\pgfpathlineto{\pgfqpoint{2.736963in}{2.951471in}}%
\pgfpathlineto{\pgfqpoint{3.206425in}{2.951410in}}%
\pgfpathlineto{\pgfqpoint{3.222286in}{2.950180in}}%
\pgfpathlineto{\pgfqpoint{3.241318in}{2.949470in}}%
\pgfpathlineto{\pgfqpoint{3.247662in}{2.948193in}}%
\pgfpathlineto{\pgfqpoint{3.380888in}{2.946420in}}%
\pgfpathlineto{\pgfqpoint{3.412608in}{2.946106in}}%
\pgfpathlineto{\pgfqpoint{3.437984in}{2.945470in}}%
\pgfpathlineto{\pgfqpoint{3.485565in}{2.943951in}}%
\pgfpathlineto{\pgfqpoint{3.504597in}{2.943901in}}%
\pgfpathlineto{\pgfqpoint{3.561694in}{2.942189in}}%
\pgfpathlineto{\pgfqpoint{3.602931in}{2.942430in}}%
\pgfpathlineto{\pgfqpoint{3.660027in}{2.942428in}}%
\pgfpathlineto{\pgfqpoint{3.694920in}{2.941413in}}%
\pgfpathlineto{\pgfqpoint{3.799597in}{2.937442in}}%
\pgfpathlineto{\pgfqpoint{3.815457in}{2.937670in}}%
\pgfpathlineto{\pgfqpoint{3.878898in}{2.936470in}}%
\pgfpathlineto{\pgfqpoint{3.894758in}{2.935834in}}%
\pgfpathlineto{\pgfqpoint{4.031156in}{2.935247in}}%
\pgfpathlineto{\pgfqpoint{4.059704in}{2.934793in}}%
\pgfpathlineto{\pgfqpoint{4.196102in}{2.934020in}}%
\pgfpathlineto{\pgfqpoint{4.208790in}{2.933893in}}%
\pgfpathlineto{\pgfqpoint{4.224650in}{2.932668in}}%
\pgfpathlineto{\pgfqpoint{4.250027in}{2.931863in}}%
\pgfpathlineto{\pgfqpoint{4.303952in}{2.931470in}}%
\pgfpathlineto{\pgfqpoint{4.322984in}{2.931867in}}%
\pgfpathlineto{\pgfqpoint{4.345188in}{2.931820in}}%
\pgfpathlineto{\pgfqpoint{4.427661in}{2.929963in}}%
\pgfpathlineto{\pgfqpoint{4.440349in}{2.929548in}}%
\pgfpathlineto{\pgfqpoint{4.554543in}{2.927129in}}%
\pgfpathlineto{\pgfqpoint{4.592607in}{2.926364in}}%
\pgfpathlineto{\pgfqpoint{4.611639in}{2.926060in}}%
\pgfpathlineto{\pgfqpoint{4.656048in}{2.923632in}}%
\pgfpathlineto{\pgfqpoint{4.694112in}{2.923279in}}%
\pgfpathlineto{\pgfqpoint{4.719489in}{2.921533in}}%
\pgfpathlineto{\pgfqpoint{4.779758in}{2.920871in}}%
\pgfpathlineto{\pgfqpoint{4.795618in}{2.919829in}}%
\pgfpathlineto{\pgfqpoint{4.960564in}{2.917668in}}%
\pgfpathlineto{\pgfqpoint{5.058897in}{2.914097in}}%
\pgfpathlineto{\pgfqpoint{5.081101in}{2.913865in}}%
\pgfpathlineto{\pgfqpoint{5.103306in}{2.913232in}}%
\pgfpathlineto{\pgfqpoint{5.125510in}{2.913087in}}%
\pgfpathlineto{\pgfqpoint{5.195295in}{2.911418in}}%
\pgfpathlineto{\pgfqpoint{5.230187in}{2.910181in}}%
\pgfpathlineto{\pgfqpoint{5.410994in}{2.907163in}}%
\pgfpathlineto{\pgfqpoint{5.430026in}{2.906676in}}%
\pgfpathlineto{\pgfqpoint{5.461746in}{2.906521in}}%
\pgfpathlineto{\pgfqpoint{5.721854in}{2.898470in}}%
\pgfpathlineto{\pgfqpoint{5.744058in}{2.897584in}}%
\pgfpathlineto{\pgfqpoint{5.797983in}{2.896081in}}%
\pgfpathlineto{\pgfqpoint{5.820187in}{2.896206in}}%
\pgfpathlineto{\pgfqpoint{5.867768in}{2.895064in}}%
\pgfpathlineto{\pgfqpoint{5.883628in}{2.894874in}}%
\pgfpathlineto{\pgfqpoint{5.918520in}{2.894349in}}%
\pgfpathlineto{\pgfqpoint{5.934380in}{2.894295in}}%
\pgfpathlineto{\pgfqpoint{6.073950in}{2.890711in}}%
\pgfpathlineto{\pgfqpoint{6.089810in}{2.890367in}}%
\pgfpathlineto{\pgfqpoint{6.108843in}{2.889352in}}%
\pgfpathlineto{\pgfqpoint{6.140563in}{2.888470in}}%
\pgfpathlineto{\pgfqpoint{6.216692in}{2.886127in}}%
\pgfpathlineto{\pgfqpoint{6.235724in}{2.885775in}}%
\pgfpathlineto{\pgfqpoint{6.261101in}{2.884984in}}%
\pgfpathlineto{\pgfqpoint{6.489487in}{2.879550in}}%
\pgfpathlineto{\pgfqpoint{6.511692in}{2.878747in}}%
\pgfpathlineto{\pgfqpoint{6.537068in}{2.876900in}}%
\pgfpathlineto{\pgfqpoint{6.581477in}{2.875699in}}%
\pgfpathlineto{\pgfqpoint{6.610025in}{2.875179in}}%
\pgfpathlineto{\pgfqpoint{6.629057in}{2.874157in}}%
\pgfpathlineto{\pgfqpoint{6.670294in}{2.873249in}}%
\pgfpathlineto{\pgfqpoint{6.730563in}{2.871378in}}%
\pgfpathlineto{\pgfqpoint{6.749595in}{2.870750in}}%
\pgfpathlineto{\pgfqpoint{6.778143in}{2.870022in}}%
\pgfpathlineto{\pgfqpoint{6.854272in}{2.868638in}}%
\pgfpathlineto{\pgfqpoint{7.491852in}{2.852119in}}%
\pgfpathlineto{\pgfqpoint{7.510885in}{2.851827in}}%
\pgfpathlineto{\pgfqpoint{7.577497in}{2.849650in}}%
\pgfpathlineto{\pgfqpoint{7.606046in}{2.849091in}}%
\pgfpathlineto{\pgfqpoint{7.679003in}{2.847542in}}%
\pgfpathlineto{\pgfqpoint{7.694863in}{2.847973in}}%
\pgfpathlineto{\pgfqpoint{7.729755in}{2.847286in}}%
\pgfpathlineto{\pgfqpoint{7.761476in}{2.845795in}}%
\pgfpathlineto{\pgfqpoint{7.783680in}{2.845266in}}%
\pgfpathlineto{\pgfqpoint{7.786852in}{2.845302in}}%
\pgfpathlineto{\pgfqpoint{7.818572in}{2.876152in}}%
\pgfpathlineto{\pgfqpoint{7.894701in}{2.949293in}}%
\pgfpathlineto{\pgfqpoint{7.901046in}{2.958781in}}%
\pgfpathlineto{\pgfqpoint{8.281690in}{3.325550in}}%
\pgfpathlineto{\pgfqpoint{8.281690in}{3.325550in}}%
\pgfusepath{stroke}%
\end{pgfscope}%
\begin{pgfscope}%
\pgfpathrectangle{\pgfqpoint{1.623736in}{1.000625in}}{\pgfqpoint{6.975000in}{3.020000in}} %
\pgfusepath{clip}%
\pgfsetbuttcap%
\pgfsetroundjoin%
\pgfsetlinewidth{3.011250pt}%
\definecolor{currentstroke}{rgb}{0.768627,0.305882,0.321569}%
\pgfsetstrokecolor{currentstroke}%
\pgfsetdash{{18.000000pt}{12.000000pt}}{0.000000pt}%
\pgfpathmoveto{\pgfqpoint{1.940781in}{1.128560in}}%
\pgfpathlineto{\pgfqpoint{1.947125in}{1.128264in}}%
\pgfpathlineto{\pgfqpoint{1.962986in}{1.230705in}}%
\pgfpathlineto{\pgfqpoint{1.975674in}{1.307475in}}%
\pgfpathlineto{\pgfqpoint{1.988362in}{1.378966in}}%
\pgfpathlineto{\pgfqpoint{2.016910in}{1.483552in}}%
\pgfpathlineto{\pgfqpoint{2.334114in}{2.651729in}}%
\pgfpathlineto{\pgfqpoint{2.372179in}{2.698465in}}%
\pgfpathlineto{\pgfqpoint{2.403899in}{2.699094in}}%
\pgfpathlineto{\pgfqpoint{2.435620in}{2.699635in}}%
\pgfpathlineto{\pgfqpoint{2.591050in}{2.702336in}}%
\pgfpathlineto{\pgfqpoint{2.641802in}{2.702446in}}%
\pgfpathlineto{\pgfqpoint{2.698899in}{2.702962in}}%
\pgfpathlineto{\pgfqpoint{2.724275in}{2.703138in}}%
\pgfpathlineto{\pgfqpoint{2.746480in}{2.703803in}}%
\pgfpathlineto{\pgfqpoint{2.860673in}{2.704426in}}%
\pgfpathlineto{\pgfqpoint{2.930458in}{2.706167in}}%
\pgfpathlineto{\pgfqpoint{2.984383in}{2.707517in}}%
\pgfpathlineto{\pgfqpoint{3.009759in}{2.708248in}}%
\pgfpathlineto{\pgfqpoint{3.063684in}{2.708395in}}%
\pgfpathlineto{\pgfqpoint{3.257178in}{2.711236in}}%
\pgfpathlineto{\pgfqpoint{3.276210in}{2.711833in}}%
\pgfpathlineto{\pgfqpoint{3.298415in}{2.713454in}}%
\pgfpathlineto{\pgfqpoint{3.320619in}{2.714554in}}%
\pgfpathlineto{\pgfqpoint{3.371372in}{2.716253in}}%
\pgfpathlineto{\pgfqpoint{3.387232in}{2.717397in}}%
\pgfpathlineto{\pgfqpoint{3.755188in}{2.724966in}}%
\pgfpathlineto{\pgfqpoint{3.774221in}{2.724594in}}%
\pgfpathlineto{\pgfqpoint{3.818629in}{2.725936in}}%
\pgfpathlineto{\pgfqpoint{3.872554in}{2.725754in}}%
\pgfpathlineto{\pgfqpoint{3.888414in}{2.726323in}}%
\pgfpathlineto{\pgfqpoint{3.916963in}{2.727263in}}%
\pgfpathlineto{\pgfqpoint{4.224650in}{2.734015in}}%
\pgfpathlineto{\pgfqpoint{4.234167in}{2.734657in}}%
\pgfpathlineto{\pgfqpoint{4.307124in}{2.735910in}}%
\pgfpathlineto{\pgfqpoint{4.357876in}{2.737405in}}%
\pgfpathlineto{\pgfqpoint{4.453037in}{2.740634in}}%
\pgfpathlineto{\pgfqpoint{4.484758in}{2.740370in}}%
\pgfpathlineto{\pgfqpoint{4.583091in}{2.741853in}}%
\pgfpathlineto{\pgfqpoint{4.709973in}{2.741693in}}%
\pgfpathlineto{\pgfqpoint{5.246048in}{2.751571in}}%
\pgfpathlineto{\pgfqpoint{5.261908in}{2.752335in}}%
\pgfpathlineto{\pgfqpoint{5.287284in}{2.752016in}}%
\pgfpathlineto{\pgfqpoint{5.303144in}{2.752166in}}%
\pgfpathlineto{\pgfqpoint{5.350725in}{2.753141in}}%
\pgfpathlineto{\pgfqpoint{5.366585in}{2.753533in}}%
\pgfpathlineto{\pgfqpoint{5.398306in}{2.754976in}}%
\pgfpathlineto{\pgfqpoint{5.518843in}{2.756587in}}%
\pgfpathlineto{\pgfqpoint{5.537875in}{2.756935in}}%
\pgfpathlineto{\pgfqpoint{5.591800in}{2.758197in}}%
\pgfpathlineto{\pgfqpoint{5.604488in}{2.757129in}}%
\pgfpathlineto{\pgfqpoint{5.620348in}{2.758013in}}%
\pgfpathlineto{\pgfqpoint{5.680617in}{2.757818in}}%
\pgfpathlineto{\pgfqpoint{5.756746in}{2.759060in}}%
\pgfpathlineto{\pgfqpoint{5.769434in}{2.759962in}}%
\pgfpathlineto{\pgfqpoint{5.810671in}{2.759707in}}%
\pgfpathlineto{\pgfqpoint{5.820187in}{2.760449in}}%
\pgfpathlineto{\pgfqpoint{5.845563in}{2.759736in}}%
\pgfpathlineto{\pgfqpoint{5.861423in}{2.759538in}}%
\pgfpathlineto{\pgfqpoint{5.880456in}{2.761051in}}%
\pgfpathlineto{\pgfqpoint{5.912176in}{2.760785in}}%
\pgfpathlineto{\pgfqpoint{5.943896in}{2.761822in}}%
\pgfpathlineto{\pgfqpoint{5.994649in}{2.760974in}}%
\pgfpathlineto{\pgfqpoint{6.000993in}{2.760406in}}%
\pgfpathlineto{\pgfqpoint{6.039058in}{2.761268in}}%
\pgfpathlineto{\pgfqpoint{6.112015in}{2.761432in}}%
\pgfpathlineto{\pgfqpoint{6.143735in}{2.760988in}}%
\pgfpathlineto{\pgfqpoint{6.156423in}{2.760853in}}%
\pgfpathlineto{\pgfqpoint{6.226208in}{2.762726in}}%
\pgfpathlineto{\pgfqpoint{6.245240in}{2.762844in}}%
\pgfpathlineto{\pgfqpoint{6.308681in}{2.763926in}}%
\pgfpathlineto{\pgfqpoint{6.448251in}{2.765513in}}%
\pgfpathlineto{\pgfqpoint{6.483143in}{2.766732in}}%
\pgfpathlineto{\pgfqpoint{6.568788in}{2.766498in}}%
\pgfpathlineto{\pgfqpoint{6.587821in}{2.767373in}}%
\pgfpathlineto{\pgfqpoint{6.613197in}{2.767473in}}%
\pgfpathlineto{\pgfqpoint{6.670294in}{2.767424in}}%
\pgfpathlineto{\pgfqpoint{6.781315in}{2.768526in}}%
\pgfpathlineto{\pgfqpoint{6.879648in}{2.769413in}}%
\pgfpathlineto{\pgfqpoint{6.895509in}{2.769871in}}%
\pgfpathlineto{\pgfqpoint{6.905025in}{2.770679in}}%
\pgfpathlineto{\pgfqpoint{7.092175in}{2.773160in}}%
\pgfpathlineto{\pgfqpoint{7.117551in}{2.773541in}}%
\pgfpathlineto{\pgfqpoint{7.193680in}{2.774518in}}%
\pgfpathlineto{\pgfqpoint{7.234917in}{2.774515in}}%
\pgfpathlineto{\pgfqpoint{7.263465in}{2.775570in}}%
\pgfpathlineto{\pgfqpoint{7.279326in}{2.775304in}}%
\pgfpathlineto{\pgfqpoint{7.326906in}{2.774939in}}%
\pgfpathlineto{\pgfqpoint{7.355455in}{2.774559in}}%
\pgfpathlineto{\pgfqpoint{7.399863in}{2.776122in}}%
\pgfpathlineto{\pgfqpoint{7.431584in}{2.776462in}}%
\pgfpathlineto{\pgfqpoint{7.482336in}{2.777069in}}%
\pgfpathlineto{\pgfqpoint{7.571153in}{2.776335in}}%
\pgfpathlineto{\pgfqpoint{7.590186in}{2.777742in}}%
\pgfpathlineto{\pgfqpoint{7.640938in}{2.778307in}}%
\pgfpathlineto{\pgfqpoint{7.726583in}{2.780761in}}%
\pgfpathlineto{\pgfqpoint{7.764648in}{2.779426in}}%
\pgfpathlineto{\pgfqpoint{7.786852in}{2.779914in}}%
\pgfpathlineto{\pgfqpoint{7.869325in}{2.861223in}}%
\pgfpathlineto{\pgfqpoint{8.281690in}{3.267123in}}%
\pgfpathlineto{\pgfqpoint{8.281690in}{3.267123in}}%
\pgfusepath{stroke}%
\end{pgfscope}%
\begin{pgfscope}%
\pgfpathrectangle{\pgfqpoint{1.623736in}{1.000625in}}{\pgfqpoint{6.975000in}{3.020000in}} %
\pgfusepath{clip}%
\pgfsetbuttcap%
\pgfsetroundjoin%
\pgfsetlinewidth{3.011250pt}%
\definecolor{currentstroke}{rgb}{0.298039,0.447059,0.690196}%
\pgfsetstrokecolor{currentstroke}%
\pgfsetdash{{3.000000pt}{0.000000pt}}{0.000000pt}%
\pgfpathmoveto{\pgfqpoint{1.940781in}{1.127175in}}%
\pgfpathlineto{\pgfqpoint{1.947125in}{1.128438in}}%
\pgfpathlineto{\pgfqpoint{1.953469in}{1.172446in}}%
\pgfpathlineto{\pgfqpoint{1.978846in}{1.332680in}}%
\pgfpathlineto{\pgfqpoint{1.988362in}{1.386172in}}%
\pgfpathlineto{\pgfqpoint{2.007394in}{1.455752in}}%
\pgfpathlineto{\pgfqpoint{2.045459in}{1.591172in}}%
\pgfpathlineto{\pgfqpoint{2.140620in}{1.932935in}}%
\pgfpathlineto{\pgfqpoint{2.146964in}{1.944706in}}%
\pgfpathlineto{\pgfqpoint{2.153308in}{1.952768in}}%
\pgfpathlineto{\pgfqpoint{2.267501in}{1.953395in}}%
\pgfpathlineto{\pgfqpoint{2.321426in}{1.953213in}}%
\pgfpathlineto{\pgfqpoint{2.527609in}{1.954004in}}%
\pgfpathlineto{\pgfqpoint{2.578361in}{1.954123in}}%
\pgfpathlineto{\pgfqpoint{2.606910in}{1.954171in}}%
\pgfpathlineto{\pgfqpoint{2.832125in}{1.954141in}}%
\pgfpathlineto{\pgfqpoint{2.911426in}{1.954455in}}%
\pgfpathlineto{\pgfqpoint{3.149329in}{1.955229in}}%
\pgfpathlineto{\pgfqpoint{3.155673in}{1.956460in}}%
\pgfpathlineto{\pgfqpoint{3.196909in}{1.956979in}}%
\pgfpathlineto{\pgfqpoint{3.225458in}{1.956792in}}%
\pgfpathlineto{\pgfqpoint{3.257178in}{1.954299in}}%
\pgfpathlineto{\pgfqpoint{3.273038in}{1.954616in}}%
\pgfpathlineto{\pgfqpoint{3.295243in}{1.954295in}}%
\pgfpathlineto{\pgfqpoint{3.307931in}{1.953246in}}%
\pgfpathlineto{\pgfqpoint{3.393576in}{1.954373in}}%
\pgfpathlineto{\pgfqpoint{3.406264in}{1.953919in}}%
\pgfpathlineto{\pgfqpoint{3.460189in}{1.953876in}}%
\pgfpathlineto{\pgfqpoint{3.479221in}{1.953527in}}%
\pgfpathlineto{\pgfqpoint{3.523630in}{1.952688in}}%
\pgfpathlineto{\pgfqpoint{3.568038in}{1.951978in}}%
\pgfpathlineto{\pgfqpoint{3.596586in}{1.951950in}}%
\pgfpathlineto{\pgfqpoint{4.119973in}{1.945503in}}%
\pgfpathlineto{\pgfqpoint{4.135833in}{1.946105in}}%
\pgfpathlineto{\pgfqpoint{4.161210in}{1.945785in}}%
\pgfpathlineto{\pgfqpoint{4.180242in}{1.945855in}}%
\pgfpathlineto{\pgfqpoint{4.224650in}{1.944787in}}%
\pgfpathlineto{\pgfqpoint{4.243683in}{1.944585in}}%
\pgfpathlineto{\pgfqpoint{4.281747in}{1.943855in}}%
\pgfpathlineto{\pgfqpoint{4.307124in}{1.944207in}}%
\pgfpathlineto{\pgfqpoint{4.443521in}{1.943489in}}%
\pgfpathlineto{\pgfqpoint{4.462554in}{1.942540in}}%
\pgfpathlineto{\pgfqpoint{4.535510in}{1.942846in}}%
\pgfpathlineto{\pgfqpoint{4.573575in}{1.942283in}}%
\pgfpathlineto{\pgfqpoint{4.608467in}{1.942764in}}%
\pgfpathlineto{\pgfqpoint{4.633844in}{1.942107in}}%
\pgfpathlineto{\pgfqpoint{4.751209in}{1.939494in}}%
\pgfpathlineto{\pgfqpoint{4.776586in}{1.939723in}}%
\pgfpathlineto{\pgfqpoint{4.798790in}{1.939462in}}%
\pgfpathlineto{\pgfqpoint{4.855887in}{1.940381in}}%
\pgfpathlineto{\pgfqpoint{5.004972in}{1.938825in}}%
\pgfpathlineto{\pgfqpoint{5.052553in}{1.936368in}}%
\pgfpathlineto{\pgfqpoint{5.112822in}{1.936800in}}%
\pgfpathlineto{\pgfqpoint{5.147714in}{1.936821in}}%
\pgfpathlineto{\pgfqpoint{5.173091in}{1.936380in}}%
\pgfpathlineto{\pgfqpoint{5.395133in}{1.934286in}}%
\pgfpathlineto{\pgfqpoint{5.439542in}{1.933716in}}%
\pgfpathlineto{\pgfqpoint{5.461746in}{1.933697in}}%
\pgfpathlineto{\pgfqpoint{5.493467in}{1.933046in}}%
\pgfpathlineto{\pgfqpoint{5.512499in}{1.933135in}}%
\pgfpathlineto{\pgfqpoint{5.560080in}{1.931774in}}%
\pgfpathlineto{\pgfqpoint{5.693305in}{1.927709in}}%
\pgfpathlineto{\pgfqpoint{5.709165in}{1.927682in}}%
\pgfpathlineto{\pgfqpoint{5.734542in}{1.927423in}}%
\pgfpathlineto{\pgfqpoint{5.804327in}{1.926366in}}%
\pgfpathlineto{\pgfqpoint{5.836047in}{1.926319in}}%
\pgfpathlineto{\pgfqpoint{5.902660in}{1.924776in}}%
\pgfpathlineto{\pgfqpoint{5.943896in}{1.925014in}}%
\pgfpathlineto{\pgfqpoint{5.959757in}{1.924862in}}%
\pgfpathlineto{\pgfqpoint{6.086638in}{1.920412in}}%
\pgfpathlineto{\pgfqpoint{6.197660in}{1.917836in}}%
\pgfpathlineto{\pgfqpoint{6.210348in}{1.917801in}}%
\pgfpathlineto{\pgfqpoint{6.226208in}{1.917654in}}%
\pgfpathlineto{\pgfqpoint{6.267445in}{1.917513in}}%
\pgfpathlineto{\pgfqpoint{6.289649in}{1.917109in}}%
\pgfpathlineto{\pgfqpoint{6.308681in}{1.916953in}}%
\pgfpathlineto{\pgfqpoint{6.337230in}{1.917249in}}%
\pgfpathlineto{\pgfqpoint{6.387982in}{1.917814in}}%
\pgfpathlineto{\pgfqpoint{6.422875in}{1.916252in}}%
\pgfpathlineto{\pgfqpoint{6.441907in}{1.915990in}}%
\pgfpathlineto{\pgfqpoint{6.486315in}{1.916078in}}%
\pgfpathlineto{\pgfqpoint{6.521208in}{1.915560in}}%
\pgfpathlineto{\pgfqpoint{6.533896in}{1.914933in}}%
\pgfpathlineto{\pgfqpoint{6.673466in}{1.912944in}}%
\pgfpathlineto{\pgfqpoint{6.692498in}{1.912376in}}%
\pgfpathlineto{\pgfqpoint{6.721046in}{1.912399in}}%
\pgfpathlineto{\pgfqpoint{6.762283in}{1.911270in}}%
\pgfpathlineto{\pgfqpoint{6.797175in}{1.911543in}}%
\pgfpathlineto{\pgfqpoint{6.835240in}{1.910124in}}%
\pgfpathlineto{\pgfqpoint{6.847928in}{1.909998in}}%
\pgfpathlineto{\pgfqpoint{6.870132in}{1.909654in}}%
\pgfpathlineto{\pgfqpoint{6.889165in}{1.909365in}}%
\pgfpathlineto{\pgfqpoint{6.917713in}{1.909191in}}%
\pgfpathlineto{\pgfqpoint{6.952605in}{1.909061in}}%
\pgfpathlineto{\pgfqpoint{6.984326in}{1.910250in}}%
\pgfpathlineto{\pgfqpoint{7.003358in}{1.909465in}}%
\pgfpathlineto{\pgfqpoint{7.016046in}{1.909121in}}%
\pgfpathlineto{\pgfqpoint{7.031906in}{1.908852in}}%
\pgfpathlineto{\pgfqpoint{7.066799in}{1.908643in}}%
\pgfpathlineto{\pgfqpoint{7.098519in}{1.907163in}}%
\pgfpathlineto{\pgfqpoint{7.130240in}{1.907399in}}%
\pgfpathlineto{\pgfqpoint{7.168304in}{1.906219in}}%
\pgfpathlineto{\pgfqpoint{7.244433in}{1.906008in}}%
\pgfpathlineto{\pgfqpoint{7.260293in}{1.905402in}}%
\pgfpathlineto{\pgfqpoint{7.288842in}{1.904135in}}%
\pgfpathlineto{\pgfqpoint{7.330078in}{1.902511in}}%
\pgfpathlineto{\pgfqpoint{7.387175in}{1.901920in}}%
\pgfpathlineto{\pgfqpoint{7.406207in}{1.901782in}}%
\pgfpathlineto{\pgfqpoint{7.425239in}{1.901673in}}%
\pgfpathlineto{\pgfqpoint{7.479164in}{1.901942in}}%
\pgfpathlineto{\pgfqpoint{7.520401in}{1.900679in}}%
\pgfpathlineto{\pgfqpoint{7.567981in}{1.900282in}}%
\pgfpathlineto{\pgfqpoint{7.587013in}{1.900291in}}%
\pgfpathlineto{\pgfqpoint{7.612390in}{1.899833in}}%
\pgfpathlineto{\pgfqpoint{7.644110in}{1.899199in}}%
\pgfpathlineto{\pgfqpoint{7.669487in}{1.899948in}}%
\pgfpathlineto{\pgfqpoint{7.707551in}{1.900030in}}%
\pgfpathlineto{\pgfqpoint{7.720239in}{1.899581in}}%
\pgfpathlineto{\pgfqpoint{7.748788in}{1.898633in}}%
\pgfpathlineto{\pgfqpoint{7.758304in}{1.898292in}}%
\pgfpathlineto{\pgfqpoint{7.786852in}{1.899045in}}%
\pgfpathlineto{\pgfqpoint{7.859809in}{1.969805in}}%
\pgfpathlineto{\pgfqpoint{7.894701in}{2.003756in}}%
\pgfpathlineto{\pgfqpoint{7.901046in}{2.012831in}}%
\pgfpathlineto{\pgfqpoint{8.161153in}{2.265305in}}%
\pgfpathlineto{\pgfqpoint{8.205561in}{2.309104in}}%
\pgfpathlineto{\pgfqpoint{8.281690in}{2.383047in}}%
\pgfpathlineto{\pgfqpoint{8.281690in}{2.383047in}}%
\pgfusepath{stroke}%
\end{pgfscope}%
\begin{pgfscope}%
\pgfpathrectangle{\pgfqpoint{1.623736in}{1.000625in}}{\pgfqpoint{6.975000in}{3.020000in}} %
\pgfusepath{clip}%
\pgfsetbuttcap%
\pgfsetroundjoin%
\pgfsetlinewidth{3.011250pt}%
\definecolor{currentstroke}{rgb}{0.505882,0.447059,0.701961}%
\pgfsetstrokecolor{currentstroke}%
\pgfsetdash{{6.000000pt}{3.000000pt}}{0.000000pt}%
\pgfpathmoveto{\pgfqpoint{1.940781in}{1.126981in}}%
\pgfpathlineto{\pgfqpoint{1.947125in}{1.126818in}}%
\pgfpathlineto{\pgfqpoint{1.953469in}{1.169134in}}%
\pgfpathlineto{\pgfqpoint{1.985190in}{1.366920in}}%
\pgfpathlineto{\pgfqpoint{2.099383in}{1.362055in}}%
\pgfpathlineto{\pgfqpoint{2.235781in}{1.363344in}}%
\pgfpathlineto{\pgfqpoint{2.518093in}{1.365569in}}%
\pgfpathlineto{\pgfqpoint{2.746480in}{1.366839in}}%
\pgfpathlineto{\pgfqpoint{2.816264in}{1.366911in}}%
\pgfpathlineto{\pgfqpoint{2.901910in}{1.367544in}}%
\pgfpathlineto{\pgfqpoint{2.955834in}{1.368375in}}%
\pgfpathlineto{\pgfqpoint{2.965350in}{1.369140in}}%
\pgfpathlineto{\pgfqpoint{3.117608in}{1.370062in}}%
\pgfpathlineto{\pgfqpoint{3.149329in}{1.370377in}}%
\pgfpathlineto{\pgfqpoint{3.155673in}{1.371439in}}%
\pgfpathlineto{\pgfqpoint{3.184221in}{1.372147in}}%
\pgfpathlineto{\pgfqpoint{3.203253in}{1.374824in}}%
\pgfpathlineto{\pgfqpoint{3.222286in}{1.374784in}}%
\pgfpathlineto{\pgfqpoint{3.250834in}{1.374626in}}%
\pgfpathlineto{\pgfqpoint{3.276210in}{1.374593in}}%
\pgfpathlineto{\pgfqpoint{3.304759in}{1.374938in}}%
\pgfpathlineto{\pgfqpoint{3.339651in}{1.376214in}}%
\pgfpathlineto{\pgfqpoint{3.352339in}{1.376446in}}%
\pgfpathlineto{\pgfqpoint{3.371372in}{1.377302in}}%
\pgfpathlineto{\pgfqpoint{3.384060in}{1.378687in}}%
\pgfpathlineto{\pgfqpoint{3.447501in}{1.380361in}}%
\pgfpathlineto{\pgfqpoint{3.514113in}{1.381075in}}%
\pgfpathlineto{\pgfqpoint{3.533146in}{1.381135in}}%
\pgfpathlineto{\pgfqpoint{3.552178in}{1.381132in}}%
\pgfpathlineto{\pgfqpoint{3.580726in}{1.381243in}}%
\pgfpathlineto{\pgfqpoint{3.590242in}{1.381210in}}%
\pgfpathlineto{\pgfqpoint{3.609275in}{1.381516in}}%
\pgfpathlineto{\pgfqpoint{3.977231in}{1.381215in}}%
\pgfpathlineto{\pgfqpoint{3.999436in}{1.381800in}}%
\pgfpathlineto{\pgfqpoint{4.034328in}{1.383639in}}%
\pgfpathlineto{\pgfqpoint{4.107285in}{1.385160in}}%
\pgfpathlineto{\pgfqpoint{4.196102in}{1.389030in}}%
\pgfpathlineto{\pgfqpoint{4.208790in}{1.389729in}}%
\pgfpathlineto{\pgfqpoint{4.218306in}{1.388869in}}%
\pgfpathlineto{\pgfqpoint{4.253199in}{1.389585in}}%
\pgfpathlineto{\pgfqpoint{4.269059in}{1.390845in}}%
\pgfpathlineto{\pgfqpoint{4.897123in}{1.401352in}}%
\pgfpathlineto{\pgfqpoint{4.982768in}{1.403102in}}%
\pgfpathlineto{\pgfqpoint{5.039865in}{1.403197in}}%
\pgfpathlineto{\pgfqpoint{5.062069in}{1.403737in}}%
\pgfpathlineto{\pgfqpoint{5.096962in}{1.404143in}}%
\pgfpathlineto{\pgfqpoint{5.131854in}{1.405115in}}%
\pgfpathlineto{\pgfqpoint{5.173091in}{1.404861in}}%
\pgfpathlineto{\pgfqpoint{5.201639in}{1.404129in}}%
\pgfpathlineto{\pgfqpoint{5.338037in}{1.407388in}}%
\pgfpathlineto{\pgfqpoint{5.360241in}{1.407539in}}%
\pgfpathlineto{\pgfqpoint{5.404650in}{1.408457in}}%
\pgfpathlineto{\pgfqpoint{5.801155in}{1.412363in}}%
\pgfpathlineto{\pgfqpoint{5.826531in}{1.413052in}}%
\pgfpathlineto{\pgfqpoint{5.994649in}{1.412996in}}%
\pgfpathlineto{\pgfqpoint{6.004165in}{1.413116in}}%
\pgfpathlineto{\pgfqpoint{6.064434in}{1.414652in}}%
\pgfpathlineto{\pgfqpoint{6.105671in}{1.414402in}}%
\pgfpathlineto{\pgfqpoint{6.137391in}{1.414028in}}%
\pgfpathlineto{\pgfqpoint{6.188144in}{1.415347in}}%
\pgfpathlineto{\pgfqpoint{6.204004in}{1.416242in}}%
\pgfpathlineto{\pgfqpoint{6.251584in}{1.415775in}}%
\pgfpathlineto{\pgfqpoint{6.305509in}{1.416739in}}%
\pgfpathlineto{\pgfqpoint{6.327713in}{1.416496in}}%
\pgfpathlineto{\pgfqpoint{6.391154in}{1.417507in}}%
\pgfpathlineto{\pgfqpoint{6.413358in}{1.416930in}}%
\pgfpathlineto{\pgfqpoint{6.438735in}{1.417175in}}%
\pgfpathlineto{\pgfqpoint{6.454595in}{1.417164in}}%
\pgfpathlineto{\pgfqpoint{6.486315in}{1.418527in}}%
\pgfpathlineto{\pgfqpoint{6.505348in}{1.418429in}}%
\pgfpathlineto{\pgfqpoint{6.518036in}{1.418531in}}%
\pgfpathlineto{\pgfqpoint{6.552928in}{1.418047in}}%
\pgfpathlineto{\pgfqpoint{6.660778in}{1.420126in}}%
\pgfpathlineto{\pgfqpoint{6.692498in}{1.419688in}}%
\pgfpathlineto{\pgfqpoint{6.711530in}{1.419735in}}%
\pgfpathlineto{\pgfqpoint{6.784487in}{1.421007in}}%
\pgfpathlineto{\pgfqpoint{6.813036in}{1.422036in}}%
\pgfpathlineto{\pgfqpoint{6.828896in}{1.421692in}}%
\pgfpathlineto{\pgfqpoint{6.851100in}{1.422158in}}%
\pgfpathlineto{\pgfqpoint{6.927229in}{1.422044in}}%
\pgfpathlineto{\pgfqpoint{7.016046in}{1.422194in}}%
\pgfpathlineto{\pgfqpoint{7.031906in}{1.422264in}}%
\pgfpathlineto{\pgfqpoint{7.060455in}{1.422300in}}%
\pgfpathlineto{\pgfqpoint{7.079487in}{1.422106in}}%
\pgfpathlineto{\pgfqpoint{7.174648in}{1.422946in}}%
\pgfpathlineto{\pgfqpoint{7.272981in}{1.425601in}}%
\pgfpathlineto{\pgfqpoint{7.339594in}{1.425506in}}%
\pgfpathlineto{\pgfqpoint{7.368143in}{1.426264in}}%
\pgfpathlineto{\pgfqpoint{7.387175in}{1.426680in}}%
\pgfpathlineto{\pgfqpoint{7.422067in}{1.426445in}}%
\pgfpathlineto{\pgfqpoint{7.482336in}{1.427733in}}%
\pgfpathlineto{\pgfqpoint{7.536261in}{1.427471in}}%
\pgfpathlineto{\pgfqpoint{7.567981in}{1.427873in}}%
\pgfpathlineto{\pgfqpoint{7.609218in}{1.428619in}}%
\pgfpathlineto{\pgfqpoint{7.637766in}{1.429879in}}%
\pgfpathlineto{\pgfqpoint{7.679003in}{1.431005in}}%
\pgfpathlineto{\pgfqpoint{7.732927in}{1.432648in}}%
\pgfpathlineto{\pgfqpoint{7.786852in}{1.432116in}}%
\pgfpathlineto{\pgfqpoint{7.834433in}{1.479956in}}%
\pgfpathlineto{\pgfqpoint{8.281690in}{1.926624in}}%
\pgfpathlineto{\pgfqpoint{8.281690in}{1.926624in}}%
\pgfusepath{stroke}%
\end{pgfscope}%
\begin{pgfscope}%
\pgfpathrectangle{\pgfqpoint{1.623736in}{1.000625in}}{\pgfqpoint{6.975000in}{3.020000in}} %
\pgfusepath{clip}%
\pgfsetbuttcap%
\pgfsetroundjoin%
\pgfsetlinewidth{3.011250pt}%
\definecolor{currentstroke}{rgb}{0.576471,0.470588,0.376471}%
\pgfsetstrokecolor{currentstroke}%
\pgfsetdash{{12.000000pt}{3.000000pt}}{0.000000pt}%
\pgfpathmoveto{\pgfqpoint{1.940781in}{1.128371in}}%
\pgfpathlineto{\pgfqpoint{1.947125in}{1.127312in}}%
\pgfpathlineto{\pgfqpoint{1.953469in}{1.170907in}}%
\pgfpathlineto{\pgfqpoint{1.959814in}{1.205140in}}%
\pgfpathlineto{\pgfqpoint{1.962986in}{1.212223in}}%
\pgfpathlineto{\pgfqpoint{1.969330in}{1.211681in}}%
\pgfpathlineto{\pgfqpoint{1.975674in}{1.210393in}}%
\pgfpathlineto{\pgfqpoint{1.982018in}{1.210956in}}%
\pgfpathlineto{\pgfqpoint{1.985190in}{1.232014in}}%
\pgfpathlineto{\pgfqpoint{2.013738in}{1.230756in}}%
\pgfpathlineto{\pgfqpoint{2.048631in}{1.226242in}}%
\pgfpathlineto{\pgfqpoint{2.213577in}{1.227727in}}%
\pgfpathlineto{\pgfqpoint{2.302394in}{1.227957in}}%
\pgfpathlineto{\pgfqpoint{2.353147in}{1.227810in}}%
\pgfpathlineto{\pgfqpoint{3.196909in}{1.237283in}}%
\pgfpathlineto{\pgfqpoint{3.241318in}{1.238175in}}%
\pgfpathlineto{\pgfqpoint{3.247662in}{1.237030in}}%
\pgfpathlineto{\pgfqpoint{3.374544in}{1.239143in}}%
\pgfpathlineto{\pgfqpoint{3.428468in}{1.239857in}}%
\pgfpathlineto{\pgfqpoint{3.666371in}{1.239749in}}%
\pgfpathlineto{\pgfqpoint{3.675887in}{1.239176in}}%
\pgfpathlineto{\pgfqpoint{3.710780in}{1.238811in}}%
\pgfpathlineto{\pgfqpoint{3.736156in}{1.239014in}}%
\pgfpathlineto{\pgfqpoint{3.764705in}{1.238605in}}%
\pgfpathlineto{\pgfqpoint{3.796425in}{1.238097in}}%
\pgfpathlineto{\pgfqpoint{3.824973in}{1.238100in}}%
\pgfpathlineto{\pgfqpoint{3.901102in}{1.238346in}}%
\pgfpathlineto{\pgfqpoint{3.910618in}{1.239062in}}%
\pgfpathlineto{\pgfqpoint{3.932823in}{1.238645in}}%
\pgfpathlineto{\pgfqpoint{3.974059in}{1.237854in}}%
\pgfpathlineto{\pgfqpoint{4.008952in}{1.238686in}}%
\pgfpathlineto{\pgfqpoint{4.196102in}{1.240772in}}%
\pgfpathlineto{\pgfqpoint{4.224650in}{1.240408in}}%
\pgfpathlineto{\pgfqpoint{4.265887in}{1.241327in}}%
\pgfpathlineto{\pgfqpoint{4.284919in}{1.241738in}}%
\pgfpathlineto{\pgfqpoint{4.329328in}{1.242487in}}%
\pgfpathlineto{\pgfqpoint{4.342016in}{1.243073in}}%
\pgfpathlineto{\pgfqpoint{4.380080in}{1.243216in}}%
\pgfpathlineto{\pgfqpoint{4.399113in}{1.243870in}}%
\pgfpathlineto{\pgfqpoint{4.446693in}{1.243424in}}%
\pgfpathlineto{\pgfqpoint{4.468898in}{1.243201in}}%
\pgfpathlineto{\pgfqpoint{4.513306in}{1.243310in}}%
\pgfpathlineto{\pgfqpoint{4.541855in}{1.243252in}}%
\pgfpathlineto{\pgfqpoint{4.576747in}{1.243112in}}%
\pgfpathlineto{\pgfqpoint{4.592607in}{1.243722in}}%
\pgfpathlineto{\pgfqpoint{4.608467in}{1.244313in}}%
\pgfpathlineto{\pgfqpoint{4.640188in}{1.243640in}}%
\pgfpathlineto{\pgfqpoint{4.671908in}{1.243304in}}%
\pgfpathlineto{\pgfqpoint{4.690940in}{1.243651in}}%
\pgfpathlineto{\pgfqpoint{4.735349in}{1.242232in}}%
\pgfpathlineto{\pgfqpoint{4.963736in}{1.243787in}}%
\pgfpathlineto{\pgfqpoint{4.973252in}{1.244072in}}%
\pgfpathlineto{\pgfqpoint{5.157230in}{1.244961in}}%
\pgfpathlineto{\pgfqpoint{5.179435in}{1.244661in}}%
\pgfpathlineto{\pgfqpoint{5.230187in}{1.243424in}}%
\pgfpathlineto{\pgfqpoint{5.255564in}{1.243354in}}%
\pgfpathlineto{\pgfqpoint{5.277768in}{1.242898in}}%
\pgfpathlineto{\pgfqpoint{5.306316in}{1.243257in}}%
\pgfpathlineto{\pgfqpoint{5.328521in}{1.244005in}}%
\pgfpathlineto{\pgfqpoint{5.360241in}{1.245067in}}%
\pgfpathlineto{\pgfqpoint{5.458574in}{1.246757in}}%
\pgfpathlineto{\pgfqpoint{5.493467in}{1.247076in}}%
\pgfpathlineto{\pgfqpoint{5.563252in}{1.247455in}}%
\pgfpathlineto{\pgfqpoint{5.588628in}{1.248041in}}%
\pgfpathlineto{\pgfqpoint{5.639381in}{1.248276in}}%
\pgfpathlineto{\pgfqpoint{5.658413in}{1.248738in}}%
\pgfpathlineto{\pgfqpoint{5.728198in}{1.249266in}}%
\pgfpathlineto{\pgfqpoint{5.839219in}{1.250912in}}%
\pgfpathlineto{\pgfqpoint{5.867768in}{1.251051in}}%
\pgfpathlineto{\pgfqpoint{5.883628in}{1.251320in}}%
\pgfpathlineto{\pgfqpoint{5.896316in}{1.251209in}}%
\pgfpathlineto{\pgfqpoint{5.924864in}{1.252275in}}%
\pgfpathlineto{\pgfqpoint{6.070778in}{1.255082in}}%
\pgfpathlineto{\pgfqpoint{6.089810in}{1.255650in}}%
\pgfpathlineto{\pgfqpoint{6.108843in}{1.255265in}}%
\pgfpathlineto{\pgfqpoint{6.334057in}{1.260018in}}%
\pgfpathlineto{\pgfqpoint{6.349918in}{1.260733in}}%
\pgfpathlineto{\pgfqpoint{6.403842in}{1.261886in}}%
\pgfpathlineto{\pgfqpoint{6.419703in}{1.261864in}}%
\pgfpathlineto{\pgfqpoint{6.635401in}{1.264614in}}%
\pgfpathlineto{\pgfqpoint{6.657606in}{1.264895in}}%
\pgfpathlineto{\pgfqpoint{6.676638in}{1.264882in}}%
\pgfpathlineto{\pgfqpoint{6.714702in}{1.266524in}}%
\pgfpathlineto{\pgfqpoint{6.755939in}{1.266285in}}%
\pgfpathlineto{\pgfqpoint{6.787659in}{1.266836in}}%
\pgfpathlineto{\pgfqpoint{7.019218in}{1.271828in}}%
\pgfpathlineto{\pgfqpoint{7.111207in}{1.272178in}}%
\pgfpathlineto{\pgfqpoint{7.152444in}{1.272879in}}%
\pgfpathlineto{\pgfqpoint{7.180992in}{1.272890in}}%
\pgfpathlineto{\pgfqpoint{7.231745in}{1.273682in}}%
\pgfpathlineto{\pgfqpoint{7.263465in}{1.273881in}}%
\pgfpathlineto{\pgfqpoint{7.295186in}{1.275024in}}%
\pgfpathlineto{\pgfqpoint{7.403035in}{1.277185in}}%
\pgfpathlineto{\pgfqpoint{7.441100in}{1.277666in}}%
\pgfpathlineto{\pgfqpoint{7.479164in}{1.279188in}}%
\pgfpathlineto{\pgfqpoint{7.517229in}{1.279988in}}%
\pgfpathlineto{\pgfqpoint{7.539433in}{1.279875in}}%
\pgfpathlineto{\pgfqpoint{7.748788in}{1.283900in}}%
\pgfpathlineto{\pgfqpoint{7.758304in}{1.283551in}}%
\pgfpathlineto{\pgfqpoint{7.790024in}{1.286029in}}%
\pgfpathlineto{\pgfqpoint{7.891529in}{1.354121in}}%
\pgfpathlineto{\pgfqpoint{8.281690in}{1.734145in}}%
\pgfpathlineto{\pgfqpoint{8.281690in}{1.734145in}}%
\pgfusepath{stroke}%
\end{pgfscope}%
\begin{pgfscope}%
\pgfpathrectangle{\pgfqpoint{1.623736in}{1.000625in}}{\pgfqpoint{6.975000in}{3.020000in}} %
\pgfusepath{clip}%
\pgfsetbuttcap%
\pgfsetroundjoin%
\pgfsetlinewidth{3.011250pt}%
\definecolor{currentstroke}{rgb}{0.576471,0.470588,0.376471}%
\pgfsetstrokecolor{currentstroke}%
\pgfsetdash{{12.000000pt}{3.000000pt}}{0.000000pt}%
\pgfpathmoveto{\pgfqpoint{1.940781in}{1.128371in}}%
\pgfpathlineto{\pgfqpoint{1.947125in}{1.127312in}}%
\pgfpathlineto{\pgfqpoint{1.953469in}{1.170907in}}%
\pgfpathlineto{\pgfqpoint{1.959814in}{1.205140in}}%
\pgfpathlineto{\pgfqpoint{1.962986in}{1.212223in}}%
\pgfpathlineto{\pgfqpoint{1.969330in}{1.211681in}}%
\pgfpathlineto{\pgfqpoint{1.975674in}{1.210393in}}%
\pgfpathlineto{\pgfqpoint{1.982018in}{1.210956in}}%
\pgfpathlineto{\pgfqpoint{1.985190in}{1.232014in}}%
\pgfpathlineto{\pgfqpoint{2.013738in}{1.230756in}}%
\pgfpathlineto{\pgfqpoint{2.048631in}{1.226242in}}%
\pgfpathlineto{\pgfqpoint{2.213577in}{1.227727in}}%
\pgfpathlineto{\pgfqpoint{2.302394in}{1.227957in}}%
\pgfpathlineto{\pgfqpoint{2.353147in}{1.227810in}}%
\pgfpathlineto{\pgfqpoint{3.196909in}{1.237283in}}%
\pgfpathlineto{\pgfqpoint{3.241318in}{1.238175in}}%
\pgfpathlineto{\pgfqpoint{3.247662in}{1.237030in}}%
\pgfpathlineto{\pgfqpoint{3.374544in}{1.239143in}}%
\pgfpathlineto{\pgfqpoint{3.428468in}{1.239857in}}%
\pgfpathlineto{\pgfqpoint{3.666371in}{1.239749in}}%
\pgfpathlineto{\pgfqpoint{3.675887in}{1.239176in}}%
\pgfpathlineto{\pgfqpoint{3.710780in}{1.238811in}}%
\pgfpathlineto{\pgfqpoint{3.736156in}{1.239014in}}%
\pgfpathlineto{\pgfqpoint{3.764705in}{1.238605in}}%
\pgfpathlineto{\pgfqpoint{3.796425in}{1.238097in}}%
\pgfpathlineto{\pgfqpoint{3.824973in}{1.238100in}}%
\pgfpathlineto{\pgfqpoint{3.901102in}{1.238346in}}%
\pgfpathlineto{\pgfqpoint{3.910618in}{1.239062in}}%
\pgfpathlineto{\pgfqpoint{3.932823in}{1.238645in}}%
\pgfpathlineto{\pgfqpoint{3.974059in}{1.237854in}}%
\pgfpathlineto{\pgfqpoint{4.008952in}{1.238686in}}%
\pgfpathlineto{\pgfqpoint{4.196102in}{1.240772in}}%
\pgfpathlineto{\pgfqpoint{4.224650in}{1.240408in}}%
\pgfpathlineto{\pgfqpoint{4.265887in}{1.241327in}}%
\pgfpathlineto{\pgfqpoint{4.284919in}{1.241738in}}%
\pgfpathlineto{\pgfqpoint{4.329328in}{1.242487in}}%
\pgfpathlineto{\pgfqpoint{4.342016in}{1.243073in}}%
\pgfpathlineto{\pgfqpoint{4.380080in}{1.243216in}}%
\pgfpathlineto{\pgfqpoint{4.399113in}{1.243870in}}%
\pgfpathlineto{\pgfqpoint{4.446693in}{1.243424in}}%
\pgfpathlineto{\pgfqpoint{4.468898in}{1.243201in}}%
\pgfpathlineto{\pgfqpoint{4.513306in}{1.243310in}}%
\pgfpathlineto{\pgfqpoint{4.541855in}{1.243252in}}%
\pgfpathlineto{\pgfqpoint{4.576747in}{1.243112in}}%
\pgfpathlineto{\pgfqpoint{4.592607in}{1.243722in}}%
\pgfpathlineto{\pgfqpoint{4.608467in}{1.244313in}}%
\pgfpathlineto{\pgfqpoint{4.640188in}{1.243640in}}%
\pgfpathlineto{\pgfqpoint{4.671908in}{1.243304in}}%
\pgfpathlineto{\pgfqpoint{4.690940in}{1.243651in}}%
\pgfpathlineto{\pgfqpoint{4.735349in}{1.242232in}}%
\pgfpathlineto{\pgfqpoint{4.963736in}{1.243787in}}%
\pgfpathlineto{\pgfqpoint{4.973252in}{1.244072in}}%
\pgfpathlineto{\pgfqpoint{5.157230in}{1.244961in}}%
\pgfpathlineto{\pgfqpoint{5.179435in}{1.244661in}}%
\pgfpathlineto{\pgfqpoint{5.230187in}{1.243424in}}%
\pgfpathlineto{\pgfqpoint{5.255564in}{1.243354in}}%
\pgfpathlineto{\pgfqpoint{5.277768in}{1.242898in}}%
\pgfpathlineto{\pgfqpoint{5.306316in}{1.243257in}}%
\pgfpathlineto{\pgfqpoint{5.328521in}{1.244005in}}%
\pgfpathlineto{\pgfqpoint{5.360241in}{1.245067in}}%
\pgfpathlineto{\pgfqpoint{5.458574in}{1.246757in}}%
\pgfpathlineto{\pgfqpoint{5.493467in}{1.247076in}}%
\pgfpathlineto{\pgfqpoint{5.563252in}{1.247455in}}%
\pgfpathlineto{\pgfqpoint{5.588628in}{1.248041in}}%
\pgfpathlineto{\pgfqpoint{5.639381in}{1.248276in}}%
\pgfpathlineto{\pgfqpoint{5.658413in}{1.248738in}}%
\pgfpathlineto{\pgfqpoint{5.728198in}{1.249266in}}%
\pgfpathlineto{\pgfqpoint{5.839219in}{1.250912in}}%
\pgfpathlineto{\pgfqpoint{5.867768in}{1.251051in}}%
\pgfpathlineto{\pgfqpoint{5.883628in}{1.251320in}}%
\pgfpathlineto{\pgfqpoint{5.896316in}{1.251209in}}%
\pgfpathlineto{\pgfqpoint{5.924864in}{1.252275in}}%
\pgfpathlineto{\pgfqpoint{6.070778in}{1.255082in}}%
\pgfpathlineto{\pgfqpoint{6.089810in}{1.255650in}}%
\pgfpathlineto{\pgfqpoint{6.108843in}{1.255265in}}%
\pgfpathlineto{\pgfqpoint{6.334057in}{1.260018in}}%
\pgfpathlineto{\pgfqpoint{6.349918in}{1.260733in}}%
\pgfpathlineto{\pgfqpoint{6.403842in}{1.261886in}}%
\pgfpathlineto{\pgfqpoint{6.419703in}{1.261864in}}%
\pgfpathlineto{\pgfqpoint{6.635401in}{1.264614in}}%
\pgfpathlineto{\pgfqpoint{6.657606in}{1.264895in}}%
\pgfpathlineto{\pgfqpoint{6.676638in}{1.264882in}}%
\pgfpathlineto{\pgfqpoint{6.714702in}{1.266524in}}%
\pgfpathlineto{\pgfqpoint{6.755939in}{1.266285in}}%
\pgfpathlineto{\pgfqpoint{6.787659in}{1.266836in}}%
\pgfpathlineto{\pgfqpoint{7.019218in}{1.271828in}}%
\pgfpathlineto{\pgfqpoint{7.111207in}{1.272178in}}%
\pgfpathlineto{\pgfqpoint{7.152444in}{1.272879in}}%
\pgfpathlineto{\pgfqpoint{7.180992in}{1.272890in}}%
\pgfpathlineto{\pgfqpoint{7.231745in}{1.273682in}}%
\pgfpathlineto{\pgfqpoint{7.263465in}{1.273881in}}%
\pgfpathlineto{\pgfqpoint{7.295186in}{1.275024in}}%
\pgfpathlineto{\pgfqpoint{7.403035in}{1.277185in}}%
\pgfpathlineto{\pgfqpoint{7.441100in}{1.277666in}}%
\pgfpathlineto{\pgfqpoint{7.479164in}{1.279188in}}%
\pgfpathlineto{\pgfqpoint{7.517229in}{1.279988in}}%
\pgfpathlineto{\pgfqpoint{7.539433in}{1.279875in}}%
\pgfpathlineto{\pgfqpoint{7.748788in}{1.283900in}}%
\pgfpathlineto{\pgfqpoint{7.758304in}{1.283551in}}%
\pgfpathlineto{\pgfqpoint{7.790024in}{1.286029in}}%
\pgfpathlineto{\pgfqpoint{7.891529in}{1.354121in}}%
\pgfpathlineto{\pgfqpoint{8.281690in}{1.734145in}}%
\pgfpathlineto{\pgfqpoint{8.281690in}{1.734145in}}%
\pgfusepath{stroke}%
\end{pgfscope}%
\begin{pgfscope}%
\pgfsetrectcap%
\pgfsetmiterjoin%
\pgfsetlinewidth{1.003750pt}%
\definecolor{currentstroke}{rgb}{0.800000,0.800000,0.800000}%
\pgfsetstrokecolor{currentstroke}%
\pgfsetdash{}{0pt}%
\pgfpathmoveto{\pgfqpoint{1.623736in}{1.000625in}}%
\pgfpathlineto{\pgfqpoint{1.623736in}{4.020625in}}%
\pgfusepath{stroke}%
\end{pgfscope}%
\begin{pgfscope}%
\pgfsetrectcap%
\pgfsetmiterjoin%
\pgfsetlinewidth{1.003750pt}%
\definecolor{currentstroke}{rgb}{0.800000,0.800000,0.800000}%
\pgfsetstrokecolor{currentstroke}%
\pgfsetdash{}{0pt}%
\pgfpathmoveto{\pgfqpoint{8.598736in}{1.000625in}}%
\pgfpathlineto{\pgfqpoint{8.598736in}{4.020625in}}%
\pgfusepath{stroke}%
\end{pgfscope}%
\begin{pgfscope}%
\pgfsetrectcap%
\pgfsetmiterjoin%
\pgfsetlinewidth{1.003750pt}%
\definecolor{currentstroke}{rgb}{0.800000,0.800000,0.800000}%
\pgfsetstrokecolor{currentstroke}%
\pgfsetdash{}{0pt}%
\pgfpathmoveto{\pgfqpoint{1.623736in}{1.000625in}}%
\pgfpathlineto{\pgfqpoint{8.598736in}{1.000625in}}%
\pgfusepath{stroke}%
\end{pgfscope}%
\begin{pgfscope}%
\pgfsetrectcap%
\pgfsetmiterjoin%
\pgfsetlinewidth{1.003750pt}%
\definecolor{currentstroke}{rgb}{0.800000,0.800000,0.800000}%
\pgfsetstrokecolor{currentstroke}%
\pgfsetdash{}{0pt}%
\pgfpathmoveto{\pgfqpoint{1.623736in}{4.020625in}}%
\pgfpathlineto{\pgfqpoint{8.598736in}{4.020625in}}%
\pgfusepath{stroke}%
\end{pgfscope}%
\begin{pgfscope}%
\pgfsetbuttcap%
\pgfsetroundjoin%
\pgfsetlinewidth{3.011250pt}%
\definecolor{currentstroke}{rgb}{0.866667,0.517647,0.321569}%
\pgfsetstrokecolor{currentstroke}%
\pgfsetdash{{6.000000pt}{12.000000pt}}{0.000000pt}%
\pgfpathmoveto{\pgfqpoint{8.632236in}{3.284597in}}%
\pgfpathlineto{\pgfqpoint{9.326680in}{3.284597in}}%
\pgfusepath{stroke}%
\end{pgfscope}%
\begin{pgfscope}%
\definecolor{textcolor}{rgb}{0.150000,0.150000,0.150000}%
\pgfsetstrokecolor{textcolor}%
\pgfsetfillcolor{textcolor}%
\pgftext[x=9.361402in,y=3.163069in,left,base]{\color{textcolor}\rmfamily\fontsize{25.000000}{30.000000}\selectfont α=0.0}%
\end{pgfscope}%
\begin{pgfscope}%
\pgfsetbuttcap%
\pgfsetroundjoin%
\pgfsetlinewidth{3.011250pt}%
\definecolor{currentstroke}{rgb}{0.333333,0.658824,0.407843}%
\pgfsetstrokecolor{currentstroke}%
\pgfsetdash{{12.000000pt}{12.000000pt}}{0.000000pt}%
\pgfpathmoveto{\pgfqpoint{8.632236in}{2.973833in}}%
\pgfpathlineto{\pgfqpoint{9.326680in}{2.973833in}}%
\pgfusepath{stroke}%
\end{pgfscope}%
\begin{pgfscope}%
\definecolor{textcolor}{rgb}{0.150000,0.150000,0.150000}%
\pgfsetstrokecolor{textcolor}%
\pgfsetfillcolor{textcolor}%
\pgftext[x=9.361402in,y=2.852305in,left,base]{\color{textcolor}\rmfamily\fontsize{25.000000}{30.000000}\selectfont α=0.1}%
\end{pgfscope}%
\begin{pgfscope}%
\pgfsetbuttcap%
\pgfsetroundjoin%
\pgfsetlinewidth{3.011250pt}%
\definecolor{currentstroke}{rgb}{0.768627,0.305882,0.321569}%
\pgfsetstrokecolor{currentstroke}%
\pgfsetdash{{18.000000pt}{12.000000pt}}{0.000000pt}%
\pgfpathmoveto{\pgfqpoint{8.632236in}{2.663070in}}%
\pgfpathlineto{\pgfqpoint{9.326680in}{2.663070in}}%
\pgfusepath{stroke}%
\end{pgfscope}%
\begin{pgfscope}%
\definecolor{textcolor}{rgb}{0.150000,0.150000,0.150000}%
\pgfsetstrokecolor{textcolor}%
\pgfsetfillcolor{textcolor}%
\pgftext[x=9.361402in,y=2.541542in,left,base]{\color{textcolor}\rmfamily\fontsize{25.000000}{30.000000}\selectfont α=0.25}%
\end{pgfscope}%
\begin{pgfscope}%
\pgfsetbuttcap%
\pgfsetroundjoin%
\pgfsetlinewidth{3.011250pt}%
\definecolor{currentstroke}{rgb}{0.298039,0.447059,0.690196}%
\pgfsetstrokecolor{currentstroke}%
\pgfsetdash{{3.000000pt}{0.000000pt}}{0.000000pt}%
\pgfpathmoveto{\pgfqpoint{8.632236in}{2.352306in}}%
\pgfpathlineto{\pgfqpoint{9.326680in}{2.352306in}}%
\pgfusepath{stroke}%
\end{pgfscope}%
\begin{pgfscope}%
\definecolor{textcolor}{rgb}{0.150000,0.150000,0.150000}%
\pgfsetstrokecolor{textcolor}%
\pgfsetfillcolor{textcolor}%
\pgftext[x=9.361402in,y=2.230778in,left,base]{\color{textcolor}\rmfamily\fontsize{25.000000}{30.000000}\selectfont α=0.50}%
\end{pgfscope}%
\begin{pgfscope}%
\pgfsetbuttcap%
\pgfsetroundjoin%
\pgfsetlinewidth{3.011250pt}%
\definecolor{currentstroke}{rgb}{0.505882,0.447059,0.701961}%
\pgfsetstrokecolor{currentstroke}%
\pgfsetdash{{6.000000pt}{3.000000pt}}{0.000000pt}%
\pgfpathmoveto{\pgfqpoint{8.632236in}{2.041542in}}%
\pgfpathlineto{\pgfqpoint{9.326680in}{2.041542in}}%
\pgfusepath{stroke}%
\end{pgfscope}%
\begin{pgfscope}%
\definecolor{textcolor}{rgb}{0.150000,0.150000,0.150000}%
\pgfsetstrokecolor{textcolor}%
\pgfsetfillcolor{textcolor}%
\pgftext[x=9.361402in,y=1.920015in,left,base]{\color{textcolor}\rmfamily\fontsize{25.000000}{30.000000}\selectfont α=0.75}%
\end{pgfscope}%
\begin{pgfscope}%
\pgfsetbuttcap%
\pgfsetroundjoin%
\pgfsetlinewidth{3.011250pt}%
\definecolor{currentstroke}{rgb}{0.576471,0.470588,0.376471}%
\pgfsetstrokecolor{currentstroke}%
\pgfsetdash{{12.000000pt}{3.000000pt}}{0.000000pt}%
\pgfpathmoveto{\pgfqpoint{8.632236in}{1.730779in}}%
\pgfpathlineto{\pgfqpoint{9.326680in}{1.730779in}}%
\pgfusepath{stroke}%
\end{pgfscope}%
\begin{pgfscope}%
\definecolor{textcolor}{rgb}{0.150000,0.150000,0.150000}%
\pgfsetstrokecolor{textcolor}%
\pgfsetfillcolor{textcolor}%
\pgftext[x=9.361402in,y=1.609251in,left,base]{\color{textcolor}\rmfamily\fontsize{25.000000}{30.000000}\selectfont α=0.9}%
\end{pgfscope}%
\end{pgfpicture}%
\makeatother%
\endgroup%
%
}
\caption{Difference in run-time when compared to $\alpha=1$}
\end{subfigure}
\caption{Impact of model materialization and $\alpha$ on executing the model-benchmarking scenario (OML = OpenML (baseline), CO = Collaborative Optimizer)}
\label{exp-model-materialization}
\end{figure}
We also investigate the effect of different values of $\alpha$, the parameter controlling the importance of model potential in our materialization algorithm, on run-time.
If $\alpha$ is close to 1, then the materializer aggressively stores high-quality models and ignores their recreation time and size.
If $\alpha$ is close to 0, then the materializer prioritizes the recreation time and size over quality.
When the materialization budget is large, it is difficult to show the effect of $\alpha$.
For example, the materialization budget for the OpenML workloads is 100 MB.
Furthermore, the size of the OpenML models is typically less than 100 KB; thus, regardless of the $\alpha$ value, the materializer stores many of the models with a higher than average quality.
Therefore, in this experiment, we design a materializer that only materializes one model artifact from the entire EG.
This highlights the impact of $\alpha$ on the utility function.
We run the OpenML workloads one by one and compare them with the best model so far.
We vary the value of $\alpha$ from 0 to 1.
When $\alpha$ is 1, the materializer always selects the best model to materialize, since it only considers the quality.
Therefore, $\alpha=1$ incurs the smallest cumulative run-time in the model-benchmarking scenario.
In Figure \ref{exp-model-materialization}(c), we report the difference in cumulative run-time between $\alpha=1$ and other values of $\alpha$ (i.e., the line y=0 corresponds to the delta in cumulative run-time when $\alpha=1$).
Note that because of the small y-scale, the variation of the reported result (i.e., the error band) is more visible when compared to the previous experiments.
In the scenario, we repeatedly execute the pipeline with the best model; thus, the faster we materialize the best model, the smaller the cumulative run-time would be.
Once we materialize a model, the delta in cumulative run-time reaches a plateau.
This is because the overhead of re-executing the best model is negligible; thus, cumulative run-time becomes similar to when $\alpha=1$.
The first time we encounter a new model with higher quality than the previous workloads is at workload number 14.
However, smaller $\alpha$ values ($\alpha<0.5$) only materialize this model after more than 100 repeated execution.
As a result, their delta in cumulative run-time reaches a plateau later than large $\alpha$ values.
The long delay in the materialization of the best model contributes to the higher cumulative run-time for smaller values of $\alpha$.

In our collaborative optimizer, the default value of $\alpha$ is 0.5. 
This value provides a good balance between workloads that have the goal of training high-quality models (e.g., the model-benchmarking scenario) and workloads that are more exploratory in nature.
However, when we have prior knowledge of the nature of the workload in a collaborative environment, then we can set $\alpha$ accordingly.
Therefore, we recommend $\alpha>0.5$ for workloads with the goal of training high-quality models and $\alpha<0.5$ for workloads with exploratory data analysis.
\begin{figure}[h]
\begin{subfigure}[b]{0.5\linewidth}
\centering
 \resizebox{\columnwidth}{!}{%
%% Creator: Matplotlib, PGF backend
%%
%% To include the figure in your LaTeX document, write
%%   \input{<filename>.pgf}
%%
%% Make sure the required packages are loaded in your preamble
%%   \usepackage{pgf}
%%
%% Figures using additional raster images can only be included by \input if
%% they are in the same directory as the main LaTeX file. For loading figures
%% from other directories you can use the `import` package
%%   \usepackage{import}
%% and then include the figures with
%%   \import{<path to file>}{<filename>.pgf}
%%
%% Matplotlib used the following preamble
%%   \usepackage{fontspec}
%%   \setmonofont{Andale Mono}
%%
\begingroup%
\makeatletter%
\begin{pgfpicture}%
\pgfpathrectangle{\pgfpointorigin}{\pgfqpoint{8.287291in}{5.023028in}}%
\pgfusepath{use as bounding box, clip}%
\begin{pgfscope}%
\pgfsetbuttcap%
\pgfsetmiterjoin%
\definecolor{currentfill}{rgb}{1.000000,1.000000,1.000000}%
\pgfsetfillcolor{currentfill}%
\pgfsetlinewidth{0.000000pt}%
\definecolor{currentstroke}{rgb}{1.000000,1.000000,1.000000}%
\pgfsetstrokecolor{currentstroke}%
\pgfsetdash{}{0pt}%
\pgfpathmoveto{\pgfqpoint{0.000000in}{0.000000in}}%
\pgfpathlineto{\pgfqpoint{8.287291in}{0.000000in}}%
\pgfpathlineto{\pgfqpoint{8.287291in}{5.023028in}}%
\pgfpathlineto{\pgfqpoint{0.000000in}{5.023028in}}%
\pgfpathclose%
\pgfusepath{fill}%
\end{pgfscope}%
\begin{pgfscope}%
\pgfsetbuttcap%
\pgfsetmiterjoin%
\definecolor{currentfill}{rgb}{1.000000,1.000000,1.000000}%
\pgfsetfillcolor{currentfill}%
\pgfsetlinewidth{0.000000pt}%
\definecolor{currentstroke}{rgb}{0.000000,0.000000,0.000000}%
\pgfsetstrokecolor{currentstroke}%
\pgfsetstrokeopacity{0.000000}%
\pgfsetdash{}{0pt}%
\pgfpathmoveto{\pgfqpoint{2.201583in}{1.233139in}}%
\pgfpathlineto{\pgfqpoint{7.626583in}{1.233139in}}%
\pgfpathlineto{\pgfqpoint{7.626583in}{4.253139in}}%
\pgfpathlineto{\pgfqpoint{2.201583in}{4.253139in}}%
\pgfpathclose%
\pgfusepath{fill}%
\end{pgfscope}%
\begin{pgfscope}%
\pgfpathrectangle{\pgfqpoint{2.201583in}{1.233139in}}{\pgfqpoint{5.425000in}{3.020000in}} %
\pgfusepath{clip}%
\pgfsetroundcap%
\pgfsetroundjoin%
\pgfsetlinewidth{0.803000pt}%
\definecolor{currentstroke}{rgb}{0.800000,0.800000,0.800000}%
\pgfsetstrokecolor{currentstroke}%
\pgfsetdash{}{0pt}%
\pgfpathmoveto{\pgfqpoint{2.448174in}{1.233139in}}%
\pgfpathlineto{\pgfqpoint{2.448174in}{4.253139in}}%
\pgfusepath{stroke}%
\end{pgfscope}%
\begin{pgfscope}%
\definecolor{textcolor}{rgb}{0.150000,0.150000,0.150000}%
\pgfsetstrokecolor{textcolor}%
\pgfsetfillcolor{textcolor}%
\pgftext[x=2.448174in,y=1.069250in,,top]{\color{textcolor}\rmfamily\fontsize{34.000000}{40.800000}\selectfont 1}%
\end{pgfscope}%
\begin{pgfscope}%
\pgfpathrectangle{\pgfqpoint{2.201583in}{1.233139in}}{\pgfqpoint{5.425000in}{3.020000in}} %
\pgfusepath{clip}%
\pgfsetroundcap%
\pgfsetroundjoin%
\pgfsetlinewidth{0.803000pt}%
\definecolor{currentstroke}{rgb}{0.800000,0.800000,0.800000}%
\pgfsetstrokecolor{currentstroke}%
\pgfsetdash{}{0pt}%
\pgfpathmoveto{\pgfqpoint{3.152719in}{1.233139in}}%
\pgfpathlineto{\pgfqpoint{3.152719in}{4.253139in}}%
\pgfusepath{stroke}%
\end{pgfscope}%
\begin{pgfscope}%
\definecolor{textcolor}{rgb}{0.150000,0.150000,0.150000}%
\pgfsetstrokecolor{textcolor}%
\pgfsetfillcolor{textcolor}%
\pgftext[x=3.152719in,y=1.069250in,,top]{\color{textcolor}\rmfamily\fontsize{34.000000}{40.800000}\selectfont 2}%
\end{pgfscope}%
\begin{pgfscope}%
\pgfpathrectangle{\pgfqpoint{2.201583in}{1.233139in}}{\pgfqpoint{5.425000in}{3.020000in}} %
\pgfusepath{clip}%
\pgfsetroundcap%
\pgfsetroundjoin%
\pgfsetlinewidth{0.803000pt}%
\definecolor{currentstroke}{rgb}{0.800000,0.800000,0.800000}%
\pgfsetstrokecolor{currentstroke}%
\pgfsetdash{}{0pt}%
\pgfpathmoveto{\pgfqpoint{3.857265in}{1.233139in}}%
\pgfpathlineto{\pgfqpoint{3.857265in}{4.253139in}}%
\pgfusepath{stroke}%
\end{pgfscope}%
\begin{pgfscope}%
\definecolor{textcolor}{rgb}{0.150000,0.150000,0.150000}%
\pgfsetstrokecolor{textcolor}%
\pgfsetfillcolor{textcolor}%
\pgftext[x=3.857265in,y=1.069250in,,top]{\color{textcolor}\rmfamily\fontsize{34.000000}{40.800000}\selectfont 3}%
\end{pgfscope}%
\begin{pgfscope}%
\pgfpathrectangle{\pgfqpoint{2.201583in}{1.233139in}}{\pgfqpoint{5.425000in}{3.020000in}} %
\pgfusepath{clip}%
\pgfsetroundcap%
\pgfsetroundjoin%
\pgfsetlinewidth{0.803000pt}%
\definecolor{currentstroke}{rgb}{0.800000,0.800000,0.800000}%
\pgfsetstrokecolor{currentstroke}%
\pgfsetdash{}{0pt}%
\pgfpathmoveto{\pgfqpoint{4.561810in}{1.233139in}}%
\pgfpathlineto{\pgfqpoint{4.561810in}{4.253139in}}%
\pgfusepath{stroke}%
\end{pgfscope}%
\begin{pgfscope}%
\definecolor{textcolor}{rgb}{0.150000,0.150000,0.150000}%
\pgfsetstrokecolor{textcolor}%
\pgfsetfillcolor{textcolor}%
\pgftext[x=4.561810in,y=1.069250in,,top]{\color{textcolor}\rmfamily\fontsize{34.000000}{40.800000}\selectfont 4}%
\end{pgfscope}%
\begin{pgfscope}%
\pgfpathrectangle{\pgfqpoint{2.201583in}{1.233139in}}{\pgfqpoint{5.425000in}{3.020000in}} %
\pgfusepath{clip}%
\pgfsetroundcap%
\pgfsetroundjoin%
\pgfsetlinewidth{0.803000pt}%
\definecolor{currentstroke}{rgb}{0.800000,0.800000,0.800000}%
\pgfsetstrokecolor{currentstroke}%
\pgfsetdash{}{0pt}%
\pgfpathmoveto{\pgfqpoint{5.266356in}{1.233139in}}%
\pgfpathlineto{\pgfqpoint{5.266356in}{4.253139in}}%
\pgfusepath{stroke}%
\end{pgfscope}%
\begin{pgfscope}%
\definecolor{textcolor}{rgb}{0.150000,0.150000,0.150000}%
\pgfsetstrokecolor{textcolor}%
\pgfsetfillcolor{textcolor}%
\pgftext[x=5.266356in,y=1.069250in,,top]{\color{textcolor}\rmfamily\fontsize{34.000000}{40.800000}\selectfont 5}%
\end{pgfscope}%
\begin{pgfscope}%
\pgfpathrectangle{\pgfqpoint{2.201583in}{1.233139in}}{\pgfqpoint{5.425000in}{3.020000in}} %
\pgfusepath{clip}%
\pgfsetroundcap%
\pgfsetroundjoin%
\pgfsetlinewidth{0.803000pt}%
\definecolor{currentstroke}{rgb}{0.800000,0.800000,0.800000}%
\pgfsetstrokecolor{currentstroke}%
\pgfsetdash{}{0pt}%
\pgfpathmoveto{\pgfqpoint{5.970901in}{1.233139in}}%
\pgfpathlineto{\pgfqpoint{5.970901in}{4.253139in}}%
\pgfusepath{stroke}%
\end{pgfscope}%
\begin{pgfscope}%
\definecolor{textcolor}{rgb}{0.150000,0.150000,0.150000}%
\pgfsetstrokecolor{textcolor}%
\pgfsetfillcolor{textcolor}%
\pgftext[x=5.970901in,y=1.069250in,,top]{\color{textcolor}\rmfamily\fontsize{34.000000}{40.800000}\selectfont 6}%
\end{pgfscope}%
\begin{pgfscope}%
\pgfpathrectangle{\pgfqpoint{2.201583in}{1.233139in}}{\pgfqpoint{5.425000in}{3.020000in}} %
\pgfusepath{clip}%
\pgfsetroundcap%
\pgfsetroundjoin%
\pgfsetlinewidth{0.803000pt}%
\definecolor{currentstroke}{rgb}{0.800000,0.800000,0.800000}%
\pgfsetstrokecolor{currentstroke}%
\pgfsetdash{}{0pt}%
\pgfpathmoveto{\pgfqpoint{6.675447in}{1.233139in}}%
\pgfpathlineto{\pgfqpoint{6.675447in}{4.253139in}}%
\pgfusepath{stroke}%
\end{pgfscope}%
\begin{pgfscope}%
\definecolor{textcolor}{rgb}{0.150000,0.150000,0.150000}%
\pgfsetstrokecolor{textcolor}%
\pgfsetfillcolor{textcolor}%
\pgftext[x=6.675447in,y=1.069250in,,top]{\color{textcolor}\rmfamily\fontsize{34.000000}{40.800000}\selectfont 7}%
\end{pgfscope}%
\begin{pgfscope}%
\pgfpathrectangle{\pgfqpoint{2.201583in}{1.233139in}}{\pgfqpoint{5.425000in}{3.020000in}} %
\pgfusepath{clip}%
\pgfsetroundcap%
\pgfsetroundjoin%
\pgfsetlinewidth{0.803000pt}%
\definecolor{currentstroke}{rgb}{0.800000,0.800000,0.800000}%
\pgfsetstrokecolor{currentstroke}%
\pgfsetdash{}{0pt}%
\pgfpathmoveto{\pgfqpoint{7.379992in}{1.233139in}}%
\pgfpathlineto{\pgfqpoint{7.379992in}{4.253139in}}%
\pgfusepath{stroke}%
\end{pgfscope}%
\begin{pgfscope}%
\definecolor{textcolor}{rgb}{0.150000,0.150000,0.150000}%
\pgfsetstrokecolor{textcolor}%
\pgfsetfillcolor{textcolor}%
\pgftext[x=7.379992in,y=1.069250in,,top]{\color{textcolor}\rmfamily\fontsize{34.000000}{40.800000}\selectfont 8}%
\end{pgfscope}%
\begin{pgfscope}%
\definecolor{textcolor}{rgb}{0.150000,0.150000,0.150000}%
\pgfsetstrokecolor{textcolor}%
\pgfsetfillcolor{textcolor}%
\pgftext[x=4.914083in,y=0.593889in,,top]{\color{textcolor}\rmfamily\fontsize{40.000000}{48.000000}\selectfont Workload}%
\end{pgfscope}%
\begin{pgfscope}%
\pgfpathrectangle{\pgfqpoint{2.201583in}{1.233139in}}{\pgfqpoint{5.425000in}{3.020000in}} %
\pgfusepath{clip}%
\pgfsetroundcap%
\pgfsetroundjoin%
\pgfsetlinewidth{0.803000pt}%
\definecolor{currentstroke}{rgb}{0.800000,0.800000,0.800000}%
\pgfsetstrokecolor{currentstroke}%
\pgfsetdash{}{0pt}%
\pgfpathmoveto{\pgfqpoint{2.201583in}{1.233139in}}%
\pgfpathlineto{\pgfqpoint{7.626583in}{1.233139in}}%
\pgfusepath{stroke}%
\end{pgfscope}%
\begin{pgfscope}%
\definecolor{textcolor}{rgb}{0.150000,0.150000,0.150000}%
\pgfsetstrokecolor{textcolor}%
\pgfsetfillcolor{textcolor}%
\pgftext[x=1.820944in,y=1.069278in,left,base]{\color{textcolor}\rmfamily\fontsize{34.000000}{40.800000}\selectfont 0}%
\end{pgfscope}%
\begin{pgfscope}%
\pgfpathrectangle{\pgfqpoint{2.201583in}{1.233139in}}{\pgfqpoint{5.425000in}{3.020000in}} %
\pgfusepath{clip}%
\pgfsetroundcap%
\pgfsetroundjoin%
\pgfsetlinewidth{0.803000pt}%
\definecolor{currentstroke}{rgb}{0.800000,0.800000,0.800000}%
\pgfsetstrokecolor{currentstroke}%
\pgfsetdash{}{0pt}%
\pgfpathmoveto{\pgfqpoint{2.201583in}{1.919502in}}%
\pgfpathlineto{\pgfqpoint{7.626583in}{1.919502in}}%
\pgfusepath{stroke}%
\end{pgfscope}%
\begin{pgfscope}%
\definecolor{textcolor}{rgb}{0.150000,0.150000,0.150000}%
\pgfsetstrokecolor{textcolor}%
\pgfsetfillcolor{textcolor}%
\pgftext[x=1.387444in,y=1.755641in,left,base]{\color{textcolor}\rmfamily\fontsize{34.000000}{40.800000}\selectfont 500}%
\end{pgfscope}%
\begin{pgfscope}%
\pgfpathrectangle{\pgfqpoint{2.201583in}{1.233139in}}{\pgfqpoint{5.425000in}{3.020000in}} %
\pgfusepath{clip}%
\pgfsetroundcap%
\pgfsetroundjoin%
\pgfsetlinewidth{0.803000pt}%
\definecolor{currentstroke}{rgb}{0.800000,0.800000,0.800000}%
\pgfsetstrokecolor{currentstroke}%
\pgfsetdash{}{0pt}%
\pgfpathmoveto{\pgfqpoint{2.201583in}{2.605866in}}%
\pgfpathlineto{\pgfqpoint{7.626583in}{2.605866in}}%
\pgfusepath{stroke}%
\end{pgfscope}%
\begin{pgfscope}%
\definecolor{textcolor}{rgb}{0.150000,0.150000,0.150000}%
\pgfsetstrokecolor{textcolor}%
\pgfsetfillcolor{textcolor}%
\pgftext[x=1.591916in,y=2.442005in,left,base]{\color{textcolor}\rmfamily\fontsize{34.000000}{40.800000}\selectfont 1k}%
\end{pgfscope}%
\begin{pgfscope}%
\pgfpathrectangle{\pgfqpoint{2.201583in}{1.233139in}}{\pgfqpoint{5.425000in}{3.020000in}} %
\pgfusepath{clip}%
\pgfsetroundcap%
\pgfsetroundjoin%
\pgfsetlinewidth{0.803000pt}%
\definecolor{currentstroke}{rgb}{0.800000,0.800000,0.800000}%
\pgfsetstrokecolor{currentstroke}%
\pgfsetdash{}{0pt}%
\pgfpathmoveto{\pgfqpoint{2.201583in}{3.292230in}}%
\pgfpathlineto{\pgfqpoint{7.626583in}{3.292230in}}%
\pgfusepath{stroke}%
\end{pgfscope}%
\begin{pgfscope}%
\definecolor{textcolor}{rgb}{0.150000,0.150000,0.150000}%
\pgfsetstrokecolor{textcolor}%
\pgfsetfillcolor{textcolor}%
\pgftext[x=1.257111in,y=3.128368in,left,base]{\color{textcolor}\rmfamily\fontsize{34.000000}{40.800000}\selectfont 1.5k}%
\end{pgfscope}%
\begin{pgfscope}%
\pgfpathrectangle{\pgfqpoint{2.201583in}{1.233139in}}{\pgfqpoint{5.425000in}{3.020000in}} %
\pgfusepath{clip}%
\pgfsetroundcap%
\pgfsetroundjoin%
\pgfsetlinewidth{0.803000pt}%
\definecolor{currentstroke}{rgb}{0.800000,0.800000,0.800000}%
\pgfsetstrokecolor{currentstroke}%
\pgfsetdash{}{0pt}%
\pgfpathmoveto{\pgfqpoint{2.201583in}{3.978593in}}%
\pgfpathlineto{\pgfqpoint{7.626583in}{3.978593in}}%
\pgfusepath{stroke}%
\end{pgfscope}%
\begin{pgfscope}%
\definecolor{textcolor}{rgb}{0.150000,0.150000,0.150000}%
\pgfsetstrokecolor{textcolor}%
\pgfsetfillcolor{textcolor}%
\pgftext[x=1.591916in,y=3.814732in,left,base]{\color{textcolor}\rmfamily\fontsize{34.000000}{40.800000}\selectfont 2k}%
\end{pgfscope}%
\begin{pgfscope}%
\definecolor{textcolor}{rgb}{0.150000,0.150000,0.150000}%
\pgfsetstrokecolor{textcolor}%
\pgfsetfillcolor{textcolor}%
\pgftext[x=0.455000in,y=1.377583in,left,base,rotate=90.000000]{\color{textcolor}\rmfamily\fontsize{40.000000}{48.000000}\selectfont Cumulative }%
\end{pgfscope}%
\begin{pgfscope}%
\definecolor{textcolor}{rgb}{0.150000,0.150000,0.150000}%
\pgfsetstrokecolor{textcolor}%
\pgfsetfillcolor{textcolor}%
\pgftext[x=1.062667in,y=1.231194in,left,base,rotate=90.000000]{\color{textcolor}\rmfamily\fontsize{40.000000}{48.000000}\selectfont Run Time (s)}%
\end{pgfscope}%
\begin{pgfscope}%
\pgfpathrectangle{\pgfqpoint{2.201583in}{1.233139in}}{\pgfqpoint{5.425000in}{3.020000in}} %
\pgfusepath{clip}%
\pgfsetbuttcap%
\pgfsetroundjoin%
\definecolor{currentfill}{rgb}{0.298039,0.447059,0.690196}%
\pgfsetfillcolor{currentfill}%
\pgfsetfillopacity{0.200000}%
\pgfsetlinewidth{0.803000pt}%
\definecolor{currentstroke}{rgb}{0.298039,0.447059,0.690196}%
\pgfsetstrokecolor{currentstroke}%
\pgfsetstrokeopacity{0.200000}%
\pgfsetdash{}{0pt}%
\pgfpathmoveto{\pgfqpoint{2.448174in}{1.508497in}}%
\pgfpathlineto{\pgfqpoint{2.448174in}{1.507370in}}%
\pgfpathlineto{\pgfqpoint{3.152719in}{1.742423in}}%
\pgfpathlineto{\pgfqpoint{3.857265in}{2.148105in}}%
\pgfpathlineto{\pgfqpoint{4.561810in}{2.316058in}}%
\pgfpathlineto{\pgfqpoint{5.266356in}{2.445289in}}%
\pgfpathlineto{\pgfqpoint{5.970901in}{2.648629in}}%
\pgfpathlineto{\pgfqpoint{6.675447in}{2.846294in}}%
\pgfpathlineto{\pgfqpoint{7.379992in}{3.167960in}}%
\pgfpathlineto{\pgfqpoint{7.379992in}{3.299033in}}%
\pgfpathlineto{\pgfqpoint{7.379992in}{3.299033in}}%
\pgfpathlineto{\pgfqpoint{6.675447in}{2.976185in}}%
\pgfpathlineto{\pgfqpoint{5.970901in}{2.667834in}}%
\pgfpathlineto{\pgfqpoint{5.266356in}{2.456845in}}%
\pgfpathlineto{\pgfqpoint{4.561810in}{2.319752in}}%
\pgfpathlineto{\pgfqpoint{3.857265in}{2.153598in}}%
\pgfpathlineto{\pgfqpoint{3.152719in}{1.744247in}}%
\pgfpathlineto{\pgfqpoint{2.448174in}{1.508497in}}%
\pgfpathclose%
\pgfusepath{stroke,fill}%
\end{pgfscope}%
\begin{pgfscope}%
\pgfpathrectangle{\pgfqpoint{2.201583in}{1.233139in}}{\pgfqpoint{5.425000in}{3.020000in}} %
\pgfusepath{clip}%
\pgfsetbuttcap%
\pgfsetroundjoin%
\definecolor{currentfill}{rgb}{0.866667,0.517647,0.321569}%
\pgfsetfillcolor{currentfill}%
\pgfsetfillopacity{0.200000}%
\pgfsetlinewidth{0.803000pt}%
\definecolor{currentstroke}{rgb}{0.866667,0.517647,0.321569}%
\pgfsetstrokecolor{currentstroke}%
\pgfsetstrokeopacity{0.200000}%
\pgfsetdash{}{0pt}%
\pgfpathmoveto{\pgfqpoint{2.448174in}{1.508011in}}%
\pgfpathlineto{\pgfqpoint{2.448174in}{1.507333in}}%
\pgfpathlineto{\pgfqpoint{3.152719in}{1.741972in}}%
\pgfpathlineto{\pgfqpoint{3.857265in}{2.140387in}}%
\pgfpathlineto{\pgfqpoint{4.561810in}{2.332333in}}%
\pgfpathlineto{\pgfqpoint{5.266356in}{2.462460in}}%
\pgfpathlineto{\pgfqpoint{5.970901in}{2.672417in}}%
\pgfpathlineto{\pgfqpoint{6.675447in}{3.229427in}}%
\pgfpathlineto{\pgfqpoint{7.379992in}{3.564088in}}%
\pgfpathlineto{\pgfqpoint{7.379992in}{3.682373in}}%
\pgfpathlineto{\pgfqpoint{7.379992in}{3.682373in}}%
\pgfpathlineto{\pgfqpoint{6.675447in}{3.347229in}}%
\pgfpathlineto{\pgfqpoint{5.970901in}{2.712367in}}%
\pgfpathlineto{\pgfqpoint{5.266356in}{2.499810in}}%
\pgfpathlineto{\pgfqpoint{4.561810in}{2.357934in}}%
\pgfpathlineto{\pgfqpoint{3.857265in}{2.149377in}}%
\pgfpathlineto{\pgfqpoint{3.152719in}{1.744075in}}%
\pgfpathlineto{\pgfqpoint{2.448174in}{1.508011in}}%
\pgfpathclose%
\pgfusepath{stroke,fill}%
\end{pgfscope}%
\begin{pgfscope}%
\pgfpathrectangle{\pgfqpoint{2.201583in}{1.233139in}}{\pgfqpoint{5.425000in}{3.020000in}} %
\pgfusepath{clip}%
\pgfsetbuttcap%
\pgfsetroundjoin%
\definecolor{currentfill}{rgb}{0.333333,0.658824,0.407843}%
\pgfsetfillcolor{currentfill}%
\pgfsetfillopacity{0.200000}%
\pgfsetlinewidth{0.803000pt}%
\definecolor{currentstroke}{rgb}{0.333333,0.658824,0.407843}%
\pgfsetstrokecolor{currentstroke}%
\pgfsetstrokeopacity{0.200000}%
\pgfsetdash{}{0pt}%
\pgfpathmoveto{\pgfqpoint{2.448174in}{1.509714in}}%
\pgfpathlineto{\pgfqpoint{2.448174in}{1.508531in}}%
\pgfpathlineto{\pgfqpoint{3.152719in}{1.744177in}}%
\pgfpathlineto{\pgfqpoint{3.857265in}{2.142255in}}%
\pgfpathlineto{\pgfqpoint{4.561810in}{2.361249in}}%
\pgfpathlineto{\pgfqpoint{5.266356in}{2.509094in}}%
\pgfpathlineto{\pgfqpoint{5.970901in}{2.719127in}}%
\pgfpathlineto{\pgfqpoint{6.675447in}{3.672070in}}%
\pgfpathlineto{\pgfqpoint{7.379992in}{4.013839in}}%
\pgfpathlineto{\pgfqpoint{7.379992in}{4.128747in}}%
\pgfpathlineto{\pgfqpoint{7.379992in}{4.128747in}}%
\pgfpathlineto{\pgfqpoint{6.675447in}{3.826368in}}%
\pgfpathlineto{\pgfqpoint{5.970901in}{2.742087in}}%
\pgfpathlineto{\pgfqpoint{5.266356in}{2.527887in}}%
\pgfpathlineto{\pgfqpoint{4.561810in}{2.382329in}}%
\pgfpathlineto{\pgfqpoint{3.857265in}{2.161396in}}%
\pgfpathlineto{\pgfqpoint{3.152719in}{1.745536in}}%
\pgfpathlineto{\pgfqpoint{2.448174in}{1.509714in}}%
\pgfpathclose%
\pgfusepath{stroke,fill}%
\end{pgfscope}%
\begin{pgfscope}%
\pgfpathrectangle{\pgfqpoint{2.201583in}{1.233139in}}{\pgfqpoint{5.425000in}{3.020000in}} %
\pgfusepath{clip}%
\pgfsetbuttcap%
\pgfsetroundjoin%
\pgfsetlinewidth{3.011250pt}%
\definecolor{currentstroke}{rgb}{0.298039,0.447059,0.690196}%
\pgfsetstrokecolor{currentstroke}%
\pgfsetdash{{3.000000pt}{0.000000pt}}{0.000000pt}%
\pgfpathmoveto{\pgfqpoint{2.448174in}{1.507971in}}%
\pgfpathlineto{\pgfqpoint{3.152719in}{1.743500in}}%
\pgfpathlineto{\pgfqpoint{3.857265in}{2.151684in}}%
\pgfpathlineto{\pgfqpoint{4.561810in}{2.317605in}}%
\pgfpathlineto{\pgfqpoint{5.266356in}{2.450822in}}%
\pgfpathlineto{\pgfqpoint{5.970901in}{2.658721in}}%
\pgfpathlineto{\pgfqpoint{6.675447in}{2.924311in}}%
\pgfpathlineto{\pgfqpoint{7.379992in}{3.247233in}}%
\pgfusepath{stroke}%
\end{pgfscope}%
\begin{pgfscope}%
\pgfpathrectangle{\pgfqpoint{2.201583in}{1.233139in}}{\pgfqpoint{5.425000in}{3.020000in}} %
\pgfusepath{clip}%
\pgfsetbuttcap%
\pgfsetroundjoin%
\definecolor{currentfill}{rgb}{0.298039,0.447059,0.690196}%
\pgfsetfillcolor{currentfill}%
\pgfsetlinewidth{0.752812pt}%
\definecolor{currentstroke}{rgb}{1.000000,1.000000,1.000000}%
\pgfsetstrokecolor{currentstroke}%
\pgfsetdash{}{0pt}%
\pgfsys@defobject{currentmarker}{\pgfqpoint{-0.104167in}{-0.104167in}}{\pgfqpoint{0.104167in}{0.104167in}}{%
\pgfpathmoveto{\pgfqpoint{0.000000in}{-0.104167in}}%
\pgfpathcurveto{\pgfqpoint{0.027625in}{-0.104167in}}{\pgfqpoint{0.054123in}{-0.093191in}}{\pgfqpoint{0.073657in}{-0.073657in}}%
\pgfpathcurveto{\pgfqpoint{0.093191in}{-0.054123in}}{\pgfqpoint{0.104167in}{-0.027625in}}{\pgfqpoint{0.104167in}{0.000000in}}%
\pgfpathcurveto{\pgfqpoint{0.104167in}{0.027625in}}{\pgfqpoint{0.093191in}{0.054123in}}{\pgfqpoint{0.073657in}{0.073657in}}%
\pgfpathcurveto{\pgfqpoint{0.054123in}{0.093191in}}{\pgfqpoint{0.027625in}{0.104167in}}{\pgfqpoint{0.000000in}{0.104167in}}%
\pgfpathcurveto{\pgfqpoint{-0.027625in}{0.104167in}}{\pgfqpoint{-0.054123in}{0.093191in}}{\pgfqpoint{-0.073657in}{0.073657in}}%
\pgfpathcurveto{\pgfqpoint{-0.093191in}{0.054123in}}{\pgfqpoint{-0.104167in}{0.027625in}}{\pgfqpoint{-0.104167in}{0.000000in}}%
\pgfpathcurveto{\pgfqpoint{-0.104167in}{-0.027625in}}{\pgfqpoint{-0.093191in}{-0.054123in}}{\pgfqpoint{-0.073657in}{-0.073657in}}%
\pgfpathcurveto{\pgfqpoint{-0.054123in}{-0.093191in}}{\pgfqpoint{-0.027625in}{-0.104167in}}{\pgfqpoint{0.000000in}{-0.104167in}}%
\pgfpathclose%
\pgfusepath{stroke,fill}%
}%
\begin{pgfscope}%
\pgfsys@transformshift{2.448174in}{1.507971in}%
\pgfsys@useobject{currentmarker}{}%
\end{pgfscope}%
\begin{pgfscope}%
\pgfsys@transformshift{3.152719in}{1.743500in}%
\pgfsys@useobject{currentmarker}{}%
\end{pgfscope}%
\begin{pgfscope}%
\pgfsys@transformshift{3.857265in}{2.151684in}%
\pgfsys@useobject{currentmarker}{}%
\end{pgfscope}%
\begin{pgfscope}%
\pgfsys@transformshift{4.561810in}{2.317605in}%
\pgfsys@useobject{currentmarker}{}%
\end{pgfscope}%
\begin{pgfscope}%
\pgfsys@transformshift{5.266356in}{2.450822in}%
\pgfsys@useobject{currentmarker}{}%
\end{pgfscope}%
\begin{pgfscope}%
\pgfsys@transformshift{5.970901in}{2.658721in}%
\pgfsys@useobject{currentmarker}{}%
\end{pgfscope}%
\begin{pgfscope}%
\pgfsys@transformshift{6.675447in}{2.924311in}%
\pgfsys@useobject{currentmarker}{}%
\end{pgfscope}%
\begin{pgfscope}%
\pgfsys@transformshift{7.379992in}{3.247233in}%
\pgfsys@useobject{currentmarker}{}%
\end{pgfscope}%
\end{pgfscope}%
\begin{pgfscope}%
\pgfpathrectangle{\pgfqpoint{2.201583in}{1.233139in}}{\pgfqpoint{5.425000in}{3.020000in}} %
\pgfusepath{clip}%
\pgfsetbuttcap%
\pgfsetroundjoin%
\pgfsetlinewidth{3.011250pt}%
\definecolor{currentstroke}{rgb}{0.866667,0.517647,0.321569}%
\pgfsetstrokecolor{currentstroke}%
\pgfsetdash{{9.000000pt}{3.000000pt}}{0.000000pt}%
\pgfpathmoveto{\pgfqpoint{2.448174in}{1.507686in}}%
\pgfpathlineto{\pgfqpoint{3.152719in}{1.743151in}}%
\pgfpathlineto{\pgfqpoint{3.857265in}{2.145642in}}%
\pgfpathlineto{\pgfqpoint{4.561810in}{2.347501in}}%
\pgfpathlineto{\pgfqpoint{5.266356in}{2.487033in}}%
\pgfpathlineto{\pgfqpoint{5.970901in}{2.696357in}}%
\pgfpathlineto{\pgfqpoint{6.675447in}{3.279924in}}%
\pgfpathlineto{\pgfqpoint{7.379992in}{3.614254in}}%
\pgfusepath{stroke}%
\end{pgfscope}%
\begin{pgfscope}%
\pgfpathrectangle{\pgfqpoint{2.201583in}{1.233139in}}{\pgfqpoint{5.425000in}{3.020000in}} %
\pgfusepath{clip}%
\pgfsetbuttcap%
\pgfsetmiterjoin%
\definecolor{currentfill}{rgb}{0.866667,0.517647,0.321569}%
\pgfsetfillcolor{currentfill}%
\pgfsetlinewidth{0.752812pt}%
\definecolor{currentstroke}{rgb}{1.000000,1.000000,1.000000}%
\pgfsetstrokecolor{currentstroke}%
\pgfsetdash{}{0pt}%
\pgfsys@defobject{currentmarker}{\pgfqpoint{-0.104167in}{-0.104167in}}{\pgfqpoint{0.104167in}{0.104167in}}{%
\pgfpathmoveto{\pgfqpoint{0.000000in}{0.104167in}}%
\pgfpathlineto{\pgfqpoint{-0.104167in}{-0.104167in}}%
\pgfpathlineto{\pgfqpoint{0.104167in}{-0.104167in}}%
\pgfpathclose%
\pgfusepath{stroke,fill}%
}%
\begin{pgfscope}%
\pgfsys@transformshift{2.448174in}{1.507686in}%
\pgfsys@useobject{currentmarker}{}%
\end{pgfscope}%
\begin{pgfscope}%
\pgfsys@transformshift{3.152719in}{1.743151in}%
\pgfsys@useobject{currentmarker}{}%
\end{pgfscope}%
\begin{pgfscope}%
\pgfsys@transformshift{3.857265in}{2.145642in}%
\pgfsys@useobject{currentmarker}{}%
\end{pgfscope}%
\begin{pgfscope}%
\pgfsys@transformshift{4.561810in}{2.347501in}%
\pgfsys@useobject{currentmarker}{}%
\end{pgfscope}%
\begin{pgfscope}%
\pgfsys@transformshift{5.266356in}{2.487033in}%
\pgfsys@useobject{currentmarker}{}%
\end{pgfscope}%
\begin{pgfscope}%
\pgfsys@transformshift{5.970901in}{2.696357in}%
\pgfsys@useobject{currentmarker}{}%
\end{pgfscope}%
\begin{pgfscope}%
\pgfsys@transformshift{6.675447in}{3.279924in}%
\pgfsys@useobject{currentmarker}{}%
\end{pgfscope}%
\begin{pgfscope}%
\pgfsys@transformshift{7.379992in}{3.614254in}%
\pgfsys@useobject{currentmarker}{}%
\end{pgfscope}%
\end{pgfscope}%
\begin{pgfscope}%
\pgfpathrectangle{\pgfqpoint{2.201583in}{1.233139in}}{\pgfqpoint{5.425000in}{3.020000in}} %
\pgfusepath{clip}%
\pgfsetbuttcap%
\pgfsetroundjoin%
\pgfsetlinewidth{3.011250pt}%
\definecolor{currentstroke}{rgb}{0.333333,0.658824,0.407843}%
\pgfsetstrokecolor{currentstroke}%
\pgfsetdash{{3.000000pt}{3.000000pt}}{0.000000pt}%
\pgfpathmoveto{\pgfqpoint{2.448174in}{1.509007in}}%
\pgfpathlineto{\pgfqpoint{3.152719in}{1.745017in}}%
\pgfpathlineto{\pgfqpoint{3.857265in}{2.152627in}}%
\pgfpathlineto{\pgfqpoint{4.561810in}{2.373692in}}%
\pgfpathlineto{\pgfqpoint{5.266356in}{2.518968in}}%
\pgfpathlineto{\pgfqpoint{5.970901in}{2.729108in}}%
\pgfpathlineto{\pgfqpoint{6.675447in}{3.738494in}}%
\pgfpathlineto{\pgfqpoint{7.379992in}{4.078162in}}%
\pgfusepath{stroke}%
\end{pgfscope}%
\begin{pgfscope}%
\pgfpathrectangle{\pgfqpoint{2.201583in}{1.233139in}}{\pgfqpoint{5.425000in}{3.020000in}} %
\pgfusepath{clip}%
\pgfsetbuttcap%
\pgfsetmiterjoin%
\definecolor{currentfill}{rgb}{0.333333,0.658824,0.407843}%
\pgfsetfillcolor{currentfill}%
\pgfsetlinewidth{0.752812pt}%
\definecolor{currentstroke}{rgb}{1.000000,1.000000,1.000000}%
\pgfsetstrokecolor{currentstroke}%
\pgfsetdash{}{0pt}%
\pgfsys@defobject{currentmarker}{\pgfqpoint{-0.104167in}{-0.104167in}}{\pgfqpoint{0.104167in}{0.104167in}}{%
\pgfpathmoveto{\pgfqpoint{-0.000000in}{-0.104167in}}%
\pgfpathlineto{\pgfqpoint{0.104167in}{0.104167in}}%
\pgfpathlineto{\pgfqpoint{-0.104167in}{0.104167in}}%
\pgfpathclose%
\pgfusepath{stroke,fill}%
}%
\begin{pgfscope}%
\pgfsys@transformshift{2.448174in}{1.509007in}%
\pgfsys@useobject{currentmarker}{}%
\end{pgfscope}%
\begin{pgfscope}%
\pgfsys@transformshift{3.152719in}{1.745017in}%
\pgfsys@useobject{currentmarker}{}%
\end{pgfscope}%
\begin{pgfscope}%
\pgfsys@transformshift{3.857265in}{2.152627in}%
\pgfsys@useobject{currentmarker}{}%
\end{pgfscope}%
\begin{pgfscope}%
\pgfsys@transformshift{4.561810in}{2.373692in}%
\pgfsys@useobject{currentmarker}{}%
\end{pgfscope}%
\begin{pgfscope}%
\pgfsys@transformshift{5.266356in}{2.518968in}%
\pgfsys@useobject{currentmarker}{}%
\end{pgfscope}%
\begin{pgfscope}%
\pgfsys@transformshift{5.970901in}{2.729108in}%
\pgfsys@useobject{currentmarker}{}%
\end{pgfscope}%
\begin{pgfscope}%
\pgfsys@transformshift{6.675447in}{3.738494in}%
\pgfsys@useobject{currentmarker}{}%
\end{pgfscope}%
\begin{pgfscope}%
\pgfsys@transformshift{7.379992in}{4.078162in}%
\pgfsys@useobject{currentmarker}{}%
\end{pgfscope}%
\end{pgfscope}%
\begin{pgfscope}%
\pgfsetrectcap%
\pgfsetmiterjoin%
\pgfsetlinewidth{1.003750pt}%
\definecolor{currentstroke}{rgb}{0.800000,0.800000,0.800000}%
\pgfsetstrokecolor{currentstroke}%
\pgfsetdash{}{0pt}%
\pgfpathmoveto{\pgfqpoint{2.201583in}{1.233139in}}%
\pgfpathlineto{\pgfqpoint{2.201583in}{4.253139in}}%
\pgfusepath{stroke}%
\end{pgfscope}%
\begin{pgfscope}%
\pgfsetrectcap%
\pgfsetmiterjoin%
\pgfsetlinewidth{1.003750pt}%
\definecolor{currentstroke}{rgb}{0.800000,0.800000,0.800000}%
\pgfsetstrokecolor{currentstroke}%
\pgfsetdash{}{0pt}%
\pgfpathmoveto{\pgfqpoint{7.626583in}{1.233139in}}%
\pgfpathlineto{\pgfqpoint{7.626583in}{4.253139in}}%
\pgfusepath{stroke}%
\end{pgfscope}%
\begin{pgfscope}%
\pgfsetrectcap%
\pgfsetmiterjoin%
\pgfsetlinewidth{1.003750pt}%
\definecolor{currentstroke}{rgb}{0.800000,0.800000,0.800000}%
\pgfsetstrokecolor{currentstroke}%
\pgfsetdash{}{0pt}%
\pgfpathmoveto{\pgfqpoint{2.201583in}{1.233139in}}%
\pgfpathlineto{\pgfqpoint{7.626583in}{1.233139in}}%
\pgfusepath{stroke}%
\end{pgfscope}%
\begin{pgfscope}%
\pgfsetrectcap%
\pgfsetmiterjoin%
\pgfsetlinewidth{1.003750pt}%
\definecolor{currentstroke}{rgb}{0.800000,0.800000,0.800000}%
\pgfsetstrokecolor{currentstroke}%
\pgfsetdash{}{0pt}%
\pgfpathmoveto{\pgfqpoint{2.201583in}{4.253139in}}%
\pgfpathlineto{\pgfqpoint{7.626583in}{4.253139in}}%
\pgfusepath{stroke}%
\end{pgfscope}%
\begin{pgfscope}%
\pgfsetbuttcap%
\pgfsetroundjoin%
\pgfsetlinewidth{4.015000pt}%
\definecolor{currentstroke}{rgb}{0.298039,0.447059,0.690196}%
\pgfsetstrokecolor{currentstroke}%
\pgfsetdash{{4.000000pt}{0.000000pt}}{0.000000pt}%
\pgfpathmoveto{\pgfqpoint{1.829764in}{4.568861in}}%
\pgfpathlineto{\pgfqpoint{2.538097in}{4.568861in}}%
\pgfusepath{stroke}%
\end{pgfscope}%
\begin{pgfscope}%
\pgfsetbuttcap%
\pgfsetroundjoin%
\definecolor{currentfill}{rgb}{0.298039,0.447059,0.690196}%
\pgfsetfillcolor{currentfill}%
\pgfsetlinewidth{0.000000pt}%
\definecolor{currentstroke}{rgb}{0.298039,0.447059,0.690196}%
\pgfsetstrokecolor{currentstroke}%
\pgfsetdash{}{0pt}%
\pgfsys@defobject{currentmarker}{\pgfqpoint{-0.104167in}{-0.104167in}}{\pgfqpoint{0.104167in}{0.104167in}}{%
\pgfpathmoveto{\pgfqpoint{0.000000in}{-0.104167in}}%
\pgfpathcurveto{\pgfqpoint{0.027625in}{-0.104167in}}{\pgfqpoint{0.054123in}{-0.093191in}}{\pgfqpoint{0.073657in}{-0.073657in}}%
\pgfpathcurveto{\pgfqpoint{0.093191in}{-0.054123in}}{\pgfqpoint{0.104167in}{-0.027625in}}{\pgfqpoint{0.104167in}{0.000000in}}%
\pgfpathcurveto{\pgfqpoint{0.104167in}{0.027625in}}{\pgfqpoint{0.093191in}{0.054123in}}{\pgfqpoint{0.073657in}{0.073657in}}%
\pgfpathcurveto{\pgfqpoint{0.054123in}{0.093191in}}{\pgfqpoint{0.027625in}{0.104167in}}{\pgfqpoint{0.000000in}{0.104167in}}%
\pgfpathcurveto{\pgfqpoint{-0.027625in}{0.104167in}}{\pgfqpoint{-0.054123in}{0.093191in}}{\pgfqpoint{-0.073657in}{0.073657in}}%
\pgfpathcurveto{\pgfqpoint{-0.093191in}{0.054123in}}{\pgfqpoint{-0.104167in}{0.027625in}}{\pgfqpoint{-0.104167in}{0.000000in}}%
\pgfpathcurveto{\pgfqpoint{-0.104167in}{-0.027625in}}{\pgfqpoint{-0.093191in}{-0.054123in}}{\pgfqpoint{-0.073657in}{-0.073657in}}%
\pgfpathcurveto{\pgfqpoint{-0.054123in}{-0.093191in}}{\pgfqpoint{-0.027625in}{-0.104167in}}{\pgfqpoint{0.000000in}{-0.104167in}}%
\pgfpathclose%
\pgfusepath{fill}%
}%
\begin{pgfscope}%
\pgfsys@transformshift{2.183930in}{4.568861in}%
\pgfsys@useobject{currentmarker}{}%
\end{pgfscope}%
\end{pgfscope}%
\begin{pgfscope}%
\definecolor{textcolor}{rgb}{0.150000,0.150000,0.150000}%
\pgfsetstrokecolor{textcolor}%
\pgfsetfillcolor{textcolor}%
\pgftext[x=2.585319in,y=4.403583in,left,base]{\color{textcolor}\rmfamily\fontsize{34.000000}{40.800000}\selectfont LN}%
\end{pgfscope}%
\begin{pgfscope}%
\pgfsetbuttcap%
\pgfsetroundjoin%
\pgfsetlinewidth{4.015000pt}%
\definecolor{currentstroke}{rgb}{0.866667,0.517647,0.321569}%
\pgfsetstrokecolor{currentstroke}%
\pgfsetdash{{12.000000pt}{4.000000pt}}{0.000000pt}%
\pgfpathmoveto{\pgfqpoint{3.279014in}{4.568861in}}%
\pgfpathlineto{\pgfqpoint{3.987347in}{4.568861in}}%
\pgfusepath{stroke}%
\end{pgfscope}%
\begin{pgfscope}%
\pgfsetbuttcap%
\pgfsetmiterjoin%
\definecolor{currentfill}{rgb}{0.866667,0.517647,0.321569}%
\pgfsetfillcolor{currentfill}%
\pgfsetlinewidth{0.000000pt}%
\definecolor{currentstroke}{rgb}{0.866667,0.517647,0.321569}%
\pgfsetstrokecolor{currentstroke}%
\pgfsetdash{}{0pt}%
\pgfsys@defobject{currentmarker}{\pgfqpoint{-0.104167in}{-0.104167in}}{\pgfqpoint{0.104167in}{0.104167in}}{%
\pgfpathmoveto{\pgfqpoint{0.000000in}{0.104167in}}%
\pgfpathlineto{\pgfqpoint{-0.104167in}{-0.104167in}}%
\pgfpathlineto{\pgfqpoint{0.104167in}{-0.104167in}}%
\pgfpathclose%
\pgfusepath{fill}%
}%
\begin{pgfscope}%
\pgfsys@transformshift{3.633180in}{4.568861in}%
\pgfsys@useobject{currentmarker}{}%
\end{pgfscope}%
\end{pgfscope}%
\begin{pgfscope}%
\definecolor{textcolor}{rgb}{0.150000,0.150000,0.150000}%
\pgfsetstrokecolor{textcolor}%
\pgfsetfillcolor{textcolor}%
\pgftext[x=4.034569in,y=4.403583in,left,base]{\color{textcolor}\rmfamily\fontsize{34.000000}{40.800000}\selectfont ALL\_M}%
\end{pgfscope}%
\begin{pgfscope}%
\pgfsetbuttcap%
\pgfsetroundjoin%
\pgfsetlinewidth{4.015000pt}%
\definecolor{currentstroke}{rgb}{0.333333,0.658824,0.407843}%
\pgfsetstrokecolor{currentstroke}%
\pgfsetdash{{4.000000pt}{4.000000pt}}{0.000000pt}%
\pgfpathmoveto{\pgfqpoint{5.728903in}{4.568861in}}%
\pgfpathlineto{\pgfqpoint{6.437236in}{4.568861in}}%
\pgfusepath{stroke}%
\end{pgfscope}%
\begin{pgfscope}%
\pgfsetbuttcap%
\pgfsetmiterjoin%
\definecolor{currentfill}{rgb}{0.333333,0.658824,0.407843}%
\pgfsetfillcolor{currentfill}%
\pgfsetlinewidth{0.000000pt}%
\definecolor{currentstroke}{rgb}{0.333333,0.658824,0.407843}%
\pgfsetstrokecolor{currentstroke}%
\pgfsetdash{}{0pt}%
\pgfsys@defobject{currentmarker}{\pgfqpoint{-0.104167in}{-0.104167in}}{\pgfqpoint{0.104167in}{0.104167in}}{%
\pgfpathmoveto{\pgfqpoint{-0.000000in}{-0.104167in}}%
\pgfpathlineto{\pgfqpoint{0.104167in}{0.104167in}}%
\pgfpathlineto{\pgfqpoint{-0.104167in}{0.104167in}}%
\pgfpathclose%
\pgfusepath{fill}%
}%
\begin{pgfscope}%
\pgfsys@transformshift{6.083069in}{4.568861in}%
\pgfsys@useobject{currentmarker}{}%
\end{pgfscope}%
\end{pgfscope}%
\begin{pgfscope}%
\definecolor{textcolor}{rgb}{0.150000,0.150000,0.150000}%
\pgfsetstrokecolor{textcolor}%
\pgfsetfillcolor{textcolor}%
\pgftext[x=6.484458in,y=4.403583in,left,base]{\color{textcolor}\rmfamily\fontsize{34.000000}{40.800000}\selectfont ALL\_C}%
\end{pgfscope}%
\end{pgfpicture}%
\makeatother%
\endgroup%
%
}
\caption{Heuristics-based}
\end{subfigure}%
\begin{subfigure}[b]{0.5\linewidth}
\centering
 \resizebox{\columnwidth}{!}{%
%% Creator: Matplotlib, PGF backend
%%
%% To include the figure in your LaTeX document, write
%%   \input{<filename>.pgf}
%%
%% Make sure the required packages are loaded in your preamble
%%   \usepackage{pgf}
%%
%% Figures using additional raster images can only be included by \input if
%% they are in the same directory as the main LaTeX file. For loading figures
%% from other directories you can use the `import` package
%%   \usepackage{import}
%% and then include the figures with
%%   \import{<path to file>}{<filename>.pgf}
%%
%% Matplotlib used the following preamble
%%   \usepackage{fontspec}
%%   \setmonofont{Andale Mono}
%%
\begingroup%
\makeatletter%
\begin{pgfpicture}%
\pgfpathrectangle{\pgfpointorigin}{\pgfqpoint{7.224961in}{4.906038in}}%
\pgfusepath{use as bounding box, clip}%
\begin{pgfscope}%
\pgfsetbuttcap%
\pgfsetmiterjoin%
\definecolor{currentfill}{rgb}{1.000000,1.000000,1.000000}%
\pgfsetfillcolor{currentfill}%
\pgfsetlinewidth{0.000000pt}%
\definecolor{currentstroke}{rgb}{1.000000,1.000000,1.000000}%
\pgfsetstrokecolor{currentstroke}%
\pgfsetdash{}{0pt}%
\pgfpathmoveto{\pgfqpoint{0.000000in}{0.000000in}}%
\pgfpathlineto{\pgfqpoint{7.224961in}{0.000000in}}%
\pgfpathlineto{\pgfqpoint{7.224961in}{4.906038in}}%
\pgfpathlineto{\pgfqpoint{0.000000in}{4.906038in}}%
\pgfpathclose%
\pgfusepath{fill}%
\end{pgfscope}%
\begin{pgfscope}%
\pgfsetbuttcap%
\pgfsetmiterjoin%
\definecolor{currentfill}{rgb}{1.000000,1.000000,1.000000}%
\pgfsetfillcolor{currentfill}%
\pgfsetlinewidth{0.000000pt}%
\definecolor{currentstroke}{rgb}{0.000000,0.000000,0.000000}%
\pgfsetstrokecolor{currentstroke}%
\pgfsetstrokeopacity{0.000000}%
\pgfsetdash{}{0pt}%
\pgfpathmoveto{\pgfqpoint{1.633294in}{1.124483in}}%
\pgfpathlineto{\pgfqpoint{7.058294in}{1.124483in}}%
\pgfpathlineto{\pgfqpoint{7.058294in}{4.144483in}}%
\pgfpathlineto{\pgfqpoint{1.633294in}{4.144483in}}%
\pgfpathclose%
\pgfusepath{fill}%
\end{pgfscope}%
\begin{pgfscope}%
\pgfpathrectangle{\pgfqpoint{1.633294in}{1.124483in}}{\pgfqpoint{5.425000in}{3.020000in}} %
\pgfusepath{clip}%
\pgfsetroundcap%
\pgfsetroundjoin%
\pgfsetlinewidth{0.803000pt}%
\definecolor{currentstroke}{rgb}{0.800000,0.800000,0.800000}%
\pgfsetstrokecolor{currentstroke}%
\pgfsetdash{}{0pt}%
\pgfpathmoveto{\pgfqpoint{1.879885in}{1.124483in}}%
\pgfpathlineto{\pgfqpoint{1.879885in}{4.144483in}}%
\pgfusepath{stroke}%
\end{pgfscope}%
\begin{pgfscope}%
\definecolor{textcolor}{rgb}{0.150000,0.150000,0.150000}%
\pgfsetstrokecolor{textcolor}%
\pgfsetfillcolor{textcolor}%
\pgftext[x=1.879885in,y=0.960594in,,top]{\color{textcolor}\rmfamily\fontsize{35.200000}{42.240000}\selectfont 1}%
\end{pgfscope}%
\begin{pgfscope}%
\pgfpathrectangle{\pgfqpoint{1.633294in}{1.124483in}}{\pgfqpoint{5.425000in}{3.020000in}} %
\pgfusepath{clip}%
\pgfsetroundcap%
\pgfsetroundjoin%
\pgfsetlinewidth{0.803000pt}%
\definecolor{currentstroke}{rgb}{0.800000,0.800000,0.800000}%
\pgfsetstrokecolor{currentstroke}%
\pgfsetdash{}{0pt}%
\pgfpathmoveto{\pgfqpoint{2.584431in}{1.124483in}}%
\pgfpathlineto{\pgfqpoint{2.584431in}{4.144483in}}%
\pgfusepath{stroke}%
\end{pgfscope}%
\begin{pgfscope}%
\definecolor{textcolor}{rgb}{0.150000,0.150000,0.150000}%
\pgfsetstrokecolor{textcolor}%
\pgfsetfillcolor{textcolor}%
\pgftext[x=2.584431in,y=0.960594in,,top]{\color{textcolor}\rmfamily\fontsize{35.200000}{42.240000}\selectfont 2}%
\end{pgfscope}%
\begin{pgfscope}%
\pgfpathrectangle{\pgfqpoint{1.633294in}{1.124483in}}{\pgfqpoint{5.425000in}{3.020000in}} %
\pgfusepath{clip}%
\pgfsetroundcap%
\pgfsetroundjoin%
\pgfsetlinewidth{0.803000pt}%
\definecolor{currentstroke}{rgb}{0.800000,0.800000,0.800000}%
\pgfsetstrokecolor{currentstroke}%
\pgfsetdash{}{0pt}%
\pgfpathmoveto{\pgfqpoint{3.288976in}{1.124483in}}%
\pgfpathlineto{\pgfqpoint{3.288976in}{4.144483in}}%
\pgfusepath{stroke}%
\end{pgfscope}%
\begin{pgfscope}%
\definecolor{textcolor}{rgb}{0.150000,0.150000,0.150000}%
\pgfsetstrokecolor{textcolor}%
\pgfsetfillcolor{textcolor}%
\pgftext[x=3.288976in,y=0.960594in,,top]{\color{textcolor}\rmfamily\fontsize{35.200000}{42.240000}\selectfont 3}%
\end{pgfscope}%
\begin{pgfscope}%
\pgfpathrectangle{\pgfqpoint{1.633294in}{1.124483in}}{\pgfqpoint{5.425000in}{3.020000in}} %
\pgfusepath{clip}%
\pgfsetroundcap%
\pgfsetroundjoin%
\pgfsetlinewidth{0.803000pt}%
\definecolor{currentstroke}{rgb}{0.800000,0.800000,0.800000}%
\pgfsetstrokecolor{currentstroke}%
\pgfsetdash{}{0pt}%
\pgfpathmoveto{\pgfqpoint{3.993521in}{1.124483in}}%
\pgfpathlineto{\pgfqpoint{3.993521in}{4.144483in}}%
\pgfusepath{stroke}%
\end{pgfscope}%
\begin{pgfscope}%
\definecolor{textcolor}{rgb}{0.150000,0.150000,0.150000}%
\pgfsetstrokecolor{textcolor}%
\pgfsetfillcolor{textcolor}%
\pgftext[x=3.993521in,y=0.960594in,,top]{\color{textcolor}\rmfamily\fontsize{35.200000}{42.240000}\selectfont 4}%
\end{pgfscope}%
\begin{pgfscope}%
\pgfpathrectangle{\pgfqpoint{1.633294in}{1.124483in}}{\pgfqpoint{5.425000in}{3.020000in}} %
\pgfusepath{clip}%
\pgfsetroundcap%
\pgfsetroundjoin%
\pgfsetlinewidth{0.803000pt}%
\definecolor{currentstroke}{rgb}{0.800000,0.800000,0.800000}%
\pgfsetstrokecolor{currentstroke}%
\pgfsetdash{}{0pt}%
\pgfpathmoveto{\pgfqpoint{4.698067in}{1.124483in}}%
\pgfpathlineto{\pgfqpoint{4.698067in}{4.144483in}}%
\pgfusepath{stroke}%
\end{pgfscope}%
\begin{pgfscope}%
\definecolor{textcolor}{rgb}{0.150000,0.150000,0.150000}%
\pgfsetstrokecolor{textcolor}%
\pgfsetfillcolor{textcolor}%
\pgftext[x=4.698067in,y=0.960594in,,top]{\color{textcolor}\rmfamily\fontsize{35.200000}{42.240000}\selectfont 5}%
\end{pgfscope}%
\begin{pgfscope}%
\pgfpathrectangle{\pgfqpoint{1.633294in}{1.124483in}}{\pgfqpoint{5.425000in}{3.020000in}} %
\pgfusepath{clip}%
\pgfsetroundcap%
\pgfsetroundjoin%
\pgfsetlinewidth{0.803000pt}%
\definecolor{currentstroke}{rgb}{0.800000,0.800000,0.800000}%
\pgfsetstrokecolor{currentstroke}%
\pgfsetdash{}{0pt}%
\pgfpathmoveto{\pgfqpoint{5.402612in}{1.124483in}}%
\pgfpathlineto{\pgfqpoint{5.402612in}{4.144483in}}%
\pgfusepath{stroke}%
\end{pgfscope}%
\begin{pgfscope}%
\definecolor{textcolor}{rgb}{0.150000,0.150000,0.150000}%
\pgfsetstrokecolor{textcolor}%
\pgfsetfillcolor{textcolor}%
\pgftext[x=5.402612in,y=0.960594in,,top]{\color{textcolor}\rmfamily\fontsize{35.200000}{42.240000}\selectfont 6}%
\end{pgfscope}%
\begin{pgfscope}%
\pgfpathrectangle{\pgfqpoint{1.633294in}{1.124483in}}{\pgfqpoint{5.425000in}{3.020000in}} %
\pgfusepath{clip}%
\pgfsetroundcap%
\pgfsetroundjoin%
\pgfsetlinewidth{0.803000pt}%
\definecolor{currentstroke}{rgb}{0.800000,0.800000,0.800000}%
\pgfsetstrokecolor{currentstroke}%
\pgfsetdash{}{0pt}%
\pgfpathmoveto{\pgfqpoint{6.107158in}{1.124483in}}%
\pgfpathlineto{\pgfqpoint{6.107158in}{4.144483in}}%
\pgfusepath{stroke}%
\end{pgfscope}%
\begin{pgfscope}%
\definecolor{textcolor}{rgb}{0.150000,0.150000,0.150000}%
\pgfsetstrokecolor{textcolor}%
\pgfsetfillcolor{textcolor}%
\pgftext[x=6.107158in,y=0.960594in,,top]{\color{textcolor}\rmfamily\fontsize{35.200000}{42.240000}\selectfont 7}%
\end{pgfscope}%
\begin{pgfscope}%
\pgfpathrectangle{\pgfqpoint{1.633294in}{1.124483in}}{\pgfqpoint{5.425000in}{3.020000in}} %
\pgfusepath{clip}%
\pgfsetroundcap%
\pgfsetroundjoin%
\pgfsetlinewidth{0.803000pt}%
\definecolor{currentstroke}{rgb}{0.800000,0.800000,0.800000}%
\pgfsetstrokecolor{currentstroke}%
\pgfsetdash{}{0pt}%
\pgfpathmoveto{\pgfqpoint{6.811703in}{1.124483in}}%
\pgfpathlineto{\pgfqpoint{6.811703in}{4.144483in}}%
\pgfusepath{stroke}%
\end{pgfscope}%
\begin{pgfscope}%
\definecolor{textcolor}{rgb}{0.150000,0.150000,0.150000}%
\pgfsetstrokecolor{textcolor}%
\pgfsetfillcolor{textcolor}%
\pgftext[x=6.811703in,y=0.960594in,,top]{\color{textcolor}\rmfamily\fontsize{35.200000}{42.240000}\selectfont 8}%
\end{pgfscope}%
\begin{pgfscope}%
\definecolor{textcolor}{rgb}{0.150000,0.150000,0.150000}%
\pgfsetstrokecolor{textcolor}%
\pgfsetfillcolor{textcolor}%
\pgftext[x=4.345794in,y=0.470417in,,top]{\color{textcolor}\rmfamily\fontsize{30.000000}{36.000000}\selectfont Workload}%
\end{pgfscope}%
\begin{pgfscope}%
\pgfpathrectangle{\pgfqpoint{1.633294in}{1.124483in}}{\pgfqpoint{5.425000in}{3.020000in}} %
\pgfusepath{clip}%
\pgfsetroundcap%
\pgfsetroundjoin%
\pgfsetlinewidth{0.803000pt}%
\definecolor{currentstroke}{rgb}{0.800000,0.800000,0.800000}%
\pgfsetstrokecolor{currentstroke}%
\pgfsetdash{}{0pt}%
\pgfpathmoveto{\pgfqpoint{1.633294in}{1.124483in}}%
\pgfpathlineto{\pgfqpoint{7.058294in}{1.124483in}}%
\pgfusepath{stroke}%
\end{pgfscope}%
\begin{pgfscope}%
\definecolor{textcolor}{rgb}{0.150000,0.150000,0.150000}%
\pgfsetstrokecolor{textcolor}%
\pgfsetfillcolor{textcolor}%
\pgftext[x=1.245005in,y=0.954838in,left,base]{\color{textcolor}\rmfamily\fontsize{35.200000}{42.240000}\selectfont 0}%
\end{pgfscope}%
\begin{pgfscope}%
\pgfpathrectangle{\pgfqpoint{1.633294in}{1.124483in}}{\pgfqpoint{5.425000in}{3.020000in}} %
\pgfusepath{clip}%
\pgfsetroundcap%
\pgfsetroundjoin%
\pgfsetlinewidth{0.803000pt}%
\definecolor{currentstroke}{rgb}{0.800000,0.800000,0.800000}%
\pgfsetstrokecolor{currentstroke}%
\pgfsetdash{}{0pt}%
\pgfpathmoveto{\pgfqpoint{1.633294in}{2.497210in}}%
\pgfpathlineto{\pgfqpoint{7.058294in}{2.497210in}}%
\pgfusepath{stroke}%
\end{pgfscope}%
\begin{pgfscope}%
\definecolor{textcolor}{rgb}{0.150000,0.150000,0.150000}%
\pgfsetstrokecolor{textcolor}%
\pgfsetfillcolor{textcolor}%
\pgftext[x=0.571805in,y=2.327566in,left,base]{\color{textcolor}\rmfamily\fontsize{35.200000}{42.240000}\selectfont 1000}%
\end{pgfscope}%
\begin{pgfscope}%
\pgfpathrectangle{\pgfqpoint{1.633294in}{1.124483in}}{\pgfqpoint{5.425000in}{3.020000in}} %
\pgfusepath{clip}%
\pgfsetroundcap%
\pgfsetroundjoin%
\pgfsetlinewidth{0.803000pt}%
\definecolor{currentstroke}{rgb}{0.800000,0.800000,0.800000}%
\pgfsetstrokecolor{currentstroke}%
\pgfsetdash{}{0pt}%
\pgfpathmoveto{\pgfqpoint{1.633294in}{3.869937in}}%
\pgfpathlineto{\pgfqpoint{7.058294in}{3.869937in}}%
\pgfusepath{stroke}%
\end{pgfscope}%
\begin{pgfscope}%
\definecolor{textcolor}{rgb}{0.150000,0.150000,0.150000}%
\pgfsetstrokecolor{textcolor}%
\pgfsetfillcolor{textcolor}%
\pgftext[x=0.571805in,y=3.700293in,left,base]{\color{textcolor}\rmfamily\fontsize{35.200000}{42.240000}\selectfont 2000}%
\end{pgfscope}%
\begin{pgfscope}%
\definecolor{textcolor}{rgb}{0.150000,0.150000,0.150000}%
\pgfsetstrokecolor{textcolor}%
\pgfsetfillcolor{textcolor}%
\pgftext[x=0.516250in,y=2.634483in,,bottom,rotate=90.000000]{\color{textcolor}\rmfamily\fontsize{30.000000}{36.000000}\selectfont Cumulative Run Time (s)}%
\end{pgfscope}%
\begin{pgfscope}%
\pgfpathrectangle{\pgfqpoint{1.633294in}{1.124483in}}{\pgfqpoint{5.425000in}{3.020000in}} %
\pgfusepath{clip}%
\pgfsetbuttcap%
\pgfsetroundjoin%
\pgfsetlinewidth{3.011250pt}%
\definecolor{currentstroke}{rgb}{0.298039,0.447059,0.690196}%
\pgfsetstrokecolor{currentstroke}%
\pgfsetdash{{3.000000pt}{0.000000pt}}{0.000000pt}%
\pgfpathmoveto{\pgfqpoint{1.879885in}{1.399546in}}%
\pgfpathlineto{\pgfqpoint{2.584431in}{1.634268in}}%
\pgfpathlineto{\pgfqpoint{3.288976in}{2.039460in}}%
\pgfpathlineto{\pgfqpoint{3.993521in}{2.116980in}}%
\pgfpathlineto{\pgfqpoint{4.698067in}{2.171519in}}%
\pgfpathlineto{\pgfqpoint{5.402612in}{2.201384in}}%
\pgfpathlineto{\pgfqpoint{6.107158in}{2.276724in}}%
\pgfpathlineto{\pgfqpoint{6.811703in}{2.360706in}}%
\pgfusepath{stroke}%
\end{pgfscope}%
\begin{pgfscope}%
\pgfpathrectangle{\pgfqpoint{1.633294in}{1.124483in}}{\pgfqpoint{5.425000in}{3.020000in}} %
\pgfusepath{clip}%
\pgfsetbuttcap%
\pgfsetroundjoin%
\definecolor{currentfill}{rgb}{0.298039,0.447059,0.690196}%
\pgfsetfillcolor{currentfill}%
\pgfsetlinewidth{0.752812pt}%
\definecolor{currentstroke}{rgb}{1.000000,1.000000,1.000000}%
\pgfsetstrokecolor{currentstroke}%
\pgfsetdash{}{0pt}%
\pgfsys@defobject{currentmarker}{\pgfqpoint{-0.104167in}{-0.104167in}}{\pgfqpoint{0.104167in}{0.104167in}}{%
\pgfpathmoveto{\pgfqpoint{0.000000in}{-0.104167in}}%
\pgfpathcurveto{\pgfqpoint{0.027625in}{-0.104167in}}{\pgfqpoint{0.054123in}{-0.093191in}}{\pgfqpoint{0.073657in}{-0.073657in}}%
\pgfpathcurveto{\pgfqpoint{0.093191in}{-0.054123in}}{\pgfqpoint{0.104167in}{-0.027625in}}{\pgfqpoint{0.104167in}{0.000000in}}%
\pgfpathcurveto{\pgfqpoint{0.104167in}{0.027625in}}{\pgfqpoint{0.093191in}{0.054123in}}{\pgfqpoint{0.073657in}{0.073657in}}%
\pgfpathcurveto{\pgfqpoint{0.054123in}{0.093191in}}{\pgfqpoint{0.027625in}{0.104167in}}{\pgfqpoint{0.000000in}{0.104167in}}%
\pgfpathcurveto{\pgfqpoint{-0.027625in}{0.104167in}}{\pgfqpoint{-0.054123in}{0.093191in}}{\pgfqpoint{-0.073657in}{0.073657in}}%
\pgfpathcurveto{\pgfqpoint{-0.093191in}{0.054123in}}{\pgfqpoint{-0.104167in}{0.027625in}}{\pgfqpoint{-0.104167in}{0.000000in}}%
\pgfpathcurveto{\pgfqpoint{-0.104167in}{-0.027625in}}{\pgfqpoint{-0.093191in}{-0.054123in}}{\pgfqpoint{-0.073657in}{-0.073657in}}%
\pgfpathcurveto{\pgfqpoint{-0.054123in}{-0.093191in}}{\pgfqpoint{-0.027625in}{-0.104167in}}{\pgfqpoint{0.000000in}{-0.104167in}}%
\pgfpathclose%
\pgfusepath{stroke,fill}%
}%
\begin{pgfscope}%
\pgfsys@transformshift{1.879885in}{1.399546in}%
\pgfsys@useobject{currentmarker}{}%
\end{pgfscope}%
\begin{pgfscope}%
\pgfsys@transformshift{2.584431in}{1.634268in}%
\pgfsys@useobject{currentmarker}{}%
\end{pgfscope}%
\begin{pgfscope}%
\pgfsys@transformshift{3.288976in}{2.039460in}%
\pgfsys@useobject{currentmarker}{}%
\end{pgfscope}%
\begin{pgfscope}%
\pgfsys@transformshift{3.993521in}{2.116980in}%
\pgfsys@useobject{currentmarker}{}%
\end{pgfscope}%
\begin{pgfscope}%
\pgfsys@transformshift{4.698067in}{2.171519in}%
\pgfsys@useobject{currentmarker}{}%
\end{pgfscope}%
\begin{pgfscope}%
\pgfsys@transformshift{5.402612in}{2.201384in}%
\pgfsys@useobject{currentmarker}{}%
\end{pgfscope}%
\begin{pgfscope}%
\pgfsys@transformshift{6.107158in}{2.276724in}%
\pgfsys@useobject{currentmarker}{}%
\end{pgfscope}%
\begin{pgfscope}%
\pgfsys@transformshift{6.811703in}{2.360706in}%
\pgfsys@useobject{currentmarker}{}%
\end{pgfscope}%
\end{pgfscope}%
\begin{pgfscope}%
\pgfpathrectangle{\pgfqpoint{1.633294in}{1.124483in}}{\pgfqpoint{5.425000in}{3.020000in}} %
\pgfusepath{clip}%
\pgfsetbuttcap%
\pgfsetroundjoin%
\pgfsetlinewidth{3.011250pt}%
\definecolor{currentstroke}{rgb}{0.866667,0.517647,0.321569}%
\pgfsetstrokecolor{currentstroke}%
\pgfsetdash{{6.000000pt}{6.000000pt}}{0.000000pt}%
\pgfpathmoveto{\pgfqpoint{1.879885in}{1.397720in}}%
\pgfpathlineto{\pgfqpoint{2.584431in}{1.634267in}}%
\pgfpathlineto{\pgfqpoint{3.288976in}{2.042740in}}%
\pgfpathlineto{\pgfqpoint{3.993521in}{2.135089in}}%
\pgfpathlineto{\pgfqpoint{4.698067in}{2.203416in}}%
\pgfpathlineto{\pgfqpoint{5.402612in}{2.251375in}}%
\pgfpathlineto{\pgfqpoint{6.107158in}{2.549296in}}%
\pgfpathlineto{\pgfqpoint{6.811703in}{2.665939in}}%
\pgfusepath{stroke}%
\end{pgfscope}%
\begin{pgfscope}%
\pgfpathrectangle{\pgfqpoint{1.633294in}{1.124483in}}{\pgfqpoint{5.425000in}{3.020000in}} %
\pgfusepath{clip}%
\pgfsetbuttcap%
\pgfsetmiterjoin%
\definecolor{currentfill}{rgb}{0.866667,0.517647,0.321569}%
\pgfsetfillcolor{currentfill}%
\pgfsetlinewidth{0.752812pt}%
\definecolor{currentstroke}{rgb}{1.000000,1.000000,1.000000}%
\pgfsetstrokecolor{currentstroke}%
\pgfsetdash{}{0pt}%
\pgfsys@defobject{currentmarker}{\pgfqpoint{-0.104167in}{-0.104167in}}{\pgfqpoint{0.104167in}{0.104167in}}{%
\pgfpathmoveto{\pgfqpoint{0.000000in}{0.104167in}}%
\pgfpathlineto{\pgfqpoint{-0.104167in}{-0.104167in}}%
\pgfpathlineto{\pgfqpoint{0.104167in}{-0.104167in}}%
\pgfpathclose%
\pgfusepath{stroke,fill}%
}%
\begin{pgfscope}%
\pgfsys@transformshift{1.879885in}{1.397720in}%
\pgfsys@useobject{currentmarker}{}%
\end{pgfscope}%
\begin{pgfscope}%
\pgfsys@transformshift{2.584431in}{1.634267in}%
\pgfsys@useobject{currentmarker}{}%
\end{pgfscope}%
\begin{pgfscope}%
\pgfsys@transformshift{3.288976in}{2.042740in}%
\pgfsys@useobject{currentmarker}{}%
\end{pgfscope}%
\begin{pgfscope}%
\pgfsys@transformshift{3.993521in}{2.135089in}%
\pgfsys@useobject{currentmarker}{}%
\end{pgfscope}%
\begin{pgfscope}%
\pgfsys@transformshift{4.698067in}{2.203416in}%
\pgfsys@useobject{currentmarker}{}%
\end{pgfscope}%
\begin{pgfscope}%
\pgfsys@transformshift{5.402612in}{2.251375in}%
\pgfsys@useobject{currentmarker}{}%
\end{pgfscope}%
\begin{pgfscope}%
\pgfsys@transformshift{6.107158in}{2.549296in}%
\pgfsys@useobject{currentmarker}{}%
\end{pgfscope}%
\begin{pgfscope}%
\pgfsys@transformshift{6.811703in}{2.665939in}%
\pgfsys@useobject{currentmarker}{}%
\end{pgfscope}%
\end{pgfscope}%
\begin{pgfscope}%
\pgfpathrectangle{\pgfqpoint{1.633294in}{1.124483in}}{\pgfqpoint{5.425000in}{3.020000in}} %
\pgfusepath{clip}%
\pgfsetbuttcap%
\pgfsetroundjoin%
\pgfsetlinewidth{3.011250pt}%
\definecolor{currentstroke}{rgb}{0.333333,0.658824,0.407843}%
\pgfsetstrokecolor{currentstroke}%
\pgfsetdash{{3.000000pt}{3.000000pt}}{0.000000pt}%
\pgfpathmoveto{\pgfqpoint{1.879885in}{1.398004in}}%
\pgfpathlineto{\pgfqpoint{2.584431in}{1.631590in}}%
\pgfpathlineto{\pgfqpoint{3.288976in}{2.033293in}}%
\pgfpathlineto{\pgfqpoint{3.993521in}{2.251201in}}%
\pgfpathlineto{\pgfqpoint{4.698067in}{2.398961in}}%
\pgfpathlineto{\pgfqpoint{5.402612in}{2.609379in}}%
\pgfpathlineto{\pgfqpoint{6.107158in}{3.503321in}}%
\pgfpathlineto{\pgfqpoint{6.811703in}{3.832312in}}%
\pgfusepath{stroke}%
\end{pgfscope}%
\begin{pgfscope}%
\pgfpathrectangle{\pgfqpoint{1.633294in}{1.124483in}}{\pgfqpoint{5.425000in}{3.020000in}} %
\pgfusepath{clip}%
\pgfsetbuttcap%
\pgfsetmiterjoin%
\definecolor{currentfill}{rgb}{0.333333,0.658824,0.407843}%
\pgfsetfillcolor{currentfill}%
\pgfsetlinewidth{0.752812pt}%
\definecolor{currentstroke}{rgb}{1.000000,1.000000,1.000000}%
\pgfsetstrokecolor{currentstroke}%
\pgfsetdash{}{0pt}%
\pgfsys@defobject{currentmarker}{\pgfqpoint{-0.104167in}{-0.104167in}}{\pgfqpoint{0.104167in}{0.104167in}}{%
\pgfpathmoveto{\pgfqpoint{-0.000000in}{-0.104167in}}%
\pgfpathlineto{\pgfqpoint{0.104167in}{0.104167in}}%
\pgfpathlineto{\pgfqpoint{-0.104167in}{0.104167in}}%
\pgfpathclose%
\pgfusepath{stroke,fill}%
}%
\begin{pgfscope}%
\pgfsys@transformshift{1.879885in}{1.398004in}%
\pgfsys@useobject{currentmarker}{}%
\end{pgfscope}%
\begin{pgfscope}%
\pgfsys@transformshift{2.584431in}{1.631590in}%
\pgfsys@useobject{currentmarker}{}%
\end{pgfscope}%
\begin{pgfscope}%
\pgfsys@transformshift{3.288976in}{2.033293in}%
\pgfsys@useobject{currentmarker}{}%
\end{pgfscope}%
\begin{pgfscope}%
\pgfsys@transformshift{3.993521in}{2.251201in}%
\pgfsys@useobject{currentmarker}{}%
\end{pgfscope}%
\begin{pgfscope}%
\pgfsys@transformshift{4.698067in}{2.398961in}%
\pgfsys@useobject{currentmarker}{}%
\end{pgfscope}%
\begin{pgfscope}%
\pgfsys@transformshift{5.402612in}{2.609379in}%
\pgfsys@useobject{currentmarker}{}%
\end{pgfscope}%
\begin{pgfscope}%
\pgfsys@transformshift{6.107158in}{3.503321in}%
\pgfsys@useobject{currentmarker}{}%
\end{pgfscope}%
\begin{pgfscope}%
\pgfsys@transformshift{6.811703in}{3.832312in}%
\pgfsys@useobject{currentmarker}{}%
\end{pgfscope}%
\end{pgfscope}%
\begin{pgfscope}%
\pgfsetrectcap%
\pgfsetmiterjoin%
\pgfsetlinewidth{1.003750pt}%
\definecolor{currentstroke}{rgb}{0.800000,0.800000,0.800000}%
\pgfsetstrokecolor{currentstroke}%
\pgfsetdash{}{0pt}%
\pgfpathmoveto{\pgfqpoint{1.633294in}{1.124483in}}%
\pgfpathlineto{\pgfqpoint{1.633294in}{4.144483in}}%
\pgfusepath{stroke}%
\end{pgfscope}%
\begin{pgfscope}%
\pgfsetrectcap%
\pgfsetmiterjoin%
\pgfsetlinewidth{1.003750pt}%
\definecolor{currentstroke}{rgb}{0.800000,0.800000,0.800000}%
\pgfsetstrokecolor{currentstroke}%
\pgfsetdash{}{0pt}%
\pgfpathmoveto{\pgfqpoint{7.058294in}{1.124483in}}%
\pgfpathlineto{\pgfqpoint{7.058294in}{4.144483in}}%
\pgfusepath{stroke}%
\end{pgfscope}%
\begin{pgfscope}%
\pgfsetrectcap%
\pgfsetmiterjoin%
\pgfsetlinewidth{1.003750pt}%
\definecolor{currentstroke}{rgb}{0.800000,0.800000,0.800000}%
\pgfsetstrokecolor{currentstroke}%
\pgfsetdash{}{0pt}%
\pgfpathmoveto{\pgfqpoint{1.633294in}{1.124483in}}%
\pgfpathlineto{\pgfqpoint{7.058294in}{1.124483in}}%
\pgfusepath{stroke}%
\end{pgfscope}%
\begin{pgfscope}%
\pgfsetrectcap%
\pgfsetmiterjoin%
\pgfsetlinewidth{1.003750pt}%
\definecolor{currentstroke}{rgb}{0.800000,0.800000,0.800000}%
\pgfsetstrokecolor{currentstroke}%
\pgfsetdash{}{0pt}%
\pgfpathmoveto{\pgfqpoint{1.633294in}{4.144483in}}%
\pgfpathlineto{\pgfqpoint{7.058294in}{4.144483in}}%
\pgfusepath{stroke}%
\end{pgfscope}%
\begin{pgfscope}%
\pgfsetroundcap%
\pgfsetroundjoin%
\pgfsetlinewidth{3.011250pt}%
\definecolor{currentstroke}{rgb}{1.000000,1.000000,1.000000}%
\pgfsetstrokecolor{currentstroke}%
\pgfsetdash{}{0pt}%
\pgfpathmoveto{\pgfqpoint{1.639294in}{4.439372in}}%
\pgfpathlineto{\pgfqpoint{2.617072in}{4.439372in}}%
\pgfusepath{stroke}%
\end{pgfscope}%
\begin{pgfscope}%
\pgfsetbuttcap%
\pgfsetroundjoin%
\pgfsetlinewidth{3.011250pt}%
\definecolor{currentstroke}{rgb}{0.298039,0.447059,0.690196}%
\pgfsetstrokecolor{currentstroke}%
\pgfsetdash{{3.000000pt}{0.000000pt}}{0.000000pt}%
\pgfpathmoveto{\pgfqpoint{1.639294in}{4.001817in}}%
\pgfpathlineto{\pgfqpoint{2.617072in}{4.001817in}}%
\pgfusepath{stroke}%
\end{pgfscope}%
\begin{pgfscope}%
\pgfsetbuttcap%
\pgfsetroundjoin%
\definecolor{currentfill}{rgb}{0.298039,0.447059,0.690196}%
\pgfsetfillcolor{currentfill}%
\pgfsetlinewidth{0.000000pt}%
\definecolor{currentstroke}{rgb}{0.298039,0.447059,0.690196}%
\pgfsetstrokecolor{currentstroke}%
\pgfsetdash{}{0pt}%
\pgfsys@defobject{currentmarker}{\pgfqpoint{-0.104167in}{-0.104167in}}{\pgfqpoint{0.104167in}{0.104167in}}{%
\pgfpathmoveto{\pgfqpoint{0.000000in}{-0.104167in}}%
\pgfpathcurveto{\pgfqpoint{0.027625in}{-0.104167in}}{\pgfqpoint{0.054123in}{-0.093191in}}{\pgfqpoint{0.073657in}{-0.073657in}}%
\pgfpathcurveto{\pgfqpoint{0.093191in}{-0.054123in}}{\pgfqpoint{0.104167in}{-0.027625in}}{\pgfqpoint{0.104167in}{0.000000in}}%
\pgfpathcurveto{\pgfqpoint{0.104167in}{0.027625in}}{\pgfqpoint{0.093191in}{0.054123in}}{\pgfqpoint{0.073657in}{0.073657in}}%
\pgfpathcurveto{\pgfqpoint{0.054123in}{0.093191in}}{\pgfqpoint{0.027625in}{0.104167in}}{\pgfqpoint{0.000000in}{0.104167in}}%
\pgfpathcurveto{\pgfqpoint{-0.027625in}{0.104167in}}{\pgfqpoint{-0.054123in}{0.093191in}}{\pgfqpoint{-0.073657in}{0.073657in}}%
\pgfpathcurveto{\pgfqpoint{-0.093191in}{0.054123in}}{\pgfqpoint{-0.104167in}{0.027625in}}{\pgfqpoint{-0.104167in}{0.000000in}}%
\pgfpathcurveto{\pgfqpoint{-0.104167in}{-0.027625in}}{\pgfqpoint{-0.093191in}{-0.054123in}}{\pgfqpoint{-0.073657in}{-0.073657in}}%
\pgfpathcurveto{\pgfqpoint{-0.054123in}{-0.093191in}}{\pgfqpoint{-0.027625in}{-0.104167in}}{\pgfqpoint{0.000000in}{-0.104167in}}%
\pgfpathclose%
\pgfusepath{fill}%
}%
\begin{pgfscope}%
\pgfsys@transformshift{2.128183in}{4.001817in}%
\pgfsys@useobject{currentmarker}{}%
\end{pgfscope}%
\end{pgfscope}%
\begin{pgfscope}%
\definecolor{textcolor}{rgb}{0.150000,0.150000,0.150000}%
\pgfsetstrokecolor{textcolor}%
\pgfsetfillcolor{textcolor}%
\pgftext[x=2.665961in,y=3.830705in,left,base]{\color{textcolor}\rmfamily\fontsize{35.200000}{42.240000}\selectfont LN}%
\end{pgfscope}%
\begin{pgfscope}%
\pgfsetbuttcap%
\pgfsetroundjoin%
\pgfsetlinewidth{3.011250pt}%
\definecolor{currentstroke}{rgb}{0.866667,0.517647,0.321569}%
\pgfsetstrokecolor{currentstroke}%
\pgfsetdash{{6.000000pt}{6.000000pt}}{0.000000pt}%
\pgfpathmoveto{\pgfqpoint{1.639294in}{3.564261in}}%
\pgfpathlineto{\pgfqpoint{2.617072in}{3.564261in}}%
\pgfusepath{stroke}%
\end{pgfscope}%
\begin{pgfscope}%
\pgfsetbuttcap%
\pgfsetmiterjoin%
\definecolor{currentfill}{rgb}{0.866667,0.517647,0.321569}%
\pgfsetfillcolor{currentfill}%
\pgfsetlinewidth{0.000000pt}%
\definecolor{currentstroke}{rgb}{0.866667,0.517647,0.321569}%
\pgfsetstrokecolor{currentstroke}%
\pgfsetdash{}{0pt}%
\pgfsys@defobject{currentmarker}{\pgfqpoint{-0.104167in}{-0.104167in}}{\pgfqpoint{0.104167in}{0.104167in}}{%
\pgfpathmoveto{\pgfqpoint{0.000000in}{0.104167in}}%
\pgfpathlineto{\pgfqpoint{-0.104167in}{-0.104167in}}%
\pgfpathlineto{\pgfqpoint{0.104167in}{-0.104167in}}%
\pgfpathclose%
\pgfusepath{fill}%
}%
\begin{pgfscope}%
\pgfsys@transformshift{2.128183in}{3.564261in}%
\pgfsys@useobject{currentmarker}{}%
\end{pgfscope}%
\end{pgfscope}%
\begin{pgfscope}%
\definecolor{textcolor}{rgb}{0.150000,0.150000,0.150000}%
\pgfsetstrokecolor{textcolor}%
\pgfsetfillcolor{textcolor}%
\pgftext[x=2.665961in,y=3.393150in,left,base]{\color{textcolor}\rmfamily\fontsize{35.200000}{42.240000}\selectfont ALL\_M}%
\end{pgfscope}%
\begin{pgfscope}%
\pgfsetbuttcap%
\pgfsetroundjoin%
\pgfsetlinewidth{3.011250pt}%
\definecolor{currentstroke}{rgb}{0.333333,0.658824,0.407843}%
\pgfsetstrokecolor{currentstroke}%
\pgfsetdash{{3.000000pt}{3.000000pt}}{0.000000pt}%
\pgfpathmoveto{\pgfqpoint{1.639294in}{3.126706in}}%
\pgfpathlineto{\pgfqpoint{2.617072in}{3.126706in}}%
\pgfusepath{stroke}%
\end{pgfscope}%
\begin{pgfscope}%
\pgfsetbuttcap%
\pgfsetmiterjoin%
\definecolor{currentfill}{rgb}{0.333333,0.658824,0.407843}%
\pgfsetfillcolor{currentfill}%
\pgfsetlinewidth{0.000000pt}%
\definecolor{currentstroke}{rgb}{0.333333,0.658824,0.407843}%
\pgfsetstrokecolor{currentstroke}%
\pgfsetdash{}{0pt}%
\pgfsys@defobject{currentmarker}{\pgfqpoint{-0.104167in}{-0.104167in}}{\pgfqpoint{0.104167in}{0.104167in}}{%
\pgfpathmoveto{\pgfqpoint{-0.000000in}{-0.104167in}}%
\pgfpathlineto{\pgfqpoint{0.104167in}{0.104167in}}%
\pgfpathlineto{\pgfqpoint{-0.104167in}{0.104167in}}%
\pgfpathclose%
\pgfusepath{fill}%
}%
\begin{pgfscope}%
\pgfsys@transformshift{2.128183in}{3.126706in}%
\pgfsys@useobject{currentmarker}{}%
\end{pgfscope}%
\end{pgfscope}%
\begin{pgfscope}%
\definecolor{textcolor}{rgb}{0.150000,0.150000,0.150000}%
\pgfsetstrokecolor{textcolor}%
\pgfsetfillcolor{textcolor}%
\pgftext[x=2.665961in,y=2.955595in,left,base]{\color{textcolor}\rmfamily\fontsize{35.200000}{42.240000}\selectfont ALL\_C}%
\end{pgfscope}%
\end{pgfpicture}%
\makeatother%
\endgroup%
%
}
\caption{Storage-aware}
\end{subfigure}
\caption{Run-time of linear-time (LN), all compute (ALL\_C), and all materialized (ALL\_M) reuse methods.}
\label{reuse-experiment}
\vspace{-8mm}
\end{figure}
\subsection{Reuse}
In this experiment, we evaluate the performance of our reuse algorithm on Kaggle workloads.
We compare our linear time reuse algorithm (LN) with two baselines.
In the first baseline (ALL\_M), we reuse every materialized artifact.
In the second baseline (ALL\_C), we recompute every artifact.
Figure \ref{reuse-experiment} shows the total run-time of the Kaggle workloads with different reuse approaches under different materialization algorithms.
ALL\_C, independent of the materialization algorithm and budget, finishes the execution of all the Kaggle workloads in around 2000.
For HM materialization, all three reuse algorithms have similar performance until Workload 6.
Since Workload 3 has large artifacts, HM exhausts its budget by materializing them (as explained the materialization experiments).
Furthermore, Workloads 4, 5, and 6 are modified versions of Workloads 1 and 2 (Table \ref{kaggle-workload}).
Therefore, there are not many reuse opportunities until Workload 7.
However, for SA materialization, both ALL\_M and LN outperform ALL\_C starting from Workload 4.
SA has better utilization of the budget and materializes some of the artifacts of the Workload 1 and 2.
ALL\_M and LN reuse these artifacts in Workloads 4, 5, and 6; thus, incurring a smaller run-time.
For both materialization algorithms, ALL\_M has a similar performance to LN until Workload 7.
Many of the artifacts of Workload 7 incur larger load costs than compute costs.
As a result, LN recomputes these artifacts and results in a smaller cumulative run-time than ALL\_M, i.e., around 300 seconds in both HM and SA materialization.
In this experiment, EG is inside the memory of the machine; thus, load times are generally low.
In scenarios where EG is on disk, we expect LN to outperform ALL\_M with a larger margin due to the higher load cost of the materialized artifacts.
\begin{figure}[t]
\begin{subfigure}[b]{0.5\linewidth}
\centering
 \resizebox{\columnwidth}{!}{%
%% Creator: Matplotlib, PGF backend
%%
%% To include the figure in your LaTeX document, write
%%   \input{<filename>.pgf}
%%
%% Make sure the required packages are loaded in your preamble
%%   \usepackage{pgf}
%%
%% Figures using additional raster images can only be included by \input if
%% they are in the same directory as the main LaTeX file. For loading figures
%% from other directories you can use the `import` package
%%   \usepackage{import}
%% and then include the figures with
%%   \import{<path to file>}{<filename>.pgf}
%%
%% Matplotlib used the following preamble
%%   \usepackage{fontspec}
%%   \setmonofont{Andale Mono}
%%
\begingroup%
\makeatletter%
\begin{pgfpicture}%
\pgfpathrectangle{\pgfpointorigin}{\pgfqpoint{7.571861in}{4.830486in}}%
\pgfusepath{use as bounding box, clip}%
\begin{pgfscope}%
\pgfsetbuttcap%
\pgfsetmiterjoin%
\definecolor{currentfill}{rgb}{1.000000,1.000000,1.000000}%
\pgfsetfillcolor{currentfill}%
\pgfsetlinewidth{0.000000pt}%
\definecolor{currentstroke}{rgb}{1.000000,1.000000,1.000000}%
\pgfsetstrokecolor{currentstroke}%
\pgfsetdash{}{0pt}%
\pgfpathmoveto{\pgfqpoint{0.000000in}{0.000000in}}%
\pgfpathlineto{\pgfqpoint{7.571861in}{0.000000in}}%
\pgfpathlineto{\pgfqpoint{7.571861in}{4.830486in}}%
\pgfpathlineto{\pgfqpoint{0.000000in}{4.830486in}}%
\pgfpathclose%
\pgfusepath{fill}%
\end{pgfscope}%
\begin{pgfscope}%
\pgfsetbuttcap%
\pgfsetmiterjoin%
\definecolor{currentfill}{rgb}{1.000000,1.000000,1.000000}%
\pgfsetfillcolor{currentfill}%
\pgfsetlinewidth{0.000000pt}%
\definecolor{currentstroke}{rgb}{0.000000,0.000000,0.000000}%
\pgfsetstrokecolor{currentstroke}%
\pgfsetstrokeopacity{0.000000}%
\pgfsetdash{}{0pt}%
\pgfpathmoveto{\pgfqpoint{1.309444in}{1.062361in}}%
\pgfpathlineto{\pgfqpoint{6.734444in}{1.062361in}}%
\pgfpathlineto{\pgfqpoint{6.734444in}{4.082361in}}%
\pgfpathlineto{\pgfqpoint{1.309444in}{4.082361in}}%
\pgfpathclose%
\pgfusepath{fill}%
\end{pgfscope}%
\begin{pgfscope}%
\pgfpathrectangle{\pgfqpoint{1.309444in}{1.062361in}}{\pgfqpoint{5.425000in}{3.020000in}} %
\pgfusepath{clip}%
\pgfsetroundcap%
\pgfsetroundjoin%
\pgfsetlinewidth{0.803000pt}%
\definecolor{currentstroke}{rgb}{0.800000,0.800000,0.800000}%
\pgfsetstrokecolor{currentstroke}%
\pgfsetdash{}{0pt}%
\pgfpathmoveto{\pgfqpoint{1.553568in}{1.062361in}}%
\pgfpathlineto{\pgfqpoint{1.553568in}{4.082361in}}%
\pgfusepath{stroke}%
\end{pgfscope}%
\begin{pgfscope}%
\definecolor{textcolor}{rgb}{0.150000,0.150000,0.150000}%
\pgfsetstrokecolor{textcolor}%
\pgfsetfillcolor{textcolor}%
\pgftext[x=1.553568in,y=0.898472in,,top]{\color{textcolor}\rmfamily\fontsize{30.000000}{36.000000}\selectfont 0}%
\end{pgfscope}%
\begin{pgfscope}%
\pgfpathrectangle{\pgfqpoint{1.309444in}{1.062361in}}{\pgfqpoint{5.425000in}{3.020000in}} %
\pgfusepath{clip}%
\pgfsetroundcap%
\pgfsetroundjoin%
\pgfsetlinewidth{0.803000pt}%
\definecolor{currentstroke}{rgb}{0.800000,0.800000,0.800000}%
\pgfsetstrokecolor{currentstroke}%
\pgfsetdash{}{0pt}%
\pgfpathmoveto{\pgfqpoint{4.020711in}{1.062361in}}%
\pgfpathlineto{\pgfqpoint{4.020711in}{4.082361in}}%
\pgfusepath{stroke}%
\end{pgfscope}%
\begin{pgfscope}%
\definecolor{textcolor}{rgb}{0.150000,0.150000,0.150000}%
\pgfsetstrokecolor{textcolor}%
\pgfsetfillcolor{textcolor}%
\pgftext[x=4.020711in,y=0.898472in,,top]{\color{textcolor}\rmfamily\fontsize{30.000000}{36.000000}\selectfont 1000}%
\end{pgfscope}%
\begin{pgfscope}%
\pgfpathrectangle{\pgfqpoint{1.309444in}{1.062361in}}{\pgfqpoint{5.425000in}{3.020000in}} %
\pgfusepath{clip}%
\pgfsetroundcap%
\pgfsetroundjoin%
\pgfsetlinewidth{0.803000pt}%
\definecolor{currentstroke}{rgb}{0.800000,0.800000,0.800000}%
\pgfsetstrokecolor{currentstroke}%
\pgfsetdash{}{0pt}%
\pgfpathmoveto{\pgfqpoint{6.487853in}{1.062361in}}%
\pgfpathlineto{\pgfqpoint{6.487853in}{4.082361in}}%
\pgfusepath{stroke}%
\end{pgfscope}%
\begin{pgfscope}%
\definecolor{textcolor}{rgb}{0.150000,0.150000,0.150000}%
\pgfsetstrokecolor{textcolor}%
\pgfsetfillcolor{textcolor}%
\pgftext[x=6.487853in,y=0.898472in,,top]{\color{textcolor}\rmfamily\fontsize{30.000000}{36.000000}\selectfont 2000}%
\end{pgfscope}%
\begin{pgfscope}%
\definecolor{textcolor}{rgb}{0.150000,0.150000,0.150000}%
\pgfsetstrokecolor{textcolor}%
\pgfsetfillcolor{textcolor}%
\pgftext[x=4.021944in,y=0.472500in,,top]{\color{textcolor}\rmfamily\fontsize{30.000000}{36.000000}\selectfont OpenML Workload}%
\end{pgfscope}%
\begin{pgfscope}%
\pgfpathrectangle{\pgfqpoint{1.309444in}{1.062361in}}{\pgfqpoint{5.425000in}{3.020000in}} %
\pgfusepath{clip}%
\pgfsetroundcap%
\pgfsetroundjoin%
\pgfsetlinewidth{0.803000pt}%
\definecolor{currentstroke}{rgb}{0.800000,0.800000,0.800000}%
\pgfsetstrokecolor{currentstroke}%
\pgfsetdash{}{0pt}%
\pgfpathmoveto{\pgfqpoint{1.309444in}{1.193121in}}%
\pgfpathlineto{\pgfqpoint{6.734444in}{1.193121in}}%
\pgfusepath{stroke}%
\end{pgfscope}%
\begin{pgfscope}%
\definecolor{textcolor}{rgb}{0.150000,0.150000,0.150000}%
\pgfsetstrokecolor{textcolor}%
\pgfsetfillcolor{textcolor}%
\pgftext[x=0.954305in,y=1.048538in,left,base]{\color{textcolor}\rmfamily\fontsize{30.000000}{36.000000}\selectfont 0}%
\end{pgfscope}%
\begin{pgfscope}%
\pgfpathrectangle{\pgfqpoint{1.309444in}{1.062361in}}{\pgfqpoint{5.425000in}{3.020000in}} %
\pgfusepath{clip}%
\pgfsetroundcap%
\pgfsetroundjoin%
\pgfsetlinewidth{0.803000pt}%
\definecolor{currentstroke}{rgb}{0.800000,0.800000,0.800000}%
\pgfsetstrokecolor{currentstroke}%
\pgfsetdash{}{0pt}%
\pgfpathmoveto{\pgfqpoint{1.309444in}{3.187222in}}%
\pgfpathlineto{\pgfqpoint{6.734444in}{3.187222in}}%
\pgfusepath{stroke}%
\end{pgfscope}%
\begin{pgfscope}%
\definecolor{textcolor}{rgb}{0.150000,0.150000,0.150000}%
\pgfsetstrokecolor{textcolor}%
\pgfsetfillcolor{textcolor}%
\pgftext[x=0.571805in,y=3.042639in,left,base]{\color{textcolor}\rmfamily\fontsize{30.000000}{36.000000}\selectfont 500}%
\end{pgfscope}%
\begin{pgfscope}%
\definecolor{textcolor}{rgb}{0.150000,0.150000,0.150000}%
\pgfsetstrokecolor{textcolor}%
\pgfsetfillcolor{textcolor}%
\pgftext[x=0.516250in,y=2.572361in,,bottom,rotate=90.000000]{\color{textcolor}\rmfamily\fontsize{30.000000}{36.000000}\selectfont Cumulative Run Time (s)}%
\end{pgfscope}%
\begin{pgfscope}%
\pgfpathrectangle{\pgfqpoint{1.309444in}{1.062361in}}{\pgfqpoint{5.425000in}{3.020000in}} %
\pgfusepath{clip}%
\pgfsetbuttcap%
\pgfsetroundjoin%
\pgfsetlinewidth{3.011250pt}%
\definecolor{currentstroke}{rgb}{0.298039,0.447059,0.690196}%
\pgfsetstrokecolor{currentstroke}%
\pgfsetdash{{3.000000pt}{0.000000pt}}{0.000000pt}%
\pgfpathmoveto{\pgfqpoint{1.556035in}{1.199634in}}%
\pgfpathlineto{\pgfqpoint{1.563437in}{1.216234in}}%
\pgfpathlineto{\pgfqpoint{1.565904in}{1.293472in}}%
\pgfpathlineto{\pgfqpoint{1.588108in}{1.305585in}}%
\pgfpathlineto{\pgfqpoint{1.593042in}{1.311166in}}%
\pgfpathlineto{\pgfqpoint{1.627582in}{1.329271in}}%
\pgfpathlineto{\pgfqpoint{1.630049in}{1.353675in}}%
\pgfpathlineto{\pgfqpoint{1.642385in}{1.358992in}}%
\pgfpathlineto{\pgfqpoint{1.780545in}{1.362734in}}%
\pgfpathlineto{\pgfqpoint{2.488615in}{1.382824in}}%
\pgfpathlineto{\pgfqpoint{2.496016in}{1.386085in}}%
\pgfpathlineto{\pgfqpoint{2.498484in}{1.438779in}}%
\pgfpathlineto{\pgfqpoint{2.503418in}{1.441101in}}%
\pgfpathlineto{\pgfqpoint{2.508352in}{1.441407in}}%
\pgfpathlineto{\pgfqpoint{2.515754in}{1.449624in}}%
\pgfpathlineto{\pgfqpoint{2.528089in}{1.458801in}}%
\pgfpathlineto{\pgfqpoint{2.533024in}{1.460293in}}%
\pgfpathlineto{\pgfqpoint{2.542892in}{1.471378in}}%
\pgfpathlineto{\pgfqpoint{2.545359in}{1.474578in}}%
\pgfpathlineto{\pgfqpoint{2.552761in}{1.477430in}}%
\pgfpathlineto{\pgfqpoint{2.562629in}{1.485645in}}%
\pgfpathlineto{\pgfqpoint{2.565096in}{1.486181in}}%
\pgfpathlineto{\pgfqpoint{2.570031in}{1.489820in}}%
\pgfpathlineto{\pgfqpoint{2.577432in}{1.491349in}}%
\pgfpathlineto{\pgfqpoint{2.582366in}{1.493738in}}%
\pgfpathlineto{\pgfqpoint{2.584834in}{1.494587in}}%
\pgfpathlineto{\pgfqpoint{2.589768in}{1.498509in}}%
\pgfpathlineto{\pgfqpoint{2.592235in}{1.498976in}}%
\pgfpathlineto{\pgfqpoint{2.599636in}{1.504504in}}%
\pgfpathlineto{\pgfqpoint{2.604571in}{1.506927in}}%
\pgfpathlineto{\pgfqpoint{2.607038in}{1.510361in}}%
\pgfpathlineto{\pgfqpoint{2.609505in}{1.511051in}}%
\pgfpathlineto{\pgfqpoint{2.616906in}{1.519169in}}%
\pgfpathlineto{\pgfqpoint{2.648979in}{1.533930in}}%
\pgfpathlineto{\pgfqpoint{2.653914in}{1.540065in}}%
\pgfpathlineto{\pgfqpoint{2.668716in}{1.547318in}}%
\pgfpathlineto{\pgfqpoint{2.671184in}{1.550205in}}%
\pgfpathlineto{\pgfqpoint{2.673651in}{1.550555in}}%
\pgfpathlineto{\pgfqpoint{2.681052in}{1.555845in}}%
\pgfpathlineto{\pgfqpoint{2.688454in}{1.558739in}}%
\pgfpathlineto{\pgfqpoint{2.700789in}{1.564270in}}%
\pgfpathlineto{\pgfqpoint{2.713125in}{1.568325in}}%
\pgfpathlineto{\pgfqpoint{2.722994in}{1.569175in}}%
\pgfpathlineto{\pgfqpoint{2.727928in}{1.571302in}}%
\pgfpathlineto{\pgfqpoint{3.902288in}{1.674371in}}%
\pgfpathlineto{\pgfqpoint{4.144068in}{1.696051in}}%
\pgfpathlineto{\pgfqpoint{4.427789in}{1.720360in}}%
\pgfpathlineto{\pgfqpoint{4.876809in}{1.759564in}}%
\pgfpathlineto{\pgfqpoint{4.879276in}{1.764038in}}%
\pgfpathlineto{\pgfqpoint{5.138326in}{1.786942in}}%
\pgfpathlineto{\pgfqpoint{5.143261in}{1.788794in}}%
\pgfpathlineto{\pgfqpoint{5.352968in}{1.808035in}}%
\pgfpathlineto{\pgfqpoint{5.360369in}{1.809219in}}%
\pgfpathlineto{\pgfqpoint{5.471390in}{1.819363in}}%
\pgfpathlineto{\pgfqpoint{5.476325in}{1.821720in}}%
\pgfpathlineto{\pgfqpoint{5.513332in}{1.826273in}}%
\pgfpathlineto{\pgfqpoint{5.518266in}{1.828153in}}%
\pgfpathlineto{\pgfqpoint{5.609550in}{1.837442in}}%
\pgfpathlineto{\pgfqpoint{5.616952in}{1.838625in}}%
\pgfpathlineto{\pgfqpoint{5.686032in}{1.846060in}}%
\pgfpathlineto{\pgfqpoint{5.690966in}{1.848077in}}%
\pgfpathlineto{\pgfqpoint{5.801988in}{1.858807in}}%
\pgfpathlineto{\pgfqpoint{5.806922in}{1.860615in}}%
\pgfpathlineto{\pgfqpoint{5.977155in}{1.876003in}}%
\pgfpathlineto{\pgfqpoint{5.982089in}{1.877775in}}%
\pgfpathlineto{\pgfqpoint{5.989490in}{1.878967in}}%
\pgfpathlineto{\pgfqpoint{5.994425in}{1.880693in}}%
\pgfpathlineto{\pgfqpoint{6.102979in}{1.890115in}}%
\pgfpathlineto{\pgfqpoint{6.107913in}{1.892142in}}%
\pgfpathlineto{\pgfqpoint{6.186862in}{1.899016in}}%
\pgfpathlineto{\pgfqpoint{6.189329in}{2.083790in}}%
\pgfpathlineto{\pgfqpoint{6.263343in}{2.090806in}}%
\pgfpathlineto{\pgfqpoint{6.268278in}{2.092527in}}%
\pgfpathlineto{\pgfqpoint{6.401503in}{2.105192in}}%
\pgfpathlineto{\pgfqpoint{6.408905in}{2.106376in}}%
\pgfpathlineto{\pgfqpoint{6.487853in}{2.114606in}}%
\pgfpathlineto{\pgfqpoint{6.487853in}{2.114606in}}%
\pgfusepath{stroke}%
\end{pgfscope}%
\begin{pgfscope}%
\pgfpathrectangle{\pgfqpoint{1.309444in}{1.062361in}}{\pgfqpoint{5.425000in}{3.020000in}} %
\pgfusepath{clip}%
\pgfsetbuttcap%
\pgfsetroundjoin%
\pgfsetlinewidth{3.011250pt}%
\definecolor{currentstroke}{rgb}{0.866667,0.517647,0.321569}%
\pgfsetstrokecolor{currentstroke}%
\pgfsetdash{{3.000000pt}{3.000000pt}}{0.000000pt}%
\pgfpathmoveto{\pgfqpoint{1.556035in}{1.199713in}}%
\pgfpathlineto{\pgfqpoint{1.563437in}{1.217084in}}%
\pgfpathlineto{\pgfqpoint{1.565904in}{1.274201in}}%
\pgfpathlineto{\pgfqpoint{1.585641in}{1.328259in}}%
\pgfpathlineto{\pgfqpoint{1.588108in}{1.329375in}}%
\pgfpathlineto{\pgfqpoint{1.600444in}{1.350481in}}%
\pgfpathlineto{\pgfqpoint{1.625115in}{1.437793in}}%
\pgfpathlineto{\pgfqpoint{1.627582in}{1.443502in}}%
\pgfpathlineto{\pgfqpoint{1.632516in}{1.484346in}}%
\pgfpathlineto{\pgfqpoint{1.639918in}{1.506253in}}%
\pgfpathlineto{\pgfqpoint{1.686794in}{1.507706in}}%
\pgfpathlineto{\pgfqpoint{2.486148in}{1.531101in}}%
\pgfpathlineto{\pgfqpoint{2.496016in}{1.535918in}}%
\pgfpathlineto{\pgfqpoint{2.498484in}{1.588609in}}%
\pgfpathlineto{\pgfqpoint{2.503418in}{1.590997in}}%
\pgfpathlineto{\pgfqpoint{2.508352in}{1.591267in}}%
\pgfpathlineto{\pgfqpoint{2.515754in}{1.599409in}}%
\pgfpathlineto{\pgfqpoint{2.528089in}{1.608491in}}%
\pgfpathlineto{\pgfqpoint{2.533024in}{1.609926in}}%
\pgfpathlineto{\pgfqpoint{2.540425in}{1.618764in}}%
\pgfpathlineto{\pgfqpoint{2.547826in}{1.625166in}}%
\pgfpathlineto{\pgfqpoint{2.555228in}{1.630117in}}%
\pgfpathlineto{\pgfqpoint{2.570031in}{1.639797in}}%
\pgfpathlineto{\pgfqpoint{2.577432in}{1.641379in}}%
\pgfpathlineto{\pgfqpoint{2.582366in}{1.643772in}}%
\pgfpathlineto{\pgfqpoint{2.584834in}{1.644632in}}%
\pgfpathlineto{\pgfqpoint{2.589768in}{1.648545in}}%
\pgfpathlineto{\pgfqpoint{2.592235in}{1.649033in}}%
\pgfpathlineto{\pgfqpoint{2.599636in}{1.654637in}}%
\pgfpathlineto{\pgfqpoint{2.604571in}{1.657066in}}%
\pgfpathlineto{\pgfqpoint{2.607038in}{1.660447in}}%
\pgfpathlineto{\pgfqpoint{2.609505in}{1.661147in}}%
\pgfpathlineto{\pgfqpoint{2.616906in}{1.669134in}}%
\pgfpathlineto{\pgfqpoint{2.639111in}{1.679568in}}%
\pgfpathlineto{\pgfqpoint{2.644045in}{1.683092in}}%
\pgfpathlineto{\pgfqpoint{2.648979in}{1.684460in}}%
\pgfpathlineto{\pgfqpoint{2.653914in}{1.690505in}}%
\pgfpathlineto{\pgfqpoint{2.668716in}{1.697750in}}%
\pgfpathlineto{\pgfqpoint{2.671184in}{1.700588in}}%
\pgfpathlineto{\pgfqpoint{2.673651in}{1.700952in}}%
\pgfpathlineto{\pgfqpoint{2.681052in}{1.706184in}}%
\pgfpathlineto{\pgfqpoint{2.688454in}{1.709096in}}%
\pgfpathlineto{\pgfqpoint{2.700789in}{1.714656in}}%
\pgfpathlineto{\pgfqpoint{2.722994in}{1.726921in}}%
\pgfpathlineto{\pgfqpoint{2.730395in}{1.731026in}}%
\pgfpathlineto{\pgfqpoint{2.762468in}{1.748260in}}%
\pgfpathlineto{\pgfqpoint{2.772336in}{1.754900in}}%
\pgfpathlineto{\pgfqpoint{2.797008in}{1.769021in}}%
\pgfpathlineto{\pgfqpoint{2.806876in}{1.774750in}}%
\pgfpathlineto{\pgfqpoint{2.816745in}{1.780224in}}%
\pgfpathlineto{\pgfqpoint{2.831548in}{1.787574in}}%
\pgfpathlineto{\pgfqpoint{2.836482in}{1.791187in}}%
\pgfpathlineto{\pgfqpoint{2.856219in}{1.799931in}}%
\pgfpathlineto{\pgfqpoint{2.878424in}{1.812857in}}%
\pgfpathlineto{\pgfqpoint{2.925299in}{1.844366in}}%
\pgfpathlineto{\pgfqpoint{2.945036in}{1.853931in}}%
\pgfpathlineto{\pgfqpoint{2.954905in}{1.859494in}}%
\pgfpathlineto{\pgfqpoint{2.967241in}{1.866825in}}%
\pgfpathlineto{\pgfqpoint{2.989445in}{1.878699in}}%
\pgfpathlineto{\pgfqpoint{3.001781in}{1.884237in}}%
\pgfpathlineto{\pgfqpoint{3.014116in}{1.892152in}}%
\pgfpathlineto{\pgfqpoint{3.028919in}{1.898937in}}%
\pgfpathlineto{\pgfqpoint{3.058525in}{1.913525in}}%
\pgfpathlineto{\pgfqpoint{3.075795in}{1.924334in}}%
\pgfpathlineto{\pgfqpoint{3.088131in}{1.931927in}}%
\pgfpathlineto{\pgfqpoint{3.102934in}{1.940981in}}%
\pgfpathlineto{\pgfqpoint{3.130072in}{1.958629in}}%
\pgfpathlineto{\pgfqpoint{3.142408in}{1.964720in}}%
\pgfpathlineto{\pgfqpoint{3.149809in}{1.969579in}}%
\pgfpathlineto{\pgfqpoint{3.169546in}{1.980559in}}%
\pgfpathlineto{\pgfqpoint{3.181882in}{1.986735in}}%
\pgfpathlineto{\pgfqpoint{3.223824in}{2.008131in}}%
\pgfpathlineto{\pgfqpoint{3.255896in}{2.025015in}}%
\pgfpathlineto{\pgfqpoint{3.292904in}{2.046004in}}%
\pgfpathlineto{\pgfqpoint{3.426129in}{2.117001in}}%
\pgfpathlineto{\pgfqpoint{3.438465in}{2.123754in}}%
\pgfpathlineto{\pgfqpoint{3.448333in}{2.129028in}}%
\pgfpathlineto{\pgfqpoint{3.463136in}{2.136240in}}%
\pgfpathlineto{\pgfqpoint{3.485341in}{2.148249in}}%
\pgfpathlineto{\pgfqpoint{3.492742in}{2.152477in}}%
\pgfpathlineto{\pgfqpoint{3.507545in}{2.160386in}}%
\pgfpathlineto{\pgfqpoint{3.532216in}{2.173360in}}%
\pgfpathlineto{\pgfqpoint{3.547019in}{2.182114in}}%
\pgfpathlineto{\pgfqpoint{3.556888in}{2.187443in}}%
\pgfpathlineto{\pgfqpoint{3.574158in}{2.195550in}}%
\pgfpathlineto{\pgfqpoint{3.586493in}{2.201180in}}%
\pgfpathlineto{\pgfqpoint{3.635836in}{2.229449in}}%
\pgfpathlineto{\pgfqpoint{3.645705in}{2.234829in}}%
\pgfpathlineto{\pgfqpoint{3.660508in}{2.241653in}}%
\pgfpathlineto{\pgfqpoint{3.680245in}{2.252371in}}%
\pgfpathlineto{\pgfqpoint{3.692581in}{2.258719in}}%
\pgfpathlineto{\pgfqpoint{3.759193in}{2.296026in}}%
\pgfpathlineto{\pgfqpoint{3.793733in}{2.313732in}}%
\pgfpathlineto{\pgfqpoint{3.803602in}{2.318528in}}%
\pgfpathlineto{\pgfqpoint{3.825806in}{2.330694in}}%
\pgfpathlineto{\pgfqpoint{3.835675in}{2.336548in}}%
\pgfpathlineto{\pgfqpoint{3.870215in}{2.356127in}}%
\pgfpathlineto{\pgfqpoint{3.880083in}{2.361049in}}%
\pgfpathlineto{\pgfqpoint{4.020711in}{2.437232in}}%
\pgfpathlineto{\pgfqpoint{4.037981in}{2.447928in}}%
\pgfpathlineto{\pgfqpoint{4.079922in}{2.471350in}}%
\pgfpathlineto{\pgfqpoint{4.087323in}{2.475559in}}%
\pgfpathlineto{\pgfqpoint{4.097192in}{2.480905in}}%
\pgfpathlineto{\pgfqpoint{4.104593in}{2.485486in}}%
\pgfpathlineto{\pgfqpoint{4.134199in}{2.500999in}}%
\pgfpathlineto{\pgfqpoint{4.158871in}{2.516222in}}%
\pgfpathlineto{\pgfqpoint{4.176141in}{2.524686in}}%
\pgfpathlineto{\pgfqpoint{4.183542in}{2.528593in}}%
\pgfpathlineto{\pgfqpoint{4.190943in}{2.532548in}}%
\pgfpathlineto{\pgfqpoint{4.200812in}{2.539713in}}%
\pgfpathlineto{\pgfqpoint{4.215615in}{2.548587in}}%
\pgfpathlineto{\pgfqpoint{4.289629in}{2.589164in}}%
\pgfpathlineto{\pgfqpoint{4.336505in}{2.612861in}}%
\pgfpathlineto{\pgfqpoint{4.341439in}{2.616189in}}%
\pgfpathlineto{\pgfqpoint{4.368578in}{2.631341in}}%
\pgfpathlineto{\pgfqpoint{4.385848in}{2.639698in}}%
\pgfpathlineto{\pgfqpoint{4.393249in}{2.644050in}}%
\pgfpathlineto{\pgfqpoint{4.432723in}{2.666296in}}%
\pgfpathlineto{\pgfqpoint{4.447526in}{2.677058in}}%
\pgfpathlineto{\pgfqpoint{4.454928in}{2.680999in}}%
\pgfpathlineto{\pgfqpoint{4.479599in}{2.694222in}}%
\pgfpathlineto{\pgfqpoint{4.496869in}{2.703357in}}%
\pgfpathlineto{\pgfqpoint{4.501803in}{2.707067in}}%
\pgfpathlineto{\pgfqpoint{4.511672in}{2.711778in}}%
\pgfpathlineto{\pgfqpoint{4.519073in}{2.716341in}}%
\pgfpathlineto{\pgfqpoint{4.541278in}{2.728695in}}%
\pgfpathlineto{\pgfqpoint{4.551146in}{2.734062in}}%
\pgfpathlineto{\pgfqpoint{4.578285in}{2.746016in}}%
\pgfpathlineto{\pgfqpoint{4.598022in}{2.756002in}}%
\pgfpathlineto{\pgfqpoint{4.607891in}{2.760992in}}%
\pgfpathlineto{\pgfqpoint{4.642431in}{2.779667in}}%
\pgfpathlineto{\pgfqpoint{4.659701in}{2.789728in}}%
\pgfpathlineto{\pgfqpoint{4.664635in}{2.793642in}}%
\pgfpathlineto{\pgfqpoint{4.674503in}{2.798328in}}%
\pgfpathlineto{\pgfqpoint{4.691773in}{2.807441in}}%
\pgfpathlineto{\pgfqpoint{4.704109in}{2.814758in}}%
\pgfpathlineto{\pgfqpoint{4.709043in}{2.818563in}}%
\pgfpathlineto{\pgfqpoint{4.718912in}{2.824251in}}%
\pgfpathlineto{\pgfqpoint{4.773189in}{2.850502in}}%
\pgfpathlineto{\pgfqpoint{4.783058in}{2.855105in}}%
\pgfpathlineto{\pgfqpoint{4.817598in}{2.871763in}}%
\pgfpathlineto{\pgfqpoint{4.827466in}{2.877882in}}%
\pgfpathlineto{\pgfqpoint{4.876809in}{2.902690in}}%
\pgfpathlineto{\pgfqpoint{4.879276in}{2.907179in}}%
\pgfpathlineto{\pgfqpoint{4.894079in}{2.915326in}}%
\pgfpathlineto{\pgfqpoint{4.903948in}{2.920032in}}%
\pgfpathlineto{\pgfqpoint{5.123523in}{3.033987in}}%
\pgfpathlineto{\pgfqpoint{5.133392in}{3.039231in}}%
\pgfpathlineto{\pgfqpoint{5.140793in}{3.043440in}}%
\pgfpathlineto{\pgfqpoint{5.153129in}{3.049963in}}%
\pgfpathlineto{\pgfqpoint{5.167932in}{3.056636in}}%
\pgfpathlineto{\pgfqpoint{5.244413in}{3.098707in}}%
\pgfpathlineto{\pgfqpoint{5.251815in}{3.103458in}}%
\pgfpathlineto{\pgfqpoint{5.318428in}{3.138362in}}%
\pgfpathlineto{\pgfqpoint{5.330763in}{3.143602in}}%
\pgfpathlineto{\pgfqpoint{5.419580in}{3.187911in}}%
\pgfpathlineto{\pgfqpoint{5.424515in}{3.190871in}}%
\pgfpathlineto{\pgfqpoint{5.439318in}{3.197398in}}%
\pgfpathlineto{\pgfqpoint{5.446719in}{3.201607in}}%
\pgfpathlineto{\pgfqpoint{5.463989in}{3.211911in}}%
\pgfpathlineto{\pgfqpoint{5.496062in}{3.229437in}}%
\pgfpathlineto{\pgfqpoint{5.513332in}{3.237069in}}%
\pgfpathlineto{\pgfqpoint{5.525668in}{3.244054in}}%
\pgfpathlineto{\pgfqpoint{5.540470in}{3.252366in}}%
\pgfpathlineto{\pgfqpoint{5.550339in}{3.257842in}}%
\pgfpathlineto{\pgfqpoint{5.560208in}{3.262348in}}%
\pgfpathlineto{\pgfqpoint{5.575010in}{3.269890in}}%
\pgfpathlineto{\pgfqpoint{5.592280in}{3.278069in}}%
\pgfpathlineto{\pgfqpoint{5.604616in}{3.285447in}}%
\pgfpathlineto{\pgfqpoint{5.616952in}{3.291255in}}%
\pgfpathlineto{\pgfqpoint{5.629288in}{3.298391in}}%
\pgfpathlineto{\pgfqpoint{5.646558in}{3.308648in}}%
\pgfpathlineto{\pgfqpoint{5.661360in}{3.315407in}}%
\pgfpathlineto{\pgfqpoint{5.668762in}{3.319708in}}%
\pgfpathlineto{\pgfqpoint{5.676163in}{3.324298in}}%
\pgfpathlineto{\pgfqpoint{5.705769in}{3.340286in}}%
\pgfpathlineto{\pgfqpoint{5.713170in}{3.344151in}}%
\pgfpathlineto{\pgfqpoint{5.720572in}{3.348366in}}%
\pgfpathlineto{\pgfqpoint{5.730440in}{3.353400in}}%
\pgfpathlineto{\pgfqpoint{5.737842in}{3.357199in}}%
\pgfpathlineto{\pgfqpoint{5.755112in}{3.365106in}}%
\pgfpathlineto{\pgfqpoint{5.794586in}{3.387273in}}%
\pgfpathlineto{\pgfqpoint{5.809389in}{3.394734in}}%
\pgfpathlineto{\pgfqpoint{6.051169in}{3.521472in}}%
\pgfpathlineto{\pgfqpoint{6.061038in}{3.528031in}}%
\pgfpathlineto{\pgfqpoint{6.098045in}{3.548421in}}%
\pgfpathlineto{\pgfqpoint{6.107913in}{3.554224in}}%
\pgfpathlineto{\pgfqpoint{6.132585in}{3.568337in}}%
\pgfpathlineto{\pgfqpoint{6.139986in}{3.573923in}}%
\pgfpathlineto{\pgfqpoint{6.144920in}{3.576990in}}%
\pgfpathlineto{\pgfqpoint{6.159723in}{3.585699in}}%
\pgfpathlineto{\pgfqpoint{6.176993in}{3.593207in}}%
\pgfpathlineto{\pgfqpoint{6.186862in}{3.599142in}}%
\pgfpathlineto{\pgfqpoint{6.189329in}{3.783516in}}%
\pgfpathlineto{\pgfqpoint{6.199198in}{3.790261in}}%
\pgfpathlineto{\pgfqpoint{6.233738in}{3.808923in}}%
\pgfpathlineto{\pgfqpoint{6.241139in}{3.812764in}}%
\pgfpathlineto{\pgfqpoint{6.258409in}{3.821282in}}%
\pgfpathlineto{\pgfqpoint{6.265810in}{3.824615in}}%
\pgfpathlineto{\pgfqpoint{6.292949in}{3.839571in}}%
\pgfpathlineto{\pgfqpoint{6.332423in}{3.860944in}}%
\pgfpathlineto{\pgfqpoint{6.342292in}{3.866544in}}%
\pgfpathlineto{\pgfqpoint{6.406438in}{3.898099in}}%
\pgfpathlineto{\pgfqpoint{6.413839in}{3.903265in}}%
\pgfpathlineto{\pgfqpoint{6.428642in}{3.911277in}}%
\pgfpathlineto{\pgfqpoint{6.445912in}{3.919877in}}%
\pgfpathlineto{\pgfqpoint{6.453313in}{3.925764in}}%
\pgfpathlineto{\pgfqpoint{6.465649in}{3.932164in}}%
\pgfpathlineto{\pgfqpoint{6.473050in}{3.937540in}}%
\pgfpathlineto{\pgfqpoint{6.487853in}{3.945088in}}%
\pgfpathlineto{\pgfqpoint{6.487853in}{3.945088in}}%
\pgfusepath{stroke}%
\end{pgfscope}%
\begin{pgfscope}%
\pgfpathrectangle{\pgfqpoint{1.309444in}{1.062361in}}{\pgfqpoint{5.425000in}{3.020000in}} %
\pgfusepath{clip}%
\pgfsetbuttcap%
\pgfsetroundjoin%
\pgfsetlinewidth{3.011250pt}%
\definecolor{currentstroke}{rgb}{0.333333,0.658824,0.407843}%
\pgfsetstrokecolor{currentstroke}%
\pgfsetdash{{6.000000pt}{3.000000pt}}{0.000000pt}%
\pgfpathmoveto{\pgfqpoint{1.556035in}{1.199644in}}%
\pgfpathlineto{\pgfqpoint{1.563437in}{1.216977in}}%
\pgfpathlineto{\pgfqpoint{1.565904in}{1.295355in}}%
\pgfpathlineto{\pgfqpoint{1.585641in}{1.349282in}}%
\pgfpathlineto{\pgfqpoint{1.588108in}{1.350373in}}%
\pgfpathlineto{\pgfqpoint{1.595509in}{1.363322in}}%
\pgfpathlineto{\pgfqpoint{1.597977in}{1.364582in}}%
\pgfpathlineto{\pgfqpoint{1.602911in}{1.380025in}}%
\pgfpathlineto{\pgfqpoint{1.612779in}{1.413373in}}%
\pgfpathlineto{\pgfqpoint{1.615246in}{1.413436in}}%
\pgfpathlineto{\pgfqpoint{1.620181in}{1.426947in}}%
\pgfpathlineto{\pgfqpoint{1.625115in}{1.444714in}}%
\pgfpathlineto{\pgfqpoint{1.627582in}{1.450450in}}%
\pgfpathlineto{\pgfqpoint{1.632516in}{1.489733in}}%
\pgfpathlineto{\pgfqpoint{1.639918in}{1.511089in}}%
\pgfpathlineto{\pgfqpoint{2.009989in}{1.520874in}}%
\pgfpathlineto{\pgfqpoint{2.486148in}{1.533314in}}%
\pgfpathlineto{\pgfqpoint{2.496016in}{1.538338in}}%
\pgfpathlineto{\pgfqpoint{2.498484in}{1.590951in}}%
\pgfpathlineto{\pgfqpoint{2.503418in}{1.593355in}}%
\pgfpathlineto{\pgfqpoint{2.508352in}{1.593656in}}%
\pgfpathlineto{\pgfqpoint{2.515754in}{1.601638in}}%
\pgfpathlineto{\pgfqpoint{2.528089in}{1.610520in}}%
\pgfpathlineto{\pgfqpoint{2.533024in}{1.611980in}}%
\pgfpathlineto{\pgfqpoint{2.542892in}{1.622800in}}%
\pgfpathlineto{\pgfqpoint{2.545359in}{1.625785in}}%
\pgfpathlineto{\pgfqpoint{2.552761in}{1.628530in}}%
\pgfpathlineto{\pgfqpoint{2.560162in}{1.635849in}}%
\pgfpathlineto{\pgfqpoint{2.565096in}{1.637103in}}%
\pgfpathlineto{\pgfqpoint{2.570031in}{1.640636in}}%
\pgfpathlineto{\pgfqpoint{2.577432in}{1.642134in}}%
\pgfpathlineto{\pgfqpoint{2.582366in}{1.644465in}}%
\pgfpathlineto{\pgfqpoint{2.584834in}{1.645297in}}%
\pgfpathlineto{\pgfqpoint{2.589768in}{1.649177in}}%
\pgfpathlineto{\pgfqpoint{2.592235in}{1.649635in}}%
\pgfpathlineto{\pgfqpoint{2.599636in}{1.655031in}}%
\pgfpathlineto{\pgfqpoint{2.604571in}{1.657401in}}%
\pgfpathlineto{\pgfqpoint{2.607038in}{1.660753in}}%
\pgfpathlineto{\pgfqpoint{2.609505in}{1.661432in}}%
\pgfpathlineto{\pgfqpoint{2.616906in}{1.669306in}}%
\pgfpathlineto{\pgfqpoint{2.648979in}{1.683648in}}%
\pgfpathlineto{\pgfqpoint{2.653914in}{1.689610in}}%
\pgfpathlineto{\pgfqpoint{2.668716in}{1.696659in}}%
\pgfpathlineto{\pgfqpoint{2.671184in}{1.699457in}}%
\pgfpathlineto{\pgfqpoint{2.673651in}{1.699795in}}%
\pgfpathlineto{\pgfqpoint{2.681052in}{1.704913in}}%
\pgfpathlineto{\pgfqpoint{2.688454in}{1.707725in}}%
\pgfpathlineto{\pgfqpoint{2.700789in}{1.713071in}}%
\pgfpathlineto{\pgfqpoint{2.720526in}{1.723040in}}%
\pgfpathlineto{\pgfqpoint{2.730395in}{1.728944in}}%
\pgfpathlineto{\pgfqpoint{2.752599in}{1.740097in}}%
\pgfpathlineto{\pgfqpoint{2.760001in}{1.743881in}}%
\pgfpathlineto{\pgfqpoint{2.814278in}{1.775430in}}%
\pgfpathlineto{\pgfqpoint{2.873489in}{1.804784in}}%
\pgfpathlineto{\pgfqpoint{2.895694in}{1.817883in}}%
\pgfpathlineto{\pgfqpoint{2.900628in}{1.821206in}}%
\pgfpathlineto{\pgfqpoint{2.912964in}{1.829048in}}%
\pgfpathlineto{\pgfqpoint{3.009182in}{1.879747in}}%
\pgfpathlineto{\pgfqpoint{3.016584in}{1.883351in}}%
\pgfpathlineto{\pgfqpoint{3.031386in}{1.890260in}}%
\pgfpathlineto{\pgfqpoint{3.041255in}{1.895379in}}%
\pgfpathlineto{\pgfqpoint{3.056058in}{1.902131in}}%
\pgfpathlineto{\pgfqpoint{3.095532in}{1.925618in}}%
\pgfpathlineto{\pgfqpoint{3.100466in}{1.928878in}}%
\pgfpathlineto{\pgfqpoint{3.110335in}{1.934638in}}%
\pgfpathlineto{\pgfqpoint{3.125138in}{1.944380in}}%
\pgfpathlineto{\pgfqpoint{3.142408in}{1.953311in}}%
\pgfpathlineto{\pgfqpoint{3.149809in}{1.958099in}}%
\pgfpathlineto{\pgfqpoint{3.172014in}{1.969929in}}%
\pgfpathlineto{\pgfqpoint{3.231225in}{1.999749in}}%
\pgfpathlineto{\pgfqpoint{3.253429in}{2.012193in}}%
\pgfpathlineto{\pgfqpoint{3.273166in}{2.023513in}}%
\pgfpathlineto{\pgfqpoint{3.283035in}{2.028968in}}%
\pgfpathlineto{\pgfqpoint{3.315108in}{2.044959in}}%
\pgfpathlineto{\pgfqpoint{3.329911in}{2.052562in}}%
\pgfpathlineto{\pgfqpoint{3.339779in}{2.057617in}}%
\pgfpathlineto{\pgfqpoint{3.349648in}{2.062805in}}%
\pgfpathlineto{\pgfqpoint{3.357049in}{2.066567in}}%
\pgfpathlineto{\pgfqpoint{3.369385in}{2.073564in}}%
\pgfpathlineto{\pgfqpoint{3.408859in}{2.092852in}}%
\pgfpathlineto{\pgfqpoint{3.421195in}{2.100519in}}%
\pgfpathlineto{\pgfqpoint{3.431063in}{2.106052in}}%
\pgfpathlineto{\pgfqpoint{3.450801in}{2.115819in}}%
\pgfpathlineto{\pgfqpoint{3.764128in}{2.281354in}}%
\pgfpathlineto{\pgfqpoint{3.771529in}{2.285611in}}%
\pgfpathlineto{\pgfqpoint{3.845543in}{2.324588in}}%
\pgfpathlineto{\pgfqpoint{3.852945in}{2.329617in}}%
\pgfpathlineto{\pgfqpoint{3.867748in}{2.336320in}}%
\pgfpathlineto{\pgfqpoint{3.909689in}{2.358177in}}%
\pgfpathlineto{\pgfqpoint{3.919558in}{2.363119in}}%
\pgfpathlineto{\pgfqpoint{3.924492in}{2.366411in}}%
\pgfpathlineto{\pgfqpoint{3.936828in}{2.372263in}}%
\pgfpathlineto{\pgfqpoint{3.951631in}{2.381519in}}%
\pgfpathlineto{\pgfqpoint{4.025645in}{2.420451in}}%
\pgfpathlineto{\pgfqpoint{4.042915in}{2.430751in}}%
\pgfpathlineto{\pgfqpoint{4.077455in}{2.448103in}}%
\pgfpathlineto{\pgfqpoint{4.094725in}{2.458325in}}%
\pgfpathlineto{\pgfqpoint{4.102126in}{2.462638in}}%
\pgfpathlineto{\pgfqpoint{4.134199in}{2.479179in}}%
\pgfpathlineto{\pgfqpoint{4.161338in}{2.495359in}}%
\pgfpathlineto{\pgfqpoint{4.183542in}{2.506661in}}%
\pgfpathlineto{\pgfqpoint{4.190943in}{2.510599in}}%
\pgfpathlineto{\pgfqpoint{4.200812in}{2.517800in}}%
\pgfpathlineto{\pgfqpoint{4.215615in}{2.526596in}}%
\pgfpathlineto{\pgfqpoint{4.257556in}{2.548697in}}%
\pgfpathlineto{\pgfqpoint{4.267425in}{2.553831in}}%
\pgfpathlineto{\pgfqpoint{4.289629in}{2.566432in}}%
\pgfpathlineto{\pgfqpoint{4.336505in}{2.590077in}}%
\pgfpathlineto{\pgfqpoint{4.341439in}{2.593372in}}%
\pgfpathlineto{\pgfqpoint{4.368578in}{2.608545in}}%
\pgfpathlineto{\pgfqpoint{4.385848in}{2.616922in}}%
\pgfpathlineto{\pgfqpoint{4.393249in}{2.621297in}}%
\pgfpathlineto{\pgfqpoint{4.432723in}{2.643658in}}%
\pgfpathlineto{\pgfqpoint{4.442592in}{2.649948in}}%
\pgfpathlineto{\pgfqpoint{4.454928in}{2.657718in}}%
\pgfpathlineto{\pgfqpoint{4.479599in}{2.670964in}}%
\pgfpathlineto{\pgfqpoint{4.496869in}{2.680098in}}%
\pgfpathlineto{\pgfqpoint{4.501803in}{2.683778in}}%
\pgfpathlineto{\pgfqpoint{4.511672in}{2.688491in}}%
\pgfpathlineto{\pgfqpoint{4.519073in}{2.693053in}}%
\pgfpathlineto{\pgfqpoint{4.541278in}{2.705367in}}%
\pgfpathlineto{\pgfqpoint{4.551146in}{2.710781in}}%
\pgfpathlineto{\pgfqpoint{4.578285in}{2.722774in}}%
\pgfpathlineto{\pgfqpoint{4.598022in}{2.732741in}}%
\pgfpathlineto{\pgfqpoint{4.607891in}{2.737787in}}%
\pgfpathlineto{\pgfqpoint{4.642431in}{2.756458in}}%
\pgfpathlineto{\pgfqpoint{4.662168in}{2.768358in}}%
\pgfpathlineto{\pgfqpoint{4.686839in}{2.781513in}}%
\pgfpathlineto{\pgfqpoint{4.694241in}{2.784917in}}%
\pgfpathlineto{\pgfqpoint{4.699175in}{2.788821in}}%
\pgfpathlineto{\pgfqpoint{4.704109in}{2.790985in}}%
\pgfpathlineto{\pgfqpoint{4.709043in}{2.794761in}}%
\pgfpathlineto{\pgfqpoint{4.718912in}{2.800402in}}%
\pgfpathlineto{\pgfqpoint{4.765788in}{2.823109in}}%
\pgfpathlineto{\pgfqpoint{4.805262in}{2.841881in}}%
\pgfpathlineto{\pgfqpoint{4.829933in}{2.855520in}}%
\pgfpathlineto{\pgfqpoint{4.844736in}{2.861824in}}%
\pgfpathlineto{\pgfqpoint{4.854605in}{2.867682in}}%
\pgfpathlineto{\pgfqpoint{4.866941in}{2.873494in}}%
\pgfpathlineto{\pgfqpoint{4.908882in}{2.898650in}}%
\pgfpathlineto{\pgfqpoint{4.943422in}{2.916163in}}%
\pgfpathlineto{\pgfqpoint{5.195071in}{3.047091in}}%
\pgfpathlineto{\pgfqpoint{5.313493in}{3.111577in}}%
\pgfpathlineto{\pgfqpoint{5.330763in}{3.119038in}}%
\pgfpathlineto{\pgfqpoint{5.380106in}{3.141868in}}%
\pgfpathlineto{\pgfqpoint{5.394909in}{3.149665in}}%
\pgfpathlineto{\pgfqpoint{5.424515in}{3.165832in}}%
\pgfpathlineto{\pgfqpoint{5.439318in}{3.172397in}}%
\pgfpathlineto{\pgfqpoint{5.446719in}{3.176607in}}%
\pgfpathlineto{\pgfqpoint{5.461522in}{3.185519in}}%
\pgfpathlineto{\pgfqpoint{5.510865in}{3.211086in}}%
\pgfpathlineto{\pgfqpoint{5.520733in}{3.216349in}}%
\pgfpathlineto{\pgfqpoint{5.686032in}{3.303973in}}%
\pgfpathlineto{\pgfqpoint{5.698368in}{3.311336in}}%
\pgfpathlineto{\pgfqpoint{5.710703in}{3.316341in}}%
\pgfpathlineto{\pgfqpoint{5.720572in}{3.322471in}}%
\pgfpathlineto{\pgfqpoint{5.730440in}{3.327503in}}%
\pgfpathlineto{\pgfqpoint{5.737842in}{3.331287in}}%
\pgfpathlineto{\pgfqpoint{5.755112in}{3.339219in}}%
\pgfpathlineto{\pgfqpoint{5.794586in}{3.361436in}}%
\pgfpathlineto{\pgfqpoint{5.809389in}{3.368909in}}%
\pgfpathlineto{\pgfqpoint{6.065972in}{3.504028in}}%
\pgfpathlineto{\pgfqpoint{6.083242in}{3.512890in}}%
\pgfpathlineto{\pgfqpoint{6.098045in}{3.520863in}}%
\pgfpathlineto{\pgfqpoint{6.107913in}{3.526588in}}%
\pgfpathlineto{\pgfqpoint{6.132585in}{3.540004in}}%
\pgfpathlineto{\pgfqpoint{6.139986in}{3.545516in}}%
\pgfpathlineto{\pgfqpoint{6.144920in}{3.548547in}}%
\pgfpathlineto{\pgfqpoint{6.162190in}{3.558400in}}%
\pgfpathlineto{\pgfqpoint{6.186862in}{3.570945in}}%
\pgfpathlineto{\pgfqpoint{6.189329in}{3.755179in}}%
\pgfpathlineto{\pgfqpoint{6.199198in}{3.761857in}}%
\pgfpathlineto{\pgfqpoint{6.233738in}{3.780475in}}%
\pgfpathlineto{\pgfqpoint{6.241139in}{3.784293in}}%
\pgfpathlineto{\pgfqpoint{6.258409in}{3.792797in}}%
\pgfpathlineto{\pgfqpoint{6.265810in}{3.796122in}}%
\pgfpathlineto{\pgfqpoint{6.292949in}{3.811079in}}%
\pgfpathlineto{\pgfqpoint{6.332423in}{3.832410in}}%
\pgfpathlineto{\pgfqpoint{6.342292in}{3.838001in}}%
\pgfpathlineto{\pgfqpoint{6.408905in}{3.871159in}}%
\pgfpathlineto{\pgfqpoint{6.421240in}{3.878226in}}%
\pgfpathlineto{\pgfqpoint{6.445912in}{3.890519in}}%
\pgfpathlineto{\pgfqpoint{6.453313in}{3.896364in}}%
\pgfpathlineto{\pgfqpoint{6.465649in}{3.902738in}}%
\pgfpathlineto{\pgfqpoint{6.473050in}{3.908057in}}%
\pgfpathlineto{\pgfqpoint{6.487853in}{3.915601in}}%
\pgfpathlineto{\pgfqpoint{6.487853in}{3.915601in}}%
\pgfusepath{stroke}%
\end{pgfscope}%
\begin{pgfscope}%
\pgfsetrectcap%
\pgfsetmiterjoin%
\pgfsetlinewidth{1.003750pt}%
\definecolor{currentstroke}{rgb}{0.800000,0.800000,0.800000}%
\pgfsetstrokecolor{currentstroke}%
\pgfsetdash{}{0pt}%
\pgfpathmoveto{\pgfqpoint{1.309444in}{1.062361in}}%
\pgfpathlineto{\pgfqpoint{1.309444in}{4.082361in}}%
\pgfusepath{stroke}%
\end{pgfscope}%
\begin{pgfscope}%
\pgfsetrectcap%
\pgfsetmiterjoin%
\pgfsetlinewidth{1.003750pt}%
\definecolor{currentstroke}{rgb}{0.800000,0.800000,0.800000}%
\pgfsetstrokecolor{currentstroke}%
\pgfsetdash{}{0pt}%
\pgfpathmoveto{\pgfqpoint{6.734444in}{1.062361in}}%
\pgfpathlineto{\pgfqpoint{6.734444in}{4.082361in}}%
\pgfusepath{stroke}%
\end{pgfscope}%
\begin{pgfscope}%
\pgfsetrectcap%
\pgfsetmiterjoin%
\pgfsetlinewidth{1.003750pt}%
\definecolor{currentstroke}{rgb}{0.800000,0.800000,0.800000}%
\pgfsetstrokecolor{currentstroke}%
\pgfsetdash{}{0pt}%
\pgfpathmoveto{\pgfqpoint{1.309444in}{1.062361in}}%
\pgfpathlineto{\pgfqpoint{6.734444in}{1.062361in}}%
\pgfusepath{stroke}%
\end{pgfscope}%
\begin{pgfscope}%
\pgfsetrectcap%
\pgfsetmiterjoin%
\pgfsetlinewidth{1.003750pt}%
\definecolor{currentstroke}{rgb}{0.800000,0.800000,0.800000}%
\pgfsetstrokecolor{currentstroke}%
\pgfsetdash{}{0pt}%
\pgfpathmoveto{\pgfqpoint{1.309444in}{4.082361in}}%
\pgfpathlineto{\pgfqpoint{6.734444in}{4.082361in}}%
\pgfusepath{stroke}%
\end{pgfscope}%
\begin{pgfscope}%
\pgfsetbuttcap%
\pgfsetroundjoin%
\pgfsetlinewidth{4.015000pt}%
\definecolor{currentstroke}{rgb}{0.298039,0.447059,0.690196}%
\pgfsetstrokecolor{currentstroke}%
\pgfsetdash{{4.000000pt}{0.000000pt}}{0.000000pt}%
\pgfpathmoveto{\pgfqpoint{1.292305in}{4.281805in}}%
\pgfpathlineto{\pgfqpoint{1.958972in}{4.281805in}}%
\pgfusepath{stroke}%
\end{pgfscope}%
\begin{pgfscope}%
\definecolor{textcolor}{rgb}{0.150000,0.150000,0.150000}%
\pgfsetstrokecolor{textcolor}%
\pgfsetfillcolor{textcolor}%
\pgftext[x=1.981194in,y=4.126250in,left,base]{\color{textcolor}\rmfamily\fontsize{32.000000}{38.400000}\selectfont CO+W}%
\end{pgfscope}%
\begin{pgfscope}%
\pgfsetbuttcap%
\pgfsetroundjoin%
\pgfsetlinewidth{4.015000pt}%
\definecolor{currentstroke}{rgb}{0.866667,0.517647,0.321569}%
\pgfsetstrokecolor{currentstroke}%
\pgfsetdash{{4.000000pt}{4.000000pt}}{0.000000pt}%
\pgfpathmoveto{\pgfqpoint{3.564750in}{4.281805in}}%
\pgfpathlineto{\pgfqpoint{4.231416in}{4.281805in}}%
\pgfusepath{stroke}%
\end{pgfscope}%
\begin{pgfscope}%
\definecolor{textcolor}{rgb}{0.150000,0.150000,0.150000}%
\pgfsetstrokecolor{textcolor}%
\pgfsetfillcolor{textcolor}%
\pgftext[x=4.253639in,y=4.126250in,left,base]{\color{textcolor}\rmfamily\fontsize{32.000000}{38.400000}\selectfont OML}%
\end{pgfscope}%
\begin{pgfscope}%
\pgfsetbuttcap%
\pgfsetroundjoin%
\pgfsetlinewidth{4.015000pt}%
\definecolor{currentstroke}{rgb}{0.333333,0.658824,0.407843}%
\pgfsetstrokecolor{currentstroke}%
\pgfsetdash{{8.000000pt}{4.000000pt}}{0.000000pt}%
\pgfpathmoveto{\pgfqpoint{5.429639in}{4.281805in}}%
\pgfpathlineto{\pgfqpoint{6.096305in}{4.281805in}}%
\pgfusepath{stroke}%
\end{pgfscope}%
\begin{pgfscope}%
\definecolor{textcolor}{rgb}{0.150000,0.150000,0.150000}%
\pgfsetstrokecolor{textcolor}%
\pgfsetfillcolor{textcolor}%
\pgftext[x=6.118528in,y=4.126250in,left,base]{\color{textcolor}\rmfamily\fontsize{32.000000}{38.400000}\selectfont CO-W}%
\end{pgfscope}%
\end{pgfpicture}%
\makeatother%
\endgroup%
%
}
\caption{Run Time}
\end{subfigure}%
\begin{subfigure}[b]{0.5\linewidth}
\centering
 \resizebox{\columnwidth}{!}{%
%% Creator: Matplotlib, PGF backend
%%
%% To include the figure in your LaTeX document, write
%%   \input{<filename>.pgf}
%%
%% Make sure the required packages are loaded in your preamble
%%   \usepackage{pgf}
%%
%% Figures using additional raster images can only be included by \input if
%% they are in the same directory as the main LaTeX file. For loading figures
%% from other directories you can use the `import` package
%%   \usepackage{import}
%% and then include the figures with
%%   \import{<path to file>}{<filename>.pgf}
%%
%% Matplotlib used the following preamble
%%   \usepackage{fontspec}
%%   \setmonofont{Andale Mono}
%%
\begingroup%
\makeatletter%
\begin{pgfpicture}%
\pgfpathrectangle{\pgfpointorigin}{\pgfqpoint{7.835611in}{4.715139in}}%
\pgfusepath{use as bounding box, clip}%
\begin{pgfscope}%
\pgfsetbuttcap%
\pgfsetmiterjoin%
\definecolor{currentfill}{rgb}{1.000000,1.000000,1.000000}%
\pgfsetfillcolor{currentfill}%
\pgfsetlinewidth{0.000000pt}%
\definecolor{currentstroke}{rgb}{1.000000,1.000000,1.000000}%
\pgfsetstrokecolor{currentstroke}%
\pgfsetdash{}{0pt}%
\pgfpathmoveto{\pgfqpoint{0.000000in}{0.000000in}}%
\pgfpathlineto{\pgfqpoint{7.835611in}{0.000000in}}%
\pgfpathlineto{\pgfqpoint{7.835611in}{4.715139in}}%
\pgfpathlineto{\pgfqpoint{0.000000in}{4.715139in}}%
\pgfpathclose%
\pgfusepath{fill}%
\end{pgfscope}%
\begin{pgfscope}%
\pgfsetbuttcap%
\pgfsetmiterjoin%
\definecolor{currentfill}{rgb}{1.000000,1.000000,1.000000}%
\pgfsetfillcolor{currentfill}%
\pgfsetlinewidth{0.000000pt}%
\definecolor{currentstroke}{rgb}{0.000000,0.000000,0.000000}%
\pgfsetstrokecolor{currentstroke}%
\pgfsetstrokeopacity{0.000000}%
\pgfsetdash{}{0pt}%
\pgfpathmoveto{\pgfqpoint{1.573194in}{1.062361in}}%
\pgfpathlineto{\pgfqpoint{6.998194in}{1.062361in}}%
\pgfpathlineto{\pgfqpoint{6.998194in}{4.082361in}}%
\pgfpathlineto{\pgfqpoint{1.573194in}{4.082361in}}%
\pgfpathclose%
\pgfusepath{fill}%
\end{pgfscope}%
\begin{pgfscope}%
\pgfpathrectangle{\pgfqpoint{1.573194in}{1.062361in}}{\pgfqpoint{5.425000in}{3.020000in}} %
\pgfusepath{clip}%
\pgfsetroundcap%
\pgfsetroundjoin%
\pgfsetlinewidth{0.803000pt}%
\definecolor{currentstroke}{rgb}{0.800000,0.800000,0.800000}%
\pgfsetstrokecolor{currentstroke}%
\pgfsetdash{}{0pt}%
\pgfpathmoveto{\pgfqpoint{1.817318in}{1.062361in}}%
\pgfpathlineto{\pgfqpoint{1.817318in}{4.082361in}}%
\pgfusepath{stroke}%
\end{pgfscope}%
\begin{pgfscope}%
\definecolor{textcolor}{rgb}{0.150000,0.150000,0.150000}%
\pgfsetstrokecolor{textcolor}%
\pgfsetfillcolor{textcolor}%
\pgftext[x=1.817318in,y=0.898472in,,top]{\color{textcolor}\rmfamily\fontsize{30.000000}{36.000000}\selectfont 0}%
\end{pgfscope}%
\begin{pgfscope}%
\pgfpathrectangle{\pgfqpoint{1.573194in}{1.062361in}}{\pgfqpoint{5.425000in}{3.020000in}} %
\pgfusepath{clip}%
\pgfsetroundcap%
\pgfsetroundjoin%
\pgfsetlinewidth{0.803000pt}%
\definecolor{currentstroke}{rgb}{0.800000,0.800000,0.800000}%
\pgfsetstrokecolor{currentstroke}%
\pgfsetdash{}{0pt}%
\pgfpathmoveto{\pgfqpoint{4.284461in}{1.062361in}}%
\pgfpathlineto{\pgfqpoint{4.284461in}{4.082361in}}%
\pgfusepath{stroke}%
\end{pgfscope}%
\begin{pgfscope}%
\definecolor{textcolor}{rgb}{0.150000,0.150000,0.150000}%
\pgfsetstrokecolor{textcolor}%
\pgfsetfillcolor{textcolor}%
\pgftext[x=4.284461in,y=0.898472in,,top]{\color{textcolor}\rmfamily\fontsize{30.000000}{36.000000}\selectfont 1000}%
\end{pgfscope}%
\begin{pgfscope}%
\pgfpathrectangle{\pgfqpoint{1.573194in}{1.062361in}}{\pgfqpoint{5.425000in}{3.020000in}} %
\pgfusepath{clip}%
\pgfsetroundcap%
\pgfsetroundjoin%
\pgfsetlinewidth{0.803000pt}%
\definecolor{currentstroke}{rgb}{0.800000,0.800000,0.800000}%
\pgfsetstrokecolor{currentstroke}%
\pgfsetdash{}{0pt}%
\pgfpathmoveto{\pgfqpoint{6.751603in}{1.062361in}}%
\pgfpathlineto{\pgfqpoint{6.751603in}{4.082361in}}%
\pgfusepath{stroke}%
\end{pgfscope}%
\begin{pgfscope}%
\definecolor{textcolor}{rgb}{0.150000,0.150000,0.150000}%
\pgfsetstrokecolor{textcolor}%
\pgfsetfillcolor{textcolor}%
\pgftext[x=6.751603in,y=0.898472in,,top]{\color{textcolor}\rmfamily\fontsize{30.000000}{36.000000}\selectfont 2000}%
\end{pgfscope}%
\begin{pgfscope}%
\definecolor{textcolor}{rgb}{0.150000,0.150000,0.150000}%
\pgfsetstrokecolor{textcolor}%
\pgfsetfillcolor{textcolor}%
\pgftext[x=4.285694in,y=0.472500in,,top]{\color{textcolor}\rmfamily\fontsize{30.000000}{36.000000}\selectfont OpenML Workload}%
\end{pgfscope}%
\begin{pgfscope}%
\pgfpathrectangle{\pgfqpoint{1.573194in}{1.062361in}}{\pgfqpoint{5.425000in}{3.020000in}} %
\pgfusepath{clip}%
\pgfsetroundcap%
\pgfsetroundjoin%
\pgfsetlinewidth{0.803000pt}%
\definecolor{currentstroke}{rgb}{0.800000,0.800000,0.800000}%
\pgfsetstrokecolor{currentstroke}%
\pgfsetdash{}{0pt}%
\pgfpathmoveto{\pgfqpoint{1.573194in}{1.197250in}}%
\pgfpathlineto{\pgfqpoint{6.998194in}{1.197250in}}%
\pgfusepath{stroke}%
\end{pgfscope}%
\begin{pgfscope}%
\definecolor{textcolor}{rgb}{0.150000,0.150000,0.150000}%
\pgfsetstrokecolor{textcolor}%
\pgfsetfillcolor{textcolor}%
\pgftext[x=1.218055in,y=1.052667in,left,base]{\color{textcolor}\rmfamily\fontsize{30.000000}{36.000000}\selectfont 0}%
\end{pgfscope}%
\begin{pgfscope}%
\pgfpathrectangle{\pgfqpoint{1.573194in}{1.062361in}}{\pgfqpoint{5.425000in}{3.020000in}} %
\pgfusepath{clip}%
\pgfsetroundcap%
\pgfsetroundjoin%
\pgfsetlinewidth{0.803000pt}%
\definecolor{currentstroke}{rgb}{0.800000,0.800000,0.800000}%
\pgfsetstrokecolor{currentstroke}%
\pgfsetdash{}{0pt}%
\pgfpathmoveto{\pgfqpoint{1.573194in}{1.774272in}}%
\pgfpathlineto{\pgfqpoint{6.998194in}{1.774272in}}%
\pgfusepath{stroke}%
\end{pgfscope}%
\begin{pgfscope}%
\definecolor{textcolor}{rgb}{0.150000,0.150000,0.150000}%
\pgfsetstrokecolor{textcolor}%
\pgfsetfillcolor{textcolor}%
\pgftext[x=0.529305in,y=1.629689in,left,base]{\color{textcolor}\rmfamily\fontsize{30.000000}{36.000000}\selectfont 12.5k}%
\end{pgfscope}%
\begin{pgfscope}%
\pgfpathrectangle{\pgfqpoint{1.573194in}{1.062361in}}{\pgfqpoint{5.425000in}{3.020000in}} %
\pgfusepath{clip}%
\pgfsetroundcap%
\pgfsetroundjoin%
\pgfsetlinewidth{0.803000pt}%
\definecolor{currentstroke}{rgb}{0.800000,0.800000,0.800000}%
\pgfsetstrokecolor{currentstroke}%
\pgfsetdash{}{0pt}%
\pgfpathmoveto{\pgfqpoint{1.573194in}{2.351294in}}%
\pgfpathlineto{\pgfqpoint{6.998194in}{2.351294in}}%
\pgfusepath{stroke}%
\end{pgfscope}%
\begin{pgfscope}%
\definecolor{textcolor}{rgb}{0.150000,0.150000,0.150000}%
\pgfsetstrokecolor{textcolor}%
\pgfsetfillcolor{textcolor}%
\pgftext[x=0.824722in,y=2.206711in,left,base]{\color{textcolor}\rmfamily\fontsize{30.000000}{36.000000}\selectfont 25k}%
\end{pgfscope}%
\begin{pgfscope}%
\pgfpathrectangle{\pgfqpoint{1.573194in}{1.062361in}}{\pgfqpoint{5.425000in}{3.020000in}} %
\pgfusepath{clip}%
\pgfsetroundcap%
\pgfsetroundjoin%
\pgfsetlinewidth{0.803000pt}%
\definecolor{currentstroke}{rgb}{0.800000,0.800000,0.800000}%
\pgfsetstrokecolor{currentstroke}%
\pgfsetdash{}{0pt}%
\pgfpathmoveto{\pgfqpoint{1.573194in}{2.928317in}}%
\pgfpathlineto{\pgfqpoint{6.998194in}{2.928317in}}%
\pgfusepath{stroke}%
\end{pgfscope}%
\begin{pgfscope}%
\definecolor{textcolor}{rgb}{0.150000,0.150000,0.150000}%
\pgfsetstrokecolor{textcolor}%
\pgfsetfillcolor{textcolor}%
\pgftext[x=0.529305in,y=2.783733in,left,base]{\color{textcolor}\rmfamily\fontsize{30.000000}{36.000000}\selectfont 37.5k}%
\end{pgfscope}%
\begin{pgfscope}%
\pgfpathrectangle{\pgfqpoint{1.573194in}{1.062361in}}{\pgfqpoint{5.425000in}{3.020000in}} %
\pgfusepath{clip}%
\pgfsetroundcap%
\pgfsetroundjoin%
\pgfsetlinewidth{0.803000pt}%
\definecolor{currentstroke}{rgb}{0.800000,0.800000,0.800000}%
\pgfsetstrokecolor{currentstroke}%
\pgfsetdash{}{0pt}%
\pgfpathmoveto{\pgfqpoint{1.573194in}{3.505339in}}%
\pgfpathlineto{\pgfqpoint{6.998194in}{3.505339in}}%
\pgfusepath{stroke}%
\end{pgfscope}%
\begin{pgfscope}%
\definecolor{textcolor}{rgb}{0.150000,0.150000,0.150000}%
\pgfsetstrokecolor{textcolor}%
\pgfsetfillcolor{textcolor}%
\pgftext[x=0.824722in,y=3.360755in,left,base]{\color{textcolor}\rmfamily\fontsize{30.000000}{36.000000}\selectfont 50k}%
\end{pgfscope}%
\begin{pgfscope}%
\pgfpathrectangle{\pgfqpoint{1.573194in}{1.062361in}}{\pgfqpoint{5.425000in}{3.020000in}} %
\pgfusepath{clip}%
\pgfsetroundcap%
\pgfsetroundjoin%
\pgfsetlinewidth{0.803000pt}%
\definecolor{currentstroke}{rgb}{0.800000,0.800000,0.800000}%
\pgfsetstrokecolor{currentstroke}%
\pgfsetdash{}{0pt}%
\pgfpathmoveto{\pgfqpoint{1.573194in}{4.082361in}}%
\pgfpathlineto{\pgfqpoint{6.998194in}{4.082361in}}%
\pgfusepath{stroke}%
\end{pgfscope}%
\begin{pgfscope}%
\definecolor{textcolor}{rgb}{0.150000,0.150000,0.150000}%
\pgfsetstrokecolor{textcolor}%
\pgfsetfillcolor{textcolor}%
\pgftext[x=0.529305in,y=3.937778in,left,base]{\color{textcolor}\rmfamily\fontsize{30.000000}{36.000000}\selectfont 62.5k}%
\end{pgfscope}%
\begin{pgfscope}%
\definecolor{textcolor}{rgb}{0.150000,0.150000,0.150000}%
\pgfsetstrokecolor{textcolor}%
\pgfsetfillcolor{textcolor}%
\pgftext[x=0.473750in,y=2.572361in,,bottom,rotate=90.000000]{\color{textcolor}\rmfamily\fontsize{30.000000}{36.000000}\selectfont Num of Operations}%
\end{pgfscope}%
\begin{pgfscope}%
\pgfpathrectangle{\pgfqpoint{1.573194in}{1.062361in}}{\pgfqpoint{5.425000in}{3.020000in}} %
\pgfusepath{clip}%
\pgfsetbuttcap%
\pgfsetroundjoin%
\pgfsetlinewidth{3.011250pt}%
\definecolor{currentstroke}{rgb}{0.298039,0.447059,0.690196}%
\pgfsetstrokecolor{currentstroke}%
\pgfsetdash{{3.000000pt}{0.000000pt}}{0.000000pt}%
\pgfpathmoveto{\pgfqpoint{1.819785in}{1.198081in}}%
\pgfpathlineto{\pgfqpoint{1.829654in}{1.201451in}}%
\pgfpathlineto{\pgfqpoint{1.911069in}{1.213361in}}%
\pgfpathlineto{\pgfqpoint{1.920938in}{1.215069in}}%
\pgfpathlineto{\pgfqpoint{2.749898in}{1.324703in}}%
\pgfpathlineto{\pgfqpoint{2.759767in}{1.329180in}}%
\pgfpathlineto{\pgfqpoint{2.772102in}{1.331858in}}%
\pgfpathlineto{\pgfqpoint{2.777037in}{1.333566in}}%
\pgfpathlineto{\pgfqpoint{6.751603in}{1.855194in}}%
\pgfpathlineto{\pgfqpoint{6.751603in}{1.855194in}}%
\pgfusepath{stroke}%
\end{pgfscope}%
\begin{pgfscope}%
\pgfpathrectangle{\pgfqpoint{1.573194in}{1.062361in}}{\pgfqpoint{5.425000in}{3.020000in}} %
\pgfusepath{clip}%
\pgfsetbuttcap%
\pgfsetroundjoin%
\pgfsetlinewidth{3.011250pt}%
\definecolor{currentstroke}{rgb}{0.866667,0.517647,0.321569}%
\pgfsetstrokecolor{currentstroke}%
\pgfsetdash{{3.000000pt}{3.000000pt}}{0.000000pt}%
\pgfpathmoveto{\pgfqpoint{1.819785in}{1.198081in}}%
\pgfpathlineto{\pgfqpoint{1.864194in}{1.217238in}}%
\pgfpathlineto{\pgfqpoint{2.752365in}{1.517982in}}%
\pgfpathlineto{\pgfqpoint{6.751603in}{3.912486in}}%
\pgfpathlineto{\pgfqpoint{6.751603in}{3.912486in}}%
\pgfusepath{stroke}%
\end{pgfscope}%
\begin{pgfscope}%
\pgfpathrectangle{\pgfqpoint{1.573194in}{1.062361in}}{\pgfqpoint{5.425000in}{3.020000in}} %
\pgfusepath{clip}%
\pgfsetbuttcap%
\pgfsetroundjoin%
\pgfsetlinewidth{3.011250pt}%
\definecolor{currentstroke}{rgb}{0.333333,0.658824,0.407843}%
\pgfsetstrokecolor{currentstroke}%
\pgfsetdash{{6.000000pt}{3.000000pt}}{0.000000pt}%
\pgfpathmoveto{\pgfqpoint{1.819785in}{1.198081in}}%
\pgfpathlineto{\pgfqpoint{1.829654in}{1.201174in}}%
\pgfpathlineto{\pgfqpoint{1.911069in}{1.213084in}}%
\pgfpathlineto{\pgfqpoint{1.920938in}{1.214792in}}%
\pgfpathlineto{\pgfqpoint{2.749898in}{1.324426in}}%
\pgfpathlineto{\pgfqpoint{2.759767in}{1.328904in}}%
\pgfpathlineto{\pgfqpoint{2.772102in}{1.331581in}}%
\pgfpathlineto{\pgfqpoint{2.777037in}{1.333289in}}%
\pgfpathlineto{\pgfqpoint{6.751603in}{1.854917in}}%
\pgfpathlineto{\pgfqpoint{6.751603in}{1.854917in}}%
\pgfusepath{stroke}%
\end{pgfscope}%
\begin{pgfscope}%
\pgfsetrectcap%
\pgfsetmiterjoin%
\pgfsetlinewidth{1.003750pt}%
\definecolor{currentstroke}{rgb}{0.800000,0.800000,0.800000}%
\pgfsetstrokecolor{currentstroke}%
\pgfsetdash{}{0pt}%
\pgfpathmoveto{\pgfqpoint{1.573194in}{1.062361in}}%
\pgfpathlineto{\pgfqpoint{1.573194in}{4.082361in}}%
\pgfusepath{stroke}%
\end{pgfscope}%
\begin{pgfscope}%
\pgfsetrectcap%
\pgfsetmiterjoin%
\pgfsetlinewidth{1.003750pt}%
\definecolor{currentstroke}{rgb}{0.800000,0.800000,0.800000}%
\pgfsetstrokecolor{currentstroke}%
\pgfsetdash{}{0pt}%
\pgfpathmoveto{\pgfqpoint{6.998194in}{1.062361in}}%
\pgfpathlineto{\pgfqpoint{6.998194in}{4.082361in}}%
\pgfusepath{stroke}%
\end{pgfscope}%
\begin{pgfscope}%
\pgfsetrectcap%
\pgfsetmiterjoin%
\pgfsetlinewidth{1.003750pt}%
\definecolor{currentstroke}{rgb}{0.800000,0.800000,0.800000}%
\pgfsetstrokecolor{currentstroke}%
\pgfsetdash{}{0pt}%
\pgfpathmoveto{\pgfqpoint{1.573194in}{1.062361in}}%
\pgfpathlineto{\pgfqpoint{6.998194in}{1.062361in}}%
\pgfusepath{stroke}%
\end{pgfscope}%
\begin{pgfscope}%
\pgfsetrectcap%
\pgfsetmiterjoin%
\pgfsetlinewidth{1.003750pt}%
\definecolor{currentstroke}{rgb}{0.800000,0.800000,0.800000}%
\pgfsetstrokecolor{currentstroke}%
\pgfsetdash{}{0pt}%
\pgfpathmoveto{\pgfqpoint{1.573194in}{4.082361in}}%
\pgfpathlineto{\pgfqpoint{6.998194in}{4.082361in}}%
\pgfusepath{stroke}%
\end{pgfscope}%
\begin{pgfscope}%
\pgfsetbuttcap%
\pgfsetroundjoin%
\pgfsetlinewidth{4.015000pt}%
\definecolor{currentstroke}{rgb}{0.298039,0.447059,0.690196}%
\pgfsetstrokecolor{currentstroke}%
\pgfsetdash{{4.000000pt}{0.000000pt}}{0.000000pt}%
\pgfpathmoveto{\pgfqpoint{1.556055in}{4.281805in}}%
\pgfpathlineto{\pgfqpoint{2.222722in}{4.281805in}}%
\pgfusepath{stroke}%
\end{pgfscope}%
\begin{pgfscope}%
\definecolor{textcolor}{rgb}{0.150000,0.150000,0.150000}%
\pgfsetstrokecolor{textcolor}%
\pgfsetfillcolor{textcolor}%
\pgftext[x=2.244944in,y=4.126250in,left,base]{\color{textcolor}\rmfamily\fontsize{32.000000}{38.400000}\selectfont CO+W}%
\end{pgfscope}%
\begin{pgfscope}%
\pgfsetbuttcap%
\pgfsetroundjoin%
\pgfsetlinewidth{4.015000pt}%
\definecolor{currentstroke}{rgb}{0.866667,0.517647,0.321569}%
\pgfsetstrokecolor{currentstroke}%
\pgfsetdash{{4.000000pt}{4.000000pt}}{0.000000pt}%
\pgfpathmoveto{\pgfqpoint{3.828500in}{4.281805in}}%
\pgfpathlineto{\pgfqpoint{4.495167in}{4.281805in}}%
\pgfusepath{stroke}%
\end{pgfscope}%
\begin{pgfscope}%
\definecolor{textcolor}{rgb}{0.150000,0.150000,0.150000}%
\pgfsetstrokecolor{textcolor}%
\pgfsetfillcolor{textcolor}%
\pgftext[x=4.517389in,y=4.126250in,left,base]{\color{textcolor}\rmfamily\fontsize{32.000000}{38.400000}\selectfont OML}%
\end{pgfscope}%
\begin{pgfscope}%
\pgfsetbuttcap%
\pgfsetroundjoin%
\pgfsetlinewidth{4.015000pt}%
\definecolor{currentstroke}{rgb}{0.333333,0.658824,0.407843}%
\pgfsetstrokecolor{currentstroke}%
\pgfsetdash{{8.000000pt}{4.000000pt}}{0.000000pt}%
\pgfpathmoveto{\pgfqpoint{5.693389in}{4.281805in}}%
\pgfpathlineto{\pgfqpoint{6.360055in}{4.281805in}}%
\pgfusepath{stroke}%
\end{pgfscope}%
\begin{pgfscope}%
\definecolor{textcolor}{rgb}{0.150000,0.150000,0.150000}%
\pgfsetstrokecolor{textcolor}%
\pgfsetfillcolor{textcolor}%
\pgftext[x=6.382278in,y=4.126250in,left,base]{\color{textcolor}\rmfamily\fontsize{32.000000}{38.400000}\selectfont CO-W}%
\end{pgfscope}%
\end{pgfpicture}%
\makeatother%
\endgroup%
%
}
\caption{Num of Operations}
\end{subfigure}
\caption{Impact of warmstarting on OpenML workloads}
\label{exp-model-warmstarting}
\vspace{-6mm}
\end{figure}
\subsection{Warmstarting}
In this experiment, we evaluate the effect of our warmstarting method.
We execute all the OpenML workloads with and without warmstarting.
Figure \ref{exp-model-warmstarting} shows the effect of warmstarting on the OpenML workloads.
In Figure \ref{exp-model-warmstarting}(a), we observe that without warmstarting, the cumulative run-time of the baseline (OML) and our optimizer (CO-W) is nearly identical.
Warmstarting (CO+W) reduces the total run-time by a third.
In OpenML workloads, because of the small size of the datasets, the run-time of the data transformation operations is very small, i.e., typically only a few milliseconds.
The model training operations are the main contributors to the total run-time.
Another characteristic of the OpenML workloads is that nearly every model training operation has a unique set of hyperparameters.
For example, out of the 2000 workloads that we execute, only 4 have similar hyperparameters.
Thus, the result of the model training operations cannot be reused, which results in CO-W having the same cumulative run-time as OML.
To further show that warmstarting is the main reason behind the smaller cumulative run-time of CO+W, we also plot the total number of executed operations in Figure \ref{exp-model-warmstarting}(b).
There are, on average, 25 operations in every OpenML workload.
OML does not reuse any artifacts; thus, it has to execute every operation.
In CO-W or CO+W, the average number of executed operations per workload is 7, i.e., 4 times smaller than the number of operations in OML.
Since the reuse procedure in CO-W and CO+W is identical, the cumulative number of executed operations is also identical.
However, due to the small run-time of preprocessing operations in OpenML workloads, we only observe an improvement in run-time with warmstarting.  
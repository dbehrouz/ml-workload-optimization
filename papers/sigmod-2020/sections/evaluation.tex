\section{Evaluation} \label{sec-evaluation} 
In this section, we evaluate the performance of our collaborative optimizer system.
We focus on investigating the effect of our optimizations on the execution cost of workloads in collaborative environments.
We first describe the setup, including the hardware and the workloads we execute.
Then, we show the run-time improvement of our optimizer.
Then, we investigate the effect of the individual contributions, i.e., materialization and reuse algorithms, on the execution and storage cost.
\begin{table*}[t]
\begin{tabular}{lp{0.84\textwidth}rr}
\hline
\textbf{$ID$} & \textbf{$Description$}& \textbf{$N$}& \textbf{$S$}   \\
\hline
1 &  \parbox[t]{0.84\textwidth}{\linespread{0.5}\selectfont \small Real script titled 'start-here-a-gentle-introduction'. It includes several feature engineering operations before training logistic regression, random forest, and gradient boosted tree models.} & 397 & 14.5\\[0.4cm]

2 &   \parbox[t]{0.84\textwidth}{\linespread{0.5}\selectfont \small Real script titled 'introduction-to-manual-feature-engineering'. It joins multiple tables to generate a dataset and trains gradient boosted tree models.} & 406 & 25\\[0.4cm]

3 &   \parbox[t]{0.84\textwidth}{\linespread{0.5}\selectfont \small Real script titled 'introduction-to-manual-feature-engineering-p2'. It is similar to workload 2, but with larger datasets.} & 146 & 83.5\\[0.15cm]

4 & \parbox[t]{0.84\textwidth}{\linespread{0.5}\selectfont \small A modified version workload 1 submitted by the Kaggle user: 'crldata'. It modifies the hyperparameters of the gradient boosted tree.} & 280 & 10\\[0.4cm]

5 & \parbox[t]{0.84\textwidth}{\linespread{0.5}\selectfont \small A modified version workload 1 submitted by the Kaggle user: 'taozhongxiao'. It runs random and grid search for gradient boosted tree model on the dataset of workload 1.} & 402 & 13.8\\[0.4cm]

6 & \parbox[t]{0.84\textwidth}{\linespread{0.5}\selectfont \small Custom script based on workload 2. It trains a gradient boosted tree on the dataset of workload 2.} & 121 & 21\\[0.15cm]

7 & \parbox[t]{0.84\textwidth}{\linespread{0.5}\selectfont \small Custom script based on workload 3. It trains a gradient boosted tree on the dataset of workload 3.} & 145 & 83\\[0.15cm]

8 & \parbox[t]{0.84\textwidth}{\linespread{0.5}\selectfont \small Custom script that joins the features of workload 1 and 2. Then, it trains a gradient boosted tree on the joined dataset.} & 341 & 21.1\\
\hline
\end{tabular}
\caption{Kaggle workloads description. $N$ is number of the artifacts and $S$ is total size of the artifacts in GB.}
\label{kaggle-workload}
\end{table*}

\subsection{Setup}
\hl{We execute all the experiments on a server running Linux xxx with 128 GB of RAM and xxx GB of SSD.
We implement a prototype of our system in python 2.7 supporting both in-memory and disk-based Experiment Graph.
In the prototype, we implement the Experiment Graph using python's NetworkX library.
Unless specified otherwise, we run every experiment 5 times and report the average results (error bar included).}
We evaluate our proposed optimizations on two different sets of workloads.

\textbf{OpenML workloads.} In the OpenML workloads, we retrieve all the scikit-learn pipelines for task 31\footnote{https://www.openml.org/t/31}, i.e., classifying customers as good or bad credit risks using the German Credit data from the UCI repository \cite{asuncion2007uci}.
\todo[inline]{more details after i finish kaggle experiments}
%Table \ref{tab-openml-pipelines} shows the id, components, and number of executions of the OpenML pipelines.\footnote{information about each pipeline is available at https://www.openml.org/f/id}
%
%\begin{table}
%\setlength\tabcolsep{1.5pt} % This is to ensure the table does not go out of bound
%\begin{tabular}{llr}
%\hline
%\textbf{id} & \textbf{operations} & \textbf{\#exec}   \\
%\hline
% 5981&  \makecell[l]{Imputer$\rightarrow$ Standard scaler$\rightarrow$Logistic regression} &11        \\
% 7707&  \makecell[l]{Imputer$\rightarrow$Onehot encoder$\rightarrow$Standard scaler\\$\rightarrow $Variance thresholder$\rightarrow$SVM }&594 \\
% 8315&  \makecell[l]{Imputer$\rightarrow$Onehot encoder\\ $\rightarrow$Variance thresholder$\rightarrow $Random Forest} &1084  \\
%8353 & \makecell[l]{Imputer$\rightarrow$Onehot encoder\\$\rightarrow$Variance thresholder $\rightarrow $Svm}  & 1000\\
%8568 &  \makecell[l]{Imputer$\rightarrow$Onehot encoder\\$\rightarrow$Variance thresholder$\rightarrow $Random Forest} &555 \\
%\hline
%\end{tabular}
%\caption{OpenML pipeline descriptions.}
%\label{tab-openml-pipelines}
%\end{table}

\textbf{Kaggle workloads.} 
In the second set of workloads, we recreate the collaborative environment of the use case in Section \ref{sec-background}.
We introduced three workloads in the use case.
We retrieve two more workloads from the Kaggle competition, Home Credit Default Risk.
We also design 3 workloads based on the existing ones.
Table \ref{kaggle-workload} shows details of the workloads we utilize in our experiments.
The competition has 9 datasets with a total size of 2.5 GB\footnote{https://www.kaggle.com/c/home-credit-default-risk/data}.

\subsection{Execution Time}
\textbf{Kaggle Workloads.}
We start by showing the impact of our collaborative optimizer on the use case of Section \ref{sec-background}.
In the use case, we describe three workloads (Workloads 1-3 of Table \ref{kaggle-workload}).
Kaggle reports different users have copied and modified these workloads a total of 7000 times.
Therefore, at the very least, users execute these workloads 7000 times.
Figures \ref{exp-reuse-kaggle-same-workload} (a)-(c) show the result of repeating the execution of each workload.
The Experiment Graph is empty before the first run; therefore, both the baseline and our collaborative optimizer have the same run-time.
By materializing and reusing the artifacts, our collaborative optimizer reduces the run-time of the consecutive executions by one order of magnitude.
Workload 1 executes an external visualization command that plots a bivariate kernel density estimate, which incurs a large overhead.
Since our collaborative optimizer does not store the visualization information, it must re-execute the visualization command; thus, resulting in a smaller run-time reduction.
\begin{figure}
\begin{subfigure}[b]{0.33\linewidth}
\centering
%\includegraphics[width=\linewidth]{../images/experiment-results/kaggle_home_credit/execution_time/repetition/start_here_a_gentle_introduction}
 \resizebox{\columnwidth}{!}{%
\input{../images/experiment-results/kaggle_home_credit/execution_time/repetition/start_here_a_gentle_introduction.pgf}%
}

\caption{Workload 1}
\end{subfigure}%
\begin{subfigure}[b]{0.33\linewidth}
\centering
%\includegraphics[width=\linewidth]{../images/experiment-results/kaggle_home_credit/execution_time/repetition/introduction_to_manual_feature_engineering}
 \resizebox{\columnwidth}{!}{%
%% Creator: Matplotlib, PGF backend
%%
%% To include the figure in your LaTeX document, write
%%   \input{<filename>.pgf}
%%
%% Make sure the required packages are loaded in your preamble
%%   \usepackage{pgf}
%%
%% Figures using additional raster images can only be included by \input if
%% they are in the same directory as the main LaTeX file. For loading figures
%% from other directories you can use the `import` package
%%   \usepackage{import}
%% and then include the figures with
%%   \import{<path to file>}{<filename>.pgf}
%%
%% Matplotlib used the following preamble
%%   \usepackage{fontspec}
%%   \setmonofont{Andale Mono}
%%
\begingroup%
\makeatletter%
\begin{pgfpicture}%
\pgfpathrectangle{\pgfpointorigin}{\pgfqpoint{4.964458in}{5.268083in}}%
\pgfusepath{use as bounding box, clip}%
\begin{pgfscope}%
\pgfsetbuttcap%
\pgfsetmiterjoin%
\definecolor{currentfill}{rgb}{1.000000,1.000000,1.000000}%
\pgfsetfillcolor{currentfill}%
\pgfsetlinewidth{0.000000pt}%
\definecolor{currentstroke}{rgb}{1.000000,1.000000,1.000000}%
\pgfsetstrokecolor{currentstroke}%
\pgfsetdash{}{0pt}%
\pgfpathmoveto{\pgfqpoint{0.000000in}{0.000000in}}%
\pgfpathlineto{\pgfqpoint{4.964458in}{0.000000in}}%
\pgfpathlineto{\pgfqpoint{4.964458in}{5.268083in}}%
\pgfpathlineto{\pgfqpoint{0.000000in}{5.268083in}}%
\pgfpathclose%
\pgfusepath{fill}%
\end{pgfscope}%
\begin{pgfscope}%
\pgfsetbuttcap%
\pgfsetmiterjoin%
\definecolor{currentfill}{rgb}{1.000000,1.000000,1.000000}%
\pgfsetfillcolor{currentfill}%
\pgfsetlinewidth{0.000000pt}%
\definecolor{currentstroke}{rgb}{0.000000,0.000000,0.000000}%
\pgfsetstrokecolor{currentstroke}%
\pgfsetstrokeopacity{0.000000}%
\pgfsetdash{}{0pt}%
\pgfpathmoveto{\pgfqpoint{1.535194in}{1.233139in}}%
\pgfpathlineto{\pgfqpoint{4.499430in}{1.233139in}}%
\pgfpathlineto{\pgfqpoint{4.499430in}{4.674805in}}%
\pgfpathlineto{\pgfqpoint{1.535194in}{4.674805in}}%
\pgfpathclose%
\pgfusepath{fill}%
\end{pgfscope}%
\begin{pgfscope}%
\definecolor{textcolor}{rgb}{0.150000,0.150000,0.150000}%
\pgfsetstrokecolor{textcolor}%
\pgfsetfillcolor{textcolor}%
\pgftext[x=2.276253in,y=1.069250in,,top]{\color{textcolor}\rmfamily\fontsize{36.000000}{43.200000}\selectfont 1}%
\end{pgfscope}%
\begin{pgfscope}%
\definecolor{textcolor}{rgb}{0.150000,0.150000,0.150000}%
\pgfsetstrokecolor{textcolor}%
\pgfsetfillcolor{textcolor}%
\pgftext[x=3.758371in,y=1.069250in,,top]{\color{textcolor}\rmfamily\fontsize{36.000000}{43.200000}\selectfont 2}%
\end{pgfscope}%
\begin{pgfscope}%
\definecolor{textcolor}{rgb}{0.150000,0.150000,0.150000}%
\pgfsetstrokecolor{textcolor}%
\pgfsetfillcolor{textcolor}%
\pgftext[x=3.017312in,y=0.569194in,,top]{\color{textcolor}\rmfamily\fontsize{38.000000}{45.600000}\selectfont Run}%
\end{pgfscope}%
\begin{pgfscope}%
\pgfpathrectangle{\pgfqpoint{1.535194in}{1.233139in}}{\pgfqpoint{2.964236in}{3.441667in}} %
\pgfusepath{clip}%
\pgfsetroundcap%
\pgfsetroundjoin%
\pgfsetlinewidth{0.803000pt}%
\definecolor{currentstroke}{rgb}{0.800000,0.800000,0.800000}%
\pgfsetstrokecolor{currentstroke}%
\pgfsetdash{}{0pt}%
\pgfpathmoveto{\pgfqpoint{1.535194in}{1.233139in}}%
\pgfpathlineto{\pgfqpoint{4.499430in}{1.233139in}}%
\pgfusepath{stroke}%
\end{pgfscope}%
\begin{pgfscope}%
\definecolor{textcolor}{rgb}{0.150000,0.150000,0.150000}%
\pgfsetstrokecolor{textcolor}%
\pgfsetfillcolor{textcolor}%
\pgftext[x=1.141805in,y=1.059639in,left,base]{\color{textcolor}\rmfamily\fontsize{36.000000}{43.200000}\selectfont 0}%
\end{pgfscope}%
\begin{pgfscope}%
\pgfpathrectangle{\pgfqpoint{1.535194in}{1.233139in}}{\pgfqpoint{2.964236in}{3.441667in}} %
\pgfusepath{clip}%
\pgfsetroundcap%
\pgfsetroundjoin%
\pgfsetlinewidth{0.803000pt}%
\definecolor{currentstroke}{rgb}{0.800000,0.800000,0.800000}%
\pgfsetstrokecolor{currentstroke}%
\pgfsetdash{}{0pt}%
\pgfpathmoveto{\pgfqpoint{1.535194in}{3.381485in}}%
\pgfpathlineto{\pgfqpoint{4.499430in}{3.381485in}}%
\pgfusepath{stroke}%
\end{pgfscope}%
\begin{pgfscope}%
\definecolor{textcolor}{rgb}{0.150000,0.150000,0.150000}%
\pgfsetstrokecolor{textcolor}%
\pgfsetfillcolor{textcolor}%
\pgftext[x=0.682805in,y=3.207985in,left,base]{\color{textcolor}\rmfamily\fontsize{36.000000}{43.200000}\selectfont 500}%
\end{pgfscope}%
\begin{pgfscope}%
\definecolor{textcolor}{rgb}{0.150000,0.150000,0.150000}%
\pgfsetstrokecolor{textcolor}%
\pgfsetfillcolor{textcolor}%
\pgftext[x=0.627250in,y=2.953972in,,bottom,rotate=90.000000]{\color{textcolor}\rmfamily\fontsize{38.000000}{45.600000}\selectfont Run Time (s)}%
\end{pgfscope}%
\begin{pgfscope}%
\pgfpathrectangle{\pgfqpoint{1.535194in}{1.233139in}}{\pgfqpoint{2.964236in}{3.441667in}} %
\pgfusepath{clip}%
\pgfsetbuttcap%
\pgfsetmiterjoin%
\definecolor{currentfill}{rgb}{0.798529,0.536765,0.389706}%
\pgfsetfillcolor{currentfill}%
\pgfsetlinewidth{0.803000pt}%
\definecolor{currentstroke}{rgb}{0.827451,0.827451,0.827451}%
\pgfsetstrokecolor{currentstroke}%
\pgfsetdash{}{0pt}%
\pgfpathmoveto{\pgfqpoint{1.683406in}{1.233139in}}%
\pgfpathlineto{\pgfqpoint{2.276253in}{1.233139in}}%
\pgfpathlineto{\pgfqpoint{2.276253in}{4.460613in}}%
\pgfpathlineto{\pgfqpoint{1.683406in}{4.460613in}}%
\pgfpathclose%
\pgfusepath{stroke,fill}%
\end{pgfscope}%
\begin{pgfscope}%
\pgfsetbuttcap%
\pgfsetmiterjoin%
\definecolor{currentfill}{rgb}{0.798529,0.536765,0.389706}%
\pgfsetfillcolor{currentfill}%
\pgfsetlinewidth{0.803000pt}%
\definecolor{currentstroke}{rgb}{0.827451,0.827451,0.827451}%
\pgfsetstrokecolor{currentstroke}%
\pgfsetdash{}{0pt}%
\pgfpathrectangle{\pgfqpoint{1.535194in}{1.233139in}}{\pgfqpoint{2.964236in}{3.441667in}} %
\pgfusepath{clip}%
\pgfpathmoveto{\pgfqpoint{1.683406in}{1.233139in}}%
\pgfpathlineto{\pgfqpoint{2.276253in}{1.233139in}}%
\pgfpathlineto{\pgfqpoint{2.276253in}{4.460613in}}%
\pgfpathlineto{\pgfqpoint{1.683406in}{4.460613in}}%
\pgfpathclose%
\pgfusepath{clip}%
\pgfsys@defobject{currentpattern}{\pgfqpoint{0in}{0in}}{\pgfqpoint{1in}{1in}}{%
\begin{pgfscope}%
\pgfpathrectangle{\pgfqpoint{0in}{0in}}{\pgfqpoint{1in}{1in}}%
\pgfusepath{clip}%
\pgfpathmoveto{\pgfqpoint{-0.500000in}{0.500000in}}%
\pgfpathlineto{\pgfqpoint{0.500000in}{1.500000in}}%
\pgfpathmoveto{\pgfqpoint{-0.333333in}{0.333333in}}%
\pgfpathlineto{\pgfqpoint{0.666667in}{1.333333in}}%
\pgfpathmoveto{\pgfqpoint{-0.166667in}{0.166667in}}%
\pgfpathlineto{\pgfqpoint{0.833333in}{1.166667in}}%
\pgfpathmoveto{\pgfqpoint{0.000000in}{0.000000in}}%
\pgfpathlineto{\pgfqpoint{1.000000in}{1.000000in}}%
\pgfpathmoveto{\pgfqpoint{0.166667in}{-0.166667in}}%
\pgfpathlineto{\pgfqpoint{1.166667in}{0.833333in}}%
\pgfpathmoveto{\pgfqpoint{0.333333in}{-0.333333in}}%
\pgfpathlineto{\pgfqpoint{1.333333in}{0.666667in}}%
\pgfpathmoveto{\pgfqpoint{0.500000in}{-0.500000in}}%
\pgfpathlineto{\pgfqpoint{1.500000in}{0.500000in}}%
\pgfusepath{stroke}%
\end{pgfscope}%
}%
\pgfsys@transformshift{1.683406in}{1.233139in}%
\pgfsys@useobject{currentpattern}{}%
\pgfsys@transformshift{1in}{0in}%
\pgfsys@transformshift{-1in}{0in}%
\pgfsys@transformshift{0in}{1in}%
\pgfsys@useobject{currentpattern}{}%
\pgfsys@transformshift{1in}{0in}%
\pgfsys@transformshift{-1in}{0in}%
\pgfsys@transformshift{0in}{1in}%
\pgfsys@useobject{currentpattern}{}%
\pgfsys@transformshift{1in}{0in}%
\pgfsys@transformshift{-1in}{0in}%
\pgfsys@transformshift{0in}{1in}%
\pgfsys@useobject{currentpattern}{}%
\pgfsys@transformshift{1in}{0in}%
\pgfsys@transformshift{-1in}{0in}%
\pgfsys@transformshift{0in}{1in}%
\end{pgfscope}%
\begin{pgfscope}%
\pgfpathrectangle{\pgfqpoint{1.535194in}{1.233139in}}{\pgfqpoint{2.964236in}{3.441667in}} %
\pgfusepath{clip}%
\pgfsetbuttcap%
\pgfsetmiterjoin%
\definecolor{currentfill}{rgb}{0.798529,0.536765,0.389706}%
\pgfsetfillcolor{currentfill}%
\pgfsetlinewidth{0.803000pt}%
\definecolor{currentstroke}{rgb}{0.827451,0.827451,0.827451}%
\pgfsetstrokecolor{currentstroke}%
\pgfsetdash{}{0pt}%
\pgfpathmoveto{\pgfqpoint{3.165524in}{1.233139in}}%
\pgfpathlineto{\pgfqpoint{3.758371in}{1.233139in}}%
\pgfpathlineto{\pgfqpoint{3.758371in}{4.504637in}}%
\pgfpathlineto{\pgfqpoint{3.165524in}{4.504637in}}%
\pgfpathclose%
\pgfusepath{stroke,fill}%
\end{pgfscope}%
\begin{pgfscope}%
\pgfsetbuttcap%
\pgfsetmiterjoin%
\definecolor{currentfill}{rgb}{0.798529,0.536765,0.389706}%
\pgfsetfillcolor{currentfill}%
\pgfsetlinewidth{0.803000pt}%
\definecolor{currentstroke}{rgb}{0.827451,0.827451,0.827451}%
\pgfsetstrokecolor{currentstroke}%
\pgfsetdash{}{0pt}%
\pgfpathrectangle{\pgfqpoint{1.535194in}{1.233139in}}{\pgfqpoint{2.964236in}{3.441667in}} %
\pgfusepath{clip}%
\pgfpathmoveto{\pgfqpoint{3.165524in}{1.233139in}}%
\pgfpathlineto{\pgfqpoint{3.758371in}{1.233139in}}%
\pgfpathlineto{\pgfqpoint{3.758371in}{4.504637in}}%
\pgfpathlineto{\pgfqpoint{3.165524in}{4.504637in}}%
\pgfpathclose%
\pgfusepath{clip}%
\pgfsys@defobject{currentpattern}{\pgfqpoint{0in}{0in}}{\pgfqpoint{1in}{1in}}{%
\begin{pgfscope}%
\pgfpathrectangle{\pgfqpoint{0in}{0in}}{\pgfqpoint{1in}{1in}}%
\pgfusepath{clip}%
\pgfpathmoveto{\pgfqpoint{-0.500000in}{0.500000in}}%
\pgfpathlineto{\pgfqpoint{0.500000in}{1.500000in}}%
\pgfpathmoveto{\pgfqpoint{-0.333333in}{0.333333in}}%
\pgfpathlineto{\pgfqpoint{0.666667in}{1.333333in}}%
\pgfpathmoveto{\pgfqpoint{-0.166667in}{0.166667in}}%
\pgfpathlineto{\pgfqpoint{0.833333in}{1.166667in}}%
\pgfpathmoveto{\pgfqpoint{0.000000in}{0.000000in}}%
\pgfpathlineto{\pgfqpoint{1.000000in}{1.000000in}}%
\pgfpathmoveto{\pgfqpoint{0.166667in}{-0.166667in}}%
\pgfpathlineto{\pgfqpoint{1.166667in}{0.833333in}}%
\pgfpathmoveto{\pgfqpoint{0.333333in}{-0.333333in}}%
\pgfpathlineto{\pgfqpoint{1.333333in}{0.666667in}}%
\pgfpathmoveto{\pgfqpoint{0.500000in}{-0.500000in}}%
\pgfpathlineto{\pgfqpoint{1.500000in}{0.500000in}}%
\pgfusepath{stroke}%
\end{pgfscope}%
}%
\pgfsys@transformshift{3.165524in}{1.233139in}%
\pgfsys@useobject{currentpattern}{}%
\pgfsys@transformshift{1in}{0in}%
\pgfsys@transformshift{-1in}{0in}%
\pgfsys@transformshift{0in}{1in}%
\pgfsys@useobject{currentpattern}{}%
\pgfsys@transformshift{1in}{0in}%
\pgfsys@transformshift{-1in}{0in}%
\pgfsys@transformshift{0in}{1in}%
\pgfsys@useobject{currentpattern}{}%
\pgfsys@transformshift{1in}{0in}%
\pgfsys@transformshift{-1in}{0in}%
\pgfsys@transformshift{0in}{1in}%
\pgfsys@useobject{currentpattern}{}%
\pgfsys@transformshift{1in}{0in}%
\pgfsys@transformshift{-1in}{0in}%
\pgfsys@transformshift{0in}{1in}%
\end{pgfscope}%
\begin{pgfscope}%
\pgfpathrectangle{\pgfqpoint{1.535194in}{1.233139in}}{\pgfqpoint{2.964236in}{3.441667in}} %
\pgfusepath{clip}%
\pgfsetbuttcap%
\pgfsetmiterjoin%
\definecolor{currentfill}{rgb}{0.347059,0.458824,0.641176}%
\pgfsetfillcolor{currentfill}%
\pgfsetlinewidth{0.803000pt}%
\definecolor{currentstroke}{rgb}{0.827451,0.827451,0.827451}%
\pgfsetstrokecolor{currentstroke}%
\pgfsetdash{}{0pt}%
\pgfpathmoveto{\pgfqpoint{2.276253in}{1.233139in}}%
\pgfpathlineto{\pgfqpoint{2.869100in}{1.233139in}}%
\pgfpathlineto{\pgfqpoint{2.869100in}{4.333691in}}%
\pgfpathlineto{\pgfqpoint{2.276253in}{4.333691in}}%
\pgfpathclose%
\pgfusepath{stroke,fill}%
\end{pgfscope}%
\begin{pgfscope}%
\pgfsetbuttcap%
\pgfsetmiterjoin%
\definecolor{currentfill}{rgb}{0.347059,0.458824,0.641176}%
\pgfsetfillcolor{currentfill}%
\pgfsetlinewidth{0.803000pt}%
\definecolor{currentstroke}{rgb}{0.827451,0.827451,0.827451}%
\pgfsetstrokecolor{currentstroke}%
\pgfsetdash{}{0pt}%
\pgfpathrectangle{\pgfqpoint{1.535194in}{1.233139in}}{\pgfqpoint{2.964236in}{3.441667in}} %
\pgfusepath{clip}%
\pgfpathmoveto{\pgfqpoint{2.276253in}{1.233139in}}%
\pgfpathlineto{\pgfqpoint{2.869100in}{1.233139in}}%
\pgfpathlineto{\pgfqpoint{2.869100in}{4.333691in}}%
\pgfpathlineto{\pgfqpoint{2.276253in}{4.333691in}}%
\pgfpathclose%
\pgfusepath{clip}%
\pgfsys@defobject{currentpattern}{\pgfqpoint{0in}{0in}}{\pgfqpoint{1in}{1in}}{%
\begin{pgfscope}%
\pgfpathrectangle{\pgfqpoint{0in}{0in}}{\pgfqpoint{1in}{1in}}%
\pgfusepath{clip}%
\pgfpathmoveto{\pgfqpoint{-0.500000in}{0.500000in}}%
\pgfpathlineto{\pgfqpoint{0.500000in}{1.500000in}}%
\pgfpathmoveto{\pgfqpoint{-0.333333in}{0.333333in}}%
\pgfpathlineto{\pgfqpoint{0.666667in}{1.333333in}}%
\pgfpathmoveto{\pgfqpoint{-0.166667in}{0.166667in}}%
\pgfpathlineto{\pgfqpoint{0.833333in}{1.166667in}}%
\pgfpathmoveto{\pgfqpoint{0.000000in}{0.000000in}}%
\pgfpathlineto{\pgfqpoint{1.000000in}{1.000000in}}%
\pgfpathmoveto{\pgfqpoint{0.166667in}{-0.166667in}}%
\pgfpathlineto{\pgfqpoint{1.166667in}{0.833333in}}%
\pgfpathmoveto{\pgfqpoint{0.333333in}{-0.333333in}}%
\pgfpathlineto{\pgfqpoint{1.333333in}{0.666667in}}%
\pgfpathmoveto{\pgfqpoint{0.500000in}{-0.500000in}}%
\pgfpathlineto{\pgfqpoint{1.500000in}{0.500000in}}%
\pgfpathmoveto{\pgfqpoint{-0.500000in}{0.500000in}}%
\pgfpathlineto{\pgfqpoint{0.500000in}{-0.500000in}}%
\pgfpathmoveto{\pgfqpoint{-0.333333in}{0.666667in}}%
\pgfpathlineto{\pgfqpoint{0.666667in}{-0.333333in}}%
\pgfpathmoveto{\pgfqpoint{-0.166667in}{0.833333in}}%
\pgfpathlineto{\pgfqpoint{0.833333in}{-0.166667in}}%
\pgfpathmoveto{\pgfqpoint{0.000000in}{1.000000in}}%
\pgfpathlineto{\pgfqpoint{1.000000in}{0.000000in}}%
\pgfpathmoveto{\pgfqpoint{0.166667in}{1.166667in}}%
\pgfpathlineto{\pgfqpoint{1.166667in}{0.166667in}}%
\pgfpathmoveto{\pgfqpoint{0.333333in}{1.333333in}}%
\pgfpathlineto{\pgfqpoint{1.333333in}{0.333333in}}%
\pgfpathmoveto{\pgfqpoint{0.500000in}{1.500000in}}%
\pgfpathlineto{\pgfqpoint{1.500000in}{0.500000in}}%
\pgfusepath{stroke}%
\end{pgfscope}%
}%
\pgfsys@transformshift{2.276253in}{1.233139in}%
\pgfsys@useobject{currentpattern}{}%
\pgfsys@transformshift{1in}{0in}%
\pgfsys@transformshift{-1in}{0in}%
\pgfsys@transformshift{0in}{1in}%
\pgfsys@useobject{currentpattern}{}%
\pgfsys@transformshift{1in}{0in}%
\pgfsys@transformshift{-1in}{0in}%
\pgfsys@transformshift{0in}{1in}%
\pgfsys@useobject{currentpattern}{}%
\pgfsys@transformshift{1in}{0in}%
\pgfsys@transformshift{-1in}{0in}%
\pgfsys@transformshift{0in}{1in}%
\pgfsys@useobject{currentpattern}{}%
\pgfsys@transformshift{1in}{0in}%
\pgfsys@transformshift{-1in}{0in}%
\pgfsys@transformshift{0in}{1in}%
\end{pgfscope}%
\begin{pgfscope}%
\pgfpathrectangle{\pgfqpoint{1.535194in}{1.233139in}}{\pgfqpoint{2.964236in}{3.441667in}} %
\pgfusepath{clip}%
\pgfsetbuttcap%
\pgfsetmiterjoin%
\definecolor{currentfill}{rgb}{0.347059,0.458824,0.641176}%
\pgfsetfillcolor{currentfill}%
\pgfsetlinewidth{0.803000pt}%
\definecolor{currentstroke}{rgb}{0.827451,0.827451,0.827451}%
\pgfsetstrokecolor{currentstroke}%
\pgfsetdash{}{0pt}%
\pgfpathmoveto{\pgfqpoint{3.758371in}{1.233139in}}%
\pgfpathlineto{\pgfqpoint{4.351218in}{1.233139in}}%
\pgfpathlineto{\pgfqpoint{4.351218in}{1.267312in}}%
\pgfpathlineto{\pgfqpoint{3.758371in}{1.267312in}}%
\pgfpathclose%
\pgfusepath{stroke,fill}%
\end{pgfscope}%
\begin{pgfscope}%
\pgfsetbuttcap%
\pgfsetmiterjoin%
\definecolor{currentfill}{rgb}{0.347059,0.458824,0.641176}%
\pgfsetfillcolor{currentfill}%
\pgfsetlinewidth{0.803000pt}%
\definecolor{currentstroke}{rgb}{0.827451,0.827451,0.827451}%
\pgfsetstrokecolor{currentstroke}%
\pgfsetdash{}{0pt}%
\pgfpathrectangle{\pgfqpoint{1.535194in}{1.233139in}}{\pgfqpoint{2.964236in}{3.441667in}} %
\pgfusepath{clip}%
\pgfpathmoveto{\pgfqpoint{3.758371in}{1.233139in}}%
\pgfpathlineto{\pgfqpoint{4.351218in}{1.233139in}}%
\pgfpathlineto{\pgfqpoint{4.351218in}{1.267312in}}%
\pgfpathlineto{\pgfqpoint{3.758371in}{1.267312in}}%
\pgfpathclose%
\pgfusepath{clip}%
\pgfsys@defobject{currentpattern}{\pgfqpoint{0in}{0in}}{\pgfqpoint{1in}{1in}}{%
\begin{pgfscope}%
\pgfpathrectangle{\pgfqpoint{0in}{0in}}{\pgfqpoint{1in}{1in}}%
\pgfusepath{clip}%
\pgfpathmoveto{\pgfqpoint{-0.500000in}{0.500000in}}%
\pgfpathlineto{\pgfqpoint{0.500000in}{1.500000in}}%
\pgfpathmoveto{\pgfqpoint{-0.333333in}{0.333333in}}%
\pgfpathlineto{\pgfqpoint{0.666667in}{1.333333in}}%
\pgfpathmoveto{\pgfqpoint{-0.166667in}{0.166667in}}%
\pgfpathlineto{\pgfqpoint{0.833333in}{1.166667in}}%
\pgfpathmoveto{\pgfqpoint{0.000000in}{0.000000in}}%
\pgfpathlineto{\pgfqpoint{1.000000in}{1.000000in}}%
\pgfpathmoveto{\pgfqpoint{0.166667in}{-0.166667in}}%
\pgfpathlineto{\pgfqpoint{1.166667in}{0.833333in}}%
\pgfpathmoveto{\pgfqpoint{0.333333in}{-0.333333in}}%
\pgfpathlineto{\pgfqpoint{1.333333in}{0.666667in}}%
\pgfpathmoveto{\pgfqpoint{0.500000in}{-0.500000in}}%
\pgfpathlineto{\pgfqpoint{1.500000in}{0.500000in}}%
\pgfpathmoveto{\pgfqpoint{-0.500000in}{0.500000in}}%
\pgfpathlineto{\pgfqpoint{0.500000in}{-0.500000in}}%
\pgfpathmoveto{\pgfqpoint{-0.333333in}{0.666667in}}%
\pgfpathlineto{\pgfqpoint{0.666667in}{-0.333333in}}%
\pgfpathmoveto{\pgfqpoint{-0.166667in}{0.833333in}}%
\pgfpathlineto{\pgfqpoint{0.833333in}{-0.166667in}}%
\pgfpathmoveto{\pgfqpoint{0.000000in}{1.000000in}}%
\pgfpathlineto{\pgfqpoint{1.000000in}{0.000000in}}%
\pgfpathmoveto{\pgfqpoint{0.166667in}{1.166667in}}%
\pgfpathlineto{\pgfqpoint{1.166667in}{0.166667in}}%
\pgfpathmoveto{\pgfqpoint{0.333333in}{1.333333in}}%
\pgfpathlineto{\pgfqpoint{1.333333in}{0.333333in}}%
\pgfpathmoveto{\pgfqpoint{0.500000in}{1.500000in}}%
\pgfpathlineto{\pgfqpoint{1.500000in}{0.500000in}}%
\pgfusepath{stroke}%
\end{pgfscope}%
}%
\pgfsys@transformshift{3.758371in}{1.233139in}%
\pgfsys@useobject{currentpattern}{}%
\pgfsys@transformshift{1in}{0in}%
\pgfsys@transformshift{-1in}{0in}%
\pgfsys@transformshift{0in}{1in}%
\end{pgfscope}%
\begin{pgfscope}%
\pgfpathrectangle{\pgfqpoint{1.535194in}{1.233139in}}{\pgfqpoint{2.964236in}{3.441667in}} %
\pgfusepath{clip}%
\pgfsetroundcap%
\pgfsetroundjoin%
\pgfsetlinewidth{2.168100pt}%
\definecolor{currentstroke}{rgb}{0.260000,0.260000,0.260000}%
\pgfsetstrokecolor{currentstroke}%
\pgfsetdash{}{0pt}%
\pgfpathmoveto{\pgfqpoint{1.979830in}{4.427255in}}%
\pgfpathlineto{\pgfqpoint{1.979830in}{4.481699in}}%
\pgfusepath{stroke}%
\end{pgfscope}%
\begin{pgfscope}%
\pgfpathrectangle{\pgfqpoint{1.535194in}{1.233139in}}{\pgfqpoint{2.964236in}{3.441667in}} %
\pgfusepath{clip}%
\pgfsetroundcap%
\pgfsetroundjoin%
\pgfsetlinewidth{2.168100pt}%
\definecolor{currentstroke}{rgb}{0.260000,0.260000,0.260000}%
\pgfsetstrokecolor{currentstroke}%
\pgfsetdash{}{0pt}%
\pgfpathmoveto{\pgfqpoint{3.461948in}{4.495418in}}%
\pgfpathlineto{\pgfqpoint{3.461948in}{4.510916in}}%
\pgfusepath{stroke}%
\end{pgfscope}%
\begin{pgfscope}%
\pgfpathrectangle{\pgfqpoint{1.535194in}{1.233139in}}{\pgfqpoint{2.964236in}{3.441667in}} %
\pgfusepath{clip}%
\pgfsetroundcap%
\pgfsetroundjoin%
\pgfsetlinewidth{2.168100pt}%
\definecolor{currentstroke}{rgb}{0.260000,0.260000,0.260000}%
\pgfsetstrokecolor{currentstroke}%
\pgfsetdash{}{0pt}%
\pgfpathmoveto{\pgfqpoint{2.572677in}{4.292714in}}%
\pgfpathlineto{\pgfqpoint{2.572677in}{4.374333in}}%
\pgfusepath{stroke}%
\end{pgfscope}%
\begin{pgfscope}%
\pgfpathrectangle{\pgfqpoint{1.535194in}{1.233139in}}{\pgfqpoint{2.964236in}{3.441667in}} %
\pgfusepath{clip}%
\pgfsetroundcap%
\pgfsetroundjoin%
\pgfsetlinewidth{2.168100pt}%
\definecolor{currentstroke}{rgb}{0.260000,0.260000,0.260000}%
\pgfsetstrokecolor{currentstroke}%
\pgfsetdash{}{0pt}%
\pgfpathmoveto{\pgfqpoint{4.054795in}{1.266525in}}%
\pgfpathlineto{\pgfqpoint{4.054795in}{1.267745in}}%
\pgfusepath{stroke}%
\end{pgfscope}%
\begin{pgfscope}%
\pgfsetrectcap%
\pgfsetmiterjoin%
\pgfsetlinewidth{1.003750pt}%
\definecolor{currentstroke}{rgb}{0.800000,0.800000,0.800000}%
\pgfsetstrokecolor{currentstroke}%
\pgfsetdash{}{0pt}%
\pgfpathmoveto{\pgfqpoint{1.535194in}{1.233139in}}%
\pgfpathlineto{\pgfqpoint{1.535194in}{4.674805in}}%
\pgfusepath{stroke}%
\end{pgfscope}%
\begin{pgfscope}%
\pgfsetrectcap%
\pgfsetmiterjoin%
\pgfsetlinewidth{1.003750pt}%
\definecolor{currentstroke}{rgb}{0.800000,0.800000,0.800000}%
\pgfsetstrokecolor{currentstroke}%
\pgfsetdash{}{0pt}%
\pgfpathmoveto{\pgfqpoint{1.535194in}{1.233139in}}%
\pgfpathlineto{\pgfqpoint{4.499430in}{1.233139in}}%
\pgfusepath{stroke}%
\end{pgfscope}%
\begin{pgfscope}%
\pgfsetbuttcap%
\pgfsetmiterjoin%
\definecolor{currentfill}{rgb}{0.798529,0.536765,0.389706}%
\pgfsetfillcolor{currentfill}%
\pgfsetlinewidth{0.803000pt}%
\definecolor{currentstroke}{rgb}{0.827451,0.827451,0.827451}%
\pgfsetstrokecolor{currentstroke}%
\pgfsetdash{}{0pt}%
\pgfpathmoveto{\pgfqpoint{1.621764in}{4.587527in}}%
\pgfpathlineto{\pgfqpoint{2.052319in}{4.587527in}}%
\pgfpathlineto{\pgfqpoint{2.052319in}{4.956972in}}%
\pgfpathlineto{\pgfqpoint{1.621764in}{4.956972in}}%
\pgfpathclose%
\pgfusepath{stroke,fill}%
\end{pgfscope}%
\begin{pgfscope}%
\pgfsetbuttcap%
\pgfsetmiterjoin%
\definecolor{currentfill}{rgb}{0.798529,0.536765,0.389706}%
\pgfsetfillcolor{currentfill}%
\pgfsetlinewidth{0.803000pt}%
\definecolor{currentstroke}{rgb}{0.827451,0.827451,0.827451}%
\pgfsetstrokecolor{currentstroke}%
\pgfsetdash{}{0pt}%
\pgfpathmoveto{\pgfqpoint{1.621764in}{4.587527in}}%
\pgfpathlineto{\pgfqpoint{2.052319in}{4.587527in}}%
\pgfpathlineto{\pgfqpoint{2.052319in}{4.956972in}}%
\pgfpathlineto{\pgfqpoint{1.621764in}{4.956972in}}%
\pgfpathclose%
\pgfusepath{clip}%
\pgfsys@defobject{currentpattern}{\pgfqpoint{0in}{0in}}{\pgfqpoint{1in}{1in}}{%
\begin{pgfscope}%
\pgfpathrectangle{\pgfqpoint{0in}{0in}}{\pgfqpoint{1in}{1in}}%
\pgfusepath{clip}%
\pgfpathmoveto{\pgfqpoint{-0.500000in}{0.500000in}}%
\pgfpathlineto{\pgfqpoint{0.500000in}{1.500000in}}%
\pgfpathmoveto{\pgfqpoint{-0.333333in}{0.333333in}}%
\pgfpathlineto{\pgfqpoint{0.666667in}{1.333333in}}%
\pgfpathmoveto{\pgfqpoint{-0.166667in}{0.166667in}}%
\pgfpathlineto{\pgfqpoint{0.833333in}{1.166667in}}%
\pgfpathmoveto{\pgfqpoint{0.000000in}{0.000000in}}%
\pgfpathlineto{\pgfqpoint{1.000000in}{1.000000in}}%
\pgfpathmoveto{\pgfqpoint{0.166667in}{-0.166667in}}%
\pgfpathlineto{\pgfqpoint{1.166667in}{0.833333in}}%
\pgfpathmoveto{\pgfqpoint{0.333333in}{-0.333333in}}%
\pgfpathlineto{\pgfqpoint{1.333333in}{0.666667in}}%
\pgfpathmoveto{\pgfqpoint{0.500000in}{-0.500000in}}%
\pgfpathlineto{\pgfqpoint{1.500000in}{0.500000in}}%
\pgfusepath{stroke}%
\end{pgfscope}%
}%
\pgfsys@transformshift{1.621764in}{4.587527in}%
\pgfsys@useobject{currentpattern}{}%
\pgfsys@transformshift{1in}{0in}%
\pgfsys@transformshift{-1in}{0in}%
\pgfsys@transformshift{0in}{1in}%
\end{pgfscope}%
\begin{pgfscope}%
\definecolor{textcolor}{rgb}{0.150000,0.150000,0.150000}%
\pgfsetstrokecolor{textcolor}%
\pgfsetfillcolor{textcolor}%
\pgftext[x=2.266208in,y=4.587527in,left,base]{\color{textcolor}\rmfamily\fontsize{38.000000}{45.600000}\selectfont DE}%
\end{pgfscope}%
\begin{pgfscope}%
\pgfsetbuttcap%
\pgfsetmiterjoin%
\definecolor{currentfill}{rgb}{0.347059,0.458824,0.641176}%
\pgfsetfillcolor{currentfill}%
\pgfsetlinewidth{0.803000pt}%
\definecolor{currentstroke}{rgb}{0.827451,0.827451,0.827451}%
\pgfsetstrokecolor{currentstroke}%
\pgfsetdash{}{0pt}%
\pgfpathmoveto{\pgfqpoint{3.276347in}{4.587527in}}%
\pgfpathlineto{\pgfqpoint{3.706902in}{4.587527in}}%
\pgfpathlineto{\pgfqpoint{3.706902in}{4.956972in}}%
\pgfpathlineto{\pgfqpoint{3.276347in}{4.956972in}}%
\pgfpathclose%
\pgfusepath{stroke,fill}%
\end{pgfscope}%
\begin{pgfscope}%
\pgfsetbuttcap%
\pgfsetmiterjoin%
\definecolor{currentfill}{rgb}{0.347059,0.458824,0.641176}%
\pgfsetfillcolor{currentfill}%
\pgfsetlinewidth{0.803000pt}%
\definecolor{currentstroke}{rgb}{0.827451,0.827451,0.827451}%
\pgfsetstrokecolor{currentstroke}%
\pgfsetdash{}{0pt}%
\pgfpathmoveto{\pgfqpoint{3.276347in}{4.587527in}}%
\pgfpathlineto{\pgfqpoint{3.706902in}{4.587527in}}%
\pgfpathlineto{\pgfqpoint{3.706902in}{4.956972in}}%
\pgfpathlineto{\pgfqpoint{3.276347in}{4.956972in}}%
\pgfpathclose%
\pgfusepath{clip}%
\pgfsys@defobject{currentpattern}{\pgfqpoint{0in}{0in}}{\pgfqpoint{1in}{1in}}{%
\begin{pgfscope}%
\pgfpathrectangle{\pgfqpoint{0in}{0in}}{\pgfqpoint{1in}{1in}}%
\pgfusepath{clip}%
\pgfpathmoveto{\pgfqpoint{-0.500000in}{0.500000in}}%
\pgfpathlineto{\pgfqpoint{0.500000in}{1.500000in}}%
\pgfpathmoveto{\pgfqpoint{-0.333333in}{0.333333in}}%
\pgfpathlineto{\pgfqpoint{0.666667in}{1.333333in}}%
\pgfpathmoveto{\pgfqpoint{-0.166667in}{0.166667in}}%
\pgfpathlineto{\pgfqpoint{0.833333in}{1.166667in}}%
\pgfpathmoveto{\pgfqpoint{0.000000in}{0.000000in}}%
\pgfpathlineto{\pgfqpoint{1.000000in}{1.000000in}}%
\pgfpathmoveto{\pgfqpoint{0.166667in}{-0.166667in}}%
\pgfpathlineto{\pgfqpoint{1.166667in}{0.833333in}}%
\pgfpathmoveto{\pgfqpoint{0.333333in}{-0.333333in}}%
\pgfpathlineto{\pgfqpoint{1.333333in}{0.666667in}}%
\pgfpathmoveto{\pgfqpoint{0.500000in}{-0.500000in}}%
\pgfpathlineto{\pgfqpoint{1.500000in}{0.500000in}}%
\pgfpathmoveto{\pgfqpoint{-0.500000in}{0.500000in}}%
\pgfpathlineto{\pgfqpoint{0.500000in}{-0.500000in}}%
\pgfpathmoveto{\pgfqpoint{-0.333333in}{0.666667in}}%
\pgfpathlineto{\pgfqpoint{0.666667in}{-0.333333in}}%
\pgfpathmoveto{\pgfqpoint{-0.166667in}{0.833333in}}%
\pgfpathlineto{\pgfqpoint{0.833333in}{-0.166667in}}%
\pgfpathmoveto{\pgfqpoint{0.000000in}{1.000000in}}%
\pgfpathlineto{\pgfqpoint{1.000000in}{0.000000in}}%
\pgfpathmoveto{\pgfqpoint{0.166667in}{1.166667in}}%
\pgfpathlineto{\pgfqpoint{1.166667in}{0.166667in}}%
\pgfpathmoveto{\pgfqpoint{0.333333in}{1.333333in}}%
\pgfpathlineto{\pgfqpoint{1.333333in}{0.333333in}}%
\pgfpathmoveto{\pgfqpoint{0.500000in}{1.500000in}}%
\pgfpathlineto{\pgfqpoint{1.500000in}{0.500000in}}%
\pgfusepath{stroke}%
\end{pgfscope}%
}%
\pgfsys@transformshift{3.276347in}{4.587527in}%
\pgfsys@useobject{currentpattern}{}%
\pgfsys@transformshift{1in}{0in}%
\pgfsys@transformshift{-1in}{0in}%
\pgfsys@transformshift{0in}{1in}%
\end{pgfscope}%
\begin{pgfscope}%
\definecolor{textcolor}{rgb}{0.150000,0.150000,0.150000}%
\pgfsetstrokecolor{textcolor}%
\pgfsetfillcolor{textcolor}%
\pgftext[x=3.920791in,y=4.587527in,left,base]{\color{textcolor}\rmfamily\fontsize{38.000000}{45.600000}\selectfont CO}%
\end{pgfscope}%
\end{pgfpicture}%
\makeatother%
\endgroup%
%
}

\caption{Workload 2}
\end{subfigure}%
\begin{subfigure}[b]{0.33\linewidth}
\centering
%\includegraphics[width=\linewidth]{../images/experiment-results/kaggle_home_credit/execution_time/repetition/introduction_to_manual_feature_engineering_p2}
 \resizebox{\columnwidth}{!}{%
%% Creator: Matplotlib, PGF backend
%%
%% To include the figure in your LaTeX document, write
%%   \input{<filename>.pgf}
%%
%% Make sure the required packages are loaded in your preamble
%%   \usepackage{pgf}
%%
%% Figures using additional raster images can only be included by \input if
%% they are in the same directory as the main LaTeX file. For loading figures
%% from other directories you can use the `import` package
%%   \usepackage{import}
%% and then include the figures with
%%   \import{<path to file>}{<filename>.pgf}
%%
%% Matplotlib used the following preamble
%%   \usepackage{fontspec}
%%   \setmonofont{Andale Mono}
%%
\begingroup%
\makeatletter%
\begin{pgfpicture}%
\pgfpathrectangle{\pgfpointorigin}{\pgfqpoint{5.068958in}{5.268083in}}%
\pgfusepath{use as bounding box, clip}%
\begin{pgfscope}%
\pgfsetbuttcap%
\pgfsetmiterjoin%
\definecolor{currentfill}{rgb}{1.000000,1.000000,1.000000}%
\pgfsetfillcolor{currentfill}%
\pgfsetlinewidth{0.000000pt}%
\definecolor{currentstroke}{rgb}{1.000000,1.000000,1.000000}%
\pgfsetstrokecolor{currentstroke}%
\pgfsetdash{}{0pt}%
\pgfpathmoveto{\pgfqpoint{0.000000in}{0.000000in}}%
\pgfpathlineto{\pgfqpoint{5.068958in}{0.000000in}}%
\pgfpathlineto{\pgfqpoint{5.068958in}{5.268083in}}%
\pgfpathlineto{\pgfqpoint{0.000000in}{5.268083in}}%
\pgfpathclose%
\pgfusepath{fill}%
\end{pgfscope}%
\begin{pgfscope}%
\pgfsetbuttcap%
\pgfsetmiterjoin%
\definecolor{currentfill}{rgb}{1.000000,1.000000,1.000000}%
\pgfsetfillcolor{currentfill}%
\pgfsetlinewidth{0.000000pt}%
\definecolor{currentstroke}{rgb}{0.000000,0.000000,0.000000}%
\pgfsetstrokecolor{currentstroke}%
\pgfsetstrokeopacity{0.000000}%
\pgfsetdash{}{0pt}%
\pgfpathmoveto{\pgfqpoint{1.764694in}{1.233139in}}%
\pgfpathlineto{\pgfqpoint{4.416430in}{1.233139in}}%
\pgfpathlineto{\pgfqpoint{4.416430in}{4.674805in}}%
\pgfpathlineto{\pgfqpoint{1.764694in}{4.674805in}}%
\pgfpathclose%
\pgfusepath{fill}%
\end{pgfscope}%
\begin{pgfscope}%
\definecolor{textcolor}{rgb}{0.150000,0.150000,0.150000}%
\pgfsetstrokecolor{textcolor}%
\pgfsetfillcolor{textcolor}%
\pgftext[x=2.427628in,y=1.069250in,,top]{\color{textcolor}\rmfamily\fontsize{36.000000}{43.200000}\selectfont 1}%
\end{pgfscope}%
\begin{pgfscope}%
\definecolor{textcolor}{rgb}{0.150000,0.150000,0.150000}%
\pgfsetstrokecolor{textcolor}%
\pgfsetfillcolor{textcolor}%
\pgftext[x=3.753496in,y=1.069250in,,top]{\color{textcolor}\rmfamily\fontsize{36.000000}{43.200000}\selectfont 2}%
\end{pgfscope}%
\begin{pgfscope}%
\definecolor{textcolor}{rgb}{0.150000,0.150000,0.150000}%
\pgfsetstrokecolor{textcolor}%
\pgfsetfillcolor{textcolor}%
\pgftext[x=3.090562in,y=0.569194in,,top]{\color{textcolor}\rmfamily\fontsize{38.000000}{45.600000}\selectfont Run}%
\end{pgfscope}%
\begin{pgfscope}%
\pgfpathrectangle{\pgfqpoint{1.764694in}{1.233139in}}{\pgfqpoint{2.651736in}{3.441667in}} %
\pgfusepath{clip}%
\pgfsetroundcap%
\pgfsetroundjoin%
\pgfsetlinewidth{0.803000pt}%
\definecolor{currentstroke}{rgb}{0.800000,0.800000,0.800000}%
\pgfsetstrokecolor{currentstroke}%
\pgfsetdash{}{0pt}%
\pgfpathmoveto{\pgfqpoint{1.764694in}{1.233139in}}%
\pgfpathlineto{\pgfqpoint{4.416430in}{1.233139in}}%
\pgfusepath{stroke}%
\end{pgfscope}%
\begin{pgfscope}%
\definecolor{textcolor}{rgb}{0.150000,0.150000,0.150000}%
\pgfsetstrokecolor{textcolor}%
\pgfsetfillcolor{textcolor}%
\pgftext[x=1.371305in,y=1.059639in,left,base]{\color{textcolor}\rmfamily\fontsize{36.000000}{43.200000}\selectfont 0}%
\end{pgfscope}%
\begin{pgfscope}%
\pgfpathrectangle{\pgfqpoint{1.764694in}{1.233139in}}{\pgfqpoint{2.651736in}{3.441667in}} %
\pgfusepath{clip}%
\pgfsetroundcap%
\pgfsetroundjoin%
\pgfsetlinewidth{0.803000pt}%
\definecolor{currentstroke}{rgb}{0.800000,0.800000,0.800000}%
\pgfsetstrokecolor{currentstroke}%
\pgfsetdash{}{0pt}%
\pgfpathmoveto{\pgfqpoint{1.764694in}{2.532223in}}%
\pgfpathlineto{\pgfqpoint{4.416430in}{2.532223in}}%
\pgfusepath{stroke}%
\end{pgfscope}%
\begin{pgfscope}%
\definecolor{textcolor}{rgb}{0.150000,0.150000,0.150000}%
\pgfsetstrokecolor{textcolor}%
\pgfsetfillcolor{textcolor}%
\pgftext[x=0.912305in,y=2.358723in,left,base]{\color{textcolor}\rmfamily\fontsize{36.000000}{43.200000}\selectfont 500}%
\end{pgfscope}%
\begin{pgfscope}%
\pgfpathrectangle{\pgfqpoint{1.764694in}{1.233139in}}{\pgfqpoint{2.651736in}{3.441667in}} %
\pgfusepath{clip}%
\pgfsetroundcap%
\pgfsetroundjoin%
\pgfsetlinewidth{0.803000pt}%
\definecolor{currentstroke}{rgb}{0.800000,0.800000,0.800000}%
\pgfsetstrokecolor{currentstroke}%
\pgfsetdash{}{0pt}%
\pgfpathmoveto{\pgfqpoint{1.764694in}{3.831307in}}%
\pgfpathlineto{\pgfqpoint{4.416430in}{3.831307in}}%
\pgfusepath{stroke}%
\end{pgfscope}%
\begin{pgfscope}%
\definecolor{textcolor}{rgb}{0.150000,0.150000,0.150000}%
\pgfsetstrokecolor{textcolor}%
\pgfsetfillcolor{textcolor}%
\pgftext[x=0.682805in,y=3.657807in,left,base]{\color{textcolor}\rmfamily\fontsize{36.000000}{43.200000}\selectfont 1000}%
\end{pgfscope}%
\begin{pgfscope}%
\definecolor{textcolor}{rgb}{0.150000,0.150000,0.150000}%
\pgfsetstrokecolor{textcolor}%
\pgfsetfillcolor{textcolor}%
\pgftext[x=0.627250in,y=2.953972in,,bottom,rotate=90.000000]{\color{textcolor}\rmfamily\fontsize{38.000000}{45.600000}\selectfont Run Time (s)}%
\end{pgfscope}%
\begin{pgfscope}%
\pgfpathrectangle{\pgfqpoint{1.764694in}{1.233139in}}{\pgfqpoint{2.651736in}{3.441667in}} %
\pgfusepath{clip}%
\pgfsetbuttcap%
\pgfsetmiterjoin%
\definecolor{currentfill}{rgb}{0.798529,0.536765,0.389706}%
\pgfsetfillcolor{currentfill}%
\pgfsetlinewidth{0.803000pt}%
\definecolor{currentstroke}{rgb}{0.827451,0.827451,0.827451}%
\pgfsetstrokecolor{currentstroke}%
\pgfsetdash{}{0pt}%
\pgfpathmoveto{\pgfqpoint{1.897281in}{1.233139in}}%
\pgfpathlineto{\pgfqpoint{2.427628in}{1.233139in}}%
\pgfpathlineto{\pgfqpoint{2.427628in}{4.246158in}}%
\pgfpathlineto{\pgfqpoint{1.897281in}{4.246158in}}%
\pgfpathclose%
\pgfusepath{stroke,fill}%
\end{pgfscope}%
\begin{pgfscope}%
\pgfsetbuttcap%
\pgfsetmiterjoin%
\definecolor{currentfill}{rgb}{0.798529,0.536765,0.389706}%
\pgfsetfillcolor{currentfill}%
\pgfsetlinewidth{0.803000pt}%
\definecolor{currentstroke}{rgb}{0.827451,0.827451,0.827451}%
\pgfsetstrokecolor{currentstroke}%
\pgfsetdash{}{0pt}%
\pgfpathrectangle{\pgfqpoint{1.764694in}{1.233139in}}{\pgfqpoint{2.651736in}{3.441667in}} %
\pgfusepath{clip}%
\pgfpathmoveto{\pgfqpoint{1.897281in}{1.233139in}}%
\pgfpathlineto{\pgfqpoint{2.427628in}{1.233139in}}%
\pgfpathlineto{\pgfqpoint{2.427628in}{4.246158in}}%
\pgfpathlineto{\pgfqpoint{1.897281in}{4.246158in}}%
\pgfpathclose%
\pgfusepath{clip}%
\pgfsys@defobject{currentpattern}{\pgfqpoint{0in}{0in}}{\pgfqpoint{1in}{1in}}{%
\begin{pgfscope}%
\pgfpathrectangle{\pgfqpoint{0in}{0in}}{\pgfqpoint{1in}{1in}}%
\pgfusepath{clip}%
\pgfpathmoveto{\pgfqpoint{-0.500000in}{0.500000in}}%
\pgfpathlineto{\pgfqpoint{0.500000in}{1.500000in}}%
\pgfpathmoveto{\pgfqpoint{-0.333333in}{0.333333in}}%
\pgfpathlineto{\pgfqpoint{0.666667in}{1.333333in}}%
\pgfpathmoveto{\pgfqpoint{-0.166667in}{0.166667in}}%
\pgfpathlineto{\pgfqpoint{0.833333in}{1.166667in}}%
\pgfpathmoveto{\pgfqpoint{0.000000in}{0.000000in}}%
\pgfpathlineto{\pgfqpoint{1.000000in}{1.000000in}}%
\pgfpathmoveto{\pgfqpoint{0.166667in}{-0.166667in}}%
\pgfpathlineto{\pgfqpoint{1.166667in}{0.833333in}}%
\pgfpathmoveto{\pgfqpoint{0.333333in}{-0.333333in}}%
\pgfpathlineto{\pgfqpoint{1.333333in}{0.666667in}}%
\pgfpathmoveto{\pgfqpoint{0.500000in}{-0.500000in}}%
\pgfpathlineto{\pgfqpoint{1.500000in}{0.500000in}}%
\pgfusepath{stroke}%
\end{pgfscope}%
}%
\pgfsys@transformshift{1.897281in}{1.233139in}%
\pgfsys@useobject{currentpattern}{}%
\pgfsys@transformshift{1in}{0in}%
\pgfsys@transformshift{-1in}{0in}%
\pgfsys@transformshift{0in}{1in}%
\pgfsys@useobject{currentpattern}{}%
\pgfsys@transformshift{1in}{0in}%
\pgfsys@transformshift{-1in}{0in}%
\pgfsys@transformshift{0in}{1in}%
\pgfsys@useobject{currentpattern}{}%
\pgfsys@transformshift{1in}{0in}%
\pgfsys@transformshift{-1in}{0in}%
\pgfsys@transformshift{0in}{1in}%
\pgfsys@useobject{currentpattern}{}%
\pgfsys@transformshift{1in}{0in}%
\pgfsys@transformshift{-1in}{0in}%
\pgfsys@transformshift{0in}{1in}%
\end{pgfscope}%
\begin{pgfscope}%
\pgfpathrectangle{\pgfqpoint{1.764694in}{1.233139in}}{\pgfqpoint{2.651736in}{3.441667in}} %
\pgfusepath{clip}%
\pgfsetbuttcap%
\pgfsetmiterjoin%
\definecolor{currentfill}{rgb}{0.798529,0.536765,0.389706}%
\pgfsetfillcolor{currentfill}%
\pgfsetlinewidth{0.803000pt}%
\definecolor{currentstroke}{rgb}{0.827451,0.827451,0.827451}%
\pgfsetstrokecolor{currentstroke}%
\pgfsetdash{}{0pt}%
\pgfpathmoveto{\pgfqpoint{3.223149in}{1.233139in}}%
\pgfpathlineto{\pgfqpoint{3.753496in}{1.233139in}}%
\pgfpathlineto{\pgfqpoint{3.753496in}{4.372438in}}%
\pgfpathlineto{\pgfqpoint{3.223149in}{4.372438in}}%
\pgfpathclose%
\pgfusepath{stroke,fill}%
\end{pgfscope}%
\begin{pgfscope}%
\pgfsetbuttcap%
\pgfsetmiterjoin%
\definecolor{currentfill}{rgb}{0.798529,0.536765,0.389706}%
\pgfsetfillcolor{currentfill}%
\pgfsetlinewidth{0.803000pt}%
\definecolor{currentstroke}{rgb}{0.827451,0.827451,0.827451}%
\pgfsetstrokecolor{currentstroke}%
\pgfsetdash{}{0pt}%
\pgfpathrectangle{\pgfqpoint{1.764694in}{1.233139in}}{\pgfqpoint{2.651736in}{3.441667in}} %
\pgfusepath{clip}%
\pgfpathmoveto{\pgfqpoint{3.223149in}{1.233139in}}%
\pgfpathlineto{\pgfqpoint{3.753496in}{1.233139in}}%
\pgfpathlineto{\pgfqpoint{3.753496in}{4.372438in}}%
\pgfpathlineto{\pgfqpoint{3.223149in}{4.372438in}}%
\pgfpathclose%
\pgfusepath{clip}%
\pgfsys@defobject{currentpattern}{\pgfqpoint{0in}{0in}}{\pgfqpoint{1in}{1in}}{%
\begin{pgfscope}%
\pgfpathrectangle{\pgfqpoint{0in}{0in}}{\pgfqpoint{1in}{1in}}%
\pgfusepath{clip}%
\pgfpathmoveto{\pgfqpoint{-0.500000in}{0.500000in}}%
\pgfpathlineto{\pgfqpoint{0.500000in}{1.500000in}}%
\pgfpathmoveto{\pgfqpoint{-0.333333in}{0.333333in}}%
\pgfpathlineto{\pgfqpoint{0.666667in}{1.333333in}}%
\pgfpathmoveto{\pgfqpoint{-0.166667in}{0.166667in}}%
\pgfpathlineto{\pgfqpoint{0.833333in}{1.166667in}}%
\pgfpathmoveto{\pgfqpoint{0.000000in}{0.000000in}}%
\pgfpathlineto{\pgfqpoint{1.000000in}{1.000000in}}%
\pgfpathmoveto{\pgfqpoint{0.166667in}{-0.166667in}}%
\pgfpathlineto{\pgfqpoint{1.166667in}{0.833333in}}%
\pgfpathmoveto{\pgfqpoint{0.333333in}{-0.333333in}}%
\pgfpathlineto{\pgfqpoint{1.333333in}{0.666667in}}%
\pgfpathmoveto{\pgfqpoint{0.500000in}{-0.500000in}}%
\pgfpathlineto{\pgfqpoint{1.500000in}{0.500000in}}%
\pgfusepath{stroke}%
\end{pgfscope}%
}%
\pgfsys@transformshift{3.223149in}{1.233139in}%
\pgfsys@useobject{currentpattern}{}%
\pgfsys@transformshift{1in}{0in}%
\pgfsys@transformshift{-1in}{0in}%
\pgfsys@transformshift{0in}{1in}%
\pgfsys@useobject{currentpattern}{}%
\pgfsys@transformshift{1in}{0in}%
\pgfsys@transformshift{-1in}{0in}%
\pgfsys@transformshift{0in}{1in}%
\pgfsys@useobject{currentpattern}{}%
\pgfsys@transformshift{1in}{0in}%
\pgfsys@transformshift{-1in}{0in}%
\pgfsys@transformshift{0in}{1in}%
\pgfsys@useobject{currentpattern}{}%
\pgfsys@transformshift{1in}{0in}%
\pgfsys@transformshift{-1in}{0in}%
\pgfsys@transformshift{0in}{1in}%
\end{pgfscope}%
\begin{pgfscope}%
\pgfpathrectangle{\pgfqpoint{1.764694in}{1.233139in}}{\pgfqpoint{2.651736in}{3.441667in}} %
\pgfusepath{clip}%
\pgfsetbuttcap%
\pgfsetmiterjoin%
\definecolor{currentfill}{rgb}{0.347059,0.458824,0.641176}%
\pgfsetfillcolor{currentfill}%
\pgfsetlinewidth{0.803000pt}%
\definecolor{currentstroke}{rgb}{0.827451,0.827451,0.827451}%
\pgfsetstrokecolor{currentstroke}%
\pgfsetdash{}{0pt}%
\pgfpathmoveto{\pgfqpoint{2.427628in}{1.233139in}}%
\pgfpathlineto{\pgfqpoint{2.957976in}{1.233139in}}%
\pgfpathlineto{\pgfqpoint{2.957976in}{4.103945in}}%
\pgfpathlineto{\pgfqpoint{2.427628in}{4.103945in}}%
\pgfpathclose%
\pgfusepath{stroke,fill}%
\end{pgfscope}%
\begin{pgfscope}%
\pgfsetbuttcap%
\pgfsetmiterjoin%
\definecolor{currentfill}{rgb}{0.347059,0.458824,0.641176}%
\pgfsetfillcolor{currentfill}%
\pgfsetlinewidth{0.803000pt}%
\definecolor{currentstroke}{rgb}{0.827451,0.827451,0.827451}%
\pgfsetstrokecolor{currentstroke}%
\pgfsetdash{}{0pt}%
\pgfpathrectangle{\pgfqpoint{1.764694in}{1.233139in}}{\pgfqpoint{2.651736in}{3.441667in}} %
\pgfusepath{clip}%
\pgfpathmoveto{\pgfqpoint{2.427628in}{1.233139in}}%
\pgfpathlineto{\pgfqpoint{2.957976in}{1.233139in}}%
\pgfpathlineto{\pgfqpoint{2.957976in}{4.103945in}}%
\pgfpathlineto{\pgfqpoint{2.427628in}{4.103945in}}%
\pgfpathclose%
\pgfusepath{clip}%
\pgfsys@defobject{currentpattern}{\pgfqpoint{0in}{0in}}{\pgfqpoint{1in}{1in}}{%
\begin{pgfscope}%
\pgfpathrectangle{\pgfqpoint{0in}{0in}}{\pgfqpoint{1in}{1in}}%
\pgfusepath{clip}%
\pgfpathmoveto{\pgfqpoint{-0.500000in}{0.500000in}}%
\pgfpathlineto{\pgfqpoint{0.500000in}{1.500000in}}%
\pgfpathmoveto{\pgfqpoint{-0.333333in}{0.333333in}}%
\pgfpathlineto{\pgfqpoint{0.666667in}{1.333333in}}%
\pgfpathmoveto{\pgfqpoint{-0.166667in}{0.166667in}}%
\pgfpathlineto{\pgfqpoint{0.833333in}{1.166667in}}%
\pgfpathmoveto{\pgfqpoint{0.000000in}{0.000000in}}%
\pgfpathlineto{\pgfqpoint{1.000000in}{1.000000in}}%
\pgfpathmoveto{\pgfqpoint{0.166667in}{-0.166667in}}%
\pgfpathlineto{\pgfqpoint{1.166667in}{0.833333in}}%
\pgfpathmoveto{\pgfqpoint{0.333333in}{-0.333333in}}%
\pgfpathlineto{\pgfqpoint{1.333333in}{0.666667in}}%
\pgfpathmoveto{\pgfqpoint{0.500000in}{-0.500000in}}%
\pgfpathlineto{\pgfqpoint{1.500000in}{0.500000in}}%
\pgfpathmoveto{\pgfqpoint{-0.500000in}{0.500000in}}%
\pgfpathlineto{\pgfqpoint{0.500000in}{-0.500000in}}%
\pgfpathmoveto{\pgfqpoint{-0.333333in}{0.666667in}}%
\pgfpathlineto{\pgfqpoint{0.666667in}{-0.333333in}}%
\pgfpathmoveto{\pgfqpoint{-0.166667in}{0.833333in}}%
\pgfpathlineto{\pgfqpoint{0.833333in}{-0.166667in}}%
\pgfpathmoveto{\pgfqpoint{0.000000in}{1.000000in}}%
\pgfpathlineto{\pgfqpoint{1.000000in}{0.000000in}}%
\pgfpathmoveto{\pgfqpoint{0.166667in}{1.166667in}}%
\pgfpathlineto{\pgfqpoint{1.166667in}{0.166667in}}%
\pgfpathmoveto{\pgfqpoint{0.333333in}{1.333333in}}%
\pgfpathlineto{\pgfqpoint{1.333333in}{0.333333in}}%
\pgfpathmoveto{\pgfqpoint{0.500000in}{1.500000in}}%
\pgfpathlineto{\pgfqpoint{1.500000in}{0.500000in}}%
\pgfusepath{stroke}%
\end{pgfscope}%
}%
\pgfsys@transformshift{2.427628in}{1.233139in}%
\pgfsys@useobject{currentpattern}{}%
\pgfsys@transformshift{1in}{0in}%
\pgfsys@transformshift{-1in}{0in}%
\pgfsys@transformshift{0in}{1in}%
\pgfsys@useobject{currentpattern}{}%
\pgfsys@transformshift{1in}{0in}%
\pgfsys@transformshift{-1in}{0in}%
\pgfsys@transformshift{0in}{1in}%
\pgfsys@useobject{currentpattern}{}%
\pgfsys@transformshift{1in}{0in}%
\pgfsys@transformshift{-1in}{0in}%
\pgfsys@transformshift{0in}{1in}%
\end{pgfscope}%
\begin{pgfscope}%
\pgfpathrectangle{\pgfqpoint{1.764694in}{1.233139in}}{\pgfqpoint{2.651736in}{3.441667in}} %
\pgfusepath{clip}%
\pgfsetbuttcap%
\pgfsetmiterjoin%
\definecolor{currentfill}{rgb}{0.347059,0.458824,0.641176}%
\pgfsetfillcolor{currentfill}%
\pgfsetlinewidth{0.803000pt}%
\definecolor{currentstroke}{rgb}{0.827451,0.827451,0.827451}%
\pgfsetstrokecolor{currentstroke}%
\pgfsetdash{}{0pt}%
\pgfpathmoveto{\pgfqpoint{3.753496in}{1.233139in}}%
\pgfpathlineto{\pgfqpoint{4.283844in}{1.233139in}}%
\pgfpathlineto{\pgfqpoint{4.283844in}{1.427924in}}%
\pgfpathlineto{\pgfqpoint{3.753496in}{1.427924in}}%
\pgfpathclose%
\pgfusepath{stroke,fill}%
\end{pgfscope}%
\begin{pgfscope}%
\pgfsetbuttcap%
\pgfsetmiterjoin%
\definecolor{currentfill}{rgb}{0.347059,0.458824,0.641176}%
\pgfsetfillcolor{currentfill}%
\pgfsetlinewidth{0.803000pt}%
\definecolor{currentstroke}{rgb}{0.827451,0.827451,0.827451}%
\pgfsetstrokecolor{currentstroke}%
\pgfsetdash{}{0pt}%
\pgfpathrectangle{\pgfqpoint{1.764694in}{1.233139in}}{\pgfqpoint{2.651736in}{3.441667in}} %
\pgfusepath{clip}%
\pgfpathmoveto{\pgfqpoint{3.753496in}{1.233139in}}%
\pgfpathlineto{\pgfqpoint{4.283844in}{1.233139in}}%
\pgfpathlineto{\pgfqpoint{4.283844in}{1.427924in}}%
\pgfpathlineto{\pgfqpoint{3.753496in}{1.427924in}}%
\pgfpathclose%
\pgfusepath{clip}%
\pgfsys@defobject{currentpattern}{\pgfqpoint{0in}{0in}}{\pgfqpoint{1in}{1in}}{%
\begin{pgfscope}%
\pgfpathrectangle{\pgfqpoint{0in}{0in}}{\pgfqpoint{1in}{1in}}%
\pgfusepath{clip}%
\pgfpathmoveto{\pgfqpoint{-0.500000in}{0.500000in}}%
\pgfpathlineto{\pgfqpoint{0.500000in}{1.500000in}}%
\pgfpathmoveto{\pgfqpoint{-0.333333in}{0.333333in}}%
\pgfpathlineto{\pgfqpoint{0.666667in}{1.333333in}}%
\pgfpathmoveto{\pgfqpoint{-0.166667in}{0.166667in}}%
\pgfpathlineto{\pgfqpoint{0.833333in}{1.166667in}}%
\pgfpathmoveto{\pgfqpoint{0.000000in}{0.000000in}}%
\pgfpathlineto{\pgfqpoint{1.000000in}{1.000000in}}%
\pgfpathmoveto{\pgfqpoint{0.166667in}{-0.166667in}}%
\pgfpathlineto{\pgfqpoint{1.166667in}{0.833333in}}%
\pgfpathmoveto{\pgfqpoint{0.333333in}{-0.333333in}}%
\pgfpathlineto{\pgfqpoint{1.333333in}{0.666667in}}%
\pgfpathmoveto{\pgfqpoint{0.500000in}{-0.500000in}}%
\pgfpathlineto{\pgfqpoint{1.500000in}{0.500000in}}%
\pgfpathmoveto{\pgfqpoint{-0.500000in}{0.500000in}}%
\pgfpathlineto{\pgfqpoint{0.500000in}{-0.500000in}}%
\pgfpathmoveto{\pgfqpoint{-0.333333in}{0.666667in}}%
\pgfpathlineto{\pgfqpoint{0.666667in}{-0.333333in}}%
\pgfpathmoveto{\pgfqpoint{-0.166667in}{0.833333in}}%
\pgfpathlineto{\pgfqpoint{0.833333in}{-0.166667in}}%
\pgfpathmoveto{\pgfqpoint{0.000000in}{1.000000in}}%
\pgfpathlineto{\pgfqpoint{1.000000in}{0.000000in}}%
\pgfpathmoveto{\pgfqpoint{0.166667in}{1.166667in}}%
\pgfpathlineto{\pgfqpoint{1.166667in}{0.166667in}}%
\pgfpathmoveto{\pgfqpoint{0.333333in}{1.333333in}}%
\pgfpathlineto{\pgfqpoint{1.333333in}{0.333333in}}%
\pgfpathmoveto{\pgfqpoint{0.500000in}{1.500000in}}%
\pgfpathlineto{\pgfqpoint{1.500000in}{0.500000in}}%
\pgfusepath{stroke}%
\end{pgfscope}%
}%
\pgfsys@transformshift{3.753496in}{1.233139in}%
\pgfsys@useobject{currentpattern}{}%
\pgfsys@transformshift{1in}{0in}%
\pgfsys@transformshift{-1in}{0in}%
\pgfsys@transformshift{0in}{1in}%
\end{pgfscope}%
\begin{pgfscope}%
\pgfpathrectangle{\pgfqpoint{1.764694in}{1.233139in}}{\pgfqpoint{2.651736in}{3.441667in}} %
\pgfusepath{clip}%
\pgfsetroundcap%
\pgfsetroundjoin%
\pgfsetlinewidth{2.168100pt}%
\definecolor{currentstroke}{rgb}{0.260000,0.260000,0.260000}%
\pgfsetstrokecolor{currentstroke}%
\pgfsetdash{}{0pt}%
\pgfpathmoveto{\pgfqpoint{2.162455in}{4.091738in}}%
\pgfpathlineto{\pgfqpoint{2.162455in}{4.353052in}}%
\pgfusepath{stroke}%
\end{pgfscope}%
\begin{pgfscope}%
\pgfpathrectangle{\pgfqpoint{1.764694in}{1.233139in}}{\pgfqpoint{2.651736in}{3.441667in}} %
\pgfusepath{clip}%
\pgfsetroundcap%
\pgfsetroundjoin%
\pgfsetlinewidth{2.168100pt}%
\definecolor{currentstroke}{rgb}{0.260000,0.260000,0.260000}%
\pgfsetstrokecolor{currentstroke}%
\pgfsetdash{}{0pt}%
\pgfpathmoveto{\pgfqpoint{3.488323in}{4.147363in}}%
\pgfpathlineto{\pgfqpoint{3.488323in}{4.510916in}}%
\pgfusepath{stroke}%
\end{pgfscope}%
\begin{pgfscope}%
\pgfpathrectangle{\pgfqpoint{1.764694in}{1.233139in}}{\pgfqpoint{2.651736in}{3.441667in}} %
\pgfusepath{clip}%
\pgfsetroundcap%
\pgfsetroundjoin%
\pgfsetlinewidth{2.168100pt}%
\definecolor{currentstroke}{rgb}{0.260000,0.260000,0.260000}%
\pgfsetstrokecolor{currentstroke}%
\pgfsetdash{}{0pt}%
\pgfpathmoveto{\pgfqpoint{2.692802in}{4.099214in}}%
\pgfpathlineto{\pgfqpoint{2.692802in}{4.109645in}}%
\pgfusepath{stroke}%
\end{pgfscope}%
\begin{pgfscope}%
\pgfpathrectangle{\pgfqpoint{1.764694in}{1.233139in}}{\pgfqpoint{2.651736in}{3.441667in}} %
\pgfusepath{clip}%
\pgfsetroundcap%
\pgfsetroundjoin%
\pgfsetlinewidth{2.168100pt}%
\definecolor{currentstroke}{rgb}{0.260000,0.260000,0.260000}%
\pgfsetstrokecolor{currentstroke}%
\pgfsetdash{}{0pt}%
\pgfpathmoveto{\pgfqpoint{4.018670in}{1.422578in}}%
\pgfpathlineto{\pgfqpoint{4.018670in}{1.438206in}}%
\pgfusepath{stroke}%
\end{pgfscope}%
\begin{pgfscope}%
\pgfsetrectcap%
\pgfsetmiterjoin%
\pgfsetlinewidth{1.003750pt}%
\definecolor{currentstroke}{rgb}{0.800000,0.800000,0.800000}%
\pgfsetstrokecolor{currentstroke}%
\pgfsetdash{}{0pt}%
\pgfpathmoveto{\pgfqpoint{1.764694in}{1.233139in}}%
\pgfpathlineto{\pgfqpoint{1.764694in}{4.674805in}}%
\pgfusepath{stroke}%
\end{pgfscope}%
\begin{pgfscope}%
\pgfsetrectcap%
\pgfsetmiterjoin%
\pgfsetlinewidth{1.003750pt}%
\definecolor{currentstroke}{rgb}{0.800000,0.800000,0.800000}%
\pgfsetstrokecolor{currentstroke}%
\pgfsetdash{}{0pt}%
\pgfpathmoveto{\pgfqpoint{1.764694in}{1.233139in}}%
\pgfpathlineto{\pgfqpoint{4.416430in}{1.233139in}}%
\pgfusepath{stroke}%
\end{pgfscope}%
\begin{pgfscope}%
\pgfsetbuttcap%
\pgfsetmiterjoin%
\definecolor{currentfill}{rgb}{0.798529,0.536765,0.389706}%
\pgfsetfillcolor{currentfill}%
\pgfsetlinewidth{0.803000pt}%
\definecolor{currentstroke}{rgb}{0.827451,0.827451,0.827451}%
\pgfsetstrokecolor{currentstroke}%
\pgfsetdash{}{0pt}%
\pgfpathmoveto{\pgfqpoint{1.726264in}{4.587527in}}%
\pgfpathlineto{\pgfqpoint{2.156819in}{4.587527in}}%
\pgfpathlineto{\pgfqpoint{2.156819in}{4.956972in}}%
\pgfpathlineto{\pgfqpoint{1.726264in}{4.956972in}}%
\pgfpathclose%
\pgfusepath{stroke,fill}%
\end{pgfscope}%
\begin{pgfscope}%
\pgfsetbuttcap%
\pgfsetmiterjoin%
\definecolor{currentfill}{rgb}{0.798529,0.536765,0.389706}%
\pgfsetfillcolor{currentfill}%
\pgfsetlinewidth{0.803000pt}%
\definecolor{currentstroke}{rgb}{0.827451,0.827451,0.827451}%
\pgfsetstrokecolor{currentstroke}%
\pgfsetdash{}{0pt}%
\pgfpathmoveto{\pgfqpoint{1.726264in}{4.587527in}}%
\pgfpathlineto{\pgfqpoint{2.156819in}{4.587527in}}%
\pgfpathlineto{\pgfqpoint{2.156819in}{4.956972in}}%
\pgfpathlineto{\pgfqpoint{1.726264in}{4.956972in}}%
\pgfpathclose%
\pgfusepath{clip}%
\pgfsys@defobject{currentpattern}{\pgfqpoint{0in}{0in}}{\pgfqpoint{1in}{1in}}{%
\begin{pgfscope}%
\pgfpathrectangle{\pgfqpoint{0in}{0in}}{\pgfqpoint{1in}{1in}}%
\pgfusepath{clip}%
\pgfpathmoveto{\pgfqpoint{-0.500000in}{0.500000in}}%
\pgfpathlineto{\pgfqpoint{0.500000in}{1.500000in}}%
\pgfpathmoveto{\pgfqpoint{-0.333333in}{0.333333in}}%
\pgfpathlineto{\pgfqpoint{0.666667in}{1.333333in}}%
\pgfpathmoveto{\pgfqpoint{-0.166667in}{0.166667in}}%
\pgfpathlineto{\pgfqpoint{0.833333in}{1.166667in}}%
\pgfpathmoveto{\pgfqpoint{0.000000in}{0.000000in}}%
\pgfpathlineto{\pgfqpoint{1.000000in}{1.000000in}}%
\pgfpathmoveto{\pgfqpoint{0.166667in}{-0.166667in}}%
\pgfpathlineto{\pgfqpoint{1.166667in}{0.833333in}}%
\pgfpathmoveto{\pgfqpoint{0.333333in}{-0.333333in}}%
\pgfpathlineto{\pgfqpoint{1.333333in}{0.666667in}}%
\pgfpathmoveto{\pgfqpoint{0.500000in}{-0.500000in}}%
\pgfpathlineto{\pgfqpoint{1.500000in}{0.500000in}}%
\pgfusepath{stroke}%
\end{pgfscope}%
}%
\pgfsys@transformshift{1.726264in}{4.587527in}%
\pgfsys@useobject{currentpattern}{}%
\pgfsys@transformshift{1in}{0in}%
\pgfsys@transformshift{-1in}{0in}%
\pgfsys@transformshift{0in}{1in}%
\end{pgfscope}%
\begin{pgfscope}%
\definecolor{textcolor}{rgb}{0.150000,0.150000,0.150000}%
\pgfsetstrokecolor{textcolor}%
\pgfsetfillcolor{textcolor}%
\pgftext[x=2.370708in,y=4.587527in,left,base]{\color{textcolor}\rmfamily\fontsize{38.000000}{45.600000}\selectfont DE}%
\end{pgfscope}%
\begin{pgfscope}%
\pgfsetbuttcap%
\pgfsetmiterjoin%
\definecolor{currentfill}{rgb}{0.347059,0.458824,0.641176}%
\pgfsetfillcolor{currentfill}%
\pgfsetlinewidth{0.803000pt}%
\definecolor{currentstroke}{rgb}{0.827451,0.827451,0.827451}%
\pgfsetstrokecolor{currentstroke}%
\pgfsetdash{}{0pt}%
\pgfpathmoveto{\pgfqpoint{3.380847in}{4.587527in}}%
\pgfpathlineto{\pgfqpoint{3.811403in}{4.587527in}}%
\pgfpathlineto{\pgfqpoint{3.811403in}{4.956972in}}%
\pgfpathlineto{\pgfqpoint{3.380847in}{4.956972in}}%
\pgfpathclose%
\pgfusepath{stroke,fill}%
\end{pgfscope}%
\begin{pgfscope}%
\pgfsetbuttcap%
\pgfsetmiterjoin%
\definecolor{currentfill}{rgb}{0.347059,0.458824,0.641176}%
\pgfsetfillcolor{currentfill}%
\pgfsetlinewidth{0.803000pt}%
\definecolor{currentstroke}{rgb}{0.827451,0.827451,0.827451}%
\pgfsetstrokecolor{currentstroke}%
\pgfsetdash{}{0pt}%
\pgfpathmoveto{\pgfqpoint{3.380847in}{4.587527in}}%
\pgfpathlineto{\pgfqpoint{3.811403in}{4.587527in}}%
\pgfpathlineto{\pgfqpoint{3.811403in}{4.956972in}}%
\pgfpathlineto{\pgfqpoint{3.380847in}{4.956972in}}%
\pgfpathclose%
\pgfusepath{clip}%
\pgfsys@defobject{currentpattern}{\pgfqpoint{0in}{0in}}{\pgfqpoint{1in}{1in}}{%
\begin{pgfscope}%
\pgfpathrectangle{\pgfqpoint{0in}{0in}}{\pgfqpoint{1in}{1in}}%
\pgfusepath{clip}%
\pgfpathmoveto{\pgfqpoint{-0.500000in}{0.500000in}}%
\pgfpathlineto{\pgfqpoint{0.500000in}{1.500000in}}%
\pgfpathmoveto{\pgfqpoint{-0.333333in}{0.333333in}}%
\pgfpathlineto{\pgfqpoint{0.666667in}{1.333333in}}%
\pgfpathmoveto{\pgfqpoint{-0.166667in}{0.166667in}}%
\pgfpathlineto{\pgfqpoint{0.833333in}{1.166667in}}%
\pgfpathmoveto{\pgfqpoint{0.000000in}{0.000000in}}%
\pgfpathlineto{\pgfqpoint{1.000000in}{1.000000in}}%
\pgfpathmoveto{\pgfqpoint{0.166667in}{-0.166667in}}%
\pgfpathlineto{\pgfqpoint{1.166667in}{0.833333in}}%
\pgfpathmoveto{\pgfqpoint{0.333333in}{-0.333333in}}%
\pgfpathlineto{\pgfqpoint{1.333333in}{0.666667in}}%
\pgfpathmoveto{\pgfqpoint{0.500000in}{-0.500000in}}%
\pgfpathlineto{\pgfqpoint{1.500000in}{0.500000in}}%
\pgfpathmoveto{\pgfqpoint{-0.500000in}{0.500000in}}%
\pgfpathlineto{\pgfqpoint{0.500000in}{-0.500000in}}%
\pgfpathmoveto{\pgfqpoint{-0.333333in}{0.666667in}}%
\pgfpathlineto{\pgfqpoint{0.666667in}{-0.333333in}}%
\pgfpathmoveto{\pgfqpoint{-0.166667in}{0.833333in}}%
\pgfpathlineto{\pgfqpoint{0.833333in}{-0.166667in}}%
\pgfpathmoveto{\pgfqpoint{0.000000in}{1.000000in}}%
\pgfpathlineto{\pgfqpoint{1.000000in}{0.000000in}}%
\pgfpathmoveto{\pgfqpoint{0.166667in}{1.166667in}}%
\pgfpathlineto{\pgfqpoint{1.166667in}{0.166667in}}%
\pgfpathmoveto{\pgfqpoint{0.333333in}{1.333333in}}%
\pgfpathlineto{\pgfqpoint{1.333333in}{0.333333in}}%
\pgfpathmoveto{\pgfqpoint{0.500000in}{1.500000in}}%
\pgfpathlineto{\pgfqpoint{1.500000in}{0.500000in}}%
\pgfusepath{stroke}%
\end{pgfscope}%
}%
\pgfsys@transformshift{3.380847in}{4.587527in}%
\pgfsys@useobject{currentpattern}{}%
\pgfsys@transformshift{1in}{0in}%
\pgfsys@transformshift{-1in}{0in}%
\pgfsys@transformshift{0in}{1in}%
\end{pgfscope}%
\begin{pgfscope}%
\definecolor{textcolor}{rgb}{0.150000,0.150000,0.150000}%
\pgfsetstrokecolor{textcolor}%
\pgfsetfillcolor{textcolor}%
\pgftext[x=4.025292in,y=4.587527in,left,base]{\color{textcolor}\rmfamily\fontsize{38.000000}{45.600000}\selectfont CO}%
\end{pgfscope}%
\end{pgfpicture}%
\makeatother%
\endgroup%
%
}
\caption{Workload 3}
\end{subfigure}
\begin{subfigure}[b]{\linewidth}
\centering
%\includegraphics[width=\linewidth]{../images/experiment-results/kaggle_home_credit/execution_time/different_workloads}
 \resizebox{\columnwidth}{!}{%
%% Creator: Matplotlib, PGF backend
%%
%% To include the figure in your LaTeX document, write
%%   \input{<filename>.pgf}
%%
%% Make sure the required packages are loaded in your preamble
%%   \usepackage{pgf}
%%
%% Figures using additional raster images can only be included by \input if
%% they are in the same directory as the main LaTeX file. For loading figures
%% from other directories you can use the `import` package
%%   \usepackage{import}
%% and then include the figures with
%%   \import{<path to file>}{<filename>.pgf}
%%
%% Matplotlib used the following preamble
%%   \usepackage{fontspec}
%%   \setmonofont{Andale Mono}
%%
\begingroup%
\makeatletter%
\begin{pgfpicture}%
\pgfpathrectangle{\pgfpointorigin}{\pgfqpoint{8.887494in}{3.778216in}}%
\pgfusepath{use as bounding box, clip}%
\begin{pgfscope}%
\pgfsetbuttcap%
\pgfsetmiterjoin%
\definecolor{currentfill}{rgb}{1.000000,1.000000,1.000000}%
\pgfsetfillcolor{currentfill}%
\pgfsetlinewidth{0.000000pt}%
\definecolor{currentstroke}{rgb}{1.000000,1.000000,1.000000}%
\pgfsetstrokecolor{currentstroke}%
\pgfsetdash{}{0pt}%
\pgfpathmoveto{\pgfqpoint{0.000000in}{0.000000in}}%
\pgfpathlineto{\pgfqpoint{8.887494in}{0.000000in}}%
\pgfpathlineto{\pgfqpoint{8.887494in}{3.778216in}}%
\pgfpathlineto{\pgfqpoint{0.000000in}{3.778216in}}%
\pgfpathclose%
\pgfusepath{fill}%
\end{pgfscope}%
\begin{pgfscope}%
\pgfsetbuttcap%
\pgfsetmiterjoin%
\definecolor{currentfill}{rgb}{1.000000,1.000000,1.000000}%
\pgfsetfillcolor{currentfill}%
\pgfsetlinewidth{0.000000pt}%
\definecolor{currentstroke}{rgb}{0.000000,0.000000,0.000000}%
\pgfsetstrokecolor{currentstroke}%
\pgfsetstrokeopacity{0.000000}%
\pgfsetdash{}{0pt}%
\pgfpathmoveto{\pgfqpoint{0.970827in}{0.917049in}}%
\pgfpathlineto{\pgfqpoint{8.720828in}{0.917049in}}%
\pgfpathlineto{\pgfqpoint{8.720828in}{3.182049in}}%
\pgfpathlineto{\pgfqpoint{0.970827in}{3.182049in}}%
\pgfpathclose%
\pgfusepath{fill}%
\end{pgfscope}%
\begin{pgfscope}%
\pgfpathrectangle{\pgfqpoint{0.970827in}{0.917049in}}{\pgfqpoint{7.750000in}{2.265000in}} %
\pgfusepath{clip}%
\pgfsetroundcap%
\pgfsetroundjoin%
\pgfsetlinewidth{0.803000pt}%
\definecolor{currentstroke}{rgb}{0.800000,0.800000,0.800000}%
\pgfsetstrokecolor{currentstroke}%
\pgfsetdash{}{0pt}%
\pgfpathmoveto{\pgfqpoint{1.323100in}{0.917049in}}%
\pgfpathlineto{\pgfqpoint{1.323100in}{3.182049in}}%
\pgfusepath{stroke}%
\end{pgfscope}%
\begin{pgfscope}%
\definecolor{textcolor}{rgb}{0.150000,0.150000,0.150000}%
\pgfsetstrokecolor{textcolor}%
\pgfsetfillcolor{textcolor}%
\pgftext[x=1.323100in,y=0.753161in,,top]{\color{textcolor}\rmfamily\fontsize{26.400000}{31.680000}\selectfont 1}%
\end{pgfscope}%
\begin{pgfscope}%
\pgfpathrectangle{\pgfqpoint{0.970827in}{0.917049in}}{\pgfqpoint{7.750000in}{2.265000in}} %
\pgfusepath{clip}%
\pgfsetroundcap%
\pgfsetroundjoin%
\pgfsetlinewidth{0.803000pt}%
\definecolor{currentstroke}{rgb}{0.800000,0.800000,0.800000}%
\pgfsetstrokecolor{currentstroke}%
\pgfsetdash{}{0pt}%
\pgfpathmoveto{\pgfqpoint{2.329594in}{0.917049in}}%
\pgfpathlineto{\pgfqpoint{2.329594in}{3.182049in}}%
\pgfusepath{stroke}%
\end{pgfscope}%
\begin{pgfscope}%
\definecolor{textcolor}{rgb}{0.150000,0.150000,0.150000}%
\pgfsetstrokecolor{textcolor}%
\pgfsetfillcolor{textcolor}%
\pgftext[x=2.329594in,y=0.753161in,,top]{\color{textcolor}\rmfamily\fontsize{26.400000}{31.680000}\selectfont 2}%
\end{pgfscope}%
\begin{pgfscope}%
\pgfpathrectangle{\pgfqpoint{0.970827in}{0.917049in}}{\pgfqpoint{7.750000in}{2.265000in}} %
\pgfusepath{clip}%
\pgfsetroundcap%
\pgfsetroundjoin%
\pgfsetlinewidth{0.803000pt}%
\definecolor{currentstroke}{rgb}{0.800000,0.800000,0.800000}%
\pgfsetstrokecolor{currentstroke}%
\pgfsetdash{}{0pt}%
\pgfpathmoveto{\pgfqpoint{3.336087in}{0.917049in}}%
\pgfpathlineto{\pgfqpoint{3.336087in}{3.182049in}}%
\pgfusepath{stroke}%
\end{pgfscope}%
\begin{pgfscope}%
\definecolor{textcolor}{rgb}{0.150000,0.150000,0.150000}%
\pgfsetstrokecolor{textcolor}%
\pgfsetfillcolor{textcolor}%
\pgftext[x=3.336087in,y=0.753161in,,top]{\color{textcolor}\rmfamily\fontsize{26.400000}{31.680000}\selectfont 3}%
\end{pgfscope}%
\begin{pgfscope}%
\pgfpathrectangle{\pgfqpoint{0.970827in}{0.917049in}}{\pgfqpoint{7.750000in}{2.265000in}} %
\pgfusepath{clip}%
\pgfsetroundcap%
\pgfsetroundjoin%
\pgfsetlinewidth{0.803000pt}%
\definecolor{currentstroke}{rgb}{0.800000,0.800000,0.800000}%
\pgfsetstrokecolor{currentstroke}%
\pgfsetdash{}{0pt}%
\pgfpathmoveto{\pgfqpoint{4.342581in}{0.917049in}}%
\pgfpathlineto{\pgfqpoint{4.342581in}{3.182049in}}%
\pgfusepath{stroke}%
\end{pgfscope}%
\begin{pgfscope}%
\definecolor{textcolor}{rgb}{0.150000,0.150000,0.150000}%
\pgfsetstrokecolor{textcolor}%
\pgfsetfillcolor{textcolor}%
\pgftext[x=4.342581in,y=0.753161in,,top]{\color{textcolor}\rmfamily\fontsize{26.400000}{31.680000}\selectfont 4}%
\end{pgfscope}%
\begin{pgfscope}%
\pgfpathrectangle{\pgfqpoint{0.970827in}{0.917049in}}{\pgfqpoint{7.750000in}{2.265000in}} %
\pgfusepath{clip}%
\pgfsetroundcap%
\pgfsetroundjoin%
\pgfsetlinewidth{0.803000pt}%
\definecolor{currentstroke}{rgb}{0.800000,0.800000,0.800000}%
\pgfsetstrokecolor{currentstroke}%
\pgfsetdash{}{0pt}%
\pgfpathmoveto{\pgfqpoint{5.349074in}{0.917049in}}%
\pgfpathlineto{\pgfqpoint{5.349074in}{3.182049in}}%
\pgfusepath{stroke}%
\end{pgfscope}%
\begin{pgfscope}%
\definecolor{textcolor}{rgb}{0.150000,0.150000,0.150000}%
\pgfsetstrokecolor{textcolor}%
\pgfsetfillcolor{textcolor}%
\pgftext[x=5.349074in,y=0.753161in,,top]{\color{textcolor}\rmfamily\fontsize{26.400000}{31.680000}\selectfont 5}%
\end{pgfscope}%
\begin{pgfscope}%
\pgfpathrectangle{\pgfqpoint{0.970827in}{0.917049in}}{\pgfqpoint{7.750000in}{2.265000in}} %
\pgfusepath{clip}%
\pgfsetroundcap%
\pgfsetroundjoin%
\pgfsetlinewidth{0.803000pt}%
\definecolor{currentstroke}{rgb}{0.800000,0.800000,0.800000}%
\pgfsetstrokecolor{currentstroke}%
\pgfsetdash{}{0pt}%
\pgfpathmoveto{\pgfqpoint{6.355568in}{0.917049in}}%
\pgfpathlineto{\pgfqpoint{6.355568in}{3.182049in}}%
\pgfusepath{stroke}%
\end{pgfscope}%
\begin{pgfscope}%
\definecolor{textcolor}{rgb}{0.150000,0.150000,0.150000}%
\pgfsetstrokecolor{textcolor}%
\pgfsetfillcolor{textcolor}%
\pgftext[x=6.355568in,y=0.753161in,,top]{\color{textcolor}\rmfamily\fontsize{26.400000}{31.680000}\selectfont 6}%
\end{pgfscope}%
\begin{pgfscope}%
\pgfpathrectangle{\pgfqpoint{0.970827in}{0.917049in}}{\pgfqpoint{7.750000in}{2.265000in}} %
\pgfusepath{clip}%
\pgfsetroundcap%
\pgfsetroundjoin%
\pgfsetlinewidth{0.803000pt}%
\definecolor{currentstroke}{rgb}{0.800000,0.800000,0.800000}%
\pgfsetstrokecolor{currentstroke}%
\pgfsetdash{}{0pt}%
\pgfpathmoveto{\pgfqpoint{7.362061in}{0.917049in}}%
\pgfpathlineto{\pgfqpoint{7.362061in}{3.182049in}}%
\pgfusepath{stroke}%
\end{pgfscope}%
\begin{pgfscope}%
\definecolor{textcolor}{rgb}{0.150000,0.150000,0.150000}%
\pgfsetstrokecolor{textcolor}%
\pgfsetfillcolor{textcolor}%
\pgftext[x=7.362061in,y=0.753161in,,top]{\color{textcolor}\rmfamily\fontsize{26.400000}{31.680000}\selectfont 7}%
\end{pgfscope}%
\begin{pgfscope}%
\pgfpathrectangle{\pgfqpoint{0.970827in}{0.917049in}}{\pgfqpoint{7.750000in}{2.265000in}} %
\pgfusepath{clip}%
\pgfsetroundcap%
\pgfsetroundjoin%
\pgfsetlinewidth{0.803000pt}%
\definecolor{currentstroke}{rgb}{0.800000,0.800000,0.800000}%
\pgfsetstrokecolor{currentstroke}%
\pgfsetdash{}{0pt}%
\pgfpathmoveto{\pgfqpoint{8.368555in}{0.917049in}}%
\pgfpathlineto{\pgfqpoint{8.368555in}{3.182049in}}%
\pgfusepath{stroke}%
\end{pgfscope}%
\begin{pgfscope}%
\definecolor{textcolor}{rgb}{0.150000,0.150000,0.150000}%
\pgfsetstrokecolor{textcolor}%
\pgfsetfillcolor{textcolor}%
\pgftext[x=8.368555in,y=0.753161in,,top]{\color{textcolor}\rmfamily\fontsize{26.400000}{31.680000}\selectfont 8}%
\end{pgfscope}%
\begin{pgfscope}%
\definecolor{textcolor}{rgb}{0.150000,0.150000,0.150000}%
\pgfsetstrokecolor{textcolor}%
\pgfsetfillcolor{textcolor}%
\pgftext[x=4.845827in,y=0.371639in,,top]{\color{textcolor}\rmfamily\fontsize{22.000000}{26.400000}\selectfont Workload}%
\end{pgfscope}%
\begin{pgfscope}%
\pgfpathrectangle{\pgfqpoint{0.970827in}{0.917049in}}{\pgfqpoint{7.750000in}{2.265000in}} %
\pgfusepath{clip}%
\pgfsetroundcap%
\pgfsetroundjoin%
\pgfsetlinewidth{0.803000pt}%
\definecolor{currentstroke}{rgb}{0.800000,0.800000,0.800000}%
\pgfsetstrokecolor{currentstroke}%
\pgfsetdash{}{0pt}%
\pgfpathmoveto{\pgfqpoint{0.970827in}{0.917049in}}%
\pgfpathlineto{\pgfqpoint{8.720828in}{0.917049in}}%
\pgfusepath{stroke}%
\end{pgfscope}%
\begin{pgfscope}%
\definecolor{textcolor}{rgb}{0.150000,0.150000,0.150000}%
\pgfsetstrokecolor{textcolor}%
\pgfsetfillcolor{textcolor}%
\pgftext[x=0.638639in,y=0.789816in,left,base]{\color{textcolor}\rmfamily\fontsize{26.400000}{31.680000}\selectfont 0}%
\end{pgfscope}%
\begin{pgfscope}%
\pgfpathrectangle{\pgfqpoint{0.970827in}{0.917049in}}{\pgfqpoint{7.750000in}{2.265000in}} %
\pgfusepath{clip}%
\pgfsetroundcap%
\pgfsetroundjoin%
\pgfsetlinewidth{0.803000pt}%
\definecolor{currentstroke}{rgb}{0.800000,0.800000,0.800000}%
\pgfsetstrokecolor{currentstroke}%
\pgfsetdash{}{0pt}%
\pgfpathmoveto{\pgfqpoint{0.970827in}{2.049549in}}%
\pgfpathlineto{\pgfqpoint{8.720828in}{2.049549in}}%
\pgfusepath{stroke}%
\end{pgfscope}%
\begin{pgfscope}%
\definecolor{textcolor}{rgb}{0.150000,0.150000,0.150000}%
\pgfsetstrokecolor{textcolor}%
\pgfsetfillcolor{textcolor}%
\pgftext[x=0.460805in,y=1.922316in,left,base]{\color{textcolor}\rmfamily\fontsize{26.400000}{31.680000}\selectfont 1k}%
\end{pgfscope}%
\begin{pgfscope}%
\pgfpathrectangle{\pgfqpoint{0.970827in}{0.917049in}}{\pgfqpoint{7.750000in}{2.265000in}} %
\pgfusepath{clip}%
\pgfsetroundcap%
\pgfsetroundjoin%
\pgfsetlinewidth{0.803000pt}%
\definecolor{currentstroke}{rgb}{0.800000,0.800000,0.800000}%
\pgfsetstrokecolor{currentstroke}%
\pgfsetdash{}{0pt}%
\pgfpathmoveto{\pgfqpoint{0.970827in}{3.182049in}}%
\pgfpathlineto{\pgfqpoint{8.720828in}{3.182049in}}%
\pgfusepath{stroke}%
\end{pgfscope}%
\begin{pgfscope}%
\definecolor{textcolor}{rgb}{0.150000,0.150000,0.150000}%
\pgfsetstrokecolor{textcolor}%
\pgfsetfillcolor{textcolor}%
\pgftext[x=0.460805in,y=3.054816in,left,base]{\color{textcolor}\rmfamily\fontsize{26.400000}{31.680000}\selectfont 2k}%
\end{pgfscope}%
\begin{pgfscope}%
\definecolor{textcolor}{rgb}{0.150000,0.150000,0.150000}%
\pgfsetstrokecolor{textcolor}%
\pgfsetfillcolor{textcolor}%
\pgftext[x=0.405250in,y=2.049549in,,bottom,rotate=90.000000]{\color{textcolor}\rmfamily\fontsize{22.000000}{26.400000}\selectfont Cumulative Run Time (s)}%
\end{pgfscope}%
\begin{pgfscope}%
\pgfpathrectangle{\pgfqpoint{0.970827in}{0.917049in}}{\pgfqpoint{7.750000in}{2.265000in}} %
\pgfusepath{clip}%
\pgfsetbuttcap%
\pgfsetroundjoin%
\definecolor{currentfill}{rgb}{0.298039,0.447059,0.690196}%
\pgfsetfillcolor{currentfill}%
\pgfsetfillopacity{0.200000}%
\pgfsetlinewidth{0.803000pt}%
\definecolor{currentstroke}{rgb}{0.298039,0.447059,0.690196}%
\pgfsetstrokecolor{currentstroke}%
\pgfsetstrokeopacity{0.200000}%
\pgfsetdash{}{0pt}%
\pgfpathmoveto{\pgfqpoint{1.323100in}{1.144789in}}%
\pgfpathlineto{\pgfqpoint{1.323100in}{1.142805in}}%
\pgfpathlineto{\pgfqpoint{2.329594in}{1.337373in}}%
\pgfpathlineto{\pgfqpoint{3.336087in}{1.662518in}}%
\pgfpathlineto{\pgfqpoint{4.342581in}{1.727351in}}%
\pgfpathlineto{\pgfqpoint{5.349074in}{1.777113in}}%
\pgfpathlineto{\pgfqpoint{6.355568in}{1.803907in}}%
\pgfpathlineto{\pgfqpoint{7.362061in}{1.828708in}}%
\pgfpathlineto{\pgfqpoint{8.368555in}{1.899170in}}%
\pgfpathlineto{\pgfqpoint{8.368555in}{1.926724in}}%
\pgfpathlineto{\pgfqpoint{8.368555in}{1.926724in}}%
\pgfpathlineto{\pgfqpoint{7.362061in}{1.856800in}}%
\pgfpathlineto{\pgfqpoint{6.355568in}{1.809984in}}%
\pgfpathlineto{\pgfqpoint{5.349074in}{1.784201in}}%
\pgfpathlineto{\pgfqpoint{4.342581in}{1.730547in}}%
\pgfpathlineto{\pgfqpoint{3.336087in}{1.667423in}}%
\pgfpathlineto{\pgfqpoint{2.329594in}{1.338300in}}%
\pgfpathlineto{\pgfqpoint{1.323100in}{1.144789in}}%
\pgfpathclose%
\pgfusepath{stroke,fill}%
\end{pgfscope}%
\begin{pgfscope}%
\pgfpathrectangle{\pgfqpoint{0.970827in}{0.917049in}}{\pgfqpoint{7.750000in}{2.265000in}} %
\pgfusepath{clip}%
\pgfsetbuttcap%
\pgfsetroundjoin%
\definecolor{currentfill}{rgb}{0.866667,0.517647,0.321569}%
\pgfsetfillcolor{currentfill}%
\pgfsetfillopacity{0.200000}%
\pgfsetlinewidth{0.803000pt}%
\definecolor{currentstroke}{rgb}{0.866667,0.517647,0.321569}%
\pgfsetstrokecolor{currentstroke}%
\pgfsetstrokeopacity{0.200000}%
\pgfsetdash{}{0pt}%
\pgfpathmoveto{\pgfqpoint{1.323100in}{1.134397in}}%
\pgfpathlineto{\pgfqpoint{1.323100in}{1.132277in}}%
\pgfpathlineto{\pgfqpoint{2.329594in}{1.359844in}}%
\pgfpathlineto{\pgfqpoint{3.336087in}{1.764130in}}%
\pgfpathlineto{\pgfqpoint{4.342581in}{1.943740in}}%
\pgfpathlineto{\pgfqpoint{5.349074in}{2.067041in}}%
\pgfpathlineto{\pgfqpoint{6.355568in}{2.245476in}}%
\pgfpathlineto{\pgfqpoint{7.362061in}{2.638406in}}%
\pgfpathlineto{\pgfqpoint{8.368555in}{2.930629in}}%
\pgfpathlineto{\pgfqpoint{8.368555in}{2.947915in}}%
\pgfpathlineto{\pgfqpoint{8.368555in}{2.947915in}}%
\pgfpathlineto{\pgfqpoint{7.362061in}{2.657878in}}%
\pgfpathlineto{\pgfqpoint{6.355568in}{2.262720in}}%
\pgfpathlineto{\pgfqpoint{5.349074in}{2.084099in}}%
\pgfpathlineto{\pgfqpoint{4.342581in}{1.953815in}}%
\pgfpathlineto{\pgfqpoint{3.336087in}{1.776360in}}%
\pgfpathlineto{\pgfqpoint{2.329594in}{1.364670in}}%
\pgfpathlineto{\pgfqpoint{1.323100in}{1.134397in}}%
\pgfpathclose%
\pgfusepath{stroke,fill}%
\end{pgfscope}%
\begin{pgfscope}%
\pgfpathrectangle{\pgfqpoint{0.970827in}{0.917049in}}{\pgfqpoint{7.750000in}{2.265000in}} %
\pgfusepath{clip}%
\pgfsetbuttcap%
\pgfsetroundjoin%
\pgfsetlinewidth{3.011250pt}%
\definecolor{currentstroke}{rgb}{0.298039,0.447059,0.690196}%
\pgfsetstrokecolor{currentstroke}%
\pgfsetdash{{3.000000pt}{0.000000pt}}{0.000000pt}%
\pgfpathmoveto{\pgfqpoint{1.323100in}{1.143755in}}%
\pgfpathlineto{\pgfqpoint{2.329594in}{1.337852in}}%
\pgfpathlineto{\pgfqpoint{3.336087in}{1.665208in}}%
\pgfpathlineto{\pgfqpoint{4.342581in}{1.728975in}}%
\pgfpathlineto{\pgfqpoint{5.349074in}{1.781277in}}%
\pgfpathlineto{\pgfqpoint{6.355568in}{1.807699in}}%
\pgfpathlineto{\pgfqpoint{7.362061in}{1.841603in}}%
\pgfpathlineto{\pgfqpoint{8.368555in}{1.912593in}}%
\pgfusepath{stroke}%
\end{pgfscope}%
\begin{pgfscope}%
\pgfpathrectangle{\pgfqpoint{0.970827in}{0.917049in}}{\pgfqpoint{7.750000in}{2.265000in}} %
\pgfusepath{clip}%
\pgfsetbuttcap%
\pgfsetroundjoin%
\definecolor{currentfill}{rgb}{0.298039,0.447059,0.690196}%
\pgfsetfillcolor{currentfill}%
\pgfsetlinewidth{0.752812pt}%
\definecolor{currentstroke}{rgb}{1.000000,1.000000,1.000000}%
\pgfsetstrokecolor{currentstroke}%
\pgfsetdash{}{0pt}%
\pgfsys@defobject{currentmarker}{\pgfqpoint{-0.104167in}{-0.104167in}}{\pgfqpoint{0.104167in}{0.104167in}}{%
\pgfpathmoveto{\pgfqpoint{0.000000in}{-0.104167in}}%
\pgfpathcurveto{\pgfqpoint{0.027625in}{-0.104167in}}{\pgfqpoint{0.054123in}{-0.093191in}}{\pgfqpoint{0.073657in}{-0.073657in}}%
\pgfpathcurveto{\pgfqpoint{0.093191in}{-0.054123in}}{\pgfqpoint{0.104167in}{-0.027625in}}{\pgfqpoint{0.104167in}{0.000000in}}%
\pgfpathcurveto{\pgfqpoint{0.104167in}{0.027625in}}{\pgfqpoint{0.093191in}{0.054123in}}{\pgfqpoint{0.073657in}{0.073657in}}%
\pgfpathcurveto{\pgfqpoint{0.054123in}{0.093191in}}{\pgfqpoint{0.027625in}{0.104167in}}{\pgfqpoint{0.000000in}{0.104167in}}%
\pgfpathcurveto{\pgfqpoint{-0.027625in}{0.104167in}}{\pgfqpoint{-0.054123in}{0.093191in}}{\pgfqpoint{-0.073657in}{0.073657in}}%
\pgfpathcurveto{\pgfqpoint{-0.093191in}{0.054123in}}{\pgfqpoint{-0.104167in}{0.027625in}}{\pgfqpoint{-0.104167in}{0.000000in}}%
\pgfpathcurveto{\pgfqpoint{-0.104167in}{-0.027625in}}{\pgfqpoint{-0.093191in}{-0.054123in}}{\pgfqpoint{-0.073657in}{-0.073657in}}%
\pgfpathcurveto{\pgfqpoint{-0.054123in}{-0.093191in}}{\pgfqpoint{-0.027625in}{-0.104167in}}{\pgfqpoint{0.000000in}{-0.104167in}}%
\pgfpathclose%
\pgfusepath{stroke,fill}%
}%
\begin{pgfscope}%
\pgfsys@transformshift{1.323100in}{1.143755in}%
\pgfsys@useobject{currentmarker}{}%
\end{pgfscope}%
\begin{pgfscope}%
\pgfsys@transformshift{2.329594in}{1.337852in}%
\pgfsys@useobject{currentmarker}{}%
\end{pgfscope}%
\begin{pgfscope}%
\pgfsys@transformshift{3.336087in}{1.665208in}%
\pgfsys@useobject{currentmarker}{}%
\end{pgfscope}%
\begin{pgfscope}%
\pgfsys@transformshift{4.342581in}{1.728975in}%
\pgfsys@useobject{currentmarker}{}%
\end{pgfscope}%
\begin{pgfscope}%
\pgfsys@transformshift{5.349074in}{1.781277in}%
\pgfsys@useobject{currentmarker}{}%
\end{pgfscope}%
\begin{pgfscope}%
\pgfsys@transformshift{6.355568in}{1.807699in}%
\pgfsys@useobject{currentmarker}{}%
\end{pgfscope}%
\begin{pgfscope}%
\pgfsys@transformshift{7.362061in}{1.841603in}%
\pgfsys@useobject{currentmarker}{}%
\end{pgfscope}%
\begin{pgfscope}%
\pgfsys@transformshift{8.368555in}{1.912593in}%
\pgfsys@useobject{currentmarker}{}%
\end{pgfscope}%
\end{pgfscope}%
\begin{pgfscope}%
\pgfpathrectangle{\pgfqpoint{0.970827in}{0.917049in}}{\pgfqpoint{7.750000in}{2.265000in}} %
\pgfusepath{clip}%
\pgfsetbuttcap%
\pgfsetroundjoin%
\pgfsetlinewidth{3.011250pt}%
\definecolor{currentstroke}{rgb}{0.866667,0.517647,0.321569}%
\pgfsetstrokecolor{currentstroke}%
\pgfsetdash{{3.000000pt}{3.000000pt}}{0.000000pt}%
\pgfpathmoveto{\pgfqpoint{1.323100in}{1.133511in}}%
\pgfpathlineto{\pgfqpoint{2.329594in}{1.362479in}}%
\pgfpathlineto{\pgfqpoint{3.336087in}{1.769893in}}%
\pgfpathlineto{\pgfqpoint{4.342581in}{1.949548in}}%
\pgfpathlineto{\pgfqpoint{5.349074in}{2.075956in}}%
\pgfpathlineto{\pgfqpoint{6.355568in}{2.255243in}}%
\pgfpathlineto{\pgfqpoint{7.362061in}{2.651155in}}%
\pgfpathlineto{\pgfqpoint{8.368555in}{2.941970in}}%
\pgfusepath{stroke}%
\end{pgfscope}%
\begin{pgfscope}%
\pgfpathrectangle{\pgfqpoint{0.970827in}{0.917049in}}{\pgfqpoint{7.750000in}{2.265000in}} %
\pgfusepath{clip}%
\pgfsetbuttcap%
\pgfsetmiterjoin%
\definecolor{currentfill}{rgb}{0.866667,0.517647,0.321569}%
\pgfsetfillcolor{currentfill}%
\pgfsetlinewidth{0.752812pt}%
\definecolor{currentstroke}{rgb}{1.000000,1.000000,1.000000}%
\pgfsetstrokecolor{currentstroke}%
\pgfsetdash{}{0pt}%
\pgfsys@defobject{currentmarker}{\pgfqpoint{-0.104167in}{-0.104167in}}{\pgfqpoint{0.104167in}{0.104167in}}{%
\pgfpathmoveto{\pgfqpoint{0.000000in}{0.104167in}}%
\pgfpathlineto{\pgfqpoint{-0.104167in}{-0.104167in}}%
\pgfpathlineto{\pgfqpoint{0.104167in}{-0.104167in}}%
\pgfpathclose%
\pgfusepath{stroke,fill}%
}%
\begin{pgfscope}%
\pgfsys@transformshift{1.323100in}{1.133511in}%
\pgfsys@useobject{currentmarker}{}%
\end{pgfscope}%
\begin{pgfscope}%
\pgfsys@transformshift{2.329594in}{1.362479in}%
\pgfsys@useobject{currentmarker}{}%
\end{pgfscope}%
\begin{pgfscope}%
\pgfsys@transformshift{3.336087in}{1.769893in}%
\pgfsys@useobject{currentmarker}{}%
\end{pgfscope}%
\begin{pgfscope}%
\pgfsys@transformshift{4.342581in}{1.949548in}%
\pgfsys@useobject{currentmarker}{}%
\end{pgfscope}%
\begin{pgfscope}%
\pgfsys@transformshift{5.349074in}{2.075956in}%
\pgfsys@useobject{currentmarker}{}%
\end{pgfscope}%
\begin{pgfscope}%
\pgfsys@transformshift{6.355568in}{2.255243in}%
\pgfsys@useobject{currentmarker}{}%
\end{pgfscope}%
\begin{pgfscope}%
\pgfsys@transformshift{7.362061in}{2.651155in}%
\pgfsys@useobject{currentmarker}{}%
\end{pgfscope}%
\begin{pgfscope}%
\pgfsys@transformshift{8.368555in}{2.941970in}%
\pgfsys@useobject{currentmarker}{}%
\end{pgfscope}%
\end{pgfscope}%
\begin{pgfscope}%
\pgfsetrectcap%
\pgfsetmiterjoin%
\pgfsetlinewidth{1.003750pt}%
\definecolor{currentstroke}{rgb}{0.800000,0.800000,0.800000}%
\pgfsetstrokecolor{currentstroke}%
\pgfsetdash{}{0pt}%
\pgfpathmoveto{\pgfqpoint{0.970827in}{0.917049in}}%
\pgfpathlineto{\pgfqpoint{0.970827in}{3.182049in}}%
\pgfusepath{stroke}%
\end{pgfscope}%
\begin{pgfscope}%
\pgfsetrectcap%
\pgfsetmiterjoin%
\pgfsetlinewidth{1.003750pt}%
\definecolor{currentstroke}{rgb}{0.800000,0.800000,0.800000}%
\pgfsetstrokecolor{currentstroke}%
\pgfsetdash{}{0pt}%
\pgfpathmoveto{\pgfqpoint{8.720828in}{0.917049in}}%
\pgfpathlineto{\pgfqpoint{8.720828in}{3.182049in}}%
\pgfusepath{stroke}%
\end{pgfscope}%
\begin{pgfscope}%
\pgfsetrectcap%
\pgfsetmiterjoin%
\pgfsetlinewidth{1.003750pt}%
\definecolor{currentstroke}{rgb}{0.800000,0.800000,0.800000}%
\pgfsetstrokecolor{currentstroke}%
\pgfsetdash{}{0pt}%
\pgfpathmoveto{\pgfqpoint{0.970827in}{0.917049in}}%
\pgfpathlineto{\pgfqpoint{8.720828in}{0.917049in}}%
\pgfusepath{stroke}%
\end{pgfscope}%
\begin{pgfscope}%
\pgfsetrectcap%
\pgfsetmiterjoin%
\pgfsetlinewidth{1.003750pt}%
\definecolor{currentstroke}{rgb}{0.800000,0.800000,0.800000}%
\pgfsetstrokecolor{currentstroke}%
\pgfsetdash{}{0pt}%
\pgfpathmoveto{\pgfqpoint{0.970827in}{3.182049in}}%
\pgfpathlineto{\pgfqpoint{8.720828in}{3.182049in}}%
\pgfusepath{stroke}%
\end{pgfscope}%
\begin{pgfscope}%
\pgfsetroundcap%
\pgfsetroundjoin%
\pgfsetlinewidth{3.011250pt}%
\definecolor{currentstroke}{rgb}{1.000000,1.000000,1.000000}%
\pgfsetstrokecolor{currentstroke}%
\pgfsetdash{}{0pt}%
\pgfpathmoveto{\pgfqpoint{1.418861in}{3.403216in}}%
\pgfpathlineto{\pgfqpoint{2.152194in}{3.403216in}}%
\pgfusepath{stroke}%
\end{pgfscope}%
\begin{pgfscope}%
\pgfsetbuttcap%
\pgfsetroundjoin%
\pgfsetlinewidth{3.011250pt}%
\definecolor{currentstroke}{rgb}{0.298039,0.447059,0.690196}%
\pgfsetstrokecolor{currentstroke}%
\pgfsetdash{{3.000000pt}{0.000000pt}}{0.000000pt}%
\pgfpathmoveto{\pgfqpoint{1.418861in}{2.891716in}}%
\pgfpathlineto{\pgfqpoint{2.152194in}{2.891716in}}%
\pgfusepath{stroke}%
\end{pgfscope}%
\begin{pgfscope}%
\pgfsetbuttcap%
\pgfsetroundjoin%
\definecolor{currentfill}{rgb}{0.298039,0.447059,0.690196}%
\pgfsetfillcolor{currentfill}%
\pgfsetlinewidth{0.000000pt}%
\definecolor{currentstroke}{rgb}{0.298039,0.447059,0.690196}%
\pgfsetstrokecolor{currentstroke}%
\pgfsetdash{}{0pt}%
\pgfsys@defobject{currentmarker}{\pgfqpoint{-0.104167in}{-0.104167in}}{\pgfqpoint{0.104167in}{0.104167in}}{%
\pgfpathmoveto{\pgfqpoint{0.000000in}{-0.104167in}}%
\pgfpathcurveto{\pgfqpoint{0.027625in}{-0.104167in}}{\pgfqpoint{0.054123in}{-0.093191in}}{\pgfqpoint{0.073657in}{-0.073657in}}%
\pgfpathcurveto{\pgfqpoint{0.093191in}{-0.054123in}}{\pgfqpoint{0.104167in}{-0.027625in}}{\pgfqpoint{0.104167in}{0.000000in}}%
\pgfpathcurveto{\pgfqpoint{0.104167in}{0.027625in}}{\pgfqpoint{0.093191in}{0.054123in}}{\pgfqpoint{0.073657in}{0.073657in}}%
\pgfpathcurveto{\pgfqpoint{0.054123in}{0.093191in}}{\pgfqpoint{0.027625in}{0.104167in}}{\pgfqpoint{0.000000in}{0.104167in}}%
\pgfpathcurveto{\pgfqpoint{-0.027625in}{0.104167in}}{\pgfqpoint{-0.054123in}{0.093191in}}{\pgfqpoint{-0.073657in}{0.073657in}}%
\pgfpathcurveto{\pgfqpoint{-0.093191in}{0.054123in}}{\pgfqpoint{-0.104167in}{0.027625in}}{\pgfqpoint{-0.104167in}{0.000000in}}%
\pgfpathcurveto{\pgfqpoint{-0.104167in}{-0.027625in}}{\pgfqpoint{-0.093191in}{-0.054123in}}{\pgfqpoint{-0.073657in}{-0.073657in}}%
\pgfpathcurveto{\pgfqpoint{-0.054123in}{-0.093191in}}{\pgfqpoint{-0.027625in}{-0.104167in}}{\pgfqpoint{0.000000in}{-0.104167in}}%
\pgfpathclose%
\pgfusepath{fill}%
}%
\begin{pgfscope}%
\pgfsys@transformshift{1.785527in}{2.891716in}%
\pgfsys@useobject{currentmarker}{}%
\end{pgfscope}%
\end{pgfscope}%
\begin{pgfscope}%
\definecolor{textcolor}{rgb}{0.150000,0.150000,0.150000}%
\pgfsetstrokecolor{textcolor}%
\pgfsetfillcolor{textcolor}%
\pgftext[x=2.445527in,y=2.763383in,left,base]{\color{textcolor}\rmfamily\fontsize{26.400000}{31.680000}\selectfont CO}%
\end{pgfscope}%
\begin{pgfscope}%
\pgfsetbuttcap%
\pgfsetroundjoin%
\pgfsetlinewidth{3.011250pt}%
\definecolor{currentstroke}{rgb}{0.866667,0.517647,0.321569}%
\pgfsetstrokecolor{currentstroke}%
\pgfsetdash{{3.000000pt}{3.000000pt}}{0.000000pt}%
\pgfpathmoveto{\pgfqpoint{1.418861in}{2.380217in}}%
\pgfpathlineto{\pgfqpoint{2.152194in}{2.380217in}}%
\pgfusepath{stroke}%
\end{pgfscope}%
\begin{pgfscope}%
\pgfsetbuttcap%
\pgfsetmiterjoin%
\definecolor{currentfill}{rgb}{0.866667,0.517647,0.321569}%
\pgfsetfillcolor{currentfill}%
\pgfsetlinewidth{0.000000pt}%
\definecolor{currentstroke}{rgb}{0.866667,0.517647,0.321569}%
\pgfsetstrokecolor{currentstroke}%
\pgfsetdash{}{0pt}%
\pgfsys@defobject{currentmarker}{\pgfqpoint{-0.104167in}{-0.104167in}}{\pgfqpoint{0.104167in}{0.104167in}}{%
\pgfpathmoveto{\pgfqpoint{0.000000in}{0.104167in}}%
\pgfpathlineto{\pgfqpoint{-0.104167in}{-0.104167in}}%
\pgfpathlineto{\pgfqpoint{0.104167in}{-0.104167in}}%
\pgfpathclose%
\pgfusepath{fill}%
}%
\begin{pgfscope}%
\pgfsys@transformshift{1.785527in}{2.380217in}%
\pgfsys@useobject{currentmarker}{}%
\end{pgfscope}%
\end{pgfscope}%
\begin{pgfscope}%
\definecolor{textcolor}{rgb}{0.150000,0.150000,0.150000}%
\pgfsetstrokecolor{textcolor}%
\pgfsetfillcolor{textcolor}%
\pgftext[x=2.445527in,y=2.251883in,left,base]{\color{textcolor}\rmfamily\fontsize{26.400000}{31.680000}\selectfont KG}%
\end{pgfscope}%
\end{pgfpicture}%
\makeatother%
\endgroup%
%
}
\caption{Execution of several workloads one by one}
\end{subfigure}
\caption{Execution of Kaggle workloads (materialization budget = 32 GB). KG = Kaggle's infrastructure (our baseline), CO = Collaborative Optimizer.}
\label{exp-reuse-kaggle-same-workload}
\end{figure}

Figure \ref{exp-reuse-kaggle-same-workload} (d) shows the cumulative run-time of executing the 8 workloads of Table \ref{kaggle-workload} one after another.
This corresponds to a real scenario, where after some scripts of a collaborative environment gain popularity, i.e., Workloads 1-3, other users modify and try to improve them, i.e., Workloads 4-8.
The figure shows that even running every workload once decreases the cumulative run-time reduction by half.
In a real collaborative environment, there are hundreds of more modified scripts and possibly thousands of repeated execution of these scripts, resulting in 1000s of hours of reduction in the cumulative run-time.

\subsection{Materialization}
In this experiment, we run the workloads of Table \ref{kaggle-workload} under different materialization budgets.
Figure \ref{exp-sa-vs-simple-size} shows the total run-time of all the workloads under different materialization budgets for the storage-aware and simple materialization algorithms.
\todo[inline]{There seems to be a bug reporting the result.}
\begin{figure}
\begin{subfigure}[b]{\linewidth}
\centering
%\includegraphics[width=\linewidth]{../images/experiment-results/kaggle_home_credit/execution_time/repetition/start_here_a_gentle_introduction}
 \resizebox{0.5\columnwidth}{!}{%
%% Creator: Matplotlib, PGF backend
%%
%% To include the figure in your LaTeX document, write
%%   \input{<filename>.pgf}
%%
%% Make sure the required packages are loaded in your preamble
%%   \usepackage{pgf}
%%
%% Figures using additional raster images can only be included by \input if
%% they are in the same directory as the main LaTeX file. For loading figures
%% from other directories you can use the `import` package
%%   \usepackage{import}
%% and then include the figures with
%%   \import{<path to file>}{<filename>.pgf}
%%
%% Matplotlib used the following preamble
%%   \usepackage{fontspec}
%%   \setmonofont{Andale Mono}
%%
\begingroup%
\makeatletter%
\begin{pgfpicture}%
\pgfpathrectangle{\pgfpointorigin}{\pgfqpoint{7.769500in}{5.538722in}}%
\pgfusepath{use as bounding box, clip}%
\begin{pgfscope}%
\pgfsetbuttcap%
\pgfsetmiterjoin%
\definecolor{currentfill}{rgb}{1.000000,1.000000,1.000000}%
\pgfsetfillcolor{currentfill}%
\pgfsetlinewidth{0.000000pt}%
\definecolor{currentstroke}{rgb}{1.000000,1.000000,1.000000}%
\pgfsetstrokecolor{currentstroke}%
\pgfsetdash{}{0pt}%
\pgfpathmoveto{\pgfqpoint{0.000000in}{0.000000in}}%
\pgfpathlineto{\pgfqpoint{7.769500in}{0.000000in}}%
\pgfpathlineto{\pgfqpoint{7.769500in}{5.538722in}}%
\pgfpathlineto{\pgfqpoint{0.000000in}{5.538722in}}%
\pgfpathclose%
\pgfusepath{fill}%
\end{pgfscope}%
\begin{pgfscope}%
\pgfsetbuttcap%
\pgfsetmiterjoin%
\definecolor{currentfill}{rgb}{1.000000,1.000000,1.000000}%
\pgfsetfillcolor{currentfill}%
\pgfsetlinewidth{0.000000pt}%
\definecolor{currentstroke}{rgb}{0.000000,0.000000,0.000000}%
\pgfsetstrokecolor{currentstroke}%
\pgfsetstrokeopacity{0.000000}%
\pgfsetdash{}{0pt}%
\pgfpathmoveto{\pgfqpoint{1.584777in}{1.109166in}}%
\pgfpathlineto{\pgfqpoint{7.602833in}{1.109166in}}%
\pgfpathlineto{\pgfqpoint{7.602833in}{5.009166in}}%
\pgfpathlineto{\pgfqpoint{1.584777in}{5.009166in}}%
\pgfpathclose%
\pgfusepath{fill}%
\end{pgfscope}%
\begin{pgfscope}%
\definecolor{textcolor}{rgb}{0.150000,0.150000,0.150000}%
\pgfsetstrokecolor{textcolor}%
\pgfsetfillcolor{textcolor}%
\pgftext[x=2.631396in,y=0.945277in,,top]{\color{textcolor}\rmfamily\fontsize{28.000000}{33.600000}\selectfont 8}%
\end{pgfscope}%
\begin{pgfscope}%
\definecolor{textcolor}{rgb}{0.150000,0.150000,0.150000}%
\pgfsetstrokecolor{textcolor}%
\pgfsetfillcolor{textcolor}%
\pgftext[x=3.939669in,y=0.945277in,,top]{\color{textcolor}\rmfamily\fontsize{28.000000}{33.600000}\selectfont 16}%
\end{pgfscope}%
\begin{pgfscope}%
\definecolor{textcolor}{rgb}{0.150000,0.150000,0.150000}%
\pgfsetstrokecolor{textcolor}%
\pgfsetfillcolor{textcolor}%
\pgftext[x=5.247942in,y=0.945277in,,top]{\color{textcolor}\rmfamily\fontsize{28.000000}{33.600000}\selectfont 32}%
\end{pgfscope}%
\begin{pgfscope}%
\definecolor{textcolor}{rgb}{0.150000,0.150000,0.150000}%
\pgfsetstrokecolor{textcolor}%
\pgfsetfillcolor{textcolor}%
\pgftext[x=6.556215in,y=0.945277in,,top]{\color{textcolor}\rmfamily\fontsize{28.000000}{33.600000}\selectfont 64}%
\end{pgfscope}%
\begin{pgfscope}%
\definecolor{textcolor}{rgb}{0.150000,0.150000,0.150000}%
\pgfsetstrokecolor{textcolor}%
\pgfsetfillcolor{textcolor}%
\pgftext[x=4.593805in,y=0.544000in,,top]{\color{textcolor}\rmfamily\fontsize{32.000000}{38.400000}\selectfont Budget (GB)}%
\end{pgfscope}%
\begin{pgfscope}%
\pgfpathrectangle{\pgfqpoint{1.584777in}{1.109166in}}{\pgfqpoint{6.018056in}{3.900000in}} %
\pgfusepath{clip}%
\pgfsetroundcap%
\pgfsetroundjoin%
\pgfsetlinewidth{0.803000pt}%
\definecolor{currentstroke}{rgb}{0.800000,0.800000,0.800000}%
\pgfsetstrokecolor{currentstroke}%
\pgfsetdash{}{0pt}%
\pgfpathmoveto{\pgfqpoint{1.584777in}{1.109166in}}%
\pgfpathlineto{\pgfqpoint{7.602833in}{1.109166in}}%
\pgfusepath{stroke}%
\end{pgfscope}%
\begin{pgfscope}%
\definecolor{textcolor}{rgb}{0.150000,0.150000,0.150000}%
\pgfsetstrokecolor{textcolor}%
\pgfsetfillcolor{textcolor}%
\pgftext[x=1.242389in,y=0.974222in,left,base]{\color{textcolor}\rmfamily\fontsize{28.000000}{33.600000}\selectfont 0}%
\end{pgfscope}%
\begin{pgfscope}%
\pgfpathrectangle{\pgfqpoint{1.584777in}{1.109166in}}{\pgfqpoint{6.018056in}{3.900000in}} %
\pgfusepath{clip}%
\pgfsetroundcap%
\pgfsetroundjoin%
\pgfsetlinewidth{0.803000pt}%
\definecolor{currentstroke}{rgb}{0.800000,0.800000,0.800000}%
\pgfsetstrokecolor{currentstroke}%
\pgfsetdash{}{0pt}%
\pgfpathmoveto{\pgfqpoint{1.584777in}{1.666309in}}%
\pgfpathlineto{\pgfqpoint{7.602833in}{1.666309in}}%
\pgfusepath{stroke}%
\end{pgfscope}%
\begin{pgfscope}%
\definecolor{textcolor}{rgb}{0.150000,0.150000,0.150000}%
\pgfsetstrokecolor{textcolor}%
\pgfsetfillcolor{textcolor}%
\pgftext[x=0.885389in,y=1.531365in,left,base]{\color{textcolor}\rmfamily\fontsize{28.000000}{33.600000}\selectfont 250}%
\end{pgfscope}%
\begin{pgfscope}%
\pgfpathrectangle{\pgfqpoint{1.584777in}{1.109166in}}{\pgfqpoint{6.018056in}{3.900000in}} %
\pgfusepath{clip}%
\pgfsetroundcap%
\pgfsetroundjoin%
\pgfsetlinewidth{0.803000pt}%
\definecolor{currentstroke}{rgb}{0.800000,0.800000,0.800000}%
\pgfsetstrokecolor{currentstroke}%
\pgfsetdash{}{0pt}%
\pgfpathmoveto{\pgfqpoint{1.584777in}{2.223452in}}%
\pgfpathlineto{\pgfqpoint{7.602833in}{2.223452in}}%
\pgfusepath{stroke}%
\end{pgfscope}%
\begin{pgfscope}%
\definecolor{textcolor}{rgb}{0.150000,0.150000,0.150000}%
\pgfsetstrokecolor{textcolor}%
\pgfsetfillcolor{textcolor}%
\pgftext[x=0.885389in,y=2.088508in,left,base]{\color{textcolor}\rmfamily\fontsize{28.000000}{33.600000}\selectfont 500}%
\end{pgfscope}%
\begin{pgfscope}%
\pgfpathrectangle{\pgfqpoint{1.584777in}{1.109166in}}{\pgfqpoint{6.018056in}{3.900000in}} %
\pgfusepath{clip}%
\pgfsetroundcap%
\pgfsetroundjoin%
\pgfsetlinewidth{0.803000pt}%
\definecolor{currentstroke}{rgb}{0.800000,0.800000,0.800000}%
\pgfsetstrokecolor{currentstroke}%
\pgfsetdash{}{0pt}%
\pgfpathmoveto{\pgfqpoint{1.584777in}{2.780595in}}%
\pgfpathlineto{\pgfqpoint{7.602833in}{2.780595in}}%
\pgfusepath{stroke}%
\end{pgfscope}%
\begin{pgfscope}%
\definecolor{textcolor}{rgb}{0.150000,0.150000,0.150000}%
\pgfsetstrokecolor{textcolor}%
\pgfsetfillcolor{textcolor}%
\pgftext[x=0.885389in,y=2.645650in,left,base]{\color{textcolor}\rmfamily\fontsize{28.000000}{33.600000}\selectfont 750}%
\end{pgfscope}%
\begin{pgfscope}%
\pgfpathrectangle{\pgfqpoint{1.584777in}{1.109166in}}{\pgfqpoint{6.018056in}{3.900000in}} %
\pgfusepath{clip}%
\pgfsetroundcap%
\pgfsetroundjoin%
\pgfsetlinewidth{0.803000pt}%
\definecolor{currentstroke}{rgb}{0.800000,0.800000,0.800000}%
\pgfsetstrokecolor{currentstroke}%
\pgfsetdash{}{0pt}%
\pgfpathmoveto{\pgfqpoint{1.584777in}{3.337738in}}%
\pgfpathlineto{\pgfqpoint{7.602833in}{3.337738in}}%
\pgfusepath{stroke}%
\end{pgfscope}%
\begin{pgfscope}%
\definecolor{textcolor}{rgb}{0.150000,0.150000,0.150000}%
\pgfsetstrokecolor{textcolor}%
\pgfsetfillcolor{textcolor}%
\pgftext[x=1.053778in,y=3.202793in,left,base]{\color{textcolor}\rmfamily\fontsize{28.000000}{33.600000}\selectfont 1k}%
\end{pgfscope}%
\begin{pgfscope}%
\pgfpathrectangle{\pgfqpoint{1.584777in}{1.109166in}}{\pgfqpoint{6.018056in}{3.900000in}} %
\pgfusepath{clip}%
\pgfsetroundcap%
\pgfsetroundjoin%
\pgfsetlinewidth{0.803000pt}%
\definecolor{currentstroke}{rgb}{0.800000,0.800000,0.800000}%
\pgfsetstrokecolor{currentstroke}%
\pgfsetdash{}{0pt}%
\pgfpathmoveto{\pgfqpoint{1.584777in}{3.894881in}}%
\pgfpathlineto{\pgfqpoint{7.602833in}{3.894881in}}%
\pgfusepath{stroke}%
\end{pgfscope}%
\begin{pgfscope}%
\definecolor{textcolor}{rgb}{0.150000,0.150000,0.150000}%
\pgfsetstrokecolor{textcolor}%
\pgfsetfillcolor{textcolor}%
\pgftext[x=0.599555in,y=3.759936in,left,base]{\color{textcolor}\rmfamily\fontsize{28.000000}{33.600000}\selectfont 1.25k}%
\end{pgfscope}%
\begin{pgfscope}%
\pgfpathrectangle{\pgfqpoint{1.584777in}{1.109166in}}{\pgfqpoint{6.018056in}{3.900000in}} %
\pgfusepath{clip}%
\pgfsetroundcap%
\pgfsetroundjoin%
\pgfsetlinewidth{0.803000pt}%
\definecolor{currentstroke}{rgb}{0.800000,0.800000,0.800000}%
\pgfsetstrokecolor{currentstroke}%
\pgfsetdash{}{0pt}%
\pgfpathmoveto{\pgfqpoint{1.584777in}{4.452023in}}%
\pgfpathlineto{\pgfqpoint{7.602833in}{4.452023in}}%
\pgfusepath{stroke}%
\end{pgfscope}%
\begin{pgfscope}%
\definecolor{textcolor}{rgb}{0.150000,0.150000,0.150000}%
\pgfsetstrokecolor{textcolor}%
\pgfsetfillcolor{textcolor}%
\pgftext[x=0.778055in,y=4.317079in,left,base]{\color{textcolor}\rmfamily\fontsize{28.000000}{33.600000}\selectfont 1.5k}%
\end{pgfscope}%
\begin{pgfscope}%
\pgfpathrectangle{\pgfqpoint{1.584777in}{1.109166in}}{\pgfqpoint{6.018056in}{3.900000in}} %
\pgfusepath{clip}%
\pgfsetroundcap%
\pgfsetroundjoin%
\pgfsetlinewidth{0.803000pt}%
\definecolor{currentstroke}{rgb}{0.800000,0.800000,0.800000}%
\pgfsetstrokecolor{currentstroke}%
\pgfsetdash{}{0pt}%
\pgfpathmoveto{\pgfqpoint{1.584777in}{5.009166in}}%
\pgfpathlineto{\pgfqpoint{7.602833in}{5.009166in}}%
\pgfusepath{stroke}%
\end{pgfscope}%
\begin{pgfscope}%
\definecolor{textcolor}{rgb}{0.150000,0.150000,0.150000}%
\pgfsetstrokecolor{textcolor}%
\pgfsetfillcolor{textcolor}%
\pgftext[x=0.599555in,y=4.874222in,left,base]{\color{textcolor}\rmfamily\fontsize{28.000000}{33.600000}\selectfont 1.75k}%
\end{pgfscope}%
\begin{pgfscope}%
\definecolor{textcolor}{rgb}{0.150000,0.150000,0.150000}%
\pgfsetstrokecolor{textcolor}%
\pgfsetfillcolor{textcolor}%
\pgftext[x=0.544000in,y=3.059166in,,bottom,rotate=90.000000]{\color{textcolor}\rmfamily\fontsize{32.000000}{38.400000}\selectfont Total Run Time (s)}%
\end{pgfscope}%
\begin{pgfscope}%
\pgfpathrectangle{\pgfqpoint{1.584777in}{1.109166in}}{\pgfqpoint{6.018056in}{3.900000in}} %
\pgfusepath{clip}%
\pgfsetbuttcap%
\pgfsetmiterjoin%
\definecolor{currentfill}{rgb}{0.347059,0.458824,0.641176}%
\pgfsetfillcolor{currentfill}%
\pgfsetlinewidth{0.803000pt}%
\definecolor{currentstroke}{rgb}{0.000000,0.000000,0.000000}%
\pgfsetstrokecolor{currentstroke}%
\pgfsetdash{}{0pt}%
\pgfpathmoveto{\pgfqpoint{2.108087in}{1.109166in}}%
\pgfpathlineto{\pgfqpoint{2.631396in}{1.109166in}}%
\pgfpathlineto{\pgfqpoint{2.631396in}{3.270798in}}%
\pgfpathlineto{\pgfqpoint{2.108087in}{3.270798in}}%
\pgfpathclose%
\pgfusepath{stroke,fill}%
\end{pgfscope}%
\begin{pgfscope}%
\pgfsetbuttcap%
\pgfsetmiterjoin%
\definecolor{currentfill}{rgb}{0.347059,0.458824,0.641176}%
\pgfsetfillcolor{currentfill}%
\pgfsetlinewidth{0.803000pt}%
\definecolor{currentstroke}{rgb}{0.000000,0.000000,0.000000}%
\pgfsetstrokecolor{currentstroke}%
\pgfsetdash{}{0pt}%
\pgfpathrectangle{\pgfqpoint{1.584777in}{1.109166in}}{\pgfqpoint{6.018056in}{3.900000in}} %
\pgfusepath{clip}%
\pgfpathmoveto{\pgfqpoint{2.108087in}{1.109166in}}%
\pgfpathlineto{\pgfqpoint{2.631396in}{1.109166in}}%
\pgfpathlineto{\pgfqpoint{2.631396in}{3.270798in}}%
\pgfpathlineto{\pgfqpoint{2.108087in}{3.270798in}}%
\pgfpathclose%
\pgfusepath{clip}%
\pgfsys@defobject{currentpattern}{\pgfqpoint{0in}{0in}}{\pgfqpoint{1in}{1in}}{%
\begin{pgfscope}%
\pgfpathrectangle{\pgfqpoint{0in}{0in}}{\pgfqpoint{1in}{1in}}%
\pgfusepath{clip}%
\pgfpathmoveto{\pgfqpoint{-0.500000in}{0.500000in}}%
\pgfpathlineto{\pgfqpoint{0.500000in}{1.500000in}}%
\pgfpathmoveto{\pgfqpoint{-0.333333in}{0.333333in}}%
\pgfpathlineto{\pgfqpoint{0.666667in}{1.333333in}}%
\pgfpathmoveto{\pgfqpoint{-0.166667in}{0.166667in}}%
\pgfpathlineto{\pgfqpoint{0.833333in}{1.166667in}}%
\pgfpathmoveto{\pgfqpoint{0.000000in}{0.000000in}}%
\pgfpathlineto{\pgfqpoint{1.000000in}{1.000000in}}%
\pgfpathmoveto{\pgfqpoint{0.166667in}{-0.166667in}}%
\pgfpathlineto{\pgfqpoint{1.166667in}{0.833333in}}%
\pgfpathmoveto{\pgfqpoint{0.333333in}{-0.333333in}}%
\pgfpathlineto{\pgfqpoint{1.333333in}{0.666667in}}%
\pgfpathmoveto{\pgfqpoint{0.500000in}{-0.500000in}}%
\pgfpathlineto{\pgfqpoint{1.500000in}{0.500000in}}%
\pgfusepath{stroke}%
\end{pgfscope}%
}%
\pgfsys@transformshift{2.108087in}{1.109166in}%
\pgfsys@useobject{currentpattern}{}%
\pgfsys@transformshift{1in}{0in}%
\pgfsys@transformshift{-1in}{0in}%
\pgfsys@transformshift{0in}{1in}%
\pgfsys@useobject{currentpattern}{}%
\pgfsys@transformshift{1in}{0in}%
\pgfsys@transformshift{-1in}{0in}%
\pgfsys@transformshift{0in}{1in}%
\pgfsys@useobject{currentpattern}{}%
\pgfsys@transformshift{1in}{0in}%
\pgfsys@transformshift{-1in}{0in}%
\pgfsys@transformshift{0in}{1in}%
\end{pgfscope}%
\begin{pgfscope}%
\pgfpathrectangle{\pgfqpoint{1.584777in}{1.109166in}}{\pgfqpoint{6.018056in}{3.900000in}} %
\pgfusepath{clip}%
\pgfsetbuttcap%
\pgfsetmiterjoin%
\definecolor{currentfill}{rgb}{0.347059,0.458824,0.641176}%
\pgfsetfillcolor{currentfill}%
\pgfsetlinewidth{0.803000pt}%
\definecolor{currentstroke}{rgb}{0.000000,0.000000,0.000000}%
\pgfsetstrokecolor{currentstroke}%
\pgfsetdash{}{0pt}%
\pgfpathmoveto{\pgfqpoint{3.416360in}{1.109166in}}%
\pgfpathlineto{\pgfqpoint{3.939669in}{1.109166in}}%
\pgfpathlineto{\pgfqpoint{3.939669in}{3.210967in}}%
\pgfpathlineto{\pgfqpoint{3.416360in}{3.210967in}}%
\pgfpathclose%
\pgfusepath{stroke,fill}%
\end{pgfscope}%
\begin{pgfscope}%
\pgfsetbuttcap%
\pgfsetmiterjoin%
\definecolor{currentfill}{rgb}{0.347059,0.458824,0.641176}%
\pgfsetfillcolor{currentfill}%
\pgfsetlinewidth{0.803000pt}%
\definecolor{currentstroke}{rgb}{0.000000,0.000000,0.000000}%
\pgfsetstrokecolor{currentstroke}%
\pgfsetdash{}{0pt}%
\pgfpathrectangle{\pgfqpoint{1.584777in}{1.109166in}}{\pgfqpoint{6.018056in}{3.900000in}} %
\pgfusepath{clip}%
\pgfpathmoveto{\pgfqpoint{3.416360in}{1.109166in}}%
\pgfpathlineto{\pgfqpoint{3.939669in}{1.109166in}}%
\pgfpathlineto{\pgfqpoint{3.939669in}{3.210967in}}%
\pgfpathlineto{\pgfqpoint{3.416360in}{3.210967in}}%
\pgfpathclose%
\pgfusepath{clip}%
\pgfsys@defobject{currentpattern}{\pgfqpoint{0in}{0in}}{\pgfqpoint{1in}{1in}}{%
\begin{pgfscope}%
\pgfpathrectangle{\pgfqpoint{0in}{0in}}{\pgfqpoint{1in}{1in}}%
\pgfusepath{clip}%
\pgfpathmoveto{\pgfqpoint{-0.500000in}{0.500000in}}%
\pgfpathlineto{\pgfqpoint{0.500000in}{1.500000in}}%
\pgfpathmoveto{\pgfqpoint{-0.333333in}{0.333333in}}%
\pgfpathlineto{\pgfqpoint{0.666667in}{1.333333in}}%
\pgfpathmoveto{\pgfqpoint{-0.166667in}{0.166667in}}%
\pgfpathlineto{\pgfqpoint{0.833333in}{1.166667in}}%
\pgfpathmoveto{\pgfqpoint{0.000000in}{0.000000in}}%
\pgfpathlineto{\pgfqpoint{1.000000in}{1.000000in}}%
\pgfpathmoveto{\pgfqpoint{0.166667in}{-0.166667in}}%
\pgfpathlineto{\pgfqpoint{1.166667in}{0.833333in}}%
\pgfpathmoveto{\pgfqpoint{0.333333in}{-0.333333in}}%
\pgfpathlineto{\pgfqpoint{1.333333in}{0.666667in}}%
\pgfpathmoveto{\pgfqpoint{0.500000in}{-0.500000in}}%
\pgfpathlineto{\pgfqpoint{1.500000in}{0.500000in}}%
\pgfusepath{stroke}%
\end{pgfscope}%
}%
\pgfsys@transformshift{3.416360in}{1.109166in}%
\pgfsys@useobject{currentpattern}{}%
\pgfsys@transformshift{1in}{0in}%
\pgfsys@transformshift{-1in}{0in}%
\pgfsys@transformshift{0in}{1in}%
\pgfsys@useobject{currentpattern}{}%
\pgfsys@transformshift{1in}{0in}%
\pgfsys@transformshift{-1in}{0in}%
\pgfsys@transformshift{0in}{1in}%
\pgfsys@useobject{currentpattern}{}%
\pgfsys@transformshift{1in}{0in}%
\pgfsys@transformshift{-1in}{0in}%
\pgfsys@transformshift{0in}{1in}%
\end{pgfscope}%
\begin{pgfscope}%
\pgfpathrectangle{\pgfqpoint{1.584777in}{1.109166in}}{\pgfqpoint{6.018056in}{3.900000in}} %
\pgfusepath{clip}%
\pgfsetbuttcap%
\pgfsetmiterjoin%
\definecolor{currentfill}{rgb}{0.347059,0.458824,0.641176}%
\pgfsetfillcolor{currentfill}%
\pgfsetlinewidth{0.803000pt}%
\definecolor{currentstroke}{rgb}{0.000000,0.000000,0.000000}%
\pgfsetstrokecolor{currentstroke}%
\pgfsetdash{}{0pt}%
\pgfpathmoveto{\pgfqpoint{4.724633in}{1.109166in}}%
\pgfpathlineto{\pgfqpoint{5.247942in}{1.109166in}}%
\pgfpathlineto{\pgfqpoint{5.247942in}{3.134253in}}%
\pgfpathlineto{\pgfqpoint{4.724633in}{3.134253in}}%
\pgfpathclose%
\pgfusepath{stroke,fill}%
\end{pgfscope}%
\begin{pgfscope}%
\pgfsetbuttcap%
\pgfsetmiterjoin%
\definecolor{currentfill}{rgb}{0.347059,0.458824,0.641176}%
\pgfsetfillcolor{currentfill}%
\pgfsetlinewidth{0.803000pt}%
\definecolor{currentstroke}{rgb}{0.000000,0.000000,0.000000}%
\pgfsetstrokecolor{currentstroke}%
\pgfsetdash{}{0pt}%
\pgfpathrectangle{\pgfqpoint{1.584777in}{1.109166in}}{\pgfqpoint{6.018056in}{3.900000in}} %
\pgfusepath{clip}%
\pgfpathmoveto{\pgfqpoint{4.724633in}{1.109166in}}%
\pgfpathlineto{\pgfqpoint{5.247942in}{1.109166in}}%
\pgfpathlineto{\pgfqpoint{5.247942in}{3.134253in}}%
\pgfpathlineto{\pgfqpoint{4.724633in}{3.134253in}}%
\pgfpathclose%
\pgfusepath{clip}%
\pgfsys@defobject{currentpattern}{\pgfqpoint{0in}{0in}}{\pgfqpoint{1in}{1in}}{%
\begin{pgfscope}%
\pgfpathrectangle{\pgfqpoint{0in}{0in}}{\pgfqpoint{1in}{1in}}%
\pgfusepath{clip}%
\pgfpathmoveto{\pgfqpoint{-0.500000in}{0.500000in}}%
\pgfpathlineto{\pgfqpoint{0.500000in}{1.500000in}}%
\pgfpathmoveto{\pgfqpoint{-0.333333in}{0.333333in}}%
\pgfpathlineto{\pgfqpoint{0.666667in}{1.333333in}}%
\pgfpathmoveto{\pgfqpoint{-0.166667in}{0.166667in}}%
\pgfpathlineto{\pgfqpoint{0.833333in}{1.166667in}}%
\pgfpathmoveto{\pgfqpoint{0.000000in}{0.000000in}}%
\pgfpathlineto{\pgfqpoint{1.000000in}{1.000000in}}%
\pgfpathmoveto{\pgfqpoint{0.166667in}{-0.166667in}}%
\pgfpathlineto{\pgfqpoint{1.166667in}{0.833333in}}%
\pgfpathmoveto{\pgfqpoint{0.333333in}{-0.333333in}}%
\pgfpathlineto{\pgfqpoint{1.333333in}{0.666667in}}%
\pgfpathmoveto{\pgfqpoint{0.500000in}{-0.500000in}}%
\pgfpathlineto{\pgfqpoint{1.500000in}{0.500000in}}%
\pgfusepath{stroke}%
\end{pgfscope}%
}%
\pgfsys@transformshift{4.724633in}{1.109166in}%
\pgfsys@useobject{currentpattern}{}%
\pgfsys@transformshift{1in}{0in}%
\pgfsys@transformshift{-1in}{0in}%
\pgfsys@transformshift{0in}{1in}%
\pgfsys@useobject{currentpattern}{}%
\pgfsys@transformshift{1in}{0in}%
\pgfsys@transformshift{-1in}{0in}%
\pgfsys@transformshift{0in}{1in}%
\pgfsys@useobject{currentpattern}{}%
\pgfsys@transformshift{1in}{0in}%
\pgfsys@transformshift{-1in}{0in}%
\pgfsys@transformshift{0in}{1in}%
\end{pgfscope}%
\begin{pgfscope}%
\pgfpathrectangle{\pgfqpoint{1.584777in}{1.109166in}}{\pgfqpoint{6.018056in}{3.900000in}} %
\pgfusepath{clip}%
\pgfsetbuttcap%
\pgfsetmiterjoin%
\definecolor{currentfill}{rgb}{0.347059,0.458824,0.641176}%
\pgfsetfillcolor{currentfill}%
\pgfsetlinewidth{0.803000pt}%
\definecolor{currentstroke}{rgb}{0.000000,0.000000,0.000000}%
\pgfsetstrokecolor{currentstroke}%
\pgfsetdash{}{0pt}%
\pgfpathmoveto{\pgfqpoint{6.032906in}{1.109166in}}%
\pgfpathlineto{\pgfqpoint{6.556215in}{1.109166in}}%
\pgfpathlineto{\pgfqpoint{6.556215in}{3.106925in}}%
\pgfpathlineto{\pgfqpoint{6.032906in}{3.106925in}}%
\pgfpathclose%
\pgfusepath{stroke,fill}%
\end{pgfscope}%
\begin{pgfscope}%
\pgfsetbuttcap%
\pgfsetmiterjoin%
\definecolor{currentfill}{rgb}{0.347059,0.458824,0.641176}%
\pgfsetfillcolor{currentfill}%
\pgfsetlinewidth{0.803000pt}%
\definecolor{currentstroke}{rgb}{0.000000,0.000000,0.000000}%
\pgfsetstrokecolor{currentstroke}%
\pgfsetdash{}{0pt}%
\pgfpathrectangle{\pgfqpoint{1.584777in}{1.109166in}}{\pgfqpoint{6.018056in}{3.900000in}} %
\pgfusepath{clip}%
\pgfpathmoveto{\pgfqpoint{6.032906in}{1.109166in}}%
\pgfpathlineto{\pgfqpoint{6.556215in}{1.109166in}}%
\pgfpathlineto{\pgfqpoint{6.556215in}{3.106925in}}%
\pgfpathlineto{\pgfqpoint{6.032906in}{3.106925in}}%
\pgfpathclose%
\pgfusepath{clip}%
\pgfsys@defobject{currentpattern}{\pgfqpoint{0in}{0in}}{\pgfqpoint{1in}{1in}}{%
\begin{pgfscope}%
\pgfpathrectangle{\pgfqpoint{0in}{0in}}{\pgfqpoint{1in}{1in}}%
\pgfusepath{clip}%
\pgfpathmoveto{\pgfqpoint{-0.500000in}{0.500000in}}%
\pgfpathlineto{\pgfqpoint{0.500000in}{1.500000in}}%
\pgfpathmoveto{\pgfqpoint{-0.333333in}{0.333333in}}%
\pgfpathlineto{\pgfqpoint{0.666667in}{1.333333in}}%
\pgfpathmoveto{\pgfqpoint{-0.166667in}{0.166667in}}%
\pgfpathlineto{\pgfqpoint{0.833333in}{1.166667in}}%
\pgfpathmoveto{\pgfqpoint{0.000000in}{0.000000in}}%
\pgfpathlineto{\pgfqpoint{1.000000in}{1.000000in}}%
\pgfpathmoveto{\pgfqpoint{0.166667in}{-0.166667in}}%
\pgfpathlineto{\pgfqpoint{1.166667in}{0.833333in}}%
\pgfpathmoveto{\pgfqpoint{0.333333in}{-0.333333in}}%
\pgfpathlineto{\pgfqpoint{1.333333in}{0.666667in}}%
\pgfpathmoveto{\pgfqpoint{0.500000in}{-0.500000in}}%
\pgfpathlineto{\pgfqpoint{1.500000in}{0.500000in}}%
\pgfusepath{stroke}%
\end{pgfscope}%
}%
\pgfsys@transformshift{6.032906in}{1.109166in}%
\pgfsys@useobject{currentpattern}{}%
\pgfsys@transformshift{1in}{0in}%
\pgfsys@transformshift{-1in}{0in}%
\pgfsys@transformshift{0in}{1in}%
\pgfsys@useobject{currentpattern}{}%
\pgfsys@transformshift{1in}{0in}%
\pgfsys@transformshift{-1in}{0in}%
\pgfsys@transformshift{0in}{1in}%
\end{pgfscope}%
\begin{pgfscope}%
\pgfpathrectangle{\pgfqpoint{1.584777in}{1.109166in}}{\pgfqpoint{6.018056in}{3.900000in}} %
\pgfusepath{clip}%
\pgfsetbuttcap%
\pgfsetmiterjoin%
\definecolor{currentfill}{rgb}{0.798529,0.536765,0.389706}%
\pgfsetfillcolor{currentfill}%
\pgfsetlinewidth{0.803000pt}%
\definecolor{currentstroke}{rgb}{0.000000,0.000000,0.000000}%
\pgfsetstrokecolor{currentstroke}%
\pgfsetdash{}{0pt}%
\pgfpathmoveto{\pgfqpoint{2.631396in}{1.109166in}}%
\pgfpathlineto{\pgfqpoint{3.154705in}{1.109166in}}%
\pgfpathlineto{\pgfqpoint{3.154705in}{4.713932in}}%
\pgfpathlineto{\pgfqpoint{2.631396in}{4.713932in}}%
\pgfpathclose%
\pgfusepath{stroke,fill}%
\end{pgfscope}%
\begin{pgfscope}%
\pgfsetbuttcap%
\pgfsetmiterjoin%
\definecolor{currentfill}{rgb}{0.798529,0.536765,0.389706}%
\pgfsetfillcolor{currentfill}%
\pgfsetlinewidth{0.803000pt}%
\definecolor{currentstroke}{rgb}{0.000000,0.000000,0.000000}%
\pgfsetstrokecolor{currentstroke}%
\pgfsetdash{}{0pt}%
\pgfpathrectangle{\pgfqpoint{1.584777in}{1.109166in}}{\pgfqpoint{6.018056in}{3.900000in}} %
\pgfusepath{clip}%
\pgfpathmoveto{\pgfqpoint{2.631396in}{1.109166in}}%
\pgfpathlineto{\pgfqpoint{3.154705in}{1.109166in}}%
\pgfpathlineto{\pgfqpoint{3.154705in}{4.713932in}}%
\pgfpathlineto{\pgfqpoint{2.631396in}{4.713932in}}%
\pgfpathclose%
\pgfusepath{clip}%
\pgfsys@defobject{currentpattern}{\pgfqpoint{0in}{0in}}{\pgfqpoint{1in}{1in}}{%
\begin{pgfscope}%
\pgfpathrectangle{\pgfqpoint{0in}{0in}}{\pgfqpoint{1in}{1in}}%
\pgfusepath{clip}%
\pgfpathmoveto{\pgfqpoint{-0.500000in}{0.500000in}}%
\pgfpathlineto{\pgfqpoint{0.500000in}{1.500000in}}%
\pgfpathmoveto{\pgfqpoint{-0.333333in}{0.333333in}}%
\pgfpathlineto{\pgfqpoint{0.666667in}{1.333333in}}%
\pgfpathmoveto{\pgfqpoint{-0.166667in}{0.166667in}}%
\pgfpathlineto{\pgfqpoint{0.833333in}{1.166667in}}%
\pgfpathmoveto{\pgfqpoint{0.000000in}{0.000000in}}%
\pgfpathlineto{\pgfqpoint{1.000000in}{1.000000in}}%
\pgfpathmoveto{\pgfqpoint{0.166667in}{-0.166667in}}%
\pgfpathlineto{\pgfqpoint{1.166667in}{0.833333in}}%
\pgfpathmoveto{\pgfqpoint{0.333333in}{-0.333333in}}%
\pgfpathlineto{\pgfqpoint{1.333333in}{0.666667in}}%
\pgfpathmoveto{\pgfqpoint{0.500000in}{-0.500000in}}%
\pgfpathlineto{\pgfqpoint{1.500000in}{0.500000in}}%
\pgfpathmoveto{\pgfqpoint{-0.500000in}{0.500000in}}%
\pgfpathlineto{\pgfqpoint{0.500000in}{-0.500000in}}%
\pgfpathmoveto{\pgfqpoint{-0.333333in}{0.666667in}}%
\pgfpathlineto{\pgfqpoint{0.666667in}{-0.333333in}}%
\pgfpathmoveto{\pgfqpoint{-0.166667in}{0.833333in}}%
\pgfpathlineto{\pgfqpoint{0.833333in}{-0.166667in}}%
\pgfpathmoveto{\pgfqpoint{0.000000in}{1.000000in}}%
\pgfpathlineto{\pgfqpoint{1.000000in}{0.000000in}}%
\pgfpathmoveto{\pgfqpoint{0.166667in}{1.166667in}}%
\pgfpathlineto{\pgfqpoint{1.166667in}{0.166667in}}%
\pgfpathmoveto{\pgfqpoint{0.333333in}{1.333333in}}%
\pgfpathlineto{\pgfqpoint{1.333333in}{0.333333in}}%
\pgfpathmoveto{\pgfqpoint{0.500000in}{1.500000in}}%
\pgfpathlineto{\pgfqpoint{1.500000in}{0.500000in}}%
\pgfusepath{stroke}%
\end{pgfscope}%
}%
\pgfsys@transformshift{2.631396in}{1.109166in}%
\pgfsys@useobject{currentpattern}{}%
\pgfsys@transformshift{1in}{0in}%
\pgfsys@transformshift{-1in}{0in}%
\pgfsys@transformshift{0in}{1in}%
\pgfsys@useobject{currentpattern}{}%
\pgfsys@transformshift{1in}{0in}%
\pgfsys@transformshift{-1in}{0in}%
\pgfsys@transformshift{0in}{1in}%
\pgfsys@useobject{currentpattern}{}%
\pgfsys@transformshift{1in}{0in}%
\pgfsys@transformshift{-1in}{0in}%
\pgfsys@transformshift{0in}{1in}%
\pgfsys@useobject{currentpattern}{}%
\pgfsys@transformshift{1in}{0in}%
\pgfsys@transformshift{-1in}{0in}%
\pgfsys@transformshift{0in}{1in}%
\end{pgfscope}%
\begin{pgfscope}%
\pgfpathrectangle{\pgfqpoint{1.584777in}{1.109166in}}{\pgfqpoint{6.018056in}{3.900000in}} %
\pgfusepath{clip}%
\pgfsetbuttcap%
\pgfsetmiterjoin%
\definecolor{currentfill}{rgb}{0.798529,0.536765,0.389706}%
\pgfsetfillcolor{currentfill}%
\pgfsetlinewidth{0.803000pt}%
\definecolor{currentstroke}{rgb}{0.000000,0.000000,0.000000}%
\pgfsetstrokecolor{currentstroke}%
\pgfsetdash{}{0pt}%
\pgfpathmoveto{\pgfqpoint{3.939669in}{1.109166in}}%
\pgfpathlineto{\pgfqpoint{4.462978in}{1.109166in}}%
\pgfpathlineto{\pgfqpoint{4.462978in}{4.309457in}}%
\pgfpathlineto{\pgfqpoint{3.939669in}{4.309457in}}%
\pgfpathclose%
\pgfusepath{stroke,fill}%
\end{pgfscope}%
\begin{pgfscope}%
\pgfsetbuttcap%
\pgfsetmiterjoin%
\definecolor{currentfill}{rgb}{0.798529,0.536765,0.389706}%
\pgfsetfillcolor{currentfill}%
\pgfsetlinewidth{0.803000pt}%
\definecolor{currentstroke}{rgb}{0.000000,0.000000,0.000000}%
\pgfsetstrokecolor{currentstroke}%
\pgfsetdash{}{0pt}%
\pgfpathrectangle{\pgfqpoint{1.584777in}{1.109166in}}{\pgfqpoint{6.018056in}{3.900000in}} %
\pgfusepath{clip}%
\pgfpathmoveto{\pgfqpoint{3.939669in}{1.109166in}}%
\pgfpathlineto{\pgfqpoint{4.462978in}{1.109166in}}%
\pgfpathlineto{\pgfqpoint{4.462978in}{4.309457in}}%
\pgfpathlineto{\pgfqpoint{3.939669in}{4.309457in}}%
\pgfpathclose%
\pgfusepath{clip}%
\pgfsys@defobject{currentpattern}{\pgfqpoint{0in}{0in}}{\pgfqpoint{1in}{1in}}{%
\begin{pgfscope}%
\pgfpathrectangle{\pgfqpoint{0in}{0in}}{\pgfqpoint{1in}{1in}}%
\pgfusepath{clip}%
\pgfpathmoveto{\pgfqpoint{-0.500000in}{0.500000in}}%
\pgfpathlineto{\pgfqpoint{0.500000in}{1.500000in}}%
\pgfpathmoveto{\pgfqpoint{-0.333333in}{0.333333in}}%
\pgfpathlineto{\pgfqpoint{0.666667in}{1.333333in}}%
\pgfpathmoveto{\pgfqpoint{-0.166667in}{0.166667in}}%
\pgfpathlineto{\pgfqpoint{0.833333in}{1.166667in}}%
\pgfpathmoveto{\pgfqpoint{0.000000in}{0.000000in}}%
\pgfpathlineto{\pgfqpoint{1.000000in}{1.000000in}}%
\pgfpathmoveto{\pgfqpoint{0.166667in}{-0.166667in}}%
\pgfpathlineto{\pgfqpoint{1.166667in}{0.833333in}}%
\pgfpathmoveto{\pgfqpoint{0.333333in}{-0.333333in}}%
\pgfpathlineto{\pgfqpoint{1.333333in}{0.666667in}}%
\pgfpathmoveto{\pgfqpoint{0.500000in}{-0.500000in}}%
\pgfpathlineto{\pgfqpoint{1.500000in}{0.500000in}}%
\pgfpathmoveto{\pgfqpoint{-0.500000in}{0.500000in}}%
\pgfpathlineto{\pgfqpoint{0.500000in}{-0.500000in}}%
\pgfpathmoveto{\pgfqpoint{-0.333333in}{0.666667in}}%
\pgfpathlineto{\pgfqpoint{0.666667in}{-0.333333in}}%
\pgfpathmoveto{\pgfqpoint{-0.166667in}{0.833333in}}%
\pgfpathlineto{\pgfqpoint{0.833333in}{-0.166667in}}%
\pgfpathmoveto{\pgfqpoint{0.000000in}{1.000000in}}%
\pgfpathlineto{\pgfqpoint{1.000000in}{0.000000in}}%
\pgfpathmoveto{\pgfqpoint{0.166667in}{1.166667in}}%
\pgfpathlineto{\pgfqpoint{1.166667in}{0.166667in}}%
\pgfpathmoveto{\pgfqpoint{0.333333in}{1.333333in}}%
\pgfpathlineto{\pgfqpoint{1.333333in}{0.333333in}}%
\pgfpathmoveto{\pgfqpoint{0.500000in}{1.500000in}}%
\pgfpathlineto{\pgfqpoint{1.500000in}{0.500000in}}%
\pgfusepath{stroke}%
\end{pgfscope}%
}%
\pgfsys@transformshift{3.939669in}{1.109166in}%
\pgfsys@useobject{currentpattern}{}%
\pgfsys@transformshift{1in}{0in}%
\pgfsys@transformshift{-1in}{0in}%
\pgfsys@transformshift{0in}{1in}%
\pgfsys@useobject{currentpattern}{}%
\pgfsys@transformshift{1in}{0in}%
\pgfsys@transformshift{-1in}{0in}%
\pgfsys@transformshift{0in}{1in}%
\pgfsys@useobject{currentpattern}{}%
\pgfsys@transformshift{1in}{0in}%
\pgfsys@transformshift{-1in}{0in}%
\pgfsys@transformshift{0in}{1in}%
\pgfsys@useobject{currentpattern}{}%
\pgfsys@transformshift{1in}{0in}%
\pgfsys@transformshift{-1in}{0in}%
\pgfsys@transformshift{0in}{1in}%
\end{pgfscope}%
\begin{pgfscope}%
\pgfpathrectangle{\pgfqpoint{1.584777in}{1.109166in}}{\pgfqpoint{6.018056in}{3.900000in}} %
\pgfusepath{clip}%
\pgfsetbuttcap%
\pgfsetmiterjoin%
\definecolor{currentfill}{rgb}{0.798529,0.536765,0.389706}%
\pgfsetfillcolor{currentfill}%
\pgfsetlinewidth{0.803000pt}%
\definecolor{currentstroke}{rgb}{0.000000,0.000000,0.000000}%
\pgfsetstrokecolor{currentstroke}%
\pgfsetdash{}{0pt}%
\pgfpathmoveto{\pgfqpoint{5.247942in}{1.109166in}}%
\pgfpathlineto{\pgfqpoint{5.771251in}{1.109166in}}%
\pgfpathlineto{\pgfqpoint{5.771251in}{3.301951in}}%
\pgfpathlineto{\pgfqpoint{5.247942in}{3.301951in}}%
\pgfpathclose%
\pgfusepath{stroke,fill}%
\end{pgfscope}%
\begin{pgfscope}%
\pgfsetbuttcap%
\pgfsetmiterjoin%
\definecolor{currentfill}{rgb}{0.798529,0.536765,0.389706}%
\pgfsetfillcolor{currentfill}%
\pgfsetlinewidth{0.803000pt}%
\definecolor{currentstroke}{rgb}{0.000000,0.000000,0.000000}%
\pgfsetstrokecolor{currentstroke}%
\pgfsetdash{}{0pt}%
\pgfpathrectangle{\pgfqpoint{1.584777in}{1.109166in}}{\pgfqpoint{6.018056in}{3.900000in}} %
\pgfusepath{clip}%
\pgfpathmoveto{\pgfqpoint{5.247942in}{1.109166in}}%
\pgfpathlineto{\pgfqpoint{5.771251in}{1.109166in}}%
\pgfpathlineto{\pgfqpoint{5.771251in}{3.301951in}}%
\pgfpathlineto{\pgfqpoint{5.247942in}{3.301951in}}%
\pgfpathclose%
\pgfusepath{clip}%
\pgfsys@defobject{currentpattern}{\pgfqpoint{0in}{0in}}{\pgfqpoint{1in}{1in}}{%
\begin{pgfscope}%
\pgfpathrectangle{\pgfqpoint{0in}{0in}}{\pgfqpoint{1in}{1in}}%
\pgfusepath{clip}%
\pgfpathmoveto{\pgfqpoint{-0.500000in}{0.500000in}}%
\pgfpathlineto{\pgfqpoint{0.500000in}{1.500000in}}%
\pgfpathmoveto{\pgfqpoint{-0.333333in}{0.333333in}}%
\pgfpathlineto{\pgfqpoint{0.666667in}{1.333333in}}%
\pgfpathmoveto{\pgfqpoint{-0.166667in}{0.166667in}}%
\pgfpathlineto{\pgfqpoint{0.833333in}{1.166667in}}%
\pgfpathmoveto{\pgfqpoint{0.000000in}{0.000000in}}%
\pgfpathlineto{\pgfqpoint{1.000000in}{1.000000in}}%
\pgfpathmoveto{\pgfqpoint{0.166667in}{-0.166667in}}%
\pgfpathlineto{\pgfqpoint{1.166667in}{0.833333in}}%
\pgfpathmoveto{\pgfqpoint{0.333333in}{-0.333333in}}%
\pgfpathlineto{\pgfqpoint{1.333333in}{0.666667in}}%
\pgfpathmoveto{\pgfqpoint{0.500000in}{-0.500000in}}%
\pgfpathlineto{\pgfqpoint{1.500000in}{0.500000in}}%
\pgfpathmoveto{\pgfqpoint{-0.500000in}{0.500000in}}%
\pgfpathlineto{\pgfqpoint{0.500000in}{-0.500000in}}%
\pgfpathmoveto{\pgfqpoint{-0.333333in}{0.666667in}}%
\pgfpathlineto{\pgfqpoint{0.666667in}{-0.333333in}}%
\pgfpathmoveto{\pgfqpoint{-0.166667in}{0.833333in}}%
\pgfpathlineto{\pgfqpoint{0.833333in}{-0.166667in}}%
\pgfpathmoveto{\pgfqpoint{0.000000in}{1.000000in}}%
\pgfpathlineto{\pgfqpoint{1.000000in}{0.000000in}}%
\pgfpathmoveto{\pgfqpoint{0.166667in}{1.166667in}}%
\pgfpathlineto{\pgfqpoint{1.166667in}{0.166667in}}%
\pgfpathmoveto{\pgfqpoint{0.333333in}{1.333333in}}%
\pgfpathlineto{\pgfqpoint{1.333333in}{0.333333in}}%
\pgfpathmoveto{\pgfqpoint{0.500000in}{1.500000in}}%
\pgfpathlineto{\pgfqpoint{1.500000in}{0.500000in}}%
\pgfusepath{stroke}%
\end{pgfscope}%
}%
\pgfsys@transformshift{5.247942in}{1.109166in}%
\pgfsys@useobject{currentpattern}{}%
\pgfsys@transformshift{1in}{0in}%
\pgfsys@transformshift{-1in}{0in}%
\pgfsys@transformshift{0in}{1in}%
\pgfsys@useobject{currentpattern}{}%
\pgfsys@transformshift{1in}{0in}%
\pgfsys@transformshift{-1in}{0in}%
\pgfsys@transformshift{0in}{1in}%
\pgfsys@useobject{currentpattern}{}%
\pgfsys@transformshift{1in}{0in}%
\pgfsys@transformshift{-1in}{0in}%
\pgfsys@transformshift{0in}{1in}%
\end{pgfscope}%
\begin{pgfscope}%
\pgfpathrectangle{\pgfqpoint{1.584777in}{1.109166in}}{\pgfqpoint{6.018056in}{3.900000in}} %
\pgfusepath{clip}%
\pgfsetbuttcap%
\pgfsetmiterjoin%
\definecolor{currentfill}{rgb}{0.798529,0.536765,0.389706}%
\pgfsetfillcolor{currentfill}%
\pgfsetlinewidth{0.803000pt}%
\definecolor{currentstroke}{rgb}{0.000000,0.000000,0.000000}%
\pgfsetstrokecolor{currentstroke}%
\pgfsetdash{}{0pt}%
\pgfpathmoveto{\pgfqpoint{6.556215in}{1.109166in}}%
\pgfpathlineto{\pgfqpoint{7.079524in}{1.109166in}}%
\pgfpathlineto{\pgfqpoint{7.079524in}{3.183124in}}%
\pgfpathlineto{\pgfqpoint{6.556215in}{3.183124in}}%
\pgfpathclose%
\pgfusepath{stroke,fill}%
\end{pgfscope}%
\begin{pgfscope}%
\pgfsetbuttcap%
\pgfsetmiterjoin%
\definecolor{currentfill}{rgb}{0.798529,0.536765,0.389706}%
\pgfsetfillcolor{currentfill}%
\pgfsetlinewidth{0.803000pt}%
\definecolor{currentstroke}{rgb}{0.000000,0.000000,0.000000}%
\pgfsetstrokecolor{currentstroke}%
\pgfsetdash{}{0pt}%
\pgfpathrectangle{\pgfqpoint{1.584777in}{1.109166in}}{\pgfqpoint{6.018056in}{3.900000in}} %
\pgfusepath{clip}%
\pgfpathmoveto{\pgfqpoint{6.556215in}{1.109166in}}%
\pgfpathlineto{\pgfqpoint{7.079524in}{1.109166in}}%
\pgfpathlineto{\pgfqpoint{7.079524in}{3.183124in}}%
\pgfpathlineto{\pgfqpoint{6.556215in}{3.183124in}}%
\pgfpathclose%
\pgfusepath{clip}%
\pgfsys@defobject{currentpattern}{\pgfqpoint{0in}{0in}}{\pgfqpoint{1in}{1in}}{%
\begin{pgfscope}%
\pgfpathrectangle{\pgfqpoint{0in}{0in}}{\pgfqpoint{1in}{1in}}%
\pgfusepath{clip}%
\pgfpathmoveto{\pgfqpoint{-0.500000in}{0.500000in}}%
\pgfpathlineto{\pgfqpoint{0.500000in}{1.500000in}}%
\pgfpathmoveto{\pgfqpoint{-0.333333in}{0.333333in}}%
\pgfpathlineto{\pgfqpoint{0.666667in}{1.333333in}}%
\pgfpathmoveto{\pgfqpoint{-0.166667in}{0.166667in}}%
\pgfpathlineto{\pgfqpoint{0.833333in}{1.166667in}}%
\pgfpathmoveto{\pgfqpoint{0.000000in}{0.000000in}}%
\pgfpathlineto{\pgfqpoint{1.000000in}{1.000000in}}%
\pgfpathmoveto{\pgfqpoint{0.166667in}{-0.166667in}}%
\pgfpathlineto{\pgfqpoint{1.166667in}{0.833333in}}%
\pgfpathmoveto{\pgfqpoint{0.333333in}{-0.333333in}}%
\pgfpathlineto{\pgfqpoint{1.333333in}{0.666667in}}%
\pgfpathmoveto{\pgfqpoint{0.500000in}{-0.500000in}}%
\pgfpathlineto{\pgfqpoint{1.500000in}{0.500000in}}%
\pgfpathmoveto{\pgfqpoint{-0.500000in}{0.500000in}}%
\pgfpathlineto{\pgfqpoint{0.500000in}{-0.500000in}}%
\pgfpathmoveto{\pgfqpoint{-0.333333in}{0.666667in}}%
\pgfpathlineto{\pgfqpoint{0.666667in}{-0.333333in}}%
\pgfpathmoveto{\pgfqpoint{-0.166667in}{0.833333in}}%
\pgfpathlineto{\pgfqpoint{0.833333in}{-0.166667in}}%
\pgfpathmoveto{\pgfqpoint{0.000000in}{1.000000in}}%
\pgfpathlineto{\pgfqpoint{1.000000in}{0.000000in}}%
\pgfpathmoveto{\pgfqpoint{0.166667in}{1.166667in}}%
\pgfpathlineto{\pgfqpoint{1.166667in}{0.166667in}}%
\pgfpathmoveto{\pgfqpoint{0.333333in}{1.333333in}}%
\pgfpathlineto{\pgfqpoint{1.333333in}{0.333333in}}%
\pgfpathmoveto{\pgfqpoint{0.500000in}{1.500000in}}%
\pgfpathlineto{\pgfqpoint{1.500000in}{0.500000in}}%
\pgfusepath{stroke}%
\end{pgfscope}%
}%
\pgfsys@transformshift{6.556215in}{1.109166in}%
\pgfsys@useobject{currentpattern}{}%
\pgfsys@transformshift{1in}{0in}%
\pgfsys@transformshift{-1in}{0in}%
\pgfsys@transformshift{0in}{1in}%
\pgfsys@useobject{currentpattern}{}%
\pgfsys@transformshift{1in}{0in}%
\pgfsys@transformshift{-1in}{0in}%
\pgfsys@transformshift{0in}{1in}%
\pgfsys@useobject{currentpattern}{}%
\pgfsys@transformshift{1in}{0in}%
\pgfsys@transformshift{-1in}{0in}%
\pgfsys@transformshift{0in}{1in}%
\end{pgfscope}%
\begin{pgfscope}%
\pgfpathrectangle{\pgfqpoint{1.584777in}{1.109166in}}{\pgfqpoint{6.018056in}{3.900000in}} %
\pgfusepath{clip}%
\pgfsetroundcap%
\pgfsetroundjoin%
\pgfsetlinewidth{3.011250pt}%
\definecolor{currentstroke}{rgb}{0.000000,0.000000,0.000000}%
\pgfsetstrokecolor{currentstroke}%
\pgfsetdash{}{0pt}%
\pgfusepath{stroke}%
\end{pgfscope}%
\begin{pgfscope}%
\pgfpathrectangle{\pgfqpoint{1.584777in}{1.109166in}}{\pgfqpoint{6.018056in}{3.900000in}} %
\pgfusepath{clip}%
\pgfsetroundcap%
\pgfsetroundjoin%
\pgfsetlinewidth{3.011250pt}%
\definecolor{currentstroke}{rgb}{0.000000,0.000000,0.000000}%
\pgfsetstrokecolor{currentstroke}%
\pgfsetdash{}{0pt}%
\pgfusepath{stroke}%
\end{pgfscope}%
\begin{pgfscope}%
\pgfpathrectangle{\pgfqpoint{1.584777in}{1.109166in}}{\pgfqpoint{6.018056in}{3.900000in}} %
\pgfusepath{clip}%
\pgfsetroundcap%
\pgfsetroundjoin%
\pgfsetlinewidth{3.011250pt}%
\definecolor{currentstroke}{rgb}{0.000000,0.000000,0.000000}%
\pgfsetstrokecolor{currentstroke}%
\pgfsetdash{}{0pt}%
\pgfusepath{stroke}%
\end{pgfscope}%
\begin{pgfscope}%
\pgfpathrectangle{\pgfqpoint{1.584777in}{1.109166in}}{\pgfqpoint{6.018056in}{3.900000in}} %
\pgfusepath{clip}%
\pgfsetroundcap%
\pgfsetroundjoin%
\pgfsetlinewidth{3.011250pt}%
\definecolor{currentstroke}{rgb}{0.000000,0.000000,0.000000}%
\pgfsetstrokecolor{currentstroke}%
\pgfsetdash{}{0pt}%
\pgfusepath{stroke}%
\end{pgfscope}%
\begin{pgfscope}%
\pgfpathrectangle{\pgfqpoint{1.584777in}{1.109166in}}{\pgfqpoint{6.018056in}{3.900000in}} %
\pgfusepath{clip}%
\pgfsetroundcap%
\pgfsetroundjoin%
\pgfsetlinewidth{3.011250pt}%
\definecolor{currentstroke}{rgb}{0.000000,0.000000,0.000000}%
\pgfsetstrokecolor{currentstroke}%
\pgfsetdash{}{0pt}%
\pgfusepath{stroke}%
\end{pgfscope}%
\begin{pgfscope}%
\pgfpathrectangle{\pgfqpoint{1.584777in}{1.109166in}}{\pgfqpoint{6.018056in}{3.900000in}} %
\pgfusepath{clip}%
\pgfsetroundcap%
\pgfsetroundjoin%
\pgfsetlinewidth{3.011250pt}%
\definecolor{currentstroke}{rgb}{0.000000,0.000000,0.000000}%
\pgfsetstrokecolor{currentstroke}%
\pgfsetdash{}{0pt}%
\pgfusepath{stroke}%
\end{pgfscope}%
\begin{pgfscope}%
\pgfpathrectangle{\pgfqpoint{1.584777in}{1.109166in}}{\pgfqpoint{6.018056in}{3.900000in}} %
\pgfusepath{clip}%
\pgfsetroundcap%
\pgfsetroundjoin%
\pgfsetlinewidth{3.011250pt}%
\definecolor{currentstroke}{rgb}{0.000000,0.000000,0.000000}%
\pgfsetstrokecolor{currentstroke}%
\pgfsetdash{}{0pt}%
\pgfusepath{stroke}%
\end{pgfscope}%
\begin{pgfscope}%
\pgfpathrectangle{\pgfqpoint{1.584777in}{1.109166in}}{\pgfqpoint{6.018056in}{3.900000in}} %
\pgfusepath{clip}%
\pgfsetroundcap%
\pgfsetroundjoin%
\pgfsetlinewidth{3.011250pt}%
\definecolor{currentstroke}{rgb}{0.000000,0.000000,0.000000}%
\pgfsetstrokecolor{currentstroke}%
\pgfsetdash{}{0pt}%
\pgfusepath{stroke}%
\end{pgfscope}%
\begin{pgfscope}%
\pgfpathrectangle{\pgfqpoint{1.584777in}{1.109166in}}{\pgfqpoint{6.018056in}{3.900000in}} %
\pgfusepath{clip}%
\pgfsetbuttcap%
\pgfsetroundjoin%
\pgfsetlinewidth{4.015000pt}%
\definecolor{currentstroke}{rgb}{0.333333,0.658824,0.407843}%
\pgfsetstrokecolor{currentstroke}%
\pgfsetdash{{8.000000pt}{4.000000pt}}{0.000000pt}%
\pgfpathmoveto{\pgfqpoint{1.584777in}{3.080004in}}%
\pgfpathlineto{\pgfqpoint{7.602833in}{3.080004in}}%
\pgfusepath{stroke}%
\end{pgfscope}%
\begin{pgfscope}%
\pgfsetrectcap%
\pgfsetmiterjoin%
\pgfsetlinewidth{1.003750pt}%
\definecolor{currentstroke}{rgb}{0.800000,0.800000,0.800000}%
\pgfsetstrokecolor{currentstroke}%
\pgfsetdash{}{0pt}%
\pgfpathmoveto{\pgfqpoint{1.584777in}{1.109166in}}%
\pgfpathlineto{\pgfqpoint{1.584777in}{5.009166in}}%
\pgfusepath{stroke}%
\end{pgfscope}%
\begin{pgfscope}%
\pgfsetrectcap%
\pgfsetmiterjoin%
\pgfsetlinewidth{1.003750pt}%
\definecolor{currentstroke}{rgb}{0.800000,0.800000,0.800000}%
\pgfsetstrokecolor{currentstroke}%
\pgfsetdash{}{0pt}%
\pgfpathmoveto{\pgfqpoint{1.584777in}{1.109166in}}%
\pgfpathlineto{\pgfqpoint{7.602833in}{1.109166in}}%
\pgfusepath{stroke}%
\end{pgfscope}%
\begin{pgfscope}%
\pgfsetbuttcap%
\pgfsetmiterjoin%
\definecolor{currentfill}{rgb}{0.347059,0.458824,0.641176}%
\pgfsetfillcolor{currentfill}%
\pgfsetlinewidth{0.803000pt}%
\definecolor{currentstroke}{rgb}{0.000000,0.000000,0.000000}%
\pgfsetstrokecolor{currentstroke}%
\pgfsetdash{}{0pt}%
\pgfpathmoveto{\pgfqpoint{2.514805in}{5.010944in}}%
\pgfpathlineto{\pgfqpoint{3.137027in}{5.010944in}}%
\pgfpathlineto{\pgfqpoint{3.137027in}{5.283166in}}%
\pgfpathlineto{\pgfqpoint{2.514805in}{5.283166in}}%
\pgfpathclose%
\pgfusepath{stroke,fill}%
\end{pgfscope}%
\begin{pgfscope}%
\pgfsetbuttcap%
\pgfsetmiterjoin%
\definecolor{currentfill}{rgb}{0.347059,0.458824,0.641176}%
\pgfsetfillcolor{currentfill}%
\pgfsetlinewidth{0.803000pt}%
\definecolor{currentstroke}{rgb}{0.000000,0.000000,0.000000}%
\pgfsetstrokecolor{currentstroke}%
\pgfsetdash{}{0pt}%
\pgfpathmoveto{\pgfqpoint{2.514805in}{5.010944in}}%
\pgfpathlineto{\pgfqpoint{3.137027in}{5.010944in}}%
\pgfpathlineto{\pgfqpoint{3.137027in}{5.283166in}}%
\pgfpathlineto{\pgfqpoint{2.514805in}{5.283166in}}%
\pgfpathclose%
\pgfusepath{clip}%
\pgfsys@defobject{currentpattern}{\pgfqpoint{0in}{0in}}{\pgfqpoint{1in}{1in}}{%
\begin{pgfscope}%
\pgfpathrectangle{\pgfqpoint{0in}{0in}}{\pgfqpoint{1in}{1in}}%
\pgfusepath{clip}%
\pgfpathmoveto{\pgfqpoint{-0.500000in}{0.500000in}}%
\pgfpathlineto{\pgfqpoint{0.500000in}{1.500000in}}%
\pgfpathmoveto{\pgfqpoint{-0.333333in}{0.333333in}}%
\pgfpathlineto{\pgfqpoint{0.666667in}{1.333333in}}%
\pgfpathmoveto{\pgfqpoint{-0.166667in}{0.166667in}}%
\pgfpathlineto{\pgfqpoint{0.833333in}{1.166667in}}%
\pgfpathmoveto{\pgfqpoint{0.000000in}{0.000000in}}%
\pgfpathlineto{\pgfqpoint{1.000000in}{1.000000in}}%
\pgfpathmoveto{\pgfqpoint{0.166667in}{-0.166667in}}%
\pgfpathlineto{\pgfqpoint{1.166667in}{0.833333in}}%
\pgfpathmoveto{\pgfqpoint{0.333333in}{-0.333333in}}%
\pgfpathlineto{\pgfqpoint{1.333333in}{0.666667in}}%
\pgfpathmoveto{\pgfqpoint{0.500000in}{-0.500000in}}%
\pgfpathlineto{\pgfqpoint{1.500000in}{0.500000in}}%
\pgfusepath{stroke}%
\end{pgfscope}%
}%
\pgfsys@transformshift{2.514805in}{5.010944in}%
\pgfsys@useobject{currentpattern}{}%
\pgfsys@transformshift{1in}{0in}%
\pgfsys@transformshift{-1in}{0in}%
\pgfsys@transformshift{0in}{1in}%
\end{pgfscope}%
\begin{pgfscope}%
\definecolor{textcolor}{rgb}{0.150000,0.150000,0.150000}%
\pgfsetstrokecolor{textcolor}%
\pgfsetfillcolor{textcolor}%
\pgftext[x=3.175916in,y=5.010944in,left,base]{\color{textcolor}\rmfamily\fontsize{28.000000}{33.600000}\selectfont SA}%
\end{pgfscope}%
\begin{pgfscope}%
\pgfsetbuttcap%
\pgfsetmiterjoin%
\definecolor{currentfill}{rgb}{0.798529,0.536765,0.389706}%
\pgfsetfillcolor{currentfill}%
\pgfsetlinewidth{0.803000pt}%
\definecolor{currentstroke}{rgb}{0.000000,0.000000,0.000000}%
\pgfsetstrokecolor{currentstroke}%
\pgfsetdash{}{0pt}%
\pgfpathmoveto{\pgfqpoint{3.838583in}{5.010944in}}%
\pgfpathlineto{\pgfqpoint{4.460805in}{5.010944in}}%
\pgfpathlineto{\pgfqpoint{4.460805in}{5.283166in}}%
\pgfpathlineto{\pgfqpoint{3.838583in}{5.283166in}}%
\pgfpathclose%
\pgfusepath{stroke,fill}%
\end{pgfscope}%
\begin{pgfscope}%
\pgfsetbuttcap%
\pgfsetmiterjoin%
\definecolor{currentfill}{rgb}{0.798529,0.536765,0.389706}%
\pgfsetfillcolor{currentfill}%
\pgfsetlinewidth{0.803000pt}%
\definecolor{currentstroke}{rgb}{0.000000,0.000000,0.000000}%
\pgfsetstrokecolor{currentstroke}%
\pgfsetdash{}{0pt}%
\pgfpathmoveto{\pgfqpoint{3.838583in}{5.010944in}}%
\pgfpathlineto{\pgfqpoint{4.460805in}{5.010944in}}%
\pgfpathlineto{\pgfqpoint{4.460805in}{5.283166in}}%
\pgfpathlineto{\pgfqpoint{3.838583in}{5.283166in}}%
\pgfpathclose%
\pgfusepath{clip}%
\pgfsys@defobject{currentpattern}{\pgfqpoint{0in}{0in}}{\pgfqpoint{1in}{1in}}{%
\begin{pgfscope}%
\pgfpathrectangle{\pgfqpoint{0in}{0in}}{\pgfqpoint{1in}{1in}}%
\pgfusepath{clip}%
\pgfpathmoveto{\pgfqpoint{-0.500000in}{0.500000in}}%
\pgfpathlineto{\pgfqpoint{0.500000in}{1.500000in}}%
\pgfpathmoveto{\pgfqpoint{-0.333333in}{0.333333in}}%
\pgfpathlineto{\pgfqpoint{0.666667in}{1.333333in}}%
\pgfpathmoveto{\pgfqpoint{-0.166667in}{0.166667in}}%
\pgfpathlineto{\pgfqpoint{0.833333in}{1.166667in}}%
\pgfpathmoveto{\pgfqpoint{0.000000in}{0.000000in}}%
\pgfpathlineto{\pgfqpoint{1.000000in}{1.000000in}}%
\pgfpathmoveto{\pgfqpoint{0.166667in}{-0.166667in}}%
\pgfpathlineto{\pgfqpoint{1.166667in}{0.833333in}}%
\pgfpathmoveto{\pgfqpoint{0.333333in}{-0.333333in}}%
\pgfpathlineto{\pgfqpoint{1.333333in}{0.666667in}}%
\pgfpathmoveto{\pgfqpoint{0.500000in}{-0.500000in}}%
\pgfpathlineto{\pgfqpoint{1.500000in}{0.500000in}}%
\pgfpathmoveto{\pgfqpoint{-0.500000in}{0.500000in}}%
\pgfpathlineto{\pgfqpoint{0.500000in}{-0.500000in}}%
\pgfpathmoveto{\pgfqpoint{-0.333333in}{0.666667in}}%
\pgfpathlineto{\pgfqpoint{0.666667in}{-0.333333in}}%
\pgfpathmoveto{\pgfqpoint{-0.166667in}{0.833333in}}%
\pgfpathlineto{\pgfqpoint{0.833333in}{-0.166667in}}%
\pgfpathmoveto{\pgfqpoint{0.000000in}{1.000000in}}%
\pgfpathlineto{\pgfqpoint{1.000000in}{0.000000in}}%
\pgfpathmoveto{\pgfqpoint{0.166667in}{1.166667in}}%
\pgfpathlineto{\pgfqpoint{1.166667in}{0.166667in}}%
\pgfpathmoveto{\pgfqpoint{0.333333in}{1.333333in}}%
\pgfpathlineto{\pgfqpoint{1.333333in}{0.333333in}}%
\pgfpathmoveto{\pgfqpoint{0.500000in}{1.500000in}}%
\pgfpathlineto{\pgfqpoint{1.500000in}{0.500000in}}%
\pgfusepath{stroke}%
\end{pgfscope}%
}%
\pgfsys@transformshift{3.838583in}{5.010944in}%
\pgfsys@useobject{currentpattern}{}%
\pgfsys@transformshift{1in}{0in}%
\pgfsys@transformshift{-1in}{0in}%
\pgfsys@transformshift{0in}{1in}%
\end{pgfscope}%
\begin{pgfscope}%
\definecolor{textcolor}{rgb}{0.150000,0.150000,0.150000}%
\pgfsetstrokecolor{textcolor}%
\pgfsetfillcolor{textcolor}%
\pgftext[x=4.499694in,y=5.010944in,left,base]{\color{textcolor}\rmfamily\fontsize{28.000000}{33.600000}\selectfont HM}%
\end{pgfscope}%
\begin{pgfscope}%
\pgfsetbuttcap%
\pgfsetroundjoin%
\pgfsetlinewidth{4.015000pt}%
\definecolor{currentstroke}{rgb}{0.333333,0.658824,0.407843}%
\pgfsetstrokecolor{currentstroke}%
\pgfsetdash{{8.000000pt}{4.000000pt}}{0.000000pt}%
\pgfpathmoveto{\pgfqpoint{5.294194in}{5.147055in}}%
\pgfpathlineto{\pgfqpoint{5.916416in}{5.147055in}}%
\pgfusepath{stroke}%
\end{pgfscope}%
\begin{pgfscope}%
\definecolor{textcolor}{rgb}{0.150000,0.150000,0.150000}%
\pgfsetstrokecolor{textcolor}%
\pgfsetfillcolor{textcolor}%
\pgftext[x=5.955305in,y=5.010944in,left,base]{\color{textcolor}\rmfamily\fontsize{28.000000}{33.600000}\selectfont ALL}%
\end{pgfscope}%
\end{pgfpicture}%
\makeatother%
\endgroup%
%
}
\caption{Total run time of all the workloads with different budgets}
\label{exp-runtime-vs-mat-budget}
\end{subfigure}
\begin{subfigure}[b]{0.5\linewidth}
\centering
 \resizebox{\columnwidth}{!}{%
\input{../images/experiment-results/kaggle_home_credit/materialization/size-8.pgf}%
}
\caption{Budget = 8 GB}
\end{subfigure}%
\begin{subfigure}[b]{0.5\linewidth}
\centering
 \resizebox{\columnwidth}{!}{%
%% Creator: Matplotlib, PGF backend
%%
%% To include the figure in your LaTeX document, write
%%   \input{<filename>.pgf}
%%
%% Make sure the required packages are loaded in your preamble
%%   \usepackage{pgf}
%%
%% Figures using additional raster images can only be included by \input if
%% they are in the same directory as the main LaTeX file. For loading figures
%% from other directories you can use the `import` package
%%   \usepackage{import}
%% and then include the figures with
%%   \import{<path to file>}{<filename>.pgf}
%%
%% Matplotlib used the following preamble
%%   \usepackage[utf8x]{inputenc}
%%   \usepackage[T1]{fontenc}
%%
\begingroup%
\makeatletter%
\begin{pgfpicture}%
\pgfpathrectangle{\pgfpointorigin}{\pgfqpoint{7.974532in}{4.884138in}}%
\pgfusepath{use as bounding box, clip}%
\begin{pgfscope}%
\pgfsetbuttcap%
\pgfsetmiterjoin%
\definecolor{currentfill}{rgb}{1.000000,1.000000,1.000000}%
\pgfsetfillcolor{currentfill}%
\pgfsetlinewidth{0.000000pt}%
\definecolor{currentstroke}{rgb}{1.000000,1.000000,1.000000}%
\pgfsetstrokecolor{currentstroke}%
\pgfsetdash{}{0pt}%
\pgfpathmoveto{\pgfqpoint{0.000000in}{0.000000in}}%
\pgfpathlineto{\pgfqpoint{7.974532in}{0.000000in}}%
\pgfpathlineto{\pgfqpoint{7.974532in}{4.884138in}}%
\pgfpathlineto{\pgfqpoint{0.000000in}{4.884138in}}%
\pgfpathclose%
\pgfusepath{fill}%
\end{pgfscope}%
\begin{pgfscope}%
\pgfsetbuttcap%
\pgfsetmiterjoin%
\definecolor{currentfill}{rgb}{1.000000,1.000000,1.000000}%
\pgfsetfillcolor{currentfill}%
\pgfsetlinewidth{0.000000pt}%
\definecolor{currentstroke}{rgb}{0.000000,0.000000,0.000000}%
\pgfsetstrokecolor{currentstroke}%
\pgfsetstrokeopacity{0.000000}%
\pgfsetdash{}{0pt}%
\pgfpathmoveto{\pgfqpoint{1.307384in}{1.215049in}}%
\pgfpathlineto{\pgfqpoint{7.507384in}{1.215049in}}%
\pgfpathlineto{\pgfqpoint{7.507384in}{4.235049in}}%
\pgfpathlineto{\pgfqpoint{1.307384in}{4.235049in}}%
\pgfpathclose%
\pgfusepath{fill}%
\end{pgfscope}%
\begin{pgfscope}%
\pgfpathrectangle{\pgfqpoint{1.307384in}{1.215049in}}{\pgfqpoint{6.200000in}{3.020000in}} %
\pgfusepath{clip}%
\pgfsetroundcap%
\pgfsetroundjoin%
\pgfsetlinewidth{1.003750pt}%
\definecolor{currentstroke}{rgb}{0.500000,0.500000,0.500000}%
\pgfsetstrokecolor{currentstroke}%
\pgfsetdash{}{0pt}%
\pgfpathmoveto{\pgfqpoint{1.589202in}{1.215049in}}%
\pgfpathlineto{\pgfqpoint{1.589202in}{4.235049in}}%
\pgfusepath{stroke}%
\end{pgfscope}%
\begin{pgfscope}%
\definecolor{textcolor}{rgb}{0.150000,0.150000,0.150000}%
\pgfsetstrokecolor{textcolor}%
\pgfsetfillcolor{textcolor}%
\pgftext[x=1.589202in,y=1.034493in,,top]{\color{textcolor}\rmfamily\fontsize{34.000000}{40.800000}\selectfont 1}%
\end{pgfscope}%
\begin{pgfscope}%
\pgfpathrectangle{\pgfqpoint{1.307384in}{1.215049in}}{\pgfqpoint{6.200000in}{3.020000in}} %
\pgfusepath{clip}%
\pgfsetroundcap%
\pgfsetroundjoin%
\pgfsetlinewidth{1.003750pt}%
\definecolor{currentstroke}{rgb}{0.500000,0.500000,0.500000}%
\pgfsetstrokecolor{currentstroke}%
\pgfsetdash{}{0pt}%
\pgfpathmoveto{\pgfqpoint{2.394397in}{1.215049in}}%
\pgfpathlineto{\pgfqpoint{2.394397in}{4.235049in}}%
\pgfusepath{stroke}%
\end{pgfscope}%
\begin{pgfscope}%
\definecolor{textcolor}{rgb}{0.150000,0.150000,0.150000}%
\pgfsetstrokecolor{textcolor}%
\pgfsetfillcolor{textcolor}%
\pgftext[x=2.394397in,y=1.034493in,,top]{\color{textcolor}\rmfamily\fontsize{34.000000}{40.800000}\selectfont 2}%
\end{pgfscope}%
\begin{pgfscope}%
\pgfpathrectangle{\pgfqpoint{1.307384in}{1.215049in}}{\pgfqpoint{6.200000in}{3.020000in}} %
\pgfusepath{clip}%
\pgfsetroundcap%
\pgfsetroundjoin%
\pgfsetlinewidth{1.003750pt}%
\definecolor{currentstroke}{rgb}{0.500000,0.500000,0.500000}%
\pgfsetstrokecolor{currentstroke}%
\pgfsetdash{}{0pt}%
\pgfpathmoveto{\pgfqpoint{3.199592in}{1.215049in}}%
\pgfpathlineto{\pgfqpoint{3.199592in}{4.235049in}}%
\pgfusepath{stroke}%
\end{pgfscope}%
\begin{pgfscope}%
\definecolor{textcolor}{rgb}{0.150000,0.150000,0.150000}%
\pgfsetstrokecolor{textcolor}%
\pgfsetfillcolor{textcolor}%
\pgftext[x=3.199592in,y=1.034493in,,top]{\color{textcolor}\rmfamily\fontsize{34.000000}{40.800000}\selectfont 3}%
\end{pgfscope}%
\begin{pgfscope}%
\pgfpathrectangle{\pgfqpoint{1.307384in}{1.215049in}}{\pgfqpoint{6.200000in}{3.020000in}} %
\pgfusepath{clip}%
\pgfsetroundcap%
\pgfsetroundjoin%
\pgfsetlinewidth{1.003750pt}%
\definecolor{currentstroke}{rgb}{0.500000,0.500000,0.500000}%
\pgfsetstrokecolor{currentstroke}%
\pgfsetdash{}{0pt}%
\pgfpathmoveto{\pgfqpoint{4.004786in}{1.215049in}}%
\pgfpathlineto{\pgfqpoint{4.004786in}{4.235049in}}%
\pgfusepath{stroke}%
\end{pgfscope}%
\begin{pgfscope}%
\definecolor{textcolor}{rgb}{0.150000,0.150000,0.150000}%
\pgfsetstrokecolor{textcolor}%
\pgfsetfillcolor{textcolor}%
\pgftext[x=4.004786in,y=1.034493in,,top]{\color{textcolor}\rmfamily\fontsize{34.000000}{40.800000}\selectfont 4}%
\end{pgfscope}%
\begin{pgfscope}%
\pgfpathrectangle{\pgfqpoint{1.307384in}{1.215049in}}{\pgfqpoint{6.200000in}{3.020000in}} %
\pgfusepath{clip}%
\pgfsetroundcap%
\pgfsetroundjoin%
\pgfsetlinewidth{1.003750pt}%
\definecolor{currentstroke}{rgb}{0.500000,0.500000,0.500000}%
\pgfsetstrokecolor{currentstroke}%
\pgfsetdash{}{0pt}%
\pgfpathmoveto{\pgfqpoint{4.809981in}{1.215049in}}%
\pgfpathlineto{\pgfqpoint{4.809981in}{4.235049in}}%
\pgfusepath{stroke}%
\end{pgfscope}%
\begin{pgfscope}%
\definecolor{textcolor}{rgb}{0.150000,0.150000,0.150000}%
\pgfsetstrokecolor{textcolor}%
\pgfsetfillcolor{textcolor}%
\pgftext[x=4.809981in,y=1.034493in,,top]{\color{textcolor}\rmfamily\fontsize{34.000000}{40.800000}\selectfont 5}%
\end{pgfscope}%
\begin{pgfscope}%
\pgfpathrectangle{\pgfqpoint{1.307384in}{1.215049in}}{\pgfqpoint{6.200000in}{3.020000in}} %
\pgfusepath{clip}%
\pgfsetroundcap%
\pgfsetroundjoin%
\pgfsetlinewidth{1.003750pt}%
\definecolor{currentstroke}{rgb}{0.500000,0.500000,0.500000}%
\pgfsetstrokecolor{currentstroke}%
\pgfsetdash{}{0pt}%
\pgfpathmoveto{\pgfqpoint{5.615176in}{1.215049in}}%
\pgfpathlineto{\pgfqpoint{5.615176in}{4.235049in}}%
\pgfusepath{stroke}%
\end{pgfscope}%
\begin{pgfscope}%
\definecolor{textcolor}{rgb}{0.150000,0.150000,0.150000}%
\pgfsetstrokecolor{textcolor}%
\pgfsetfillcolor{textcolor}%
\pgftext[x=5.615176in,y=1.034493in,,top]{\color{textcolor}\rmfamily\fontsize{34.000000}{40.800000}\selectfont 6}%
\end{pgfscope}%
\begin{pgfscope}%
\pgfpathrectangle{\pgfqpoint{1.307384in}{1.215049in}}{\pgfqpoint{6.200000in}{3.020000in}} %
\pgfusepath{clip}%
\pgfsetroundcap%
\pgfsetroundjoin%
\pgfsetlinewidth{1.003750pt}%
\definecolor{currentstroke}{rgb}{0.500000,0.500000,0.500000}%
\pgfsetstrokecolor{currentstroke}%
\pgfsetdash{}{0pt}%
\pgfpathmoveto{\pgfqpoint{6.420371in}{1.215049in}}%
\pgfpathlineto{\pgfqpoint{6.420371in}{4.235049in}}%
\pgfusepath{stroke}%
\end{pgfscope}%
\begin{pgfscope}%
\definecolor{textcolor}{rgb}{0.150000,0.150000,0.150000}%
\pgfsetstrokecolor{textcolor}%
\pgfsetfillcolor{textcolor}%
\pgftext[x=6.420371in,y=1.034493in,,top]{\color{textcolor}\rmfamily\fontsize{34.000000}{40.800000}\selectfont 7}%
\end{pgfscope}%
\begin{pgfscope}%
\pgfpathrectangle{\pgfqpoint{1.307384in}{1.215049in}}{\pgfqpoint{6.200000in}{3.020000in}} %
\pgfusepath{clip}%
\pgfsetroundcap%
\pgfsetroundjoin%
\pgfsetlinewidth{1.003750pt}%
\definecolor{currentstroke}{rgb}{0.500000,0.500000,0.500000}%
\pgfsetstrokecolor{currentstroke}%
\pgfsetdash{}{0pt}%
\pgfpathmoveto{\pgfqpoint{7.225566in}{1.215049in}}%
\pgfpathlineto{\pgfqpoint{7.225566in}{4.235049in}}%
\pgfusepath{stroke}%
\end{pgfscope}%
\begin{pgfscope}%
\definecolor{textcolor}{rgb}{0.150000,0.150000,0.150000}%
\pgfsetstrokecolor{textcolor}%
\pgfsetfillcolor{textcolor}%
\pgftext[x=7.225566in,y=1.034493in,,top]{\color{textcolor}\rmfamily\fontsize{34.000000}{40.800000}\selectfont 8}%
\end{pgfscope}%
\begin{pgfscope}%
\definecolor{textcolor}{rgb}{0.150000,0.150000,0.150000}%
\pgfsetstrokecolor{textcolor}%
\pgfsetfillcolor{textcolor}%
\pgftext[x=4.407384in,y=0.539469in,,top]{\color{textcolor}\rmfamily\fontsize{40.000000}{48.000000}\selectfont Workload}%
\end{pgfscope}%
\begin{pgfscope}%
\pgfpathrectangle{\pgfqpoint{1.307384in}{1.215049in}}{\pgfqpoint{6.200000in}{3.020000in}} %
\pgfusepath{clip}%
\pgfsetroundcap%
\pgfsetroundjoin%
\pgfsetlinewidth{1.003750pt}%
\definecolor{currentstroke}{rgb}{0.500000,0.500000,0.500000}%
\pgfsetstrokecolor{currentstroke}%
\pgfsetdash{}{0pt}%
\pgfpathmoveto{\pgfqpoint{1.307384in}{1.215049in}}%
\pgfpathlineto{\pgfqpoint{7.507384in}{1.215049in}}%
\pgfusepath{stroke}%
\end{pgfscope}%
\begin{pgfscope}%
\definecolor{textcolor}{rgb}{0.150000,0.150000,0.150000}%
\pgfsetstrokecolor{textcolor}%
\pgfsetfillcolor{textcolor}%
\pgftext[x=0.968350in,y=1.043716in,left,base]{\color{textcolor}\rmfamily\fontsize{34.000000}{40.800000}\selectfont \(\displaystyle 0\)}%
\end{pgfscope}%
\begin{pgfscope}%
\pgfpathrectangle{\pgfqpoint{1.307384in}{1.215049in}}{\pgfqpoint{6.200000in}{3.020000in}} %
\pgfusepath{clip}%
\pgfsetroundcap%
\pgfsetroundjoin%
\pgfsetlinewidth{1.003750pt}%
\definecolor{currentstroke}{rgb}{0.500000,0.500000,0.500000}%
\pgfsetstrokecolor{currentstroke}%
\pgfsetdash{}{0pt}%
\pgfpathmoveto{\pgfqpoint{1.307384in}{1.754335in}}%
\pgfpathlineto{\pgfqpoint{7.507384in}{1.754335in}}%
\pgfusepath{stroke}%
\end{pgfscope}%
\begin{pgfscope}%
\definecolor{textcolor}{rgb}{0.150000,0.150000,0.150000}%
\pgfsetstrokecolor{textcolor}%
\pgfsetfillcolor{textcolor}%
\pgftext[x=0.809872in,y=1.583001in,left,base]{\color{textcolor}\rmfamily\fontsize{34.000000}{40.800000}\selectfont \(\displaystyle 25\)}%
\end{pgfscope}%
\begin{pgfscope}%
\pgfpathrectangle{\pgfqpoint{1.307384in}{1.215049in}}{\pgfqpoint{6.200000in}{3.020000in}} %
\pgfusepath{clip}%
\pgfsetroundcap%
\pgfsetroundjoin%
\pgfsetlinewidth{1.003750pt}%
\definecolor{currentstroke}{rgb}{0.500000,0.500000,0.500000}%
\pgfsetstrokecolor{currentstroke}%
\pgfsetdash{}{0pt}%
\pgfpathmoveto{\pgfqpoint{1.307384in}{2.293620in}}%
\pgfpathlineto{\pgfqpoint{7.507384in}{2.293620in}}%
\pgfusepath{stroke}%
\end{pgfscope}%
\begin{pgfscope}%
\definecolor{textcolor}{rgb}{0.150000,0.150000,0.150000}%
\pgfsetstrokecolor{textcolor}%
\pgfsetfillcolor{textcolor}%
\pgftext[x=0.809872in,y=2.122287in,left,base]{\color{textcolor}\rmfamily\fontsize{34.000000}{40.800000}\selectfont \(\displaystyle 50\)}%
\end{pgfscope}%
\begin{pgfscope}%
\pgfpathrectangle{\pgfqpoint{1.307384in}{1.215049in}}{\pgfqpoint{6.200000in}{3.020000in}} %
\pgfusepath{clip}%
\pgfsetroundcap%
\pgfsetroundjoin%
\pgfsetlinewidth{1.003750pt}%
\definecolor{currentstroke}{rgb}{0.500000,0.500000,0.500000}%
\pgfsetstrokecolor{currentstroke}%
\pgfsetdash{}{0pt}%
\pgfpathmoveto{\pgfqpoint{1.307384in}{2.832906in}}%
\pgfpathlineto{\pgfqpoint{7.507384in}{2.832906in}}%
\pgfusepath{stroke}%
\end{pgfscope}%
\begin{pgfscope}%
\definecolor{textcolor}{rgb}{0.150000,0.150000,0.150000}%
\pgfsetstrokecolor{textcolor}%
\pgfsetfillcolor{textcolor}%
\pgftext[x=0.809872in,y=2.661573in,left,base]{\color{textcolor}\rmfamily\fontsize{34.000000}{40.800000}\selectfont \(\displaystyle 75\)}%
\end{pgfscope}%
\begin{pgfscope}%
\pgfpathrectangle{\pgfqpoint{1.307384in}{1.215049in}}{\pgfqpoint{6.200000in}{3.020000in}} %
\pgfusepath{clip}%
\pgfsetroundcap%
\pgfsetroundjoin%
\pgfsetlinewidth{1.003750pt}%
\definecolor{currentstroke}{rgb}{0.500000,0.500000,0.500000}%
\pgfsetstrokecolor{currentstroke}%
\pgfsetdash{}{0pt}%
\pgfpathmoveto{\pgfqpoint{1.307384in}{3.372192in}}%
\pgfpathlineto{\pgfqpoint{7.507384in}{3.372192in}}%
\pgfusepath{stroke}%
\end{pgfscope}%
\begin{pgfscope}%
\definecolor{textcolor}{rgb}{0.150000,0.150000,0.150000}%
\pgfsetstrokecolor{textcolor}%
\pgfsetfillcolor{textcolor}%
\pgftext[x=0.651394in,y=3.200859in,left,base]{\color{textcolor}\rmfamily\fontsize{34.000000}{40.800000}\selectfont \(\displaystyle 100\)}%
\end{pgfscope}%
\begin{pgfscope}%
\pgfpathrectangle{\pgfqpoint{1.307384in}{1.215049in}}{\pgfqpoint{6.200000in}{3.020000in}} %
\pgfusepath{clip}%
\pgfsetroundcap%
\pgfsetroundjoin%
\pgfsetlinewidth{1.003750pt}%
\definecolor{currentstroke}{rgb}{0.500000,0.500000,0.500000}%
\pgfsetstrokecolor{currentstroke}%
\pgfsetdash{}{0pt}%
\pgfpathmoveto{\pgfqpoint{1.307384in}{3.911477in}}%
\pgfpathlineto{\pgfqpoint{7.507384in}{3.911477in}}%
\pgfusepath{stroke}%
\end{pgfscope}%
\begin{pgfscope}%
\definecolor{textcolor}{rgb}{0.150000,0.150000,0.150000}%
\pgfsetstrokecolor{textcolor}%
\pgfsetfillcolor{textcolor}%
\pgftext[x=0.651394in,y=3.740144in,left,base]{\color{textcolor}\rmfamily\fontsize{34.000000}{40.800000}\selectfont \(\displaystyle 125\)}%
\end{pgfscope}%
\begin{pgfscope}%
\definecolor{textcolor}{rgb}{0.150000,0.150000,0.150000}%
\pgfsetstrokecolor{textcolor}%
\pgfsetfillcolor{textcolor}%
\pgftext[x=0.595839in,y=2.725049in,,bottom,rotate=90.000000]{\color{textcolor}\rmfamily\fontsize{40.000000}{48.000000}\selectfont Size (GB)}%
\end{pgfscope}%
\begin{pgfscope}%
\pgfpathrectangle{\pgfqpoint{1.307384in}{1.215049in}}{\pgfqpoint{6.200000in}{3.020000in}} %
\pgfusepath{clip}%
\pgfsetbuttcap%
\pgfsetroundjoin%
\definecolor{currentfill}{rgb}{0.298039,0.447059,0.690196}%
\pgfsetfillcolor{currentfill}%
\pgfsetfillopacity{0.200000}%
\pgfsetlinewidth{1.003750pt}%
\definecolor{currentstroke}{rgb}{0.298039,0.447059,0.690196}%
\pgfsetstrokecolor{currentstroke}%
\pgfsetstrokeopacity{0.200000}%
\pgfsetdash{}{0pt}%
\pgfpathmoveto{\pgfqpoint{1.589202in}{1.528456in}}%
\pgfpathlineto{\pgfqpoint{1.589202in}{1.528456in}}%
\pgfpathlineto{\pgfqpoint{2.394397in}{2.045252in}}%
\pgfpathlineto{\pgfqpoint{3.199592in}{2.355195in}}%
\pgfpathlineto{\pgfqpoint{4.004786in}{2.480319in}}%
\pgfpathlineto{\pgfqpoint{4.809981in}{2.602371in}}%
\pgfpathlineto{\pgfqpoint{5.615176in}{2.861339in}}%
\pgfpathlineto{\pgfqpoint{6.420371in}{3.248992in}}%
\pgfpathlineto{\pgfqpoint{7.225566in}{3.283720in}}%
\pgfpathlineto{\pgfqpoint{7.225566in}{3.325435in}}%
\pgfpathlineto{\pgfqpoint{7.225566in}{3.325435in}}%
\pgfpathlineto{\pgfqpoint{6.420371in}{3.290708in}}%
\pgfpathlineto{\pgfqpoint{5.615176in}{2.886451in}}%
\pgfpathlineto{\pgfqpoint{4.809981in}{2.633246in}}%
\pgfpathlineto{\pgfqpoint{4.004786in}{2.511193in}}%
\pgfpathlineto{\pgfqpoint{3.199592in}{2.386070in}}%
\pgfpathlineto{\pgfqpoint{2.394397in}{2.045528in}}%
\pgfpathlineto{\pgfqpoint{1.589202in}{1.528456in}}%
\pgfpathclose%
\pgfusepath{stroke,fill}%
\end{pgfscope}%
\begin{pgfscope}%
\pgfpathrectangle{\pgfqpoint{1.307384in}{1.215049in}}{\pgfqpoint{6.200000in}{3.020000in}} %
\pgfusepath{clip}%
\pgfsetbuttcap%
\pgfsetroundjoin%
\definecolor{currentfill}{rgb}{0.866667,0.517647,0.321569}%
\pgfsetfillcolor{currentfill}%
\pgfsetfillopacity{0.200000}%
\pgfsetlinewidth{1.003750pt}%
\definecolor{currentstroke}{rgb}{0.866667,0.517647,0.321569}%
\pgfsetstrokecolor{currentstroke}%
\pgfsetstrokeopacity{0.200000}%
\pgfsetdash{}{0pt}%
\pgfpathmoveto{\pgfqpoint{1.589202in}{1.528456in}}%
\pgfpathlineto{\pgfqpoint{1.589202in}{1.528456in}}%
\pgfpathlineto{\pgfqpoint{2.394397in}{1.614056in}}%
\pgfpathlineto{\pgfqpoint{3.199592in}{1.704411in}}%
\pgfpathlineto{\pgfqpoint{4.004786in}{1.704411in}}%
\pgfpathlineto{\pgfqpoint{4.809981in}{1.704410in}}%
\pgfpathlineto{\pgfqpoint{5.615176in}{1.704411in}}%
\pgfpathlineto{\pgfqpoint{6.420371in}{1.704165in}}%
\pgfpathlineto{\pgfqpoint{7.225566in}{1.704411in}}%
\pgfpathlineto{\pgfqpoint{7.225566in}{1.704411in}}%
\pgfpathlineto{\pgfqpoint{7.225566in}{1.704411in}}%
\pgfpathlineto{\pgfqpoint{6.420371in}{1.704266in}}%
\pgfpathlineto{\pgfqpoint{5.615176in}{1.704411in}}%
\pgfpathlineto{\pgfqpoint{4.809981in}{1.704411in}}%
\pgfpathlineto{\pgfqpoint{4.004786in}{1.704411in}}%
\pgfpathlineto{\pgfqpoint{3.199592in}{1.704411in}}%
\pgfpathlineto{\pgfqpoint{2.394397in}{1.614056in}}%
\pgfpathlineto{\pgfqpoint{1.589202in}{1.528456in}}%
\pgfpathclose%
\pgfusepath{stroke,fill}%
\end{pgfscope}%
\begin{pgfscope}%
\pgfpathrectangle{\pgfqpoint{1.307384in}{1.215049in}}{\pgfqpoint{6.200000in}{3.020000in}} %
\pgfusepath{clip}%
\pgfsetbuttcap%
\pgfsetroundjoin%
\definecolor{currentfill}{rgb}{0.333333,0.658824,0.407843}%
\pgfsetfillcolor{currentfill}%
\pgfsetfillopacity{0.200000}%
\pgfsetlinewidth{1.003750pt}%
\definecolor{currentstroke}{rgb}{0.333333,0.658824,0.407843}%
\pgfsetstrokecolor{currentstroke}%
\pgfsetstrokeopacity{0.200000}%
\pgfsetdash{}{0pt}%
\pgfpathmoveto{\pgfqpoint{1.589202in}{1.510416in}}%
\pgfpathlineto{\pgfqpoint{1.589202in}{1.510416in}}%
\pgfpathlineto{\pgfqpoint{2.394397in}{1.594156in}}%
\pgfpathlineto{\pgfqpoint{3.199592in}{1.605531in}}%
\pgfpathlineto{\pgfqpoint{4.004786in}{1.605531in}}%
\pgfpathlineto{\pgfqpoint{4.809981in}{1.605531in}}%
\pgfpathlineto{\pgfqpoint{5.615176in}{1.567572in}}%
\pgfpathlineto{\pgfqpoint{6.420371in}{1.625569in}}%
\pgfpathlineto{\pgfqpoint{7.225566in}{1.647065in}}%
\pgfpathlineto{\pgfqpoint{7.225566in}{1.647065in}}%
\pgfpathlineto{\pgfqpoint{7.225566in}{1.647065in}}%
\pgfpathlineto{\pgfqpoint{6.420371in}{1.625569in}}%
\pgfpathlineto{\pgfqpoint{5.615176in}{1.567572in}}%
\pgfpathlineto{\pgfqpoint{4.809981in}{1.605531in}}%
\pgfpathlineto{\pgfqpoint{4.004786in}{1.605531in}}%
\pgfpathlineto{\pgfqpoint{3.199592in}{1.605531in}}%
\pgfpathlineto{\pgfqpoint{2.394397in}{1.594156in}}%
\pgfpathlineto{\pgfqpoint{1.589202in}{1.510416in}}%
\pgfpathclose%
\pgfusepath{stroke,fill}%
\end{pgfscope}%
\begin{pgfscope}%
\pgfpathrectangle{\pgfqpoint{1.307384in}{1.215049in}}{\pgfqpoint{6.200000in}{3.020000in}} %
\pgfusepath{clip}%
\pgfsetbuttcap%
\pgfsetroundjoin%
\definecolor{currentfill}{rgb}{0.768627,0.305882,0.321569}%
\pgfsetfillcolor{currentfill}%
\pgfsetfillopacity{0.200000}%
\pgfsetlinewidth{1.003750pt}%
\definecolor{currentstroke}{rgb}{0.768627,0.305882,0.321569}%
\pgfsetstrokecolor{currentstroke}%
\pgfsetstrokeopacity{0.200000}%
\pgfsetdash{}{0pt}%
\pgfpathmoveto{\pgfqpoint{1.589202in}{1.528456in}}%
\pgfpathlineto{\pgfqpoint{1.589202in}{1.528456in}}%
\pgfpathlineto{\pgfqpoint{2.394397in}{2.055351in}}%
\pgfpathlineto{\pgfqpoint{3.199592in}{3.845298in}}%
\pgfpathlineto{\pgfqpoint{4.004786in}{3.845319in}}%
\pgfpathlineto{\pgfqpoint{4.809981in}{3.967372in}}%
\pgfpathlineto{\pgfqpoint{5.615176in}{3.967441in}}%
\pgfpathlineto{\pgfqpoint{6.420371in}{3.967441in}}%
\pgfpathlineto{\pgfqpoint{7.225566in}{3.977407in}}%
\pgfpathlineto{\pgfqpoint{7.225566in}{3.977407in}}%
\pgfpathlineto{\pgfqpoint{7.225566in}{3.977407in}}%
\pgfpathlineto{\pgfqpoint{6.420371in}{3.967441in}}%
\pgfpathlineto{\pgfqpoint{5.615176in}{3.967441in}}%
\pgfpathlineto{\pgfqpoint{4.809981in}{3.967372in}}%
\pgfpathlineto{\pgfqpoint{4.004786in}{3.845319in}}%
\pgfpathlineto{\pgfqpoint{3.199592in}{3.845298in}}%
\pgfpathlineto{\pgfqpoint{2.394397in}{2.055351in}}%
\pgfpathlineto{\pgfqpoint{1.589202in}{1.528456in}}%
\pgfpathclose%
\pgfusepath{stroke,fill}%
\end{pgfscope}%
\begin{pgfscope}%
\pgfpathrectangle{\pgfqpoint{1.307384in}{1.215049in}}{\pgfqpoint{6.200000in}{3.020000in}} %
\pgfusepath{clip}%
\pgfsetbuttcap%
\pgfsetroundjoin%
\pgfsetlinewidth{8.030000pt}%
\definecolor{currentstroke}{rgb}{0.298039,0.447059,0.690196}%
\pgfsetstrokecolor{currentstroke}%
\pgfsetdash{{8.000000pt}{0.000000pt}}{0.000000pt}%
\pgfpathmoveto{\pgfqpoint{1.589202in}{1.528456in}}%
\pgfpathlineto{\pgfqpoint{2.394397in}{2.045436in}}%
\pgfpathlineto{\pgfqpoint{3.199592in}{2.372609in}}%
\pgfpathlineto{\pgfqpoint{4.004786in}{2.497733in}}%
\pgfpathlineto{\pgfqpoint{4.809981in}{2.619785in}}%
\pgfpathlineto{\pgfqpoint{5.615176in}{2.871680in}}%
\pgfpathlineto{\pgfqpoint{6.420371in}{3.264868in}}%
\pgfpathlineto{\pgfqpoint{7.225566in}{3.299596in}}%
\pgfusepath{stroke}%
\end{pgfscope}%
\begin{pgfscope}%
\pgfpathrectangle{\pgfqpoint{1.307384in}{1.215049in}}{\pgfqpoint{6.200000in}{3.020000in}} %
\pgfusepath{clip}%
\pgfsetbuttcap%
\pgfsetroundjoin%
\definecolor{currentfill}{rgb}{0.298039,0.447059,0.690196}%
\pgfsetfillcolor{currentfill}%
\pgfsetlinewidth{0.752812pt}%
\definecolor{currentstroke}{rgb}{1.000000,1.000000,1.000000}%
\pgfsetstrokecolor{currentstroke}%
\pgfsetdash{}{0pt}%
\pgfsys@defobject{currentmarker}{\pgfqpoint{-0.104167in}{-0.104167in}}{\pgfqpoint{0.104167in}{0.104167in}}{%
\pgfpathmoveto{\pgfqpoint{0.000000in}{-0.104167in}}%
\pgfpathcurveto{\pgfqpoint{0.027625in}{-0.104167in}}{\pgfqpoint{0.054123in}{-0.093191in}}{\pgfqpoint{0.073657in}{-0.073657in}}%
\pgfpathcurveto{\pgfqpoint{0.093191in}{-0.054123in}}{\pgfqpoint{0.104167in}{-0.027625in}}{\pgfqpoint{0.104167in}{0.000000in}}%
\pgfpathcurveto{\pgfqpoint{0.104167in}{0.027625in}}{\pgfqpoint{0.093191in}{0.054123in}}{\pgfqpoint{0.073657in}{0.073657in}}%
\pgfpathcurveto{\pgfqpoint{0.054123in}{0.093191in}}{\pgfqpoint{0.027625in}{0.104167in}}{\pgfqpoint{0.000000in}{0.104167in}}%
\pgfpathcurveto{\pgfqpoint{-0.027625in}{0.104167in}}{\pgfqpoint{-0.054123in}{0.093191in}}{\pgfqpoint{-0.073657in}{0.073657in}}%
\pgfpathcurveto{\pgfqpoint{-0.093191in}{0.054123in}}{\pgfqpoint{-0.104167in}{0.027625in}}{\pgfqpoint{-0.104167in}{0.000000in}}%
\pgfpathcurveto{\pgfqpoint{-0.104167in}{-0.027625in}}{\pgfqpoint{-0.093191in}{-0.054123in}}{\pgfqpoint{-0.073657in}{-0.073657in}}%
\pgfpathcurveto{\pgfqpoint{-0.054123in}{-0.093191in}}{\pgfqpoint{-0.027625in}{-0.104167in}}{\pgfqpoint{0.000000in}{-0.104167in}}%
\pgfpathclose%
\pgfusepath{stroke,fill}%
}%
\begin{pgfscope}%
\pgfsys@transformshift{1.589202in}{1.528456in}%
\pgfsys@useobject{currentmarker}{}%
\end{pgfscope}%
\begin{pgfscope}%
\pgfsys@transformshift{2.394397in}{2.045436in}%
\pgfsys@useobject{currentmarker}{}%
\end{pgfscope}%
\begin{pgfscope}%
\pgfsys@transformshift{3.199592in}{2.372609in}%
\pgfsys@useobject{currentmarker}{}%
\end{pgfscope}%
\begin{pgfscope}%
\pgfsys@transformshift{4.004786in}{2.497733in}%
\pgfsys@useobject{currentmarker}{}%
\end{pgfscope}%
\begin{pgfscope}%
\pgfsys@transformshift{4.809981in}{2.619785in}%
\pgfsys@useobject{currentmarker}{}%
\end{pgfscope}%
\begin{pgfscope}%
\pgfsys@transformshift{5.615176in}{2.871680in}%
\pgfsys@useobject{currentmarker}{}%
\end{pgfscope}%
\begin{pgfscope}%
\pgfsys@transformshift{6.420371in}{3.264868in}%
\pgfsys@useobject{currentmarker}{}%
\end{pgfscope}%
\begin{pgfscope}%
\pgfsys@transformshift{7.225566in}{3.299596in}%
\pgfsys@useobject{currentmarker}{}%
\end{pgfscope}%
\end{pgfscope}%
\begin{pgfscope}%
\pgfpathrectangle{\pgfqpoint{1.307384in}{1.215049in}}{\pgfqpoint{6.200000in}{3.020000in}} %
\pgfusepath{clip}%
\pgfsetbuttcap%
\pgfsetroundjoin%
\pgfsetlinewidth{8.030000pt}%
\definecolor{currentstroke}{rgb}{0.866667,0.517647,0.321569}%
\pgfsetstrokecolor{currentstroke}%
\pgfsetdash{{24.000000pt}{8.000000pt}}{0.000000pt}%
\pgfpathmoveto{\pgfqpoint{1.589202in}{1.528456in}}%
\pgfpathlineto{\pgfqpoint{2.394397in}{1.614056in}}%
\pgfpathlineto{\pgfqpoint{3.199592in}{1.704411in}}%
\pgfpathlineto{\pgfqpoint{4.004786in}{1.704411in}}%
\pgfpathlineto{\pgfqpoint{4.809981in}{1.704411in}}%
\pgfpathlineto{\pgfqpoint{5.615176in}{1.704411in}}%
\pgfpathlineto{\pgfqpoint{6.420371in}{1.704232in}}%
\pgfpathlineto{\pgfqpoint{7.225566in}{1.704411in}}%
\pgfusepath{stroke}%
\end{pgfscope}%
\begin{pgfscope}%
\pgfpathrectangle{\pgfqpoint{1.307384in}{1.215049in}}{\pgfqpoint{6.200000in}{3.020000in}} %
\pgfusepath{clip}%
\pgfsetbuttcap%
\pgfsetmiterjoin%
\definecolor{currentfill}{rgb}{0.866667,0.517647,0.321569}%
\pgfsetfillcolor{currentfill}%
\pgfsetlinewidth{0.752812pt}%
\definecolor{currentstroke}{rgb}{1.000000,1.000000,1.000000}%
\pgfsetstrokecolor{currentstroke}%
\pgfsetdash{}{0pt}%
\pgfsys@defobject{currentmarker}{\pgfqpoint{-0.104167in}{-0.104167in}}{\pgfqpoint{0.104167in}{0.104167in}}{%
\pgfpathmoveto{\pgfqpoint{0.000000in}{0.104167in}}%
\pgfpathlineto{\pgfqpoint{-0.104167in}{-0.104167in}}%
\pgfpathlineto{\pgfqpoint{0.104167in}{-0.104167in}}%
\pgfpathclose%
\pgfusepath{stroke,fill}%
}%
\begin{pgfscope}%
\pgfsys@transformshift{1.589202in}{1.528456in}%
\pgfsys@useobject{currentmarker}{}%
\end{pgfscope}%
\begin{pgfscope}%
\pgfsys@transformshift{2.394397in}{1.614056in}%
\pgfsys@useobject{currentmarker}{}%
\end{pgfscope}%
\begin{pgfscope}%
\pgfsys@transformshift{3.199592in}{1.704411in}%
\pgfsys@useobject{currentmarker}{}%
\end{pgfscope}%
\begin{pgfscope}%
\pgfsys@transformshift{4.004786in}{1.704411in}%
\pgfsys@useobject{currentmarker}{}%
\end{pgfscope}%
\begin{pgfscope}%
\pgfsys@transformshift{4.809981in}{1.704411in}%
\pgfsys@useobject{currentmarker}{}%
\end{pgfscope}%
\begin{pgfscope}%
\pgfsys@transformshift{5.615176in}{1.704411in}%
\pgfsys@useobject{currentmarker}{}%
\end{pgfscope}%
\begin{pgfscope}%
\pgfsys@transformshift{6.420371in}{1.704232in}%
\pgfsys@useobject{currentmarker}{}%
\end{pgfscope}%
\begin{pgfscope}%
\pgfsys@transformshift{7.225566in}{1.704411in}%
\pgfsys@useobject{currentmarker}{}%
\end{pgfscope}%
\end{pgfscope}%
\begin{pgfscope}%
\pgfpathrectangle{\pgfqpoint{1.307384in}{1.215049in}}{\pgfqpoint{6.200000in}{3.020000in}} %
\pgfusepath{clip}%
\pgfsetbuttcap%
\pgfsetroundjoin%
\pgfsetlinewidth{8.030000pt}%
\definecolor{currentstroke}{rgb}{0.333333,0.658824,0.407843}%
\pgfsetstrokecolor{currentstroke}%
\pgfsetdash{{16.000000pt}{16.000000pt}}{0.000000pt}%
\pgfpathmoveto{\pgfqpoint{1.589202in}{1.510416in}}%
\pgfpathlineto{\pgfqpoint{2.394397in}{1.594156in}}%
\pgfpathlineto{\pgfqpoint{3.199592in}{1.605531in}}%
\pgfpathlineto{\pgfqpoint{4.004786in}{1.605531in}}%
\pgfpathlineto{\pgfqpoint{4.809981in}{1.605531in}}%
\pgfpathlineto{\pgfqpoint{5.615176in}{1.567572in}}%
\pgfpathlineto{\pgfqpoint{6.420371in}{1.625569in}}%
\pgfpathlineto{\pgfqpoint{7.225566in}{1.647065in}}%
\pgfusepath{stroke}%
\end{pgfscope}%
\begin{pgfscope}%
\pgfpathrectangle{\pgfqpoint{1.307384in}{1.215049in}}{\pgfqpoint{6.200000in}{3.020000in}} %
\pgfusepath{clip}%
\pgfsetbuttcap%
\pgfsetmiterjoin%
\definecolor{currentfill}{rgb}{0.333333,0.658824,0.407843}%
\pgfsetfillcolor{currentfill}%
\pgfsetlinewidth{0.752812pt}%
\definecolor{currentstroke}{rgb}{1.000000,1.000000,1.000000}%
\pgfsetstrokecolor{currentstroke}%
\pgfsetdash{}{0pt}%
\pgfsys@defobject{currentmarker}{\pgfqpoint{-0.104167in}{-0.104167in}}{\pgfqpoint{0.104167in}{0.104167in}}{%
\pgfpathmoveto{\pgfqpoint{-0.052083in}{-0.104167in}}%
\pgfpathlineto{\pgfqpoint{0.000000in}{-0.052083in}}%
\pgfpathlineto{\pgfqpoint{0.052083in}{-0.104167in}}%
\pgfpathlineto{\pgfqpoint{0.104167in}{-0.052083in}}%
\pgfpathlineto{\pgfqpoint{0.052083in}{0.000000in}}%
\pgfpathlineto{\pgfqpoint{0.104167in}{0.052083in}}%
\pgfpathlineto{\pgfqpoint{0.052083in}{0.104167in}}%
\pgfpathlineto{\pgfqpoint{0.000000in}{0.052083in}}%
\pgfpathlineto{\pgfqpoint{-0.052083in}{0.104167in}}%
\pgfpathlineto{\pgfqpoint{-0.104167in}{0.052083in}}%
\pgfpathlineto{\pgfqpoint{-0.052083in}{0.000000in}}%
\pgfpathlineto{\pgfqpoint{-0.104167in}{-0.052083in}}%
\pgfpathclose%
\pgfusepath{stroke,fill}%
}%
\begin{pgfscope}%
\pgfsys@transformshift{1.589202in}{1.510416in}%
\pgfsys@useobject{currentmarker}{}%
\end{pgfscope}%
\begin{pgfscope}%
\pgfsys@transformshift{2.394397in}{1.594156in}%
\pgfsys@useobject{currentmarker}{}%
\end{pgfscope}%
\begin{pgfscope}%
\pgfsys@transformshift{3.199592in}{1.605531in}%
\pgfsys@useobject{currentmarker}{}%
\end{pgfscope}%
\begin{pgfscope}%
\pgfsys@transformshift{4.004786in}{1.605531in}%
\pgfsys@useobject{currentmarker}{}%
\end{pgfscope}%
\begin{pgfscope}%
\pgfsys@transformshift{4.809981in}{1.605531in}%
\pgfsys@useobject{currentmarker}{}%
\end{pgfscope}%
\begin{pgfscope}%
\pgfsys@transformshift{5.615176in}{1.567572in}%
\pgfsys@useobject{currentmarker}{}%
\end{pgfscope}%
\begin{pgfscope}%
\pgfsys@transformshift{6.420371in}{1.625569in}%
\pgfsys@useobject{currentmarker}{}%
\end{pgfscope}%
\begin{pgfscope}%
\pgfsys@transformshift{7.225566in}{1.647065in}%
\pgfsys@useobject{currentmarker}{}%
\end{pgfscope}%
\end{pgfscope}%
\begin{pgfscope}%
\pgfpathrectangle{\pgfqpoint{1.307384in}{1.215049in}}{\pgfqpoint{6.200000in}{3.020000in}} %
\pgfusepath{clip}%
\pgfsetbuttcap%
\pgfsetroundjoin%
\pgfsetlinewidth{8.030000pt}%
\definecolor{currentstroke}{rgb}{0.768627,0.305882,0.321569}%
\pgfsetstrokecolor{currentstroke}%
\pgfsetdash{{8.000000pt}{8.000000pt}}{0.000000pt}%
\pgfpathmoveto{\pgfqpoint{1.589202in}{1.528456in}}%
\pgfpathlineto{\pgfqpoint{2.394397in}{2.055351in}}%
\pgfpathlineto{\pgfqpoint{3.199592in}{3.845298in}}%
\pgfpathlineto{\pgfqpoint{4.004786in}{3.845319in}}%
\pgfpathlineto{\pgfqpoint{4.809981in}{3.967372in}}%
\pgfpathlineto{\pgfqpoint{5.615176in}{3.967441in}}%
\pgfpathlineto{\pgfqpoint{6.420371in}{3.967441in}}%
\pgfpathlineto{\pgfqpoint{7.225566in}{3.977407in}}%
\pgfusepath{stroke}%
\end{pgfscope}%
\begin{pgfscope}%
\pgfpathrectangle{\pgfqpoint{1.307384in}{1.215049in}}{\pgfqpoint{6.200000in}{3.020000in}} %
\pgfusepath{clip}%
\pgfsetbuttcap%
\pgfsetmiterjoin%
\definecolor{currentfill}{rgb}{0.768627,0.305882,0.321569}%
\pgfsetfillcolor{currentfill}%
\pgfsetlinewidth{0.752812pt}%
\definecolor{currentstroke}{rgb}{1.000000,1.000000,1.000000}%
\pgfsetstrokecolor{currentstroke}%
\pgfsetdash{}{0pt}%
\pgfsys@defobject{currentmarker}{\pgfqpoint{-0.104167in}{-0.104167in}}{\pgfqpoint{0.104167in}{0.104167in}}{%
\pgfpathmoveto{\pgfqpoint{-0.000000in}{-0.104167in}}%
\pgfpathlineto{\pgfqpoint{0.104167in}{0.104167in}}%
\pgfpathlineto{\pgfqpoint{-0.104167in}{0.104167in}}%
\pgfpathclose%
\pgfusepath{stroke,fill}%
}%
\begin{pgfscope}%
\pgfsys@transformshift{1.589202in}{1.528456in}%
\pgfsys@useobject{currentmarker}{}%
\end{pgfscope}%
\begin{pgfscope}%
\pgfsys@transformshift{2.394397in}{2.055351in}%
\pgfsys@useobject{currentmarker}{}%
\end{pgfscope}%
\begin{pgfscope}%
\pgfsys@transformshift{3.199592in}{3.845298in}%
\pgfsys@useobject{currentmarker}{}%
\end{pgfscope}%
\begin{pgfscope}%
\pgfsys@transformshift{4.004786in}{3.845319in}%
\pgfsys@useobject{currentmarker}{}%
\end{pgfscope}%
\begin{pgfscope}%
\pgfsys@transformshift{4.809981in}{3.967372in}%
\pgfsys@useobject{currentmarker}{}%
\end{pgfscope}%
\begin{pgfscope}%
\pgfsys@transformshift{5.615176in}{3.967441in}%
\pgfsys@useobject{currentmarker}{}%
\end{pgfscope}%
\begin{pgfscope}%
\pgfsys@transformshift{6.420371in}{3.967441in}%
\pgfsys@useobject{currentmarker}{}%
\end{pgfscope}%
\begin{pgfscope}%
\pgfsys@transformshift{7.225566in}{3.977407in}%
\pgfsys@useobject{currentmarker}{}%
\end{pgfscope}%
\end{pgfscope}%
\begin{pgfscope}%
\pgfsetrectcap%
\pgfsetmiterjoin%
\pgfsetlinewidth{1.254687pt}%
\definecolor{currentstroke}{rgb}{0.800000,0.800000,0.800000}%
\pgfsetstrokecolor{currentstroke}%
\pgfsetdash{}{0pt}%
\pgfpathmoveto{\pgfqpoint{1.307384in}{1.215049in}}%
\pgfpathlineto{\pgfqpoint{1.307384in}{4.235049in}}%
\pgfusepath{stroke}%
\end{pgfscope}%
\begin{pgfscope}%
\pgfsetrectcap%
\pgfsetmiterjoin%
\pgfsetlinewidth{1.254687pt}%
\definecolor{currentstroke}{rgb}{0.800000,0.800000,0.800000}%
\pgfsetstrokecolor{currentstroke}%
\pgfsetdash{}{0pt}%
\pgfpathmoveto{\pgfqpoint{7.507384in}{1.215049in}}%
\pgfpathlineto{\pgfqpoint{7.507384in}{4.235049in}}%
\pgfusepath{stroke}%
\end{pgfscope}%
\begin{pgfscope}%
\pgfsetrectcap%
\pgfsetmiterjoin%
\pgfsetlinewidth{1.254687pt}%
\definecolor{currentstroke}{rgb}{0.800000,0.800000,0.800000}%
\pgfsetstrokecolor{currentstroke}%
\pgfsetdash{}{0pt}%
\pgfpathmoveto{\pgfqpoint{1.307384in}{1.215049in}}%
\pgfpathlineto{\pgfqpoint{7.507384in}{1.215049in}}%
\pgfusepath{stroke}%
\end{pgfscope}%
\begin{pgfscope}%
\pgfsetrectcap%
\pgfsetmiterjoin%
\pgfsetlinewidth{1.254687pt}%
\definecolor{currentstroke}{rgb}{0.800000,0.800000,0.800000}%
\pgfsetstrokecolor{currentstroke}%
\pgfsetdash{}{0pt}%
\pgfpathmoveto{\pgfqpoint{1.307384in}{4.235049in}}%
\pgfpathlineto{\pgfqpoint{7.507384in}{4.235049in}}%
\pgfusepath{stroke}%
\end{pgfscope}%
\begin{pgfscope}%
\pgfsetbuttcap%
\pgfsetroundjoin%
\pgfsetlinewidth{6.022500pt}%
\definecolor{currentstroke}{rgb}{0.298039,0.447059,0.690196}%
\pgfsetstrokecolor{currentstroke}%
\pgfsetdash{{6.000000pt}{0.000000pt}}{0.000000pt}%
\pgfpathmoveto{\pgfqpoint{1.129125in}{4.417860in}}%
\pgfpathlineto{\pgfqpoint{1.837458in}{4.417860in}}%
\pgfusepath{stroke}%
\end{pgfscope}%
\begin{pgfscope}%
\pgfsetbuttcap%
\pgfsetroundjoin%
\definecolor{currentfill}{rgb}{0.298039,0.447059,0.690196}%
\pgfsetfillcolor{currentfill}%
\pgfsetlinewidth{0.000000pt}%
\definecolor{currentstroke}{rgb}{0.298039,0.447059,0.690196}%
\pgfsetstrokecolor{currentstroke}%
\pgfsetdash{}{0pt}%
\pgfsys@defobject{currentmarker}{\pgfqpoint{-0.104167in}{-0.104167in}}{\pgfqpoint{0.104167in}{0.104167in}}{%
\pgfpathmoveto{\pgfqpoint{0.000000in}{-0.104167in}}%
\pgfpathcurveto{\pgfqpoint{0.027625in}{-0.104167in}}{\pgfqpoint{0.054123in}{-0.093191in}}{\pgfqpoint{0.073657in}{-0.073657in}}%
\pgfpathcurveto{\pgfqpoint{0.093191in}{-0.054123in}}{\pgfqpoint{0.104167in}{-0.027625in}}{\pgfqpoint{0.104167in}{0.000000in}}%
\pgfpathcurveto{\pgfqpoint{0.104167in}{0.027625in}}{\pgfqpoint{0.093191in}{0.054123in}}{\pgfqpoint{0.073657in}{0.073657in}}%
\pgfpathcurveto{\pgfqpoint{0.054123in}{0.093191in}}{\pgfqpoint{0.027625in}{0.104167in}}{\pgfqpoint{0.000000in}{0.104167in}}%
\pgfpathcurveto{\pgfqpoint{-0.027625in}{0.104167in}}{\pgfqpoint{-0.054123in}{0.093191in}}{\pgfqpoint{-0.073657in}{0.073657in}}%
\pgfpathcurveto{\pgfqpoint{-0.093191in}{0.054123in}}{\pgfqpoint{-0.104167in}{0.027625in}}{\pgfqpoint{-0.104167in}{0.000000in}}%
\pgfpathcurveto{\pgfqpoint{-0.104167in}{-0.027625in}}{\pgfqpoint{-0.093191in}{-0.054123in}}{\pgfqpoint{-0.073657in}{-0.073657in}}%
\pgfpathcurveto{\pgfqpoint{-0.054123in}{-0.093191in}}{\pgfqpoint{-0.027625in}{-0.104167in}}{\pgfqpoint{0.000000in}{-0.104167in}}%
\pgfpathclose%
\pgfusepath{fill}%
}%
\begin{pgfscope}%
\pgfsys@transformshift{1.483291in}{4.417860in}%
\pgfsys@useobject{currentmarker}{}%
\end{pgfscope}%
\end{pgfscope}%
\begin{pgfscope}%
\definecolor{textcolor}{rgb}{0.150000,0.150000,0.150000}%
\pgfsetstrokecolor{textcolor}%
\pgfsetfillcolor{textcolor}%
\pgftext[x=1.884680in,y=4.252583in,left,base]{\color{textcolor}\rmfamily\fontsize{34.000000}{40.800000}\selectfont SA}%
\end{pgfscope}%
\begin{pgfscope}%
\pgfsetbuttcap%
\pgfsetroundjoin%
\pgfsetlinewidth{6.022500pt}%
\definecolor{currentstroke}{rgb}{0.866667,0.517647,0.321569}%
\pgfsetstrokecolor{currentstroke}%
\pgfsetdash{{18.000000pt}{6.000000pt}}{0.000000pt}%
\pgfpathmoveto{\pgfqpoint{2.702113in}{4.417860in}}%
\pgfpathlineto{\pgfqpoint{3.410446in}{4.417860in}}%
\pgfusepath{stroke}%
\end{pgfscope}%
\begin{pgfscope}%
\pgfsetbuttcap%
\pgfsetmiterjoin%
\definecolor{currentfill}{rgb}{0.866667,0.517647,0.321569}%
\pgfsetfillcolor{currentfill}%
\pgfsetlinewidth{0.000000pt}%
\definecolor{currentstroke}{rgb}{0.866667,0.517647,0.321569}%
\pgfsetstrokecolor{currentstroke}%
\pgfsetdash{}{0pt}%
\pgfsys@defobject{currentmarker}{\pgfqpoint{-0.104167in}{-0.104167in}}{\pgfqpoint{0.104167in}{0.104167in}}{%
\pgfpathmoveto{\pgfqpoint{0.000000in}{0.104167in}}%
\pgfpathlineto{\pgfqpoint{-0.104167in}{-0.104167in}}%
\pgfpathlineto{\pgfqpoint{0.104167in}{-0.104167in}}%
\pgfpathclose%
\pgfusepath{fill}%
}%
\begin{pgfscope}%
\pgfsys@transformshift{3.056279in}{4.417860in}%
\pgfsys@useobject{currentmarker}{}%
\end{pgfscope}%
\end{pgfscope}%
\begin{pgfscope}%
\definecolor{textcolor}{rgb}{0.150000,0.150000,0.150000}%
\pgfsetstrokecolor{textcolor}%
\pgfsetfillcolor{textcolor}%
\pgftext[x=3.457668in,y=4.252583in,left,base]{\color{textcolor}\rmfamily\fontsize{34.000000}{40.800000}\selectfont HM}%
\end{pgfscope}%
\begin{pgfscope}%
\pgfsetbuttcap%
\pgfsetroundjoin%
\pgfsetlinewidth{6.022500pt}%
\definecolor{currentstroke}{rgb}{0.333333,0.658824,0.407843}%
\pgfsetstrokecolor{currentstroke}%
\pgfsetdash{{12.000000pt}{12.000000pt}}{0.000000pt}%
\pgfpathmoveto{\pgfqpoint{4.435854in}{4.417860in}}%
\pgfpathlineto{\pgfqpoint{5.144188in}{4.417860in}}%
\pgfusepath{stroke}%
\end{pgfscope}%
\begin{pgfscope}%
\pgfsetbuttcap%
\pgfsetmiterjoin%
\definecolor{currentfill}{rgb}{0.333333,0.658824,0.407843}%
\pgfsetfillcolor{currentfill}%
\pgfsetlinewidth{0.000000pt}%
\definecolor{currentstroke}{rgb}{0.333333,0.658824,0.407843}%
\pgfsetstrokecolor{currentstroke}%
\pgfsetdash{}{0pt}%
\pgfsys@defobject{currentmarker}{\pgfqpoint{-0.104167in}{-0.104167in}}{\pgfqpoint{0.104167in}{0.104167in}}{%
\pgfpathmoveto{\pgfqpoint{-0.052083in}{-0.104167in}}%
\pgfpathlineto{\pgfqpoint{0.000000in}{-0.052083in}}%
\pgfpathlineto{\pgfqpoint{0.052083in}{-0.104167in}}%
\pgfpathlineto{\pgfqpoint{0.104167in}{-0.052083in}}%
\pgfpathlineto{\pgfqpoint{0.052083in}{0.000000in}}%
\pgfpathlineto{\pgfqpoint{0.104167in}{0.052083in}}%
\pgfpathlineto{\pgfqpoint{0.052083in}{0.104167in}}%
\pgfpathlineto{\pgfqpoint{0.000000in}{0.052083in}}%
\pgfpathlineto{\pgfqpoint{-0.052083in}{0.104167in}}%
\pgfpathlineto{\pgfqpoint{-0.104167in}{0.052083in}}%
\pgfpathlineto{\pgfqpoint{-0.052083in}{0.000000in}}%
\pgfpathlineto{\pgfqpoint{-0.104167in}{-0.052083in}}%
\pgfpathclose%
\pgfusepath{fill}%
}%
\begin{pgfscope}%
\pgfsys@transformshift{4.790021in}{4.417860in}%
\pgfsys@useobject{currentmarker}{}%
\end{pgfscope}%
\end{pgfscope}%
\begin{pgfscope}%
\definecolor{textcolor}{rgb}{0.150000,0.150000,0.150000}%
\pgfsetstrokecolor{textcolor}%
\pgfsetfillcolor{textcolor}%
\pgftext[x=5.191410in,y=4.252583in,left,base]{\color{textcolor}\rmfamily\fontsize{34.000000}{40.800000}\selectfont HL}%
\end{pgfscope}%
\begin{pgfscope}%
\pgfsetbuttcap%
\pgfsetroundjoin%
\pgfsetlinewidth{6.022500pt}%
\definecolor{currentstroke}{rgb}{0.768627,0.305882,0.321569}%
\pgfsetstrokecolor{currentstroke}%
\pgfsetdash{{6.000000pt}{6.000000pt}}{0.000000pt}%
\pgfpathmoveto{\pgfqpoint{6.039668in}{4.417860in}}%
\pgfpathlineto{\pgfqpoint{6.748001in}{4.417860in}}%
\pgfusepath{stroke}%
\end{pgfscope}%
\begin{pgfscope}%
\pgfsetbuttcap%
\pgfsetmiterjoin%
\definecolor{currentfill}{rgb}{0.768627,0.305882,0.321569}%
\pgfsetfillcolor{currentfill}%
\pgfsetlinewidth{0.000000pt}%
\definecolor{currentstroke}{rgb}{0.768627,0.305882,0.321569}%
\pgfsetstrokecolor{currentstroke}%
\pgfsetdash{}{0pt}%
\pgfsys@defobject{currentmarker}{\pgfqpoint{-0.104167in}{-0.104167in}}{\pgfqpoint{0.104167in}{0.104167in}}{%
\pgfpathmoveto{\pgfqpoint{-0.000000in}{-0.104167in}}%
\pgfpathlineto{\pgfqpoint{0.104167in}{0.104167in}}%
\pgfpathlineto{\pgfqpoint{-0.104167in}{0.104167in}}%
\pgfpathclose%
\pgfusepath{fill}%
}%
\begin{pgfscope}%
\pgfsys@transformshift{6.393834in}{4.417860in}%
\pgfsys@useobject{currentmarker}{}%
\end{pgfscope}%
\end{pgfscope}%
\begin{pgfscope}%
\definecolor{textcolor}{rgb}{0.150000,0.150000,0.150000}%
\pgfsetstrokecolor{textcolor}%
\pgfsetfillcolor{textcolor}%
\pgftext[x=6.795223in,y=4.252583in,left,base]{\color{textcolor}\rmfamily\fontsize{34.000000}{40.800000}\selectfont ALL}%
\end{pgfscope}%
\end{pgfpicture}%
\makeatother%
\endgroup%
%
}

\caption{Budget = 16 GB}
\end{subfigure}
\begin{subfigure}[b]{0.5\linewidth}
\centering
 \resizebox{\columnwidth}{!}{%
%% Creator: Matplotlib, PGF backend
%%
%% To include the figure in your LaTeX document, write
%%   \input{<filename>.pgf}
%%
%% Make sure the required packages are loaded in your preamble
%%   \usepackage{pgf}
%%
%% Figures using additional raster images can only be included by \input if
%% they are in the same directory as the main LaTeX file. For loading figures
%% from other directories you can use the `import` package
%%   \usepackage{import}
%% and then include the figures with
%%   \import{<path to file>}{<filename>.pgf}
%%
%% Matplotlib used the following preamble
%%   \usepackage{fontspec}
%%   \setmonofont{Andale Mono}
%%
\begingroup%
\makeatletter%
\begin{pgfpicture}%
\pgfpathrectangle{\pgfpointorigin}{\pgfqpoint{5.576861in}{4.844250in}}%
\pgfusepath{use as bounding box, clip}%
\begin{pgfscope}%
\pgfsetbuttcap%
\pgfsetmiterjoin%
\definecolor{currentfill}{rgb}{1.000000,1.000000,1.000000}%
\pgfsetfillcolor{currentfill}%
\pgfsetlinewidth{0.000000pt}%
\definecolor{currentstroke}{rgb}{1.000000,1.000000,1.000000}%
\pgfsetstrokecolor{currentstroke}%
\pgfsetdash{}{0pt}%
\pgfpathmoveto{\pgfqpoint{0.000000in}{0.000000in}}%
\pgfpathlineto{\pgfqpoint{5.576861in}{0.000000in}}%
\pgfpathlineto{\pgfqpoint{5.576861in}{4.844250in}}%
\pgfpathlineto{\pgfqpoint{0.000000in}{4.844250in}}%
\pgfpathclose%
\pgfusepath{fill}%
\end{pgfscope}%
\begin{pgfscope}%
\pgfsetbuttcap%
\pgfsetmiterjoin%
\definecolor{currentfill}{rgb}{1.000000,1.000000,1.000000}%
\pgfsetfillcolor{currentfill}%
\pgfsetlinewidth{0.000000pt}%
\definecolor{currentstroke}{rgb}{0.000000,0.000000,0.000000}%
\pgfsetstrokecolor{currentstroke}%
\pgfsetstrokeopacity{0.000000}%
\pgfsetdash{}{0pt}%
\pgfpathmoveto{\pgfqpoint{1.535194in}{1.233139in}}%
\pgfpathlineto{\pgfqpoint{5.410194in}{1.233139in}}%
\pgfpathlineto{\pgfqpoint{5.410194in}{4.253139in}}%
\pgfpathlineto{\pgfqpoint{1.535194in}{4.253139in}}%
\pgfpathclose%
\pgfusepath{fill}%
\end{pgfscope}%
\begin{pgfscope}%
\pgfpathrectangle{\pgfqpoint{1.535194in}{1.233139in}}{\pgfqpoint{3.875000in}{3.020000in}} %
\pgfusepath{clip}%
\pgfsetroundcap%
\pgfsetroundjoin%
\pgfsetlinewidth{0.803000pt}%
\definecolor{currentstroke}{rgb}{0.800000,0.800000,0.800000}%
\pgfsetstrokecolor{currentstroke}%
\pgfsetdash{}{0pt}%
\pgfpathmoveto{\pgfqpoint{1.711331in}{1.233139in}}%
\pgfpathlineto{\pgfqpoint{1.711331in}{4.253139in}}%
\pgfusepath{stroke}%
\end{pgfscope}%
\begin{pgfscope}%
\definecolor{textcolor}{rgb}{0.150000,0.150000,0.150000}%
\pgfsetstrokecolor{textcolor}%
\pgfsetfillcolor{textcolor}%
\pgftext[x=1.711331in,y=1.069250in,,top]{\color{textcolor}\rmfamily\fontsize{36.000000}{43.200000}\selectfont 1}%
\end{pgfscope}%
\begin{pgfscope}%
\pgfpathrectangle{\pgfqpoint{1.535194in}{1.233139in}}{\pgfqpoint{3.875000in}{3.020000in}} %
\pgfusepath{clip}%
\pgfsetroundcap%
\pgfsetroundjoin%
\pgfsetlinewidth{0.803000pt}%
\definecolor{currentstroke}{rgb}{0.800000,0.800000,0.800000}%
\pgfsetstrokecolor{currentstroke}%
\pgfsetdash{}{0pt}%
\pgfpathmoveto{\pgfqpoint{2.592012in}{1.233139in}}%
\pgfpathlineto{\pgfqpoint{2.592012in}{4.253139in}}%
\pgfusepath{stroke}%
\end{pgfscope}%
\begin{pgfscope}%
\definecolor{textcolor}{rgb}{0.150000,0.150000,0.150000}%
\pgfsetstrokecolor{textcolor}%
\pgfsetfillcolor{textcolor}%
\pgftext[x=2.592012in,y=1.069250in,,top]{\color{textcolor}\rmfamily\fontsize{36.000000}{43.200000}\selectfont 2}%
\end{pgfscope}%
\begin{pgfscope}%
\pgfpathrectangle{\pgfqpoint{1.535194in}{1.233139in}}{\pgfqpoint{3.875000in}{3.020000in}} %
\pgfusepath{clip}%
\pgfsetroundcap%
\pgfsetroundjoin%
\pgfsetlinewidth{0.803000pt}%
\definecolor{currentstroke}{rgb}{0.800000,0.800000,0.800000}%
\pgfsetstrokecolor{currentstroke}%
\pgfsetdash{}{0pt}%
\pgfpathmoveto{\pgfqpoint{3.472694in}{1.233139in}}%
\pgfpathlineto{\pgfqpoint{3.472694in}{4.253139in}}%
\pgfusepath{stroke}%
\end{pgfscope}%
\begin{pgfscope}%
\definecolor{textcolor}{rgb}{0.150000,0.150000,0.150000}%
\pgfsetstrokecolor{textcolor}%
\pgfsetfillcolor{textcolor}%
\pgftext[x=3.472694in,y=1.069250in,,top]{\color{textcolor}\rmfamily\fontsize{36.000000}{43.200000}\selectfont 3}%
\end{pgfscope}%
\begin{pgfscope}%
\pgfpathrectangle{\pgfqpoint{1.535194in}{1.233139in}}{\pgfqpoint{3.875000in}{3.020000in}} %
\pgfusepath{clip}%
\pgfsetroundcap%
\pgfsetroundjoin%
\pgfsetlinewidth{0.803000pt}%
\definecolor{currentstroke}{rgb}{0.800000,0.800000,0.800000}%
\pgfsetstrokecolor{currentstroke}%
\pgfsetdash{}{0pt}%
\pgfpathmoveto{\pgfqpoint{4.353376in}{1.233139in}}%
\pgfpathlineto{\pgfqpoint{4.353376in}{4.253139in}}%
\pgfusepath{stroke}%
\end{pgfscope}%
\begin{pgfscope}%
\definecolor{textcolor}{rgb}{0.150000,0.150000,0.150000}%
\pgfsetstrokecolor{textcolor}%
\pgfsetfillcolor{textcolor}%
\pgftext[x=4.353376in,y=1.069250in,,top]{\color{textcolor}\rmfamily\fontsize{36.000000}{43.200000}\selectfont 4}%
\end{pgfscope}%
\begin{pgfscope}%
\pgfpathrectangle{\pgfqpoint{1.535194in}{1.233139in}}{\pgfqpoint{3.875000in}{3.020000in}} %
\pgfusepath{clip}%
\pgfsetroundcap%
\pgfsetroundjoin%
\pgfsetlinewidth{0.803000pt}%
\definecolor{currentstroke}{rgb}{0.800000,0.800000,0.800000}%
\pgfsetstrokecolor{currentstroke}%
\pgfsetdash{}{0pt}%
\pgfpathmoveto{\pgfqpoint{5.234058in}{1.233139in}}%
\pgfpathlineto{\pgfqpoint{5.234058in}{4.253139in}}%
\pgfusepath{stroke}%
\end{pgfscope}%
\begin{pgfscope}%
\definecolor{textcolor}{rgb}{0.150000,0.150000,0.150000}%
\pgfsetstrokecolor{textcolor}%
\pgfsetfillcolor{textcolor}%
\pgftext[x=5.234058in,y=1.069250in,,top]{\color{textcolor}\rmfamily\fontsize{36.000000}{43.200000}\selectfont 5}%
\end{pgfscope}%
\begin{pgfscope}%
\definecolor{textcolor}{rgb}{0.150000,0.150000,0.150000}%
\pgfsetstrokecolor{textcolor}%
\pgfsetfillcolor{textcolor}%
\pgftext[x=3.472694in,y=0.569194in,,top]{\color{textcolor}\rmfamily\fontsize{38.000000}{45.600000}\selectfont Workload}%
\end{pgfscope}%
\begin{pgfscope}%
\pgfpathrectangle{\pgfqpoint{1.535194in}{1.233139in}}{\pgfqpoint{3.875000in}{3.020000in}} %
\pgfusepath{clip}%
\pgfsetroundcap%
\pgfsetroundjoin%
\pgfsetlinewidth{0.803000pt}%
\definecolor{currentstroke}{rgb}{0.800000,0.800000,0.800000}%
\pgfsetstrokecolor{currentstroke}%
\pgfsetdash{}{0pt}%
\pgfpathmoveto{\pgfqpoint{1.535194in}{1.233139in}}%
\pgfpathlineto{\pgfqpoint{5.410194in}{1.233139in}}%
\pgfusepath{stroke}%
\end{pgfscope}%
\begin{pgfscope}%
\definecolor{textcolor}{rgb}{0.150000,0.150000,0.150000}%
\pgfsetstrokecolor{textcolor}%
\pgfsetfillcolor{textcolor}%
\pgftext[x=1.141805in,y=1.059639in,left,base]{\color{textcolor}\rmfamily\fontsize{36.000000}{43.200000}\selectfont 0}%
\end{pgfscope}%
\begin{pgfscope}%
\pgfpathrectangle{\pgfqpoint{1.535194in}{1.233139in}}{\pgfqpoint{3.875000in}{3.020000in}} %
\pgfusepath{clip}%
\pgfsetroundcap%
\pgfsetroundjoin%
\pgfsetlinewidth{0.803000pt}%
\definecolor{currentstroke}{rgb}{0.800000,0.800000,0.800000}%
\pgfsetstrokecolor{currentstroke}%
\pgfsetdash{}{0pt}%
\pgfpathmoveto{\pgfqpoint{1.535194in}{2.311710in}}%
\pgfpathlineto{\pgfqpoint{5.410194in}{2.311710in}}%
\pgfusepath{stroke}%
\end{pgfscope}%
\begin{pgfscope}%
\definecolor{textcolor}{rgb}{0.150000,0.150000,0.150000}%
\pgfsetstrokecolor{textcolor}%
\pgfsetfillcolor{textcolor}%
\pgftext[x=0.912305in,y=2.138210in,left,base]{\color{textcolor}\rmfamily\fontsize{36.000000}{43.200000}\selectfont 50}%
\end{pgfscope}%
\begin{pgfscope}%
\pgfpathrectangle{\pgfqpoint{1.535194in}{1.233139in}}{\pgfqpoint{3.875000in}{3.020000in}} %
\pgfusepath{clip}%
\pgfsetroundcap%
\pgfsetroundjoin%
\pgfsetlinewidth{0.803000pt}%
\definecolor{currentstroke}{rgb}{0.800000,0.800000,0.800000}%
\pgfsetstrokecolor{currentstroke}%
\pgfsetdash{}{0pt}%
\pgfpathmoveto{\pgfqpoint{1.535194in}{3.390281in}}%
\pgfpathlineto{\pgfqpoint{5.410194in}{3.390281in}}%
\pgfusepath{stroke}%
\end{pgfscope}%
\begin{pgfscope}%
\definecolor{textcolor}{rgb}{0.150000,0.150000,0.150000}%
\pgfsetstrokecolor{textcolor}%
\pgfsetfillcolor{textcolor}%
\pgftext[x=0.682805in,y=3.216781in,left,base]{\color{textcolor}\rmfamily\fontsize{36.000000}{43.200000}\selectfont 100}%
\end{pgfscope}%
\begin{pgfscope}%
\definecolor{textcolor}{rgb}{0.150000,0.150000,0.150000}%
\pgfsetstrokecolor{textcolor}%
\pgfsetfillcolor{textcolor}%
\pgftext[x=0.627250in,y=2.743139in,,bottom,rotate=90.000000]{\color{textcolor}\rmfamily\fontsize{38.000000}{45.600000}\selectfont Size (GB)}%
\end{pgfscope}%
\begin{pgfscope}%
\pgfpathrectangle{\pgfqpoint{1.535194in}{1.233139in}}{\pgfqpoint{3.875000in}{3.020000in}} %
\pgfusepath{clip}%
\pgfsetbuttcap%
\pgfsetroundjoin%
\pgfsetlinewidth{3.011250pt}%
\definecolor{currentstroke}{rgb}{0.866667,0.517647,0.321569}%
\pgfsetstrokecolor{currentstroke}%
\pgfsetdash{{3.000000pt}{0.000000pt}}{0.000000pt}%
\pgfpathmoveto{\pgfqpoint{1.711331in}{1.529100in}}%
\pgfpathlineto{\pgfqpoint{2.592012in}{1.923359in}}%
\pgfpathlineto{\pgfqpoint{3.472694in}{1.923209in}}%
\pgfpathlineto{\pgfqpoint{4.353376in}{1.923366in}}%
\pgfpathlineto{\pgfqpoint{5.234058in}{1.923405in}}%
\pgfusepath{stroke}%
\end{pgfscope}%
\begin{pgfscope}%
\pgfpathrectangle{\pgfqpoint{1.535194in}{1.233139in}}{\pgfqpoint{3.875000in}{3.020000in}} %
\pgfusepath{clip}%
\pgfsetbuttcap%
\pgfsetroundjoin%
\definecolor{currentfill}{rgb}{0.866667,0.517647,0.321569}%
\pgfsetfillcolor{currentfill}%
\pgfsetlinewidth{0.752812pt}%
\definecolor{currentstroke}{rgb}{1.000000,1.000000,1.000000}%
\pgfsetstrokecolor{currentstroke}%
\pgfsetdash{}{0pt}%
\pgfsys@defobject{currentmarker}{\pgfqpoint{-0.104167in}{-0.104167in}}{\pgfqpoint{0.104167in}{0.104167in}}{%
\pgfpathmoveto{\pgfqpoint{0.000000in}{-0.104167in}}%
\pgfpathcurveto{\pgfqpoint{0.027625in}{-0.104167in}}{\pgfqpoint{0.054123in}{-0.093191in}}{\pgfqpoint{0.073657in}{-0.073657in}}%
\pgfpathcurveto{\pgfqpoint{0.093191in}{-0.054123in}}{\pgfqpoint{0.104167in}{-0.027625in}}{\pgfqpoint{0.104167in}{0.000000in}}%
\pgfpathcurveto{\pgfqpoint{0.104167in}{0.027625in}}{\pgfqpoint{0.093191in}{0.054123in}}{\pgfqpoint{0.073657in}{0.073657in}}%
\pgfpathcurveto{\pgfqpoint{0.054123in}{0.093191in}}{\pgfqpoint{0.027625in}{0.104167in}}{\pgfqpoint{0.000000in}{0.104167in}}%
\pgfpathcurveto{\pgfqpoint{-0.027625in}{0.104167in}}{\pgfqpoint{-0.054123in}{0.093191in}}{\pgfqpoint{-0.073657in}{0.073657in}}%
\pgfpathcurveto{\pgfqpoint{-0.093191in}{0.054123in}}{\pgfqpoint{-0.104167in}{0.027625in}}{\pgfqpoint{-0.104167in}{0.000000in}}%
\pgfpathcurveto{\pgfqpoint{-0.104167in}{-0.027625in}}{\pgfqpoint{-0.093191in}{-0.054123in}}{\pgfqpoint{-0.073657in}{-0.073657in}}%
\pgfpathcurveto{\pgfqpoint{-0.054123in}{-0.093191in}}{\pgfqpoint{-0.027625in}{-0.104167in}}{\pgfqpoint{0.000000in}{-0.104167in}}%
\pgfpathclose%
\pgfusepath{stroke,fill}%
}%
\begin{pgfscope}%
\pgfsys@transformshift{1.711331in}{1.529100in}%
\pgfsys@useobject{currentmarker}{}%
\end{pgfscope}%
\begin{pgfscope}%
\pgfsys@transformshift{2.592012in}{1.923359in}%
\pgfsys@useobject{currentmarker}{}%
\end{pgfscope}%
\begin{pgfscope}%
\pgfsys@transformshift{3.472694in}{1.923209in}%
\pgfsys@useobject{currentmarker}{}%
\end{pgfscope}%
\begin{pgfscope}%
\pgfsys@transformshift{4.353376in}{1.923366in}%
\pgfsys@useobject{currentmarker}{}%
\end{pgfscope}%
\begin{pgfscope}%
\pgfsys@transformshift{5.234058in}{1.923405in}%
\pgfsys@useobject{currentmarker}{}%
\end{pgfscope}%
\end{pgfscope}%
\begin{pgfscope}%
\pgfpathrectangle{\pgfqpoint{1.535194in}{1.233139in}}{\pgfqpoint{3.875000in}{3.020000in}} %
\pgfusepath{clip}%
\pgfsetbuttcap%
\pgfsetroundjoin%
\pgfsetlinewidth{3.011250pt}%
\definecolor{currentstroke}{rgb}{0.298039,0.447059,0.690196}%
\pgfsetstrokecolor{currentstroke}%
\pgfsetdash{{6.000000pt}{6.000000pt}}{0.000000pt}%
\pgfpathmoveto{\pgfqpoint{1.711331in}{1.529100in}}%
\pgfpathlineto{\pgfqpoint{2.592012in}{2.055994in}}%
\pgfpathlineto{\pgfqpoint{3.472694in}{3.845944in}}%
\pgfpathlineto{\pgfqpoint{4.353376in}{3.850800in}}%
\pgfpathlineto{\pgfqpoint{5.234058in}{3.979227in}}%
\pgfusepath{stroke}%
\end{pgfscope}%
\begin{pgfscope}%
\pgfpathrectangle{\pgfqpoint{1.535194in}{1.233139in}}{\pgfqpoint{3.875000in}{3.020000in}} %
\pgfusepath{clip}%
\pgfsetbuttcap%
\pgfsetmiterjoin%
\definecolor{currentfill}{rgb}{0.298039,0.447059,0.690196}%
\pgfsetfillcolor{currentfill}%
\pgfsetlinewidth{0.752812pt}%
\definecolor{currentstroke}{rgb}{1.000000,1.000000,1.000000}%
\pgfsetstrokecolor{currentstroke}%
\pgfsetdash{}{0pt}%
\pgfsys@defobject{currentmarker}{\pgfqpoint{-0.104167in}{-0.104167in}}{\pgfqpoint{0.104167in}{0.104167in}}{%
\pgfpathmoveto{\pgfqpoint{0.000000in}{0.104167in}}%
\pgfpathlineto{\pgfqpoint{-0.104167in}{-0.104167in}}%
\pgfpathlineto{\pgfqpoint{0.104167in}{-0.104167in}}%
\pgfpathclose%
\pgfusepath{stroke,fill}%
}%
\begin{pgfscope}%
\pgfsys@transformshift{1.711331in}{1.529100in}%
\pgfsys@useobject{currentmarker}{}%
\end{pgfscope}%
\begin{pgfscope}%
\pgfsys@transformshift{2.592012in}{2.055994in}%
\pgfsys@useobject{currentmarker}{}%
\end{pgfscope}%
\begin{pgfscope}%
\pgfsys@transformshift{3.472694in}{3.845944in}%
\pgfsys@useobject{currentmarker}{}%
\end{pgfscope}%
\begin{pgfscope}%
\pgfsys@transformshift{4.353376in}{3.850800in}%
\pgfsys@useobject{currentmarker}{}%
\end{pgfscope}%
\begin{pgfscope}%
\pgfsys@transformshift{5.234058in}{3.979227in}%
\pgfsys@useobject{currentmarker}{}%
\end{pgfscope}%
\end{pgfscope}%
\begin{pgfscope}%
\pgfsetrectcap%
\pgfsetmiterjoin%
\pgfsetlinewidth{1.003750pt}%
\definecolor{currentstroke}{rgb}{0.800000,0.800000,0.800000}%
\pgfsetstrokecolor{currentstroke}%
\pgfsetdash{}{0pt}%
\pgfpathmoveto{\pgfqpoint{1.535194in}{1.233139in}}%
\pgfpathlineto{\pgfqpoint{1.535194in}{4.253139in}}%
\pgfusepath{stroke}%
\end{pgfscope}%
\begin{pgfscope}%
\pgfsetrectcap%
\pgfsetmiterjoin%
\pgfsetlinewidth{1.003750pt}%
\definecolor{currentstroke}{rgb}{0.800000,0.800000,0.800000}%
\pgfsetstrokecolor{currentstroke}%
\pgfsetdash{}{0pt}%
\pgfpathmoveto{\pgfqpoint{5.410194in}{1.233139in}}%
\pgfpathlineto{\pgfqpoint{5.410194in}{4.253139in}}%
\pgfusepath{stroke}%
\end{pgfscope}%
\begin{pgfscope}%
\pgfsetrectcap%
\pgfsetmiterjoin%
\pgfsetlinewidth{1.003750pt}%
\definecolor{currentstroke}{rgb}{0.800000,0.800000,0.800000}%
\pgfsetstrokecolor{currentstroke}%
\pgfsetdash{}{0pt}%
\pgfpathmoveto{\pgfqpoint{1.535194in}{1.233139in}}%
\pgfpathlineto{\pgfqpoint{5.410194in}{1.233139in}}%
\pgfusepath{stroke}%
\end{pgfscope}%
\begin{pgfscope}%
\pgfsetrectcap%
\pgfsetmiterjoin%
\pgfsetlinewidth{1.003750pt}%
\definecolor{currentstroke}{rgb}{0.800000,0.800000,0.800000}%
\pgfsetstrokecolor{currentstroke}%
\pgfsetdash{}{0pt}%
\pgfpathmoveto{\pgfqpoint{1.535194in}{4.253139in}}%
\pgfpathlineto{\pgfqpoint{5.410194in}{4.253139in}}%
\pgfusepath{stroke}%
\end{pgfscope}%
\begin{pgfscope}%
\pgfsetroundcap%
\pgfsetroundjoin%
\pgfsetlinewidth{3.011250pt}%
\definecolor{currentstroke}{rgb}{1.000000,1.000000,1.000000}%
\pgfsetstrokecolor{currentstroke}%
\pgfsetdash{}{0pt}%
\pgfpathmoveto{\pgfqpoint{1.700194in}{4.348416in}}%
\pgfpathlineto{\pgfqpoint{2.333527in}{4.348416in}}%
\pgfusepath{stroke}%
\end{pgfscope}%
\begin{pgfscope}%
\pgfsetbuttcap%
\pgfsetroundjoin%
\pgfsetlinewidth{3.011250pt}%
\definecolor{currentstroke}{rgb}{0.866667,0.517647,0.321569}%
\pgfsetstrokecolor{currentstroke}%
\pgfsetdash{{3.000000pt}{0.000000pt}}{0.000000pt}%
\pgfpathmoveto{\pgfqpoint{1.700194in}{3.928833in}}%
\pgfpathlineto{\pgfqpoint{2.333527in}{3.928833in}}%
\pgfusepath{stroke}%
\end{pgfscope}%
\begin{pgfscope}%
\pgfsetbuttcap%
\pgfsetroundjoin%
\definecolor{currentfill}{rgb}{0.866667,0.517647,0.321569}%
\pgfsetfillcolor{currentfill}%
\pgfsetlinewidth{0.000000pt}%
\definecolor{currentstroke}{rgb}{0.866667,0.517647,0.321569}%
\pgfsetstrokecolor{currentstroke}%
\pgfsetdash{}{0pt}%
\pgfsys@defobject{currentmarker}{\pgfqpoint{-0.104167in}{-0.104167in}}{\pgfqpoint{0.104167in}{0.104167in}}{%
\pgfpathmoveto{\pgfqpoint{0.000000in}{-0.104167in}}%
\pgfpathcurveto{\pgfqpoint{0.027625in}{-0.104167in}}{\pgfqpoint{0.054123in}{-0.093191in}}{\pgfqpoint{0.073657in}{-0.073657in}}%
\pgfpathcurveto{\pgfqpoint{0.093191in}{-0.054123in}}{\pgfqpoint{0.104167in}{-0.027625in}}{\pgfqpoint{0.104167in}{0.000000in}}%
\pgfpathcurveto{\pgfqpoint{0.104167in}{0.027625in}}{\pgfqpoint{0.093191in}{0.054123in}}{\pgfqpoint{0.073657in}{0.073657in}}%
\pgfpathcurveto{\pgfqpoint{0.054123in}{0.093191in}}{\pgfqpoint{0.027625in}{0.104167in}}{\pgfqpoint{0.000000in}{0.104167in}}%
\pgfpathcurveto{\pgfqpoint{-0.027625in}{0.104167in}}{\pgfqpoint{-0.054123in}{0.093191in}}{\pgfqpoint{-0.073657in}{0.073657in}}%
\pgfpathcurveto{\pgfqpoint{-0.093191in}{0.054123in}}{\pgfqpoint{-0.104167in}{0.027625in}}{\pgfqpoint{-0.104167in}{0.000000in}}%
\pgfpathcurveto{\pgfqpoint{-0.104167in}{-0.027625in}}{\pgfqpoint{-0.093191in}{-0.054123in}}{\pgfqpoint{-0.073657in}{-0.073657in}}%
\pgfpathcurveto{\pgfqpoint{-0.054123in}{-0.093191in}}{\pgfqpoint{-0.027625in}{-0.104167in}}{\pgfqpoint{0.000000in}{-0.104167in}}%
\pgfpathclose%
\pgfusepath{fill}%
}%
\begin{pgfscope}%
\pgfsys@transformshift{2.016861in}{3.928833in}%
\pgfsys@useobject{currentmarker}{}%
\end{pgfscope}%
\end{pgfscope}%
\begin{pgfscope}%
\definecolor{textcolor}{rgb}{0.150000,0.150000,0.150000}%
\pgfsetstrokecolor{textcolor}%
\pgfsetfillcolor{textcolor}%
\pgftext[x=2.386305in,y=3.744111in,left,base]{\color{textcolor}\rmfamily\fontsize{38.000000}{45.600000}\selectfont SM}%
\end{pgfscope}%
\begin{pgfscope}%
\pgfsetbuttcap%
\pgfsetroundjoin%
\pgfsetlinewidth{3.011250pt}%
\definecolor{currentstroke}{rgb}{0.298039,0.447059,0.690196}%
\pgfsetstrokecolor{currentstroke}%
\pgfsetdash{{6.000000pt}{6.000000pt}}{0.000000pt}%
\pgfpathmoveto{\pgfqpoint{1.700194in}{3.509250in}}%
\pgfpathlineto{\pgfqpoint{2.333527in}{3.509250in}}%
\pgfusepath{stroke}%
\end{pgfscope}%
\begin{pgfscope}%
\pgfsetbuttcap%
\pgfsetmiterjoin%
\definecolor{currentfill}{rgb}{0.298039,0.447059,0.690196}%
\pgfsetfillcolor{currentfill}%
\pgfsetlinewidth{0.000000pt}%
\definecolor{currentstroke}{rgb}{0.298039,0.447059,0.690196}%
\pgfsetstrokecolor{currentstroke}%
\pgfsetdash{}{0pt}%
\pgfsys@defobject{currentmarker}{\pgfqpoint{-0.104167in}{-0.104167in}}{\pgfqpoint{0.104167in}{0.104167in}}{%
\pgfpathmoveto{\pgfqpoint{0.000000in}{0.104167in}}%
\pgfpathlineto{\pgfqpoint{-0.104167in}{-0.104167in}}%
\pgfpathlineto{\pgfqpoint{0.104167in}{-0.104167in}}%
\pgfpathclose%
\pgfusepath{fill}%
}%
\begin{pgfscope}%
\pgfsys@transformshift{2.016861in}{3.509250in}%
\pgfsys@useobject{currentmarker}{}%
\end{pgfscope}%
\end{pgfscope}%
\begin{pgfscope}%
\definecolor{textcolor}{rgb}{0.150000,0.150000,0.150000}%
\pgfsetstrokecolor{textcolor}%
\pgfsetfillcolor{textcolor}%
\pgftext[x=2.386305in,y=3.324528in,left,base]{\color{textcolor}\rmfamily\fontsize{38.000000}{45.600000}\selectfont SA}%
\end{pgfscope}%
\end{pgfpicture}%
\makeatother%
\endgroup%
%
}

\caption{Budget = 32 GB}
\end{subfigure}%
\begin{subfigure}[b]{0.5\linewidth}
\centering
 \resizebox{\columnwidth}{!}{%
%% Creator: Matplotlib, PGF backend
%%
%% To include the figure in your LaTeX document, write
%%   \input{<filename>.pgf}
%%
%% Make sure the required packages are loaded in your preamble
%%   \usepackage{pgf}
%%
%% Figures using additional raster images can only be included by \input if
%% they are in the same directory as the main LaTeX file. For loading figures
%% from other directories you can use the `import` package
%%   \usepackage{import}
%% and then include the figures with
%%   \import{<path to file>}{<filename>.pgf}
%%
%% Matplotlib used the following preamble
%%   \usepackage{fontspec}
%%   \setmonofont{Andale Mono}
%%
\begingroup%
\makeatletter%
\begin{pgfpicture}%
\pgfpathrectangle{\pgfpointorigin}{\pgfqpoint{7.126861in}{4.844250in}}%
\pgfusepath{use as bounding box, clip}%
\begin{pgfscope}%
\pgfsetbuttcap%
\pgfsetmiterjoin%
\definecolor{currentfill}{rgb}{1.000000,1.000000,1.000000}%
\pgfsetfillcolor{currentfill}%
\pgfsetlinewidth{0.000000pt}%
\definecolor{currentstroke}{rgb}{1.000000,1.000000,1.000000}%
\pgfsetstrokecolor{currentstroke}%
\pgfsetdash{}{0pt}%
\pgfpathmoveto{\pgfqpoint{-0.000000in}{0.000000in}}%
\pgfpathlineto{\pgfqpoint{7.126861in}{0.000000in}}%
\pgfpathlineto{\pgfqpoint{7.126861in}{4.844250in}}%
\pgfpathlineto{\pgfqpoint{-0.000000in}{4.844250in}}%
\pgfpathclose%
\pgfusepath{fill}%
\end{pgfscope}%
\begin{pgfscope}%
\pgfsetbuttcap%
\pgfsetmiterjoin%
\definecolor{currentfill}{rgb}{1.000000,1.000000,1.000000}%
\pgfsetfillcolor{currentfill}%
\pgfsetlinewidth{0.000000pt}%
\definecolor{currentstroke}{rgb}{0.000000,0.000000,0.000000}%
\pgfsetstrokecolor{currentstroke}%
\pgfsetstrokeopacity{0.000000}%
\pgfsetdash{}{0pt}%
\pgfpathmoveto{\pgfqpoint{1.535194in}{1.233139in}}%
\pgfpathlineto{\pgfqpoint{6.960194in}{1.233139in}}%
\pgfpathlineto{\pgfqpoint{6.960194in}{4.253139in}}%
\pgfpathlineto{\pgfqpoint{1.535194in}{4.253139in}}%
\pgfpathclose%
\pgfusepath{fill}%
\end{pgfscope}%
\begin{pgfscope}%
\pgfpathrectangle{\pgfqpoint{1.535194in}{1.233139in}}{\pgfqpoint{5.425000in}{3.020000in}} %
\pgfusepath{clip}%
\pgfsetroundcap%
\pgfsetroundjoin%
\pgfsetlinewidth{0.803000pt}%
\definecolor{currentstroke}{rgb}{0.800000,0.800000,0.800000}%
\pgfsetstrokecolor{currentstroke}%
\pgfsetdash{}{0pt}%
\pgfpathmoveto{\pgfqpoint{1.781785in}{1.233139in}}%
\pgfpathlineto{\pgfqpoint{1.781785in}{4.253139in}}%
\pgfusepath{stroke}%
\end{pgfscope}%
\begin{pgfscope}%
\definecolor{textcolor}{rgb}{0.150000,0.150000,0.150000}%
\pgfsetstrokecolor{textcolor}%
\pgfsetfillcolor{textcolor}%
\pgftext[x=1.781785in,y=1.069250in,,top]{\color{textcolor}\rmfamily\fontsize{36.000000}{43.200000}\selectfont 1}%
\end{pgfscope}%
\begin{pgfscope}%
\pgfpathrectangle{\pgfqpoint{1.535194in}{1.233139in}}{\pgfqpoint{5.425000in}{3.020000in}} %
\pgfusepath{clip}%
\pgfsetroundcap%
\pgfsetroundjoin%
\pgfsetlinewidth{0.803000pt}%
\definecolor{currentstroke}{rgb}{0.800000,0.800000,0.800000}%
\pgfsetstrokecolor{currentstroke}%
\pgfsetdash{}{0pt}%
\pgfpathmoveto{\pgfqpoint{2.486331in}{1.233139in}}%
\pgfpathlineto{\pgfqpoint{2.486331in}{4.253139in}}%
\pgfusepath{stroke}%
\end{pgfscope}%
\begin{pgfscope}%
\definecolor{textcolor}{rgb}{0.150000,0.150000,0.150000}%
\pgfsetstrokecolor{textcolor}%
\pgfsetfillcolor{textcolor}%
\pgftext[x=2.486331in,y=1.069250in,,top]{\color{textcolor}\rmfamily\fontsize{36.000000}{43.200000}\selectfont 2}%
\end{pgfscope}%
\begin{pgfscope}%
\pgfpathrectangle{\pgfqpoint{1.535194in}{1.233139in}}{\pgfqpoint{5.425000in}{3.020000in}} %
\pgfusepath{clip}%
\pgfsetroundcap%
\pgfsetroundjoin%
\pgfsetlinewidth{0.803000pt}%
\definecolor{currentstroke}{rgb}{0.800000,0.800000,0.800000}%
\pgfsetstrokecolor{currentstroke}%
\pgfsetdash{}{0pt}%
\pgfpathmoveto{\pgfqpoint{3.190876in}{1.233139in}}%
\pgfpathlineto{\pgfqpoint{3.190876in}{4.253139in}}%
\pgfusepath{stroke}%
\end{pgfscope}%
\begin{pgfscope}%
\definecolor{textcolor}{rgb}{0.150000,0.150000,0.150000}%
\pgfsetstrokecolor{textcolor}%
\pgfsetfillcolor{textcolor}%
\pgftext[x=3.190876in,y=1.069250in,,top]{\color{textcolor}\rmfamily\fontsize{36.000000}{43.200000}\selectfont 3}%
\end{pgfscope}%
\begin{pgfscope}%
\pgfpathrectangle{\pgfqpoint{1.535194in}{1.233139in}}{\pgfqpoint{5.425000in}{3.020000in}} %
\pgfusepath{clip}%
\pgfsetroundcap%
\pgfsetroundjoin%
\pgfsetlinewidth{0.803000pt}%
\definecolor{currentstroke}{rgb}{0.800000,0.800000,0.800000}%
\pgfsetstrokecolor{currentstroke}%
\pgfsetdash{}{0pt}%
\pgfpathmoveto{\pgfqpoint{3.895421in}{1.233139in}}%
\pgfpathlineto{\pgfqpoint{3.895421in}{4.253139in}}%
\pgfusepath{stroke}%
\end{pgfscope}%
\begin{pgfscope}%
\definecolor{textcolor}{rgb}{0.150000,0.150000,0.150000}%
\pgfsetstrokecolor{textcolor}%
\pgfsetfillcolor{textcolor}%
\pgftext[x=3.895421in,y=1.069250in,,top]{\color{textcolor}\rmfamily\fontsize{36.000000}{43.200000}\selectfont 4}%
\end{pgfscope}%
\begin{pgfscope}%
\pgfpathrectangle{\pgfqpoint{1.535194in}{1.233139in}}{\pgfqpoint{5.425000in}{3.020000in}} %
\pgfusepath{clip}%
\pgfsetroundcap%
\pgfsetroundjoin%
\pgfsetlinewidth{0.803000pt}%
\definecolor{currentstroke}{rgb}{0.800000,0.800000,0.800000}%
\pgfsetstrokecolor{currentstroke}%
\pgfsetdash{}{0pt}%
\pgfpathmoveto{\pgfqpoint{4.599967in}{1.233139in}}%
\pgfpathlineto{\pgfqpoint{4.599967in}{4.253139in}}%
\pgfusepath{stroke}%
\end{pgfscope}%
\begin{pgfscope}%
\definecolor{textcolor}{rgb}{0.150000,0.150000,0.150000}%
\pgfsetstrokecolor{textcolor}%
\pgfsetfillcolor{textcolor}%
\pgftext[x=4.599967in,y=1.069250in,,top]{\color{textcolor}\rmfamily\fontsize{36.000000}{43.200000}\selectfont 5}%
\end{pgfscope}%
\begin{pgfscope}%
\pgfpathrectangle{\pgfqpoint{1.535194in}{1.233139in}}{\pgfqpoint{5.425000in}{3.020000in}} %
\pgfusepath{clip}%
\pgfsetroundcap%
\pgfsetroundjoin%
\pgfsetlinewidth{0.803000pt}%
\definecolor{currentstroke}{rgb}{0.800000,0.800000,0.800000}%
\pgfsetstrokecolor{currentstroke}%
\pgfsetdash{}{0pt}%
\pgfpathmoveto{\pgfqpoint{5.304512in}{1.233139in}}%
\pgfpathlineto{\pgfqpoint{5.304512in}{4.253139in}}%
\pgfusepath{stroke}%
\end{pgfscope}%
\begin{pgfscope}%
\definecolor{textcolor}{rgb}{0.150000,0.150000,0.150000}%
\pgfsetstrokecolor{textcolor}%
\pgfsetfillcolor{textcolor}%
\pgftext[x=5.304512in,y=1.069250in,,top]{\color{textcolor}\rmfamily\fontsize{36.000000}{43.200000}\selectfont 6}%
\end{pgfscope}%
\begin{pgfscope}%
\pgfpathrectangle{\pgfqpoint{1.535194in}{1.233139in}}{\pgfqpoint{5.425000in}{3.020000in}} %
\pgfusepath{clip}%
\pgfsetroundcap%
\pgfsetroundjoin%
\pgfsetlinewidth{0.803000pt}%
\definecolor{currentstroke}{rgb}{0.800000,0.800000,0.800000}%
\pgfsetstrokecolor{currentstroke}%
\pgfsetdash{}{0pt}%
\pgfpathmoveto{\pgfqpoint{6.009058in}{1.233139in}}%
\pgfpathlineto{\pgfqpoint{6.009058in}{4.253139in}}%
\pgfusepath{stroke}%
\end{pgfscope}%
\begin{pgfscope}%
\definecolor{textcolor}{rgb}{0.150000,0.150000,0.150000}%
\pgfsetstrokecolor{textcolor}%
\pgfsetfillcolor{textcolor}%
\pgftext[x=6.009058in,y=1.069250in,,top]{\color{textcolor}\rmfamily\fontsize{36.000000}{43.200000}\selectfont 7}%
\end{pgfscope}%
\begin{pgfscope}%
\pgfpathrectangle{\pgfqpoint{1.535194in}{1.233139in}}{\pgfqpoint{5.425000in}{3.020000in}} %
\pgfusepath{clip}%
\pgfsetroundcap%
\pgfsetroundjoin%
\pgfsetlinewidth{0.803000pt}%
\definecolor{currentstroke}{rgb}{0.800000,0.800000,0.800000}%
\pgfsetstrokecolor{currentstroke}%
\pgfsetdash{}{0pt}%
\pgfpathmoveto{\pgfqpoint{6.713603in}{1.233139in}}%
\pgfpathlineto{\pgfqpoint{6.713603in}{4.253139in}}%
\pgfusepath{stroke}%
\end{pgfscope}%
\begin{pgfscope}%
\definecolor{textcolor}{rgb}{0.150000,0.150000,0.150000}%
\pgfsetstrokecolor{textcolor}%
\pgfsetfillcolor{textcolor}%
\pgftext[x=6.713603in,y=1.069250in,,top]{\color{textcolor}\rmfamily\fontsize{36.000000}{43.200000}\selectfont 8}%
\end{pgfscope}%
\begin{pgfscope}%
\definecolor{textcolor}{rgb}{0.150000,0.150000,0.150000}%
\pgfsetstrokecolor{textcolor}%
\pgfsetfillcolor{textcolor}%
\pgftext[x=4.247694in,y=0.569194in,,top]{\color{textcolor}\rmfamily\fontsize{38.000000}{45.600000}\selectfont Workload}%
\end{pgfscope}%
\begin{pgfscope}%
\pgfpathrectangle{\pgfqpoint{1.535194in}{1.233139in}}{\pgfqpoint{5.425000in}{3.020000in}} %
\pgfusepath{clip}%
\pgfsetroundcap%
\pgfsetroundjoin%
\pgfsetlinewidth{0.803000pt}%
\definecolor{currentstroke}{rgb}{0.800000,0.800000,0.800000}%
\pgfsetstrokecolor{currentstroke}%
\pgfsetdash{}{0pt}%
\pgfpathmoveto{\pgfqpoint{1.535194in}{1.233139in}}%
\pgfpathlineto{\pgfqpoint{6.960194in}{1.233139in}}%
\pgfusepath{stroke}%
\end{pgfscope}%
\begin{pgfscope}%
\definecolor{textcolor}{rgb}{0.150000,0.150000,0.150000}%
\pgfsetstrokecolor{textcolor}%
\pgfsetfillcolor{textcolor}%
\pgftext[x=1.141805in,y=1.059639in,left,base]{\color{textcolor}\rmfamily\fontsize{36.000000}{43.200000}\selectfont 0}%
\end{pgfscope}%
\begin{pgfscope}%
\pgfpathrectangle{\pgfqpoint{1.535194in}{1.233139in}}{\pgfqpoint{5.425000in}{3.020000in}} %
\pgfusepath{clip}%
\pgfsetroundcap%
\pgfsetroundjoin%
\pgfsetlinewidth{0.803000pt}%
\definecolor{currentstroke}{rgb}{0.800000,0.800000,0.800000}%
\pgfsetstrokecolor{currentstroke}%
\pgfsetdash{}{0pt}%
\pgfpathmoveto{\pgfqpoint{1.535194in}{2.311710in}}%
\pgfpathlineto{\pgfqpoint{6.960194in}{2.311710in}}%
\pgfusepath{stroke}%
\end{pgfscope}%
\begin{pgfscope}%
\definecolor{textcolor}{rgb}{0.150000,0.150000,0.150000}%
\pgfsetstrokecolor{textcolor}%
\pgfsetfillcolor{textcolor}%
\pgftext[x=0.912305in,y=2.138210in,left,base]{\color{textcolor}\rmfamily\fontsize{36.000000}{43.200000}\selectfont 50}%
\end{pgfscope}%
\begin{pgfscope}%
\pgfpathrectangle{\pgfqpoint{1.535194in}{1.233139in}}{\pgfqpoint{5.425000in}{3.020000in}} %
\pgfusepath{clip}%
\pgfsetroundcap%
\pgfsetroundjoin%
\pgfsetlinewidth{0.803000pt}%
\definecolor{currentstroke}{rgb}{0.800000,0.800000,0.800000}%
\pgfsetstrokecolor{currentstroke}%
\pgfsetdash{}{0pt}%
\pgfpathmoveto{\pgfqpoint{1.535194in}{3.390281in}}%
\pgfpathlineto{\pgfqpoint{6.960194in}{3.390281in}}%
\pgfusepath{stroke}%
\end{pgfscope}%
\begin{pgfscope}%
\definecolor{textcolor}{rgb}{0.150000,0.150000,0.150000}%
\pgfsetstrokecolor{textcolor}%
\pgfsetfillcolor{textcolor}%
\pgftext[x=0.682805in,y=3.216781in,left,base]{\color{textcolor}\rmfamily\fontsize{36.000000}{43.200000}\selectfont 100}%
\end{pgfscope}%
\begin{pgfscope}%
\definecolor{textcolor}{rgb}{0.150000,0.150000,0.150000}%
\pgfsetstrokecolor{textcolor}%
\pgfsetfillcolor{textcolor}%
\pgftext[x=0.627250in,y=2.743139in,,bottom,rotate=90.000000]{\color{textcolor}\rmfamily\fontsize{38.000000}{45.600000}\selectfont Size (GB)}%
\end{pgfscope}%
\begin{pgfscope}%
\pgfpathrectangle{\pgfqpoint{1.535194in}{1.233139in}}{\pgfqpoint{5.425000in}{3.020000in}} %
\pgfusepath{clip}%
\pgfsetbuttcap%
\pgfsetroundjoin%
\pgfsetlinewidth{3.011250pt}%
\definecolor{currentstroke}{rgb}{0.866667,0.517647,0.321569}%
\pgfsetstrokecolor{currentstroke}%
\pgfsetdash{{3.000000pt}{0.000000pt}}{0.000000pt}%
\pgfpathmoveto{\pgfqpoint{1.781785in}{1.546546in}}%
\pgfpathlineto{\pgfqpoint{2.486331in}{2.063342in}}%
\pgfpathlineto{\pgfqpoint{3.190876in}{2.757665in}}%
\pgfpathlineto{\pgfqpoint{3.895421in}{2.757687in}}%
\pgfpathlineto{\pgfqpoint{4.599967in}{2.757719in}}%
\pgfpathlineto{\pgfqpoint{5.304512in}{2.757788in}}%
\pgfpathlineto{\pgfqpoint{6.009058in}{2.757910in}}%
\pgfpathlineto{\pgfqpoint{6.713603in}{2.757922in}}%
\pgfusepath{stroke}%
\end{pgfscope}%
\begin{pgfscope}%
\pgfpathrectangle{\pgfqpoint{1.535194in}{1.233139in}}{\pgfqpoint{5.425000in}{3.020000in}} %
\pgfusepath{clip}%
\pgfsetbuttcap%
\pgfsetroundjoin%
\definecolor{currentfill}{rgb}{0.866667,0.517647,0.321569}%
\pgfsetfillcolor{currentfill}%
\pgfsetlinewidth{0.752812pt}%
\definecolor{currentstroke}{rgb}{1.000000,1.000000,1.000000}%
\pgfsetstrokecolor{currentstroke}%
\pgfsetdash{}{0pt}%
\pgfsys@defobject{currentmarker}{\pgfqpoint{-0.104167in}{-0.104167in}}{\pgfqpoint{0.104167in}{0.104167in}}{%
\pgfpathmoveto{\pgfqpoint{0.000000in}{-0.104167in}}%
\pgfpathcurveto{\pgfqpoint{0.027625in}{-0.104167in}}{\pgfqpoint{0.054123in}{-0.093191in}}{\pgfqpoint{0.073657in}{-0.073657in}}%
\pgfpathcurveto{\pgfqpoint{0.093191in}{-0.054123in}}{\pgfqpoint{0.104167in}{-0.027625in}}{\pgfqpoint{0.104167in}{0.000000in}}%
\pgfpathcurveto{\pgfqpoint{0.104167in}{0.027625in}}{\pgfqpoint{0.093191in}{0.054123in}}{\pgfqpoint{0.073657in}{0.073657in}}%
\pgfpathcurveto{\pgfqpoint{0.054123in}{0.093191in}}{\pgfqpoint{0.027625in}{0.104167in}}{\pgfqpoint{0.000000in}{0.104167in}}%
\pgfpathcurveto{\pgfqpoint{-0.027625in}{0.104167in}}{\pgfqpoint{-0.054123in}{0.093191in}}{\pgfqpoint{-0.073657in}{0.073657in}}%
\pgfpathcurveto{\pgfqpoint{-0.093191in}{0.054123in}}{\pgfqpoint{-0.104167in}{0.027625in}}{\pgfqpoint{-0.104167in}{0.000000in}}%
\pgfpathcurveto{\pgfqpoint{-0.104167in}{-0.027625in}}{\pgfqpoint{-0.093191in}{-0.054123in}}{\pgfqpoint{-0.073657in}{-0.073657in}}%
\pgfpathcurveto{\pgfqpoint{-0.054123in}{-0.093191in}}{\pgfqpoint{-0.027625in}{-0.104167in}}{\pgfqpoint{0.000000in}{-0.104167in}}%
\pgfpathclose%
\pgfusepath{stroke,fill}%
}%
\begin{pgfscope}%
\pgfsys@transformshift{1.781785in}{1.546546in}%
\pgfsys@useobject{currentmarker}{}%
\end{pgfscope}%
\begin{pgfscope}%
\pgfsys@transformshift{2.486331in}{2.063342in}%
\pgfsys@useobject{currentmarker}{}%
\end{pgfscope}%
\begin{pgfscope}%
\pgfsys@transformshift{3.190876in}{2.757665in}%
\pgfsys@useobject{currentmarker}{}%
\end{pgfscope}%
\begin{pgfscope}%
\pgfsys@transformshift{3.895421in}{2.757687in}%
\pgfsys@useobject{currentmarker}{}%
\end{pgfscope}%
\begin{pgfscope}%
\pgfsys@transformshift{4.599967in}{2.757719in}%
\pgfsys@useobject{currentmarker}{}%
\end{pgfscope}%
\begin{pgfscope}%
\pgfsys@transformshift{5.304512in}{2.757788in}%
\pgfsys@useobject{currentmarker}{}%
\end{pgfscope}%
\begin{pgfscope}%
\pgfsys@transformshift{6.009058in}{2.757910in}%
\pgfsys@useobject{currentmarker}{}%
\end{pgfscope}%
\begin{pgfscope}%
\pgfsys@transformshift{6.713603in}{2.757922in}%
\pgfsys@useobject{currentmarker}{}%
\end{pgfscope}%
\end{pgfscope}%
\begin{pgfscope}%
\pgfpathrectangle{\pgfqpoint{1.535194in}{1.233139in}}{\pgfqpoint{5.425000in}{3.020000in}} %
\pgfusepath{clip}%
\pgfsetbuttcap%
\pgfsetroundjoin%
\pgfsetlinewidth{3.011250pt}%
\definecolor{currentstroke}{rgb}{0.298039,0.447059,0.690196}%
\pgfsetstrokecolor{currentstroke}%
\pgfsetdash{{6.000000pt}{6.000000pt}}{0.000000pt}%
\pgfpathmoveto{\pgfqpoint{1.781785in}{1.546546in}}%
\pgfpathlineto{\pgfqpoint{2.486331in}{2.063342in}}%
\pgfpathlineto{\pgfqpoint{3.190876in}{3.758523in}}%
\pgfpathlineto{\pgfqpoint{3.895421in}{3.758544in}}%
\pgfpathlineto{\pgfqpoint{4.599967in}{3.880597in}}%
\pgfpathlineto{\pgfqpoint{5.304512in}{3.880666in}}%
\pgfpathlineto{\pgfqpoint{6.009058in}{3.880666in}}%
\pgfpathlineto{\pgfqpoint{6.713603in}{3.899904in}}%
\pgfusepath{stroke}%
\end{pgfscope}%
\begin{pgfscope}%
\pgfpathrectangle{\pgfqpoint{1.535194in}{1.233139in}}{\pgfqpoint{5.425000in}{3.020000in}} %
\pgfusepath{clip}%
\pgfsetbuttcap%
\pgfsetmiterjoin%
\definecolor{currentfill}{rgb}{0.298039,0.447059,0.690196}%
\pgfsetfillcolor{currentfill}%
\pgfsetlinewidth{0.752812pt}%
\definecolor{currentstroke}{rgb}{1.000000,1.000000,1.000000}%
\pgfsetstrokecolor{currentstroke}%
\pgfsetdash{}{0pt}%
\pgfsys@defobject{currentmarker}{\pgfqpoint{-0.104167in}{-0.104167in}}{\pgfqpoint{0.104167in}{0.104167in}}{%
\pgfpathmoveto{\pgfqpoint{0.000000in}{0.104167in}}%
\pgfpathlineto{\pgfqpoint{-0.104167in}{-0.104167in}}%
\pgfpathlineto{\pgfqpoint{0.104167in}{-0.104167in}}%
\pgfpathclose%
\pgfusepath{stroke,fill}%
}%
\begin{pgfscope}%
\pgfsys@transformshift{1.781785in}{1.546546in}%
\pgfsys@useobject{currentmarker}{}%
\end{pgfscope}%
\begin{pgfscope}%
\pgfsys@transformshift{2.486331in}{2.063342in}%
\pgfsys@useobject{currentmarker}{}%
\end{pgfscope}%
\begin{pgfscope}%
\pgfsys@transformshift{3.190876in}{3.758523in}%
\pgfsys@useobject{currentmarker}{}%
\end{pgfscope}%
\begin{pgfscope}%
\pgfsys@transformshift{3.895421in}{3.758544in}%
\pgfsys@useobject{currentmarker}{}%
\end{pgfscope}%
\begin{pgfscope}%
\pgfsys@transformshift{4.599967in}{3.880597in}%
\pgfsys@useobject{currentmarker}{}%
\end{pgfscope}%
\begin{pgfscope}%
\pgfsys@transformshift{5.304512in}{3.880666in}%
\pgfsys@useobject{currentmarker}{}%
\end{pgfscope}%
\begin{pgfscope}%
\pgfsys@transformshift{6.009058in}{3.880666in}%
\pgfsys@useobject{currentmarker}{}%
\end{pgfscope}%
\begin{pgfscope}%
\pgfsys@transformshift{6.713603in}{3.899904in}%
\pgfsys@useobject{currentmarker}{}%
\end{pgfscope}%
\end{pgfscope}%
\begin{pgfscope}%
\pgfsetrectcap%
\pgfsetmiterjoin%
\pgfsetlinewidth{1.003750pt}%
\definecolor{currentstroke}{rgb}{0.800000,0.800000,0.800000}%
\pgfsetstrokecolor{currentstroke}%
\pgfsetdash{}{0pt}%
\pgfpathmoveto{\pgfqpoint{1.535194in}{1.233139in}}%
\pgfpathlineto{\pgfqpoint{1.535194in}{4.253139in}}%
\pgfusepath{stroke}%
\end{pgfscope}%
\begin{pgfscope}%
\pgfsetrectcap%
\pgfsetmiterjoin%
\pgfsetlinewidth{1.003750pt}%
\definecolor{currentstroke}{rgb}{0.800000,0.800000,0.800000}%
\pgfsetstrokecolor{currentstroke}%
\pgfsetdash{}{0pt}%
\pgfpathmoveto{\pgfqpoint{6.960194in}{1.233139in}}%
\pgfpathlineto{\pgfqpoint{6.960194in}{4.253139in}}%
\pgfusepath{stroke}%
\end{pgfscope}%
\begin{pgfscope}%
\pgfsetrectcap%
\pgfsetmiterjoin%
\pgfsetlinewidth{1.003750pt}%
\definecolor{currentstroke}{rgb}{0.800000,0.800000,0.800000}%
\pgfsetstrokecolor{currentstroke}%
\pgfsetdash{}{0pt}%
\pgfpathmoveto{\pgfqpoint{1.535194in}{1.233139in}}%
\pgfpathlineto{\pgfqpoint{6.960194in}{1.233139in}}%
\pgfusepath{stroke}%
\end{pgfscope}%
\begin{pgfscope}%
\pgfsetrectcap%
\pgfsetmiterjoin%
\pgfsetlinewidth{1.003750pt}%
\definecolor{currentstroke}{rgb}{0.800000,0.800000,0.800000}%
\pgfsetstrokecolor{currentstroke}%
\pgfsetdash{}{0pt}%
\pgfpathmoveto{\pgfqpoint{1.535194in}{4.253139in}}%
\pgfpathlineto{\pgfqpoint{6.960194in}{4.253139in}}%
\pgfusepath{stroke}%
\end{pgfscope}%
\begin{pgfscope}%
\pgfsetroundcap%
\pgfsetroundjoin%
\pgfsetlinewidth{3.011250pt}%
\definecolor{currentstroke}{rgb}{1.000000,1.000000,1.000000}%
\pgfsetstrokecolor{currentstroke}%
\pgfsetdash{}{0pt}%
\pgfpathmoveto{\pgfqpoint{1.576194in}{4.348416in}}%
\pgfpathlineto{\pgfqpoint{2.209528in}{4.348416in}}%
\pgfusepath{stroke}%
\end{pgfscope}%
\begin{pgfscope}%
\pgfsetbuttcap%
\pgfsetroundjoin%
\pgfsetlinewidth{3.011250pt}%
\definecolor{currentstroke}{rgb}{0.866667,0.517647,0.321569}%
\pgfsetstrokecolor{currentstroke}%
\pgfsetdash{{3.000000pt}{0.000000pt}}{0.000000pt}%
\pgfpathmoveto{\pgfqpoint{1.576194in}{3.928833in}}%
\pgfpathlineto{\pgfqpoint{2.209528in}{3.928833in}}%
\pgfusepath{stroke}%
\end{pgfscope}%
\begin{pgfscope}%
\pgfsetbuttcap%
\pgfsetroundjoin%
\definecolor{currentfill}{rgb}{0.866667,0.517647,0.321569}%
\pgfsetfillcolor{currentfill}%
\pgfsetlinewidth{0.000000pt}%
\definecolor{currentstroke}{rgb}{0.866667,0.517647,0.321569}%
\pgfsetstrokecolor{currentstroke}%
\pgfsetdash{}{0pt}%
\pgfsys@defobject{currentmarker}{\pgfqpoint{-0.104167in}{-0.104167in}}{\pgfqpoint{0.104167in}{0.104167in}}{%
\pgfpathmoveto{\pgfqpoint{0.000000in}{-0.104167in}}%
\pgfpathcurveto{\pgfqpoint{0.027625in}{-0.104167in}}{\pgfqpoint{0.054123in}{-0.093191in}}{\pgfqpoint{0.073657in}{-0.073657in}}%
\pgfpathcurveto{\pgfqpoint{0.093191in}{-0.054123in}}{\pgfqpoint{0.104167in}{-0.027625in}}{\pgfqpoint{0.104167in}{0.000000in}}%
\pgfpathcurveto{\pgfqpoint{0.104167in}{0.027625in}}{\pgfqpoint{0.093191in}{0.054123in}}{\pgfqpoint{0.073657in}{0.073657in}}%
\pgfpathcurveto{\pgfqpoint{0.054123in}{0.093191in}}{\pgfqpoint{0.027625in}{0.104167in}}{\pgfqpoint{0.000000in}{0.104167in}}%
\pgfpathcurveto{\pgfqpoint{-0.027625in}{0.104167in}}{\pgfqpoint{-0.054123in}{0.093191in}}{\pgfqpoint{-0.073657in}{0.073657in}}%
\pgfpathcurveto{\pgfqpoint{-0.093191in}{0.054123in}}{\pgfqpoint{-0.104167in}{0.027625in}}{\pgfqpoint{-0.104167in}{0.000000in}}%
\pgfpathcurveto{\pgfqpoint{-0.104167in}{-0.027625in}}{\pgfqpoint{-0.093191in}{-0.054123in}}{\pgfqpoint{-0.073657in}{-0.073657in}}%
\pgfpathcurveto{\pgfqpoint{-0.054123in}{-0.093191in}}{\pgfqpoint{-0.027625in}{-0.104167in}}{\pgfqpoint{0.000000in}{-0.104167in}}%
\pgfpathclose%
\pgfusepath{fill}%
}%
\begin{pgfscope}%
\pgfsys@transformshift{1.892861in}{3.928833in}%
\pgfsys@useobject{currentmarker}{}%
\end{pgfscope}%
\end{pgfscope}%
\begin{pgfscope}%
\definecolor{textcolor}{rgb}{0.150000,0.150000,0.150000}%
\pgfsetstrokecolor{textcolor}%
\pgfsetfillcolor{textcolor}%
\pgftext[x=2.262305in,y=3.744111in,left,base]{\color{textcolor}\rmfamily\fontsize{38.000000}{45.600000}\selectfont SM}%
\end{pgfscope}%
\begin{pgfscope}%
\pgfsetbuttcap%
\pgfsetroundjoin%
\pgfsetlinewidth{3.011250pt}%
\definecolor{currentstroke}{rgb}{0.298039,0.447059,0.690196}%
\pgfsetstrokecolor{currentstroke}%
\pgfsetdash{{6.000000pt}{6.000000pt}}{0.000000pt}%
\pgfpathmoveto{\pgfqpoint{1.576194in}{3.509250in}}%
\pgfpathlineto{\pgfqpoint{2.209528in}{3.509250in}}%
\pgfusepath{stroke}%
\end{pgfscope}%
\begin{pgfscope}%
\pgfsetbuttcap%
\pgfsetmiterjoin%
\definecolor{currentfill}{rgb}{0.298039,0.447059,0.690196}%
\pgfsetfillcolor{currentfill}%
\pgfsetlinewidth{0.000000pt}%
\definecolor{currentstroke}{rgb}{0.298039,0.447059,0.690196}%
\pgfsetstrokecolor{currentstroke}%
\pgfsetdash{}{0pt}%
\pgfsys@defobject{currentmarker}{\pgfqpoint{-0.104167in}{-0.104167in}}{\pgfqpoint{0.104167in}{0.104167in}}{%
\pgfpathmoveto{\pgfqpoint{0.000000in}{0.104167in}}%
\pgfpathlineto{\pgfqpoint{-0.104167in}{-0.104167in}}%
\pgfpathlineto{\pgfqpoint{0.104167in}{-0.104167in}}%
\pgfpathclose%
\pgfusepath{fill}%
}%
\begin{pgfscope}%
\pgfsys@transformshift{1.892861in}{3.509250in}%
\pgfsys@useobject{currentmarker}{}%
\end{pgfscope}%
\end{pgfscope}%
\begin{pgfscope}%
\definecolor{textcolor}{rgb}{0.150000,0.150000,0.150000}%
\pgfsetstrokecolor{textcolor}%
\pgfsetfillcolor{textcolor}%
\pgftext[x=2.262305in,y=3.324528in,left,base]{\color{textcolor}\rmfamily\fontsize{38.000000}{45.600000}\selectfont SA}%
\end{pgfscope}%
\end{pgfpicture}%
\makeatother%
\endgroup%
%
}
\caption{Budget = 64 GB}
\end{subfigure}
\caption{Real size of the stored artifact for storage aware and simple materialization under different budgets (SA = storage-aware materialization, SM = simple materialization)  \todo[inline]{add total size and a straight line for budget}}
\label{exp-sa-vs-simple-size}
\end{figure}
Figure \ref{exp-sa-vs-simple-size} shows the real size of the stored artifacts under different budgets.
For the simple materialization algorithm, the maximum real size of the stored artifact never surpasses the budget.
However, in the storage-aware algorithm, we observe that the real size of the stored artifacts is 2 to 3 times more than the specified budget.
This shows that the rate of the duplicated columns in the artifacts of ML workloads is generally high.
Therefore, our storage-aware materialization algorithm can materialize more artifacts by de-duplicating the columns before storing them.

\textbf{Model Materialization. }
The goal of our materialization algorithm is to materialize high-quality models as soon as they appear in a new workload.
In this experiment, we run a typical scenario, where users always compare the score of their models with the score of the best performing model in the collaborative environment.
The OpenML platform assigns a time-stamp to every executed workload.
In this experiment, we execute all the OpenML workloads one by one and keep track of the best performing workload.
In every workload, as a post-processing step, we compare the score of the model with the score of the best performing model so far.
Figure \ref{exp-model-materialization} shows the result.
\todo[inline]{plot another figure, showing after how many repeated execution of the best pipeline did our system materializes it. It should be after 1 Unless the best pipeline has a very high run time.}

\begin{figure}
\begin{subfigure}[b]{0.5\linewidth}
\centering
 \resizebox{\columnwidth}{!}{%
%% Creator: Matplotlib, PGF backend
%%
%% To include the figure in your LaTeX document, write
%%   \input{<filename>.pgf}
%%
%% Make sure the required packages are loaded in your preamble
%%   \usepackage{pgf}
%%
%% Figures using additional raster images can only be included by \input if
%% they are in the same directory as the main LaTeX file. For loading figures
%% from other directories you can use the `import` package
%%   \usepackage{import}
%% and then include the figures with
%%   \import{<path to file>}{<filename>.pgf}
%%
%% Matplotlib used the following preamble
%%   \usepackage{fontspec}
%%   \setmonofont{Andale Mono}
%%
\begingroup%
\makeatletter%
\begin{pgfpicture}%
\pgfpathrectangle{\pgfpointorigin}{\pgfqpoint{8.039987in}{5.001044in}}%
\pgfusepath{use as bounding box, clip}%
\begin{pgfscope}%
\pgfsetbuttcap%
\pgfsetmiterjoin%
\definecolor{currentfill}{rgb}{1.000000,1.000000,1.000000}%
\pgfsetfillcolor{currentfill}%
\pgfsetlinewidth{0.000000pt}%
\definecolor{currentstroke}{rgb}{1.000000,1.000000,1.000000}%
\pgfsetstrokecolor{currentstroke}%
\pgfsetdash{}{0pt}%
\pgfpathmoveto{\pgfqpoint{0.000000in}{0.000000in}}%
\pgfpathlineto{\pgfqpoint{8.039987in}{0.000000in}}%
\pgfpathlineto{\pgfqpoint{8.039987in}{5.001044in}}%
\pgfpathlineto{\pgfqpoint{0.000000in}{5.001044in}}%
\pgfpathclose%
\pgfusepath{fill}%
\end{pgfscope}%
\begin{pgfscope}%
\pgfsetbuttcap%
\pgfsetmiterjoin%
\definecolor{currentfill}{rgb}{1.000000,1.000000,1.000000}%
\pgfsetfillcolor{currentfill}%
\pgfsetlinewidth{0.000000pt}%
\definecolor{currentstroke}{rgb}{0.000000,0.000000,0.000000}%
\pgfsetstrokecolor{currentstroke}%
\pgfsetstrokeopacity{0.000000}%
\pgfsetdash{}{0pt}%
\pgfpathmoveto{\pgfqpoint{2.302578in}{1.285444in}}%
\pgfpathlineto{\pgfqpoint{7.727578in}{1.285444in}}%
\pgfpathlineto{\pgfqpoint{7.727578in}{4.305444in}}%
\pgfpathlineto{\pgfqpoint{2.302578in}{4.305444in}}%
\pgfpathclose%
\pgfusepath{fill}%
\end{pgfscope}%
\begin{pgfscope}%
\pgfpathrectangle{\pgfqpoint{2.302578in}{1.285444in}}{\pgfqpoint{5.425000in}{3.020000in}} %
\pgfusepath{clip}%
\pgfsetroundcap%
\pgfsetroundjoin%
\pgfsetlinewidth{0.803000pt}%
\definecolor{currentstroke}{rgb}{0.800000,0.800000,0.800000}%
\pgfsetstrokecolor{currentstroke}%
\pgfsetdash{}{0pt}%
\pgfpathmoveto{\pgfqpoint{2.546701in}{1.285444in}}%
\pgfpathlineto{\pgfqpoint{2.546701in}{4.305444in}}%
\pgfusepath{stroke}%
\end{pgfscope}%
\begin{pgfscope}%
\definecolor{textcolor}{rgb}{0.150000,0.150000,0.150000}%
\pgfsetstrokecolor{textcolor}%
\pgfsetfillcolor{textcolor}%
\pgftext[x=2.546701in,y=1.121555in,,top]{\color{textcolor}\rmfamily\fontsize{36.000000}{43.200000}\selectfont 0}%
\end{pgfscope}%
\begin{pgfscope}%
\pgfpathrectangle{\pgfqpoint{2.302578in}{1.285444in}}{\pgfqpoint{5.425000in}{3.020000in}} %
\pgfusepath{clip}%
\pgfsetroundcap%
\pgfsetroundjoin%
\pgfsetlinewidth{0.803000pt}%
\definecolor{currentstroke}{rgb}{0.800000,0.800000,0.800000}%
\pgfsetstrokecolor{currentstroke}%
\pgfsetdash{}{0pt}%
\pgfpathmoveto{\pgfqpoint{5.013844in}{1.285444in}}%
\pgfpathlineto{\pgfqpoint{5.013844in}{4.305444in}}%
\pgfusepath{stroke}%
\end{pgfscope}%
\begin{pgfscope}%
\definecolor{textcolor}{rgb}{0.150000,0.150000,0.150000}%
\pgfsetstrokecolor{textcolor}%
\pgfsetfillcolor{textcolor}%
\pgftext[x=5.013844in,y=1.121555in,,top]{\color{textcolor}\rmfamily\fontsize{36.000000}{43.200000}\selectfont 1000}%
\end{pgfscope}%
\begin{pgfscope}%
\pgfpathrectangle{\pgfqpoint{2.302578in}{1.285444in}}{\pgfqpoint{5.425000in}{3.020000in}} %
\pgfusepath{clip}%
\pgfsetroundcap%
\pgfsetroundjoin%
\pgfsetlinewidth{0.803000pt}%
\definecolor{currentstroke}{rgb}{0.800000,0.800000,0.800000}%
\pgfsetstrokecolor{currentstroke}%
\pgfsetdash{}{0pt}%
\pgfpathmoveto{\pgfqpoint{7.480987in}{1.285444in}}%
\pgfpathlineto{\pgfqpoint{7.480987in}{4.305444in}}%
\pgfusepath{stroke}%
\end{pgfscope}%
\begin{pgfscope}%
\definecolor{textcolor}{rgb}{0.150000,0.150000,0.150000}%
\pgfsetstrokecolor{textcolor}%
\pgfsetfillcolor{textcolor}%
\pgftext[x=7.480987in,y=1.121555in,,top]{\color{textcolor}\rmfamily\fontsize{36.000000}{43.200000}\selectfont 2000}%
\end{pgfscope}%
\begin{pgfscope}%
\definecolor{textcolor}{rgb}{0.150000,0.150000,0.150000}%
\pgfsetstrokecolor{textcolor}%
\pgfsetfillcolor{textcolor}%
\pgftext[x=5.015078in,y=0.621500in,,top]{\color{textcolor}\rmfamily\fontsize{42.000000}{50.400000}\selectfont OpenML Workload}%
\end{pgfscope}%
\begin{pgfscope}%
\pgfpathrectangle{\pgfqpoint{2.302578in}{1.285444in}}{\pgfqpoint{5.425000in}{3.020000in}} %
\pgfusepath{clip}%
\pgfsetroundcap%
\pgfsetroundjoin%
\pgfsetlinewidth{0.803000pt}%
\definecolor{currentstroke}{rgb}{0.800000,0.800000,0.800000}%
\pgfsetstrokecolor{currentstroke}%
\pgfsetdash{}{0pt}%
\pgfpathmoveto{\pgfqpoint{2.302578in}{1.285444in}}%
\pgfpathlineto{\pgfqpoint{7.727578in}{1.285444in}}%
\pgfusepath{stroke}%
\end{pgfscope}%
\begin{pgfscope}%
\definecolor{textcolor}{rgb}{0.150000,0.150000,0.150000}%
\pgfsetstrokecolor{textcolor}%
\pgfsetfillcolor{textcolor}%
\pgftext[x=1.909189in,y=1.111944in,left,base]{\color{textcolor}\rmfamily\fontsize{36.000000}{43.200000}\selectfont 0}%
\end{pgfscope}%
\begin{pgfscope}%
\pgfpathrectangle{\pgfqpoint{2.302578in}{1.285444in}}{\pgfqpoint{5.425000in}{3.020000in}} %
\pgfusepath{clip}%
\pgfsetroundcap%
\pgfsetroundjoin%
\pgfsetlinewidth{0.803000pt}%
\definecolor{currentstroke}{rgb}{0.800000,0.800000,0.800000}%
\pgfsetstrokecolor{currentstroke}%
\pgfsetdash{}{0pt}%
\pgfpathmoveto{\pgfqpoint{2.302578in}{1.806134in}}%
\pgfpathlineto{\pgfqpoint{7.727578in}{1.806134in}}%
\pgfusepath{stroke}%
\end{pgfscope}%
\begin{pgfscope}%
\definecolor{textcolor}{rgb}{0.150000,0.150000,0.150000}%
\pgfsetstrokecolor{textcolor}%
\pgfsetfillcolor{textcolor}%
\pgftext[x=1.450189in,y=1.632634in,left,base]{\color{textcolor}\rmfamily\fontsize{36.000000}{43.200000}\selectfont 500}%
\end{pgfscope}%
\begin{pgfscope}%
\pgfpathrectangle{\pgfqpoint{2.302578in}{1.285444in}}{\pgfqpoint{5.425000in}{3.020000in}} %
\pgfusepath{clip}%
\pgfsetroundcap%
\pgfsetroundjoin%
\pgfsetlinewidth{0.803000pt}%
\definecolor{currentstroke}{rgb}{0.800000,0.800000,0.800000}%
\pgfsetstrokecolor{currentstroke}%
\pgfsetdash{}{0pt}%
\pgfpathmoveto{\pgfqpoint{2.302578in}{2.326823in}}%
\pgfpathlineto{\pgfqpoint{7.727578in}{2.326823in}}%
\pgfusepath{stroke}%
\end{pgfscope}%
\begin{pgfscope}%
\definecolor{textcolor}{rgb}{0.150000,0.150000,0.150000}%
\pgfsetstrokecolor{textcolor}%
\pgfsetfillcolor{textcolor}%
\pgftext[x=1.666689in,y=2.153323in,left,base]{\color{textcolor}\rmfamily\fontsize{36.000000}{43.200000}\selectfont 1k}%
\end{pgfscope}%
\begin{pgfscope}%
\pgfpathrectangle{\pgfqpoint{2.302578in}{1.285444in}}{\pgfqpoint{5.425000in}{3.020000in}} %
\pgfusepath{clip}%
\pgfsetroundcap%
\pgfsetroundjoin%
\pgfsetlinewidth{0.803000pt}%
\definecolor{currentstroke}{rgb}{0.800000,0.800000,0.800000}%
\pgfsetstrokecolor{currentstroke}%
\pgfsetdash{}{0pt}%
\pgfpathmoveto{\pgfqpoint{2.302578in}{2.847513in}}%
\pgfpathlineto{\pgfqpoint{7.727578in}{2.847513in}}%
\pgfusepath{stroke}%
\end{pgfscope}%
\begin{pgfscope}%
\definecolor{textcolor}{rgb}{0.150000,0.150000,0.150000}%
\pgfsetstrokecolor{textcolor}%
\pgfsetfillcolor{textcolor}%
\pgftext[x=1.312189in,y=2.674013in,left,base]{\color{textcolor}\rmfamily\fontsize{36.000000}{43.200000}\selectfont 1.5k}%
\end{pgfscope}%
\begin{pgfscope}%
\pgfpathrectangle{\pgfqpoint{2.302578in}{1.285444in}}{\pgfqpoint{5.425000in}{3.020000in}} %
\pgfusepath{clip}%
\pgfsetroundcap%
\pgfsetroundjoin%
\pgfsetlinewidth{0.803000pt}%
\definecolor{currentstroke}{rgb}{0.800000,0.800000,0.800000}%
\pgfsetstrokecolor{currentstroke}%
\pgfsetdash{}{0pt}%
\pgfpathmoveto{\pgfqpoint{2.302578in}{3.368203in}}%
\pgfpathlineto{\pgfqpoint{7.727578in}{3.368203in}}%
\pgfusepath{stroke}%
\end{pgfscope}%
\begin{pgfscope}%
\definecolor{textcolor}{rgb}{0.150000,0.150000,0.150000}%
\pgfsetstrokecolor{textcolor}%
\pgfsetfillcolor{textcolor}%
\pgftext[x=1.666689in,y=3.194703in,left,base]{\color{textcolor}\rmfamily\fontsize{36.000000}{43.200000}\selectfont 2k}%
\end{pgfscope}%
\begin{pgfscope}%
\pgfpathrectangle{\pgfqpoint{2.302578in}{1.285444in}}{\pgfqpoint{5.425000in}{3.020000in}} %
\pgfusepath{clip}%
\pgfsetroundcap%
\pgfsetroundjoin%
\pgfsetlinewidth{0.803000pt}%
\definecolor{currentstroke}{rgb}{0.800000,0.800000,0.800000}%
\pgfsetstrokecolor{currentstroke}%
\pgfsetdash{}{0pt}%
\pgfpathmoveto{\pgfqpoint{2.302578in}{3.888892in}}%
\pgfpathlineto{\pgfqpoint{7.727578in}{3.888892in}}%
\pgfusepath{stroke}%
\end{pgfscope}%
\begin{pgfscope}%
\definecolor{textcolor}{rgb}{0.150000,0.150000,0.150000}%
\pgfsetstrokecolor{textcolor}%
\pgfsetfillcolor{textcolor}%
\pgftext[x=1.312189in,y=3.715392in,left,base]{\color{textcolor}\rmfamily\fontsize{36.000000}{43.200000}\selectfont 2.5k}%
\end{pgfscope}%
\begin{pgfscope}%
\definecolor{textcolor}{rgb}{0.150000,0.150000,0.150000}%
\pgfsetstrokecolor{textcolor}%
\pgfsetfillcolor{textcolor}%
\pgftext[x=0.472750in,y=1.361611in,left,base,rotate=90.000000]{\color{textcolor}\rmfamily\fontsize{42.000000}{50.400000}\selectfont Cumulative }%
\end{pgfscope}%
\begin{pgfscope}%
\definecolor{textcolor}{rgb}{0.150000,0.150000,0.150000}%
\pgfsetstrokecolor{textcolor}%
\pgfsetfillcolor{textcolor}%
\pgftext[x=1.110800in,y=1.207902in,left,base,rotate=90.000000]{\color{textcolor}\rmfamily\fontsize{42.000000}{50.400000}\selectfont Run Time (s)}%
\end{pgfscope}%
\begin{pgfscope}%
\pgfpathrectangle{\pgfqpoint{2.302578in}{1.285444in}}{\pgfqpoint{5.425000in}{3.020000in}} %
\pgfusepath{clip}%
\pgfsetbuttcap%
\pgfsetroundjoin%
\pgfsetlinewidth{3.011250pt}%
\definecolor{currentstroke}{rgb}{0.298039,0.447059,0.690196}%
\pgfsetstrokecolor{currentstroke}%
\pgfsetdash{{3.000000pt}{0.000000pt}}{0.000000pt}%
\pgfpathmoveto{\pgfqpoint{2.549169in}{1.287083in}}%
\pgfpathlineto{\pgfqpoint{2.556570in}{1.291719in}}%
\pgfpathlineto{\pgfqpoint{2.559037in}{1.312155in}}%
\pgfpathlineto{\pgfqpoint{2.588643in}{1.330736in}}%
\pgfpathlineto{\pgfqpoint{2.593577in}{1.332853in}}%
\pgfpathlineto{\pgfqpoint{2.605913in}{1.343989in}}%
\pgfpathlineto{\pgfqpoint{2.608380in}{1.344040in}}%
\pgfpathlineto{\pgfqpoint{2.620716in}{1.353818in}}%
\pgfpathlineto{\pgfqpoint{2.625650in}{1.364138in}}%
\pgfpathlineto{\pgfqpoint{2.633051in}{1.369849in}}%
\pgfpathlineto{\pgfqpoint{3.471880in}{1.386911in}}%
\pgfpathlineto{\pgfqpoint{3.489150in}{1.388507in}}%
\pgfpathlineto{\pgfqpoint{3.491617in}{1.402249in}}%
\pgfpathlineto{\pgfqpoint{3.688988in}{1.436040in}}%
\pgfpathlineto{\pgfqpoint{3.735864in}{1.442797in}}%
\pgfpathlineto{\pgfqpoint{3.812346in}{1.454905in}}%
\pgfpathlineto{\pgfqpoint{3.829616in}{1.457643in}}%
\pgfpathlineto{\pgfqpoint{3.874024in}{1.463970in}}%
\pgfpathlineto{\pgfqpoint{3.920900in}{1.471805in}}%
\pgfpathlineto{\pgfqpoint{3.999848in}{1.483666in}}%
\pgfpathlineto{\pgfqpoint{4.017118in}{1.486244in}}%
\pgfpathlineto{\pgfqpoint{4.076330in}{1.495179in}}%
\pgfpathlineto{\pgfqpoint{4.105936in}{1.500435in}}%
\pgfpathlineto{\pgfqpoint{4.140476in}{1.506326in}}%
\pgfpathlineto{\pgfqpoint{4.271234in}{1.526224in}}%
\pgfpathlineto{\pgfqpoint{5.191478in}{1.666478in}}%
\pgfpathlineto{\pgfqpoint{5.233420in}{1.673203in}}%
\pgfpathlineto{\pgfqpoint{5.749053in}{1.752883in}}%
\pgfpathlineto{\pgfqpoint{5.983431in}{1.787665in}}%
\pgfpathlineto{\pgfqpoint{7.051704in}{1.948431in}}%
\pgfpathlineto{\pgfqpoint{7.088711in}{1.954188in}}%
\pgfpathlineto{\pgfqpoint{7.135587in}{1.961775in}}%
\pgfpathlineto{\pgfqpoint{7.157791in}{1.965440in}}%
\pgfpathlineto{\pgfqpoint{7.179995in}{1.968675in}}%
\pgfpathlineto{\pgfqpoint{7.182462in}{2.016829in}}%
\pgfpathlineto{\pgfqpoint{7.202200in}{2.020000in}}%
\pgfpathlineto{\pgfqpoint{7.480987in}{2.062569in}}%
\pgfpathlineto{\pgfqpoint{7.480987in}{2.062569in}}%
\pgfusepath{stroke}%
\end{pgfscope}%
\begin{pgfscope}%
\pgfpathrectangle{\pgfqpoint{2.302578in}{1.285444in}}{\pgfqpoint{5.425000in}{3.020000in}} %
\pgfusepath{clip}%
\pgfsetbuttcap%
\pgfsetroundjoin%
\pgfsetlinewidth{3.011250pt}%
\definecolor{currentstroke}{rgb}{0.866667,0.517647,0.321569}%
\pgfsetstrokecolor{currentstroke}%
\pgfsetdash{{9.000000pt}{3.000000pt}}{0.000000pt}%
\pgfpathmoveto{\pgfqpoint{2.549169in}{1.287192in}}%
\pgfpathlineto{\pgfqpoint{2.556570in}{1.296704in}}%
\pgfpathlineto{\pgfqpoint{2.559037in}{1.313268in}}%
\pgfpathlineto{\pgfqpoint{2.583709in}{1.345900in}}%
\pgfpathlineto{\pgfqpoint{2.598511in}{1.361542in}}%
\pgfpathlineto{\pgfqpoint{2.605913in}{1.371309in}}%
\pgfpathlineto{\pgfqpoint{2.615781in}{1.385101in}}%
\pgfpathlineto{\pgfqpoint{2.620716in}{1.390675in}}%
\pgfpathlineto{\pgfqpoint{2.625650in}{1.403103in}}%
\pgfpathlineto{\pgfqpoint{2.635519in}{1.413199in}}%
\pgfpathlineto{\pgfqpoint{2.761343in}{1.471322in}}%
\pgfpathlineto{\pgfqpoint{3.143750in}{1.648714in}}%
\pgfpathlineto{\pgfqpoint{3.220231in}{1.684441in}}%
\pgfpathlineto{\pgfqpoint{3.264640in}{1.705039in}}%
\pgfpathlineto{\pgfqpoint{3.484216in}{1.807645in}}%
\pgfpathlineto{\pgfqpoint{3.489150in}{1.810520in}}%
\pgfpathlineto{\pgfqpoint{3.491617in}{1.825411in}}%
\pgfpathlineto{\pgfqpoint{3.676653in}{1.938656in}}%
\pgfpathlineto{\pgfqpoint{3.753134in}{1.983500in}}%
\pgfpathlineto{\pgfqpoint{3.802477in}{2.013689in}}%
\pgfpathlineto{\pgfqpoint{3.901163in}{2.072416in}}%
\pgfpathlineto{\pgfqpoint{3.940637in}{2.095820in}}%
\pgfpathlineto{\pgfqpoint{4.786867in}{2.599721in}}%
\pgfpathlineto{\pgfqpoint{4.823874in}{2.621584in}}%
\pgfpathlineto{\pgfqpoint{5.073056in}{2.770309in}}%
\pgfpathlineto{\pgfqpoint{5.112530in}{2.793455in}}%
\pgfpathlineto{\pgfqpoint{5.191478in}{2.841489in}}%
\pgfpathlineto{\pgfqpoint{5.233420in}{2.866875in}}%
\pgfpathlineto{\pgfqpoint{5.554148in}{3.059480in}}%
\pgfpathlineto{\pgfqpoint{5.625695in}{3.101641in}}%
\pgfpathlineto{\pgfqpoint{5.645433in}{3.113334in}}%
\pgfpathlineto{\pgfqpoint{5.721914in}{3.160262in}}%
\pgfpathlineto{\pgfqpoint{5.776191in}{3.191432in}}%
\pgfpathlineto{\pgfqpoint{5.869943in}{3.246403in}}%
\pgfpathlineto{\pgfqpoint{5.877344in}{3.251392in}}%
\pgfpathlineto{\pgfqpoint{5.904483in}{3.267705in}}%
\pgfpathlineto{\pgfqpoint{5.931621in}{3.283488in}}%
\pgfpathlineto{\pgfqpoint{5.946424in}{3.292240in}}%
\pgfpathlineto{\pgfqpoint{6.030307in}{3.341766in}}%
\pgfpathlineto{\pgfqpoint{6.099387in}{3.382458in}}%
\pgfpathlineto{\pgfqpoint{7.044303in}{3.943813in}}%
\pgfpathlineto{\pgfqpoint{7.071441in}{3.960151in}}%
\pgfpathlineto{\pgfqpoint{7.105981in}{3.977304in}}%
\pgfpathlineto{\pgfqpoint{7.115850in}{3.979916in}}%
\pgfpathlineto{\pgfqpoint{7.170127in}{3.993593in}}%
\pgfpathlineto{\pgfqpoint{7.179995in}{3.996194in}}%
\pgfpathlineto{\pgfqpoint{7.182462in}{4.044557in}}%
\pgfpathlineto{\pgfqpoint{7.207134in}{4.051096in}}%
\pgfpathlineto{\pgfqpoint{7.384768in}{4.094459in}}%
\pgfpathlineto{\pgfqpoint{7.424242in}{4.104204in}}%
\pgfpathlineto{\pgfqpoint{7.480987in}{4.118990in}}%
\pgfpathlineto{\pgfqpoint{7.480987in}{4.118990in}}%
\pgfusepath{stroke}%
\end{pgfscope}%
\begin{pgfscope}%
\pgfsetrectcap%
\pgfsetmiterjoin%
\pgfsetlinewidth{1.003750pt}%
\definecolor{currentstroke}{rgb}{0.800000,0.800000,0.800000}%
\pgfsetstrokecolor{currentstroke}%
\pgfsetdash{}{0pt}%
\pgfpathmoveto{\pgfqpoint{2.302578in}{1.285444in}}%
\pgfpathlineto{\pgfqpoint{2.302578in}{4.305444in}}%
\pgfusepath{stroke}%
\end{pgfscope}%
\begin{pgfscope}%
\pgfsetrectcap%
\pgfsetmiterjoin%
\pgfsetlinewidth{1.003750pt}%
\definecolor{currentstroke}{rgb}{0.800000,0.800000,0.800000}%
\pgfsetstrokecolor{currentstroke}%
\pgfsetdash{}{0pt}%
\pgfpathmoveto{\pgfqpoint{7.727578in}{1.285444in}}%
\pgfpathlineto{\pgfqpoint{7.727578in}{4.305444in}}%
\pgfusepath{stroke}%
\end{pgfscope}%
\begin{pgfscope}%
\pgfsetrectcap%
\pgfsetmiterjoin%
\pgfsetlinewidth{1.003750pt}%
\definecolor{currentstroke}{rgb}{0.800000,0.800000,0.800000}%
\pgfsetstrokecolor{currentstroke}%
\pgfsetdash{}{0pt}%
\pgfpathmoveto{\pgfqpoint{2.302578in}{1.285444in}}%
\pgfpathlineto{\pgfqpoint{7.727578in}{1.285444in}}%
\pgfusepath{stroke}%
\end{pgfscope}%
\begin{pgfscope}%
\pgfsetrectcap%
\pgfsetmiterjoin%
\pgfsetlinewidth{1.003750pt}%
\definecolor{currentstroke}{rgb}{0.800000,0.800000,0.800000}%
\pgfsetstrokecolor{currentstroke}%
\pgfsetdash{}{0pt}%
\pgfpathmoveto{\pgfqpoint{2.302578in}{4.305444in}}%
\pgfpathlineto{\pgfqpoint{7.727578in}{4.305444in}}%
\pgfusepath{stroke}%
\end{pgfscope}%
\begin{pgfscope}%
\pgfsetbuttcap%
\pgfsetroundjoin%
\pgfsetlinewidth{4.015000pt}%
\definecolor{currentstroke}{rgb}{0.298039,0.447059,0.690196}%
\pgfsetstrokecolor{currentstroke}%
\pgfsetdash{{4.000000pt}{0.000000pt}}{0.000000pt}%
\pgfpathmoveto{\pgfqpoint{3.281578in}{4.526044in}}%
\pgfpathlineto{\pgfqpoint{4.031578in}{4.526044in}}%
\pgfusepath{stroke}%
\end{pgfscope}%
\begin{pgfscope}%
\definecolor{textcolor}{rgb}{0.150000,0.150000,0.150000}%
\pgfsetstrokecolor{textcolor}%
\pgfsetfillcolor{textcolor}%
\pgftext[x=4.081578in,y=4.351044in,left,base]{\color{textcolor}\rmfamily\fontsize{36.000000}{43.200000}\selectfont CO}%
\end{pgfscope}%
\begin{pgfscope}%
\pgfsetbuttcap%
\pgfsetroundjoin%
\pgfsetlinewidth{4.015000pt}%
\definecolor{currentstroke}{rgb}{0.866667,0.517647,0.321569}%
\pgfsetstrokecolor{currentstroke}%
\pgfsetdash{{12.000000pt}{4.000000pt}}{0.000000pt}%
\pgfpathmoveto{\pgfqpoint{4.875578in}{4.526044in}}%
\pgfpathlineto{\pgfqpoint{5.625578in}{4.526044in}}%
\pgfusepath{stroke}%
\end{pgfscope}%
\begin{pgfscope}%
\definecolor{textcolor}{rgb}{0.150000,0.150000,0.150000}%
\pgfsetstrokecolor{textcolor}%
\pgfsetfillcolor{textcolor}%
\pgftext[x=5.675578in,y=4.351044in,left,base]{\color{textcolor}\rmfamily\fontsize{36.000000}{43.200000}\selectfont OML}%
\end{pgfscope}%
\end{pgfpicture}%
\makeatother%
\endgroup%
%
}
\caption{Combined run-time}
\end{subfigure}%
\begin{subfigure}[b]{0.5\linewidth}
\centering
 \resizebox{\columnwidth}{!}{%
%% Creator: Matplotlib, PGF backend
%%
%% To include the figure in your LaTeX document, write
%%   \input{<filename>.pgf}
%%
%% Make sure the required packages are loaded in your preamble
%%   \usepackage{pgf}
%%
%% Figures using additional raster images can only be included by \input if
%% they are in the same directory as the main LaTeX file. For loading figures
%% from other directories you can use the `import` package
%%   \usepackage{import}
%% and then include the figures with
%%   \import{<path to file>}{<filename>.pgf}
%%
%% Matplotlib used the following preamble
%%   \usepackage{fontspec}
%%   \setmonofont{Andale Mono}
%%
\begingroup%
\makeatletter%
\begin{pgfpicture}%
\pgfpathrectangle{\pgfpointorigin}{\pgfqpoint{8.039987in}{5.001044in}}%
\pgfusepath{use as bounding box, clip}%
\begin{pgfscope}%
\pgfsetbuttcap%
\pgfsetmiterjoin%
\definecolor{currentfill}{rgb}{1.000000,1.000000,1.000000}%
\pgfsetfillcolor{currentfill}%
\pgfsetlinewidth{0.000000pt}%
\definecolor{currentstroke}{rgb}{1.000000,1.000000,1.000000}%
\pgfsetstrokecolor{currentstroke}%
\pgfsetdash{}{0pt}%
\pgfpathmoveto{\pgfqpoint{0.000000in}{0.000000in}}%
\pgfpathlineto{\pgfqpoint{8.039987in}{0.000000in}}%
\pgfpathlineto{\pgfqpoint{8.039987in}{5.001044in}}%
\pgfpathlineto{\pgfqpoint{0.000000in}{5.001044in}}%
\pgfpathclose%
\pgfusepath{fill}%
\end{pgfscope}%
\begin{pgfscope}%
\pgfsetbuttcap%
\pgfsetmiterjoin%
\definecolor{currentfill}{rgb}{1.000000,1.000000,1.000000}%
\pgfsetfillcolor{currentfill}%
\pgfsetlinewidth{0.000000pt}%
\definecolor{currentstroke}{rgb}{0.000000,0.000000,0.000000}%
\pgfsetstrokecolor{currentstroke}%
\pgfsetstrokeopacity{0.000000}%
\pgfsetdash{}{0pt}%
\pgfpathmoveto{\pgfqpoint{2.302578in}{1.285444in}}%
\pgfpathlineto{\pgfqpoint{7.727578in}{1.285444in}}%
\pgfpathlineto{\pgfqpoint{7.727578in}{4.305444in}}%
\pgfpathlineto{\pgfqpoint{2.302578in}{4.305444in}}%
\pgfpathclose%
\pgfusepath{fill}%
\end{pgfscope}%
\begin{pgfscope}%
\pgfpathrectangle{\pgfqpoint{2.302578in}{1.285444in}}{\pgfqpoint{5.425000in}{3.020000in}} %
\pgfusepath{clip}%
\pgfsetroundcap%
\pgfsetroundjoin%
\pgfsetlinewidth{0.803000pt}%
\definecolor{currentstroke}{rgb}{0.800000,0.800000,0.800000}%
\pgfsetstrokecolor{currentstroke}%
\pgfsetdash{}{0pt}%
\pgfpathmoveto{\pgfqpoint{2.546701in}{1.285444in}}%
\pgfpathlineto{\pgfqpoint{2.546701in}{4.305444in}}%
\pgfusepath{stroke}%
\end{pgfscope}%
\begin{pgfscope}%
\definecolor{textcolor}{rgb}{0.150000,0.150000,0.150000}%
\pgfsetstrokecolor{textcolor}%
\pgfsetfillcolor{textcolor}%
\pgftext[x=2.546701in,y=1.121555in,,top]{\color{textcolor}\rmfamily\fontsize{36.000000}{43.200000}\selectfont 0}%
\end{pgfscope}%
\begin{pgfscope}%
\pgfpathrectangle{\pgfqpoint{2.302578in}{1.285444in}}{\pgfqpoint{5.425000in}{3.020000in}} %
\pgfusepath{clip}%
\pgfsetroundcap%
\pgfsetroundjoin%
\pgfsetlinewidth{0.803000pt}%
\definecolor{currentstroke}{rgb}{0.800000,0.800000,0.800000}%
\pgfsetstrokecolor{currentstroke}%
\pgfsetdash{}{0pt}%
\pgfpathmoveto{\pgfqpoint{5.013844in}{1.285444in}}%
\pgfpathlineto{\pgfqpoint{5.013844in}{4.305444in}}%
\pgfusepath{stroke}%
\end{pgfscope}%
\begin{pgfscope}%
\definecolor{textcolor}{rgb}{0.150000,0.150000,0.150000}%
\pgfsetstrokecolor{textcolor}%
\pgfsetfillcolor{textcolor}%
\pgftext[x=5.013844in,y=1.121555in,,top]{\color{textcolor}\rmfamily\fontsize{36.000000}{43.200000}\selectfont 1000}%
\end{pgfscope}%
\begin{pgfscope}%
\pgfpathrectangle{\pgfqpoint{2.302578in}{1.285444in}}{\pgfqpoint{5.425000in}{3.020000in}} %
\pgfusepath{clip}%
\pgfsetroundcap%
\pgfsetroundjoin%
\pgfsetlinewidth{0.803000pt}%
\definecolor{currentstroke}{rgb}{0.800000,0.800000,0.800000}%
\pgfsetstrokecolor{currentstroke}%
\pgfsetdash{}{0pt}%
\pgfpathmoveto{\pgfqpoint{7.480987in}{1.285444in}}%
\pgfpathlineto{\pgfqpoint{7.480987in}{4.305444in}}%
\pgfusepath{stroke}%
\end{pgfscope}%
\begin{pgfscope}%
\definecolor{textcolor}{rgb}{0.150000,0.150000,0.150000}%
\pgfsetstrokecolor{textcolor}%
\pgfsetfillcolor{textcolor}%
\pgftext[x=7.480987in,y=1.121555in,,top]{\color{textcolor}\rmfamily\fontsize{36.000000}{43.200000}\selectfont 2000}%
\end{pgfscope}%
\begin{pgfscope}%
\definecolor{textcolor}{rgb}{0.150000,0.150000,0.150000}%
\pgfsetstrokecolor{textcolor}%
\pgfsetfillcolor{textcolor}%
\pgftext[x=5.015078in,y=0.621500in,,top]{\color{textcolor}\rmfamily\fontsize{42.000000}{50.400000}\selectfont OpenML Workload}%
\end{pgfscope}%
\begin{pgfscope}%
\pgfpathrectangle{\pgfqpoint{2.302578in}{1.285444in}}{\pgfqpoint{5.425000in}{3.020000in}} %
\pgfusepath{clip}%
\pgfsetroundcap%
\pgfsetroundjoin%
\pgfsetlinewidth{0.803000pt}%
\definecolor{currentstroke}{rgb}{0.800000,0.800000,0.800000}%
\pgfsetstrokecolor{currentstroke}%
\pgfsetdash{}{0pt}%
\pgfpathmoveto{\pgfqpoint{2.302578in}{1.285444in}}%
\pgfpathlineto{\pgfqpoint{7.727578in}{1.285444in}}%
\pgfusepath{stroke}%
\end{pgfscope}%
\begin{pgfscope}%
\definecolor{textcolor}{rgb}{0.150000,0.150000,0.150000}%
\pgfsetstrokecolor{textcolor}%
\pgfsetfillcolor{textcolor}%
\pgftext[x=1.909189in,y=1.111944in,left,base]{\color{textcolor}\rmfamily\fontsize{36.000000}{43.200000}\selectfont 0}%
\end{pgfscope}%
\begin{pgfscope}%
\pgfpathrectangle{\pgfqpoint{2.302578in}{1.285444in}}{\pgfqpoint{5.425000in}{3.020000in}} %
\pgfusepath{clip}%
\pgfsetroundcap%
\pgfsetroundjoin%
\pgfsetlinewidth{0.803000pt}%
\definecolor{currentstroke}{rgb}{0.800000,0.800000,0.800000}%
\pgfsetstrokecolor{currentstroke}%
\pgfsetdash{}{0pt}%
\pgfpathmoveto{\pgfqpoint{2.302578in}{1.806134in}}%
\pgfpathlineto{\pgfqpoint{7.727578in}{1.806134in}}%
\pgfusepath{stroke}%
\end{pgfscope}%
\begin{pgfscope}%
\definecolor{textcolor}{rgb}{0.150000,0.150000,0.150000}%
\pgfsetstrokecolor{textcolor}%
\pgfsetfillcolor{textcolor}%
\pgftext[x=1.450189in,y=1.632634in,left,base]{\color{textcolor}\rmfamily\fontsize{36.000000}{43.200000}\selectfont 500}%
\end{pgfscope}%
\begin{pgfscope}%
\pgfpathrectangle{\pgfqpoint{2.302578in}{1.285444in}}{\pgfqpoint{5.425000in}{3.020000in}} %
\pgfusepath{clip}%
\pgfsetroundcap%
\pgfsetroundjoin%
\pgfsetlinewidth{0.803000pt}%
\definecolor{currentstroke}{rgb}{0.800000,0.800000,0.800000}%
\pgfsetstrokecolor{currentstroke}%
\pgfsetdash{}{0pt}%
\pgfpathmoveto{\pgfqpoint{2.302578in}{2.326823in}}%
\pgfpathlineto{\pgfqpoint{7.727578in}{2.326823in}}%
\pgfusepath{stroke}%
\end{pgfscope}%
\begin{pgfscope}%
\definecolor{textcolor}{rgb}{0.150000,0.150000,0.150000}%
\pgfsetstrokecolor{textcolor}%
\pgfsetfillcolor{textcolor}%
\pgftext[x=1.666689in,y=2.153323in,left,base]{\color{textcolor}\rmfamily\fontsize{36.000000}{43.200000}\selectfont 1k}%
\end{pgfscope}%
\begin{pgfscope}%
\pgfpathrectangle{\pgfqpoint{2.302578in}{1.285444in}}{\pgfqpoint{5.425000in}{3.020000in}} %
\pgfusepath{clip}%
\pgfsetroundcap%
\pgfsetroundjoin%
\pgfsetlinewidth{0.803000pt}%
\definecolor{currentstroke}{rgb}{0.800000,0.800000,0.800000}%
\pgfsetstrokecolor{currentstroke}%
\pgfsetdash{}{0pt}%
\pgfpathmoveto{\pgfqpoint{2.302578in}{2.847513in}}%
\pgfpathlineto{\pgfqpoint{7.727578in}{2.847513in}}%
\pgfusepath{stroke}%
\end{pgfscope}%
\begin{pgfscope}%
\definecolor{textcolor}{rgb}{0.150000,0.150000,0.150000}%
\pgfsetstrokecolor{textcolor}%
\pgfsetfillcolor{textcolor}%
\pgftext[x=1.312189in,y=2.674013in,left,base]{\color{textcolor}\rmfamily\fontsize{36.000000}{43.200000}\selectfont 1.5k}%
\end{pgfscope}%
\begin{pgfscope}%
\pgfpathrectangle{\pgfqpoint{2.302578in}{1.285444in}}{\pgfqpoint{5.425000in}{3.020000in}} %
\pgfusepath{clip}%
\pgfsetroundcap%
\pgfsetroundjoin%
\pgfsetlinewidth{0.803000pt}%
\definecolor{currentstroke}{rgb}{0.800000,0.800000,0.800000}%
\pgfsetstrokecolor{currentstroke}%
\pgfsetdash{}{0pt}%
\pgfpathmoveto{\pgfqpoint{2.302578in}{3.368203in}}%
\pgfpathlineto{\pgfqpoint{7.727578in}{3.368203in}}%
\pgfusepath{stroke}%
\end{pgfscope}%
\begin{pgfscope}%
\definecolor{textcolor}{rgb}{0.150000,0.150000,0.150000}%
\pgfsetstrokecolor{textcolor}%
\pgfsetfillcolor{textcolor}%
\pgftext[x=1.666689in,y=3.194703in,left,base]{\color{textcolor}\rmfamily\fontsize{36.000000}{43.200000}\selectfont 2k}%
\end{pgfscope}%
\begin{pgfscope}%
\pgfpathrectangle{\pgfqpoint{2.302578in}{1.285444in}}{\pgfqpoint{5.425000in}{3.020000in}} %
\pgfusepath{clip}%
\pgfsetroundcap%
\pgfsetroundjoin%
\pgfsetlinewidth{0.803000pt}%
\definecolor{currentstroke}{rgb}{0.800000,0.800000,0.800000}%
\pgfsetstrokecolor{currentstroke}%
\pgfsetdash{}{0pt}%
\pgfpathmoveto{\pgfqpoint{2.302578in}{3.888892in}}%
\pgfpathlineto{\pgfqpoint{7.727578in}{3.888892in}}%
\pgfusepath{stroke}%
\end{pgfscope}%
\begin{pgfscope}%
\definecolor{textcolor}{rgb}{0.150000,0.150000,0.150000}%
\pgfsetstrokecolor{textcolor}%
\pgfsetfillcolor{textcolor}%
\pgftext[x=1.312189in,y=3.715392in,left,base]{\color{textcolor}\rmfamily\fontsize{36.000000}{43.200000}\selectfont 2.5k}%
\end{pgfscope}%
\begin{pgfscope}%
\definecolor{textcolor}{rgb}{0.150000,0.150000,0.150000}%
\pgfsetstrokecolor{textcolor}%
\pgfsetfillcolor{textcolor}%
\pgftext[x=0.472750in,y=1.361611in,left,base,rotate=90.000000]{\color{textcolor}\rmfamily\fontsize{42.000000}{50.400000}\selectfont Cumulative }%
\end{pgfscope}%
\begin{pgfscope}%
\definecolor{textcolor}{rgb}{0.150000,0.150000,0.150000}%
\pgfsetstrokecolor{textcolor}%
\pgfsetfillcolor{textcolor}%
\pgftext[x=1.110800in,y=1.207902in,left,base,rotate=90.000000]{\color{textcolor}\rmfamily\fontsize{42.000000}{50.400000}\selectfont Run Time (s)}%
\end{pgfscope}%
\begin{pgfscope}%
\pgfpathrectangle{\pgfqpoint{2.302578in}{1.285444in}}{\pgfqpoint{5.425000in}{3.020000in}} %
\pgfusepath{clip}%
\pgfsetbuttcap%
\pgfsetroundjoin%
\pgfsetlinewidth{3.011250pt}%
\definecolor{currentstroke}{rgb}{0.298039,0.447059,0.690196}%
\pgfsetstrokecolor{currentstroke}%
\pgfsetdash{{3.000000pt}{0.000000pt}}{0.000000pt}%
\pgfpathmoveto{\pgfqpoint{2.549169in}{1.287083in}}%
\pgfpathlineto{\pgfqpoint{2.677460in}{1.289529in}}%
\pgfpathlineto{\pgfqpoint{7.480987in}{1.353997in}}%
\pgfpathlineto{\pgfqpoint{7.480987in}{1.353997in}}%
\pgfusepath{stroke}%
\end{pgfscope}%
\begin{pgfscope}%
\pgfpathrectangle{\pgfqpoint{2.302578in}{1.285444in}}{\pgfqpoint{5.425000in}{3.020000in}} %
\pgfusepath{clip}%
\pgfsetbuttcap%
\pgfsetroundjoin%
\pgfsetlinewidth{3.011250pt}%
\definecolor{currentstroke}{rgb}{0.866667,0.517647,0.321569}%
\pgfsetstrokecolor{currentstroke}%
\pgfsetdash{{9.000000pt}{3.000000pt}}{0.000000pt}%
\pgfpathmoveto{\pgfqpoint{2.549169in}{1.287192in}}%
\pgfpathlineto{\pgfqpoint{2.600979in}{1.319141in}}%
\pgfpathlineto{\pgfqpoint{4.796736in}{2.302386in}}%
\pgfpathlineto{\pgfqpoint{5.124866in}{2.448575in}}%
\pgfpathlineto{\pgfqpoint{6.059913in}{2.865653in}}%
\pgfpathlineto{\pgfqpoint{6.200540in}{2.928620in}}%
\pgfpathlineto{\pgfqpoint{6.242481in}{2.947426in}}%
\pgfpathlineto{\pgfqpoint{6.370773in}{3.004253in}}%
\pgfpathlineto{\pgfqpoint{6.861734in}{3.223840in}}%
\pgfpathlineto{\pgfqpoint{7.017164in}{3.293342in}}%
\pgfpathlineto{\pgfqpoint{7.098580in}{3.328976in}}%
\pgfpathlineto{\pgfqpoint{7.480987in}{3.370665in}}%
\pgfpathlineto{\pgfqpoint{7.480987in}{3.370665in}}%
\pgfusepath{stroke}%
\end{pgfscope}%
\begin{pgfscope}%
\pgfsetrectcap%
\pgfsetmiterjoin%
\pgfsetlinewidth{1.003750pt}%
\definecolor{currentstroke}{rgb}{0.800000,0.800000,0.800000}%
\pgfsetstrokecolor{currentstroke}%
\pgfsetdash{}{0pt}%
\pgfpathmoveto{\pgfqpoint{2.302578in}{1.285444in}}%
\pgfpathlineto{\pgfqpoint{2.302578in}{4.305444in}}%
\pgfusepath{stroke}%
\end{pgfscope}%
\begin{pgfscope}%
\pgfsetrectcap%
\pgfsetmiterjoin%
\pgfsetlinewidth{1.003750pt}%
\definecolor{currentstroke}{rgb}{0.800000,0.800000,0.800000}%
\pgfsetstrokecolor{currentstroke}%
\pgfsetdash{}{0pt}%
\pgfpathmoveto{\pgfqpoint{7.727578in}{1.285444in}}%
\pgfpathlineto{\pgfqpoint{7.727578in}{4.305444in}}%
\pgfusepath{stroke}%
\end{pgfscope}%
\begin{pgfscope}%
\pgfsetrectcap%
\pgfsetmiterjoin%
\pgfsetlinewidth{1.003750pt}%
\definecolor{currentstroke}{rgb}{0.800000,0.800000,0.800000}%
\pgfsetstrokecolor{currentstroke}%
\pgfsetdash{}{0pt}%
\pgfpathmoveto{\pgfqpoint{2.302578in}{1.285444in}}%
\pgfpathlineto{\pgfqpoint{7.727578in}{1.285444in}}%
\pgfusepath{stroke}%
\end{pgfscope}%
\begin{pgfscope}%
\pgfsetrectcap%
\pgfsetmiterjoin%
\pgfsetlinewidth{1.003750pt}%
\definecolor{currentstroke}{rgb}{0.800000,0.800000,0.800000}%
\pgfsetstrokecolor{currentstroke}%
\pgfsetdash{}{0pt}%
\pgfpathmoveto{\pgfqpoint{2.302578in}{4.305444in}}%
\pgfpathlineto{\pgfqpoint{7.727578in}{4.305444in}}%
\pgfusepath{stroke}%
\end{pgfscope}%
\begin{pgfscope}%
\pgfsetbuttcap%
\pgfsetroundjoin%
\pgfsetlinewidth{4.015000pt}%
\definecolor{currentstroke}{rgb}{0.298039,0.447059,0.690196}%
\pgfsetstrokecolor{currentstroke}%
\pgfsetdash{{4.000000pt}{0.000000pt}}{0.000000pt}%
\pgfpathmoveto{\pgfqpoint{3.281578in}{4.526044in}}%
\pgfpathlineto{\pgfqpoint{4.031578in}{4.526044in}}%
\pgfusepath{stroke}%
\end{pgfscope}%
\begin{pgfscope}%
\definecolor{textcolor}{rgb}{0.150000,0.150000,0.150000}%
\pgfsetstrokecolor{textcolor}%
\pgfsetfillcolor{textcolor}%
\pgftext[x=4.081578in,y=4.351044in,left,base]{\color{textcolor}\rmfamily\fontsize{36.000000}{43.200000}\selectfont CO}%
\end{pgfscope}%
\begin{pgfscope}%
\pgfsetbuttcap%
\pgfsetroundjoin%
\pgfsetlinewidth{4.015000pt}%
\definecolor{currentstroke}{rgb}{0.866667,0.517647,0.321569}%
\pgfsetstrokecolor{currentstroke}%
\pgfsetdash{{12.000000pt}{4.000000pt}}{0.000000pt}%
\pgfpathmoveto{\pgfqpoint{4.875578in}{4.526044in}}%
\pgfpathlineto{\pgfqpoint{5.625578in}{4.526044in}}%
\pgfusepath{stroke}%
\end{pgfscope}%
\begin{pgfscope}%
\definecolor{textcolor}{rgb}{0.150000,0.150000,0.150000}%
\pgfsetstrokecolor{textcolor}%
\pgfsetfillcolor{textcolor}%
\pgftext[x=5.675578in,y=4.351044in,left,base]{\color{textcolor}\rmfamily\fontsize{36.000000}{43.200000}\selectfont OML}%
\end{pgfscope}%
\end{pgfpicture}%
\makeatother%
\endgroup%
%
}
\caption{Best pipeline run-time}
\end{subfigure}
\caption{Effect of model materialization for OpenML workloads (CO = Collaborative Optimizer, OML = OpenML Default)}
\label{exp-model-materialization}
\end{figure}

\subsection{Reuse}
In this section, we compare our reuse algorithm we 2 baselines.
\begin{figure}
\begin{subfigure}[b]{0.5\linewidth}
\centering
 \resizebox{\columnwidth}{!}{%
%% Creator: Matplotlib, PGF backend
%%
%% To include the figure in your LaTeX document, write
%%   \input{<filename>.pgf}
%%
%% Make sure the required packages are loaded in your preamble
%%   \usepackage{pgf}
%%
%% Figures using additional raster images can only be included by \input if
%% they are in the same directory as the main LaTeX file. For loading figures
%% from other directories you can use the `import` package
%%   \usepackage{import}
%% and then include the figures with
%%   \import{<path to file>}{<filename>.pgf}
%%
%% Matplotlib used the following preamble
%%   \usepackage{fontspec}
%%   \setmonofont{Andale Mono}
%%
\begingroup%
\makeatletter%
\begin{pgfpicture}%
\pgfpathrectangle{\pgfpointorigin}{\pgfqpoint{8.287291in}{5.023028in}}%
\pgfusepath{use as bounding box, clip}%
\begin{pgfscope}%
\pgfsetbuttcap%
\pgfsetmiterjoin%
\definecolor{currentfill}{rgb}{1.000000,1.000000,1.000000}%
\pgfsetfillcolor{currentfill}%
\pgfsetlinewidth{0.000000pt}%
\definecolor{currentstroke}{rgb}{1.000000,1.000000,1.000000}%
\pgfsetstrokecolor{currentstroke}%
\pgfsetdash{}{0pt}%
\pgfpathmoveto{\pgfqpoint{0.000000in}{0.000000in}}%
\pgfpathlineto{\pgfqpoint{8.287291in}{0.000000in}}%
\pgfpathlineto{\pgfqpoint{8.287291in}{5.023028in}}%
\pgfpathlineto{\pgfqpoint{0.000000in}{5.023028in}}%
\pgfpathclose%
\pgfusepath{fill}%
\end{pgfscope}%
\begin{pgfscope}%
\pgfsetbuttcap%
\pgfsetmiterjoin%
\definecolor{currentfill}{rgb}{1.000000,1.000000,1.000000}%
\pgfsetfillcolor{currentfill}%
\pgfsetlinewidth{0.000000pt}%
\definecolor{currentstroke}{rgb}{0.000000,0.000000,0.000000}%
\pgfsetstrokecolor{currentstroke}%
\pgfsetstrokeopacity{0.000000}%
\pgfsetdash{}{0pt}%
\pgfpathmoveto{\pgfqpoint{2.201583in}{1.233139in}}%
\pgfpathlineto{\pgfqpoint{7.626583in}{1.233139in}}%
\pgfpathlineto{\pgfqpoint{7.626583in}{4.253139in}}%
\pgfpathlineto{\pgfqpoint{2.201583in}{4.253139in}}%
\pgfpathclose%
\pgfusepath{fill}%
\end{pgfscope}%
\begin{pgfscope}%
\pgfpathrectangle{\pgfqpoint{2.201583in}{1.233139in}}{\pgfqpoint{5.425000in}{3.020000in}} %
\pgfusepath{clip}%
\pgfsetroundcap%
\pgfsetroundjoin%
\pgfsetlinewidth{0.803000pt}%
\definecolor{currentstroke}{rgb}{0.800000,0.800000,0.800000}%
\pgfsetstrokecolor{currentstroke}%
\pgfsetdash{}{0pt}%
\pgfpathmoveto{\pgfqpoint{2.448174in}{1.233139in}}%
\pgfpathlineto{\pgfqpoint{2.448174in}{4.253139in}}%
\pgfusepath{stroke}%
\end{pgfscope}%
\begin{pgfscope}%
\definecolor{textcolor}{rgb}{0.150000,0.150000,0.150000}%
\pgfsetstrokecolor{textcolor}%
\pgfsetfillcolor{textcolor}%
\pgftext[x=2.448174in,y=1.069250in,,top]{\color{textcolor}\rmfamily\fontsize{34.000000}{40.800000}\selectfont 1}%
\end{pgfscope}%
\begin{pgfscope}%
\pgfpathrectangle{\pgfqpoint{2.201583in}{1.233139in}}{\pgfqpoint{5.425000in}{3.020000in}} %
\pgfusepath{clip}%
\pgfsetroundcap%
\pgfsetroundjoin%
\pgfsetlinewidth{0.803000pt}%
\definecolor{currentstroke}{rgb}{0.800000,0.800000,0.800000}%
\pgfsetstrokecolor{currentstroke}%
\pgfsetdash{}{0pt}%
\pgfpathmoveto{\pgfqpoint{3.152719in}{1.233139in}}%
\pgfpathlineto{\pgfqpoint{3.152719in}{4.253139in}}%
\pgfusepath{stroke}%
\end{pgfscope}%
\begin{pgfscope}%
\definecolor{textcolor}{rgb}{0.150000,0.150000,0.150000}%
\pgfsetstrokecolor{textcolor}%
\pgfsetfillcolor{textcolor}%
\pgftext[x=3.152719in,y=1.069250in,,top]{\color{textcolor}\rmfamily\fontsize{34.000000}{40.800000}\selectfont 2}%
\end{pgfscope}%
\begin{pgfscope}%
\pgfpathrectangle{\pgfqpoint{2.201583in}{1.233139in}}{\pgfqpoint{5.425000in}{3.020000in}} %
\pgfusepath{clip}%
\pgfsetroundcap%
\pgfsetroundjoin%
\pgfsetlinewidth{0.803000pt}%
\definecolor{currentstroke}{rgb}{0.800000,0.800000,0.800000}%
\pgfsetstrokecolor{currentstroke}%
\pgfsetdash{}{0pt}%
\pgfpathmoveto{\pgfqpoint{3.857265in}{1.233139in}}%
\pgfpathlineto{\pgfqpoint{3.857265in}{4.253139in}}%
\pgfusepath{stroke}%
\end{pgfscope}%
\begin{pgfscope}%
\definecolor{textcolor}{rgb}{0.150000,0.150000,0.150000}%
\pgfsetstrokecolor{textcolor}%
\pgfsetfillcolor{textcolor}%
\pgftext[x=3.857265in,y=1.069250in,,top]{\color{textcolor}\rmfamily\fontsize{34.000000}{40.800000}\selectfont 3}%
\end{pgfscope}%
\begin{pgfscope}%
\pgfpathrectangle{\pgfqpoint{2.201583in}{1.233139in}}{\pgfqpoint{5.425000in}{3.020000in}} %
\pgfusepath{clip}%
\pgfsetroundcap%
\pgfsetroundjoin%
\pgfsetlinewidth{0.803000pt}%
\definecolor{currentstroke}{rgb}{0.800000,0.800000,0.800000}%
\pgfsetstrokecolor{currentstroke}%
\pgfsetdash{}{0pt}%
\pgfpathmoveto{\pgfqpoint{4.561810in}{1.233139in}}%
\pgfpathlineto{\pgfqpoint{4.561810in}{4.253139in}}%
\pgfusepath{stroke}%
\end{pgfscope}%
\begin{pgfscope}%
\definecolor{textcolor}{rgb}{0.150000,0.150000,0.150000}%
\pgfsetstrokecolor{textcolor}%
\pgfsetfillcolor{textcolor}%
\pgftext[x=4.561810in,y=1.069250in,,top]{\color{textcolor}\rmfamily\fontsize{34.000000}{40.800000}\selectfont 4}%
\end{pgfscope}%
\begin{pgfscope}%
\pgfpathrectangle{\pgfqpoint{2.201583in}{1.233139in}}{\pgfqpoint{5.425000in}{3.020000in}} %
\pgfusepath{clip}%
\pgfsetroundcap%
\pgfsetroundjoin%
\pgfsetlinewidth{0.803000pt}%
\definecolor{currentstroke}{rgb}{0.800000,0.800000,0.800000}%
\pgfsetstrokecolor{currentstroke}%
\pgfsetdash{}{0pt}%
\pgfpathmoveto{\pgfqpoint{5.266356in}{1.233139in}}%
\pgfpathlineto{\pgfqpoint{5.266356in}{4.253139in}}%
\pgfusepath{stroke}%
\end{pgfscope}%
\begin{pgfscope}%
\definecolor{textcolor}{rgb}{0.150000,0.150000,0.150000}%
\pgfsetstrokecolor{textcolor}%
\pgfsetfillcolor{textcolor}%
\pgftext[x=5.266356in,y=1.069250in,,top]{\color{textcolor}\rmfamily\fontsize{34.000000}{40.800000}\selectfont 5}%
\end{pgfscope}%
\begin{pgfscope}%
\pgfpathrectangle{\pgfqpoint{2.201583in}{1.233139in}}{\pgfqpoint{5.425000in}{3.020000in}} %
\pgfusepath{clip}%
\pgfsetroundcap%
\pgfsetroundjoin%
\pgfsetlinewidth{0.803000pt}%
\definecolor{currentstroke}{rgb}{0.800000,0.800000,0.800000}%
\pgfsetstrokecolor{currentstroke}%
\pgfsetdash{}{0pt}%
\pgfpathmoveto{\pgfqpoint{5.970901in}{1.233139in}}%
\pgfpathlineto{\pgfqpoint{5.970901in}{4.253139in}}%
\pgfusepath{stroke}%
\end{pgfscope}%
\begin{pgfscope}%
\definecolor{textcolor}{rgb}{0.150000,0.150000,0.150000}%
\pgfsetstrokecolor{textcolor}%
\pgfsetfillcolor{textcolor}%
\pgftext[x=5.970901in,y=1.069250in,,top]{\color{textcolor}\rmfamily\fontsize{34.000000}{40.800000}\selectfont 6}%
\end{pgfscope}%
\begin{pgfscope}%
\pgfpathrectangle{\pgfqpoint{2.201583in}{1.233139in}}{\pgfqpoint{5.425000in}{3.020000in}} %
\pgfusepath{clip}%
\pgfsetroundcap%
\pgfsetroundjoin%
\pgfsetlinewidth{0.803000pt}%
\definecolor{currentstroke}{rgb}{0.800000,0.800000,0.800000}%
\pgfsetstrokecolor{currentstroke}%
\pgfsetdash{}{0pt}%
\pgfpathmoveto{\pgfqpoint{6.675447in}{1.233139in}}%
\pgfpathlineto{\pgfqpoint{6.675447in}{4.253139in}}%
\pgfusepath{stroke}%
\end{pgfscope}%
\begin{pgfscope}%
\definecolor{textcolor}{rgb}{0.150000,0.150000,0.150000}%
\pgfsetstrokecolor{textcolor}%
\pgfsetfillcolor{textcolor}%
\pgftext[x=6.675447in,y=1.069250in,,top]{\color{textcolor}\rmfamily\fontsize{34.000000}{40.800000}\selectfont 7}%
\end{pgfscope}%
\begin{pgfscope}%
\pgfpathrectangle{\pgfqpoint{2.201583in}{1.233139in}}{\pgfqpoint{5.425000in}{3.020000in}} %
\pgfusepath{clip}%
\pgfsetroundcap%
\pgfsetroundjoin%
\pgfsetlinewidth{0.803000pt}%
\definecolor{currentstroke}{rgb}{0.800000,0.800000,0.800000}%
\pgfsetstrokecolor{currentstroke}%
\pgfsetdash{}{0pt}%
\pgfpathmoveto{\pgfqpoint{7.379992in}{1.233139in}}%
\pgfpathlineto{\pgfqpoint{7.379992in}{4.253139in}}%
\pgfusepath{stroke}%
\end{pgfscope}%
\begin{pgfscope}%
\definecolor{textcolor}{rgb}{0.150000,0.150000,0.150000}%
\pgfsetstrokecolor{textcolor}%
\pgfsetfillcolor{textcolor}%
\pgftext[x=7.379992in,y=1.069250in,,top]{\color{textcolor}\rmfamily\fontsize{34.000000}{40.800000}\selectfont 8}%
\end{pgfscope}%
\begin{pgfscope}%
\definecolor{textcolor}{rgb}{0.150000,0.150000,0.150000}%
\pgfsetstrokecolor{textcolor}%
\pgfsetfillcolor{textcolor}%
\pgftext[x=4.914083in,y=0.593889in,,top]{\color{textcolor}\rmfamily\fontsize{40.000000}{48.000000}\selectfont Workload}%
\end{pgfscope}%
\begin{pgfscope}%
\pgfpathrectangle{\pgfqpoint{2.201583in}{1.233139in}}{\pgfqpoint{5.425000in}{3.020000in}} %
\pgfusepath{clip}%
\pgfsetroundcap%
\pgfsetroundjoin%
\pgfsetlinewidth{0.803000pt}%
\definecolor{currentstroke}{rgb}{0.800000,0.800000,0.800000}%
\pgfsetstrokecolor{currentstroke}%
\pgfsetdash{}{0pt}%
\pgfpathmoveto{\pgfqpoint{2.201583in}{1.233139in}}%
\pgfpathlineto{\pgfqpoint{7.626583in}{1.233139in}}%
\pgfusepath{stroke}%
\end{pgfscope}%
\begin{pgfscope}%
\definecolor{textcolor}{rgb}{0.150000,0.150000,0.150000}%
\pgfsetstrokecolor{textcolor}%
\pgfsetfillcolor{textcolor}%
\pgftext[x=1.820944in,y=1.069278in,left,base]{\color{textcolor}\rmfamily\fontsize{34.000000}{40.800000}\selectfont 0}%
\end{pgfscope}%
\begin{pgfscope}%
\pgfpathrectangle{\pgfqpoint{2.201583in}{1.233139in}}{\pgfqpoint{5.425000in}{3.020000in}} %
\pgfusepath{clip}%
\pgfsetroundcap%
\pgfsetroundjoin%
\pgfsetlinewidth{0.803000pt}%
\definecolor{currentstroke}{rgb}{0.800000,0.800000,0.800000}%
\pgfsetstrokecolor{currentstroke}%
\pgfsetdash{}{0pt}%
\pgfpathmoveto{\pgfqpoint{2.201583in}{1.919502in}}%
\pgfpathlineto{\pgfqpoint{7.626583in}{1.919502in}}%
\pgfusepath{stroke}%
\end{pgfscope}%
\begin{pgfscope}%
\definecolor{textcolor}{rgb}{0.150000,0.150000,0.150000}%
\pgfsetstrokecolor{textcolor}%
\pgfsetfillcolor{textcolor}%
\pgftext[x=1.387444in,y=1.755641in,left,base]{\color{textcolor}\rmfamily\fontsize{34.000000}{40.800000}\selectfont 500}%
\end{pgfscope}%
\begin{pgfscope}%
\pgfpathrectangle{\pgfqpoint{2.201583in}{1.233139in}}{\pgfqpoint{5.425000in}{3.020000in}} %
\pgfusepath{clip}%
\pgfsetroundcap%
\pgfsetroundjoin%
\pgfsetlinewidth{0.803000pt}%
\definecolor{currentstroke}{rgb}{0.800000,0.800000,0.800000}%
\pgfsetstrokecolor{currentstroke}%
\pgfsetdash{}{0pt}%
\pgfpathmoveto{\pgfqpoint{2.201583in}{2.605866in}}%
\pgfpathlineto{\pgfqpoint{7.626583in}{2.605866in}}%
\pgfusepath{stroke}%
\end{pgfscope}%
\begin{pgfscope}%
\definecolor{textcolor}{rgb}{0.150000,0.150000,0.150000}%
\pgfsetstrokecolor{textcolor}%
\pgfsetfillcolor{textcolor}%
\pgftext[x=1.591916in,y=2.442005in,left,base]{\color{textcolor}\rmfamily\fontsize{34.000000}{40.800000}\selectfont 1k}%
\end{pgfscope}%
\begin{pgfscope}%
\pgfpathrectangle{\pgfqpoint{2.201583in}{1.233139in}}{\pgfqpoint{5.425000in}{3.020000in}} %
\pgfusepath{clip}%
\pgfsetroundcap%
\pgfsetroundjoin%
\pgfsetlinewidth{0.803000pt}%
\definecolor{currentstroke}{rgb}{0.800000,0.800000,0.800000}%
\pgfsetstrokecolor{currentstroke}%
\pgfsetdash{}{0pt}%
\pgfpathmoveto{\pgfqpoint{2.201583in}{3.292230in}}%
\pgfpathlineto{\pgfqpoint{7.626583in}{3.292230in}}%
\pgfusepath{stroke}%
\end{pgfscope}%
\begin{pgfscope}%
\definecolor{textcolor}{rgb}{0.150000,0.150000,0.150000}%
\pgfsetstrokecolor{textcolor}%
\pgfsetfillcolor{textcolor}%
\pgftext[x=1.257111in,y=3.128368in,left,base]{\color{textcolor}\rmfamily\fontsize{34.000000}{40.800000}\selectfont 1.5k}%
\end{pgfscope}%
\begin{pgfscope}%
\pgfpathrectangle{\pgfqpoint{2.201583in}{1.233139in}}{\pgfqpoint{5.425000in}{3.020000in}} %
\pgfusepath{clip}%
\pgfsetroundcap%
\pgfsetroundjoin%
\pgfsetlinewidth{0.803000pt}%
\definecolor{currentstroke}{rgb}{0.800000,0.800000,0.800000}%
\pgfsetstrokecolor{currentstroke}%
\pgfsetdash{}{0pt}%
\pgfpathmoveto{\pgfqpoint{2.201583in}{3.978593in}}%
\pgfpathlineto{\pgfqpoint{7.626583in}{3.978593in}}%
\pgfusepath{stroke}%
\end{pgfscope}%
\begin{pgfscope}%
\definecolor{textcolor}{rgb}{0.150000,0.150000,0.150000}%
\pgfsetstrokecolor{textcolor}%
\pgfsetfillcolor{textcolor}%
\pgftext[x=1.591916in,y=3.814732in,left,base]{\color{textcolor}\rmfamily\fontsize{34.000000}{40.800000}\selectfont 2k}%
\end{pgfscope}%
\begin{pgfscope}%
\definecolor{textcolor}{rgb}{0.150000,0.150000,0.150000}%
\pgfsetstrokecolor{textcolor}%
\pgfsetfillcolor{textcolor}%
\pgftext[x=0.455000in,y=1.377583in,left,base,rotate=90.000000]{\color{textcolor}\rmfamily\fontsize{40.000000}{48.000000}\selectfont Cumulative }%
\end{pgfscope}%
\begin{pgfscope}%
\definecolor{textcolor}{rgb}{0.150000,0.150000,0.150000}%
\pgfsetstrokecolor{textcolor}%
\pgfsetfillcolor{textcolor}%
\pgftext[x=1.062667in,y=1.231194in,left,base,rotate=90.000000]{\color{textcolor}\rmfamily\fontsize{40.000000}{48.000000}\selectfont Run Time (s)}%
\end{pgfscope}%
\begin{pgfscope}%
\pgfpathrectangle{\pgfqpoint{2.201583in}{1.233139in}}{\pgfqpoint{5.425000in}{3.020000in}} %
\pgfusepath{clip}%
\pgfsetbuttcap%
\pgfsetroundjoin%
\definecolor{currentfill}{rgb}{0.298039,0.447059,0.690196}%
\pgfsetfillcolor{currentfill}%
\pgfsetfillopacity{0.200000}%
\pgfsetlinewidth{0.803000pt}%
\definecolor{currentstroke}{rgb}{0.298039,0.447059,0.690196}%
\pgfsetstrokecolor{currentstroke}%
\pgfsetstrokeopacity{0.200000}%
\pgfsetdash{}{0pt}%
\pgfpathmoveto{\pgfqpoint{2.448174in}{1.508497in}}%
\pgfpathlineto{\pgfqpoint{2.448174in}{1.507370in}}%
\pgfpathlineto{\pgfqpoint{3.152719in}{1.742423in}}%
\pgfpathlineto{\pgfqpoint{3.857265in}{2.148105in}}%
\pgfpathlineto{\pgfqpoint{4.561810in}{2.316058in}}%
\pgfpathlineto{\pgfqpoint{5.266356in}{2.445289in}}%
\pgfpathlineto{\pgfqpoint{5.970901in}{2.648629in}}%
\pgfpathlineto{\pgfqpoint{6.675447in}{2.846294in}}%
\pgfpathlineto{\pgfqpoint{7.379992in}{3.167960in}}%
\pgfpathlineto{\pgfqpoint{7.379992in}{3.299033in}}%
\pgfpathlineto{\pgfqpoint{7.379992in}{3.299033in}}%
\pgfpathlineto{\pgfqpoint{6.675447in}{2.976185in}}%
\pgfpathlineto{\pgfqpoint{5.970901in}{2.667834in}}%
\pgfpathlineto{\pgfqpoint{5.266356in}{2.456845in}}%
\pgfpathlineto{\pgfqpoint{4.561810in}{2.319752in}}%
\pgfpathlineto{\pgfqpoint{3.857265in}{2.153598in}}%
\pgfpathlineto{\pgfqpoint{3.152719in}{1.744247in}}%
\pgfpathlineto{\pgfqpoint{2.448174in}{1.508497in}}%
\pgfpathclose%
\pgfusepath{stroke,fill}%
\end{pgfscope}%
\begin{pgfscope}%
\pgfpathrectangle{\pgfqpoint{2.201583in}{1.233139in}}{\pgfqpoint{5.425000in}{3.020000in}} %
\pgfusepath{clip}%
\pgfsetbuttcap%
\pgfsetroundjoin%
\definecolor{currentfill}{rgb}{0.866667,0.517647,0.321569}%
\pgfsetfillcolor{currentfill}%
\pgfsetfillopacity{0.200000}%
\pgfsetlinewidth{0.803000pt}%
\definecolor{currentstroke}{rgb}{0.866667,0.517647,0.321569}%
\pgfsetstrokecolor{currentstroke}%
\pgfsetstrokeopacity{0.200000}%
\pgfsetdash{}{0pt}%
\pgfpathmoveto{\pgfqpoint{2.448174in}{1.508011in}}%
\pgfpathlineto{\pgfqpoint{2.448174in}{1.507333in}}%
\pgfpathlineto{\pgfqpoint{3.152719in}{1.741972in}}%
\pgfpathlineto{\pgfqpoint{3.857265in}{2.140387in}}%
\pgfpathlineto{\pgfqpoint{4.561810in}{2.332333in}}%
\pgfpathlineto{\pgfqpoint{5.266356in}{2.462460in}}%
\pgfpathlineto{\pgfqpoint{5.970901in}{2.672417in}}%
\pgfpathlineto{\pgfqpoint{6.675447in}{3.229427in}}%
\pgfpathlineto{\pgfqpoint{7.379992in}{3.564088in}}%
\pgfpathlineto{\pgfqpoint{7.379992in}{3.682373in}}%
\pgfpathlineto{\pgfqpoint{7.379992in}{3.682373in}}%
\pgfpathlineto{\pgfqpoint{6.675447in}{3.347229in}}%
\pgfpathlineto{\pgfqpoint{5.970901in}{2.712367in}}%
\pgfpathlineto{\pgfqpoint{5.266356in}{2.499810in}}%
\pgfpathlineto{\pgfqpoint{4.561810in}{2.357934in}}%
\pgfpathlineto{\pgfqpoint{3.857265in}{2.149377in}}%
\pgfpathlineto{\pgfqpoint{3.152719in}{1.744075in}}%
\pgfpathlineto{\pgfqpoint{2.448174in}{1.508011in}}%
\pgfpathclose%
\pgfusepath{stroke,fill}%
\end{pgfscope}%
\begin{pgfscope}%
\pgfpathrectangle{\pgfqpoint{2.201583in}{1.233139in}}{\pgfqpoint{5.425000in}{3.020000in}} %
\pgfusepath{clip}%
\pgfsetbuttcap%
\pgfsetroundjoin%
\definecolor{currentfill}{rgb}{0.333333,0.658824,0.407843}%
\pgfsetfillcolor{currentfill}%
\pgfsetfillopacity{0.200000}%
\pgfsetlinewidth{0.803000pt}%
\definecolor{currentstroke}{rgb}{0.333333,0.658824,0.407843}%
\pgfsetstrokecolor{currentstroke}%
\pgfsetstrokeopacity{0.200000}%
\pgfsetdash{}{0pt}%
\pgfpathmoveto{\pgfqpoint{2.448174in}{1.509714in}}%
\pgfpathlineto{\pgfqpoint{2.448174in}{1.508531in}}%
\pgfpathlineto{\pgfqpoint{3.152719in}{1.744177in}}%
\pgfpathlineto{\pgfqpoint{3.857265in}{2.142255in}}%
\pgfpathlineto{\pgfqpoint{4.561810in}{2.361249in}}%
\pgfpathlineto{\pgfqpoint{5.266356in}{2.509094in}}%
\pgfpathlineto{\pgfqpoint{5.970901in}{2.719127in}}%
\pgfpathlineto{\pgfqpoint{6.675447in}{3.672070in}}%
\pgfpathlineto{\pgfqpoint{7.379992in}{4.013839in}}%
\pgfpathlineto{\pgfqpoint{7.379992in}{4.128747in}}%
\pgfpathlineto{\pgfqpoint{7.379992in}{4.128747in}}%
\pgfpathlineto{\pgfqpoint{6.675447in}{3.826368in}}%
\pgfpathlineto{\pgfqpoint{5.970901in}{2.742087in}}%
\pgfpathlineto{\pgfqpoint{5.266356in}{2.527887in}}%
\pgfpathlineto{\pgfqpoint{4.561810in}{2.382329in}}%
\pgfpathlineto{\pgfqpoint{3.857265in}{2.161396in}}%
\pgfpathlineto{\pgfqpoint{3.152719in}{1.745536in}}%
\pgfpathlineto{\pgfqpoint{2.448174in}{1.509714in}}%
\pgfpathclose%
\pgfusepath{stroke,fill}%
\end{pgfscope}%
\begin{pgfscope}%
\pgfpathrectangle{\pgfqpoint{2.201583in}{1.233139in}}{\pgfqpoint{5.425000in}{3.020000in}} %
\pgfusepath{clip}%
\pgfsetbuttcap%
\pgfsetroundjoin%
\pgfsetlinewidth{3.011250pt}%
\definecolor{currentstroke}{rgb}{0.298039,0.447059,0.690196}%
\pgfsetstrokecolor{currentstroke}%
\pgfsetdash{{3.000000pt}{0.000000pt}}{0.000000pt}%
\pgfpathmoveto{\pgfqpoint{2.448174in}{1.507971in}}%
\pgfpathlineto{\pgfqpoint{3.152719in}{1.743500in}}%
\pgfpathlineto{\pgfqpoint{3.857265in}{2.151684in}}%
\pgfpathlineto{\pgfqpoint{4.561810in}{2.317605in}}%
\pgfpathlineto{\pgfqpoint{5.266356in}{2.450822in}}%
\pgfpathlineto{\pgfqpoint{5.970901in}{2.658721in}}%
\pgfpathlineto{\pgfqpoint{6.675447in}{2.924311in}}%
\pgfpathlineto{\pgfqpoint{7.379992in}{3.247233in}}%
\pgfusepath{stroke}%
\end{pgfscope}%
\begin{pgfscope}%
\pgfpathrectangle{\pgfqpoint{2.201583in}{1.233139in}}{\pgfqpoint{5.425000in}{3.020000in}} %
\pgfusepath{clip}%
\pgfsetbuttcap%
\pgfsetroundjoin%
\definecolor{currentfill}{rgb}{0.298039,0.447059,0.690196}%
\pgfsetfillcolor{currentfill}%
\pgfsetlinewidth{0.752812pt}%
\definecolor{currentstroke}{rgb}{1.000000,1.000000,1.000000}%
\pgfsetstrokecolor{currentstroke}%
\pgfsetdash{}{0pt}%
\pgfsys@defobject{currentmarker}{\pgfqpoint{-0.104167in}{-0.104167in}}{\pgfqpoint{0.104167in}{0.104167in}}{%
\pgfpathmoveto{\pgfqpoint{0.000000in}{-0.104167in}}%
\pgfpathcurveto{\pgfqpoint{0.027625in}{-0.104167in}}{\pgfqpoint{0.054123in}{-0.093191in}}{\pgfqpoint{0.073657in}{-0.073657in}}%
\pgfpathcurveto{\pgfqpoint{0.093191in}{-0.054123in}}{\pgfqpoint{0.104167in}{-0.027625in}}{\pgfqpoint{0.104167in}{0.000000in}}%
\pgfpathcurveto{\pgfqpoint{0.104167in}{0.027625in}}{\pgfqpoint{0.093191in}{0.054123in}}{\pgfqpoint{0.073657in}{0.073657in}}%
\pgfpathcurveto{\pgfqpoint{0.054123in}{0.093191in}}{\pgfqpoint{0.027625in}{0.104167in}}{\pgfqpoint{0.000000in}{0.104167in}}%
\pgfpathcurveto{\pgfqpoint{-0.027625in}{0.104167in}}{\pgfqpoint{-0.054123in}{0.093191in}}{\pgfqpoint{-0.073657in}{0.073657in}}%
\pgfpathcurveto{\pgfqpoint{-0.093191in}{0.054123in}}{\pgfqpoint{-0.104167in}{0.027625in}}{\pgfqpoint{-0.104167in}{0.000000in}}%
\pgfpathcurveto{\pgfqpoint{-0.104167in}{-0.027625in}}{\pgfqpoint{-0.093191in}{-0.054123in}}{\pgfqpoint{-0.073657in}{-0.073657in}}%
\pgfpathcurveto{\pgfqpoint{-0.054123in}{-0.093191in}}{\pgfqpoint{-0.027625in}{-0.104167in}}{\pgfqpoint{0.000000in}{-0.104167in}}%
\pgfpathclose%
\pgfusepath{stroke,fill}%
}%
\begin{pgfscope}%
\pgfsys@transformshift{2.448174in}{1.507971in}%
\pgfsys@useobject{currentmarker}{}%
\end{pgfscope}%
\begin{pgfscope}%
\pgfsys@transformshift{3.152719in}{1.743500in}%
\pgfsys@useobject{currentmarker}{}%
\end{pgfscope}%
\begin{pgfscope}%
\pgfsys@transformshift{3.857265in}{2.151684in}%
\pgfsys@useobject{currentmarker}{}%
\end{pgfscope}%
\begin{pgfscope}%
\pgfsys@transformshift{4.561810in}{2.317605in}%
\pgfsys@useobject{currentmarker}{}%
\end{pgfscope}%
\begin{pgfscope}%
\pgfsys@transformshift{5.266356in}{2.450822in}%
\pgfsys@useobject{currentmarker}{}%
\end{pgfscope}%
\begin{pgfscope}%
\pgfsys@transformshift{5.970901in}{2.658721in}%
\pgfsys@useobject{currentmarker}{}%
\end{pgfscope}%
\begin{pgfscope}%
\pgfsys@transformshift{6.675447in}{2.924311in}%
\pgfsys@useobject{currentmarker}{}%
\end{pgfscope}%
\begin{pgfscope}%
\pgfsys@transformshift{7.379992in}{3.247233in}%
\pgfsys@useobject{currentmarker}{}%
\end{pgfscope}%
\end{pgfscope}%
\begin{pgfscope}%
\pgfpathrectangle{\pgfqpoint{2.201583in}{1.233139in}}{\pgfqpoint{5.425000in}{3.020000in}} %
\pgfusepath{clip}%
\pgfsetbuttcap%
\pgfsetroundjoin%
\pgfsetlinewidth{3.011250pt}%
\definecolor{currentstroke}{rgb}{0.866667,0.517647,0.321569}%
\pgfsetstrokecolor{currentstroke}%
\pgfsetdash{{9.000000pt}{3.000000pt}}{0.000000pt}%
\pgfpathmoveto{\pgfqpoint{2.448174in}{1.507686in}}%
\pgfpathlineto{\pgfqpoint{3.152719in}{1.743151in}}%
\pgfpathlineto{\pgfqpoint{3.857265in}{2.145642in}}%
\pgfpathlineto{\pgfqpoint{4.561810in}{2.347501in}}%
\pgfpathlineto{\pgfqpoint{5.266356in}{2.487033in}}%
\pgfpathlineto{\pgfqpoint{5.970901in}{2.696357in}}%
\pgfpathlineto{\pgfqpoint{6.675447in}{3.279924in}}%
\pgfpathlineto{\pgfqpoint{7.379992in}{3.614254in}}%
\pgfusepath{stroke}%
\end{pgfscope}%
\begin{pgfscope}%
\pgfpathrectangle{\pgfqpoint{2.201583in}{1.233139in}}{\pgfqpoint{5.425000in}{3.020000in}} %
\pgfusepath{clip}%
\pgfsetbuttcap%
\pgfsetmiterjoin%
\definecolor{currentfill}{rgb}{0.866667,0.517647,0.321569}%
\pgfsetfillcolor{currentfill}%
\pgfsetlinewidth{0.752812pt}%
\definecolor{currentstroke}{rgb}{1.000000,1.000000,1.000000}%
\pgfsetstrokecolor{currentstroke}%
\pgfsetdash{}{0pt}%
\pgfsys@defobject{currentmarker}{\pgfqpoint{-0.104167in}{-0.104167in}}{\pgfqpoint{0.104167in}{0.104167in}}{%
\pgfpathmoveto{\pgfqpoint{0.000000in}{0.104167in}}%
\pgfpathlineto{\pgfqpoint{-0.104167in}{-0.104167in}}%
\pgfpathlineto{\pgfqpoint{0.104167in}{-0.104167in}}%
\pgfpathclose%
\pgfusepath{stroke,fill}%
}%
\begin{pgfscope}%
\pgfsys@transformshift{2.448174in}{1.507686in}%
\pgfsys@useobject{currentmarker}{}%
\end{pgfscope}%
\begin{pgfscope}%
\pgfsys@transformshift{3.152719in}{1.743151in}%
\pgfsys@useobject{currentmarker}{}%
\end{pgfscope}%
\begin{pgfscope}%
\pgfsys@transformshift{3.857265in}{2.145642in}%
\pgfsys@useobject{currentmarker}{}%
\end{pgfscope}%
\begin{pgfscope}%
\pgfsys@transformshift{4.561810in}{2.347501in}%
\pgfsys@useobject{currentmarker}{}%
\end{pgfscope}%
\begin{pgfscope}%
\pgfsys@transformshift{5.266356in}{2.487033in}%
\pgfsys@useobject{currentmarker}{}%
\end{pgfscope}%
\begin{pgfscope}%
\pgfsys@transformshift{5.970901in}{2.696357in}%
\pgfsys@useobject{currentmarker}{}%
\end{pgfscope}%
\begin{pgfscope}%
\pgfsys@transformshift{6.675447in}{3.279924in}%
\pgfsys@useobject{currentmarker}{}%
\end{pgfscope}%
\begin{pgfscope}%
\pgfsys@transformshift{7.379992in}{3.614254in}%
\pgfsys@useobject{currentmarker}{}%
\end{pgfscope}%
\end{pgfscope}%
\begin{pgfscope}%
\pgfpathrectangle{\pgfqpoint{2.201583in}{1.233139in}}{\pgfqpoint{5.425000in}{3.020000in}} %
\pgfusepath{clip}%
\pgfsetbuttcap%
\pgfsetroundjoin%
\pgfsetlinewidth{3.011250pt}%
\definecolor{currentstroke}{rgb}{0.333333,0.658824,0.407843}%
\pgfsetstrokecolor{currentstroke}%
\pgfsetdash{{3.000000pt}{3.000000pt}}{0.000000pt}%
\pgfpathmoveto{\pgfqpoint{2.448174in}{1.509007in}}%
\pgfpathlineto{\pgfqpoint{3.152719in}{1.745017in}}%
\pgfpathlineto{\pgfqpoint{3.857265in}{2.152627in}}%
\pgfpathlineto{\pgfqpoint{4.561810in}{2.373692in}}%
\pgfpathlineto{\pgfqpoint{5.266356in}{2.518968in}}%
\pgfpathlineto{\pgfqpoint{5.970901in}{2.729108in}}%
\pgfpathlineto{\pgfqpoint{6.675447in}{3.738494in}}%
\pgfpathlineto{\pgfqpoint{7.379992in}{4.078162in}}%
\pgfusepath{stroke}%
\end{pgfscope}%
\begin{pgfscope}%
\pgfpathrectangle{\pgfqpoint{2.201583in}{1.233139in}}{\pgfqpoint{5.425000in}{3.020000in}} %
\pgfusepath{clip}%
\pgfsetbuttcap%
\pgfsetmiterjoin%
\definecolor{currentfill}{rgb}{0.333333,0.658824,0.407843}%
\pgfsetfillcolor{currentfill}%
\pgfsetlinewidth{0.752812pt}%
\definecolor{currentstroke}{rgb}{1.000000,1.000000,1.000000}%
\pgfsetstrokecolor{currentstroke}%
\pgfsetdash{}{0pt}%
\pgfsys@defobject{currentmarker}{\pgfqpoint{-0.104167in}{-0.104167in}}{\pgfqpoint{0.104167in}{0.104167in}}{%
\pgfpathmoveto{\pgfqpoint{-0.000000in}{-0.104167in}}%
\pgfpathlineto{\pgfqpoint{0.104167in}{0.104167in}}%
\pgfpathlineto{\pgfqpoint{-0.104167in}{0.104167in}}%
\pgfpathclose%
\pgfusepath{stroke,fill}%
}%
\begin{pgfscope}%
\pgfsys@transformshift{2.448174in}{1.509007in}%
\pgfsys@useobject{currentmarker}{}%
\end{pgfscope}%
\begin{pgfscope}%
\pgfsys@transformshift{3.152719in}{1.745017in}%
\pgfsys@useobject{currentmarker}{}%
\end{pgfscope}%
\begin{pgfscope}%
\pgfsys@transformshift{3.857265in}{2.152627in}%
\pgfsys@useobject{currentmarker}{}%
\end{pgfscope}%
\begin{pgfscope}%
\pgfsys@transformshift{4.561810in}{2.373692in}%
\pgfsys@useobject{currentmarker}{}%
\end{pgfscope}%
\begin{pgfscope}%
\pgfsys@transformshift{5.266356in}{2.518968in}%
\pgfsys@useobject{currentmarker}{}%
\end{pgfscope}%
\begin{pgfscope}%
\pgfsys@transformshift{5.970901in}{2.729108in}%
\pgfsys@useobject{currentmarker}{}%
\end{pgfscope}%
\begin{pgfscope}%
\pgfsys@transformshift{6.675447in}{3.738494in}%
\pgfsys@useobject{currentmarker}{}%
\end{pgfscope}%
\begin{pgfscope}%
\pgfsys@transformshift{7.379992in}{4.078162in}%
\pgfsys@useobject{currentmarker}{}%
\end{pgfscope}%
\end{pgfscope}%
\begin{pgfscope}%
\pgfsetrectcap%
\pgfsetmiterjoin%
\pgfsetlinewidth{1.003750pt}%
\definecolor{currentstroke}{rgb}{0.800000,0.800000,0.800000}%
\pgfsetstrokecolor{currentstroke}%
\pgfsetdash{}{0pt}%
\pgfpathmoveto{\pgfqpoint{2.201583in}{1.233139in}}%
\pgfpathlineto{\pgfqpoint{2.201583in}{4.253139in}}%
\pgfusepath{stroke}%
\end{pgfscope}%
\begin{pgfscope}%
\pgfsetrectcap%
\pgfsetmiterjoin%
\pgfsetlinewidth{1.003750pt}%
\definecolor{currentstroke}{rgb}{0.800000,0.800000,0.800000}%
\pgfsetstrokecolor{currentstroke}%
\pgfsetdash{}{0pt}%
\pgfpathmoveto{\pgfqpoint{7.626583in}{1.233139in}}%
\pgfpathlineto{\pgfqpoint{7.626583in}{4.253139in}}%
\pgfusepath{stroke}%
\end{pgfscope}%
\begin{pgfscope}%
\pgfsetrectcap%
\pgfsetmiterjoin%
\pgfsetlinewidth{1.003750pt}%
\definecolor{currentstroke}{rgb}{0.800000,0.800000,0.800000}%
\pgfsetstrokecolor{currentstroke}%
\pgfsetdash{}{0pt}%
\pgfpathmoveto{\pgfqpoint{2.201583in}{1.233139in}}%
\pgfpathlineto{\pgfqpoint{7.626583in}{1.233139in}}%
\pgfusepath{stroke}%
\end{pgfscope}%
\begin{pgfscope}%
\pgfsetrectcap%
\pgfsetmiterjoin%
\pgfsetlinewidth{1.003750pt}%
\definecolor{currentstroke}{rgb}{0.800000,0.800000,0.800000}%
\pgfsetstrokecolor{currentstroke}%
\pgfsetdash{}{0pt}%
\pgfpathmoveto{\pgfqpoint{2.201583in}{4.253139in}}%
\pgfpathlineto{\pgfqpoint{7.626583in}{4.253139in}}%
\pgfusepath{stroke}%
\end{pgfscope}%
\begin{pgfscope}%
\pgfsetbuttcap%
\pgfsetroundjoin%
\pgfsetlinewidth{4.015000pt}%
\definecolor{currentstroke}{rgb}{0.298039,0.447059,0.690196}%
\pgfsetstrokecolor{currentstroke}%
\pgfsetdash{{4.000000pt}{0.000000pt}}{0.000000pt}%
\pgfpathmoveto{\pgfqpoint{1.829764in}{4.568861in}}%
\pgfpathlineto{\pgfqpoint{2.538097in}{4.568861in}}%
\pgfusepath{stroke}%
\end{pgfscope}%
\begin{pgfscope}%
\pgfsetbuttcap%
\pgfsetroundjoin%
\definecolor{currentfill}{rgb}{0.298039,0.447059,0.690196}%
\pgfsetfillcolor{currentfill}%
\pgfsetlinewidth{0.000000pt}%
\definecolor{currentstroke}{rgb}{0.298039,0.447059,0.690196}%
\pgfsetstrokecolor{currentstroke}%
\pgfsetdash{}{0pt}%
\pgfsys@defobject{currentmarker}{\pgfqpoint{-0.104167in}{-0.104167in}}{\pgfqpoint{0.104167in}{0.104167in}}{%
\pgfpathmoveto{\pgfqpoint{0.000000in}{-0.104167in}}%
\pgfpathcurveto{\pgfqpoint{0.027625in}{-0.104167in}}{\pgfqpoint{0.054123in}{-0.093191in}}{\pgfqpoint{0.073657in}{-0.073657in}}%
\pgfpathcurveto{\pgfqpoint{0.093191in}{-0.054123in}}{\pgfqpoint{0.104167in}{-0.027625in}}{\pgfqpoint{0.104167in}{0.000000in}}%
\pgfpathcurveto{\pgfqpoint{0.104167in}{0.027625in}}{\pgfqpoint{0.093191in}{0.054123in}}{\pgfqpoint{0.073657in}{0.073657in}}%
\pgfpathcurveto{\pgfqpoint{0.054123in}{0.093191in}}{\pgfqpoint{0.027625in}{0.104167in}}{\pgfqpoint{0.000000in}{0.104167in}}%
\pgfpathcurveto{\pgfqpoint{-0.027625in}{0.104167in}}{\pgfqpoint{-0.054123in}{0.093191in}}{\pgfqpoint{-0.073657in}{0.073657in}}%
\pgfpathcurveto{\pgfqpoint{-0.093191in}{0.054123in}}{\pgfqpoint{-0.104167in}{0.027625in}}{\pgfqpoint{-0.104167in}{0.000000in}}%
\pgfpathcurveto{\pgfqpoint{-0.104167in}{-0.027625in}}{\pgfqpoint{-0.093191in}{-0.054123in}}{\pgfqpoint{-0.073657in}{-0.073657in}}%
\pgfpathcurveto{\pgfqpoint{-0.054123in}{-0.093191in}}{\pgfqpoint{-0.027625in}{-0.104167in}}{\pgfqpoint{0.000000in}{-0.104167in}}%
\pgfpathclose%
\pgfusepath{fill}%
}%
\begin{pgfscope}%
\pgfsys@transformshift{2.183930in}{4.568861in}%
\pgfsys@useobject{currentmarker}{}%
\end{pgfscope}%
\end{pgfscope}%
\begin{pgfscope}%
\definecolor{textcolor}{rgb}{0.150000,0.150000,0.150000}%
\pgfsetstrokecolor{textcolor}%
\pgfsetfillcolor{textcolor}%
\pgftext[x=2.585319in,y=4.403583in,left,base]{\color{textcolor}\rmfamily\fontsize{34.000000}{40.800000}\selectfont LN}%
\end{pgfscope}%
\begin{pgfscope}%
\pgfsetbuttcap%
\pgfsetroundjoin%
\pgfsetlinewidth{4.015000pt}%
\definecolor{currentstroke}{rgb}{0.866667,0.517647,0.321569}%
\pgfsetstrokecolor{currentstroke}%
\pgfsetdash{{12.000000pt}{4.000000pt}}{0.000000pt}%
\pgfpathmoveto{\pgfqpoint{3.279014in}{4.568861in}}%
\pgfpathlineto{\pgfqpoint{3.987347in}{4.568861in}}%
\pgfusepath{stroke}%
\end{pgfscope}%
\begin{pgfscope}%
\pgfsetbuttcap%
\pgfsetmiterjoin%
\definecolor{currentfill}{rgb}{0.866667,0.517647,0.321569}%
\pgfsetfillcolor{currentfill}%
\pgfsetlinewidth{0.000000pt}%
\definecolor{currentstroke}{rgb}{0.866667,0.517647,0.321569}%
\pgfsetstrokecolor{currentstroke}%
\pgfsetdash{}{0pt}%
\pgfsys@defobject{currentmarker}{\pgfqpoint{-0.104167in}{-0.104167in}}{\pgfqpoint{0.104167in}{0.104167in}}{%
\pgfpathmoveto{\pgfqpoint{0.000000in}{0.104167in}}%
\pgfpathlineto{\pgfqpoint{-0.104167in}{-0.104167in}}%
\pgfpathlineto{\pgfqpoint{0.104167in}{-0.104167in}}%
\pgfpathclose%
\pgfusepath{fill}%
}%
\begin{pgfscope}%
\pgfsys@transformshift{3.633180in}{4.568861in}%
\pgfsys@useobject{currentmarker}{}%
\end{pgfscope}%
\end{pgfscope}%
\begin{pgfscope}%
\definecolor{textcolor}{rgb}{0.150000,0.150000,0.150000}%
\pgfsetstrokecolor{textcolor}%
\pgfsetfillcolor{textcolor}%
\pgftext[x=4.034569in,y=4.403583in,left,base]{\color{textcolor}\rmfamily\fontsize{34.000000}{40.800000}\selectfont ALL\_M}%
\end{pgfscope}%
\begin{pgfscope}%
\pgfsetbuttcap%
\pgfsetroundjoin%
\pgfsetlinewidth{4.015000pt}%
\definecolor{currentstroke}{rgb}{0.333333,0.658824,0.407843}%
\pgfsetstrokecolor{currentstroke}%
\pgfsetdash{{4.000000pt}{4.000000pt}}{0.000000pt}%
\pgfpathmoveto{\pgfqpoint{5.728903in}{4.568861in}}%
\pgfpathlineto{\pgfqpoint{6.437236in}{4.568861in}}%
\pgfusepath{stroke}%
\end{pgfscope}%
\begin{pgfscope}%
\pgfsetbuttcap%
\pgfsetmiterjoin%
\definecolor{currentfill}{rgb}{0.333333,0.658824,0.407843}%
\pgfsetfillcolor{currentfill}%
\pgfsetlinewidth{0.000000pt}%
\definecolor{currentstroke}{rgb}{0.333333,0.658824,0.407843}%
\pgfsetstrokecolor{currentstroke}%
\pgfsetdash{}{0pt}%
\pgfsys@defobject{currentmarker}{\pgfqpoint{-0.104167in}{-0.104167in}}{\pgfqpoint{0.104167in}{0.104167in}}{%
\pgfpathmoveto{\pgfqpoint{-0.000000in}{-0.104167in}}%
\pgfpathlineto{\pgfqpoint{0.104167in}{0.104167in}}%
\pgfpathlineto{\pgfqpoint{-0.104167in}{0.104167in}}%
\pgfpathclose%
\pgfusepath{fill}%
}%
\begin{pgfscope}%
\pgfsys@transformshift{6.083069in}{4.568861in}%
\pgfsys@useobject{currentmarker}{}%
\end{pgfscope}%
\end{pgfscope}%
\begin{pgfscope}%
\definecolor{textcolor}{rgb}{0.150000,0.150000,0.150000}%
\pgfsetstrokecolor{textcolor}%
\pgfsetfillcolor{textcolor}%
\pgftext[x=6.484458in,y=4.403583in,left,base]{\color{textcolor}\rmfamily\fontsize{34.000000}{40.800000}\selectfont ALL\_C}%
\end{pgfscope}%
\end{pgfpicture}%
\makeatother%
\endgroup%
%
}
\caption{Simple}
\end{subfigure}%
\begin{subfigure}[b]{0.5\linewidth}
\centering
 \resizebox{\columnwidth}{!}{%
%% Creator: Matplotlib, PGF backend
%%
%% To include the figure in your LaTeX document, write
%%   \input{<filename>.pgf}
%%
%% Make sure the required packages are loaded in your preamble
%%   \usepackage{pgf}
%%
%% Figures using additional raster images can only be included by \input if
%% they are in the same directory as the main LaTeX file. For loading figures
%% from other directories you can use the `import` package
%%   \usepackage{import}
%% and then include the figures with
%%   \import{<path to file>}{<filename>.pgf}
%%
%% Matplotlib used the following preamble
%%   \usepackage{fontspec}
%%   \setmonofont{Andale Mono}
%%
\begingroup%
\makeatletter%
\begin{pgfpicture}%
\pgfpathrectangle{\pgfpointorigin}{\pgfqpoint{7.224961in}{4.906038in}}%
\pgfusepath{use as bounding box, clip}%
\begin{pgfscope}%
\pgfsetbuttcap%
\pgfsetmiterjoin%
\definecolor{currentfill}{rgb}{1.000000,1.000000,1.000000}%
\pgfsetfillcolor{currentfill}%
\pgfsetlinewidth{0.000000pt}%
\definecolor{currentstroke}{rgb}{1.000000,1.000000,1.000000}%
\pgfsetstrokecolor{currentstroke}%
\pgfsetdash{}{0pt}%
\pgfpathmoveto{\pgfqpoint{0.000000in}{0.000000in}}%
\pgfpathlineto{\pgfqpoint{7.224961in}{0.000000in}}%
\pgfpathlineto{\pgfqpoint{7.224961in}{4.906038in}}%
\pgfpathlineto{\pgfqpoint{0.000000in}{4.906038in}}%
\pgfpathclose%
\pgfusepath{fill}%
\end{pgfscope}%
\begin{pgfscope}%
\pgfsetbuttcap%
\pgfsetmiterjoin%
\definecolor{currentfill}{rgb}{1.000000,1.000000,1.000000}%
\pgfsetfillcolor{currentfill}%
\pgfsetlinewidth{0.000000pt}%
\definecolor{currentstroke}{rgb}{0.000000,0.000000,0.000000}%
\pgfsetstrokecolor{currentstroke}%
\pgfsetstrokeopacity{0.000000}%
\pgfsetdash{}{0pt}%
\pgfpathmoveto{\pgfqpoint{1.633294in}{1.124483in}}%
\pgfpathlineto{\pgfqpoint{7.058294in}{1.124483in}}%
\pgfpathlineto{\pgfqpoint{7.058294in}{4.144483in}}%
\pgfpathlineto{\pgfqpoint{1.633294in}{4.144483in}}%
\pgfpathclose%
\pgfusepath{fill}%
\end{pgfscope}%
\begin{pgfscope}%
\pgfpathrectangle{\pgfqpoint{1.633294in}{1.124483in}}{\pgfqpoint{5.425000in}{3.020000in}} %
\pgfusepath{clip}%
\pgfsetroundcap%
\pgfsetroundjoin%
\pgfsetlinewidth{0.803000pt}%
\definecolor{currentstroke}{rgb}{0.800000,0.800000,0.800000}%
\pgfsetstrokecolor{currentstroke}%
\pgfsetdash{}{0pt}%
\pgfpathmoveto{\pgfqpoint{1.879885in}{1.124483in}}%
\pgfpathlineto{\pgfqpoint{1.879885in}{4.144483in}}%
\pgfusepath{stroke}%
\end{pgfscope}%
\begin{pgfscope}%
\definecolor{textcolor}{rgb}{0.150000,0.150000,0.150000}%
\pgfsetstrokecolor{textcolor}%
\pgfsetfillcolor{textcolor}%
\pgftext[x=1.879885in,y=0.960594in,,top]{\color{textcolor}\rmfamily\fontsize{35.200000}{42.240000}\selectfont 1}%
\end{pgfscope}%
\begin{pgfscope}%
\pgfpathrectangle{\pgfqpoint{1.633294in}{1.124483in}}{\pgfqpoint{5.425000in}{3.020000in}} %
\pgfusepath{clip}%
\pgfsetroundcap%
\pgfsetroundjoin%
\pgfsetlinewidth{0.803000pt}%
\definecolor{currentstroke}{rgb}{0.800000,0.800000,0.800000}%
\pgfsetstrokecolor{currentstroke}%
\pgfsetdash{}{0pt}%
\pgfpathmoveto{\pgfqpoint{2.584431in}{1.124483in}}%
\pgfpathlineto{\pgfqpoint{2.584431in}{4.144483in}}%
\pgfusepath{stroke}%
\end{pgfscope}%
\begin{pgfscope}%
\definecolor{textcolor}{rgb}{0.150000,0.150000,0.150000}%
\pgfsetstrokecolor{textcolor}%
\pgfsetfillcolor{textcolor}%
\pgftext[x=2.584431in,y=0.960594in,,top]{\color{textcolor}\rmfamily\fontsize{35.200000}{42.240000}\selectfont 2}%
\end{pgfscope}%
\begin{pgfscope}%
\pgfpathrectangle{\pgfqpoint{1.633294in}{1.124483in}}{\pgfqpoint{5.425000in}{3.020000in}} %
\pgfusepath{clip}%
\pgfsetroundcap%
\pgfsetroundjoin%
\pgfsetlinewidth{0.803000pt}%
\definecolor{currentstroke}{rgb}{0.800000,0.800000,0.800000}%
\pgfsetstrokecolor{currentstroke}%
\pgfsetdash{}{0pt}%
\pgfpathmoveto{\pgfqpoint{3.288976in}{1.124483in}}%
\pgfpathlineto{\pgfqpoint{3.288976in}{4.144483in}}%
\pgfusepath{stroke}%
\end{pgfscope}%
\begin{pgfscope}%
\definecolor{textcolor}{rgb}{0.150000,0.150000,0.150000}%
\pgfsetstrokecolor{textcolor}%
\pgfsetfillcolor{textcolor}%
\pgftext[x=3.288976in,y=0.960594in,,top]{\color{textcolor}\rmfamily\fontsize{35.200000}{42.240000}\selectfont 3}%
\end{pgfscope}%
\begin{pgfscope}%
\pgfpathrectangle{\pgfqpoint{1.633294in}{1.124483in}}{\pgfqpoint{5.425000in}{3.020000in}} %
\pgfusepath{clip}%
\pgfsetroundcap%
\pgfsetroundjoin%
\pgfsetlinewidth{0.803000pt}%
\definecolor{currentstroke}{rgb}{0.800000,0.800000,0.800000}%
\pgfsetstrokecolor{currentstroke}%
\pgfsetdash{}{0pt}%
\pgfpathmoveto{\pgfqpoint{3.993521in}{1.124483in}}%
\pgfpathlineto{\pgfqpoint{3.993521in}{4.144483in}}%
\pgfusepath{stroke}%
\end{pgfscope}%
\begin{pgfscope}%
\definecolor{textcolor}{rgb}{0.150000,0.150000,0.150000}%
\pgfsetstrokecolor{textcolor}%
\pgfsetfillcolor{textcolor}%
\pgftext[x=3.993521in,y=0.960594in,,top]{\color{textcolor}\rmfamily\fontsize{35.200000}{42.240000}\selectfont 4}%
\end{pgfscope}%
\begin{pgfscope}%
\pgfpathrectangle{\pgfqpoint{1.633294in}{1.124483in}}{\pgfqpoint{5.425000in}{3.020000in}} %
\pgfusepath{clip}%
\pgfsetroundcap%
\pgfsetroundjoin%
\pgfsetlinewidth{0.803000pt}%
\definecolor{currentstroke}{rgb}{0.800000,0.800000,0.800000}%
\pgfsetstrokecolor{currentstroke}%
\pgfsetdash{}{0pt}%
\pgfpathmoveto{\pgfqpoint{4.698067in}{1.124483in}}%
\pgfpathlineto{\pgfqpoint{4.698067in}{4.144483in}}%
\pgfusepath{stroke}%
\end{pgfscope}%
\begin{pgfscope}%
\definecolor{textcolor}{rgb}{0.150000,0.150000,0.150000}%
\pgfsetstrokecolor{textcolor}%
\pgfsetfillcolor{textcolor}%
\pgftext[x=4.698067in,y=0.960594in,,top]{\color{textcolor}\rmfamily\fontsize{35.200000}{42.240000}\selectfont 5}%
\end{pgfscope}%
\begin{pgfscope}%
\pgfpathrectangle{\pgfqpoint{1.633294in}{1.124483in}}{\pgfqpoint{5.425000in}{3.020000in}} %
\pgfusepath{clip}%
\pgfsetroundcap%
\pgfsetroundjoin%
\pgfsetlinewidth{0.803000pt}%
\definecolor{currentstroke}{rgb}{0.800000,0.800000,0.800000}%
\pgfsetstrokecolor{currentstroke}%
\pgfsetdash{}{0pt}%
\pgfpathmoveto{\pgfqpoint{5.402612in}{1.124483in}}%
\pgfpathlineto{\pgfqpoint{5.402612in}{4.144483in}}%
\pgfusepath{stroke}%
\end{pgfscope}%
\begin{pgfscope}%
\definecolor{textcolor}{rgb}{0.150000,0.150000,0.150000}%
\pgfsetstrokecolor{textcolor}%
\pgfsetfillcolor{textcolor}%
\pgftext[x=5.402612in,y=0.960594in,,top]{\color{textcolor}\rmfamily\fontsize{35.200000}{42.240000}\selectfont 6}%
\end{pgfscope}%
\begin{pgfscope}%
\pgfpathrectangle{\pgfqpoint{1.633294in}{1.124483in}}{\pgfqpoint{5.425000in}{3.020000in}} %
\pgfusepath{clip}%
\pgfsetroundcap%
\pgfsetroundjoin%
\pgfsetlinewidth{0.803000pt}%
\definecolor{currentstroke}{rgb}{0.800000,0.800000,0.800000}%
\pgfsetstrokecolor{currentstroke}%
\pgfsetdash{}{0pt}%
\pgfpathmoveto{\pgfqpoint{6.107158in}{1.124483in}}%
\pgfpathlineto{\pgfqpoint{6.107158in}{4.144483in}}%
\pgfusepath{stroke}%
\end{pgfscope}%
\begin{pgfscope}%
\definecolor{textcolor}{rgb}{0.150000,0.150000,0.150000}%
\pgfsetstrokecolor{textcolor}%
\pgfsetfillcolor{textcolor}%
\pgftext[x=6.107158in,y=0.960594in,,top]{\color{textcolor}\rmfamily\fontsize{35.200000}{42.240000}\selectfont 7}%
\end{pgfscope}%
\begin{pgfscope}%
\pgfpathrectangle{\pgfqpoint{1.633294in}{1.124483in}}{\pgfqpoint{5.425000in}{3.020000in}} %
\pgfusepath{clip}%
\pgfsetroundcap%
\pgfsetroundjoin%
\pgfsetlinewidth{0.803000pt}%
\definecolor{currentstroke}{rgb}{0.800000,0.800000,0.800000}%
\pgfsetstrokecolor{currentstroke}%
\pgfsetdash{}{0pt}%
\pgfpathmoveto{\pgfqpoint{6.811703in}{1.124483in}}%
\pgfpathlineto{\pgfqpoint{6.811703in}{4.144483in}}%
\pgfusepath{stroke}%
\end{pgfscope}%
\begin{pgfscope}%
\definecolor{textcolor}{rgb}{0.150000,0.150000,0.150000}%
\pgfsetstrokecolor{textcolor}%
\pgfsetfillcolor{textcolor}%
\pgftext[x=6.811703in,y=0.960594in,,top]{\color{textcolor}\rmfamily\fontsize{35.200000}{42.240000}\selectfont 8}%
\end{pgfscope}%
\begin{pgfscope}%
\definecolor{textcolor}{rgb}{0.150000,0.150000,0.150000}%
\pgfsetstrokecolor{textcolor}%
\pgfsetfillcolor{textcolor}%
\pgftext[x=4.345794in,y=0.470417in,,top]{\color{textcolor}\rmfamily\fontsize{30.000000}{36.000000}\selectfont Workload}%
\end{pgfscope}%
\begin{pgfscope}%
\pgfpathrectangle{\pgfqpoint{1.633294in}{1.124483in}}{\pgfqpoint{5.425000in}{3.020000in}} %
\pgfusepath{clip}%
\pgfsetroundcap%
\pgfsetroundjoin%
\pgfsetlinewidth{0.803000pt}%
\definecolor{currentstroke}{rgb}{0.800000,0.800000,0.800000}%
\pgfsetstrokecolor{currentstroke}%
\pgfsetdash{}{0pt}%
\pgfpathmoveto{\pgfqpoint{1.633294in}{1.124483in}}%
\pgfpathlineto{\pgfqpoint{7.058294in}{1.124483in}}%
\pgfusepath{stroke}%
\end{pgfscope}%
\begin{pgfscope}%
\definecolor{textcolor}{rgb}{0.150000,0.150000,0.150000}%
\pgfsetstrokecolor{textcolor}%
\pgfsetfillcolor{textcolor}%
\pgftext[x=1.245005in,y=0.954838in,left,base]{\color{textcolor}\rmfamily\fontsize{35.200000}{42.240000}\selectfont 0}%
\end{pgfscope}%
\begin{pgfscope}%
\pgfpathrectangle{\pgfqpoint{1.633294in}{1.124483in}}{\pgfqpoint{5.425000in}{3.020000in}} %
\pgfusepath{clip}%
\pgfsetroundcap%
\pgfsetroundjoin%
\pgfsetlinewidth{0.803000pt}%
\definecolor{currentstroke}{rgb}{0.800000,0.800000,0.800000}%
\pgfsetstrokecolor{currentstroke}%
\pgfsetdash{}{0pt}%
\pgfpathmoveto{\pgfqpoint{1.633294in}{2.497210in}}%
\pgfpathlineto{\pgfqpoint{7.058294in}{2.497210in}}%
\pgfusepath{stroke}%
\end{pgfscope}%
\begin{pgfscope}%
\definecolor{textcolor}{rgb}{0.150000,0.150000,0.150000}%
\pgfsetstrokecolor{textcolor}%
\pgfsetfillcolor{textcolor}%
\pgftext[x=0.571805in,y=2.327566in,left,base]{\color{textcolor}\rmfamily\fontsize{35.200000}{42.240000}\selectfont 1000}%
\end{pgfscope}%
\begin{pgfscope}%
\pgfpathrectangle{\pgfqpoint{1.633294in}{1.124483in}}{\pgfqpoint{5.425000in}{3.020000in}} %
\pgfusepath{clip}%
\pgfsetroundcap%
\pgfsetroundjoin%
\pgfsetlinewidth{0.803000pt}%
\definecolor{currentstroke}{rgb}{0.800000,0.800000,0.800000}%
\pgfsetstrokecolor{currentstroke}%
\pgfsetdash{}{0pt}%
\pgfpathmoveto{\pgfqpoint{1.633294in}{3.869937in}}%
\pgfpathlineto{\pgfqpoint{7.058294in}{3.869937in}}%
\pgfusepath{stroke}%
\end{pgfscope}%
\begin{pgfscope}%
\definecolor{textcolor}{rgb}{0.150000,0.150000,0.150000}%
\pgfsetstrokecolor{textcolor}%
\pgfsetfillcolor{textcolor}%
\pgftext[x=0.571805in,y=3.700293in,left,base]{\color{textcolor}\rmfamily\fontsize{35.200000}{42.240000}\selectfont 2000}%
\end{pgfscope}%
\begin{pgfscope}%
\definecolor{textcolor}{rgb}{0.150000,0.150000,0.150000}%
\pgfsetstrokecolor{textcolor}%
\pgfsetfillcolor{textcolor}%
\pgftext[x=0.516250in,y=2.634483in,,bottom,rotate=90.000000]{\color{textcolor}\rmfamily\fontsize{30.000000}{36.000000}\selectfont Cumulative Run Time (s)}%
\end{pgfscope}%
\begin{pgfscope}%
\pgfpathrectangle{\pgfqpoint{1.633294in}{1.124483in}}{\pgfqpoint{5.425000in}{3.020000in}} %
\pgfusepath{clip}%
\pgfsetbuttcap%
\pgfsetroundjoin%
\pgfsetlinewidth{3.011250pt}%
\definecolor{currentstroke}{rgb}{0.298039,0.447059,0.690196}%
\pgfsetstrokecolor{currentstroke}%
\pgfsetdash{{3.000000pt}{0.000000pt}}{0.000000pt}%
\pgfpathmoveto{\pgfqpoint{1.879885in}{1.399546in}}%
\pgfpathlineto{\pgfqpoint{2.584431in}{1.634268in}}%
\pgfpathlineto{\pgfqpoint{3.288976in}{2.039460in}}%
\pgfpathlineto{\pgfqpoint{3.993521in}{2.116980in}}%
\pgfpathlineto{\pgfqpoint{4.698067in}{2.171519in}}%
\pgfpathlineto{\pgfqpoint{5.402612in}{2.201384in}}%
\pgfpathlineto{\pgfqpoint{6.107158in}{2.276724in}}%
\pgfpathlineto{\pgfqpoint{6.811703in}{2.360706in}}%
\pgfusepath{stroke}%
\end{pgfscope}%
\begin{pgfscope}%
\pgfpathrectangle{\pgfqpoint{1.633294in}{1.124483in}}{\pgfqpoint{5.425000in}{3.020000in}} %
\pgfusepath{clip}%
\pgfsetbuttcap%
\pgfsetroundjoin%
\definecolor{currentfill}{rgb}{0.298039,0.447059,0.690196}%
\pgfsetfillcolor{currentfill}%
\pgfsetlinewidth{0.752812pt}%
\definecolor{currentstroke}{rgb}{1.000000,1.000000,1.000000}%
\pgfsetstrokecolor{currentstroke}%
\pgfsetdash{}{0pt}%
\pgfsys@defobject{currentmarker}{\pgfqpoint{-0.104167in}{-0.104167in}}{\pgfqpoint{0.104167in}{0.104167in}}{%
\pgfpathmoveto{\pgfqpoint{0.000000in}{-0.104167in}}%
\pgfpathcurveto{\pgfqpoint{0.027625in}{-0.104167in}}{\pgfqpoint{0.054123in}{-0.093191in}}{\pgfqpoint{0.073657in}{-0.073657in}}%
\pgfpathcurveto{\pgfqpoint{0.093191in}{-0.054123in}}{\pgfqpoint{0.104167in}{-0.027625in}}{\pgfqpoint{0.104167in}{0.000000in}}%
\pgfpathcurveto{\pgfqpoint{0.104167in}{0.027625in}}{\pgfqpoint{0.093191in}{0.054123in}}{\pgfqpoint{0.073657in}{0.073657in}}%
\pgfpathcurveto{\pgfqpoint{0.054123in}{0.093191in}}{\pgfqpoint{0.027625in}{0.104167in}}{\pgfqpoint{0.000000in}{0.104167in}}%
\pgfpathcurveto{\pgfqpoint{-0.027625in}{0.104167in}}{\pgfqpoint{-0.054123in}{0.093191in}}{\pgfqpoint{-0.073657in}{0.073657in}}%
\pgfpathcurveto{\pgfqpoint{-0.093191in}{0.054123in}}{\pgfqpoint{-0.104167in}{0.027625in}}{\pgfqpoint{-0.104167in}{0.000000in}}%
\pgfpathcurveto{\pgfqpoint{-0.104167in}{-0.027625in}}{\pgfqpoint{-0.093191in}{-0.054123in}}{\pgfqpoint{-0.073657in}{-0.073657in}}%
\pgfpathcurveto{\pgfqpoint{-0.054123in}{-0.093191in}}{\pgfqpoint{-0.027625in}{-0.104167in}}{\pgfqpoint{0.000000in}{-0.104167in}}%
\pgfpathclose%
\pgfusepath{stroke,fill}%
}%
\begin{pgfscope}%
\pgfsys@transformshift{1.879885in}{1.399546in}%
\pgfsys@useobject{currentmarker}{}%
\end{pgfscope}%
\begin{pgfscope}%
\pgfsys@transformshift{2.584431in}{1.634268in}%
\pgfsys@useobject{currentmarker}{}%
\end{pgfscope}%
\begin{pgfscope}%
\pgfsys@transformshift{3.288976in}{2.039460in}%
\pgfsys@useobject{currentmarker}{}%
\end{pgfscope}%
\begin{pgfscope}%
\pgfsys@transformshift{3.993521in}{2.116980in}%
\pgfsys@useobject{currentmarker}{}%
\end{pgfscope}%
\begin{pgfscope}%
\pgfsys@transformshift{4.698067in}{2.171519in}%
\pgfsys@useobject{currentmarker}{}%
\end{pgfscope}%
\begin{pgfscope}%
\pgfsys@transformshift{5.402612in}{2.201384in}%
\pgfsys@useobject{currentmarker}{}%
\end{pgfscope}%
\begin{pgfscope}%
\pgfsys@transformshift{6.107158in}{2.276724in}%
\pgfsys@useobject{currentmarker}{}%
\end{pgfscope}%
\begin{pgfscope}%
\pgfsys@transformshift{6.811703in}{2.360706in}%
\pgfsys@useobject{currentmarker}{}%
\end{pgfscope}%
\end{pgfscope}%
\begin{pgfscope}%
\pgfpathrectangle{\pgfqpoint{1.633294in}{1.124483in}}{\pgfqpoint{5.425000in}{3.020000in}} %
\pgfusepath{clip}%
\pgfsetbuttcap%
\pgfsetroundjoin%
\pgfsetlinewidth{3.011250pt}%
\definecolor{currentstroke}{rgb}{0.866667,0.517647,0.321569}%
\pgfsetstrokecolor{currentstroke}%
\pgfsetdash{{6.000000pt}{6.000000pt}}{0.000000pt}%
\pgfpathmoveto{\pgfqpoint{1.879885in}{1.397720in}}%
\pgfpathlineto{\pgfqpoint{2.584431in}{1.634267in}}%
\pgfpathlineto{\pgfqpoint{3.288976in}{2.042740in}}%
\pgfpathlineto{\pgfqpoint{3.993521in}{2.135089in}}%
\pgfpathlineto{\pgfqpoint{4.698067in}{2.203416in}}%
\pgfpathlineto{\pgfqpoint{5.402612in}{2.251375in}}%
\pgfpathlineto{\pgfqpoint{6.107158in}{2.549296in}}%
\pgfpathlineto{\pgfqpoint{6.811703in}{2.665939in}}%
\pgfusepath{stroke}%
\end{pgfscope}%
\begin{pgfscope}%
\pgfpathrectangle{\pgfqpoint{1.633294in}{1.124483in}}{\pgfqpoint{5.425000in}{3.020000in}} %
\pgfusepath{clip}%
\pgfsetbuttcap%
\pgfsetmiterjoin%
\definecolor{currentfill}{rgb}{0.866667,0.517647,0.321569}%
\pgfsetfillcolor{currentfill}%
\pgfsetlinewidth{0.752812pt}%
\definecolor{currentstroke}{rgb}{1.000000,1.000000,1.000000}%
\pgfsetstrokecolor{currentstroke}%
\pgfsetdash{}{0pt}%
\pgfsys@defobject{currentmarker}{\pgfqpoint{-0.104167in}{-0.104167in}}{\pgfqpoint{0.104167in}{0.104167in}}{%
\pgfpathmoveto{\pgfqpoint{0.000000in}{0.104167in}}%
\pgfpathlineto{\pgfqpoint{-0.104167in}{-0.104167in}}%
\pgfpathlineto{\pgfqpoint{0.104167in}{-0.104167in}}%
\pgfpathclose%
\pgfusepath{stroke,fill}%
}%
\begin{pgfscope}%
\pgfsys@transformshift{1.879885in}{1.397720in}%
\pgfsys@useobject{currentmarker}{}%
\end{pgfscope}%
\begin{pgfscope}%
\pgfsys@transformshift{2.584431in}{1.634267in}%
\pgfsys@useobject{currentmarker}{}%
\end{pgfscope}%
\begin{pgfscope}%
\pgfsys@transformshift{3.288976in}{2.042740in}%
\pgfsys@useobject{currentmarker}{}%
\end{pgfscope}%
\begin{pgfscope}%
\pgfsys@transformshift{3.993521in}{2.135089in}%
\pgfsys@useobject{currentmarker}{}%
\end{pgfscope}%
\begin{pgfscope}%
\pgfsys@transformshift{4.698067in}{2.203416in}%
\pgfsys@useobject{currentmarker}{}%
\end{pgfscope}%
\begin{pgfscope}%
\pgfsys@transformshift{5.402612in}{2.251375in}%
\pgfsys@useobject{currentmarker}{}%
\end{pgfscope}%
\begin{pgfscope}%
\pgfsys@transformshift{6.107158in}{2.549296in}%
\pgfsys@useobject{currentmarker}{}%
\end{pgfscope}%
\begin{pgfscope}%
\pgfsys@transformshift{6.811703in}{2.665939in}%
\pgfsys@useobject{currentmarker}{}%
\end{pgfscope}%
\end{pgfscope}%
\begin{pgfscope}%
\pgfpathrectangle{\pgfqpoint{1.633294in}{1.124483in}}{\pgfqpoint{5.425000in}{3.020000in}} %
\pgfusepath{clip}%
\pgfsetbuttcap%
\pgfsetroundjoin%
\pgfsetlinewidth{3.011250pt}%
\definecolor{currentstroke}{rgb}{0.333333,0.658824,0.407843}%
\pgfsetstrokecolor{currentstroke}%
\pgfsetdash{{3.000000pt}{3.000000pt}}{0.000000pt}%
\pgfpathmoveto{\pgfqpoint{1.879885in}{1.398004in}}%
\pgfpathlineto{\pgfqpoint{2.584431in}{1.631590in}}%
\pgfpathlineto{\pgfqpoint{3.288976in}{2.033293in}}%
\pgfpathlineto{\pgfqpoint{3.993521in}{2.251201in}}%
\pgfpathlineto{\pgfqpoint{4.698067in}{2.398961in}}%
\pgfpathlineto{\pgfqpoint{5.402612in}{2.609379in}}%
\pgfpathlineto{\pgfqpoint{6.107158in}{3.503321in}}%
\pgfpathlineto{\pgfqpoint{6.811703in}{3.832312in}}%
\pgfusepath{stroke}%
\end{pgfscope}%
\begin{pgfscope}%
\pgfpathrectangle{\pgfqpoint{1.633294in}{1.124483in}}{\pgfqpoint{5.425000in}{3.020000in}} %
\pgfusepath{clip}%
\pgfsetbuttcap%
\pgfsetmiterjoin%
\definecolor{currentfill}{rgb}{0.333333,0.658824,0.407843}%
\pgfsetfillcolor{currentfill}%
\pgfsetlinewidth{0.752812pt}%
\definecolor{currentstroke}{rgb}{1.000000,1.000000,1.000000}%
\pgfsetstrokecolor{currentstroke}%
\pgfsetdash{}{0pt}%
\pgfsys@defobject{currentmarker}{\pgfqpoint{-0.104167in}{-0.104167in}}{\pgfqpoint{0.104167in}{0.104167in}}{%
\pgfpathmoveto{\pgfqpoint{-0.000000in}{-0.104167in}}%
\pgfpathlineto{\pgfqpoint{0.104167in}{0.104167in}}%
\pgfpathlineto{\pgfqpoint{-0.104167in}{0.104167in}}%
\pgfpathclose%
\pgfusepath{stroke,fill}%
}%
\begin{pgfscope}%
\pgfsys@transformshift{1.879885in}{1.398004in}%
\pgfsys@useobject{currentmarker}{}%
\end{pgfscope}%
\begin{pgfscope}%
\pgfsys@transformshift{2.584431in}{1.631590in}%
\pgfsys@useobject{currentmarker}{}%
\end{pgfscope}%
\begin{pgfscope}%
\pgfsys@transformshift{3.288976in}{2.033293in}%
\pgfsys@useobject{currentmarker}{}%
\end{pgfscope}%
\begin{pgfscope}%
\pgfsys@transformshift{3.993521in}{2.251201in}%
\pgfsys@useobject{currentmarker}{}%
\end{pgfscope}%
\begin{pgfscope}%
\pgfsys@transformshift{4.698067in}{2.398961in}%
\pgfsys@useobject{currentmarker}{}%
\end{pgfscope}%
\begin{pgfscope}%
\pgfsys@transformshift{5.402612in}{2.609379in}%
\pgfsys@useobject{currentmarker}{}%
\end{pgfscope}%
\begin{pgfscope}%
\pgfsys@transformshift{6.107158in}{3.503321in}%
\pgfsys@useobject{currentmarker}{}%
\end{pgfscope}%
\begin{pgfscope}%
\pgfsys@transformshift{6.811703in}{3.832312in}%
\pgfsys@useobject{currentmarker}{}%
\end{pgfscope}%
\end{pgfscope}%
\begin{pgfscope}%
\pgfsetrectcap%
\pgfsetmiterjoin%
\pgfsetlinewidth{1.003750pt}%
\definecolor{currentstroke}{rgb}{0.800000,0.800000,0.800000}%
\pgfsetstrokecolor{currentstroke}%
\pgfsetdash{}{0pt}%
\pgfpathmoveto{\pgfqpoint{1.633294in}{1.124483in}}%
\pgfpathlineto{\pgfqpoint{1.633294in}{4.144483in}}%
\pgfusepath{stroke}%
\end{pgfscope}%
\begin{pgfscope}%
\pgfsetrectcap%
\pgfsetmiterjoin%
\pgfsetlinewidth{1.003750pt}%
\definecolor{currentstroke}{rgb}{0.800000,0.800000,0.800000}%
\pgfsetstrokecolor{currentstroke}%
\pgfsetdash{}{0pt}%
\pgfpathmoveto{\pgfqpoint{7.058294in}{1.124483in}}%
\pgfpathlineto{\pgfqpoint{7.058294in}{4.144483in}}%
\pgfusepath{stroke}%
\end{pgfscope}%
\begin{pgfscope}%
\pgfsetrectcap%
\pgfsetmiterjoin%
\pgfsetlinewidth{1.003750pt}%
\definecolor{currentstroke}{rgb}{0.800000,0.800000,0.800000}%
\pgfsetstrokecolor{currentstroke}%
\pgfsetdash{}{0pt}%
\pgfpathmoveto{\pgfqpoint{1.633294in}{1.124483in}}%
\pgfpathlineto{\pgfqpoint{7.058294in}{1.124483in}}%
\pgfusepath{stroke}%
\end{pgfscope}%
\begin{pgfscope}%
\pgfsetrectcap%
\pgfsetmiterjoin%
\pgfsetlinewidth{1.003750pt}%
\definecolor{currentstroke}{rgb}{0.800000,0.800000,0.800000}%
\pgfsetstrokecolor{currentstroke}%
\pgfsetdash{}{0pt}%
\pgfpathmoveto{\pgfqpoint{1.633294in}{4.144483in}}%
\pgfpathlineto{\pgfqpoint{7.058294in}{4.144483in}}%
\pgfusepath{stroke}%
\end{pgfscope}%
\begin{pgfscope}%
\pgfsetroundcap%
\pgfsetroundjoin%
\pgfsetlinewidth{3.011250pt}%
\definecolor{currentstroke}{rgb}{1.000000,1.000000,1.000000}%
\pgfsetstrokecolor{currentstroke}%
\pgfsetdash{}{0pt}%
\pgfpathmoveto{\pgfqpoint{1.639294in}{4.439372in}}%
\pgfpathlineto{\pgfqpoint{2.617072in}{4.439372in}}%
\pgfusepath{stroke}%
\end{pgfscope}%
\begin{pgfscope}%
\pgfsetbuttcap%
\pgfsetroundjoin%
\pgfsetlinewidth{3.011250pt}%
\definecolor{currentstroke}{rgb}{0.298039,0.447059,0.690196}%
\pgfsetstrokecolor{currentstroke}%
\pgfsetdash{{3.000000pt}{0.000000pt}}{0.000000pt}%
\pgfpathmoveto{\pgfqpoint{1.639294in}{4.001817in}}%
\pgfpathlineto{\pgfqpoint{2.617072in}{4.001817in}}%
\pgfusepath{stroke}%
\end{pgfscope}%
\begin{pgfscope}%
\pgfsetbuttcap%
\pgfsetroundjoin%
\definecolor{currentfill}{rgb}{0.298039,0.447059,0.690196}%
\pgfsetfillcolor{currentfill}%
\pgfsetlinewidth{0.000000pt}%
\definecolor{currentstroke}{rgb}{0.298039,0.447059,0.690196}%
\pgfsetstrokecolor{currentstroke}%
\pgfsetdash{}{0pt}%
\pgfsys@defobject{currentmarker}{\pgfqpoint{-0.104167in}{-0.104167in}}{\pgfqpoint{0.104167in}{0.104167in}}{%
\pgfpathmoveto{\pgfqpoint{0.000000in}{-0.104167in}}%
\pgfpathcurveto{\pgfqpoint{0.027625in}{-0.104167in}}{\pgfqpoint{0.054123in}{-0.093191in}}{\pgfqpoint{0.073657in}{-0.073657in}}%
\pgfpathcurveto{\pgfqpoint{0.093191in}{-0.054123in}}{\pgfqpoint{0.104167in}{-0.027625in}}{\pgfqpoint{0.104167in}{0.000000in}}%
\pgfpathcurveto{\pgfqpoint{0.104167in}{0.027625in}}{\pgfqpoint{0.093191in}{0.054123in}}{\pgfqpoint{0.073657in}{0.073657in}}%
\pgfpathcurveto{\pgfqpoint{0.054123in}{0.093191in}}{\pgfqpoint{0.027625in}{0.104167in}}{\pgfqpoint{0.000000in}{0.104167in}}%
\pgfpathcurveto{\pgfqpoint{-0.027625in}{0.104167in}}{\pgfqpoint{-0.054123in}{0.093191in}}{\pgfqpoint{-0.073657in}{0.073657in}}%
\pgfpathcurveto{\pgfqpoint{-0.093191in}{0.054123in}}{\pgfqpoint{-0.104167in}{0.027625in}}{\pgfqpoint{-0.104167in}{0.000000in}}%
\pgfpathcurveto{\pgfqpoint{-0.104167in}{-0.027625in}}{\pgfqpoint{-0.093191in}{-0.054123in}}{\pgfqpoint{-0.073657in}{-0.073657in}}%
\pgfpathcurveto{\pgfqpoint{-0.054123in}{-0.093191in}}{\pgfqpoint{-0.027625in}{-0.104167in}}{\pgfqpoint{0.000000in}{-0.104167in}}%
\pgfpathclose%
\pgfusepath{fill}%
}%
\begin{pgfscope}%
\pgfsys@transformshift{2.128183in}{4.001817in}%
\pgfsys@useobject{currentmarker}{}%
\end{pgfscope}%
\end{pgfscope}%
\begin{pgfscope}%
\definecolor{textcolor}{rgb}{0.150000,0.150000,0.150000}%
\pgfsetstrokecolor{textcolor}%
\pgfsetfillcolor{textcolor}%
\pgftext[x=2.665961in,y=3.830705in,left,base]{\color{textcolor}\rmfamily\fontsize{35.200000}{42.240000}\selectfont LN}%
\end{pgfscope}%
\begin{pgfscope}%
\pgfsetbuttcap%
\pgfsetroundjoin%
\pgfsetlinewidth{3.011250pt}%
\definecolor{currentstroke}{rgb}{0.866667,0.517647,0.321569}%
\pgfsetstrokecolor{currentstroke}%
\pgfsetdash{{6.000000pt}{6.000000pt}}{0.000000pt}%
\pgfpathmoveto{\pgfqpoint{1.639294in}{3.564261in}}%
\pgfpathlineto{\pgfqpoint{2.617072in}{3.564261in}}%
\pgfusepath{stroke}%
\end{pgfscope}%
\begin{pgfscope}%
\pgfsetbuttcap%
\pgfsetmiterjoin%
\definecolor{currentfill}{rgb}{0.866667,0.517647,0.321569}%
\pgfsetfillcolor{currentfill}%
\pgfsetlinewidth{0.000000pt}%
\definecolor{currentstroke}{rgb}{0.866667,0.517647,0.321569}%
\pgfsetstrokecolor{currentstroke}%
\pgfsetdash{}{0pt}%
\pgfsys@defobject{currentmarker}{\pgfqpoint{-0.104167in}{-0.104167in}}{\pgfqpoint{0.104167in}{0.104167in}}{%
\pgfpathmoveto{\pgfqpoint{0.000000in}{0.104167in}}%
\pgfpathlineto{\pgfqpoint{-0.104167in}{-0.104167in}}%
\pgfpathlineto{\pgfqpoint{0.104167in}{-0.104167in}}%
\pgfpathclose%
\pgfusepath{fill}%
}%
\begin{pgfscope}%
\pgfsys@transformshift{2.128183in}{3.564261in}%
\pgfsys@useobject{currentmarker}{}%
\end{pgfscope}%
\end{pgfscope}%
\begin{pgfscope}%
\definecolor{textcolor}{rgb}{0.150000,0.150000,0.150000}%
\pgfsetstrokecolor{textcolor}%
\pgfsetfillcolor{textcolor}%
\pgftext[x=2.665961in,y=3.393150in,left,base]{\color{textcolor}\rmfamily\fontsize{35.200000}{42.240000}\selectfont ALL\_M}%
\end{pgfscope}%
\begin{pgfscope}%
\pgfsetbuttcap%
\pgfsetroundjoin%
\pgfsetlinewidth{3.011250pt}%
\definecolor{currentstroke}{rgb}{0.333333,0.658824,0.407843}%
\pgfsetstrokecolor{currentstroke}%
\pgfsetdash{{3.000000pt}{3.000000pt}}{0.000000pt}%
\pgfpathmoveto{\pgfqpoint{1.639294in}{3.126706in}}%
\pgfpathlineto{\pgfqpoint{2.617072in}{3.126706in}}%
\pgfusepath{stroke}%
\end{pgfscope}%
\begin{pgfscope}%
\pgfsetbuttcap%
\pgfsetmiterjoin%
\definecolor{currentfill}{rgb}{0.333333,0.658824,0.407843}%
\pgfsetfillcolor{currentfill}%
\pgfsetlinewidth{0.000000pt}%
\definecolor{currentstroke}{rgb}{0.333333,0.658824,0.407843}%
\pgfsetstrokecolor{currentstroke}%
\pgfsetdash{}{0pt}%
\pgfsys@defobject{currentmarker}{\pgfqpoint{-0.104167in}{-0.104167in}}{\pgfqpoint{0.104167in}{0.104167in}}{%
\pgfpathmoveto{\pgfqpoint{-0.000000in}{-0.104167in}}%
\pgfpathlineto{\pgfqpoint{0.104167in}{0.104167in}}%
\pgfpathlineto{\pgfqpoint{-0.104167in}{0.104167in}}%
\pgfpathclose%
\pgfusepath{fill}%
}%
\begin{pgfscope}%
\pgfsys@transformshift{2.128183in}{3.126706in}%
\pgfsys@useobject{currentmarker}{}%
\end{pgfscope}%
\end{pgfscope}%
\begin{pgfscope}%
\definecolor{textcolor}{rgb}{0.150000,0.150000,0.150000}%
\pgfsetstrokecolor{textcolor}%
\pgfsetfillcolor{textcolor}%
\pgftext[x=2.665961in,y=2.955595in,left,base]{\color{textcolor}\rmfamily\fontsize{35.200000}{42.240000}\selectfont ALL\_C}%
\end{pgfscope}%
\end{pgfpicture}%
\makeatother%
\endgroup%
%
}

\caption{Storage-aware}
\end{subfigure}
\caption{Comparison of our linear reuse algorithm with two baselines under different materialization schemes (LN = linear, ALL\_C = all compute, ALL\_M = all materialized).}
\end{figure}

\subsection{Discussion}
